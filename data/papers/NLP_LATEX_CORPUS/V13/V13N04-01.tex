    \documentstyle[graphicx,theorem,amsmath,jnlpbbl,fleqn]{jnlp_e_b5_2e}


\theoremstyle{break}
\newtheorem{theorem}{}[] 

\theoremstyle{change}
\newtheorem{lemma}{}[] 

\newtheorem{syoumei}{}
\renewcommand{\thesyoumei}{}
\newcommand{\proof}[1]{}


\newcommand{\myfigdir}{}

\newcommand{\myfiglabelskip}{}
\newcommand{\myfiglabelskippre}{}
\newcommand{\mygapskip}{}



    \newcommand{\mynoindent}{}


\newcommand{\myindentskipa}{}



\newif\if提出
\提出true     


\setcounter{page}{3}
\setcounter{巻数}{13}
\setcounter{号数}{4}
\setcounter{年}{2006}
\setcounter{月}{10}
\受付{2005}{10}{14}
\再受付{2006}{2}{20}
\採録{2006}{5}{1}

\title{}
\author{}
\jkeywords{}

\etitle{
Graph Branch Algorithm: An Optimum Tree\\
Search Method for Scored Dependency Graph\\
with Arc Co-occurrence Constraints
}
\eauthor{Hideki Hirakawa\affiref{toshiba}}
\headauthor{Hirakawa}
\headtitle{Graph Branch Algorithm}
\affilabel{toshiba}
          {TOSHIBA Corporate Research \& Development Center}
          {TOSHIBA Corporate Research \& Development Center}
\eabstract{ Preference Dependency Grammar (PDG) is a framework for the
morphological, syntactic and semantic analysis of natural language
sentences. PDG gives packed shared data structures for encompassing
the various ambiguities in each levels of sentence analysis with
preference scores and a method for calculating the most plausible
interpretation of a sentence. This paper proposes the Graph Branch
Algorithm for computing the optimum dependency tree (the most
plausible interpretation of a sentence) from a scored dependency
forest which is a packed shared data structure encompassing all
possible dependency trees (interpretations) of a sentence. The graph
branch algorithm adopts the branch and bound principle for managing
arbitrary arc co-occurrence constraints including the single valence
occupation constraint which is a basic semantic constraint in
PDG. This paper also reports the experiment using English texts
showing the computational complexity and behavior of the graph branch algorithm.
}
\ekeywords{Optimum Tree Search, Branch and Bound Method, Dependency Structure, Syntactic Analysis, Semantic Analysis}


\begin{document}

\maketitle
\thispagestyle{empty}


\section{Introduction}
Preference Dependency Grammar (PDG) is a framework for the
morphological, syntactic and semantic analysis of natural language
sentences, mainly designed for manipulating one of the most serious
issues in natural language analysis, i.e. ambiguities in sentence
interpretations. It requires the suppression of combinatorial
explosion, the proper treatment of various kinds of knowledge such as
hypothesis generation, constraint and preference knowledge, and the
integration of knowledge in each analysis
level \cite{Hirakawa01,Hirakawa05a}. PDG adopts the following approach
for these problems.

\begin{itemize}
\item [(a)] Utilizing multiple kinds of linguistic data structures such
as phrase structure and dependency structure to treat morphological,
syntactic and semantic knowledge with high module independency.
\item [(b)] Utilizing packed shared data structure encompassing all
ambiguities at each level of sentence analysis to suppress
combinatorial explosion without pruning.
\item [(c)] Every interpretation in each data structure has
mapping to one or more interpretations in other data structures. This
provides a basis for integrating multi-level preference knowledge
(scores).
\item [(d)] Incorporating an optimum tree search method for a packed
shared data structure with arc co-occurrence constraints to
compute the most plausible interpretation (dependency tree) of a sentence.
\end{itemize}

PDG adopts the packed shared data structures such as the ``Headed Parse
Forest'', the ``Functional Dependency Forest''\footnote{``functional
dependency forest'' is simply called ``dependency forest'' in this
paper.} and the ``Semantic Dependency Forest'' encompassing parse trees
(phrase structure trees), functional dependency trees (syntactic
dependency trees) and semantic dependency trees, respectively
\cite{Hirakawa05a}.

PDG is a framework for dependency analysis. As described in
\cite{McDonald05}, various dependency analysis methods are
proposed. Some methods utilize lexicalized phrase-structure parsers
with the ability to output dependency information
\cite{Collins99,Charniak00} and some methods obtain dependency trees
directly \cite{Ozeki94,Kato_ehara89,Eisner96,YamadaAndMatsumoto03,NivreAndScholz04}.
In this paper, parsers in the former category are called
phrase-structure based dependency parsers and those in the latter
category are called direct dependency parsers. PDG is in the former
category because it utilizes a lexicalized phrase-structure parser to
generate a dependency forest \cite{Hirakawa05a} as described below.

\cite{Collins99,Charniak00} are basically lexicalized phrase structure
parsers and work as dependency parsers by attaching a function for
conversion from a phrase-structure to a dependency structure. The
dependency tree for a sentence is generated from the headed phrase
structure tree\footnote{``headed'' means that each nonterminal symbol in the tree
has its phrase head (an input word with POS)} obtained
by the phrase structure parser. For example, each nonterminal symbol
and its child constituents in the phrase structure tree correspond to
the dependency structure that has one governor node (the phrase head
of the nonterminal symbol) and its dependant nodes (the phrase heads
of the child constituents) in \cite{Collins99}. On the other hand, PDG
generates a dependency structure based on structure mapping information
in grammar rules \cite{Hirakawa05a}. This mechanism enables 
generation of flexible dependency structures with dependency relation
labels. For example, PDG can provide phrase structure rules which
generate non-projective dependency structures which are not produced
by \cite{Collins99,Charniak00} and the majority of direct dependency
parsers as described in Section \ref{sec:WellFormednessConstraints}. 
The phrase-structure based dependency parsers have a possibility to 
utilize the descriptive power of the phrase structure rules to
prescribe the dependency structures.

Training corpora and statistical information are used for computing
the most appropriate dependency tree in many parsers. One class
of parsers adopts a history-based approach\cite{Black92} in which each
tree-building procedure uses a probability model p(A$\mid$B) to weight any
action A based on the available context, or history, B.
\cite{YamadaAndMatsumoto03,NivreAndScholz04} can be regarded as
history-based direct dependency parsers which choose the optimum
decision during the parsing process based on information obtained from
the training data. Another class of parsers generates various
dependency graphs encompassing all possible dependency trees for a
sentence\footnote{In fact, a set of possible dependency trees is
represented by a dependency graph and a set of constraints as shown in
Section \ref{sec:OptimumTreeSearchForDependencyGraph}.} and searches
for the optimum tree based on preference scores attached to the
dependency graph\footnote{Dependency graphs with preference scores are
called scored dependency graphs.}
\cite{Ozeki94,Kato_ehara89,Hirakawa01,McDonald05}. This method is
called the full-parse-based approach in this paper. In general, the
history-based method seems to be more efficient than the
full-parse-based method because it makes decisions before completing
the full parse. However, the history-based method may fail to obtain
the optimum solution because it does not utilize the full parse
information. PDG is classified as a full-parse-based method since it
searches for the optimum tree in a dependency forest\footnote{The
dependency forest consists of a dependency graph and a constraint
matrix which represents arc co-occurrence constraints as described in
Section \ref{sec:SemanticDependencyGraphAndDependencyForest}.}.

This paper focuses on full-parse-based methods and discusses some
approaches to the optimum tree search in dependency graphs and
proposes an optimum tree search algorithm for the dependency forest,
the ``Graph Branch Algorithm''. PDG (and the graph branch
algorithm) is an extension of the sentence analysis method based on the
``Semantic Dependency Graph'' \cite{Hirakawa89b_e,Hirakawa01}.

Section \ref{sec:OptimumTreeSearchForDependencyGraph} formalizes the
framework of the optimum tree search in dependency graphs and shows
some traditional methods. Section
\ref{sec:SemanticDependencyGraphAndDependencyForest} compares two
approaches based on semantic dependency graph and the dependency
forest. Sections \ref{sec:OptimumTreeSearchForDependencyForest} and
\ref{sec:ExampleOfOptimumTreeSearch} describe the graph branch
algorithm and the execution example of the algorithm,
respectively. Section \ref{sec:ExperimentOfOptSearch} shows some
experimental results for the computational complexity and behavior of the graph branch
algorithm.


\section{Optimum Tree Search in a Scored Dependency Graph}
\label{sec:OptimumTreeSearchForDependencyGraph}

 \begin{figure}[tb]
 \begin{center}
     \includegraphics[scale=0.8]{OS_スコア付き依存グラフの枠組みENG.eps}
 \end{center}
\myfiglabelskippre
\caption{Framework of the optimum tree search in a scored dependency graph}
\label{fig:ScoreAddedDependencyGraph}
\end{figure}

\subsection{Basic Framework}

Scored dependency graphs are widely used as packed shared data
structures representing a set of dependency trees. Figure
\ref{fig:ScoreAddedDependencyGraph} shows the basic framework of the
optimum dependency tree search in a scored dependency
graph\footnote{Sometimes ``dependency forest'' is used for representing
``scored dependency graph'' for simplicity.}. In general, nodes in a
dependency graph correspond to words in the sentence and the arcs show
some kind of labeled or non-labeled dependency relations between
nodes. Each arc has a preference score representing plausibility of
the relation\footnote{Scoring method is an important issue beyond the
scope of this paper.  \cite{McDonald05} proposes a discriminative
learning method for obtaining the optimum parameters for scoring
dependency arcs.}. The well-formed dependency tree constraint is a set
of well-formed constraints which should be satisfied by all dependency
trees representing sentence interpretations. A pair of a dependency
graph and a well-formed dependency tree constraint defines a set of
well-formed dependency trees. The score of a dependency tree is the
sum total of arc scores. The optimum tree is a dependency tree with
the highest score in the set of dependency trees defined by the
dependency graph and the well-formed dependency tree constraint.

\subsection{Dependency Graph}
\label{sec:DependencyGraph}

Dependency graphs are classified into some classes based on the types
of nodes and arcs. This paper assumes three types of nodes,
i.e. word-type, WPP-type\footnote{WPP is a pair of a word and a part
of speech (POS). The word ``time'' has WPPs such as ``time/n'' and
``time/v''.} and concept-type\footnote{One WPP (ex. ``time/n'') can be categorized
into one or more concepts semantically (ex. ``time/n/period\_time''
and ``time/n/clock\_time'').}. The types of dependency graphs are
called a word dependency graph, a WPP dependency graph and a concept
dependency graph, respectively, in this paper. Dependency graphs are
also classified into non-labeled and labeled graphs. There are some
types of arc labels such as syntactic label (ex. ``subject'',
``object'') and semantic label (ex. ``agent'',``target''). Various
types of dependency graphs are used in existing systems according to
these classifications, such as non-label word dependency graph
\cite{Lee97,Eisner96,McDonald05}, syntactic-label word dependency
graph \cite{Maruyama90}, semantic-label word dependency graph
\cite{Hirakawa01}, non-label WPP dependency graph
\cite{Ozeki94,Kato_ehara89}, syntactic-label WPP dependency graph
\cite{Wang04}, semantic-label concept dependency graph\footnote{This data structure encompasses semantic dependency trees
for one word-dependency tree.} \cite{Harada01}. PDG utilizes two types of dependency
graphs, i.e. a syntactic-label WPP dependency graph in the functional dependency forest and a
semantic-label concept dependency graph in the semantic dependency
forest \cite{Hirakawa05a}. 

\subsection{Well-formedness Constraints for Dependency Tree}
\label{sec:WellFormednessConstraints}

There can be a variety of well-formedness constraints for dependency
trees from very basic and language-independent constraints to specific
language-dependent constraints. This paper focuses on the following four
basic and language-independent constraints which may be embedded in
data structure and/or the optimum tree search algorithm.


\begin{itemize}
\item [(C1)] Coverage constraint: Every input word has a corresponding node in the tree
\item [(C2)] Single role constraint: No two nodes in a dependency tree
occupy the same input position

\item [(C3)] Projectivity constraint: No arc crosses another arc\footnote{Another condition for projectivity, i.e. ``no arc covers top
node'' is equivalent to the crossing arc constraint if special root
node, which is a governor of top node, is introduced at the top (or
end) of a sentence.}

\item [(C4)] Single valence occupation constraint: No two arcs in a tree occupy the same valence of a predicate 
\end{itemize}

{\mynoindent}(C1) and (C2) are basic constraints adopted by almost all
dependency parsers. (C1) and (C2) are collectively referred to as
``covering constraint''. (C3) is also adopted by the majority of
dependency parsers which are called projective dependency parsers. A
projective dependency parser fails to analyze non-projective
sentences\footnote{\cite{Nivre05} proposes a pseudo-projective
dependency parsing method which transforms projective dependency tree
to non-projective dependency tree. This kind of approach is beyond the
scope of this paper.}.
Most sentences of a language are projective, but several types of
non-projective sentences exist \cite{Meluk88}. The non-projective
parsing model obtained improvement in overall accuracy compared with
the projective model in an experiment on Czech, which has more
flexible word order than English \cite{McDonald05}. 
In this case, all possible non-projective dependency trees are
candidates for the sentence structure because no projectivity
constraint is applied in contrast to projective parsing model. The
author calls this type of non-projectivity an uncontrolled
non-projectivity. Uncontrolled acceptance of non-projectivity may
cause overall degradation of parsing accuracy for other
languages. This seems to be a main reason why the majority of
dependency parsers adopt the projectivity constraint. As described below, PDG does not adopt (C3) directly. Therefore PDG
can generate non-projective dependency trees for input sentences.
(C4) is a basic constraint for valency but is not adopted by the majority
of dependency parsers. 

(C2)-(C4) can be described as a set of co-occurrence constraints
between two arcs in a dependency graph. As described below, PDG adopts
co-occurrence constraints between two arbitrary arcs in a dependency
graph using a matrix called ``Constraint Matrix'' (CM) whose rows
and columns are a set of arcs in a dependency graph. Constraints
represented by CM are called arc co-occurrence constraints.

{\mygapskip}
\begin{itemize}
\item [(C5)] Arc co-occurrence constraint: Each arc pair in a tree has a co-occurrence constraint in CM
\end{itemize}
{\mygapskip}

{\mynoindent}More precise constraints compared with (C2) - (C4) are
representable by means of CM. For example, it can allow
non-projectivity for only some special arcs. In PDG, the mapping
between a sequence of constituents (the body of a CFG rule) and a set
of arcs (a partial dependency tree) is defined in an extended CFG rule
\cite{Hirakawa05a}. As described below, this grammar framework allows
generating non-projective structures defined by grammar rules. The
author calls this type of non-projectivity a controlled
non-projectivity. The controlled non-projectivity can reduce the
generation of illegal non-projective dependency trees compared
with the uncontrolled non-projectivity.

The optimum tree search in a scored dependency graph is the task of
searching for the dependency tree with the highest score satisfying the
well-formed dependency tree constraint. The algorithm for this task is
closely related to the types of dependency graphs and/or
well-formedness constraints. Graph search algorithms, such as the
Chu-Liu-Edmonds maximum spanning tree algorithm
\cite{ChuLiu65,Edmonds67}, algorithms based on the dynamic
programming (DP) principle \cite{Ozeki94,Eisner96} and the algorithm based
on the branch and bound (B\&B) method \cite{Hirakawa01}, are used for
the optimum tree search in scored dependency graphs. The Chu-Liu-Edmonds
algorithm is very fast ($O(n^2)$ for sentence length $n$), but it works
correctly only on word dependency graphs. Maximum spanning tree
algorithms cannot satisfy the single role constraint for WPP and
concept dependency graphs. DP-based algorithms can satisfy (C1) - (C3)
and run efficiently, but seems not to satisfy (C4). \cite{Hirakawa01} proposed a
B\&B-based algorithm working on word dependency graphs satisfying (C1)
- (C4). This paper extends this algorithm to work on
WPP and concept dependency graphs. The next section explains the
problems of the DP-based method in treating (C4).

\subsection{Single Valence Occupation Constraint and DP}


 \cite{Ozeki94} proposed an algorithm for obtaining the optimum
kakari-uke tree and its score from a set of all possible scored
kakari-uke relations\footnote{Kakari-uke relation, widely adopted in
Japanese sentence analysis, is projective dependency relation with a
constraint such that the dependent word is located at the left-hand
side of its governor word (the left to right dependency).}
corresponding to a scored dependency graph. The author calls this
algorithm the minimizing total cost method (MTCM) in this paper. MTCM
can be extended to treat general dependency relations, including the
right to left dependency relation\cite{Kato_ehara89}. This algorithm
is generalized into the minimum cost partitioning method (MCPM), which
is a partitioning computation based on the recurrence equation given
below \cite{Ozeki99_e}. MCPM is also a generalization of the
probabilistic CKY algorithm and the Viterbi algorithm\footnote{Specifically, MTCM corresponds to probabilistic CKY and the
Viterbi algorithm because it computes both the optimum tree score and
its structure.}.


Considering the phrase $(w_i,…w_j; a_i,…,a_j; A)$ partitioned into $(w_i,…,w_k;
a_i,…,a_k; B)$ and $(w_{k+1},…,w_j; a_{k+1},…, a_j: C)$ where
$w_x$, $a_x$, and $A$-$C$ mean word, analog information (such as prosodic
information), and features like phrase name, respectively. MCPM computes
the optimum solution based on the following recurrence equation for
total cost F.
\[
 F(i,j,A) = min[F(i,k,B)+F(k+1,j,C)+cost(w_i,…,w_j, a_i,…,a_j,k,A,B,C)]
\]
$F(i,j,A)$ is the total cost of phrase $A$ covering from
the $i$-th to the $j$-th word in a given sentence. $cost(w_i,...w_j,
a_i,...,a_j,k,A,B,C)$ is a cost function where $k$ is a partitioning
position. The minimum cost partition of the whole sentence is
calculated very efficiently by the DP principle for this equation. The
optimum partitioning obtained by this method constitutes a tree
covering the whole sentence satisfying the single role and
projectivity constraints. However, it is not assured that the single
valence occupation constraint adopted in PDG for basic semantic level
constraint is satisfied by MCPM.

\begin{figure}[bt]
\myfiglabelskippre
 \begin{center}
     \includegraphics[scale=0.51]{OS多重格制約を満たす最適解探索DPとBB比較ENG.eps}
 \end{center}
\myfiglabelskippre
\caption{Optimum solution search satisfying the single valence occupation constraint}
\label{fig:SingleValenceOccupationProblem}
\myfiglabelskippre
\end{figure}

Figure \ref{fig:SingleValenceOccupationProblem} shows a dependency
graph for the Japanese phrase ``Isha-mo Wakaranai Byouki-no Kanja''
encompassing dependency trees corresponding to ``a patient suffering
from a disease that the doctor doesn't know'', ``a sick patient who
does not know the doctor'', and so on. The dependency graph has two
kinds of ambiguities, i.e. semantic role ambiguity and attachment
ambiguity. For example, $Wakaranai(not\_know)$ has four outgoing arcs
with different semantic roles ($agent$ and $target$) and different
attachments ($Byouki(sickness)$ and $Kanja(patient)$) in Figure
\ref{fig:SingleValenceOccupationProblem}. The single valence
occupation constraint prevents $Wakaranai(not\_know)$ from being
connected with the same two semantic role arcs. $OS_1$ - $OS_4$
represent the optimum solutions for the phrases specified by their
brackets computed based on MCPM. For example, $OS_3$ gives an optimum
tree with a score of $22$ (consisting of $agent1$ and $target4$) for
the phrase ``Isha-mo Wakaranai Byouki-no''. The optimum solution for
the whole phrase is either $OS_1+OS_4$ or $OS_3+OS_2$ due to MCPM. The
former has the highest score $40(=15+25)$ but does not satisfy the
single valence occupation constraint because it has $agent1$ and
$agent5$ simultaneously. The optimum solutions satisfying this
constraint are $NOS_1+OS_4$ and $OS_1+NOS_2$ shown at the bottom of
Figure \ref{fig:SingleValenceOccupationProblem}. $NOS_1$ and $NOS_2$
are not optimum solutions for their word coverages. In this case, MCPM
generates a non-optimum tree in $OS_3+OS_2$ if it adopts the strategy
of neglecting inconsistent trees. Otherwise, MCPM generates a high
score but an ill-formed tree in $OS_1+OS_4$. This shows that it is not
assured that MCPM will obtain the optimum solution satisfying the
single valence occupation constraint. 
On the contrary, it is assured that the graph branch algorithm
computes the optimum solution(s) satisfying the single valence
occupation constraint because it computes the optimum solution(s)
satisfying any co-occurrence constraints in the constraint matrix
as described below. It is an open problem whether an algorithm based
on the DP framework exists which can handle the single valence
occupation constraint and arbitrary arc co-occurrence constraints.


\section{Semantic Dependency Graph and Dependency Forest}
\label{sec:SemanticDependencyGraphAndDependencyForest}

The semantic dependency graph, as shown in Section
\ref{sec:SemanticDependencyGraph}, is a semantic-label word
dependency graph designed for Japanese sentence analysis
\cite{Hirakawa89}. The optimum solution for a sentence is obtained by
searching for the optimum tree in a semantic dependency graph with
preference scores \cite{Hirakawa01}. 

The sentence analysis method based on the semantic dependency graph is
effective because it employs linguistic constraints as well as
linguistic preferences. However, this method is lacking in terms of
generality in that it cannot handle backward dependency and multiple
WPP because it depends on some linguistic features peculiar to
Japanese. PDG employs the dependency forest instead of the semantic
dependency graph. The dependency forest is a packed shared data
structure encompassing all dependency trees corresponding to parse
trees in a packed shared parse forest \cite{Tomita91} for a
sentence. Since the dependency forest has none of the
language-dependent premises that the semantic dependency graph has, it
is applicable to English and other languages. PDG has one more
advantage in that it can generate non-projective dependency trees
because the mapping from phrase structure to dependency structure is
defined in grammar rules.

The optimum tree search algorithm for the semantic dependency graph is
not applicable to the dependency forest. This paper gives a brief
explanation of the dependency forest and shows the graph branch
algorithm for obtaining the optimum solution (tree) in the dependency
forest.

\subsection{Semantic Dependency Graph and its Drawbacks} 
\label{sec:SemanticDependencyGraph}

\begin{figure}[b]
 \begin{center}
     \includegraphics[scale=0.7]{OS最適解付き意味係り受けグラフの例ENG.eps}
 \end{center}
\myfiglabelskippre
\caption{Example of semantic dependency graph and its optimum solution}
\label{fig:ImiKakariGraphWithOptSol}
\end{figure}

Figure \ref{fig:ImiKakariGraphWithOptSol} shows a semantic dependency graph
for ``Watashi-mo Kare-ga Tsukue-wo Katta Mise-ni Utta''
 \cite{Hirakawa01}. The nodes in the graph correspond to the content words in the
sentence and the arcs show possible semantic dependency relations between
the nodes. Each arc has an arc ID and a preference score. Interpretations
of a sentence are well-formed spanning trees that satisfy the
projectivity constraint and the single valence occupation constraint.
The bold arcs in the graph in
Figure \ref{fig:ImiKakariGraphWithOptSol} shows the optimum
interpretation with a maximum score of 130.

The semantic dependency graph is designed based on the Japanese kakari-uke
relation and assumes the following features of Japanese.
\begin{itemize}
\item [(a)] A dependant always locates to the left of its governor (no backward dependency)
\item [(b)] POS ambiguities are quite minor compared with English\footnote{Word boundary ambiguity corresponding to the compound word
boundary ambiguity in English exists in Japanese. Treatment of this
ambiguity is a practical problem for the semantic dependency graph even when
applied to Japanese sentence analysis.}
\end{itemize}

{\mynoindent}The semantic dependency graph and its optimum solution
search algorithm adopt these as their premises. Therefore, this method
is inherently inapplicable to languages like English that require
backward dependency and multiple POS analysis.


\subsection{Overview of Dependency Forest}

 \begin{figure}[b]
 \begin{center}
     \includegraphics[scale=0.58]{COM_TimeFliesに対するスコア付き依存森ENG.eps}
 \end{center}
\myfiglabelskippre
\caption{Scored dependency forest for ``Time flies like an arrow''}
\label{fig:WDFForExample}
\end{figure}

The dependency forest is a packed shared data structure encompassing
all possible dependency trees for a sentence. As described above, PDG
utilizes two kinds of dependency forest, i.e. the functional
dependency forest and the semantic dependency forest.  The functional
dependency forest encompasses a set of functional (or syntactic)
dependency trees which consist of WPP nodes and labeled arcs
representing the syntactic relations between them. The semantic
dependency forest encompasses a set of semantic dependency trees which
consist of lexical concept nodes and labeled arcs representing the
semantic relations between them. Each of the two dependency forests
consists of a dependency graph (DG) and a constraint matrix (CM).
This paper focuses on the functional dependency forest because it is
the target of the current PDG prototype system. Figure
\ref{fig:WDFForExample} shows a scored (functional) dependency forest
for the example sentence ``Time flies like an arrow.'' The dependency
graph consists of nodes and directed arcs. A node represents a WPP and
an arc shows the dependency relation between nodes. An arc has its ID
and preference score. CM is a matrix whose rows and columns are a set
of arcs in DG and prescribes the co-occurrence constraint between
    arcs. Only when CM(i,j) is $\bigcirc$, 
$arc_i$ and $arc_j$ are co-occurrable
in one dependency tree.

\begin{figure}[b]
 \begin{center}
     \includegraphics[scale=0.8]{COM_TimeFliesに対する整依存木ENG.eps}
 \end{center}
\myfiglabelskippre
\caption{Well-formed dependency trees for the example sentence}
\label{fig:WellFormedTreesForExample}
\end{figure}

The dependency forest has correspondence with the packed shared parse
forest\footnote{The correspondence between the parse tree and the
dependency tree is generally 1 to N and vice versa.}. This means that
the dependency forest provides a means to treat all possible
interpretations of a sentence in dependency structure representation.
One sentence interpretation is represented by one well-formed
dependency tree which satisfies the well-formed dependency tree
constraint, i.e. the covering constraint and the arc co-occurrence
constraint described in Section \ref{sec:WellFormednessConstraints}.
In semantic dependency graphs, a spanning tree of a graph is a
well-covered tree. This simplifies the development of an optimum
solution search algorithm. The algorithm for the dependency forest has
to treat the covering
constraint. 

Figure \ref{fig:WellFormedTreesForExample} shows four
well-formed dependency trees for the dependency forest in Figure
\ref{fig:WDFForExample}. Tree (a) is the optimum dependency tree with
the highest score of 51.


\subsection{Relation Between Semantic Dependency Graph and Dependency Forest}

The dependency forest and the semantic dependency graph utilize the
WPP dependency graph and the word graph respectively. The word
dependency graph can be seen as a special case of the WPP dependency
graph. Therefore, the semantic dependency graph is a subset of the
dependency graph of the dependency forest. On the other hand,
well-formedness constraints introduced to a semantic graph, i.e. the
projectivity and single valence occupation constraints, are a type of
arc co-occurrence constraints representable by means of CM. Therefore,
the dependency forest is a generalized and more powerful data
structure covering the representative power of the semantic dependency
graph.


\section{The Optimum Tree Search in the Dependency Forest Based on the Graph Branch Method}
\label{sec:OptimumTreeSearchForDependencyForest}

The graph branch method works on the branch and bound principle and
searches for the optimum well-formed tree in a dependency graph by
applying partial sub-problem expansions called graph branching. The
algorithm in \cite{Hirakawa01} applies the graph branch method to the
semantic dependency graph. Unfortunately, this algorithm is not
directly applicable to the dependency forest search problem. The
following shows a new algorithm for applying the graph branch method
to the dependency forest. This algorithm is applicable to the
functional dependency forest and the semantic dependency forest
because the basic framework is common to both of the dependency
forests.


\subsection{Outline of Branch and Bound Method}

The branch and bound method is a principle for solving computationally
hard problems such as NP-complete problems \cite{Ibaraki78}. The basic
strategy is that the original problem is decomposed into easier
partial-problems (branching) and the original problem is solved by
solving them. Pruning called a bound operation is applied if it turns
out that the optimum solution to a partial-problem is inferior to the
solution obtained from some other partial-problem (dominance test), or
if it turns out that a partial-problem gives no optimum solutions to
the original problem (maximum value test). The dominance test is not
used in the graph branch method. Usually, the branch and bound
algorithm is constructed to minimize the value of the solution. The
graph branch algorithm in this paper is constructed to maximize the
score of the solution because the best solution is the maximum tree in
the dependency forest.

\begin{figure}[b]
\begin{center}
     \includegraphics[scale=0.88]{OS分枝限定法の基本アルゴリズムENG.eps}
\end{center}
\myfiglabelskippre
\caption{Skeleton of branch and bound algorithm}
\label{fig:BBAlgorithm}
\end{figure}

The following features for the maximum bound value test with respect
to the problem $P$ and its partial-problem $P_c$ must be satisfied in
the branch and bound method\footnote{MC1-MC4 are called model conditions in this paper.}.

\begin{itemize}
\item[(MC1)] $g(P_c){\geq}f(P)$ where $g(P_c)$ is the maximum value of $P_c$, and $f(P)$ is the maximum value of $P$.
\item[(MC2)] If $g(P_c)=l(P)$ where $l$ gives a value of a feasible solution to $P$, then the feasible solution is a solution to $P$.
\item[(MC3)] If $P_c$ has no feasible solutions then $P$ has no solutions.
\item[(MC4)] If a feasible solution with an incumbent value $z$ is obtained for some partial-problem, and if $g(P_c){\leq}z$, then partial-problems branched from problem $P$ have no better solutions than $z$.
\end{itemize}

{\mynoindent}In the case of MC2-MC4, partial-problem $P_c$ can be
terminated. Figure \ref{fig:BBAlgorithm} shows a general branch and
bound algorithm for obtaining one optimum solution.


\subsection{Graph Branch Algorithm}
\label{GraphBranchAlgorithm}

\begin{figure}[b]
 \begin{center}
      \includegraphics[scale=0.93]{OSグラフ分枝アルゴリズムENG.eps}
 \end{center}
\myfiglabelskippre
\caption{Graph Branch Algorithm}
\label{fig:GraphBrachAlgorithm}
\end{figure}

Figure \ref{fig:BBAlgorithm} shows a skeleton of the algorithm. In order to make it
running code, each operation in the algorithm must be realized for
the target problem. The graph branch algorithm applies the branch and bound
method to the optimum tree search problem with the binary arc
co-occurrence constraint by introducing the graph branch operation for
the partial-problem expansion operation. Figure \ref{fig:GraphBrachAlgorithm}
shows the graph branch algorithm which has been extended from the original
skeleton to search for all the optimum trees in a dependency graph. The
following sections explain how the components of the branch and bound
method in Figure \ref{fig:BBAlgorithm} are implemented in the graph branch
algorithm.

\mygapskip\mynoindent{\bf 
(1) Partial-problem}

Partial-problem $P_i$ in the graph branch method is a problem
searching for all the well-formed optimum trees in a dependency forest
$DF_i$ consisting of the dependency graph $DG_i$ and constraint matrix
$CM_i$. Partial-problem $P_i$ consists of the following elements.

\begin{itemize}
\item[(a)] Dependency graph $DG_i$
\item[(b)] Constraint matrix $CM_i$
\item[(c)] Feasible solution value $LB_i$ (corresponding to $l(P)$ in Figure \ref{fig:BBAlgorithm})
\item[(d)] Upper bound value $UB_i$ (corresponding to $g(P)$ in Figure \ref{fig:BBAlgorithm})
\item[(e)] Inconsistent arc pair list $IAPL_i$
\end{itemize}

{\mynoindent}The constraint matrix is common to all partial-problems,
so one $CM$ is shared by all partial-problems. $DG_i$ is represented
by ``$rem[..]$'' which shows a set of arcs to be removed from the
whole dependency graph $DG_i$, i.e. $DG_i$ is obtained by removing
$rem[..]$ from $DG_i$. For example, ``$rem[b,d]$'' represents a
partial dependency graph $[a,c,e]$ in the case $DG=[a,b,c,d,e]$. This
reduces the memory space and the computation for a feasible solution
as described below. $IAPL_i$ is a list of inconsistent arc pairs. An
inconsistent arc pair is an arc pair which does not satisfy some
co-occurrence constraint.

\mygapskip\mynoindent{\bf 
(2) Algorithm for Obtaining Feasible Solution and Lower Bound Value
}

In the graph branch method, a well-formed dependency tree in the
dependency graph $DG$ of the partial-problem $P$ is assigned as the
feasible solution $FS$ (corresponding to $x$ in
Figure \ref{fig:BBAlgorithm}) of $P$~\footnote{A feasible solution may not be
optimum but is a possible interpretation of a sentence. Therefore, it
can be used as an approximate output when the search process is
aborted.}. The score of the feasible solution $FS$ is assigned as the
lower bound value $LB$ (corresponding to $l(P)$ in
Figure \ref{fig:BBAlgorithm}). The function for computing these values
$get\_fs$ is called a feasible solution/lower bound value
function. Figure \ref{fig:AlgorithmForLPi} shows the algorithm of
$get\_fs$. Basically, $get\_fs$ searches for one feasible solution in
a higher-score-first manner and a depth-first manner. When an arc which
violates co-occurrence constraint against one of the selected arcs is found,
$get\_fs$ backtracks at $step5$ to the nearest choice point
which resolves the contradiction. This assures that the obtained
solution satisfies the co-occurrence constraint. Furthermore, if $get\_fs$
finds no solution, then the problem $P$ has no solution. Since
$get\_fs$ selects one arc for each position in a sentence, the obtained
arcs satisfies the well-covered constraint. 

Arc groups $S_1$ to $S_n$ are sorted according to their scores in $step1$. This operation is introduced to obtain a better (higher score) feasible solution, since the better feasible solution lead to a higher incumbent value which bounds more partial-problems. 

\begin{figure}[t]
 \begin{center} 
    \includegraphics[scale=0.93]{OS許容解・下界値を求めるアルゴリズムENG.eps}
 \end{center}
\myfiglabelskippre
\caption{Algorithm for obtaining $FS$ and $LB$}
\label{fig:AlgorithmForLPi}
\end{figure}

\mygapskip\mynoindent{\bf 
(3) Algorithm for Obtaining Upper Bound Value}

Given a set of arcs $A$ which is a subset of a dependency graph $DG$,
if the set of dependent nodes\footnote{The dependent node of an arc
is the node located at the source of the arc.} of arcs in $A$
satisfies the covering constraint described above, the arc set $A$ is
called the well-covered arc set. The maximum well-covered arc set is
defined as a well-covered arc set with the highest score. In general,
the maximum well-covered arc set does not satisfy the single role
constraint and does not form a tree. In the graph branch method, the
score of the maximum well-covered arc set of a dependency graph $G$ is
assigned as the upper bound value $UB$ (corresponding to $g(P)$ in
Figure \ref{fig:BBAlgorithm}) of the partial-problem $P$. Upper bound
function $get\_ub$ calculates $UB$ by scanning the arc lists sorted by
the surface position of the dependent nodes of the arcs.

The above settings satisfy the model conditions. In these settings, $P$
and $get\_ub$ corresponds to $P_c$ and $g(P_c)$ respectively. (MC1) is
satisfied because $get\_ub(P){\geq}f(P)$ is true for $f(P)$ (the score
of the optimum tree). (MC2) and (MC4) are satisfied because $get\_ub$ is
the score of the maximum well-covered arc set. (MC3) is satisfied
since $get\_ub(P)$ always has its solution. Therefore, partial-problem
$P$ is prunable if the incumbent value $z$ satisfies $z{\geq}g(P)$\footnote{In the case of obtaining all optimum solutions, the terminate
condition should be changed to $z>g(P)$.}.

\mygapskip\mynoindent{\bf 
(4) Branch Operation}

\begin{figure}[b]
 \begin{center}
      \includegraphics[scale=0.77]{OSグラフ分枝の図ENG.eps}
 \end{center}
\myfiglabelskippre
\caption{Graph Branching}
\label{fig:GraphBranching}
\end{figure}

Figure \ref{fig:GraphBranching} shows a branch operation in the graph branch
method called a graph branch operation. Child partial-problems of
$P$ are constructed as follows:

\begin{itemize}
\item[(a)] Search for an inconsistent arc pair $(arc_i,arc_j)$ in the maximum
well-covered arc set of the dependency graph of $P$.
\item[(b)] Create child partial-problems $P_i$, $P_j$ which have new
dependency graphs $DG_i=DG-\{arc_j\}$ and $DG_j=DG-\{arc_i\}$ respectively.
\end{itemize}

{\mynoindent}Since a solution to $P$ cannot have both $arc_i$ and
$arc_j$ simultaneously due to the co-occurrence constraint, the
optimum solution of $P$ is obtained from either/both $P_i$ or/and
$P_j$. The child partial-problem is easier than the parent
partial-problem because the size of the dependency graph of the child
partial-problem is less than that of its parent.

In Figure \ref{fig:GraphBrachAlgorithm}, $get\_iapl$ computes the list
of inconsistent arc pairs $IAPL$(Inconsistent Arc Pair List) for the
maximum well-covered arc set of $P_i$. Then the graph branch function
$graph\_branch$ selects one inconsistent arc pair $(arc_i,arc_j)$ from
$IAPL$ for branch operation. The selection criteria for
$(arc_i,arc_j)$ affects the efficiency of the
algorithm. $graph\_branch$ selects the inconsistent arc pair
containing the highest score arc in $BACL$(Branch Arc Candidates
List). $graph\_branch$ calculates the upper bound value for a child
partial-problem by $get\_ub$ and sets it to the child
partial-problem. Simultaneously, $graph\_branch$ executes bound
operation by immediately pruning the child partial-problem whose upper
bound value is less than the incumbent value $z$.

\mygapskip\mynoindent{\bf 
(5) Selection of Partial-problem from Active Partial-problems}

$select\_problem$ in Figure \ref{fig:AlgorithmForLPi} corresponds to
the search $s(A)$ in Figure \ref{fig:BBAlgorithm}. The best bound
search is employed for $select\_problem$, i.e. it selects the
partial-problem which has the maximum bound value among the active
partial-problems. It is known that the number of partial-problems
decomposed during computation is minimized by this strategy in the case
that no dominance tests are applied \cite{Ibaraki78}.

\mygapskip\mynoindent{\bf 
(6) Computing All Optimum Solutions}


In order to obtain all optimum solutions, partial-problems whose upper
bound values are equal to the score of the optimum solution(s) are
expanded at $S8(search\ for\ more\ optimum$ $solutions)$. In the case that at least
one inconsistent arc pair remains in a partial-problem
(i.e. $IAPL{\ne}\{\}$), graph branch is performed based on the
inconsistent arc pair. Otherwise, the obtained optimum solution $FS$
is checked if one of the arcs in $FS$ has an equal rival arc by
$arcs\_with\_alternatives$ function in Figure \ref{fig:BBAlgorithm}. The
equal rival arc of arc $A$ is an arc whose position and score are equal
to those of arc $A$. If an equal rival arc of an arc in $FS$ exists, 
a new partial-problem is generated by removing the arc in $FS$. $S8$
assures that no partial-problem has an upper bound value greater than or equal
to the score of the optimum solutions when the computation stopped. 

\mygapskip\mynoindent{\bf 
(7) Correctness of the Graph Branch Algorithm}

All Dependency trees are generated by the feasible solution and lower
bound value function $get\_fs$. $get\_fs$ does not violate the
covering constraint(the single role constraint and the coverage
constraint) because it selects one arc for one input position at the
$step7$ in Figure \ref{fig:AlgorithmForLPi}. It also assures the
co-occurrence constraint by checking the CM value for every two arcs
in a tree at $step5$. Therefore, output dependency trees of the graph
branch algorithm satisfy the well-formed dependency tree constraint.


\section{Example of Optimum Tree Search}
\label{sec:ExampleOfOptimumTreeSearch}

This section presents an example showing the behavior of the graph
branch algorithm using the dependency forest in
Figure \ref{fig:WDFForExample}.


\subsection{Feasible Solution/Lower Bound Value Function}
The following section shows the behavior of feasible solution/lower bound value function $get\_fs$ for the example sentence.

\begin{figure}[b]
\begin{center}
     \includegraphics[scale=0.7]{OSグループ化,ソートされた依存片JAPENG.eps}
\end{center}
\myfiglabelskippre
\caption{Grouped and Sorted Arcs}
\label{fig:SortedDepPces}
\end{figure}

$step1(grouping\ and\ sorting\ of\ arcs)$ in Figure
\ref{fig:AlgorithmForLPi} is performed once at the beginning for the
initial dependency forest. The result of $step1$ is shown in Figure
\ref{fig:SortedDepPces}. $Pos$ and $MaxScore$ mean the position of the
arc in the sentence and the maximum arc score at that position
respectively. Arcs with no rival arc have $MaxScore$ $\infty$ and are
located at the top of the arc group list. Arc groups with start
positions $3$,$0$,$4$,$1$ and $2$ are assigned to
$S_1$,$S_2$,$S_3$,$S_4$ and $S_5$, respectively. $step2(initialize)$
initializes variables. After $step3$ and $step4$ are executed, $step5$
checks that $a(i,j)=a(1,1)=14(=det14)$ can be registered to $FS$. In
this case, no violation of the co-occurrence constraint occurs, and
then $step7$ registers $a(1,1)$ to $FS$\footnote{In fact, the arc ID
$14$ is registered to $FS$. The $a(i,j)$ form is used here for
clarity.}, then backtrack point $BP[1]$ at the position $i(=1)$ is set
to $j(=1)$.
\[
 FS=[a(1,1)](=[14]), \ BP=[1,-,-,-,-], \ \text{$i=2$, $j=1$, $k=1$, $l=0$}
\]
Next, $step3$-$5$ try the first arc $a(2,1)(=nc2)$ in $S_2$. Since
$CM(a(1,1),a(2,1))=CM(14,2)=\bigcirc$ 
in Figure \ref{fig:WDFForExample},
$a(2,1)$ and $a(1,1)$ satisfy the co-occurrence constraint and then
$a(2,1)$ is registered to $FS$.
\[
 FS=[a(1,1),a(2,1)](=[14,2]),\ BP=[1,1,-,-,-], \ \text{$i=3$, $j=1$, $k=1$, $l=0$}
\]
    $a(3,1)(=pre15)$ is skipped because $CM(a(3,1),a(2,1))=CM(15,2){\ne}\bigcirc$. 
Then $a(3,2)$ is tried.
\begin{align*}
 & FS=[a(1,1),a(2,1),a(3,2)](=[14,2,16])\\
 & BP=[1,1,2,-,-], \ \text{$i=4$, $j=1$, $k=1$, $l=1$}
\end{align*}
In a similar manner, $a(4,1)(=sub23)$ and $a(5,4)(=rt29)$ are added to $FS$, then the termination condition at $step3$ is satisfied.
\begin{align*}
 & FS=[a(1,1),a(2,1),a(3,2),a(4,1),a(5,4)] (=[14,2,16,23,29]),\\
 & BP=[1,1,2,1,4], \ \text{$i=6$, $j=1$, $k=4$, $l=4$},
\end{align*}
The $FS$ here is a feasible solution and the sum total of arc scores,
i.e. $17+17+6+10+0=50$ is the score of the feasible solution.

No backtracking occurred in this example. Backtracking occurs when all
arcs in $S_i$ are found to be inconsistent with either of the arcs in $FS$
at that point. In this case, $step6(backtracking)$ backtracks to the
$l$ position. $l$ is assured to be the rightmost position, where some
element in $S_i$ is inconsistent with the selected arc in $FS$. This
mechanism is introduced to optimize backtracking.


\subsection{Example of Graph Branch Algorithm}

The search process of the branch and bound method can be shown as a
search diagram constructing a partial-problem tree representing the
parent-child relation between the partial-problems. Figure
\ref{fig:SearchDiagramForExample} is a search diagram for the example
dependency forest showing the search process of the graph branch method.

\begin{figure}[bt]
 \begin{center}
     \includegraphics[scale=0.6]{OS例文に対する計算過程を示す探索図ENG.eps}
 \end{center}
\myfiglabelskippre
\caption{Search diagram for the example sentence}
\label{fig:SearchDiagramForExample}
\myfiglabelskippre
\end{figure}

In this figure, box $P_i$ is a partial-problem with its dependency
graph $rem$, upper bound value $UB$, feasible solution and lower bound
value $LB$ and inconsistent arc pair list $IAPL$. Suffix $i$ of $P_i$
indicates the generation order of partial-problems. Updating of global
variable $z$ (incumbent value) and $O$ (set of incumbent solutions) is
shown under the box. The value of the left-hand side of the arrow is
updated to that of right-hand side of the arrow during the
partial-problem processing. Details of the behavior of the algorithm
in Figure \ref{fig:GraphBrachAlgorithm} are described below.

In $S1(initialize)$, $z$, $O$ and $AP$ are set to $-1$,
$\{\}$ and $\{P_0\}$ respectively. The dependency graph of $P_0$ is that
of the example dependency forest. This is represented by
$rem=[]$. $get\_ub$ sets the upper bound value (=63) of $P_0$ to
$UB$. In practice, this is calculated by obtaining the maximum
well-covered arc set of $P_0$. In $S2(search)$, $select\_problem$
selects $P_0$ and $get\_fs(P_0)$ is executed. The feasible solution
$FS$ and its score $LB$ are calculated based on the algorithm in
Figure \ref{fig:AlgorithmForLPi} to set $FS=[14,2,16,23,29]$, $LB=50$ ($P_0$ in the search
diagram). $S3(incumbent\ value\ update)$ updates $z$ and $O$ to new
values. Then, $get\_iapl(P_0)$ computes the inconsistent arc pair list
$[(2,15),(15,23),(23,18),(2,18)]$ from the maximum well-covered arc
set $[14,2,15,23,18]$ and set it to $IAPL$. $S5(maximum\ value\ test)$
compares the upper bound value $UB$ and the feasible solution value
$LB$. In this case, $LB<UB$ holds, so $BACL$ is assigned the value of
$IAPL$. The next step $S6(branch\ operation)$ executes the $graph\_branch$
function. $graph\_branch$ selects the arc pair with the highest arc
score and performs the graph branch operation with the selected arc
pair. The following is a $BACL$ shown with the arc names and arc scores.
\[
 [(nc2[17],pre15[10]),(pre15[10],sub23[10]),(sub23[10],vpp18[9]),(nc2[17],vpp18[9])]
\]

Scores are shown in $[\ ]$. The arc pair containing the highest arc
score is $(2,15)$ and $(2,18)$ containing $nc2[17]$. Here, $(2,15)$ is
selected and partial-problems $P_1(rem[2])$ and $P_2(rem[15])$ are
generated. $P_0$ is removed from $AP$ and the new two partial-problems
are added to $AP$ resulting in $AP=\{P_1,P_2\}$. Then, based on the
best bound search strategy, $S2(search)$ is tried
again. $select\_problem$ selects $P_1$ because the upper bound value of
$P_1$ (=$61$) is greater than that of $P_2$ (=$59$). Since the upper
bound of $P_1$ (=$61$) is greater than the feasible solution score
(=$51$), $get\_iapl$ is executed and sets $BACL$ to the value shown in
$P_1$ in Figure \ref{fig:SearchDiagramForExample}. The graph branch function
$graph\_branch$ gets two candidates for child partial-problems
corresponding to $rem[24,2]$ and $rem[23,2]$ because the inconsistent
arc pair $(24,23)$ is selected as the source of the graph branch
operation (arc $24$ has the highest score of $15$). The former
candidate for $rem[24,2]$ is pruned immediately, because its upper bound
value (=$46$) is smaller than the incumbent value (=$51$) (termination
by the upper bound test). Therefore, $graph\_branch$ returns
$\{P_3(rem[23,2])\}$. The upper bound value $UB$ of $P_3$ is $58$
which is less than that of its parent problem $P_1$. The processing
for $P_1$ is completed and $P_1$ is removed from
$AP$. $select\_problem$ selects $P_2$ by comparing the upper bound
values of $P_2$ and $P_3$ in $AP$. Partial-problem $P_2$ is terminated
because it has no feasible solution ($FS=no\_solution$). Then, the next
partial-problem $P_3$ is processed. $P_3$ has a feasible solution with
a score of $41$. Updating of the incumbent value does not occur because the
obtained score is lower than the existing incumbent value. The next
partial-problem $P_4$ has no feasible solution, so all processing is
terminated at $S8(stop)$. At this time, the values of $O$ and $z$ are the
optimum solution (=$\{[14,24,15,31,18]\}$) and its score (=$51$)
respectively. This solution corresponds to the dependency tree (a) in
Figure \ref{fig:WellFormedTreesForExample}.




\section{Experiment}
\label{sec:ExperimentOfOptSearch}

This section describes some experimental results showing the
computational complexity and behavior of the graph branch algorithm. 

\subsection{Environment and Metric of the Experiment}

An English text corpus, PDG grammar and preference
knowledge are prepared. Preference knowledge source in this
experiment is the WPP frequencies (node frequencies) and the
dependency relation frequencies (arc frequencies) in the
corpus. Preference score is calculated from these statistical data and
attached to the arcs in the dependency graphs. 

Experiment data of 125,320 sentences extracted from English technical
documents is divided into open data (8605 sentences) and closed data
(116,715 sentences). Figure \ref{fig:SentenceDistribution} shows the
distribution of word length of sentences in the open data. The closed
data is used for producing WPP and dependency frequencies. An existing
sentence analysis system (called the oracle system) is used as a
generator of these frequencies. The oracle system is a real-world
rule-based system with a long development history
\cite{Amano89,Hirakawa00}. In order to see a brief accuracy of the
oracle system, 136 sentences are selected randomly but with similar
distribution shown in Figure \ref{fig:SentenceDistribution} from a set
of sentences which are parsable using the basic grammar described
below\footnote{Since unparsable sentences have no output, they are
neglected in the succeeding evaluation experiments. Some extra method
is required for obtaining partial parse trees for unparsable
sentences.}. The arc precision ratio\footnote{Correct arc ratio with
respect to arcs in output dependency trees \cite{Hirakawa05b}.} for
this sentence set with respect to human analysis results is
97.2\%. Therefore, the output of the oracle system is a good
approximation of human correct data.

\begin{figure}[t]
\myfiglabelskippre
 \begin{center}
     \includegraphics[scale=0.66]{EV対象文の語数分布ENG.eps}
 \end{center}
\myfiglabelskippre
\caption{Distribution of sentence length for sentences in the open data}
\label{fig:SentenceDistribution}
\myfiglabelskippre
\end{figure}

A PDG grammar called a basic grammar is prepared. The basic grammar
consists of basic grammar rules which cover sentence variations such
as noun/verb/adjective/adverbial/ prepositional phrases,
simple/complex/compound sentences, relative/subordinate clauses and
Onions' 5 sentence patterns\footnote{S+V,S+V+C,S+V+O,S+V+O+O and
S+V+O+C patterns}. The basic grammar does not accept insertion,
omission, inversion and idiomatic structures (ex. not only .. but also
..). More detailed information on the environment of this experiment
is described in \cite{Hirakawa05b}.  The expanded problem number, a
principal computational complexity factor of the B\&B method, is adopted as the
base metric. The following three metrics are used in
this experiment.

\begin{itemize}
\item[(a)] Expanded Problem Number in Total (EPN-T): The number of the
expanded problems which are generated in the entire search process.

\item[(b)] Expanded Problem Number for the First Optimum
Solution (EPN-F): 
EPN-F is the number of the expanded problems when the first optimum
solution is obtained.

\item[(c)] Expanded Problem Number for the Last Optimum
Solution (EPN-L): EPN-L is the number of the expanded problems when the last optimum
solution is obtained. At this point, all optimum solutions are obtained.

\end{itemize}

{\mynoindent}Optimum solution number (OSN) for a problem, i.e. the
number of optimum dependency trees in a given dependency forest, gives
the lower bound value for all these metrics because one problem
generates at most one solution. The minimum value of OSN is 1 because
every dependency forest has at least one dependency tree. As the search
process proceeds, the algorithm finds the first optimum solution, then
the last optimum solution, and finally terminates the process by
confirming no better solution is left. Therefore, the three metrics and
OSN have the relation OSN ${\leq}$ EPN-F ${\leq}$ EPN-L ${\leq}$
EPN-T. Average values for these are described as Ave:OSN, Ave:EPN-F,
Ave:EPN-L and Ave:EPN-T.

\subsection{Experimental Results}

An evaluation experiment for the open data is performed using a
prototype PDG system implemented in Prolog. The test sentences
containing more than 22 words are neglected due to the limitation of
Prolog system resources in the parsing process. 4334 sentences out of
6882 test sentences are parsable. Without applying special treatment
such as construction of the whole parse tree from partial parse trees,
unparsable sentences (2584 sentences) are simply neglected in this
experiment. 

\begin{figure}[b]
\myfiglabelskippre
 \begin{center}
    \includegraphics[scale=0.57]{OS_GraphBracn評価総問題展開数・第一解の展開数などグラフENG.eps}
 \end{center}
\myfiglabelskippre
\caption{Total problem number, problem number for the first\\ optimum solution and optimum solution number}
\label{fig:OptSolSrcExperimentResult}
\myfiglabelskippre
\end{figure}

All optimum trees are computed by the graph branch algorithm described
in Section
\ref{GraphBranchAlgorithm}. Figure \ref{fig:OptSolSrcExperimentResult}
shows averages of EPN-T, EPN-L, EPN-F and OSN with respect to sentence
length. Overall averages of EPN-T, EPN-L, EPN-F and OSN for the test
sentences are 3.0, 1.67, 1.43 and 1.15. The result shows that the
average number of problems required is relatively small. The feasible
solution search function based on the greedy algorithm in
Figure \ref{fig:AlgorithmForLPi} seems to give a good feasible solution
for a given problem and suppresses the number of expanded problems. The gap
between Ave:EPN-T and Ave:EPN-L (3.0-1.67=1.33) is much greater than
the gap between Ave:EPN-L and Ave:OSN (1.67-1.15=0.52). This means that
the major part of the computation is performed only for checking if
the obtained feasible solutions are optimum or not.


\cite{Hirakawa01} reported the experiment for the B\&B-based optimum
search algorithm implemented in C language using the semantic
dependency graphs obtained from 100 sentences randomly selected from
Japanese technical documents. Compared with the experiment reported in
this paper, the previous experiment was performed in different
conditions and settings with regard to, for example, the target
language(English vs. Japanese), the target dependency
graph(syntactic-label WPP dependency graph vs. semantic-label word
dependency graph), the scoring method(statistics-based vs. rule-based)
and the search target(all optimum solution search vs. one optimum
solution search). However, the two experiments have the same basic
structure, i.e. the optimum tree search in scored dependency graphs
with arc constraints based on the B\&B principle. The B\&B-based
algorithms of the two experiments have very similar components of the
branch and bound method and the main factor of the computational
complexity is the number of the expanded problems. The previous
experiment shows that overall averages of EPN-T, EPN-F are 2.91, 1.33
\footnote{OSN and EPN-L were not measured because the algorithm
searches for only one optimum solution.}. These result values are very
similar to those in the new experiment. The tendency for the optimum
solution to be obtained in the early stage of the search process was
observed in the previous experiment just as it is in this
experiment. 

\cite{Hirakawa01} introduced two improvements of the algorithm,
i.e. the introduction of an improved upper bound function g'(P) and
the optimized feasible solution search. As a result, the Ave:EPN-T is
reduced from 2.91 to 1.82 and the Ave:EPN-F is increased from 1.33 to
1.35. The average CPU time is reduced from 305.8\,ms to 162.1\,ms (on
engineering work station AS-4260). In the new experiment, the g'(P) is
introduced to the graph branch algorithm and has obtained the
reduction of the Ave:EPN-T from 3.00 to 2.68 and the reduction of the
Ave:EPN-F from 1.43 to 1.36. g'(P) is also effective to some extent in
this experiment.

\begin{figure}[bt]
\myfiglabelskippre
 \begin{center}
    \includegraphics[scale=0.82]{OS展開問題数に対する全問題展開・第一最適解・最終最適解展開の充足率ENG.eps}
 \end{center}
\myfiglabelskippre
\caption{Achievement ratios for full expansion, first optimum solution expansion and\\ last optimum solution expansion cases with respect to max problem expansion number}
\label{fig:AchivementRatioWRTMaxProblemExpansionNumber}
\myfiglabelskippre
\end{figure}

The tendency for the optimum solution to be obtained in the early
stage of the search process suggests that limiting the number of
problems to expand is an effective pruning
strategy. Figure \ref{fig:AchivementRatioWRTMaxProblemExpansionNumber}
shows the ratios of the sentences obtaining the whole problem
expansion, the first optimum solution and the last optimum solution to
whole sentences with respect to the expanded problem numbers. This
kind of ratio is called an achievement ratio (AR) in this paper. From
Figure \ref{fig:AchivementRatioWRTMaxProblemExpansionNumber}, the ARs
for EPN-T, EPN-L, EPN-F (plotted in solid lines) are
97.1\%,99.6\%,99.8\% respectively at the expanded solution number 10.
The dotted line shows the AR for EPN-T of the improved algorithm using
g'(P). The use of g'(P) increases the AR for EPN-T from 97.1\% to
99.1\% at the expanded solution number 10. However, the effect of
g'(P) is quite small for EPN-F and EPN-L. ARs for EPN-F and EPN-L in
using g'(P) is almost the same as those shown in
Figure \ref{fig:AchivementRatioWRTMaxProblemExpansionNumber}. This
result shows that the pruning strategy based on the expanded problem
number is effective and g'(P) works for the reduction of the problems
generated in the posterior part of the search processes.

Behavior of the search process should be affected by the scoring
strategy (resources of preference knowledge and their application
methods) and the structure of dependency graphs defined by grammar
rules. The search process should be analyzed in greater detail along
with scoring strategies and dependency graph structures. 

The experiment using the basic English grammar shows a promising
result. This is because EPN-T and EPN-F are similar to those in the
algorithm described in \cite{Hirakawa01} which has a performance
sufficient for real-world applications. The practical code
implementation of the graph branch algorithm and its performance
evaluation with an extended grammar are subjects for future work.





\section{Concluding Remarks}

This paper has described the graph branch algorithm for obtaining the
optimum solution for a dependency forest used in the preference
dependency grammar. The well-formedness dependency tree constraints
are represented by the constraint matrix of the dependency forest,
which has flexible and precise description ability so that controlled
non-projectivity is available in PDG framework. The graph branch
algorithm assures the search for the optimum trees with arbitrary arc
co-occurrence constraints, including the single valence occupation
constraint which has not been treated in DP-based algorithms so
far. The dependency forest has wider applicability compared with the
semantic dependency graph because it can handle whole morphological
ambiguity caused by homonyms and word boundary divisions. The
experimental result shows the averages of EPN-T, EPN-L and EPN-F for
English test sentences are 3.0, 1.67 and 1.43, respectively. This
is a promising result because EPN-T and EPN-F are similar to
those in the algorithm described in \cite{Hirakawa01} which 
has a performance sufficient for real-world applications. The
practical code implementation of the graph branch algorithm and its
performance evaluation with an extended grammar are subjects for
future work.




\bibliographystyle{nlpbbl}
\renewcommand{\refname}{}
\begin{thebibliography}{}

\bibitem[\protect\BCAY{Amano, Hirakawa, Nogami, \BBA\ Kumano}{Amano
  et~al.}{1989}]{Amano89}
Amano, S., Hirakawa, H., Nogami, H., \BBA\ Kumano, A. \BBOP 1989\BBCP.
\newblock \BBOQ The Toshiba Machine Translation System\BBCQ\
\newblock {\Bem Future Computing Systems}, {\Bem 2\/}(3).

\bibitem[\protect\BCAY{Black, Jelinek, Lafferty, Magerman, Mercer, \BBA\
  Roukos}{Black et~al.}{1992}]{Black92}
Black, E., Jelinek, F., Lafferty, J., Magerman, D., Mercer, R., \BBA\ Roukos,
  S. \BBOP 1992\BBCP.
\newblock \BBOQ Towards History-based Grammars: Using Richer Models for
  Probabilistic Parsing\BBCQ\
\newblock In {\Bem Proceedings of the 5th DARPA Speech and Natural Language
  Workshop}, \BPGS\ 134--139.

\bibitem[\protect\BCAY{Charniak}{Charniak}{2000}]{Charniak00}
Charniak, E. \BBOP 2000\BBCP.
\newblock \BBOQ A maximum-entropy-inspired parser\BBCQ\
\newblock In {\Bem Proceedings of the 1st Conference of the North American
  Chapter of the Association for Computational Linguistics}, \BPGS\ 132--139.

\bibitem[\protect\BCAY{Chu \BBA\ Liu}{Chu \BBA\ Liu}{1965}]{ChuLiu65}
Chu, Y., J.\BBACOMMA\  \BBA\ Liu, T., H. \BBOP 1965\BBCP.
\newblock \BBOQ On the shortest arborescence of a directed graph\BBCQ\
\newblock {\Bem Science Sinica}, {\Bem 14}, pp.~1396--1400.

\bibitem[\protect\BCAY{Collins}{Collins}{1999}]{Collins99}
Collins, M. \BBOP 1999\BBCP.
\newblock {\Bem Head-Driven Statistical Models for Natural Language Parsing}.
\newblock Ph.D.\ thesis, University of Pennsylvania.

\bibitem[\protect\BCAY{Edmonds}{Edmonds}{1967}]{Edmonds67}
Edmonds, J. \BBOP 1967\BBCP.
\newblock \BBOQ Optimum branchings\BBCQ\
\newblock {\Bem Journal of Research of the National Bureau of Standards}, {\Bem
  71B}, pp.~233--240.

\bibitem[\protect\BCAY{Eisner}{Eisner}{1996}]{Eisner96}
Eisner, J. \BBOP 1996\BBCP.
\newblock \BBOQ Three new probabilistic models for dependency parsing: An
  exploration\BBCQ\
\newblock In {\Bem Proceedings of COLING'96}, \BPGS\ 340--345.

\bibitem[\protect\BCAY{Harada \BBA\ Mizuno}{Harada \BBA\
  Mizuno}{2001}]{Harada01}
Harada, M.\BBACOMMA\  \BBA\ Mizuno, T. \BBOP 2001\BBCP.
\newblock \BBOQ Japanese semantic analysis system SAGE using EDR (in
  Japanese)\BBCQ\
\newblock {\Bem Transactions of the Japanese Society of Artificial
  Intelligence}, {\Bem 16\/}(1), pp.~85--93.

\bibitem[\protect\BCAY{Hirakawa}{Hirakawa}{2001}]{Hirakawa01}
Hirakawa, H. \BBOP 2001\BBCP.
\newblock \BBOQ Semantic Dependency Analysis Method for Japanese Based on
  Optimum Tree Search Algorithm\BBCQ\
\newblock In {\Bem Proceedings of the PACLING2001}, \BPGS\ 117--126.

\bibitem[\protect\BCAY{Hirakawa}{Hirakawa}{2005a}]{Hirakawa05a}
Hirakawa, H. \BBOP 2005a\BBCP.
\newblock \BBOQ Dependency Forest: Packed Shared Dependency Structure
  Corresponding to Parse Forest (in Japanese)\BBCQ\
\newblock In {\Bem Natural Language Processing NL-167-9,IPSJ}, \BPGS\ 53--62.

\bibitem[\protect\BCAY{Hirakawa}{Hirakawa}{2005b}]{Hirakawa05b}
Hirakawa, H. \BBOP 2005b\BBCP.
\newblock \BBOQ Evaluation Measures for Natural Language Analyser based on
  Preference Dependency Grammar\BBCQ\
\newblock In {\Bem Proceedings of the PACLING2005}, \BPGS\ 153--160.

\bibitem[\protect\BCAY{Hirakawa \BBA\ Amano}{Hirakawa \BBA\
  Amano}{1989a}]{Hirakawa89}
Hirakawa, H.\BBACOMMA\  \BBA\ Amano, S. \BBOP 1989a\BBCP.
\newblock \BBOQ Japanese sentence analysis using syntactic/semantic preference
  (in Japanese)\BBCQ\
\newblock In {\Bem Proceedings of the 3rd National Conference of JSAI}, \BPGS\
  363--366.

\bibitem[\protect\BCAY{Hirakawa \BBA\ Amano}{Hirakawa \BBA\
  Amano}{1989b}]{Hirakawa89b_e}
Hirakawa, H.\BBACOMMA\  \BBA\ Amano, S. \BBOP 1989b\BBCP.
\newblock \BBOQ Method for Searching Optimum Tree in Japanese Sentence Analysis
  (in Japanese)\BBCQ\
\newblock In {\Bem Natural Language Processing NL-74-2,IPSJ}, \BPGS\ 9--16.

\bibitem[\protect\BCAY{Hirakawa, Ono, \BBA\ Yoshimura}{Hirakawa
  et~al.}{2000}]{Hirakawa00}
Hirakawa, H., Ono, K., \BBA\ Yoshimura, Y. \BBOP 2000\BBCP.
\newblock \BBOQ Automatic Refinement of a POS Tagger Using a Reliable Parser
  and Plain Text Corpora\BBCQ\
\newblock In {\Bem Proceedings of the COLING'00}, \BPGS\ 313--319.

\bibitem[\protect\BCAY{Ibaraki}{Ibaraki}{1978}]{Ibaraki78}
Ibaraki, T. \BBOP 1978\BBCP.
\newblock \BBOQ Branch-and-bounding procedure and state-space representation of
  combinatorial optimization problems\BBCQ\
\newblock {\Bem Information and Control}, {\Bem 36,1-27}.

\bibitem[\protect\BCAY{Katoh \BBA\ Ehara}{Katoh \BBA\
  Ehara}{1989}]{Kato_ehara89}
Katoh, N.\BBACOMMA\  \BBA\ Ehara, T. \BBOP 1989\BBCP.
\newblock \BBOQ A fast algorithm for dependency structure analysis (in
  Japanese)\BBCQ\
\newblock In {\Bem Proceedings of 39th Annual Convention of the Information
  Processing Society}.

\bibitem[\protect\BCAY{Lee \BBA\ Choi}{Lee \BBA\ Choi}{1997}]{Lee97}
Lee, S.\BBACOMMA\  \BBA\ Choi, K., S. \BBOP 1997\BBCP.
\newblock \BBOQ Reestimation and Best-First Parsing Algorithm for Probablistic
  Dependency Grammars\BBCQ\
\newblock In {\Bem Proceedings of the Fifth Workshop on Very Large Corpora},
  \BPGS\ 41--55.

\bibitem[\protect\BCAY{Maruyama}{Maruyama}{1990}]{Maruyama90}
Maruyama, H. \BBOP 1990\BBCP.
\newblock \BBOQ Constraint Dependency Grammar and Its Weak Generative
  Capacity\BBCQ\
\newblock {\Bem Computer Software}.

\bibitem[\protect\BCAY{McDonald, Crammer, \BBA\ Pereira}{McDonald
  et~al.}{2005}]{McDonald05}
McDonald, R., Crammer, K., \BBA\ Pereira, F. \BBOP 2005\BBCP.
\newblock \BBOQ Spanning Tree Methods for Discriminative Training of Dependency
  Parsers\BBCQ\
\newblock Technical report, UPenn CIS.

\bibitem[\protect\BCAY{Mel'uk}{Mel'uk}{1988}]{Meluk88}
Mel'uk, I., A. \BBOP 1988\BBCP.
\newblock {\Bem Dependency Syntax: Theory and Practice}.
\newblock State University of New York Press, Albany, New York.

\bibitem[\protect\BCAY{Nivre \BBA\ Nilsson}{Nivre \BBA\
  Nilsson}{2005}]{Nivre05}
Nivre, J.\BBACOMMA\  \BBA\ Nilsson, J. \BBOP 2005\BBCP.
\newblock \BBOQ Pseudo-Projective Dependency Parsing\BBCQ\
\newblock In {\Bem Proceedings of the 43rd Meeting of the ACL (ACL-05)}, \BPGS\
  99--106.

\bibitem[\protect\BCAY{Nivre \BBA\ Scholz}{Nivre \BBA\
  Scholz}{2004}]{NivreAndScholz04}
Nivre, J.\BBACOMMA\  \BBA\ Scholz, M. \BBOP 2004\BBCP.
\newblock \BBOQ Deterministic Dependency Parsing of English Text\BBCQ\
\newblock In {\Bem Proceedings of COLING'04}, \BPGS\ 64--70.

\bibitem[\protect\BCAY{Ozeki}{Ozeki}{1994}]{Ozeki94}
Ozeki, K. \BBOP 1994\BBCP.
\newblock \BBOQ Dependency Structure Analysis as Combinatorial
  Optimization\BBCQ\
\newblock {\Bem Information Sciences}, {\Bem 78(1-2)}, pp.~77--99.

\bibitem[\protect\BCAY{Ozeki \BBA\ Zhang}{Ozeki \BBA\ Zhang}{1999}]{Ozeki99_e}
Ozeki, K.\BBACOMMA\  \BBA\ Zhang, Y. \BBOP 1999\BBCP.
\newblock \BBOQ 最小コスト分割問題としての係り受け解析(Kakari-uke analysis as
  minimum cost partitioning problem) (In Japanese)\BBCQ\
\newblock In {\Bem Proceeding of the Workshop of The Fifth Annual Meeting of
  The Association for Natural Language Processing}, \BPGS\ 9--14.

\bibitem[\protect\BCAY{Tomita}{Tomita}{1991}]{Tomita91}
Tomita, M. \BBOP 1991\BBCP.
\newblock {\Bem Generalized LR Parsing}.
\newblock Kluwer Academic Publishers, Boston, MA.

\bibitem[\protect\BCAY{Wang \BBA\ Harper}{Wang \BBA\ Harper}{2004}]{Wang04}
Wang, W.\BBACOMMA\  \BBA\ Harper, M., P. \BBOP 2004\BBCP.
\newblock \BBOQ A Statistical Constraint Dependency Grammar (CDG) Parser\BBCQ\
\newblock In {\Bem Workshop on Incremental Parsing: Bringing Engineering and
  Cognition Together (ACL)}, \BPGS\ 42--49.

\bibitem[\protect\BCAY{Yamada \BBA\ Matsumoto}{Yamada \BBA\
  Matsumoto}{2003}]{YamadaAndMatsumoto03}
Yamada, H.\BBACOMMA\  \BBA\ Matsumoto, Y. \BBOP 2003\BBCP.
\newblock \BBOQ Statistical dependency analysis with support vector
  machine\BBCQ\
\newblock In {\Bem Proceedings of IWPT'03}, \BPGS\ 195--206.

\end{thebibliography}

\begin{biography}

\biotitle{}

\bioauthor{Hideki Hirakawa}
{

Hideki Hirakawa received the B.E. and M.E. degrees in electrical
engineering from Kyoto University, Kyoto, Japan, in 1978 and 1980,
respectively. In April 1980, he joined the Toshiba Corp, Kawasaki,
Japan. From 1982 to 1984, he was a researcher at ICOT (Institute for
New Generation Computer Technology), Tokyo, Japan. From 1994 to 1995,
he worked as a research affiliate of the MIT Media Laboratory, Boston,
U.S.A. He is currently a Chief Research Scientist of the Knowledge
Media Laboratory, Corporate Research \& Development Center, Toshiba
Corp., Kawasaki, Japan. He is a member of the Association for Natural
Language Processing of Japan, the Information Processing Society of
Japan, the Japanese Society for Artificial Intelligence, the Gengo
Shigen Kyoukai and the Association for Computational Linguistics.  His
current research interests include natural language processing,
knowledge processing and human interface.

}

\bioreceived{Received}
\biorevised{Revised}
\bioaccepted{Accepted}

\end{biography}


\end{document}
