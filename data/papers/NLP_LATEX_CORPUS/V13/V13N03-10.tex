



    \documentstyle[epsf,jnlpbbl]{jnlp_j_b5_2e}

\setcounter{page}{243}
\setcounter{巻数}{13}
\setcounter{号数}{3}
\setcounter{年}{2006}
\setcounter{月}{7}
\受付{2005}{4}{19}
\再受付{2005}{7}{30}
\再々受付{2006}{3}{6}
\採録{2006}{3}{13}

\setcounter{secnumdepth}{2}

\title{日本語発話文における敬語の誤用を指摘するシステムの開発}
\authorC{白土 保\affiref{NICT} 
\and 丸元 聡子\affiref{IBS}
\and 村田 真樹\affiref{NICT}
\and 井佐原 均\affiref{NICT}
}

\headauthor{白土,丸元,村田,井佐原}
\headtitle{日本語発話文における敬語の誤用を指摘するシステムの開発}

\affilabel{NICT}{情報通信研究機構}
{National Institute of Information and Communications Technology}
\affilabel{IBS}{計量計画研究所}
{The Institute of Behavioral Sciences}

\jabstract{現代の日本社会において,日本語の敬語に関する様々な誤用が指摘されてきている.日本社会における敬語の誤用は,言語によるコミュニケーションを通じた社会的人間関係の構築を妨げる場合がある.敬語の誤用を避けるには,敬語の規範に関する正しい知識の習得が不可欠である.このような知識習得を効率的に行うため,敬語学習を支援する計算機システムの実現が期待される.このような背景の下,我々は日本語発話文に含まれる語形上の誤用,及び運用上の誤用を指摘するシステムを開発した.本システムは,日本語発話文,及び発話内容に関係する人物間の上下関係を表すラベルを入力とし,入力された日本語発話文における誤用の有無,及び誤用が含まれる場合にはその箇所と種類を出力する.発話に関わる人数は最大4名まで取り扱うことができる.正例,及び負例を用いた実験によってシステムの妥当性を検証したところ,一部のケースを除き,本システムが妥当な出力を行うことが確認された.本システムは,特に敬語の初学者に対する学習支援システムとして有用と考えられるが,その他の人々にとっても,文書作成における敬語の語形のケアレスミスをチェックする等の用途として幅広く活用できると考えられる.}

\jkeywords{日本語,敬語,誤用,計算機支援言語学習}

\etitle{System for Pointing Out Honorific Misusages\\
 in Japanese Speech Sentences}
\eauthor{Tamotsu Shirado\affiref{NICT} 
\and Satoko Marumoto\affiref{IBS}
\and Masaki Murata\affiref{NICT}
\and Hitoshi Isahara\affiref{NICT}
} 

\eabstract{In Japan, politeness plays an important role in social activities, especially in conversations. However, honorific Japanese expressions are increasingly being misused. This misusage is a failure to use the honorific expressions in a way appropriate to the relative social positions assumed in a conversation. One of the causes of this misusage may be a lack of education on honorific conversations. Because honorific expressions take a long time to learn, computer assisted language learning systems for honorific expressions should be developed. We developed a computational system to check the usages of honorific expressions in Japanese speech sentences. The system can point out misused words and phrases, and can also indicate how they have been misused. The validity of the system was tested using ``correct'' sentences including no misused expressions, and ``incorrect'' sentences including misused expressions. The system was able to point out all the misusages in the incorrect sentences. It also judged most of the correct sentences as ``correct'' except some cases.}

\ekeywords{Japanese, Honorific Expression, Misusage, Computer assisted language learning}

\begin{document}
\maketitle
\thispagestyle{empty}


\section{はじめに}
敬語は日本語の重要な特徴の一つとされており,日本語の敬語は単に依頼,要求あるいは人を示す代名詞において見られるだけでなく,言語体系,及び言語行動のほぼ全般にわたって発達している.このような特徴を持つ言語は日本語以外では,韓国語,チベット語,及びジャワ語等世界中に少数しか見られない\cite{Hayashi1974}.

ところが現代の日本社会において,日本語の敬語に関する様々な誤用が指摘されてきている\cite{Kikuchi1997,Ishino1986}.日本社会における敬語の誤用は,言語によるコミュニケーションを通じた社会的人間関係の構築を妨げる場合がある.特にビジネスの場面における敬語の誤用は,時として円滑なビジネスを進める上での障害にもなり得る.このため,一般的には敬語の誤用はできるだけ避けることが望ましい.敬語の誤用を避けるには,敬語の規範に関する正しい知識の習得が不可欠である.このような知識習得を効率的に行うため,敬語学習を支援する計算機システムの実現が期待される.

以上の背景の下,我々は日本語発話文に含まれる語形上の誤用,及び運用上の誤用を指摘するシステムを開発した.本システムは,日本語発話文,及び発話内容に関係する人物間の上下関係を表すラベルを入力とし,入力された日本語発話文における誤用の有無,誤用の箇所,及び誤用の種類(後者二つは誤用有りの場合のみ)を出力する.最近ではこれに類似した機能を搭載した日本語入力支援ツール等が開発されてきてはいるが,既存のシステムは主に語形上の誤用の一部のみを対象としており,運用上の誤用についても極めて限られた表現しか扱うことができなかった.

本システムのように,発話文に含まれる敬語の誤用を指摘するシステムの構築にあたっては,(1)敬語の規範を何処に求めるか?及び(2)発話状況をどう取り扱うか?が問題になる.本研究では,以下の考え方に基づきこれらの問題に対処している.

(1)敬語の規範

敬語(正確には,敬語を含む言語一般)は時代の経過と共に変化する.例えば,``お話になられる''(二重敬語)等は伝統的な日本語学においては誤用とされてきたが,近年では必ずしも誤用としては認識しない人が少なからずいることが報告されている\cite{Bunkacho1999}.このため,現代の日本において幅広く社会のコンセンサスが得られている敬語の体系的規範はないと考えられる.

この問題に対し本研究では,日本語学に関する様々な文献において共通して明示的あるいは暗示的に述べられていると解釈できる規範にできるだけ厳密に準拠する,という立場を取る.このため,現代の日本社会において敬語として概して許容されている表現であっても,本システムではその表現を規範的な敬語として見なさない可能性がある.しかしこのことは,少しでも誤用の可能性のある表現をできるだけ漏らさずピックアップできる,という利点として考えることもできる.

(2)発話状況の取り扱い

従来の敬語研究で指摘されているように,発話状況に応じた適切な敬語(即ち,運用上正しい敬語)を選択する際に考慮すべき主な要因には,発話に関わる人物間の上下関係(年齢差や社会的地位の違いに基づき話者が判断した上下関係,以下では``主観的上下関係''と呼ぶ),人物間の親疎,人物間のウチソト,及び各人物の体面に対して発話意図が及ぼすリスク,等がある.中でも人物間の主観的上下関係は,敬語が誤用である否かを判断する際の最も重要な要因であることが,日本語の敬語に関する多くの文献において明記あるいは暗示的に述べられている\cite[等]{Kikuchi1996,Kikuchi1997,Kabaya1998,Kokugoken1990,Kokugoken1992,Minami1987}.一方,このような判断の際に,人物間の親疎,人物間のウチソト,あるいは各人物の体面に対して発話意図が及ぼすリスクが,主観的上下関係より重要な要因であることを指摘した文献は殆どない.このことは,敬語の運用の規範に関わる要因としては,主観的上下関係が最も重要な要因であることを示唆する.従って本研究においては,敬語の運用上の規範を発話に関わる人物間の主観的上下関係のみに基づいて定義する.尚,実際の場面では人物間の主観的上下関係が殆ど同じ状況も想定されるため,実用的なシステムのためにはこのような状況も扱えることが望ましいが,今回は誤用指摘システム開発の最初のステップとして,明確な主観的上下関係の下での規範に焦点を当てることとし,上下関係が殆ど同じ状況の取り扱いは今後の課題としている.

\bigskip

以下では,本研究における``敬語の誤用''の定義を述べた後,それに基づいた誤用指摘システムについて述べる.更に,様々なテストデータを用いたシステムの妥当性の検証,及びシステムの今後の改善点等について述べる.

\section{敬語の誤用}
前述のように,本システムは(1)語形上の誤用,及び(2)運用上の誤用を指摘することができる.敬語の誤用は,大きくこの2種類に分けられる.

(1)語形上の誤用

語形が敬語として規範的ではない表現の使用.本研究では,敬語の文献\cite[等]{Kikuchi1996,Kikuchi1997,Horikawa1969,Miyaji1999}を参考に語形上の誤用を定義した.本システムで用いている語形誤り表現リストの一部を表1に示す.表中``〜''は任意の動詞を表す.語形誤り表現リストに登録されている表現の総数(``〜''を含むパターンの表現はパターンとして1個とカウント)は約80個である.尚,一般的に日本語として誤っているか否かという観点からは語形上の誤用を無数に想定することができるが,本システムでは特に敬語に関わる表現の中で最も典型的な語形上の誤用を登録している.


\begin{table}[htbp]
\begin{center} 
\caption{語形誤り表現リスト(一部)} 
\label{tbl:table1} 
\begin{tabular}{ll} 
\hline 
\multicolumn{1}{c}{表現(``〜''は動詞)} & \multicolumn{1}{l}{誤用のタイプ} \\ 
\hline
お/ご〜になられる&二重敬語\\
お/ご〜なされる&二重敬語\\
お/ご〜をなされる&二重敬語\\
お/ご〜される&尊敬語形+謙譲語形の混用\\
\hline 
\end{tabular} 
\end{center} 
\end{table} 

(2)運用上の誤用

語形は正しいが,発話に関わる人物間の主観的上下関係と整合しない表現の使用.伝統的な日本語学では,例えば,話者より社会的に上に位置付けられる聞き手に対する発話文の文末は丁寧にすべきとされている.従って,このような状況における発話文として,文末が丁寧でない文は運用上の誤用と見なされる.発話に関わる人物間の主観的上下関係に基づいた運用上の規範をどのように定義するかは,本システムが対象とする発話文の特徴にも依存するため,次章(システムの説明)で述べる.

\section{敬語誤用指摘システム}
\subsection{本システムが対象とする発話文}
本システムに入力される発話文は,以下三つの制約を満たすものとする.

\begin{description}
\item [制約1]発話文には述語が一つだけ(述語の主語,補語\footnote{主語以外の格要素を指す.研究者によっては他の語(例えば,``目的語'',``主格補語''等)を用いることがあるが,本研究では「敬語教育の基本問題(下)」\cite{Kokugoken1992}に倣ってこの語を用いる.}はそれぞれ一つ)含まれる.
\item [制約2]発話に関係する人数は2名〜4名.ここで,2名の時は話者(名前は``S''とする),聞き手(名前は``L''とする),3名の時は話者,聞き手,及び発話文中で参照される人物1名(名前は``A''とする),4名の時は話者,聞き手,及び発話文中で参照される人物2名(4人目の名前は``B''とする)とする.
\item [制約3]話者/聞き手/人物A/人物Bが述語の主語あるいは補語の場合はその人物の名前``S''/``L''/``A''/``B''を明記する.
\end{description}

\bigskip

以上の制約は,現行の構文解析システムや意味解析システムを用いた際の,文構造の解析(特に,述語の主語,補語の同定等)における解析エラーの生じる可能性があるような複雑な文を排除するために設けた.従って,高い精度の文解析手法が将来開発されれば,これらの制約はより緩やかにできると考えられる.

尚,発話に関わる人数が5名以上の状況はまれである(ここで,``彼ら''等のように複数人からなるグループの場合は,グループを擬人化して1名相当とみなす)と考えられるため,日常用いられている発話文の殆どは制約2を満たすと考えられる.
また,発話に関わる人物が4名(S,L,A,B)の際,AとBを一つのグループ(例えば,``AさんとBさん'')としては扱わないものとする.従って,発話に関わる人物が4名の場合はAが主語でBが補語のケースのみであり(Bが主語でAが補語のケースはこれと等価),話者(S)や聞き手(L)が主語や補語になる状況は想定しない.一方,発話に関わる人物が3名(S,L,A)の場合はSやLが主語や補語になる場合があり,このときAは主語あるいは補語のいずれかになる.発話に関わる人物が2名(S,L)の場合は,Sが主語でLが補語のケース,及びLが主語でSが補語のケースのみである.

制約3で記したように本システムでは発話に関わる人の名前を``S'',``L'',``A'',``B''に固定しているが,人名の情報等を事前にシステムに登録することによって人名を直接取り扱うことも可能である.

\subsection{敬語特徴パターン}
伝統的な日本語学の文献の多くにおいて,敬語は尊敬語,謙譲語,及び丁寧語に概ね分類できるとされている.尊敬語の中で最も典型的な語は,尊敬語を表すための形式を持つ述語,及び人物に対する敬称(例えば,``様'')である.謙譲語の中で最も典型的な語は,謙譲語を表すための形式を持つ述語である.丁寧語は主に文末における丁寧な語(例えば,``〜です.\unskip'')を指す.

前述のように,本システムが対象とする発話文には述語が一つのみ現れる.従って,このような発話文の敬語的な特徴は,述語の主語の敬称の有/無,述語の補語の敬称の有/無,文末が丁寧/丁寧でない,及び述語の敬語的な特徴(尊敬語/謙譲語/二方面敬語\footnote{尊敬語かつ謙譲語である語のこと.例えば,``ご説明して下さった''(``ご説明する''が謙譲語,``下さった''が尊敬語)等.}/尊敬語でも謙譲語でもない語)で表すことができると考えられる.本研究ではこれを``敬語タイプ''と呼ぶ.このため本システムでは,発話文を形態素解析して得た形態素の並びに対して,表2に示すような敬語タイプ辞書を用いて表3に示すパターンを作り,このパターンで発話文の敬語的特徴を表すこととする(以下,このパターンを``敬語特徴パターン''と呼ぶ).


\begin{table}[htbp]
\begin{center} 
\caption{敬語タイプ辞書の一部(``〜''は動詞)} 
\tabcolsep = 2em
\begin{tabular}{ll} 
\hline 
\multicolumn{1}{c}{形態素の部分的並び} & \multicolumn{1}{c}{敬語タイプ} \\ 
\hline
``A''+``さん''&敬称\\
``B''+``氏''&敬称\\
``L''+``様''&敬称\\
``お''+ 〜 +``する''&謙譲語\\
``ご''+ 〜 +``する''&謙譲語\\
``頂く''&謙譲語\\
``申しあげる''&謙譲語\\
``お''+ 〜 +``なさる''&尊敬語\\
``ご''+ 〜 +``に'' + ``なる''&尊敬語\\
``おっしゃる''&尊敬語\\
``いらっしゃる''&尊敬語\\
``です''+``.''&丁寧語\\
``ます''+``.''&丁寧語\\
\hline 
\end{tabular} 
\end{center} 
\end{table} 

敬語タイプ辞書に登録されている表現の総数(``〜''を含むパターンの表現はパターンとして1個とカウント.また``頂く''/``いただく''等の異表記は別個のものとしてカウント)は約250個である.また表3に示すように,敬語特徴パターンは4つの要素$s$,$o$,$e$,$p$からなる.要素$s$は,述語の主語(表中``$subj$''と標記)の敬称の有/無,に応じて1/0の値を取る.要素$o$は,述語の補語(表中``$obj$''と標記)の敬称の有/無,に応じて1/0の値を取る.要素$e$は,発話文の文末が丁寧/丁寧でない,に応じて1/0の値を取る.要素$p$は,述語の敬語タイプに応じて0/1/2/3の値を取る.尚,``$subj$''/``$obj$''は3.4節で述べる文構造解析によって述語の主語/補語と同定された人物を指す語である.

\begin{table}[htbp]
\begin{center} 
\caption{敬語特徴パターンの各要素の定義} 
\label{tbl:table3} 
\begin{tabular}{ll} 
\hline 
敬語特徴パターンの各要素の値&条件\\ 
\hline
$s=0$&$subj$の敬称なし\\
$s=1$&$subj$の敬称あり\\
\hline 
$o=0$&$obj$の敬称なし\\
$o=1$&$obj$の敬称あり\\
\hline 
$e=0$&文末が丁寧でない\\
$e=1$&文末が丁寧\\
\hline 
$p=0$&述語が尊敬語でも謙譲語でもない\\
$p=1$&述語が尊敬語\\
$p=2$&述語が謙譲語\\
$p=3$&述語が尊敬語かつ謙譲語(二方面敬語)\\
\hline 
\end{tabular} 
\end{center} 
\end{table} 

\subsection{入出力の形式}
システムの入出力の例を図1に示す.図中,記号``A$>$B$>$L$>$S''は,話者(S),聞き手(L),人物A,及び人物Bの主観的上下関係を表すラベル(以下,``上下関係ラベル''と記す)である.``$>$''の左側に現れた記号に対応する人物は右側に現れた記号に対応する人物より主観的上下関係が上であるものと定義する.本システムでは,上下関係ラベルが表す人物間の主観的上下関係は常に正しいと想定して処理を行っている.上下関係ラベルと共に,聞き手(L)に対する話者(S)の発話文が入力される.この例では,発話意図:``AがBに言った''に対応する発話文が入力されている.システムは入力された発話文の語形上の誤用,及び発話文と上下関係ラベルとの整合性をチェックし,誤用の箇所が見つかった場合には,その箇所,及び誤用の種類を出力する.この例では,語形上の誤用は見つからなかったが,話者(S)より主観的上下関係が上の聞き手(L)に対する発話文において文末が丁寧ではないため,``誤用''と判定されている.

\begin{figure}[htbp]
\begin{center}
\fbox
{
\begin{minipage}{60mm}
\baselineskip=4mm
 \\
[入力]\\ 
A$>$B$>$L$>$S\\
AさんがBさんにおっしゃったそうだ.\\
\\
[出力]\\
判定: $誤用$\\
誤用の箇所: $文末$\\
誤用の種類: $文末が丁寧ではない$\\
\end{minipage}
}
\caption{入出力の例}
\label{fig:figure1}
\end{center}
\end{figure}


\subsection{処理の流れ}
本システムにおける処理のフローを図2に示す.システムに入力された発話文はまず形態素解析され,形態素の並びに変換される(形態素解析には茶筌\footnote{奈良先端大が開発した形態素解析システム\\http://chasen.aist-nara.ac.jp/index.html.ja}を用いた).次に,語形誤り表現リスト(表1)を用いて,語形上の誤用がチェックされる.例えば,形態素の並びの中に,\mbox{``お''}+動詞+``に''+``なる''+``れる''が現れた場合には,語形上の誤用:``お〜になられる''であると判断し,その旨を出力して処理を終了する.

語形上の誤用が見つからなかった場合には,文構造解析によって述語の主語と補語の同定を行う.文構造解析では,形態素の並びの異なる箇所に【人物を指す語(+敬称)+格助詞(``が''/``を''/``に''/``から''等)】が2箇所現れるようなテンプレートを始め様々なテンプレートを用意することによって主語と補語を同定する.テンプレートの総数は約140個である.具体的なテンプレートの例を表4に示す.表中``○'',``△''は人物を指す語,``〜''は任意の文字列を表す.例えば図1に示した発話文に対しては,``AさんがBさんに''に対する形態素の並び,即ち【``A'' ``さん''(敬称) ``が'' ``B'' ``さん''(敬称) ``に''】が表4の最初に記されたテンプレートと一致するので,主語がA,補語がBと同定される.

\begin{figure}[htbp]
\begin{center}
\epsfxsize=11cm  
\epsffile{./fig2.eps} 
\caption{処理のフロー}
\label{fig:figure2}
\end{center}
\end{figure}

\begin{table}[htbp]
\begin{center} 
\caption{テンプレートの例(``○'',``△''は人物を指す語,``〜''は任意の文字列)} 
\label{tbl:table4} 
\tabcolsep = 2em
\begin{tabular}{lll} 
\hline 
形態素の並び&主語&補語\\ 
\hline
〜 ○ {敬称} が 〜 △{敬称} に 〜&○&△\\
〜 ○ {敬称} が 〜 △ に 〜&○&△\\
〜 ○ が 〜 △ に 〜& ○&△\\
〜 ○ が 〜 △ から 〜& ○&△\\
〜 △ {敬称} に 〜 ○ {敬称} が 〜&○&△\\
\hline 
\end{tabular} 
\end{center} 
\end{table} 

文構造解析の結果に基づき,敬語タイプ辞書(表2),及び敬語特徴パターンの各要素の定義(表3)を用いて,敬語特徴パターンを抽出する.例えば,図1に示した発話文に対する形態素の並びは【``A'' ``さん''(敬称) ``が'' ``B'' ``さん''(敬称) ``に'' ``おっしゃる'' ``た'' ``そう'' ``だ'' ``.''】となるが,これは主語(A)の敬称があり(``A'' ``さん''),補語(B)の敬称があり(``B'' ``さん''),文末が丁寧でなく(``だ'' ``.''),述語が尊敬語(``おっしゃる'')であるので,$s=1$, $o=1$, $e=0$, $p=1$となる.最後に,整合表(表5)を用いて,敬語特徴パターンと上下関係ラベルとの整合性がチェックされ,判定結果が出力される(図2中の「補足ルール」については3.6節で述べる).

\begin{table}[htbp]
\begin{center} 
\caption{整合表} 
\label{tbl:table5} 
\begin{tabular}{ll} 
\hline 
敬語特徴パターンの各要素の値 & 上下関係\\ 
\hline
$s=0$&S$>subj$\\
$s=1$&$subj>$S\\
\hline 
$o=0$&S$>obj$\\
$o=1$&$obj>$S\\
\hline 
$e=0$&S$>$L\\
$e=1$&L$>$S\\
\hline 
$p=0$&S$>subj$ $\wedge$ S$>obj$\\
$p=1$&$subj>obj$ $\wedge$ $subj>$S\\
$p=2$&$obj>$S$>subj$(i.e. $obj>$S $\wedge$ S$>subj$)\\
\hline 
$p=1$ or 3&$obj>subj>$S (i.e. $obj>subj$ $\wedge$ $subj>$S)\\
\hline 
\end{tabular} 
\end{center} 
\end{table} 

整合表(表5)は,部分的上下関係と,それに対応した文の敬語特徴パターンにおける特定の要素が持つべき値との間の対応を記したルールである.ここで``部分的上下関係''とは,発話に関わる人物のうち一部の人物の間の主観的上下関係を表したものであり,上下関係ラベルと同じ表記法で記される.例えば表5の$e=1$に関する行は,話者(S)より聞き手(L)の主観的上下関係が上の場合(L$>$S)には,文末は丁寧($e=1$)とする,というルールを表している.

図1の例では,A($subj$)$>$B($obj$)$>$L$>$Sと$s=1$, $o=1$, $e=0$, $p=1$との整合性がチェックされる.この時,$sub>obj>$L$>$Sは$s=1$, $o=1$, $p=1$とは整合するが,L$>$Sと$e=0$は整合しない.従って,$e=0$に対応する文中の箇所(即ち文末)が運用上の誤用と判定される.

尚,ユーザによる入力の便宜のため,上下関係ラベルは省略可能とした.この場合には,過去に入力された上下関係ラベルのうち最後の上下関係ラベルを用いることとした.

\subsection{整合表の構築}
整合表(表5)は,学習データに基づいて構築した.学習データの例を図3に示す.学習データの基本的な単位は本システムの入力と同じ形式のデータ,即ち上下関係ラベルと制約1〜3を満たす発話文のペアである(記号``:''は入力文と上下関係ラベルのセパレータ).学習データでは,発話に関わる人数を4名(S,L,A,B)に固定した.前述の通り,このときAが主語($subj$)でBが補語($obj$)のケースのみ想定される(Bが主語でAが補語のケースはこれと等価).

\begin{figure}[htbp]
\begin{center}
\fbox
{
\begin{minipage}{80mm}
\baselineskip=4mm
 \\
AはBさんからお聞きしたそうです.:L$>$B$>$S$>$A\\
AがBさんのところに伺いました.:L$>$B$>$S$>$A\\
\end{minipage}
}
\caption{学習データの例}
\label{fig:figure3}
\end{center}
\end{figure}

発話文は,上下関係ラベルが表す人物間の上下関係に対応する文として規範的に正しいと考えられる文であり,具体的には日本語学に関する様々な文献\cite[等]{Ishino1986,Kabaya1998,Kikuchi1996,Kikuchi1997,Kokugoken1990,Kokugoken1992,Suzuki1984,Hoshino1993,Horikawa1969,Masuoka1989,Miyaji1999,Moriyama2000}において共通して明示的あるいは暗示的に述べられていると解釈できる規範に基づいて我々が作成した.上下関係ラベルのバリエーションはS,L,A,Bの間の全ての上下関係バリエーション(即ち,4!=24通り)である.各々の上下関係ラベルに対し,発話文を12個ずつ作成した.従って,学習データの総数は288個(=24×12)である.

このようにして作った学習データに対して,以下の手続きによって整合表を構築した.この手続きは人手で行った.

\bigskip

[整合表構築手続き]
\begin{description}
\item[(Step 1)]学習データにおける各々のペア:{上下関係ラベル,発話文}において,発話文の敬語特徴パターンを求め,その発話文とペアとなっていた上下関係ラベルとの間でペアを作る.敬語特徴パターンは3.4節と同様の手順,即ち発話文の形態素解析,文構造解析を行い,更に敬語タイプ辞書(表2),及び敬語特徴パターンの各要素の定義(表3)を用いることにより求める.
\item[(Step 2)]24種類の上下関係ラベルの各々に関し,以下を行う.\\
Step 1で得たペア:{上下関係ラベル,敬語特徴パターン}(計288個)の中からその上下関係ラベルとペアになっている敬語特徴パターンの全ての種類をリストアップする.
\item[(Step 3)]全ての上下関係ラベルに関しStep 2で得られたリストをまとめて一つの表にする.ここで,一つの上下関係ラベルに複数の敬語特徴パターンが対応する場合は,同じ行の異なる列に記す(このようにして得られた表が表6).
\item[(Step 4)]敬語特徴パターンの要素$s$, $o$, $e$, $p$の各々に関し,以下を行う.
\item[(Step 4-1)]注目している要素の各々の値に関し,Step 3で得られた表(表6)においてこの値を含む敬語特徴パターンに対応する全ての上下関係ラベルをピックアップしてひとまとまりのグループにする.ただし要素pに関しては,表6においてpの値だけが異なるような二つの敬語特徴パターンを持つ上下関係ラベル,即ちL$>$B$>$A$>$S,B$>$A$>$L$>$S,B$>$L$>$A$>$S,B$>$A$>$S$>$Lをひとまとまりのグループにした後,$p=$0,1,2の各々についてこの処理を行う.
\item[(Step 4-2)]Step 4-1で得られた各グループにおいて,同じグループに含まれる上下関係ラベルの全てに共通し,かつ異なるグループに含まれるいかなる上下関係ラベルとも共通しないような特徴(部分的上下関係)を見つける.これには,カルノー図(Karnaugh 1953)を用いて冗長な論理式から簡潔な論理式を導く方法と同様の方法を用いる.
\item[(Step 5)]敬語特徴パターンの全ての要素に関して得られた部分的上下関係をまとめて一つの表にする.
\item[(Step 6)]表中のAを$subj$,Bを$obj$に置き換える(このようにして得られた表が表5).
\end{description}

\begin{table}[htbp]
\begin{center} 
\caption{上下関係ラベルと敬語特徴パターンの対応表} 
\label{tbl:table6} 
\begin{tabular}{lll} 
\hline 
\multicolumn{1}{c}{上下関係ラベル}&\multicolumn{1}{c}{$s$ $o$ $e$ $p$}&\multicolumn{1}{c}{$s$ $o$ $e$ $p$}\\ 
\hline
S$>$L$>$A$>$B&0 0 0 0&\\
S$>$L$>$B$>$A&0 0 0 0&\\
S$>$A$>$L$>$B&0 0 0 0&\\
S$>$A$>$B$>$L&0 0 0 0&\\
S$>$B$>$L$>$A&0 0 0 0&\\
S$>$B$>$A$>$L&0 0 0 0&\\
L$>$S$>$A$>$B&0 0 1 0&\\
L$>$S$>$B$>$A&0 0 1 0&\\
L$>$A$>$S$>$B&1 0 1 1&\\
L$>$A$>$B$>$S&1 1 1 1&\\
A$>$B$>$L$>$S&1 1 1 1&\\
A$>$L$>$B$>$S&1 1 1 1&\\
L$>$B$>$S$>$A&0 1 1 2&\\
B$>$L$>$S$>$A&0 1 1 2&\\
L$>$B$>$A$>$S&1 1 1 1&1 1 1 3\\
B$>$A$>$L$>$S&1 1 1 1&1 1 1 3\\
B$>$L$>$A$>$S&1 1 1 1&1 1 1 3\\
A$>$B$>$S$>$L&1 1 0 1\\
A$>$S$>$B$>$L&1 0 0 1&\\
A$>$S$>$L$>$B&1 0 0 1&\\
A$>$L$>$S$>$B&1 0 1 1&\\
B$>$A$>$S$>$L&1 1 0 1&1 1 0 3\\
B$>$S$>$A$>$L&0 1 0 2&\\
B$>$S$>$L$>$A&0 1 0 2&\\
\hline 
\end{tabular} 
\end{center} 
\end{table} 

例えば,表5の$s=0$,及び$s=1$に対応する行は以下のようにして得られる.まずStep 1〜Step 3で表6が得られる.次にStep 4で$s$に注目し,Step 4-1において,$s=0$に対応する上下関係ラベルのグループ:S$>$L$>$A$>$B,S$>$L$>$B$>$A,S$>$A$>$L$>$B,S$>$A$>$B$>$L,S$>$B$>$L$>$A,S$>$B$>$A$>$L,L$>$S$>$A$>$B,L$>$S$>$B$>$A,L$>$B$>$S$>$A,B$>$L$>$S$>$A,B$>$S$>$A$>$L,B$>$S$>$L$>$A及び$s=1$に対応する上下関係ラベルのグループ:L$>$A$>$S$>$B,L$>$A$>$B$>$S,A$>$B$>$L$>$S,A$>$L$>$B$>$S,L$>$B$>$A$>$S,B$>$A$>$L$>$S,B$>$L$>$A$>$S,A$>$B$>$S$>$L,A$>$S$>$B$>$L,A$>$S$>$L$>$B,A$>$L$>$S$>$B,B$>$A$>$S$>$Lが得られる.次にStep 4-2によって,$s=0$に対応する部分的上下関係:S$>$A,及び$s=1$に対応する部分的上下関係:A$>$Sが得られる.最後にAを$subj$に置き換えることにより,$s=0$に対応する部分的上下関係:S$>$$subj$,及び$s=1$に対応する部分的上下関係:$subj$$>$Sが得られる.

\bigskip

整合表の妥当性を検証するため,テストデータを用いた実験を行った.テストデータは整合表の構築に用いた学習データと同じ文献を用い,同じ要領で作成した.即ちテストデータでは,発話に関わる人数を4名(S,L,A,B)に固定し,上下関係ラベルのバリエーションを24通りとした.各々のバリエーションに関し,正例(上下関係ラベル+規範的な文),負例(上下関係ラベル+非規範的な表現を含む文)をそれぞれ6個ずつ作成した.従って,正例,負例の総数はいずれも144個である.

これらのテストデータをシステムに入力したところ,全ての正例に対してシステムは``正しい''と判定し,全ての負例に対して``誤用''と判定した(誤用の箇所,種類も正しく指摘).この結果は,本研究の枠組み(敬語の規範に関する考え方,発話状況の取り扱い,及び対象とする文に関する制約)の下で,発話に関わる人物が4名(この時,Aが主語かつBが補語,あるいはBが主語かつAが補語)の場合には,敬語運用上の規範を整合表(表5)が過不足なく表していることを示唆する.

また3.1節で述べたように,発話に関わる人物が3名(S,L,A)の場合は,Aは主語あるいは補語のいずれかになるが,この時SやLが主語や補語になるケースがある.また発話に関わる人物が2名(S,L)の場合は,Sが主語でLが補語のケース,及びLが主語でSが補語のケースがある.従って,整合表(表5)がこのようなケースでも有効に機能するようにするため,整合表の補足ルールを導入した.そして整合表と補足ルールが,発話に関わる人数,及び主語・補語のあらゆるバリエーションについて妥当か否かについての検証を行った.以下ではこれらについて述べる.

\subsection{補足ルール}
前述の通り,発話に関わる人物が4名の場合は,話者(S)や聞き手(L)が主語($subj$)や補語($obj$)になる状況は想定しないが,発話に関わる人物が2名,及び3名の場合はSやLが主語や補語になる場合がある.特にSが主語あるいは補語になる場合には,整合表(表5)における敬語特徴パターンの要素$s$,$o$,及び$p$に対応する上下関係が``$subj$$>$$subj$''や``$obj$$>$$obj$''となるケースが生じるため整合表をそのまま用いることができない.このためこのようなケースでは,``$subj$$>$$subj$''や``$obj$$>$$obj$''の項を整合表から削除(この処理は,これらの項の真理値を真とした場合と等価)した上で,以下に述べる規範から直接導いたルール,及び規範に基づき整合表を修正して導いたルール(以下これらを``補足ルール''と総称する)を用いることとする.この規範は整合表の構築に用いた文献を参考にして設けたものであり,その基本的な考えは「発話に関わる人物間の上下関係によらず,話者自身を高めるような語は用いない\footnote{例えば,「俺様は」等は扱わない.}」ということである.これを具体的に記述すると,規範a:話者自身の敬称はつけない,規範b:話者が主語の時は尊敬語を用いない,規範c:話者が補語の時は謙譲語を用いない,ということになる.尚,補足ルールは整合表において$subj$や$obj$を含むルールに関連する特徴敬語パターンの要素(即ち,$s$,$o$,及び$p$)に対するルールであり,整合表において対応する上下関係に$subj$や$obj$を含まない要素eについては適用しない(即ち,$e$に関しては整合表を用いる).具体的に,補足ルールは以下のように導かれる.

まずSが$subj$の時,規範aにより直接$s=0$(上下関係によらず)が導かれる.また$p$に関しては,規範bにより$p=1$,及び$p=3$がルール化の対象から除外され,更に整合表(表5)の``S$>subj$''の項を削除することにより,$p=0$(S$>obj$)/$p=2$($obj>$S)が得られる.尚,$o$は整合表通り$o=0$ (S$>obj$) / $o=1$ ($obj>$S)である.

一方Sが$obj$の時,規範aにより直接$o=0$(上下関係によらず)が導かれる.またpに関しては,まず整合表(表5)の中で論理値が偽となる上下関係が記述された行,即ち整合表の一番下の行(部分的上下関係は,$obj>subj$ $\wedge$ $subj>$S)をルール化の対象から除外する.次に規範cにより$p=2$がルール化の対象から除外され,更に``S$>obj$'',及び``$obj>$S''の項を削除することにより,$p=0$(S$>subj$)/$p=1$($subj>$S)が得られる.尚,$s$は整合表通り$s=0$ (S$>subj$)/$s=1$ ($subj>$S)である.以上をまとめると,補足ルールは下記となる(整合表通りのルールは省略).

\bigskip

[補足ルール]
\begin{enumerate}
\item Sが$subj$の時
\begin{itemize}
\item $s=0$(上下関係によらず)
\item $p=0$(S$>obj$)/$p=2$($obj>$S)
\end{itemize}
\item Sが$obj$の時
\begin{itemize}
\item $o=0$(上下関係によらず)
\item $p=0$(S$>subj$)/$p=1$($subj>$S)
\end{itemize}
\end{enumerate}

\bigskip

\subsection{我々が作成したテストデータを用いた妥当性の検証}
整合表と補足ルールを合わせた機能の妥当性を検証するため,テストデータを用いた実験を行った.テストデータは整合表の妥当性の検証に用いたテストデータと同様の方法で作成した.ただしここでは,発話に関わる人数を2名(S,L),3名(S,L,A),及び4名(S,L,A,B)とし,それぞれの人数において,上下関係ラベル,及び主語・補語に関する全てのバリエーションを設定した.具体的には,人数が2名の場合は,上下関係ラベル2通り×主語・補語のバリエーション2通り=計4通り,人数が3名の場合は,上下関係ラベル6通り×主語・補語のバリエーション4通り(ここで,Aは主語あるいは補語)=計24通り,人数が4名の場合は,上下関係ラベル24通りとした(Aは主語,Bは補語に固定).各々のバリエーションに関し,正例(上下関係ラベル+規範的な文),負例(上下関係ラベル+非規範的な表現を含む文)をそれぞれ5個ずつ作成した.従って,正例,負例の総数はいずれも260個である(正例,負例の一部を図4に示す).尚,文作成の際想定する発話意図として,複数の発話意図をあらかじめ用意した.

\begin{figure}[htbp]
\begin{center}
\fbox
{
\begin{minipage}{80mm}
\baselineskip=4mm
 \\
(正例)\\
SがLから聞いたんだよ.:S$>$L\\
AがSに電話をかけてきたよ.:S$>$L$>$A\\
AはBさんをご自宅までお送りした.: B$>$S$>$A$>$L\\

(負例)\\
SがLに電話をかけました.:S$>$L\\
AさんはSにお写真を見せてくださった.:L$>$A$>$S\\
AはBからお聞きしました.:L$>$S$>$B$>$A\\
\end{minipage}
}
\caption{実験に用いた正例,負例の一部}
\label{fig:figure4}
\end{center}
\end{figure}


これらのテストデータをシステムに入力したところ,全ての正例に対してシステムは``正しい''と判定し,全ての負例に対して``誤用''と判定した(誤用の箇所,種類も正しく指摘).この結果は,整合表(表5),及び補足ルールの妥当性を示唆する.

\section{システムの妥当性の検証}
本システムの妥当性(ただし,語形上の誤用に関する機能は主に語形誤り表現リストの適切さに依存するため,ここでは運用上の誤用に関する機能のみを対象とする)を更に検証するため,著者らの研究グループとは独立した日本語学の研究グループ(言語に関するデータ作成の専門機関に所属し,日本語学を主な専門とする研究グループであり,メンバは3〜5名)が作成したテストデータを用いた実験を行った.テストデータの作成に当たっては,様々な日本語学研究者の間で共通して述べられていると作業者が認識する敬語規範にできるだけ厳密に準拠し,3.7節で述べたテストデータと同じバリエーションに関し,3.1節の制約1〜3に従う発話文を記述するよう指示した.即ち,テストデータのバリエーションは,人数が2名の場合は4通り,人数が3名の場合は24通り,人数が4名の場合は24通りである.各々のバリエーションに関し,正例,負例はそれぞれ10〜12個ずつ作成された.正例,負例の総数はいずれも616個である.

これらのテストデータをシステムに入力したところ,全ての負例に対し,システムは``誤用''と判定した(誤用の箇所,種類も正しく指摘).また,正例のうち587個(正例の約95%)に対し,システムは``正しい''と判定したが,残り29個(正例の約5%)については``誤用''と判定した.

図5は,正例に対してシステムが``正しい''と判定した例である.

\begin{figure}[htbp]
\begin{center}
\fbox
{
\begin{minipage}{35mm}
\baselineskip=4mm
 \\
入出力例:\\
[入力]\\ 
L$>$S$>$A\\
SがAに言いました.\\
 \\
[出力]\\
判定:文は正しい\\
\end{minipage}
}
\caption{正例を``正しい''と判定した例}
\label{fig:figure5}
\end{center}
\end{figure}

また,システムが負例の誤用を正しく指摘した例を図6に示す.図6の入出力例1では,尊敬語でも謙譲語でもない語(この例では,``言った''等)にすべき述語が謙譲語(この例では,\mbox{``お伝えした'')}であったため誤用と指摘されている.また入出力例2では,部分的上下関係がB$>$Sであるにも関わらずBの敬称がない,という誤用が指摘されている.

\begin{figure}[htbp]
\begin{center}
\fbox
{
\begin{minipage}{65mm}
\baselineskip=4mm
 \\
入出力例1:\\
[入力]\\
S$>$A$>$L\\
AがLにお伝えしたはずだよ.\\
 \\
[出力]\\
判定:誤用\\
誤用の箇所: お伝えした\\
誤用の種類: 述語が謙譲語になっている\\

\bigskip

入出力例2:\\
[入力]\\
A$>$B$>$L$>$S\\
AさんがBのところへいらっしゃいました.\\
 \\
[出力]\\
判定:誤用 \\
誤用の箇所: B\\
誤用の種類: Bの敬称がない\\
\end{minipage}
}
\caption{負例の誤用を正しく指摘した例}
\label{fig:figure6}
\end{center}
\end{figure}

システムによって``誤用''と判定された正例29個は以下3種類に分けられる(それぞれの種類に対応する3つの入出力例を図7に示す).


(1)図7の入出力例1のように,人物の敬称はつけないことが整合表で期待される上下関係($subj$の敬称をつけないことが期待される上下関係はS$>subj$あるいはSが$subj$.また$obj$の敬称をつけないことが期待される上下関係はS$>obj$あるいはSが$obj$)において敬称がつけられていた(この例では,``Aさん'')正例は17個.

(2)図7の入出力例2のように,尊敬語(例えば,``おっしゃる''等)あるいは二方面敬語(例えば,``お話しして下さる''等)の述語が期待される上下関係($obj>subj>$S)において述語が謙譲語(この例では,``申し上げる'')だった正例は6個.

(3)図7の入出力例3のように,尊敬語でも謙譲語でもない述語(例えば,``説明する''等)が期待される上下関係(Sが$subj$でも$obj$でもない時はS$>subj$ $\wedge$ S$>obj$.Sが$subj$の時はS$>obj$.Sが$obj$の時はS$>subj$)において述語が謙譲語(この例では,``説明いたす'')だった正例は6個.この6個はいずれも部分的上下関係がL$>$S(即ち,話者より聞き手の主観的上下関係が上)のケースであった

\begin{figure}[htbp]
\begin{center}
\fbox
{
\begin{minipage}{65mm}
\baselineskip=4mm
 \\
入出力例1:\\
[入力]\\
S$>$L$>$A\\
SがAさんに言ったんだ.\\
 \\
[出力]\\
判定:誤用\\
誤用の箇所: Aさん\\
誤用の種類: Aに不要な敬称がある\\

\bigskip

入出力例2:\\
[入力]\\
L$>$A$>$S\\
AさんがLさんに申し上げました.\\
 \\
[出力]\\
判定:誤用\\
誤用の箇所: 申し上げる\\
誤用の種類: 述語が謙譲語になっている\\

\bigskip

入出力例3:\\
[入力]\\
L$>$S$>$A\\
SがAに説明いたしました.\\
 \\
[出力]\\
判定:誤用\\
誤用の箇所: 説明いたす\\
誤用の種類: 述語が謙譲語になっている\\
\end{minipage}
}
\caption{正例を``誤用''と判定した例}
\label{fig:figure7}
\end{center}
\end{figure}

\bigskip

以上の結果のうち(1)は特に女性が用いる傾向がある丁寧な用法であり,一概には非規範的と決めつけることはできないと考えられる.(2)は,謙譲語として妥当な運用法であるとする解釈がある(例えば,国語研 1992).また(3)は聞き手への丁重さを表すための表現であり,「主語を低める」という,いわゆる謙譲語の用法とは異なる用法(``丁重語''等と呼ばれることがある)として一般的になりつつあるが,今回は最も基本的と考えられる敬語の用法のみを対象としたため,このような表現を適切に取り扱えなかったものと考えられる.

以上から,著者らの研究グループとは独立した日本語学の研究グループが作成したテストセットを用いた場合でも,本システムは概ね妥当な出力を行うことが確認された.ただし,一部の表現に対してはより詳細な取り扱いが必要であることが示唆された.

\section{むすび}
日本語発話文,及び発話に関わる人物間の主観的上下関係を表すラベルを入力とし,入力された日本語発話文における誤用の有無,及び誤用が含まれる場合にはその箇所と種類を出力するシステムを開発した.本システムは,敬語における主要な2種類の誤用,即ち語形上の誤用,及び運用上の誤用の両方を指摘することができる.また発話に関わる人数としては,2名,3名,及び4名を取り扱うことができる.正例,及び負例を用いた実験によってシステムの妥当性を検証したところ,一部のケースを除き,本システムが妥当な出力を行うことが確認できた.

本システムの構築に当たっては,日本語学に関する様々な文献において共通して明示的あるいは暗示的に述べられていると解釈できる規範にできるだけ厳密に準拠するよう心がけた.本システムは,日本語を勉強中の外国人や日本の子供が敬語の最も基本的な知識を学習するためのツールとして有用であると考えられる.またこのような人々に限らず,語形上の誤用をチェックすることのできる本システムの機能は,文書作成における敬語の語形のケアレスミスをチェックする等の用途として幅広く活用できると考えられる.

一方,本システムで用いている整合表と補足ルールが表す敬語規範は,実際の社会で運用されている敬語規範に比べ,概して厳しい可能性がある.従って,より実用的な観点からは,社会言語学的調査等によって実際の運用を反映した整合表と補足ルールを獲得し,現在用いているものとの間で適宜使い分ける,あるいは両者を何らかの形で融合することが望ましいと考えられる.また実際の場面では,発話に関わる人物間の主観的上下関係がない(即ち,社会的地位が等しいと見なされる)ような状況がしばしば生じる.実用的な観点からは,このような状況に対処するルールの追加が必要であると考えられる.更に敬語学習の便宜のためには,誤用の指摘だけでなく正しい表現の候補を例示することが望ましい.このためには,発話意図や文としての自然さを損なわないような表現の選択法等を検討する必要があると考えられる.

今後は,以上の課題の検討,機能拡張を行うことによって,より実用的なシステムの構築を目指す.





\bibliographystyle{jnlpbbl}
\begin{thebibliography}{}
\bibitem[\protect\BCAY{Brown}{Brown}{1987}]{Brown1987}
Brown, P. \BBACOMMA\ \BBA\ Levinson, S. \BBOP 1987\BBCP.
\newblock{\Bem  Politeness - Some universals of language usage -}
\newblock{Cambridge.}

\bibitem[\protect\BCAY{Karnaugh}{Karnaugh}{1953}]{Karnaugh1953}
Karnaugh, M. \BBOP 1953\BBCP.
\newblock \BBOQ The Map Method for Synthesis of Combinational Logic Circuits\BBCQ\
\newblock {\Bem Trans. AIEE}, {\Bbf 72} (9).

\bibitem[\protect\BCAY{石野}{石野}{1986}]{Ishino1986}
石野博史 \BBOP 1986\BBCP.
\newblock \JBOQ 敬語の乱れ—誤用の観点から—\JBCQ
\newblock 文化庁「ことば」シリーズ24 続敬語.

\bibitem[\protect\BCAY{蒲谷, 川口, 坂本}{蒲谷}{1998}]{Kabaya1998}
蒲谷宏,川口義一,坂本惠 \BBOP 1998\BBCP.
\newblock \JBOQ 敬語表現 \JBCQ\
\newblock 大修館書店.

\bibitem[\protect\BCAY{菊池}{菊池}{1997}]{Kikuchi1997}
菊地康人 \BBOP 1997\BBCP.
\newblock \JBOQ 敬語\JBCQ\
\newblock 講談社.

\bibitem[\protect\BCAY{菊池}{菊池}{1996}]{Kikuchi1996}
菊地康人 \BBOP 1996\BBCP.
\newblock \JBOQ 敬語再入門\JBCQ\
\newblock 丸善ライブラリー.

\bibitem[\protect\BCAY{国語審議会}{国語審議会}{1952}]{Kokugoshingikai1952}
国語審議会 \BBOP 1952\BBCP.
\newblock \JBOQ これからの敬語\JBCQ\
\newblock 文部大臣に対する建議.

\bibitem[\protect\BCAY{国語研}{国語研}{1990}]{Kokugoken1990}
国立国語研究所 \BBOP 1990\BBCP
\newblock \JBOQ 日本語教育指導参考書17 敬語教育の基本問題(上)\unskip\JBCQ\
\newblock 大蔵省印刷局.

\bibitem[\protect\BCAY{国語研}{国語研}{1992}]{Kokugoken1992}
国立国語研究所 \BBOP 1992\BBCP
\newblock \JBOQ 日本語教育指導参考書18 敬語教育の基本問題(下)\unskip\JBCQ\
\newblock 大蔵省印刷局.

\bibitem[\protect\BCAY{鈴木,林}{鈴木}{1984}]{Suzuki1984}
鈴木一彦,林巨樹編 \BBOP 1984\BBCP
\newblock \JBOQ 研究資料日本文法9 敬語法編\JBCQ\
\newblock 明治書院.

\bibitem[\protect\BCAY{林,南}{林}{1973}]{Hayashi1973}
林四郎,南不二男編 \BBOP 1973\BBCP
\newblock \JBOQ 敬語講座6 現代の敬語\JBCQ\
\newblock 明治書院.

\bibitem[\protect\BCAY{林,南}{林}{1974}]{Hayashi1974}
林四郎,南不二男編 \BBOP 1974\BBCP
\newblock \JBOQ 敬語講座1 敬語の体系\JBCQ\
\newblock 明治書院.

\bibitem[\protect\BCAY{林,南}{林}{1974}]{Hayashi1974}
林四郎,南不二男編 \BBOP 1974\BBCP
\newblock \JBOQ 敬語講座8 世界の敬語\JBCQ\
\newblock 明治書院.

\bibitem[\protect\BCAY{文化庁}{文化庁}{1999}]{Bunkacho1999}
文化庁文化部国語課 \BBOP 1999\BBCP
\newblock \JBOQ 世論調査報告 平成10年度 国語に関する世論調査 −敬語・漢字・外来語−\JBCQ\
\newblock 文化庁.

\bibitem[\protect\BCAY{星野,丸山}{星野}{1993}]{Hoshino1993}
星野和子,丸山直子編 \BBOP 1993\BBCP
\newblock \JBOQ 日本語の表現\JBCQ\
\newblock 圭文社.

\bibitem[\protect\BCAY{堀川,林}{堀川}{1969}]{Horikawa1969}
堀川直義,林四郎 \BBOP 1969\BBCP
\newblock \JBOQ 敬語用例中心ガイド\JBCQ\
\newblock 明治書院.

\bibitem[\protect\BCAY{益岡,田窪}{益岡}{1989}]{Masuoka1989}
益岡隆志,田窪行則 \BBOP 1989\BBCP
\newblock \JBOQ 基礎日本語文法 −改訂版−\JBCQ\
\newblock くろしお出版.

\bibitem[\protect\BCAY{南}{南}{1987}]{Minami1987}
南不二男 \BBOP 1987\BBCP
\newblock \JBOQ 敬語\JBCQ\
\newblock 岩波書店.

\bibitem[\protect\BCAY{宮地}{宮地}{1999}]{Miyaji1999}
宮地裕 \BBOP 1999\BBCP
\newblock \JBOQ 敬語・慣用句表現論 −現代語の文法と表現の研究(二)−\JBCQ\
\newblock 明治書院.

\bibitem[\protect\BCAY{森山}{森山}{2000}]{Moriyama2000}
森山卓郎 \BBOP 2000\BBCP
\newblock \JBOQ ここからはじまる日本語文法\JBCQ\
\newblock ひつじ書房.

\bibitem[\protect\BCAY{渡辺}{渡辺}{1971}]{Watanabe1971}
渡辺実 \BBOP 1971\BBCP
\newblock \JBOQ 国語構文論\JBCQ\
\newblock 塙書房.
\end{thebibliography}
\begin{biography}


\bioreceived{受付}
\biorevised{再受付}
\biorerevised{再々受付}
\bioaccepted{採録}

\end{biography}

\end{document}
