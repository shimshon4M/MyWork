    \documentclass[japanese]{jnlp_1.3a}
\usepackage{jnlpbbl_1.1}
\usepackage[dvips]{graphicx}
\usepackage{amsmath}
\setcounter{secnumdepth}{3}


\Volume{14}
\Number{3}
\Month{Apr.}
\Year{2007}
\received{2006}{4}{19}
\revised{2006}{8}{16}
\accepted{2006}{9}{21}

\setcounter{page}{219}


\jtitle{連想メカニズムを用いた話者の感情判断手法の提案}
\jauthor{土屋 誠司\affiref{Sanyo} \and 吉村枝里子\affiref{doshisha} \and 渡部 広一\affiref{doshisha} \and 河岡  司\affiref{doshisha}}
\jabstract{
我々は,人とのコミュニケーションの仕組みを機器とのインタフェースとして実現することを目標に研究を行っている.人間は会話をする際に意識的または無意識のうちに,様々な常識的な概念をもって会話を展開している.このように会話文章から常識的な判断を行い,適切に応答するためには,ある語から概念を想起し,さらに,その概念に関係のある様々な概念を連想できる能力が重要な役割を果たす.本稿では,ある概念から様々な概念を連想できるメカニズムを基に,人間が行う常識的な判断の一つである感情に関する判断を実現する方法について提案している.「主体語」,「修飾語」,「目的語」,「変化語」の4要素から成るユーザの発話文章から,そのユーザの感情を基本感情10種類,補足感情24種類で判断する手法を提案している.また,本手法を用いた感情判断システムを構築し,その性能を評価した結果,常識的な解の正答率は76.5{\kern0pt}%,非常識ではない解を正答率に含めると88.0{\kern0pt}%であり,提案した処理手法は有効であると言える.
}
\jkeywords{感情判断,知識ベース,概念ベース,関連度}

\etitle{The Method of the Emotion Judgment Based on \\an Association Mechanism}
\eauthor{Seiji Tsuchiya\affiref{Sanyo} \and Eriko Yoshimura\affiref{doshisha} \and \\
	Hirokazu Watabe\affiref{doshisha} \and Tsukasa Kawaoka\affiref{doshisha}} 
\eabstract{
A human-like common sense and judgment is necessary to materialize a computer that can take communication with human. Because, when people talk to each other, they have the concept of emotion in our mind consciously or unconsciously. In the case, the ability to call concept in mind and to associate with many referred concepts will be an important matter. This paper proposes the method to systemize judgment concerning emotion, based on the mechanism to associate concept with many other referred concepts. As a result, the percentage of correct answers of the emotion judgment system is approximately 88.0\%. Therefore, the emotion judgment system using the technique proposed in this paper is an effective system.
}
\ekeywords{emotion judgment, knowledge base, concept base, degree of association}

\headauthor{土屋,吉村,渡部,河岡}
\headtitle{連想メカニズムを用いた話者の感情判断手法の提案}

\affilabel{Sanyo}{徳島大学大学院ソシオテクノサイエンス研究部}{
	Institute of Technology and Science, The University of Tokushima}
\affilabel{doshisha}{同志社大学大学院工学研究科}{
	Graduate School of Engineering, Doshisha University}



\begin{document}
\maketitle



\section{はじめに}

近年,機器の高機能化がますます進み,我々の生活は非常に便利になってきている.
しかし一方では,
それらの機器を使いこなせないユーザが増えてきていることもまた事実である.この原因としては,高機能化に伴い,機器の操作が複雑化していることが考えられる.この問題を解決する一つの手段に,新しいユーザインタフェースの開発を挙げることができる.これまでにも,音声認識や手書き文字認識など,日常生活で慣れ親しんでいる入力を扱うことによる使いやすい機器の開発がなされており,一定の成果を挙げてはいるが,未だ万人に受け入れられるインタフェースとしては完成していない.これは,入力されたデータを規則に沿って処理しているだけであり,ユーザが置かれている状況や立場・気持ちを理解することなく,単純に処理していることにより,便利であるはずのインタフェースが,かえって人に不便さや不快感を与える結果になっていることが原因であると考えられる.

そこで,我々は,新しいインタフェースとして,人間のコミュニケーションの仕組み,特に,常識的な判断の実現を目標に研究を行っている.人間はコミュニケーションにおいて,あいまいな情報を受け取った場合にも,適宜に解釈し円滑に会話を進めることができる.これは,人間が長年の経験により,言語における知識を蓄積し,その基本となる概念に関する「常識」を確立しているからである.人間が日常的に用いている常識には様々なものがある.例えば,言葉の論理性に関する常識,大きさや重さなどの量に関する常識,季節や時期などの時間に関する常識,暑い・騒がしい・美味しい・美しいといった感覚に関する常識,嬉しい・悲しいといった感情に関する常識などを挙げることができる.これらの常識を機器に理解させることができれば,ユーザは人とコミュニケーションをとるように機器をごく自然に使いこなすことができると考えられる.

これまでにも,前述した常識に関する判断を実現する手法についての研究がなされている\cite{horiguchi:02,watabe:04,kometani:03,tsuchiya:05}.そこで本稿では,これらの常識の中の感情に着目し,ユーザの発話文章からそのユーザの感情を判断する手法を確立し,実システムによりその有効性を検証する.本システムにより例えば,提供しようとしている内容にユーザが不快感を覚える表現や不快な事象を想起させるような内容が含まれている場合に,別の適切な表現に変更することができるなどの効果が期待できる.

本稿のように,感情に主眼を置いた研究はこれまでにもなされている.例えば,イソップワールドを研究の対象に置き,「喜び」,「悲しみ」など8種類の感情に応じた特徴を現在の状況から抽出し,それら複数の特徴を組み合わせることによってエージェントの感情を生成させる研究がある\cite{okada:92,okada:96,tokuhisa:98}.この手法では,エージェントの処理を内部から監視することによって,感情生成のための特徴を抽出している.また,\cite{mera:02}では,語彙に対する好感度を利用し,発話文章から話者の快・不快の感情を判断している.これらの先行研究では,あらかじめ知識として獲得している語彙以外は処理を行うことができない.また,判断できる感情の種類が少なく,表現力に乏しいという問題点が挙げられる.一方,本稿で提案する手法では,連想メカニズムを利用することにより,知識を獲得している語彙との意味的な関連性を評価することができ,知識として獲得していない語彙に関しても適切に処理を行うことが可能であると共に,多彩な感情を判断できることに独自性・優位性があると考えられる.


\section{感情判断システム}\label{system}


\begin{figure}[b]
\begin{center}
    \includegraphics[width=6cm]{14-3ia12f1.eps}
\end{center}
\caption{感情判断システムの構成}
\label{emotion_judgment_system}
\end{figure}

\begin{table}[b]
\caption{本研究で対象とする発話文章の例}
\label{example_of_hatsuwabunnshou}
\begin{center}
\begin{tabular}{|c|c|c|c|c|} \hline
発話文章 & 主体語 & 修飾語 & 目的語 & 変化語 \\ \hline\hline
私は綺麗な宝石を貰う & 私 & 綺麗な & 宝石 & 貰う \\ \hline
私はお化けが怖い & 私 & - & お化け & 怖い \\ \hline
\end{tabular}
\end{center}
\end{table}

構築した感情判断システムの構成を図\ref{emotion_judgment_system}に示す.本研究はまだ初期段階であることから,その基本となる条件の下で研究を遂行した.そのため,話者の感情を判断するための発話文章の形式を「主体語」,「修飾語」,「目的語」,「変化語」の4要素に限定した.表\ref{example_of_hatsuwabunnshou}に本研究で処理の対象とする発話文章の例を示す.

「主体語」とは,発話文章の主体となる名詞である.本研究では,今後,研究を発展する際に基本となる発話者自身,つまり,「私」を主体とする文章に限定した.

「修飾語」とは,後に続く「目的語」を修飾する形容詞・形容動詞である.「修飾語」に関しては,文章表現において必ずしも必要でない場合があるため,省略を許可している.

「目的語」とは,主体の行為・行動・状態の対象となる名詞である.以下,先に説明した「修飾語」と「目的語」を合わせて「対象語」と呼ぶ.

「変化語」とは,主体の行為・行動・状態を表現する動詞や形容詞・形容動詞である.

これらの「主体語」,「対象語」,「変化語」を基に発話者の感情を判断する.感情判断知識ベースには,「対象語」の「修飾語」,「変化語」,「感情判断」に関する少数の知識が登録されており,これを基に,語の連想を行うことにより,知識を常識の範囲で拡張し,多くの表現に対応している.語の連想は,複数の電子化辞書等から機械的に自動構築された大規模なデータベースである概念ベース\cite{hirose:02,kojima:02}と,語と語の間にある関連性を評価する関連度計算法\cite{watabe:01}(以下,これらを合わせて連想メカニズムと呼ぶ)を用いることにより実現している.「対象語」の「目的語」に関しては,名詞が持っている感覚・知覚の特徴を抽出することができる感覚・知覚判断メカニズム\cite{horiguchi:02,watabe:04,kometani:03}を用いて処理を行っている.

人間の抱く感情はこれまで,心理学者や哲学者などによって数多く研究されてきた\cite{hukui:90,saitou:86,rita:99,suzan:01}.しかし,感情には実態がなく,非常にあいまいなものであるため,研究者ごとに解釈が異なり,定義する感情モデルも皆様々である.例えば,「嫌悪」,「恐怖」,「怒り」,「愛」を感情の基本と置き,色を混ぜ合わせるように感情を多様に表現できると定義するもの\cite{rita:99}や,人間が表現できる顔の表情から「悲しみ」,「満足」,「嫌悪」,「怒り」,「恐怖」が基本的な感情であると定義するもの\cite{suzan:01}などがある.そこで,我々は,「あるアクションが起こった際に瞬間的に感じる」ものを感情とみなし,判断する感情を「喜び」,「悲しみ」,「怒り」,「安心」,「恐れ」,「落胆」,「恥」,「後悔」,「罪悪感」と「感情なし」の計10種類と定義した.なお,我々は常識的な判断を実現するシステムの開発を目指しているため,人の好き嫌いに左右される「嫌悪」は判断の対象として扱わないことにした.また,これら10種類の感情を基本感情と定義し,より詳細な感情表現を可能とする手法も提案する.これについては,\ref{jugdement_emotion}章で詳しく説明する.


\ref{association_mechanism}章で連想メカニズム,\ref{taishougo}章で対象語,\ref{hennkago}章で変化語,\ref{jugdement_emotion}章で感情判断について述べ,\ref{result_of_emotion_judgement_system}章で感情判断システムの性能評価を行う.

なお,本研究では,人間の常識を機器上で表現し,扱うことを目標にしているため,人間の常識的な考え方・感じ方を基準にデータベースの構築や評価を行っている.また,処理性能としては,正答率8割以上を目標値として設定している.



\section{連想メカニズム}\label{association_mechanism}
連想メカニズムは概念ベースと関連度計算法により構成されており,概念ベース\cite{hirose:02,kojima:02}は,ある語から語意の展開を行い,関連度計算法\cite{watabe:01}は,語意の展開結果を利用し,語の間にある関連性を数値として表す手法である.


\subsection{概念ベース}\label{consept_base}
概念ベースは,複数の電子化辞書などから各見出し語を概念,その見出し語の説明文中の自立語を概念の属性として,機械的に自動構築された大規模なデータベースである.本研究では,機械的に構築した後,人間の感覚からは不適切である属性を削除し,必要な属性を追加する自動精錬処理を行った概念ベース(概念数約9万)\cite{hirose:02}を利用している.

\begin{figure}[b]
\begin{center}
    \includegraphics[width=10cm]{14-3ia12f2.eps}
\end{center}
\caption{概念「電車」を二次属性まで展開した場合の例}
\label{concept_base}
\end{figure}

概念ベースにおいて,任意の概念$A$は,概念の意味特徴を表す属性$a_i$と,この属性$a_i$が概念$A$を表す上でどれだけ重要かを表す重み$w_i$の対で表現される.概念$A$の属性数を$N$個とすると,概念$A$は以下のように表すことができる.ここで,属性$a_i$を概念$A$の一次属性と呼ぶ.
\[
A = \{ (a_1, w_1), (a_2, w_2), \cdots, (a_N, w_N) \}
\]

概念$A$の一次属性$a_i$は概念ベースに定義されている概念としているため,$a_i$からも同様に属性を導くことができる.$a_i$の属性$a_{ij}$を概念$A$の二次属性と呼ぶ.
概念「電車」を二次属性まで展開した様子を図\ref{concept_base}に示す.



\subsection{関連度計算法}\label{ra}
関連度とは,概念と概念の関連の強さを定量的に評価するものであり,具体的には概念連鎖により概念を二次属性まで展開したところで,最も対応の良い一次属性同士を対応付け,それらの一致する属性の重みを評価することにより算出するものである.
概念$A$と$B$の関連度$ChainW(A,B)$は以下のアルゴリズムにより計算する\cite{watabe:01}.

\begin{enumerate}
\item まず,2つの概念$A$, $B$を一次属性$a_i, b_j$と重み$u_i, v_j$を用いて,
{\allowdisplaybreaks
\begin{align*}
A &= \{ (a_i, u_i) | i=1\sim L \} 
	\\
B &= \{ (b_j, v_j) | j=1\sim M \}  
\end{align*}
}
と定義する.
ここで,属性個数は重みの大きいものから30個を上限(実験的に検証された)\cite{watabe:01}として展開するものとする.
\item 一次属性数の少ない方の概念を概念$A$とし ($L \le M$),概念$A$の一次属性の並びを固定する.
\[
	A = ((a_1, u_1), (a_2, u_2), \cdots, (a_L, u_L)) 
\]
\item 概念$B$の各一次属性を対応する概念$A$の各一次属性との一致度($MatchW$)の合計が最大になるように並べ替える.
ただし,対応にあふれた概念$B$の一次属性($(b_{x_j}, v_{x_j}), \ j=L+1, \cdots, M$)は無視する.
\[
  B_x = ((b_{x_1}, v_{x_1}), (b_{x_2}, v_{x_2}), \cdots, (b_{x_L}, v_{x_L}))
\]
\item 概念$A$と概念$B$との関連度$ChainW(A,B)$は,
\begin{align*}
	ChainW(A,B) &= (s_A/n_A+s_B/n_B)/2  \label{Echain} \\
	s_A &= \sum_{i=1}^L u_iMatchW(a_i, b_{x_i}) \\
	s_B &= \sum_{i=1}^L v_{x_i}MatchW(a_i, b_{x_i}) \\
	n_A &= \sum_{i=1}^L u_i \\
	n_B &= \sum_{j=1}^M v_j 
\end{align*}
とする.
\end{enumerate}

また,概念$A$と概念$B$の一致度$MatchW(A,B)$は,一致する一次属性の重み(すなわち,$a_i = b_j$なる$a_i, b_j$の重み)の合計をそれぞれ$w_A, w_B$とするとき,次式で定義する.
\[
 MatchW(A,B) = (w_A/n_A + w_B/n_B)/2	
\]
この式は,概念Aと概念Bの一致割合を評価する一つの方式として,概念$A$から見たときの一致している属性の重みの割合$w_A/n_A$と概念$B$から見たときの一致している属性の重みの割合$w_B/n_B$の平均を採用している.


\section{対象語の処理}\label{taishougo}
対象語は発話者の行為・動作・状態の対象となり,修飾語(形容詞・形容動詞)と目的語(名詞)で構成される.修飾語は主に感情判断知識ベースを用いて意味分類の処理を行い,多義性の判断が処理のポイントである.目的語については,別の研究成果である感覚・知覚判断手法\cite{horiguchi:02,watabe:04,kometani:03}をサブシステムとして用いることで意味分類の処理を行う.


\subsection{修飾語の処理}
修飾語は,日常使用する語数が比較的少ないため(6358語),その全てを修飾の方法によって以下の4種類に分けて扱う.直接修飾型,依存修飾型については,それらを表現する形容詞(後述する感覚・知覚判断システムが判断する感覚語(目的語の意味分類)203分類)により意味的に分類し,感情判断知識ベースに登録している.

(1)直接修飾型(785語):感情判断に直接関与し,後に続く名詞の意味分類を修飾語の意味分類に変換するもの(表\ref{example_of_direct_shuushoku}).

\begin{table}[b]
\caption{直接修飾型の「修飾語」の例}
\label{example_of_direct_shuushoku}
\begin{center}
\begin{tabular}{|c|c|} \hline
修飾語(785語)  & 修飾語の意味分類(203分類) \\ \hline\hline
綺麗な & 美しい \\ \hline
不潔な & 汚い \\ \hline
… & … \\ \hline
\end{tabular}
\end{center}
\end{table}


(2)依存修飾型(54語):修飾される名詞によって意味分類が変化するもの(表\ref{Example_of_dependence_modification}).次節で処理方法を詳しく述べる.

\begin{table}[b]
\caption{依存修飾型の「修飾語」の例}
\label{Example_of_dependence_modification}
\begin{center}
\begin{tabular}{|c|c|c|} \hline
修飾語(54語) & 修飾語の意味分類(203分類) & 修飾する名詞\\ \hline\hline
  & 心強い & 握手,誓い,… \\
かたい & 憂鬱な & 頭,雰囲気,… \\
  & なし & 食べ物,石,… \\ \hline
… & … & … \\ \hline
\end{tabular}
\end{center}
\end{table}


(3)無修飾型:名詞の意味分類に影響を及ぼさないもの(例:丸い,赤い,…).

(4)程度表現型:後に続く名詞に関する強さ(程度)を表現するもの(例:深い,大きい,…).

本研究では,これら4種類の修飾語の内,直接修飾型と依存修飾型の修飾語のみを扱うことにした.これは,無修飾型と程度表現型の修飾語は,感情の程度を増減させる効果を持ち,感情そのものには影響を及ぼさないからである.


\subsubsection{修飾語の多義性判断}\label{judgement_of_shuushokugo}
直接修飾型の修飾語は感情判断知識ベースに登録されている知識を参照することにより,容易に意味分類の処理を行うことができる.依存修飾型の修飾語は,その後に続く名詞(目的語)によって,表現する意味が変化する.そこで,表\ref{Example_of_dependence_modification}に示すように,修飾語とその意味分類の他に,修飾する名詞で代表的な語を関連付けて感情判断知識ベースに登録している.多義性の判断は,この修飾する名詞と処理対象である修飾語の後に続く名詞との関連度(\ref{ra}節)を算出し,算出された関連度が最大となる名詞が関連付けられている修飾語の意味分類とすることで実現する.

\subsubsection{多義性判断の性能評価}\label{result_of_shuushokugo}
大学生40名に対して,「依存修飾語」を提示し,それに修飾される「名詞」を思いつくだけ挙げてもらうことにより収集したデータから,無作為に200組を抽出し評価データとして使用した.なお,関連度の有効性を評価するため,関連度計算法と同じように単語間の関連性を数値化する別の手法との比較を行う.本論文では,関連度の算出過程で用いる\ref{ra}節で説明した一致度 ($MatchW$) と\cite{nagao:96}で紹介されている以下の算出式によりシソーラス上の距離を定量化することで単語間の類似度を求める手法を比較対象とした.
\[
 sim(n_1, n_2) = 2d(c)/(d(n_1;c)+d(n_2;c))
\]
ここで,$d(a)$は$a$の深さ,すなわち,シソーラスのルートノードからノード$a$への最短パス長であり,$d(a;b)$は$b$を経由する$a$の深さ,すなわち,シソーラスのルートノードからノード$b$を経由してノード$a$へ至るパスの最短パス長である.また,実験に使用したシソーラスは,日本語語彙体系\cite{ntt:97}を使用した.

多義性の判断についての処理結果を表\ref{result_of_shuushokugo_table}に示す.関連度を用いた処理は,シソーラス上の距離を用いた処理に比べて16.5{\kern0pt}%,一致度を用いた処理に比べて3.0{\kern0pt}%正答率が向上している.また,関連度を用いた処理のみ正答率が目標値を越えており,効果のある処理手法であると考えられる.

\begin{table}[b]
\caption{依存修飾型の修飾語における多義性判断の結果}
\label{result_of_shuushokugo_table}
\begin{center}
\begin{tabular}{|c|c|c|c|} \hline
 & シソーラス距離 & 一致度 & 関連度 \\ \hline\hline
正答率 & 64.5{\kern0pt}% & 78.0{\kern0pt}% & 81.0{\kern0pt}% \\ \hline
\end{tabular}
\end{center}
\end{table}


\subsection{目的語の処理}
目的語の意味分類については,別の研究成果である感覚・知覚判断システム\cite{horiguchi:02,watabe:04,kometani:03}を用いて処理を行う.

感覚・知覚判断システムは,ある語(名詞)に対して人間が常識的に抱く特徴を形容詞・形容動詞の形で判断するシステムである.「痛い」「臭い」などの人間が五感で感じる特徴を『五感感覚語』,「めでたい」,「不幸な」などの五感以外で感じる特徴を『知覚語』と呼ぶ.また,この2種類を総称して『感覚語』と呼び,計203語を定義している.感覚・知覚判断システムの処理は,語とその特徴である感覚・知覚の関係に関する代表的な知識を感覚・知覚判断知識ベース(図\ref{sense_judgment_knowledge_base_image})に登録し,その知識を基に\ref{association_mechanism}章で説明した連想メカニズムを用いて,あらゆる単語に関する感覚語を精度良く判断できるよう工夫されている.

\begin{figure}[b]
\begin{center}
    \includegraphics[width=8.5cm]{14-3ia12f3.eps}
\end{center}
\caption{感覚判断知識ベースのイメージ図}
\label{sense_judgment_knowledge_base_image}
\end{figure}

なお,感覚・知覚判断システムが判断する感覚語203語により目的語を意味的に分類する.また,この203語の感覚語は前述した修飾語の意味分類と共通化させている.つまり,対象語の意味分類として感覚語203語を用いている.


\subsubsection{感覚・知覚判断の処理手法}
ここでは,感覚・知覚の判断についての処理手法を簡単に述べる.なお,詳細については参考文献\cite{horiguchi:02,watabe:04,kometani:03}を参照されたい.

感覚・知覚判断知識ベースは,シソーラス構造をとっており,代表的な語(名詞)に対して,その語から想起される感覚・知覚を人手により付与している.感覚・知覚判断知識ベースに登録されていない未知語が処理の対象になった場合には,感覚・知覚判断知識ベースに登録されている既知語との関連度を算出し,関連性の強い語に帰着する.これにより,大まかな感覚・知覚を得ることができる.さらに,概念ベースの属性を参照することにより,その語特有の感覚・知覚を得る.概念ベースの属性にはその構成上,想起する感覚・知覚として不適切な語も含まれるため,関連度の考え方を用いて,適切な感覚・知覚を得る工夫をしている.



\subsubsection{感覚・知覚判断システムの性能評価}
感覚に関しては447語,知覚に関しては500語の目的語(名詞)を無作為に抽出し,評価データとして使用した.評価は,大学生3名を被験者として行った.処理対象の名詞に対して,意味的に関連が強い感覚・知覚を正しく判断すると「常識的(正答)」,誤った判断をすると「非常識(誤答)」とする.また,意味的な関連は強くないが,判断結果として不適切でないもの(感覚・知覚の観点から一般的に不適切でないもの)は「非常識ではない」とする.例えば,「林檎」の場合,「赤い」は常識的,「明るい」は非常識,「緑」は非常識ではない解と判断する.目的語における感覚の判断結果を図\ref{result_of_kannkaku}に,知覚の判断結果を図\ref{result_of_chikaku}に示す.なお,\ref{result_of_shuushokugo}節で述べた修飾語における多義性判断の性能評価方法と同様に,関連度の代わりに一致度とシソーラス距離を用いた場合の結果を比較対象として評価する.

\begin{figure}[b]
\begin{center}
    \includegraphics[width=10cm]{14-3ia12f4.eps}
\end{center}
\caption{目的語における感覚判断の結果}
\label{result_of_kannkaku}
\end{figure}

\begin{figure}[t]
\begin{center}
    \includegraphics[width=10cm]{14-3ia12f5.eps}
\end{center}
\caption{目的語における知覚判断の結果}
\label{result_of_chikaku}
\end{figure}

常識的な解において,関連度を用いた処理は,シソーラス上の距離を用いた処理に比べて約10〜12{\kern0pt}%,一致度を用いた処理に比べて約3〜5{\kern0pt}%向上している.また,関連度を用いた場合,常識的な解は,感覚・知覚判断においてそれぞれ84.6{\kern0pt}%と70.8{\kern0pt}%であり,非常識ではない解も正答に含めると98.0{\kern0pt}%と88.4{\kern0pt}%と非常に高い正答率となっており,連想メカニズムを用いた感覚・知覚判断システムは,感情判断システムにおける目的語の意味分類処理において有効であるといえる.


\section{変化語の処理}\label{hennkago}
変化語は,発話者の行為・行動・状態を表現する語であり,動詞の他に,形容詞・形容動詞が対象となる.例えば,「私はお腹が痛い」という発話文章の場合,変化語は形容詞「痛い」である.変化語には,対象語から想起される感覚・知覚に関する特徴を変換する効果がある.感覚・知覚的に表現される特徴には大きくプラス的表現とマイナス的表現の2種類に分類できる.例えば,プラス的表現としては,「美しい」や「大切な」,マイナス的表現としては,「痛い」や「汚い」などを挙げることができる.また,感情も同じくプラスとマイナス的な感情の2種類に大別できる.「喜び」と「安心」がプラス的感情,「悲しみ」や「怒り」がマイナス的感情とすることができる.すると変化語には,4種類の作用を見出すことができる(図\ref{sayou_of_hennkago}).

\begin{figure}[b]
\begin{center}
    \includegraphics[width=7cm]{14-3ia12f6.eps}
\end{center}
\caption{変化語における4種類の作用のイメージ図}
\label{sayou_of_hennkago}
\end{figure}

ここで,図\ref{sayou_of_hennkago}における【A】及び【B】は,対象語の情報に依存せず,変化語のみで一意に感情が決定する働きをもつ.【A】に関しては「喜ぶ」や「楽しむ」等が,【B】に関しては「悲しむ」や「恐れる」等を例として挙げることができる.また,【C】【D】の場合,対象語の意味分類により判断される感情が異なり,【C】の場合,対象語の意味分類を『継承』した感情を判断する.逆に【D】の場合では対象語の意味分類を『逆転』する感情を判断する.【C】に関しては「見る」「貰う」等が,【D】に関しては「失う」「捨てる」等が相当する.以下,【A】【B】の変化語を【感情一意想起型】,【C】【D】を【対象語依存型】と呼ぶ.また,感情一意想起型の変化語に関しては,「喜び」の感情を判断する〔喜び型〕,「悲しみ」の感情を判断する〔悲しみ型〕等のように,判断する感情として定義した10種類に細分類することができる.

変化語に関する知識は,動詞を動作や状態によって体系付けられたシソーラス\footnote{学研シソーラス(類義語辞典)辞書,学研メディア開発事業部編}を利用して構築しており,自立語の動詞,及び,動作や状態を表す名詞(サ変接続名詞)17,676語をすべて感情判断知識ベースに登録している.動詞のシソーラスを用いることにより容易に且つ大量に分類作業を行うことができる.なお,形容詞・形容動詞については,修飾語の知識を流用する.


\subsection{変化語における多義性判断}
名詞の場合,語彙数は膨大であるが,多義性が少ないのに対し,動詞の場合,語彙数は少ないが,多義性が激しいという特徴がある.そのため,多くの動詞は複数の意味を有している.例えば,動詞「上がる」には,「継承」に分類される「低い所から高い所へ移動する」意味や「逆転」に分類される「終了する」意味,「喜び」の「地位が進む」意味,「悲しみ」の「費用が増える」意味など多岐に渡る.

変化語単体では多義性を判断することができないため,修飾語の多義性判断と同様に,対象語の目的語の意味分類を利用する.そこで,国語辞書に記載されている変化語に対する解説文章から自立語を抽出し,対象語の目的語との関連度(\ref{ra}節)を算出する.算出された関連度が最大の自立語を含む説明文章が記載されている分類を変化語に対する多義性判断の結果とする.


\subsection{変化語における多義性判断の性能評価}
感情を想起できる「名詞」と「動詞」の組み合わせを大学生40名から収集し,無作為に370セットを抽出し,評価データとして使用した.各セットに対しては,多義性の判断結果として期待する「変化語の分類」をあらかじめ3名の被験者の多数決により決定した.変化語における多義性判断の結果を表\ref{result_of_doushi}に示す.なお,\ref{result_of_shuushokugo}節と同様に比較評価を行った.

\begin{table}[b]
\caption{変化語における多義性判断の結果}
\label{result_of_doushi}
\begin{center}
\begin{tabular}{|c|c|c|c|} \hline
 & シソーラス距離 & 一致度 & 関連度 \\ \hline\hline
正答率 & 66.5{\kern0pt}% & 70.5{\kern0pt}% & 77.0{\kern0pt}% \\ \hline
\end{tabular}
\end{center}
\end{table}

関連度を用いた処理は,シソーラス上の距離を用いた処理に比べて10.5{\kern0pt}%,一致度を用いた処理に比べて6.5{\kern0pt}%正答率が向上している.


\section{感情判断の処理}\label{jugdement_emotion}
\ref{system}章で述べたように,人間が感じる感情の定義には様々な見解がある.そこで,我々は,「あるアクションが起こった際に瞬間的に感じる」ものを感情とみなし,判断する感情を「喜び」,「悲しみ」,「怒り」,「安心」,「恐れ」,「落胆」,「恥」,「後悔」,「罪悪感」と「感情なし」の計10種類と定義した.しかし,人間の感情は非常に複雑であり,高々10種類の感情だけでは,話者の感情を柔軟に判断することは困難である.そこで,先に述べた人手で定義した10種類の感情を基本感情と位置付け,より詳細に感情を表現するために機械的に多くの表現を付与する処理を行う.機械的に付与された詳細な感情を補足感情と呼ぶ.これにより作業量を最小限に抑え,且つ,より豊かに感情を表現することができる.

感情判断には対象語の意味分類(203分類)と変化語の分類(継承と逆転の2分類)の組み合わせ計406種類について,想起する感情を人手で定義して感情判断知識ベースに登録している(表\ref{example_of_emotion_table}).なお,変化語の他の10種類の分類については,「喜び」や「怒り」など感情を直接表現しているため,感情判断の規則として感情判断知識ベースには登録していない.

\begin{table}[b]
\caption{感情判断のための知識の例}
\label{example_of_emotion_table}
\begin{center}
\begin{tabular}{|c|c|c|} \hline
対象語の意味分類(203分類) & 変化語の分類(2分類)& 判断感情 \\ \hline\hline
めでたい & 継承 & 喜び \\ \hline
めでたい & 逆転 & 悲しみ \\ \hline
不吉な & 継承 & 恐れ \\ \hline
… & … & … \\ \hline
\end{tabular}
\end{center}
\end{table}

また,これらの組み合わせに対して,より柔軟に人間の感情を判断するために,補足感情を定義する必要がある.ここで問題になるのは,対象語の意味分類と補足感情との組み合わせ数が膨大になり,各組み合わせに対して手作業を行うことは効率が悪く,また主観による知識の偏りが生じる恐れがあることである.そこで,\ref{ra}節の関連度を利用して,対象語の意味分類と関連性が強い補足感情を機械的に定義する.なお,本研究において,扱う補足感情としては,唯一日本人の感情をモデル化した「情緒の系図」\cite{iki:91}に定義されている感情を用いることにした.


\begin{figure}[b]
\begin{center}
    \includegraphics[width=10cm]{14-3ia12f7.eps}
\end{center}
\caption{関連度の分布}
\label{degree_of_assosiation_average_example}
\end{figure}

ここで,意味的な関連が強いと判断するための閾値の設定方法を述べる.概念に関する関連度合いを「女性−婦人」,「山−丘」などの極めて密接な関係,「山−川」,「夕焼け−赤い」などの密な関係,「山−机」,「電車−眼鏡」などの疎な関係の3種類に分類する教師データを概念ベース全体の中から人手で無作為に抽出した.それらの関係について被験者4名で評価を行い,4名が共にその関係が正しいと判断したものを200セット(計600データ)使用し,それらの関連度の平均を求め,これを関連の強さの判断に利用する.ある概念と極めて密接な関係,密な関係,疎な関係にある語の関連度の平均は,0.47,0.16,0.01であり,図\ref{degree_of_assosiation_average_example}のように各関係の間には関連度に十分有意な差が見られる.そこで,各関係における関連度平均の中間値である0.32,0.09を閾値と設定した.

対象語の意味分類と補足感情との関連度が関連度平均0.32以上の場合は,意味的な関連が極めて強いと判断し機械的に関連付け,関連度平均0.32未満0.09以上の場合は,意味的な関連が強いと判断し手作業で関連付ける.また,関連度平均0.09未満の場合は,意味的な関連が弱いと判断し関連付けを行わない.このようにして定義した補足感情は,「愛」,「哀しみ」,「怨」,「恩」,「怪しい」,「悔しい」,「懐かしい」,「楽しい」,「希望」,「驚き」,「苦」,「誇り」,「寂しい」,「心配」,「親しみ」,「憎」,「妬み」,「美」,「満足」,
「不安」,「不満」,「蔑み」,「憐れみ」,「なし」の計24種類であり,基本感情とは完全に独立した関係になっている.また,補足感情は,基本感情の定義方法と異なり,対象語の意味分類のみを基に定義している.つまり,変化語の意味分類が「継続」であることを前提として定義している.そこで,変化語の意味分類が「逆転」の場合,表\ref{pair_of_supplementation_emotion}に示す補足感情については感情を反転させる処理(プラス的感情を対応するマイナス的感情に変換,または,その逆の処理)を行う.

\begin{table}[t]
\caption{対となる補足感情}
\label{pair_of_supplementation_emotion}
\begin{center}
\begin{tabular}{|c|c|} \hline
プラス的感情 & マイナス的感情 \\ \hline\hline
満足 & 不満  \\ \hline
楽 & 苦,哀しい  \\ \hline
希望 & 不安,心配  \\ \hline
愛 & 憎  \\ \hline
恩 & 怨  \\ \hline
親しみ & 寂しい  \\ \hline
誇り & 悔しい  \\ \hline
\end{tabular}
\end{center}
\end{table}

\ref{taishougo}章から本章までの各処理により感情を判断する流れを図\ref{flow_of_emotion_generation}に示す.

発話文章の主体語は「私」に限定しているため,主体語について特別な処理は行わない.

対象語について,修飾語がある場合には,修飾語により対象語の意味分類を行う.依存型修飾語の場合は,多義性の判断処理を行い対象語の意味分類を決定する.その他の修飾語の場合は,感情判断知識ベースを参照することにより対象語の意味分類を導き出す.修飾語がない場合には,目的語から対象語の意味分類を決定する.目的語の処理には,感覚・知覚判断システムを利用する.対象語の意味分類は203分類であるが,これは,感覚・知覚判断システムが判断可能な意味分類である203分類を用いている.つまり,修飾語の意味分類も同様の203分類を用いている.

変化語については,すべての語を対象に多義性判断の処理を行い,変化語の意味分類を決定する.

このように判断した対象語の意味分類と変化語の意味分類を用いて基本感情と補足感情を判断する.なお,基本感情と補足感情は階層構造などをとらず,完全独立なものとして定義している.


\begin{figure}[t]
\begin{center}
    \includegraphics[width=9cm]{14-3ia12f8.eps}
\end{center}
\caption{感情判断の流れ}
\label{flow_of_emotion_generation}
\end{figure}



\section{感情判断システムの性能評価}\label{result_of_emotion_judgement_system}
感情判断システムの判断感情の妥当性を評価するために,大学生40名から感情が想起される対象語と変化語のセットを収集し,無作為に抽出した200セットを評価データとして使用した.評価としては,5名の被験者にシステムが判断した感情が常識的か非常識かを判断してもらい,4名以上が常識的と答えたものは「常識的な解(正答)」,2名以上3名以下が常識と判断したものは「非常識ではない」,1名以下が常識的と判断したものは「非常識な解(誤答)」とした.また,複数の感情が判断される場合には,すべての感情が常識的であれば「常識的な解(正答)」,一つでも非常識な感情があれば「非常識な解(誤答)」と判断し,その他は「非常識ではない」としている.

\begin{figure}[b]
\begin{center}
    \includegraphics[width=10cm]{14-3ia12f9.eps}
\end{center}
\caption{基本感情のみを判断した際の感情判断結果}
\label{result_of_base_emotion_judgement}
\end{figure}

\begin{figure}[b]
\begin{center}
    \includegraphics[width=10cm]{14-3ia12f10.eps}
\end{center}
\caption{基本感情と補足感情の両方を判断した際の感情判断結果}
\label{result_of_supplementation_emotion_judgement}
\end{figure}

図\ref{result_of_base_emotion_judgement}に基本感情のみを判断した際の感情判断結果を,図\ref{result_of_supplementation_emotion_judgement}に基本感情と補足感情の両方を判断した際の感情判断結果を示す.両方の感情を判断する際には,基本感情と補足感情の両方が常識的なら常識的,どちらか一方でも非常識ならば非常識と評価している.なお,\ref{result_of_shuushokugo}節と同様に比較評価を行った.


常識的な解において,関連度を用いた処理は,シソーラス上の距離を用いた処理に比べて9〜14{\kern0pt}%,一致度を用いた処理に比べて3.5〜4.5{\kern0pt}%向上している.また,図\ref{result_of_base_emotion_judgement}と図\ref{result_of_supplementation_emotion_judgement}を比較すると,正答率に差がないことが分かる.基本感情と補足感情は前述したように完全独立して定義している.そのため,基本的には両方の感情を判断する方が正解率は低下する.このことは,シソーラス距離を用いた処理結果から読み取ることができる.本節の結果で正答率に差が生じなかったことは偶然の結果であるが,半機械的に構築した補足感情を利用することにより,人手ですべてを定義した基本感情における感情判断結果の正答率を極力低下させることなく,判断する感情の表現を多様化し,感情をよりきめ細かに表現できたと考えている.


\begin{table}[b]
\caption{評価データと感情生判断の結果の例}
\label{example_of_result}
\begin{center}
\begin{tabular}{|c|c|c|c|} \hline
評価データ & 基本感情 & 補足感情 & 評価結果 \\ \hline\hline
私はつまらない映画を見る & 落胆 & 不満 & 常識的な解 \\ \hline
私は豊かな生活を送る & 喜び & 満足 & 常識的な解 \\ \hline
私は一人ぼっちで生活する & 悲しみ & 寂しい,不安 & 常識的な解 \\ \hline
私は貧しい生活を送る & 恥 & 苦 & 常識的な解 \\ \hline
私は珍しい現象に遭遇する & なし & 驚き & 常識的な解 \\ \hline
私はいたずら電話をする & 罪悪感 & 憎 & 常識的な解 \\ \hline
私は優勝をする & 喜び & 楽しい,誇り & 常識的な解 \\ \hline
私は赤ん坊を出産する & 喜び & 愛 & 常識的な解 \\ \hline
私は別れを告げる & 悲しみ & 苦,寂しい,不安 & 常識的な解 \\ \hline
私は幽霊を見る & 恐れ & 怪しい,驚き & 常識的な解 \\ \hline
私は娘を出産する & 喜び,悲しみ & 愛,親しみ,不安 & 非常識ではない解 \\ \hline
私は不正を見逃す & 安心,喜び & なし & 非常識な解 \\ \hline
\end{tabular}
\end{center}
\end{table}

評価データとその結果の例を表\ref{example_of_result}に示す.常識的と判定された例である「私はつまらない映画を見る」という評価データでは,基本感情が「落胆」であったが,「不満」という補足感情を判断することにより,話者の感情をより詳細に表現しているといえる.このように,基本感情のみでは,10種類の感情による表現であったものが,補足感情を用いることでより多様な感情を表現することができた.また,正答率は,両者共に76.5{\kern0pt}%であり,非常識ではない解を正答率に含めるとそれぞれ89.0{\kern0pt}%,88.0{\kern0pt}%と非常に高い結果であり,本システムは有効であると言える.


\section{おわりに}
本稿では,人間がコミュニケーションの中で自然に行っている常識的な判断の一つである感情に着目し,「主体語」,「修飾語」,「目的語」,「変化語」の4要素から成るユーザの発話文章から,そのユーザの感情を基本感情10種類,補足感情24種類で判断する手法を提案した.また,実システムによりその性能を評価した結果,常識的な解の正答率は76.5{\kern0pt}%であり,非常識ではない解を正答率に含めると88.0{\kern0pt}%となった.このことから,本研究で構築した感情判断システムは非常に高い性能であり,提案した処理手法は有効であると言える.

\vspace{0.5\baselineskip}

\acknowledgment

本研究は,文部科学省からの補助を受けた同志社大学の学術フロンティア研究プロジェクトにおける研究の一環として行った.




\vspace{0.5\baselineskip}

\bibliographystyle{jnlpbbl_1.2}
\begin{thebibliography}{}

\bibitem[\protect\BCAY{広瀬\JBA 渡部\JBA 河岡}{広瀬\Jetal }{2002}]{hirose:02}
広瀬幹規\JBA 渡部広一\JBA 河岡司 \BBOP 2002\BBCP.
\newblock \JBOQ
  概念間ルールと属性としての出現頻度を考慮した概念ベースの自動精錬手法\JBCQ\
\newblock \Jem{信学技報, NLC2001-93}, \mbox{\BPGS\ 109--116}.

\bibitem[\protect\BCAY{Horiguchi, Tsuchiya, Kojima, Watabe, \BBA\
  Kawaoka}{Horiguchi et~al.}{2002}]{horiguchi:02}
Horiguchi, A., Tsuchiya, S., Kojima, K., Watabe, H., \BBA\ Kawaoka, T. \BBOP
  2002\BBCP.
\newblock \BBOQ Constructing a Sensuous Judgment System Based on Conceptual
  Processing\BBCQ\
\newblock {\Bem Computational Linguistics and Intelligent Text Processing
  (Proc. of CICLing-2002)}, \mbox{\BPGS\ 86--95}.

\bibitem[\protect\BCAY{齊藤}{齊藤}{1986}]{saitou:86}
齊藤勇 \BBOP 1986\BBCP.
\newblock \Jem{感情と人間関係の心理}.
\newblock 川島書店.

\bibitem[\protect\BCAY{福井}{福井}{1990}]{hukui:90}
福井康之 \BBOP 1990\BBCP.
\newblock \Jem{感情の心理学}.
\newblock 川島書店.

\bibitem[\protect\BCAY{九鬼}{九鬼}{2001}]{iki:91}
九鬼周造 \BBOP 2001\BBCP.
\newblock \Jem{「いき」の構造}.
\newblock 岩波書店.

\bibitem[\protect\BCAY{小島\JBA 渡部\JBA 河岡}{小島\Jetal }{2002}]{kojima:02}
小島一秀\JBA 渡部広一\JBA 河岡司 \BBOP 2002\BBCP.
\newblock \JBOQ
  連想システムのための概念ベース構成法−属性信頼度の考え方に基づく属性重みの決
定\JBCQ\
\newblock \Jem{自然言語処理}, {\Bbf 9}  (5), \mbox{\BPGS\ 93--110}.

\bibitem[\protect\BCAY{米谷\JBA 渡部\JBA 河岡}{米谷\Jetal }{2003}]{kometani:03}
米谷彩\JBA 渡部広一\JBA 河岡司 \BBOP 2003\BBCP.
\newblock \JBOQ 常識的知覚判断システムの構築\JBCQ\
\newblock \Jem{人工知能学会全国大会, 3C1-07}.

\bibitem[\protect\BCAY{目良\JBA 市村\JBA 相沢\JBA 山下}{目良\Jetal
  }{2002}]{mera:02}
目良和也\JBA 市村匠\JBA 相沢輝昭\JBA 山下利之 \BBOP 2002\BBCP.
\newblock \JBOQ 語の好感度に基づく自然言語発話からの情緒生起手法\JBCQ\
\newblock \Jem{人工知能学会論文誌}, {\Bbf 17}  (3), \mbox{\BPGS\ 186--195}.

\bibitem[\protect\BCAY{長尾}{長尾}{1996}]{nagao:96}
長尾真 \BBOP 1996\BBCP.
\newblock \Jem{岩波講座ソフトウェア科学15自然言語処理}.
\newblock 岩波書店.

\bibitem[\protect\BCAY{NTTコミュニケーション科学研究所}{NTTコミュニケーション
科学研究所}{1997}]{ntt:97}
NTTコミュニケーション科学研究所 \BBOP 1997\BBCP.
\newblock \Jem{日本語語彙体系}.
\newblock 岩波書店.

\bibitem[\protect\BCAY{Okada}{Okada}{1996}]{okada:96}
Okada, N. \BBOP 1996\BBCP.
\newblock \BBOQ Integrating Vision, Motion and Language through Mind\BBCQ\
\newblock {\Bem Artificial Intelligence Review}, {\Bbf 10}, \mbox{\BPGS\
  209--234}.

\bibitem[\protect\BCAY{Okada \BBA\ Endo}{Okada \BBA\ Endo}{1992}]{okada:92}
Okada, N.\BBACOMMA\ \BBA\ Endo, T. \BBOP 1992\BBCP.
\newblock \BBOQ Story Generation Based on Dynamics of the Mind\BBCQ\
\newblock {\Bem Computational Intelligence}, {\Bbf 8}  (1), \mbox{\BPGS\
  123--160}.

\bibitem[\protect\BCAY{リタ・カーター}{リタ・カーター}{1999}]{rita:99}
リタ・カーター \BBOP 1999\BBCP.
\newblock \Jem{脳と心の地形図}.
\newblock 原書房.

\bibitem[\protect\BCAY{スーザン・グリーンフィールド}{スーザン・グリーンフィー
ルド}{2001}]{suzan:01}
スーザン・グリーンフィールド \BBOP 2001\BBCP.
\newblock \Jem{脳の探究}.
\newblock 無名舎.

\bibitem[\protect\BCAY{徳久\JBA 岡田}{徳久\JBA 岡田}{1998}]{tokuhisa:98}
徳久雅人\JBA 岡田直之 \BBOP 1998\BBCP.
\newblock \JBOQ パターン理解的手法に基づく知能エージェントの情緒生起\JBCQ\
\newblock \Jem{情報処理学会論文誌}, {\Bbf 39}  (8), \mbox{\BPGS\ 2440--2451}.

\bibitem[\protect\BCAY{土屋\JBA 奥村\JBA 渡部\JBA 河岡}{土屋\Jetal
  }{2005}]{tsuchiya:05}
土屋誠司\JBA 奥村紀之\JBA 渡部広一\JBA 河岡司 \BBOP 2005\BBCP.
\newblock \JBOQ 連想メカニズムを用いた時間判断手法の提案\JBCQ\
\newblock \Jem{自然言語処理}, {\Bbf 12}  (5), \mbox{\BPGS\ 111--129}.

\bibitem[\protect\BCAY{渡部\JBA 河岡}{渡部\JBA 河岡}{2001}]{watabe:01}
渡部広一\JBA 河岡司 \BBOP 2001\BBCP.
\newblock \JBOQ 常識的判断のための概念間の関連度評価モデル\JBCQ\
\newblock \Jem{自然言語処理}, {\Bbf 8}  (2), \mbox{\BPGS\ 39--54}.

\bibitem[\protect\BCAY{渡部\JBA 堀口\JBA 河岡}{渡部\Jetal }{2004}]{watabe:04}
渡部広一\JBA 堀口敦史\JBA 河岡司 \BBOP 2004\BBCP.
\newblock \JBOQ 常識的感覚判断システムにおける名詞からの感覚想起手法\JBCQ\
\newblock \Jem{人工知能学会論文誌}, {\Bbf 19}  (2), \mbox{\BPGS\ 73--82}.

\end{thebibliography}


\vspace{\baselineskip}

\begin{biography}
\bioauthor{土屋 誠司}{
2000年同志社大学工学部知識工学科卒業.
2002年同大学院工学研究科知識工学専攻博士前期課程修了.
同年,三洋電機株式会社入社.
2007年同志社大学大学院工学研究科知識工学専攻博士後期課程修了.
同年,徳島大学大学院ソシオテクノサイエンス研究部助教.工学博士.
主に,知識処理,概念処理,意味解釈の研究に従事.
言語処理学会,人工知能学会,情報処理学会,電子情報通信学会各会員.
}
\bioauthor{吉村枝里子}{
2004年同志社大学工学部知識工学科卒業.
2006年同大学院工学研究科知識工学専攻博士前期課程修了.
同年,大学院工学研究科知識工学専攻博士後期課程入学.
主に,知識情報処理の研究に従事.
}
\bioauthor{渡部 広一}{
1983年北海道大学工学部精密工学科卒業.
1985年同大学院工学研究科情報工学専攻修士課程修了.
1987年同精密工学専攻博士後期課程中途退学.
同年,京都大学工学部助手.
1994年同志社大学工学部専任講師.
1998年同助教授.
2006年同教授.工学博士.
主に,進化的計算法,コンピュータビジョン,概念処理などの研究に従事.
言語処理学会,人工知能学会,情報処理学会,電子情報通信学会,システム制御情報学会,精密工学会各会員.
}
\bioauthor{河岡  司}{
1966年大阪大学工学部通信工学科卒業.
1968年同大学院修士課程修了.
同年,日本電信電話公社入社,情報通信網研究所知識処理研究部長,
NTTコミュニケーション科学研究所所長を経て,現在同志社大学工学部教授.
工学博士.
主にコンピュータネットワーク,知識情報処理の研究に従事.
言語処理学会,人工知能学会,情報処理学会,電子情報通信学会,IEEE(CS)各会員.
}


\end{biography}





\vspace{\baselineskip}

\biodate

\end{document}
