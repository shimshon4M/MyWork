    \documentclass[japanese]{jnlp_1.3a}
\usepackage{jnlpbbl_1.1}
\usepackage{udline}
\setulsep{0pt}
\usepackage[dvips]{graphicx}
\usepackage{multirow}


\Volume{14}
\Number{3}
\Month{Apr.}
\Year{2007}
\received{2006}{4}{20}
\revised{2006}{9}{11}
\rerevised{2006}{11}{6}
\accepted{2006}{11}{12}

\setcounter{page}{39}

\jtitle{図形説明課題対話におけるフィラーの分析\\
—心的マーカによる内的処理プロセスの理解へ向けて—}
\jauthor{山下 耕二\affiref{NiCT} \and 水上 悦雄\affiref{NiCT}}
\jabstract{
本研究の目的は,これまで言語学的には感動詞,言語心理学的には発話の非流暢性として扱われてきた,フィラーを中心に,情動的感動詞,言い差し(途切れ)といった話し言葉特有の発話要素を,人の内的処理プロセスが音声として外化した「心的マーカ」の一部であると捉え,それらが状況によってどのような影響を受けるかを分析し,対応する内的処理プロセスについて検討することであった.実験的統制のもと,異なる条件(役割や親近性,対面性,課題難易度)が設定され,成人男女56名(18--36歳)に対して,ペアでの協調問題解決である図形説明課題を実施し,対話データが収集された.その結果,1) それぞれの出現率は状況差の影響を受けたこと,2) 出現するフィラーの種類別出現率に差があることが示された.これらの結果が先行研究との対比,内的処理プロセスと心的マーカの対応,そして結果の応用可能性という観点から考察される.
}
\jkeywords{フィラー,情動的感動詞,言い差し,心的マーカ,内的処理プロセス,非流暢性}

\etitle{Using Fillers as Mental Makers: Effects of Familiarity, \\
	Modality, and Task Difficulty in Describing the Figure}
\eauthor{Kouji Yamashita\affiref{NiCT} \and Etsuo Mizukami\affiref{NiCT}} 
\eabstract{
We examined the effects of familiarity, modality, and task difficulty on the use of fillers when describing a figure. A total of 56 adults (aged 18--38) participated in an experiment designed to elicit examples of disfluency words, such as fillers, affective interjections, and speech discontinuities. They were asked to solve a problem in same sex pairs. One was instructed to describe the figure, and the other to identify the correct figure from a choice of six. This experiment was done in various conditions when there was variation in how familiar the participants were with the figure, i.e., variation in familiarity; when the pairs could and not see each other, i.e., variation in modality; and when the task was both easy and difficult, i.e., variation in difficulty. The results showed two things. First, the average rates of filler, affective interjections, and speech discontinuities differed in relation to situational differences. Second, the filler rates varied with the type of filler. These results are discussed in terms of the relationship between mental processing and mental markers.
}
\ekeywords{Fillers, Affective interjections, Speech discontinuities, Disfluency, Mental marker, Mental processing}

\headauthor{山下,水上}
\headtitle{図形説明課題対話におけるフィラーの分析}

\affilabel{NiCT}{独立行政法人情報通信研究機構}{
	National Institute of Information and Communications Technology}



\begin{document}
\maketitle



\renewcommand{\underline}[1]{}
\setcounter{secnumdepth}{3}

\section{はじめに}
人は必ずしも流暢に話しているわけではなく,以下の例のように,ときにつっかえながら,ときに無意味とも言える言葉を発しながら,話している.
\newcounter{cacocnt}
\begin{list}{ 例 \arabic{cacocnt}}{\usecounter{cacocnt}}
\item \underline{アッ} しまった  \underline{エッ} 本当?
\item \underline{ド} どうしよう? \underline{アシ} あさってかな?
\item\underline{エート} 今度の日曜なんですが \underline{アノー} 部屋はあいてるでしょうか
\end{list}

例1の下線部は感動詞(間投詞,interjections),例2は発話の非流暢性 (disfluency)の一部であり,例3はその両方のカテゴリーに帰属する話し言葉特有の発話要素である.これらは,近年,人の言語処理を含む内的処理プロセス (mental processing)や心の動きを映し出す「窓」として注目されてきている\cite{定延・田窪,田窪・金水,田中,Clark:02,山根,定延:05,富樫:05}.本研究では,これらを発話に伴う「心的マーカ (mental marker)」と捉え,例3のような「フィラー (fillers)」を中心に,「情動的感動詞 (affective interjections)」(例1)および「言い差し(途切れ;speech discontinuities)」(例2)と対比することで,人の内的処理プロセスとこれらの心的マーカとの対応関係について検討した.

\subsection{従来の研究アプローチ}
感動詞および非流暢性に焦点をあてた研究アプローチには,大きく分けて言語学的 (linguistic)アプローチと言語心理学的 (psycholinguistic)アプローチの2つが存在する.前者のアプローチからは,これまで主として,感動詞と感情の関係や感動詞の統語的性質が考察されてきた\cite[など]{田窪・金水,森山:96,土屋,富樫:05}.例えば,\citeA{森山:96}は「ああ」や「わあ」などの情動的感動詞を内発系と遭遇系に分類し,それらがどのような心的操作と対応するかについて詳しく考察した.一方,後者のアプローチからは,人の内的言語処理メカニズムを知るために,途切れや延伸,繰り返し,言い直しなどの非流暢性が研究されてきた\cite[など]{村井,伊藤,田中}\footnote{最近になって,\citeA{定延・中川}が非流暢性の言語学的な制約を分析するという言語学的アプローチによる考察を試みている.}.例えば,\citeA{村井}は,幼児の言語発達における言語障害的発話を分類し,言語発達過程における非流暢性の現れ方について考察した.これら2つのアプローチは,発話要素から人の内的処理メカニズムを探るという目的では類似している.しかしながら,前者は主としてそれぞれの感動詞に対応する心的操作について,後者は主として非流暢性の程度と言語処理メカニズムあるいは言語発達過程との関係について検討してきたため,共通する対象領域をカバーしながらも,それぞれ別の角度から取り組んできたといえる.本研究において中心に取り上げるフィラーは,言語学的には感動詞の一部として\cite{田窪・金水,定延・田窪},言語心理学的には非流暢性の一種である有声休止(filled pause, \cite{Goldman-Eisler,田中})として,双方のアプローチから研究されてきた音声現象である\cite{山根}.フィラーと情動的感動詞,言い差し(途切れ)を同一軸上で比較することで,両研究アプローチからの「切り口」により明らかにされる内的処理プロセスの諸側面をさらに深く理解することにつながると考えられる.以下に,本研究で扱う3つの発話要素(フィラー,情動的感動詞,言い差し)に関する先行研究を概観し,本研究の目的および特色を述べる.


\subsubsection{フィラー}
Merriam-Webster Online Dictionary (http://www.m-w.com/)によると,フィラー (fillers)には「間を埋めるもの」という意味がある.\citeA{Brown}によると,フィラーは主に発話権を維持するために,発話と発話の間を埋めるように発する発話要素とされる\footnote{\citeA{Clark-Tree}や\citeA{水上・山下}は,話し手のフィラーが長い場合,前後のポーズ長も長くなる傾向にあることを示しており,結果として,ポーズだけの場合よりも長く発話権を維持できる.}.この意味に相当する日本語の用語として,「間(場)つなぎ言葉」がある.その他に,無意味語,冗長語,繋ぎの語,遊び言葉,言い淀み,躊躇語など,これまでそれぞれの研究者の視点からさまざまに呼ばれてきている\cite{山根}.本研究では,近年の傾向にしたがい\cite{山根,定延:05},便宜的に,フィラーという名称を用いる.フィラーは,一般に命題内容を持たず,前後の発話を修飾するようなものでもない\cite{野村,山根}.例えば,例3の文からフィラーを除いたとしても,文意には何ら影響しない.そのため,古典的な日本語研究においては,感動詞や応答詞あるいは間投詞の一部として,その用法が取り上げられるにすぎなかった\cite{山根}.

しかしながら,近年,言語学的アプローチによる研究により,フィラーのさまざまな機能が注目されるようになってきた.例えば,談話の区切りを表示する「談話標識 (discourse marker\cite{Schiffrin})」の機能\cite{Swerts,Watanabe,野村}や,換言や修正のマーカ\footnote{「渡したペー アノ プリント」のように言い直しの前などに出現するフィラーを指す.}としての機能 \cite{野村}があげられる.その他にも,``uh''や``ah''などのフィラーが構文理解 (parsing)にもうまく利用されることが示されている\cite{Ferreira-Bailey}.

また,フィラーは,非流暢性あるいは停滞現象 (speech unfluency\cite{田中}),有声休止 (filled pause)と呼ばれることもあり,発話上の問題として捉えられてきた側面もある(例えば,\citeA{Hickson}).一方で,1960年代から\citeA{Goldman-Eisler}ら言語心理学者によってさかんに非流暢性が研究されてきた理由の一つは,非流暢性が話し手の言語化に関わる内的処理過程・処理能力を表示するよい指標になり得るからである.注目すべきは,表情や一部の身体動作と共に(例えば,\citeA{Ekman,Ekman-Friesen}),フィラーが話し手の心的状態や態度が外化したものと考えることができる点である\cite{定延・田窪,田窪・金水}.\citeA{定延・田窪}は,フィラーを話し手の心的操作標識と捉え,「エート」と「アノ(ー)」を取り上げて,心的操作モニター機構について考察した.\citeA{定延・田窪}によれば,「エート」は,話し手が計算や検索のために心的演算領域を確保していることを表示し,一方で「アノ(ー)」は,話し手が主に聞き手に対して適切な表現をするために言語編集中であることを表示するとされる.この例以外に,状況によって適さないフィラーや,逆に儀礼的に使われるフィラーも存在する\cite{定延:05}.これらは,フィラーが発話者の心的状態を表示する標識となる一方で,状況や場などの制約を受ける言語学的な側面を持つことを示している.

\subsubsection{情動的感動詞}
情動的感動詞とは,\citeA{森山:96}が,情動的反応を表す感動詞として分類したものである.\citeA{森山:96}は,泉の比喩を使ったモデルで「アア」のような内から湧き上がってくる感情を表す内発系と,「オヤ」「オット」「ワア」「キャア」などの遭遇系の情動を分類し,それぞれと感情との関係を考察した.また,\citeA{田窪・金水}は,感動詞を,「心的な過程が表情として声に現れたもの」と捉え,特に情報の入出力に関わるものを「入出力制御系」とし,それらを応答,意外・驚き,発見・思い出し,気付かせ・思い出させ,評価中,迷い,嘆息に分類し,それぞれについて考察した\footnote{出力の際の操作に関わるものは「言い淀み系」として,非語彙的形式,語彙的形式(内容計算,形式検索,評価)に分類された.これはほぼフィラーに対応すると考えられる.}.彼らによれば,例えば,感動詞「ア」とは,発見・思い出しの標識であり,「予期されていなかったにも関らず関連性の高い情報の存在を新規に登録したということを表す」ものである.これに対し,近年,\citeA{富樫:05}は,驚きを伝えるとされる「アッ」と「ワッ」を取り上げ,「アッ」の本質は発見や新規情報の登録を示すものではなく,単に「変化点の認識」を示すものであると述べた.さらに\citeA{富樫:05}は,従来考えられてきたような感動詞の伝達的側面を疑問視し,感動詞の本質は感動を含まず,それは聞き手の解釈による効果に過ぎないと述べている.これらの研究は,情動的感動詞が少なくとも話し手の何らかの「心の状態の変化が音声として表出したもの (change of state token\cite{Heritage})」と考えられることを示している.

\subsubsection{言い差し(途切れ)}
言い差しとは,反復や言い直しによって途切れた不完全な語断片を指す.本研究では,スラッシュ単位マニュアル\shortcite{Slash-Manual}でタグとして使用されている言い差しの用法に従う\footnote{「ちょっと用事がありまして(参加できません)」のように,重要な部分を省略した用法を「言い差し表現」と言う場合もある.}.言い差しは,言語心理学的な研究の中で,意味処理や調音運動に関連付けて研究されてきた.例えば,\citeA{田中}は,スピーチの停滞現象を反復(「ヒ ヒトハオドロイタ」),言い直し(「キカイガヘンカ コワレタ」),有声休止(フィラー),無声休止(ポーズ)などに分類し,それらが意味処理の過程とどのように関っているのかを実験に基づく考察から詳細に分析した.その結果,意味の処理には,音声を伴わない処理と音声を伴う処理の2つの様相があることが示された.この結果は,従来の考え方が前提としていた,人の発話処理過程において,意味の処理が完了してから音声出力されるという考え方に疑問を投げかけるものであった.つまり,人は考えてから話すのではなく,話しながら考えるという二重処理を行っていることを示す.言い差しとは,一旦,出力されかけた言語表現が並列的に動作する意味処理によって,中断されたものと考えられる.その意味で,言い差しは人の発話に伴う内的処理のプロセスの並列性,階層性を理解する上で,重要な鍵となると考えられている.

\subsection{本研究のアプローチ}
\subsubsection{3つの発話要素の定義}
本研究では,先行研究\cite{山根}を参考に,フィラー,情動的感動詞,言い差しといった3つの発話要素を以下のように定義した.以下の例では,フィラー,情動的感動詞,言い差しに該当する部分をそれぞれカタカナで表記して示す.
\\

\noindent\textbf{フィラー}\\
・それ自身命題内容を持たず,発話文の構成上,排除しても,意味に影響を及ぼさないもので\\
\noindent (1)他との応答・修飾・接続関係にないもの\\
○「エットソノー 3つ目の正方形の」\\
×「その角に」

\noindent ○「普通のモー 三角形ですね」\\
×「もう少し」

\noindent ○「コー ナンテイウンデスカネ」\\
× ジェスチャーを伴って「こう(こんなふうに)」

\noindent (2)他との応答関係にあっても逡巡を示すもの\\
○ 質問を受けて「ウーン 左側が長いんですよね」\\
× 「うんそう」

\noindent (3)情動的感動詞\cite{森山:96}や言い差し(途切れ)とは異なるもの\\
○「エー 左だけ書いてから」\\
×「えっ それだけ?」(情動的感動詞)\\
×「え 円を描くように」(言い差し)

\vspace{10pt}
\noindent \textbf{情動的感動詞}\\
・気付き,驚き,意外など,心的状態の変化を表出していると考えられるもの\\
 「ア わかりました」「エ 違う?」「アレ?」など

\vspace{10pt}
\noindent\textbf{言い差し(途切れ)}\\
・反復,言い直しなど,言いかけて止めることによって,単語として成立していないもの\\
「サ さんかく」「フタ 三つ目」「ホ(沈黙)」など

\vspace{10pt}
この定義により,本研究で扱う対話データ(後述)では,以下のようなものがフィラーとして認定された:アー,アノ(ー),アノナ
,アノネ,アレ\footnote{フィラーとしての「アレ」は,平坦に短く,低ピッチで発音される.「それは アレ 三角関数みたいに」という場合.同様に,代名詞と同表記である,「アノ」,「コノ」,「ソノ」もフィラーの場合には基本的に平坦かつ低ピッチで発音される.}
,アンナ,ウー,ウーン,ウ(ー)ント(ー),ウ(ー)ントネ,ウ(ー)ントナ,エ(ー),エ(ー)(ッ)ト,エ(ー)(ッ)トネ,エ(ー)(ッ)トナ,エ(ー)(ッ)トデスネ,コー,コノ(ー),ソーデスネ(ー),ソノ(ー),(ッ)ト(ー),(ッ)トネ,(ッ)トナ,ドウイエバイイノカ\footnote{{\kern-0.5zw}「ドウイエバイイノカ」に類するフィラーは,低ピッチで独り言のように発する場合であり,相手に答えを求めて「どう言えばいいんですか?」と問いかけているものではない.「ナンテイエバイイノカ」に類するフィラーも同様.これらが命題内容を持つかどうかについては議論の余地があるが,本研究では,\citeA{山根}において,フィラーとされる「ドウイウカ」「ナンテイウカ」の変形として,これらをフィラーに含めた.},ドウイエバイインダロウ,ドウイッタライイカ,ドウイッタライイノカ,ドウイッタライインデスカネ,ドウセツメイシタライイカ,ドウダロウ,ナンカ,ナンカネ,ナンカナ,ナンテイウカ,ナンテイウノカ,ナンテイウノ,ナンテイウノカナ,ナンテイイマスカ,ナンテイエバ,ナンテイエバイイカ,ナンテイエバイインデスカネ,ナンテイッタライイノカ,ナンテイウンデスカネ,ナンテイッタラインデスカネ,ハー,フーン,マ(ー),モー,ンー,ン(ー)ト,この他,方言による変異と考えられる,アンナー,ソヤネー,ナンチューカ,ナンテイエバイイトなどもフィラーとみなした.また,情動的感動詞としては,以下のものが認定された:ア(ー)(ッ),アレ(ッ),イ(ッ),ウ(ッ),エ(ー)(ッ),オ(ー)(ッ),ハ(ッ),ハイ,ヒ(ッ),ヘ(ッ),(ウ)ン.言い差しについては,不定形のため省略する.

\subsubsection{本研究の目的}

本研究の目的は,従来の言語学的アプローチと言語心理学的アプローチにより明らかにされてきた発話行為に伴う内的処理について,フィラーを中心に,情動的感動詞,言い差し(途切れ)という心的マーカを指標に検討することにある.対話において内的処理の過程に何らかの問題が発生すると,その内的状態を反映して,話し手,聞き手双方の発話中に,心的マーカが出現する.これらの心的マーカの出現率を分析することで,対応する処理プロセスとの関係を明らかにする.

話し手の内的処理プロセスには,思考に関わるもの(検索・記憶操作,計算,類推,話の組み立てなど)と,発話生成に関わるもの(構文調整,音韻調整,単語・表現選択など),聞き手の内的処理プロセスには,発話の理解に関わるもの(構文理解,文脈理解,意味解釈,意図推論など)が考えられる.これらの話し手,聞き手の処理プロセスに,状況の認識に関わる内的処理(場の認識,関係性の認識,話者間の共通知識についての認識,利用可能なモダリティの認識,時間や空間の制約の認識など)が影響を及ぼすことが予想される.つまり,状況の認識が決定されることで,思考や発話のなされ方が変化すると考える.


\begin{table}[b]
\begin{center}
\caption{話し手の内的処理プロセスおよび心的マーカと状況変数との対応}
\label{map_speaker}
\scriptsize
\begin{tabular}{c c c c c c} \hline 
\multirow{2}{12mm}{状況変数} & \multirow{2}{24mm}{喚起される主だった 状況認識のモード} & 主な思考プロセス & 主な発話生成プロセス\\
 & & [主な心的マーカ]& [主な心的マーカ]\\ \hline
親近性 & 関係性の認識 & 説明の組み立て & 表現選択\\
 & (丁寧さの意識) & [フィラー(アノ)] & [フィラー(アノ)] &\\
対面性 & モダリティの認識 & 表象の言語化 & 単語選択\\
 & (制約の意識) & [フィラー(ナンカ)] & [フィラー(アノ)]\\
難易度 & 必要な処理の認識 & 記憶・検索操作,説明方略 & 単語選択,文構成\\
 & (必要操作への意識) & [フィラー(エート,ソノ),情動的感動詞] & [フィラー(アノ),言い差し]\\ \hline
\end{tabular}
\end{center}
\end{table}


そこで本研究では,発話の言語化に関わる内的処理プロセスに影響を及ぼすと想定される3つの状況変数(親近性,対面性,課題難易度)が操作され,話し手の内的処理プロセスが状況変数の影響をどのように受け,また聞き手の理解に影響するかどうかが検討された.本研究で操作される変数以外にも,状況変数としては性別差や年齢差などが考えられる.それらと比較して,親近性,対面性,課題難易度は,それぞれ,社会性,伝達手段,処理の複雑さといった異種の認識モードを必要とし,発話の言語化に関わる内的処理プロセスにも異なる影響を及ぼすと考えられた.本研究で想定された話し手の内的プロセス(思考と発話生成のプロセス)および心的マーカと状況変数の関係が,表\ref{map_speaker}に示される.具体的には,親近性の場合,対話の相手が友人か初対面の人であるかという関係性の認識によって,丁寧さへの意識が変化し,発話生成のための言葉選びや言い回しが変化する.つまり,初対面の人に説明する場合には,思考プロセスにおいて丁寧な説明のための発話の組み立てに負荷が,発話生成プロセスにおいては,発話表現の選択に負荷がかかることが予想される.次に,対面性の場合,相手と対面して対話するかどうかという利用可能なモダリティの認識によって,表現方法への制約が意識される.つまり,非対面の場合に,思考プロセスにおいては形状の表象への変換に負荷が,そして発話生成プロセスでは説明のための単語や表現の選択に負荷がかかるだろう.最後に,難易度の場合,説明内容が難しく,必要な処理操作が増加するという認識によって,記憶や対象への注意などの必要操作への意識が高まる.つまり,思考プロセスにおいては記憶操作や単語検索,対象把握や文の組み立てなどに,発話生成プロセスではどのような言語表現を使い,いかに発話の整合性を保つかという単語選択や文構成に負荷がかかるであろう.

リアルタイムに処理可能な情報量に限界のある話し手にとって,特定の発話プロセスに負荷がかかると,その状態を表示するさまざまな心的マーカが外化することが予想される.例えば,先行研究からの予測として,単語や表現の検索・選択への負荷の増加は,「エート」や「アノ(ー)」などのフィラーの増加として表出するであろう.その他,「ナンカ」は,具現化できない何かを模索中であることの標識であり,表象の言語化過程に表出しやすいであろうし,「ソノ」は,すぐに具現化できない内容が思考プロセスに存在していることを示すとされ\cite{山根},言葉を掘り起こす負荷の高い場合に表出されやすいであろう.また,並列的に処理される思考プロセスと発話生成プロセスに同時に負荷がかかる場合,例えば,発話を始めてから言い間違いに気付いて,言い直す場合には,言い差しが表出することが予想される.一方,「ア」や「エ」などの情動的感動詞の場合には,上記の負荷の影響は間接的であり,例えば,説明しにくい(相手にも理解しにくい)対象を説明する場合に,自分が今行っている説明の仕方よりもさらによい説明の仕方を思いついたときや,説明の不備に気がついたときに表出される機会が増加することが予想される\footnote{ここでは,話し手の発話プロセスについて言及しているが,「ア」などの情動的感動詞は,理解や発見の表示として表出する場合が多く,聞き手の応答時に現れやすい(例えば,「ア,はいはいはい」).}.

以上から,3つの状況変数は以下のような心的マーカの出現率の差として現れることが予測される.1) 親近性が低いと,表現選択に関するフィラー出現率が高まり,2) 対面性がないと,表象の言語化や単語選択に関するフィラー出現率が高まり,3) 難易度が高いと,記憶・検索操作に関するフィラー出現率,情動的感動詞出現率,言い差し出現率のすべてが高まる.

また,本研究では,状況による心的マーカの現れ方を検討するため,統制された実験環境において,課題遂行型の対話である図形説明課題対話を収録,分析した.先行研究では,自然な対話収録を目的とし,自由対話を課題とするものが多く,例えば,会話分析のような社会学的手法においては日常会話が主として扱われてきた\cite{好井}.しかし,本研究で用いる図形説明課題対話は,提示された図形を説明する説明者役と,説明を受けて理解し,選択肢を答える回答者役に分かれて行う課題であり,役割の非対称性(話し手/聞き手)と情報の非対称性(説明者≫回答者)を特徴としている\footnote{ただし,回答者には,説明者に対して質問することを許可しており,局所的には話し手/聞き手が逆転する場合がある.}.役割の非対称性がある対話として,インタビュー対話\shortcite[など]{CSJ}があげられるが,ここでは,聞き手であるインタビュアの会話進行能力や質問の仕方に依存し,発話量のバランスや難易度の統制が困難である.また,本研究での課題と同様に,協同作業型課題遂行対話である地図課題対話\shortcite{堀内-99}では,説明者役と回答者役の間の情報の非対称性が完全ではない(回答者にも手がかりがある).図形説明課題を使用することで,説明者側の内的処理プロセスは,説明のための言語化に係わる処理プロセスが主となり,回答者側の内的処理プロセスは,理解に係わる処理プロセスが主となると切り分けて検討できる利点を有する.

\section{方法}

\noindent 実験参加者

成人56(男性28,女性28)名がペアで実験に参加した.実験ペアは同世代かつ同性で組み合わされた,28(男性ペア14,女性ペア14)組であった.平均年齢は25.45(SD=5.80,範囲=18--38)歳であった.参加者には実験参加に対する謝礼が支払われた.

\vspace{10pt}
\noindent 実験計画

2×2×2の3要因混合計画が用いられた.親近性(知人 vs 初対面)と対面性(対面 vs 非対面)が被験者間要因であり,課題難易度(難 vs 易)が被験者内要因であった.各群のペア数は以下の通りであった;知人/対面群7ペア,知人/非対面群6ペア,初対面/対面群7ペア,初対面/非対面群8ペア.

\vspace{10pt}
\noindent 装置・器具

実験は防音室内で実施された.防音室は防音壁と防音ガラス窓で構成される仕切り壁によって,2つの小部屋に区切られていた.実験中,各参加者はマイクロホン (SONY ECM88)とヘッドフォン (SONY MDR-CD900ST)を装着した.一方の参加者の話す音声はマイクロホンから入力され,音声ミキサー (JVC PS-M3016)を介して,他方の参加者のヘッドフォンへ出力された.これにより,各参加者の音声は互いに回り込むことなく,完全に分離した形で,オーディオワークステーション (TASCAM SX-1)に収録された.また,実験の様子はそれぞれの参加者の正面,側面について別々の小型カメラ (WATEC WAT-204CX)を使用して撮影され,画面分割器 (SONY YS-Q440)を通して,4つの画像情報を1つの画像としてデジタルビデオデッキ (SONY DSR-2000)に収録された.

\begin{figure}[t]
      \begin{center}
      \includegraphics[width=8cm]{stimulus.eps}
      \caption{提示刺激}          
      
      \label{figures}
      \end{center}
\end{figure}

\vspace{10pt}
\noindent 課題と刺激

実験では,図形説明課題が用いられた.図形説明課題とは,2人1組で実施し,参加者の一方が説明者役,他方が回答者役となり,説明者役が口頭で説明する抽象的な線画の形を,回答者役が選択肢の中から再認し,答える課題である.なお,本研究では報告されないが,実験では,写真を見てそこに写る複数の人物間の関係を類推し,2者間で回答を作成する合意形成課題も実施されたが,両課題は性質の異なる課題であると考えられた.つまり,2者間の社会的関係性(例えば,上司と部下,先輩と後輩など)を考慮せず,純粋に理論的に考えると,合意形成課題では,話者間での発話機会は均等に保証されているが,図形説明課題では,説明者役から回答者役への情報提供が主となり,説明者役の発話機会が大きくなる課題設定であり,お互いの発話機会が均等に保証されていない.また,自然な状況での会話に近い合意形成課題と比較して,図形説明課題では役割が設定されているので,個人差が反映しにくい課題であると考えられた.

図形説明課題で用いられた刺激は抽象線画であり,予備調査によって,説明の難易度が統制されていた.予備調査では,ジェスチャと発話の関係を検討した\citeA{Graham-Argyle}で用いられた抽象線画を参考にして作成された12個の刺激図形を大学生男女43名に提示し,それぞれの図形の説明しやすさについて,7段階評定(1:簡単,7:難しい)を求めた.その結果から本実験のために,説明のしやすい図形2つ,しにくい図形2つ,合計4つの図形が選択された(図\ref{figures}参照).

\vspace{10pt}
\noindent 手続き

実験は実験手順の説明,マイク類の装着と音声チェック,図形説明課題,合意形成課題の順で実施された.全体の所要時間は約50分であった.実験に関する説明,ならびに実験課題の提示はパーソナルコンピュータ (SONY VAIO PCG-GR 7/K)により制御された.実験はペアごとに防音室内で実施された.対面性の操作として,これらのペアの半数が間をガラスで隔てた対面 (FTF: face-to-face)条件で実験を行い,残る半数は相手が見えず声のみしか聞こえない非対面 (NFTF: non-face-to-face)条件で実験を実施した.また,親近性の操作として,ペアの半数はお互いが知り合い同士である知人 (F: familiar)条件,残り半数は初対面 (UF: unfamiliar)条件に割り当てられた.初対面条件のペアは事前にまったく面識がない者同士が組み合わされ,相手についてのいかなる情報(年齢や性別など)も告知されていなかった.また,実験開始前の印象やコミュニケーションが実験に何らかの影響を及ぼすことを避けるために,参加者は別々に控え室に入室し,個別に実験手順の説明を受けた.その結果,対面/初対面条件の参加者は実験開始のために防音室内に入室して初めて顔を合わすこととなり,非対面/初対面条件の参加者はマイク類の調整時点で,初めて相手の音声を耳にしたが,実験終了後まで顔を合わせることはなかった.

最初に,各参加者はそれぞれ別々の控室に入室し,コンピュータディスプレイ (SONY LMD-230W)の前に座って,本実験の目的が「ペアで会話を通していくつかの課題を協力して解く」ことであり,実験の概要と注意事項について説明された.以下では,本研究で報告される図形説明課題に絞って記述する.図形説明課題では,参加者は図形を見てその形状を説明する説明者役 (briefer)と,説明者の説明を聞いて選択肢から答えを再認する回答者役 (answerer)に分かれ,図形の形状を伝達するように求められた.各参加者は実験者から指示された役割に従って,説明者役,回答者役を交替した.つまり,1人の参加者は説明者役を2回,回答者役を2回行った.課題難易度として,説明者役が図形を説明する際に,説明のしやすい図形がディスプレイ上に常に表示され参照可能な課題易条件と,説明のしにくい図形が5秒間のみ表示され記憶して説明しなければならない課題難条件が設定された\footnote{本実験では,実験時間の制約上,図形そのものの説明の難易度と課題設定の難易度の4つの組み合わせから,もっとも困難なものが課題難条件,もっとも易しいものが課題易条件として用いられたが,この操作により説明の生成と記憶操作という2つの異なるプロセスが関与することになった.}.結果として,図形説明課題は4セッション存在し(説明者役/課題難条件,説明者役/課題易条件,回答者役/課題難条件,回答者役/課題易条件),その実施順序はペアごとにランダマイズされていた.1回のセッションは最大10分間であり,10分以内に回答に至らない場合には,課題途中でセッションを中止し,次のセッションへと移動した.説明者には回答者が一度で正解できるように,できる限り詳しく,かつ分かりやすく説明するように求められ,回答者には分からないところがあれば,説明者に何度聞き返しても構わないことが説明された.1セッションの流れを以下に示す.

\noindent 1) 説明 説明者側のディスプレイにのみ,課題図形が表示され,説明者はそれを回答者にできる限り詳しく,かつ分かりやすく説明する.この時点で,説明者役にそのセッションが課題難条件で行われるか課題易条件か行われるかが提示される.回答者役は説明を聞き,図形の形が分かれば,「分かりました」と説明者に伝える.

\noindent 2) 確認 回答者は確認のため,その形を説明者役に自分の言葉を使って説明をする.説明者役は回答者役の説明を聞いて,自分の説明がうまく伝わっていると感じたならば,OKを出す.逆に,うまく伝わっていないならば,うまく伝わるまで,1)を反復する.

\noindent 3) 回答 回答者側のディスプレイにのみ6つの選択肢が表示され,回答者はこの中から回答を選択する.回答が正解であれば,次の課題に移る.

\noindent 4) 再回答 回答が不正解の場合は,1)へ戻り,説明を再度確認し,再回答する.制限時間内であれば,正解が出るまでこれを反復する.

最後に,参加者には,密閉性の高い閉鎖環境(防音室)での実験により圧迫感や疲労感を感じた場合には,自らの判断でいつでも実験を中止する権限があり,中止しても不利益を被ることは一切ないことが書類で説明され,すべての説明について了解し,実験参加への同意を得た人のみを実験参加者とした.実験手順の説明終了後,参加者はそれぞれ防音室に移動し,マイク類の装着,及びチェックを行った後,課題を実施した.

\section{結果}

データとして,セッション開始から回答者が発する「分かりました」の合図までの区間,すなわち,説明者役が主な話し手となる「説明」フェーズで出現した心的マーカが対象とされた,セッション設定時間である10分を超過しても回答選択に至らなかったセッションを含む1ペア分のデータは,以下の分析から除外された(各群の最終的なペア数は,知人/対面群6ペア,知人/非対面群6ペア,初対面/対面群7ペア,初対面/非対面群8ペアであった).以下ではまず,収録された対話データの大まかな特徴を明らかにするため,条件別の課題所要時間を報告する.次に,本研究で検討する心的マーカであるフィラー,情動的感動詞,言い差し(途切れ)について,どのような手順でタグ付けされ,分析されたかが説明され,それぞれの発話要素の出現率を報告する.最後に,出現したフィラーの内容分類を行い,種類別出現度数について報告する.

\subsection{課題のセッション所要時間}
図\ref{time}はセッションに要した平均時間を条件別に示したものである.親近性 (2)×対面性 (2)×課題難易度 (2)の分散分析を実施したところ,親近性と課題難易度の交互作用が有意であった (F(1,23) = 5.69, $\mathrm{p}<.05$).親近性の単純主効果を検定したところ,課題難条件では有意であり (F(1,23) = 5.73, $\mathrm{p} < .05$),課題易条件では有意ではなかった (F(1,23) = 1.14).また,課題難易度の単純主効果は知人条件 (F(1,23) = 8.06, $\mathrm{p} < .01$),初対面条件 (F(1,23) = 38.57, $\mathrm{p} < .01$)の両方において有意であった.これらの結果から,本実験で用いられた課題難易度の操作は妥当なものであったことが裏付けられる.また,参加者はお互い初対面で,かつ課題が難しい場合に課題解決のための時間を要していたことが示された.

\begin{figure}[t]
      \begin{center}
      \includegraphics[width=8cm]{time.eps}
      \caption{セッションの平均所要時間(バーは標準誤差)}
      
      \label{time}
      \end{center}
\end{figure}

\subsection{分析手順とタグ付け}
収録された対話データから,心的マーカであるフィラー,情動的感動詞,言い差し(途切れ)を抽出するため,まずは録音された音声がすべて文字化され,次に,2名の判定者があらかじめ定められた定義にしたがって,それぞれ別々にタグ付けを実施した.タグ付けの不一致率は全体の4\%以下であった.判定者間で判断が分かれたものについては,協議の上,最終的な判定を行った.判定者間で揺れが生じた代表的な例は,以下のとおりである.

\noindent(例4)はいアンナーとりあえずな正方形が四つでてきてんやんか\\
\noindent 冒頭の「はい」は応答の先がなく,本研究の定義ではフィラーに属するが,この場合,これから話し始めるという合図であると考えられたため,フィラーには含まれなかった.また,「アンナー」は呼びかけとして使用されることもあるが,状況によりフィラーと区別することが困難であることが多いため,今回はすべてフィラーと判定された.

\noindent(例5)‥今線が二つコウナンテイウンデスカネ斜めに‥\\
\noindent 「コウ」は音声のみから判定が難しいケースが多く,ビデオにより,発話者がハンドジェスチャを伴い,「こんなふうに」の意味で使用している場合はフィラーから除外された.

\subsection{平均出現率}

各条件間を比較するために,分析の基本単位として,正規化された出現率が算出された.まず,100msより長い無音により区切られた一連の発話区間である間休止単位 (IPU: Inter-Pausal Unit, \shortcite{Koiso-etal})を利用し,セッションごとに説明者役,回答者役のそれぞれで課題所要時間中に発話された総IPU度数が算出された.次に,フィラー,情動的感動詞,言い差し(途切れ)の出現度数から各々の出現率(出現度数/IPU度数)が求められた.本実験での課題である図形説明課題は,話し手(説明者)と聞き手(回答者)という役割による発話の質的・量的差異を見込んだ設定であったので,以下では,3つの発話要素の平均出現率について,役割間での違いが結果に反映されているかが確認された上で,それぞれについて親近性 (2)×対面性 (2)×課題難易度 (2)の条件間の平均の差が検討された.


\begin{figure}[tbh]
      \begin{center}
      \includegraphics[width=8cm]{f.eps}
      \caption{フィラーの条件別平均出現率(バーは標準誤差)}
      
      \label{filler_rate}
      \end{center}
\end{figure}

\begin{figure}[t]
      \begin{center}
      \includegraphics[width=8cm]{e.eps}
      \caption{情動的感動詞の条件別平均出現率(バーは標準誤差)}
      \label{affect_rate}
      \end{center}
\end{figure}

\begin{figure}[t]
      \begin{center}
      \includegraphics[width=8cm]{d.eps}
      \caption{言い差しの条件別平均出現率(バーは標準誤差)}
      \label{dis_rate}
      \end{center}
\end{figure}

図\ref{filler_rate},図\ref{affect_rate},図\ref{dis_rate}は,条件別に見たフィラー,情動的感動詞,言い差し(途切れ)の平均出現率を示したものである.それぞれの平均出現率に関して,役割別の効果(説明者/回答者)を検討したところ,フィラーの場合,説明者役の出現率 (M = 0.18, SD = 0.08)は,回答者役 (M = 0.04, SD = 0.04)より約4倍高く (F(1,107) = 174.76, $\mathrm{p} < .01$),情動的感動詞の場合,回答者役 (M = 0.09, SD = 0.06)の出現率が説明者役 (M = 0.03, SD = 0.02)よりも約3倍高く (F(1,107) = 68.73, $\mathrm{p} < .01$),言い差し(途切れ)の場合,説明者役の出現率 (M = 0.08, SD = 0.06)は回答者役 (M = 0.04, SD = 0.04)より約2倍高いことが示された (F(1,107) = 25.32, $\mathrm{p} < .01$).以上から,以降では役割ごとに条件比較がなされた.

図\ref{filler_rate}をもとにした説明者役のフィラー出現率に関する分散分析の結果,親近性の主効果のみが有意であり,知人条件 (M = 0.15, SD = 0.07)は初対面条件 (M = 0.20, SD = 0.08)よりフィラー出現率が低かった (F(1,50) = 6.38, $\mathrm{p} < .05$).フィラーは,全ての条件において差が予想されたが,親近性のみであった理由については,後述のフィラーの種類別の結果を踏まえ,考察で議論される.一方,回答者役に対する分析の結果,有意な効果は存在しなかった ($\mathrm{F} < 1$).

次に,図\ref{affect_rate}をもとにした説明者役に対する情動的感動詞の平均出現率に関する分散分析の結果,親近性と課題難易度の交互作用が有意であった (F(1,50) = 5.99, $\mathrm{p} < .05$).親近性の単純主効果を検定したところ,課題が難しい場合,知人より初対面の方が出現率が高い傾向にあったが (F(1,50) = 2.95, $\mathrm{p} < .10$),課題が易しい場合には差がなかった (F(1,50) = 1.61).また,初対面同士の場合,課題が難しいと出現率が高い傾向にあるが (F(1,50) = 3.36, $\mathrm{p} < .10$),知人の場合にはそうではなかった (F(1,50) = 2.65).一方,回答者役に対する分析の結果,二次の交互作用が有意であった (F(1,50) = 4.27, $\mathrm{p} < .05$).そこで,対面・非対面条件別に親近性×課題難易度の単純交互作用を分析した.対面状況での交互作用が有意傾向にあったので (F(1,22) = 3.38, $\mathrm{p} < .10$),水準別誤差項を用いた単純・単純主効果検定の結果,課題が難しい場合には知人よりも初対面同士の方が出現率が高かったが (F = 14.44, $\mathrm{p} < .05$),課題が易しい場合にはその差はなかった.また,親近性条件による課題難易度の差はいずれも存在しなかった.一方で,非対面状況では,有意な効果は何も存在しなかった.

図\ref{dis_rate}をもとにした説明者役の言い差し(途切れ)の平均出現率に関する分散分析の結果,課題難易度の主効果 (F(1,50) = 7.93, $\mathrm{p} < .01$)と親近性と対面性の交互作用が有意であった (F(1,50) = 4.50, $\mathrm{p} < .05$).交互作用について,親近性の単純主効果を検定したところ,非対面条件では初対面より知人同士の方が出現率が高いが (F(1,50) = 26.04, $\mathrm{p} < .01$),対面条件では有意差がなかった (F(1,50) = 2.09).また,対面性の単純主効果は知人条件の場合,非対面の方が出現率が高く (F(1,50) = 13.77, $\mathrm{p} < .01$),初対面条件の場合には,対面の方が出現率が高かった (F(1,50) = 8.06, $\mathrm{p} < .01$).一方,回答者役に対する分散分析の結果,親近性と対面性の交互作用が有意傾向であった (F(1,50) = 2.93, $\mathrm{p} < .10$).親近性の単純主効果を検定したところ,対面条件では有意差がないが (F(1,50) = 0.06),非対面条件では知人同士の方が出現率が高かった (F(1,50) = 7.31, $\mathrm{p} < .05$).また,対面性の単純主効果は知人条件 (F(1,50) = 1.71),初対面条件 (F(1,50) = 2.73)ともに有意でなかった.以上,出現率に関する結果のまとめを表\ref{marker_trend}に示す.


\begin{table}[t]
\caption{心的マーカの条件別出現率の結果まとめ} \label{marker_trend}
\scriptsize
\begin{tabular}{c c c c c c c} \hline
 & \multicolumn{3}{c}{説明者役} & \multicolumn{3}{c}{回答者役}\\
 & フィラー & 情動的感動詞 & 言い差し & フィラー & 情動的感動詞 & 言い差し\\ \hline
親近性 & \multirow{2}{19mm}{知人$<$初対面**} & 知人 $<$ 初対面+
 & 知人 $>$ 初対面** &  & 知人 $<$ 初対面** & 知人 $>$ 初対面*\\
 & &(課題難条件) & (非対面条件) & & (課題難・対面条件) & (非対面条件)\\
対面性 &  &  & 対面 $<$ 非対面** &  &  & \\
 & & & (知人条件) & & & \\
 &  &  & 対面 $>$ 非対面** &  &  & \\
 & & & (初対面条件) & & & \\
課題難易度 &  & 難 $>$ 易+ & \multirow{2}{12mm}{難 $>$ 易**} &  &  & \\
 & & (初対面条件) & & & &\\ \hline
\multicolumn{5}{l}{** $\mathrm{p} < .01$, * $\mathrm{p} < .05$, + $\mathrm{p} < .10$}
\end{tabular}
\end{table}


次に,説明者役のフィラー,情動的感動詞,言い差し(途切れ)が回答者の回答選択に影響したかどうかを検討するために,セッション中の回答者役による回答が一度で正解されたか,それとも再回答を要したかどうかで条件に分け,説明者役のフィラー,情動的感動詞,言い差し(途切れ)の平均出現率を検討した.分散分析の結果,フィラーと言い差しを指標とした場合には,一度で正解に至った場合とそうでない場合とに出現率の違いはなかったが,情動的感動詞では,一度で正解に至らない場合にその出現率が高かった (F(1,106) = 4.77, $\mathrm{p} < .05$).

\subsection{フィラーの種類別分析}
表\ref{ab}はセッション中に出現したフィラー度数を種類別に分類し,説明者役と回答者役に分けて示したものである.説明者役は総数で回答者役の約6倍のフィラーを発していた.種類別に見ると,説明者役,回答者役ともに,「エート」「アノ」「エー」「ソノ」「ナンカ」で全体の8割以上を占めており,説明者役と回答者役でその構成比率にほとんど違いはなかった.

\begin{table}[b]
\caption{フィラーの種類別構成} \label{ab}
\begin{center}
\small
\begin{tabular}{llrrrrrrr} \hline
 &  & エート & アノ & エー & ソノ & ナンカ & その他 & 合計 \\ \hline
説明者役 & No & 378 & 232 & 176 & 100 & 118 & 212 & 1216 \\
 & \% & \phantom{0}31 & \phantom{0}19 & \phantom{0}14 & \phantom{00}8 & \phantom{00}10 & \phantom{0}17 & \phantom{0}100 \\
回答者役 & No & \phantom{0}80 & \phantom{0}35 & \phantom{0}22 & \phantom{0}19 & \phantom{0}11 & \phantom{0}38 & \phantom{0}205 \\
 & \% & \phantom{0}39 & \phantom{0}17 & \phantom{0}10 & \phantom{00}9 & \phantom{00}5 & \phantom{0}19 & \phantom{0}100 \\ \hline
\end{tabular}
\end{center}
\end{table} 


表\ref{filler_kind}に種類別条件別出現率を示す.表\ref{filler_kind}に基づき,説明者役に対して,親近性×対面性×難易度の分散分析を行った\footnote{本研究では主に話し手の発話プロセスに着目するので,以下の分析では,回答者役を除外する.なお,回答者役の場合,フィラーの種類によってはセッション中に現れないものが存在したことも分析対象としない1つの理由であった.}.「エート」の場合,初対面条件の方が知人条件よりも有意に出現率が高かった (F(1,50) = 10.13, $\mathrm{p} < .01$).「アノ」「エー」の場合,どちらも条件による有意な差はなかった.「ソノ」の場合,課題難条件の出現率が課題易条件よりも高かった (F(1,50) = 5.09, $\mathrm{p} < .05$).「ナンカ」の場合,親近性と対面性の交互作用が有意傾向であり (F(1,50) = 3.07, $\mathrm{p} < .10$),それぞれの単純主効果の分析の結果,非対面条件では,知人同士の方が初対面同士よりも出現率が高く (F(1,50) = 6.52, $\mathrm{p} < .05$),知人条件では,非対面条件の方が,対面条件よりも出現率が高かった (F(1,50) = 5.28, $\mathrm{p} < .05$).これらの結果を表\ref{filler_trend}にまとめる.

\begin{table}[t]
\caption{フィラーの種類別条件別出現率(説明者役)} \label{filler_kind}
\begin{center}
\footnotesize
\begin{tabular}{c c c c c c c c c c c c c c} \hline
 &  & \multicolumn{2}{c}{エート} &  \multicolumn{2}{c}{アノ} & \multicolumn{2}{c}{エー} &  \multicolumn{2}{c}{ソノ} &  \multicolumn{2}{c}{ナンカ} & \multicolumn{2}{c}{その他}\\
 &  & 難 & 易 & 難 & 易 & 難 & 易 & 難 & 易 & 難 & 易 & 難 & 易\\ \hline
F/FTF & M & 0.02  & 0.02  & 0.03  & 0.02  & 0.01  & 0.01  & 0.02  & 0.02  & 0.02  & 0.01  & 0.10  & 0.07 \\
 & SD &  0.02  &  0.03  &  0.05  &  0.03  &  0.02  &  0.02  &  0.02  &  0.03  &  0.03  &  0.01  &  0.07  &  0.05 \\
F/NFTF & M & 0.05  & 0.06  & 0.03  & 0.03  & 0.02  & 0.02  & 0.01  & 0.00  & 0.03  & 0.03  & 0.15  & 0.15 \\
 & SD &  0.04  &  0.05  &  0.03  &  0.04  &  0.04  &  0.02  &  0.02  &  0.01  &  0.03  &  0.04  &  0.07  &  0.05 \\
UF/FTF & M & 0.07  & 0.07  & 0.04  & 0.04  & 0.03  & 0.03  & 0.01  & 0.01  & 0.01  & 0.01  & 0.17  & 0.16 \\
 & SD &  0.06  &  0.06  &  0.07  &  0.06  &  0.04  &  0.04  &  0.02  &  0.02  &  0.02  &  0.01  &  0.06  &  0.08 \\
UF/NFTF & M & 0.07  & 0.07  & 0.02  & 0.02  & 0.05  & 0.04  & 0.02  & 0.01  & 0.01  & 0.01  & 0.17  & 0.16 \\
 & SD &  0.05  &  0.04  &  0.02  &  0.04  &  0.07  &  0.07  &  0.03  &  0.02  &  0.02  &  0.02  & 0.08  & 0.08 \\ \hline
\end{tabular}
\end{center}
\end{table}

\begin{table}[t]
\caption{フィラー(上位5種)の種類別条件別出現率(説明者役)のまとめ} \label{filler_trend}
\begin{center}
\scriptsize
\begin{tabular}{c c c c c c c} \hline
 & エート & アノ & エー & ソノ & ナンカ\\ \hline
\multirow{2}{13mm}{親近性} & \multirow{2}{21mm}{知人 $<$ 初対面**} & & & & 知人 $>$ 初対面*\\
 & & & & & (非対面条件)\\ 
\multirow{2}{13mm}{対面性} &  &  &  &  & 対面 $<$ 非対面*\\
 & & & & &  (知人条件)\\
\multirow{2}{13mm}{難易度} &  & & & \multirow{2}{11mm}{難 $>$ 易*} & \\
 & & & & &\\ \hline
\multicolumn{4}{l}{** $\mathrm{p} < .01$, * $\mathrm{p} < .05$, + $\mathrm{p} < .10$} & & \\
\end{tabular}
\end{center}
\end{table}

\section{考察}
実験の結果,課題難易度が高くかつ初対面の場合には,セッション所要時間が増加することが示された.単純に考えれば,時間増加に伴って発話量も増加し,フィラー,情動的感動詞,言い差し(途切れ)も増加すると考えられる.そこで,この3つの発話要素の出現率について分析したところ,フィラーでは,説明者役の場合,回答者役よりも出現率が高かった.また,初対面条件の場合,知人条件よりも出現率が高かったが,課題難易度の影響は見られなかった.これに対し,情動的感動詞では,フィラーとは逆に回答者役の場合,説明者役よりも出現率が高く,課題難・対面条件で初対面条件の場合,知人条件よりも出現率が高かった.また,言い差し(途切れ)は,説明者役の場合,回答者役よりも出現率が高く,説明者役の場合,難易度が影響した.また,一度で正解に至ったかどうかに,説明者役の心的マーカがどう影響を与えたかを分析した結果,情動的感動詞のみが,一度で正解に至らなかった場合に,正解した場合よりも出現率が高かった.さらに,フィラーの種類別の検討から,「エート」は,初対面条件で出現率が上がったのに対し,「アノ」や「エー」では,条件による出現率の差はなかった.また,「ソノ」は課題難条件において出現率が上がり,「ナンカ」に関しては,非対面条件では,知人同士のほうが出現率が高まり,知人条件では,非対面のほうが出現率が高かった.以上の結果から,本研究における状況変数が,3つの心的マーカの出現の仕方に,さらにフィラーに関しては,種類の違いによって,それぞれ異なる影響を及ぼしたことが示唆される.つまり,この違いは,心的マーカの背後にある内的処理プロセスの違いを反映して有標化したものであると考えることができる.以下では,本研究で得られた結果を,1) 従来の研究結果と本研究の結果の一致点および相違点,2) 話し手の内的処理プロセスと心的マーカとの対応,3) 本研究で得られた結果の応用可能性の観点から考察する.

まず,従来の研究結果と本研究の結果の一致点および相違点を取り上げると,これまでの研究においては,フィラーは主に「場の改まり度(フォーマルかインフォーマルか;例えば,\citeA{Philips})」の影響を受けることが示されてきたが,本研究結果で,親近性条件に差があったということは,対話相手が初対面である場合,知人同士である場合より,改まった状況であると認識されることを考慮すれば,先行研究の結果を確認したことになる.一方,先行研究においては,対話の伝達内容の複雑さがフィラーを含めた発話における非流暢性の出現量に影響を与えることが示されてきた\cite{Goldman-Eisler}.本研究のフィラー全体の結果では,課題難易度の影響は見られなかった.ただし,種類別で見ると,「ソノ」に難易度の影響が見られたことから,フィラーに関しては種類別に分けた検討が必要であることがわかる.言い差し(途切れ)は,課題難易度の影響を受けており,これは先行研究と一致していた.また,フィラーには,後続する発話の内容の傾向(複雑さや重要度)を聞き手に予期させ,聞き手の理解を促す効果があることが示されてきたが (\shortciteA{FoxTree,渡辺}),本研究における,一度での正解率には影響は見られなかった\footnote{この理由として,フィラーの有無とは別に,言葉の曖昧さ(例えば,「底辺がへこんでいる」と表現した場合のへこみは三角形の内側か外側か)や,課題の設定上,説明者役には回答者側に表示される選択肢は見えず,どこまで説明すれば選択肢間の差異を表現できるかが不明であったため,説明の詳細化に関して個人差があり,回答者役の理解が直接正解につながらない場合があったことなども考えられる.}.次に,本研究でのフィラー種類の構成比率において,「エート」が3割を超えていた.これは,平均して1割程度,多くても2割以下という構成比率を示す先行研究の結果\cite[など]{山根}よりも多かった.先行研究では,「エート」は,主に発話者が記憶操作や計算・検索などの心的操作を行っている標識であると考えられてきた\cite{定延・田窪}が,本研究ではそれに対応する難易度(記憶操作)の影響はなく,これまであまり関係ないとされていた対人的な要因が影響したことを示している.ただし,\citeA{山根}が言及するように,「エート」は沈黙を回避するために選択的に使われる傾向があり,不要な間を開けることを避けるために,知人条件よりも初対面条件において多く使用された可能性がある.これとは逆に,一般に「アノ」は対人的な言語表現編集の標識や表現の和らげの機能を持つとされてきたが,本研究の結果では,親近性の影響は見られなかった.後述するように,「アノ」の持つ,対人的な機能の側面が,本研究の課題設定では現れにくかった可能性がある.先行研究と異なる結果になった理由については,前後の文脈関係を含めたより詳細な分析が必要であり,また本研究での図形説明課題とは異なる課題設定での検討も求められるだろう.

次に,話し手の内的処理プロセスと心的マーカとの対応について,表\ref{map_speaker}で予想されたことと,本研究の結果(表\ref{marker_trend},表\ref{filler_trend})とを照らして考察する.まず,親近性に関して,フィラーに影響が見られたこと,難易度に関して,情動的感動詞,言い差しに影響が見られたことは,予想と一致した.しかしながら,対面性に関して,影響が見られると予想されたフィラーには,影響が見られなかった.その理由としては,参加者の多くが対面であることの利点の1つであると考えられるジェスチャなどの視覚的伝達手段を,あまり使用しなかったことが挙げられる.実験教示において,「ジェスチャを使用して良い(悪い)」という説明をしなかったこと,マイクとヘッドフォンを装着していたため,動きが制約されていたことなどから,結果的に,ジェスチャを使わないという潜在的な抑制が働いていた可能性が考えられる.この点は今後の実験設定,教示方法の検討材料である.また,言い差しに関して,知人同士か,初対面同士かによって,対面性の影響が逆転した.顔が見えることは,知人同士では安心感を,逆に初対面同士では緊張を生み出す要因として働いた可能性が考えられるだろう.さらに,フィラーの種類別の影響に関しては,難易度の影響を受けると予想された「ソノ」では予測通りの結果となったが,親近性条件の影響を受けると予想された表現の選択に関するマーカである「アノ」では影響は見られず,予想外の「エート」で親近性の影響が見られた.「エート」が「アノ」の代わりに使われたような結果となったのは,実験課題の性質上,「アノ」が出現しやすいとされる「言いにくいことを和らげる」という状況(依頼状況)が少なかったために,「アノ」の出現率が抑制され,一方,沈黙回避の必要性が増加したために,「エート」の出現率が増加したことが,理由の1つとして考えられる.また,「ナンカ」に関しては,非対面条件において,知人同士の出現率が初対面同士を上回っていた.通常,「ナンカ」は具現化できない何かがあり,それを模索中であることを示す標識であるとされ,特に「ナンテイウカ」は,操作に時間を要していることを表示するので,表象の言語化過程に負荷のかかる対面性の影響が見られると予測されたが,その影響は知人条件のみに限定されていた.これには前述のジェスチャの使用抑制が関係している可能性がある.また,本研究では出現率を分析対象としたが,説明所要時間を指標とした場合には難易度の影響が見られたことから,各状況変数が,フィラーの出現率以外の指標,例えば,個々のフィラーの表出継続時間や語形などに影響を与えた可能性も考えられる.本研究では,例えば,「エート」と「エーットデスネー」,「ナンカ」と「ナンテイッタライインデスカネー」を同じカテゴリに分類したが,今後,より詳細な条件設定の下で,より多くのデータを収集した上で,カテゴリ内の各語の語形や音響特性,継続時間長などの特徴別に分析することが必要となるだろう.

最後に,本研究で得られた結果の応用可能性として,例えば,心的マーカをリアルタイムに検出,分類し,類推するシステムを,ヒューマノイドロボットやエージェントに実装することによって,システム側が,人間の話し手/聞き手の内的処理プロセスを類推できれば,それに応じてシステム側は柔軟な対応を生成でき,自然なインタラクションを生み出すことができると期待される.しかしながら,本研究は実験的設定のもとに得られた対話を分析した結果であるため,その応用可能性は限定的であり,また,必ずしも心的マーカの普遍的性質を示し得たわけではない.今回取り上げたものとは異なる状況変数(例えば,上司と部下などの社会的関係,性別など)や,3つの発話要素以外の心的マーカの要素(例えば,ジェスチャや表情,視線,ポーズなど)についてさらに詳細なる分析を進めていく必要がある.現在,我々は,時間情報を含む,フィラー,情動的感動詞,言い差し(途切れ)といった心的マーカをタグ付けした対話コーパスの作成を進めている.この対話コーパスを使用して,より質的な分析,例えば,話者交替や発話の連鎖の中での心的マーカの表れ方の違いについてや,条件ごとのフィラーや情動的感動詞の音響的特性の分析を通じて,心的マーカによる内的処理プロセスの理解がより一層深化することが期待される.


\acknowledgment

論文の改稿にあたり,多くの貴重なご助言を頂きました査読者に深く感謝いたします.

\bibliographystyle{jnlpbbl_1.2}
\begin{thebibliography}{}

\bibitem[\protect\BCAY{Brown}{Brown}{1977}]{Brown}
Brown, G. \BBOP 1977\BBCP.
\newblock {\Bem Listening to spoken English}.
\newblock Longman, London.

\bibitem[\protect\BCAY{Clark}{Clark}{2002}]{Clark:02}
Clark, H.~H. \BBOP 2002\BBCP.
\newblock \BBOQ Speaking in time\BBCQ\
\newblock {\Bem Speech Communication}, {\Bbf 36}, \mbox{\BPGS\ 5--13}.

    \bibitem[\protect\BCAY{Clark \BBA\ Fox Tree}{Clark \BBA\ Fox Tree}{2002}]{Clark-Tree}
Clark, H.~H.\BBACOMMA\ \BBA\ Tree, J. E.~F. \BBOP 2002\BBCP.
\newblock \BBOQ Using uh and um in spontaneous speaking\BBCQ\
\newblock {\Bem Cognition}, {\Bbf 84}, \mbox{\BPGS\ 73--111}.

\bibitem[\protect\BCAY{Ekman}{Ekman}{1972}]{Ekman}
Ekman, P. \BBOP 1972\BBCP.
\newblock \BBOQ Universal and cultural differences in facial expressions of
  emotions\BBCQ\
\newblock In Code, J.\BED, {\Bem Nebraska Symposium on Motivation},
  \mbox{\BPGS\ 207--283}. University of Nebraska press, Lincoln.

\bibitem[\protect\BCAY{Ekman \BBA\ Friesen}{Ekman \BBA\
  Friesen}{1967}]{Ekman-Friesen}
Ekman, P.\BBACOMMA\ \BBA\ Friesen, W.~V. \BBOP 1967\BBCP.
\newblock \BBOQ Head and Body cues in the judgement of emotion: A
  reformulation\BBCQ\
\newblock {\Bem Perceptual and Motor Skills}, {\Bbf 24}, \mbox{\BPGS\
  711--724}.

\bibitem[\protect\BCAY{Ferreira \BBA\ Bailey}{Ferreira \BBA\
  Bailey}{2004}]{Ferreira-Bailey}
Ferreira, F.\BBACOMMA\ \BBA\ Bailey, K. G.~D. \BBOP 2004\BBCP.
\newblock \BBOQ Disfluencies and human language comprehension\BBCQ\
\newblock {\Bem TRENDS in Cognitive Sciences}, {\Bbf 8}(5), \mbox{\BPGS\
  231--237}.

\bibitem[\protect\BCAY{Fox~Tree}{Fox~Tree}{1995}]{FoxTree}
Fox~Tree, J.~E. \BBOP 1995\BBCP.
\newblock \BBOQ The effects of false starts and repetitions on the processing
  of subsequent words in spontaneous speech\BBCQ\
\newblock {\Bem Journal of Memory and Language}, {\Bbf 34}, \mbox{\BPGS\
  709--738}.

\bibitem[\protect\BCAY{Goldman-Eisler}{Goldman-Eisler}{1968}]{Goldman-Eisler}
Goldman-Eisler, F.~G. \BBOP 1968\BBCP.
\newblock {\Bem Psycholinguistics: Experiments in spontaneous speech}.
\newblock Academic Press, N.Y.

\bibitem[\protect\BCAY{Graham \BBA\ Argyle}{Graham \BBA\
  Argyle}{1975}]{Graham-Argyle}
Graham, J.~A.\BBACOMMA\ \BBA\ Argyle, M.~A. \BBOP 1975\BBCP.
\newblock \BBOQ A cross-cultural study of the communication of extra-verbal
  meaning by gestures\BBCQ\
\newblock {\Bem International Journal of Psychology}, {\Bbf 10}, \mbox{\BPGS\
  57--67}.

\bibitem[\protect\BCAY{Heritage}{Heritage}{1998}]{Heritage}
Heritage, J. \BBOP 1998\BBCP.
\newblock \BBOQ Oh-prefaced response to inquiry\BBCQ\
\newblock {\Bem Language in Society}, {\Bbf 27}(3), \mbox{\BPGS\ 291--334}.

\bibitem[\protect\BCAY{Hickson, Stacks, \BBA\ Moore}{Hickson
  et~al.}{2004}]{Hickson}
Hickson, I.~M., Stacks, D.~W., \BBA\ Moore, N. \BBOP 2004\BBCP.
\newblock {\Bem Nonverbal communication: studies and applications (4th ed.)}.
\newblock Roxbury Publishing Co., Los Angeles, California.

\bibitem[\protect\BCAY{堀内\JBA 中野\JBA 小磯\JBA 石崎\JBA 鈴木\JBA 岡田\JBA
  仲\JBA 土屋\JBA 市川}{堀内\Jetal }{1999}]{堀内-99}
堀内靖雄\JBA 中野有紀子\JBA 小磯花絵\JBA 石崎雅人\JBA 鈴木浩之\JBA
  岡田美智男\JBA 仲真紀子\JBA 土屋俊\JBA 市川熹 \BBOP 1999\BBCP.
\newblock \JBOQ 日本語地図課題対話コーパスの設計と特徴\JBCQ\
\newblock \Jem{人工知能学会誌}, {\Bbf 14}, \mbox{\BPGS\ 261--272}.

\bibitem[\protect\BCAY{伊藤}{伊藤}{1994}]{伊藤}
伊藤友彦 \BBOP 1994\BBCP.
\newblock \Jem{幼児の発話における非流暢性に関する言語心理学的研究}.
\newblock 風間書房, 東京.

\bibitem[\protect\BCAY{人工知能学会「談話・対話研究におけるコーパス利用」研究
グループ}{人工知能学会「談話・対話研究におけるコーパス利用」研究グループ}{2002
}]{Slash-Manual}
人工知能学会「談話・対話研究におけるコーパス利用」研究グループ \BBOP 2002\BBCP.
\newblock \Jem{日本語スラッシュ単位(発話単位)ラベリングマニュアル}.

\bibitem[\protect\BCAY{Koiso, Horiuchi, Tutiya, Ichikawa, \BBA\ Den}{Koiso
  et~al.}{1998}]{Koiso-etal}
Koiso, H., Horiuchi, Y., Tutiya, S., Ichikawa, A., \BBA\ Den, Y. \BBOP
  1998\BBCP.
\newblock \BBOQ An analysis of turn-taking and backchannels based on prosodic
  and syntactic features in Japanese Map Task dialogues\BBCQ\
\newblock {\Bem Language and Speech}, {\Bbf 41}, \mbox{\BPGS\ 295--321}.

\bibitem[\protect\BCAY{前川\JBA 籠宮\JBA 小磯\JBA 小椋\JBA 菊池}{前川\Jetal
  }{2000}]{CSJ}
前川喜久雄\JBA 籠宮隆之\JBA 小磯花絵\JBA 小椋秀樹\JBA 菊池英明 \BBOP 2000\BBCP.
\newblock \JBOQ 日本語話し言葉コーパスの設計\JBCQ\
\newblock \Jem{音声研究}, {\Bbf 4}(2), \mbox{\BPGS\ 51--61}.

\bibitem[\protect\BCAY{水上\JBA 山下}{水上\JBA 山下}{2005}]{水上・山下}
水上悦雄\JBA 山下耕二 \BBOP 2005\BBCP.
\newblock \JBOQ 図形説明課題遂行時のフィラーとポーズの関係\JBCQ\
\newblock \Jem{人工知能学会研究会資料}, SIG-SLUD-A501, \mbox{\BPGS\
  45--50}.

\bibitem[\protect\BCAY{森山}{森山}{1996}]{森山:96}
森山卓郎 \BBOP 1996\BBCP.
\newblock \JBOQ 情動的感動詞考\JBCQ\
\newblock \Jem{語文}, {\Bbf 65}, \mbox{\BPGS\ 51--62}.

\bibitem[\protect\BCAY{村井}{村井}{1970}]{村井}
村井潤一 \BBOP 1970\BBCP.
\newblock \Jem{言語機能の形成と発達}.
\newblock 風間書房, 東京.

\bibitem[\protect\BCAY{野村}{野村}{1996}]{野村}
野村美穂子 \BBOP 1996\BBCP.
\newblock \JBOQ
  大学の講義における文科系の日本語と理科系の日本語---「フィラー」に注目して---
\JBCQ\
\newblock \Jem{文教大学教育研究所紀要}, 第5号.

\bibitem[\protect\BCAY{Philips}{Philips}{1998}]{Philips}
Philips, M.~K. \BBOP 1998\BBCP.
\newblock {\Bem Discourse markers in Japanese: connectives, fillers, and
  interactional particles}.
\newblock Doctoral dissertaion, Michigan State University, East Lansing, MI.

\bibitem[\protect\BCAY{定延}{定延}{2005}]{定延:05}
定延利之 \BBOP 2005\BBCP.
\newblock \Jem{ささやく恋人,りきむレポーター---口の中の文化---}.
\newblock 岩波書店, 東京.

\bibitem[\protect\BCAY{定延\JBA 中川}{定延\JBA 中川}{2005}]{定延・中川}
定延利之\JBA 中川明子 \BBOP 2005\BBCP.
\newblock \JBOQ
  非流ちょう性への言語学的アプローチ---発音の延伸,とぎれを中心に---\JBCQ\
    \newblock 串田秀也\JBA 定延利之\JBA 伝康晴\JEDS,\Jem{活動としての文と発話},
  \mbox{\BPGS\ 209--228}. ひつじ書房.

\bibitem[\protect\BCAY{定延\JBA 田窪}{定延\JBA 田窪}{1995}]{定延・田窪}
定延利之\JBA 田窪行則 \BBOP 1995\BBCP.
\newblock \JBOQ
  談話における心的操作モニタ機構—心的操作標識「ええと」と「あの(ー)」—\JBCQ\
\newblock \Jem{言語研究}, {\Bbf 108}, \mbox{\BPGS\ 74--93}.

\bibitem[\protect\BCAY{Schiffrin}{Schiffrin}{1978}]{Schiffrin}
Schiffrin, D. \BBOP 1978\BBCP.
\newblock {\Bem Discourse markers}.
\newblock Cambridge University Press, UK.

\bibitem[\protect\BCAY{Swerts}{Swerts}{1998}]{Swerts}
Swerts, M. \BBOP 1998\BBCP.
\newblock \BBOQ Filled pauses as markers of disourse structure\BBCQ\
\newblock {\Bem Journal of Pragmatics}, {\Bbf 30}, \mbox{\BPGS\ 485--496}.

\bibitem[\protect\BCAY{田窪\JBA 金水}{田窪\JBA 金水}{1997}]{田窪・金水}
田窪行則\JBA 金水敏 \BBOP 1997\BBCP.
\newblock \JBOQ 応答詞・感動詞の談話的機能\JBCQ\
    \newblock 音声文法研究会\JED,\Jem{文法と音声}, \mbox{\BPGS\ 257--279}.
  くろしお出版.

\bibitem[\protect\BCAY{田中}{田中}{1995}]{田中}
田中敏 \BBOP 1995\BBCP.
\newblock \Jem{スピーチの言語心理学モデル}.
\newblock 風間書房, 東京.

\bibitem[\protect\BCAY{富樫}{富樫}{2005}]{富樫:05}
富樫純一 \BBOP 2005\BBCP.
\newblock \JBOQ
  驚きを伝えるということ---感動詞「あっ」と「わっ」の分析を通じて---\JBCQ\
    \newblock \linebreak 串田秀也\JBA 定延利之\JBA 伝康晴\JEDS,\Jem{活動としての文と発話},
  \mbox{\BPGS\ 229--251}. ひつじ書房.

\bibitem[\protect\BCAY{土屋}{土屋}{2000}]{土屋}
土屋菜穂子 \BBOP 2000\BBCP.
\newblock \JBOQ 対話コーパスを用いた言い淀みの統語論的考察\JBCQ\
\newblock \Jem{青山語文}, {\Bbf 30}, \mbox{\BPGS\ 13--26}.

\bibitem[\protect\BCAY{Watanabe}{Watanabe}{2002}]{Watanabe}
Watanabe, M. \BBOP 2002\BBCP.
\newblock \BBOQ Fillers and connectives as discourse segment boundary markers
  in an academic monologue in Japanese\BBCQ\
\newblock \Jem{東京大学留学生センター紀要}, {\Bbf 12}, \mbox{\BPGS\ 107--119}.

\bibitem[\protect\BCAY{渡辺\JBA 広瀬\JBA 伝\JBA 峯松}{渡辺\Jetal }{2006}]{渡辺}
渡辺美知子\JBA 広瀬啓吉\JBA 伝康晴\JBA 峯松信明 \BBOP 2006\BBCP.
\newblock \JBOQ 音声聴取時のフィラーの働き---「エート」による後続句の複雑さ予
  測---\JBCQ\
\newblock \Jem{日本音響学会誌}, {\Bbf 62}, \mbox{\BPGS\ 370--378}.

\bibitem[\protect\BCAY{山根}{山根}{2002}]{山根}
山根智恵 \BBOP 2002\BBCP.
\newblock \Jem{日本語の談話におけるフィラー}.
\newblock 日本語研究叢書15. くろしお出版.

\bibitem[\protect\BCAY{好井}{好井}{1999}]{好井}
好井裕明 \BBOP 1999\BBCP.
\newblock \Jem{会話分析への招待}.
\newblock 世界思想社, 東京.

\end{thebibliography}

\begin{biography}

\bioauthor{山下 耕二}{
1999年大阪大学大学院人間科学研究科博士課程修了.博士(人間科学).
神戸商船大学地域共同研究センター講師を経て,2001年より(独)通信総合研究所特別研究員となり,現在,(独)情報通信研究機構専攻研究員.コミュニケーション(非言語,メディア)の研究に従事.日本心理学会,日本認知科学会各会員.}

\bioauthor{水上 悦雄}{
1992年神戸大学理学部卒業.1997年同大学院博士課程修了.博士(理学).
2000年学習院大学計算機センター助手.2003年通信総合研究所専攻研究員,
現在,(独)情報通信研究機構自然言語グループ専攻研究員.
言語・非言語コミュニケーションの研究に従事.
認知科学会,人工知能学会,動物行動学会各会員.}

\end{biography}






\biodate


\end{document}
