    \documentclass[japanese]{jnlp_1.4}
\usepackage{jnlpbbl_1.1}
\usepackage[multi]{otf}
\usepackage{mlotf}
\usepackage{udline}
\usepackage{longtable}
\usepackage{array}
\usepackage[dvipdfm]{graphicx}

\Volume{14}
\Number{5}
\Month{Oct.}
\Year{2007}

\received{2007}{1}{5}
\revised{2007}{4}{14}
\accepted{2007}{6}{29}

\setcounter{page}{65}

\setulminsep{1.2ex}{0.2ex}
\newcommand{\inHZ}{}
\newcommand{\inHRei}[1]{}
\addtolength{\tabcolsep}{-0.25pt}
\makeatletter
\def\LT@makecaption#1#2#3{}
\makeatother
\jtitle{日中機械翻訳における存在表現の翻訳処理について}
\jauthor{王  軼謳\affiref{Author_1} \and 池田 尚志\affiref{Author_2}}
\jabstract{
存在文はいかなる言語にも存在し,人間のもっとも原始的な思考の言語表現の一つであって,それぞれの言語で特徴があり言語により異なりが現れてくる.存在表現の意味上と構文上の多様さのために,更に中国語との対応関係の複雑さのために,日中機械翻訳において,曖昧さを引き起こしやすい.現在の日中市販翻訳ソフトでは,存在表現に起因する誤訳(訳語選択,語順)が多く見られる.本論文では,日中両言語の存在表現における異同について考察し,日中機械翻訳のために,日本語文の構文特徴,対応名詞の属性,中国語文の構文構造などを利用して存在動詞の翻訳規則をまとめ,存在表現の翻訳方法について提案した.これらの翻訳規則を我々の研究室で開発している日中機械翻訳システムJaw/Chineseに組み込んで,翻訳実験を行った.更に手作業による翻訳実験も加えて,この規則を検証し,良好な評価を得た.
}
\jkeywords{日中機械翻訳,存在表現,意味解析,構文特徴,翻訳規則}

\etitle{A solution for the problem of Existential Expressions in Japanese-Chinese Machine Translation}
\eauthor{Wang Yiou\affiref{Author_1} \and Ikeda Takashi\affiref{Author_2}} 
\eabstract{
Existential sentence as one of primitive sentence patterns is very important for each language, and has characteristics of itself for different language. However the variety of syntactic and semantic use of existential expression and complicated correspondence to Chinese leads to ambiguities in Japanese-Chinese machine translation. Therefore there are numerous mistranslations by the currently commercially available translation software in existential expression, such as vocabulary selection and word order determination.
In this paper, we propose a method for handling the existential verbs based on the constraint of Japanese syntactic and semantic features, Chinese syntactic features, the attributes of the related nouns and so on. Furthermore we implement the translation rules in Jaw/Chinese which is the Japanese to Chinese translation system developed by our lab and evaluate our rules. And we also made manual experiment over 700 existential sentences and get an accuracy of about 90\%, which is rather high compared to the currently commercially available translation software. Both of the evaluations indicate that our method provides a high accuracy and is available.
}
\ekeywords{Japanese-Chinese machine translation, existential expression, semantics analysis, syntactic and semantic features, translation rules}

\headauthor{王,池田}
\headtitle{日中機械翻訳における存在表現の翻訳処理について}

\affilabel{Author_1}{岐阜大学大学院工学研究科}{Graduate school of Engineering, Gifu University}
\affilabel{Author_2}{岐阜大学工学部応用情報学科}{Department of Information Science, Faculty of Engineering, Gifu University}



\begin{document}
\maketitle



\section{はじめに}

存在文はいかなる言語にも存在し,人間のもっとも原始的な思考の言語表現の一つであって,それぞれの言語で特徴があり言語により異なりが現れてくる.日本語と中国語でも,前者が存在の主体が有情物か非情物かで使われる動詞が異なる(「ある/いる」)のに対し,後者では所在の意味か所有の意味かで使われる動詞が異なる(「在/有」)など,大きな違いがある.日本と中国の言語学の分野では,存在文について論述があるが(飯田隆 2001,西山佑司 2003,金水敏 2006,儲澤祥 1997),日中機械翻訳の角度からの研究は殆ど見あたらない.また日中機械翻訳において現在の日中市販翻訳ソフトでは,存在文に関する誤りが多く見られる.

本論文は言語学の側の文献を参考にしながら存在文に関する日中機械翻訳の方法について考察し,翻訳規則の提案,翻訳実験を行ったものである.

\begin{itemize}
\item[(1)]
日中両言語における存在表現を対照的に分析し,異同を起こす原因に関しても検討を試みた.

\item[(2)]
中国語の存在動詞の選択とその位置の問題を中心に,機械翻訳における存在文の翻訳規則を提案した.

\item[(3)]
提案した翻訳規則を,手作業で及び我々が開発している翻訳システムで翻訳実験を行い,評価した.

\item[(4)]
関連する問題点と今後の課題について議論した.
\end{itemize}



\section{日本語と中国語における存在表現}


\subsection{日本語の存在文}

存在文は日本語の基本文型の一つであり,広くは動詞文に含まれる.言語学の文献で通常あげられる存在文はもっぱら場所存在文が中心である.

場所存在文は,(1)(2)に例示されるようなもっとも標準的なタイプの存在文であり,LSV(L:何処何処に,S:何々が,V:ある/いる)という基本語順で,空間的場所における対象の有無を表す.

\inHRei{(1)}
机の上にペンがある.\footnote{
	本論文における日本語例文は日中.中日辞典,日英対訳コーパスから採集した,また内省によって創作した文もある.}

\inHRei{(2)}
公園に男の子がいる.

本論文では,もっと広く「ある」,「いる」,「ない」,「存在する」と「だ」によって対象の存在を表現する文を存在文としてとりあげ,日中機械翻訳における翻訳処理について検討する.

このうち「ある」と「いる」による動詞文を存在詞文という.これによる存在文は,日本語において他の言語(例えば,中国語,韓国語など)と異なった特徴を有している.存在詞文は存在主の意味的性質により,二種に大別される(小池清治 2002).存在主が非情物(無生物や現象など)の場合は,存在の意を「ある」で,非存在の意を「ない」で表す.一方,存在主が有情物(生物や人間など)の場合は,存在の意を「いる」で,非存在の意を「いない」で表す.

「存在する」による存在文は,(3) に例示されるような実在文である.

\inHRei{(3)}
ペガサスは存在しない.

「だ」による存在文は,(4) ,(5) に例示されるような所在コピュラ文である.

\inHRei{(4)}
お母さんは,居間だ.

\inHRei{(5)}
慧子ちゃんのカバンは,車のなかだ.


\subsection{中国語の存現文}

中国語では,存在を表現する文は存現文として分類される.存現文は中国語の基本文型の一つであり,意味的には,ある場所に何物かが存在していることを表す文,或いはある場所ひいてはある時間に何物かが出現または消失したことを表す文である(劉月華 1996).

前者は人,事物の存在を表す文であり,中国語では「存在句」と呼ばれる.

厳格に言えば,中国語の存在句は物事の「存在関係」,つまり物事の空間の位置の関係(「ある所にある物が存在する」)を表現する言語の形式であり,一般に次のような文型となる.

{場所語句}+{動詞}+{名詞(存在する事物)}

\inHRei{(6)}
\begin{簡体中文}\UTFC{684C}子上\ul{有}一本\UTFC{4E66}.\end{簡体中文}(机の上に本が一冊ある.)   ——「有」構文

\inHRei{(7)}
\begin{簡体中文}\UTFC{684C}子上\ul{是}一本\UTFC{4E66}.\end{簡体中文}(机の上は一冊の本だ.)    ——「是」構文

\inHRei{(8)}
\begin{簡体中文}\UTFC{684C}子上\ul{放着}一本\UTFC{4E66}.\end{簡体中文}(机の上に本が一冊置いてある)

つまり中国語の習慣では,この種の意味を表す際,場所語句を(``在''``从''等の介詞を用いずに)主語として文頭に置き,存在する事物を表す名詞は述語動詞の後に置く.場所語句は一般に不可欠である.

後者は人,事物の出現或いは消失を表す文であり,これは中国語では「隠現句」と呼ばれ,一般に次のような文型となる.時間語句と場所語句は省略できる.

{時間語句}+{場所語句}+{動詞}+{名詞(出現或いは消失事物)}

\inHRei{(9)}
\begin{簡体中文}昨天\ul{\mbox{\UTFC{53D1}生}}了一件大事.\end{簡体中文}\\
昨日大きな事件があった.


\subsection{日本語の存在文と中国語の存在文の対応の多様性}

中国語には,日本語の存在動詞「ある/いる/だ/存在する」に直接的に対応する品詞は無く,動詞,副詞,連詞など様々の品詞を用いた表現に翻訳することになる.

\noindent
[i]日本語側では同じ1つの存在動詞を用いて表現するが,中国語では,意味が違えば異なる品詞に対応する場合もある.

\inHRei{(13)}
箱に,危険\ul{とある}.\hfill (書記結果存在の意味)\\
訳文:\begin{簡体中文}箱子上,\ul{写着}``危\UTFC{9669}''.\end{簡体中文}\hfill (動詞)

\inHRei{(14)}
彼は外国で生活した\ul{ことがある}.\hfill (経験の存在の意味)\\
訳文:\begin{簡体中文}他\ul{曾\mbox{\UTFC{7ECF}}}在国外生活\UTFC{8FC7}.\end{簡体中文}\hfill (副詞)

\inHRei{(15)}
群衆の中にいても寂しく感じる\ul{こともある}.\hfill (生起の意味)\\
訳文:\begin{簡体中文}\ul{有\mbox{\UTFC{65F6}}}在人群中也会感到寂寞\end{簡体中文}\hfill (副詞)

\inHRei{(16)}
最近は,教科書以外の本は一冊も読まない学生が\ul{いる}.\hfill (限量的存在の意味)\\
訳文:\begin{簡体中文}最近,\ul{有的}学生除了教科\UTFC{4E66}一本\UTFC{4E66}也不\UTFC{8BFB}.\end{簡体中文}\hfill (指示代詞)

また一見,同じような意味にみえる文でも異なる訳語で表す場合もある.

\inHRei{(17)}
机の上に本が\ul{ある}.\hfill (存在の意味)\\
訳文:\begin{簡体中文}\UTFC{684C}子上\ul{有}\UTFC{4E66}.\end{簡体中文}\hfill (動詞)\\
英語:There is a book on the desk.

これらを次のように言い換えると,ナニカよりドコカが焦点となって,存在というよりも,もののありか,所在を示す文になる.中国の訳語も変わる.

\inHRei{(18)}
本は机の上にある.\hfill (存在の意味)\\
訳文:\begin{簡体中文}\UTFC{4E66}\ul{在}\UTFC{684C}子上.\end{簡体中文}\hfill (動詞)\\
英語:The book is on the desk.

\noindent
[ii]逆に日本語側で同じ存在動詞で異なる意味をしていても,中国語でもそれらを同じ訳語で対応できる場合がある.

\inHRei{(19)}
ここから目黒へ行く間にとても静かな自然教育園が\ul{あります}.\hfill (存在の意味)\\
訳文:\begin{簡体中文}从\UTFC{8FD9}里到\UTFC{9ED1}目之\UTFC{95F4}\ul{有}个很静的自然教育\UTFC{56ED}.\end{簡体中文}\hfill (動詞:有)

\inHRei{(20)}
恵美は音楽の才能が\ul{ある}.\hfill (所有の意味)\\
訳文:\begin{簡体中文}惠美\ul{有}音\UTFC{4E50}才能.\end{簡体中文}\hfill (動詞:有)

このように,日本語の存在文の意味用法は非常に多様であり,さらに中国語との意味的,位置的な対応も複雑多岐である.現在の市販ソフトの存在文に関する誤訳は,これらのことの十分な分析がなされていないのが原因だと考えられる.



\section{存在表現の中国語への機械翻訳}

2節の分析(日本語と中国語の対応多様性)によって,日中機械翻訳において,日本語存在文と中国語の対応関係の分析と存在動詞の意味分類が必要であることが分かった.我々は日中機械翻訳システムを開発中であるが,この翻訳システムに組み込むことを想定して,存在文の翻訳規則について分析した.

分析と評価の資料として,日英対訳コーパス(村上仁一 2002)中の日本語例文を用いた.その4万文から「ある」を含む1853文,「いる」を含む659例文と「ない」を含む1434文を抽出した(4万文中には存在文が約4千文,10\%含まれていた.)

\begin{itemize}
\item[\UTF{2460}]
分析資料として,「ある」を含む500文,「いる」を含む300文と「ない」を含む300文を調べた.

\item[\UTF{2461}]
分析した資料の中の「ある」を含む72文と「いる」を含む18文を機械翻訳システムでのクローズドテストの対象として抽出し,翻訳実験した.

\item[\UTF{2462}]
分析対象としなかった文の中から,オープンテストの対象として「ある」を含む300文,「いる」を含む200文,「ない」を含む200文を抽出し,手作業で評価した.またそれらとは別の「ある」を含む40文,「いる」を含む20文,「ない」を含む40文を機械翻訳システムでのオープンテストの対象として抽出し評価した.
\end{itemize}

以下本節では存在動詞の意味分類に基づいて,機械用の存在文の翻訳規則を提案する.その評価(翻訳実験)については4節で述べる.




\subsection{「ある」の翻訳規則}

「ある」を中国語に翻訳する場合,「所在」の意味か「所有」の意味かに焦点に置くことによって,基本的に「\begin{簡体中文}有\end{簡体中文}」か「\begin{簡体中文}在\end{簡体中文}」が対応するが,ほかに場合によっては「\begin{簡体中文}在于,\UTFC{53D1}生,\UTFC{8FDB}行\end{簡体中文}」などの別の動詞が対応することもある.「ある」の意味分類を中国語の訳語との対応関係を考慮に入れて,また文献(西山佑司 2003,金水敏 2006,劉月華 1996,金田一春彦 1988,池原悟他 1997)などを参考にして表1にまとめた.

\begin{table}[t]
\caption{「ある」の意味分類}
\input{03t01.txt}
\end{table}


以上の分析に基づき,日本語文の構文特徴(文型,助詞,テンスなど)と存在主と存在場所の意味属性を用いて,また中国語の存在動詞の組み合わせの制限を総合して分析し,判定条件(翻訳規則)を機械で処理可能な形に整理した(表2).


以降の表中のN1,N2,N3は名詞を,Xは句などを,Pは結び部分(述部)を表す.意味属性にはNTT日本語語彙体大系(池原悟他 1997)の意味属性を用いて翻訳パターンを記述している.以下の規則は,複数のパターンにマッチした場合には最適解の選択が行われる.最適解としては以下の3種の条件を総合し,コストが最も低いパターンを選択する.選択の原則は「一般的な規則より個別的,具体的な規則を優先するべきである」という考えである.具体的に以下の3種の条件で判定する(詳しい説明は付録1として記述した):

\UTF{2460} 木構造の構成に使用されているパターンの種類と数

\UTF{2461} 制約条件の種類

\UTF{2462} 意味属性間の距離


(表2など本論文の規則で,機能語条件に「が」と書いてある場合は,「が」と「は」のいずれも,「は」と書いている場合は,「は」しか適合しない.)

以上の35個の日本語パターンのうち,(11) と(12) の中には,状態名詞存在主文(「熱,金,暇」など,状態を表す名詞が存在主になるもの),意思素質名詞存在主文(「やる気,勇気,才能」など,意思や素質に関する名詞が存在主になるもの)と出現物名詞存在主文(「効果,疑問,責任,」など,作用や行為の結果出現したことが存在主になるもの)が多い(グループ・ジャマ\par

\input{03t02.txt}
\noindent
シイ 2001).この中には語彙の組み合わせの制約条件として,個々の語彙そのものを制約条件とする(字面照合)のが適切なパターンがある(現在のところ38個,付録2参照).例えば

\inHRei{(1)}
(N1:主体/精神/状態)には落ち着きがある\\
→N1 \begin{簡体中文}\UTFC{6C89}着/\UTFC{9547}定\end{簡体中文}

\inHRei{(2)}
(N1:人/抽象物/性質)には表裏がある\\
→N1 \begin{簡体中文}表里不一\end{簡体中文}

これらを含めると現在設定している「ある」の翻訳パターンは全体で73個である.



\subsection{「いる」の翻訳規則}

「いる」は中国語に翻訳すると,意味と文法上の規定により,「在,有,\UTFC{5904}于…」などの可能性がある.「いる」の意味分類を中国語の訳語との対応関係を考慮に入れて,また文献(西山佑司 2003,金水敏 2006,劉月華 1996)などを参考にして表3にまとめた.「ある」の場合と同様に,「いる」の翻訳規則を表4のようにまとめた.

\begin{table}[t]
\caption{「いる」の意味分類}
\input{03t03.txt}
\end{table}



\subsection{「存在する」と「だ」の翻訳規則}

「ある」と「いる」の場合と同様に,所在コピュラ文「だ」と実在文「存在する」の翻訳規則を表5にまとめた.

\begin{table}[t]
\caption{「いる」の翻訳規則}
\input{03t04.txt}
\end{table}


\subsection{存在否定文の翻訳処理}

「今日は風がある.」「今日は風がない.」「あそこに猫がいる.」「あそこに猫がいない.」のように,述語が「ある」の否定「ない」や「いる」の否定「いない」等で構成される文を存在否定文という.丁寧体では,「ありません/いません」の形が用いられる.

「いる」の否定「いない」の存在否定文の翻訳処理は通常の否定の処理(「いる」の翻訳処理プラス否定の翻訳処理)でできる(ト朝暉 2004).しかし,「ない」は単純に「ある」の否定としては翻訳できない場合が多い.「ない」そのものは意義をもつ単語であり,「ある」の意味分類の外の意味も持っている場合がある.例えば,「類がない」,「またとない」は「とてもすばらしい」の意味であって,「今さら泣くことはない」の「ことはない」は「不必要なこと,あってはならないこと」の意味である.これらは「ある」の否定として翻訳することはできない表現である.そこで,我々は「ない」を存在詞の一種として,翻訳規則を整理した.

まず,「ある」の場合と同様に,状態名詞存在主文,意思素質名詞存在主文と出現物名詞存在主文に対して,語彙の組み合わせの制約条件にして,個々の語彙そのものを制約条件とする(字面照合)のが適切なパターンがある.「ある」と「ない」非対称型の字面照合パターンの整理を行って,翻訳パターンにまとめた.例えば:

\inHRei{(1)}
N1は表裏がない  →  (N1\UTFC{5355}\UTFC{7EAF})

\inHRei{(2)}
N1は身長がない  →  (N1\begin{簡体中文}不高\end{簡体中文})

\inHRei{(3)}
N1は無理がない  →  (N1\begin{簡体中文}是自然/理所当然的\end{簡体中文})

次に,非存在を表す形式しかない成句と熟語を整理し,翻訳パターンにまとめた.例えば

\inHRei{(1)}
しかたない  →  (\begin{簡体中文}没\UTFC{529E}法\end{簡体中文})

\inHRei{(2)}
申し訳ない  →  (\begin{簡体中文}\UTFC{5BF9}不起\end{簡体中文})

付録3に「ない」に関するこれらの字面パターン(合計38)を示した.それら以外の「ない」の用法と慣用文型の翻訳規則を表6のようにまとめた.

\begin{table}[t]
\caption{「存在する」と「だ」の翻訳規則}
\input{03t05.txt}
\end{table}
\begin{table}[t]
\caption{「ない」の翻訳規則}
\input{03t06.txt}
\end{table}


以上述べた場合の外の「ない」は,「ある」と対応できると考えて,存在文「ある」の翻訳プラス否定の翻訳という方法で翻訳する.



\subsection{存在文の連体修飾の翻訳について}

日中翻訳の場合には,連体修飾表現の一般的な日中翻訳規則は

  日本語:\ul{\mbox{V(用言連体形)}}+N

  中国語:\ul{V}+\begin{簡体中文}的\end{簡体中文}+N

\noindent
である.(Nが形式名詞(「の」など)ではない場合)

  例文:\ul{\mbox{80歳の老人が山登りをする}} 様子を想像できますか.

  訳語:\begin{簡体中文}不能想象 \ul{\mbox{80\UTFC{5C81}}老人登山} 的 \UTFC{6837}子.\end{簡体中文}

しかし,存在文の例文の翻訳を分析する中で,この規則では対応できない事例を見出した.存在文が連体修飾節を伴う場合であって,限量的存在文の場合と修飾部が「能力,可能,方法,理由,時間」などの場合である.これらは中国語の兼語文或いは連動文に翻訳させるのが正しい.



\subsubsection{限量的存在文の翻訳について}

限量的存在文は特定の集合における要素の有無多少について述べる表現である.話し手の立場から下す,世界についての判断の一種であるということもできる.限量的存在文の存在動詞は述語というよりは,記号論理における存在限量詞の働きをすると考えられる.限量的存在文には,場所名詞句は必ずしも要しない.場所名詞句は,基本的には主語より前に置かれる(金水敏 2006).

例文:最近は,教科書以外の本は一冊も読まない学生がいる.

誤訳1:\begin{簡体中文}最近,有 \ul{\mbox{除了教科\UTFC{4E66}一本\UTFC{4E66}也不\UTFC{8BFB}(V)}}的 \ul{\mbox{学生(N)}}.\end{簡体中文}(一般的な連体修飾規則での翻訳)

正訳2:\begin{簡体中文}最近,有的 \ul{\mbox{学生(N)}} \ul{除了教科\mbox{\UTFC{4E66}}一本\mbox{\UTFC{4E66}}也不\mbox{\UTFC{8BFB}}}(V).\end{簡体中文}

限量的存在文について調べた結果,「いる」,「ある」と「ない」の中では,「ある」と「ない」については限量的存在文の割合が少ないが,「いる」の例文の中に限量的存在文の割合が高いころが分かった.分析資料のからに「ある」,「いる」,「ない」100文ずつを無作為抽出して,調査した結果を表7に示す.「いる」の中では,連体修飾の限量的存在文は約45\%であった.

\begin{table}[b]
\caption{限量的存在文の統計結果}
\input{03t07.txt}
\end{table}


限量的存在文の場合には,一般的な連体修飾規則では,次の例1,2のように,不自然な中国語となる.

\noindent
(1) パターン:V(用言連体形)+N(人/動物)がいる.

  現訳:\begin{簡体中文}有+V的+N\end{簡体中文}\hfill (一般的な連体修飾規則での翻訳)

  正訳:\begin{簡体中文}有N+V\end{簡体中文}

  例文:授業中わたしの後ろで私語している人が2,3人いた.

  現訳:\begin{簡体中文}有 2,3个 \ul{\mbox{上\UTFC{8BFE}中在我后面\UTFC{8BF4}悄悄\UTFC{8BDD}(V)}} 的 \ul{\mbox{人(N)}}.\end{簡体中文}

  正訳:\begin{簡体中文}有 2,3个 \ul{\mbox{人(N)}} \ul{\mbox{上\UTFC{8BFE}在我后面\UTFC{8BF4}\UTFC{8BDD}(V)}}.\end{簡体中文}

\begin{table}[b]
\caption{連体修飾の限量的存在文の翻訳規則}
\input{03t08.txt}
\end{table}

\noindent
(2) パターン:N(人/動物)の中には+V(用言連体形)+ものがいる.

  現訳:\begin{簡体中文}N中有+V的\end{簡体中文}\hfill (一般的な連体修飾規則での翻訳)

  正訳:\begin{簡体中文}有的N+V\end{簡体中文}

  例文:蛇の中には毒を持つものがある.

  現訳:\begin{簡体中文}蛇中有有毒的.\end{簡体中文}\hfill (一般的な連体修飾規則での翻訳)

  正訳:\begin{簡体中文}有的蛇有毒.\end{簡体中文}

以上のように,限量的存在文の場合には,日本語の連体修飾文は中国語の兼語文に翻訳するのが自然である.「\begin{簡体中文}有\end{簡体中文}+N+V」の形は中国語の兼語文であり,「\begin{簡体中文}有\end{簡体中文}」の目的語N(これを兼語という)の多くは存在する人,事物を表し,述語Vは兼語Nを説明或いは描写する.限量的存在文の翻訳規則を表8のように整理した.




\subsubsection{修飾部が「能力,可能,方法,理由,時間など」の場合について}

「V(用言連体形)+Nがある」というパターンにおいて,Nが能力,可能,方法,理由,時間,責任,条件,自信等の場合には,このような抽象名詞は「\begin{簡体中文}有\end{簡体中文}」とあわさって「\begin{簡体中文}能\end{簡体中文}」「\UTFC{5E94}\UTFC{8BE5}」「\begin{簡体中文}想\end{簡体中文}」の類の意味を表すものであり,「\begin{簡体中文}有\end{簡体中文}+N+V」のように連動文に翻訳するのが正しい.

  パターン:V(用言連体形)+Nが+ある.

  現訳:\begin{簡体中文}有+V+的+N\end{簡体中文}\hfill (一般的な連体修飾規則での訳語)

  正訳:\begin{簡体中文}有+N+V\end{簡体中文}

例えば,

  例文:あなたに相談したいことがある.

  現訳:\begin{簡体中文}有想和\UTFC{4F60}\UTFC{8C08}的事情.\end{簡体中文}

  正訳:\begin{簡体中文}有事情想和\UTFC{4F60}\UTFC{8C08}.\end{簡体中文}\hfill (連動文)

二つまたはそれ以上の動詞或いは動詞フレーズが連用されたものが述語になっている文を連動文という.連動文では,連用されている複数の動詞或いは動詞フレーズは主語を共にする(劉月華 1996).

修飾部が「能力,可能,方法,理由,時間など」の場合の翻訳規則を表9のように整理した.

\begin{table}[b]
\caption{能力などの存在文の翻訳規則}
\input{03t09.txt}
\end{table}




\subsection{英語への翻訳規則との比較}

日本語語彙大系(池原悟他 1997)は日英翻訳のための翻訳パターン辞書である.存在動詞「ある/いる/ない」に関して日本語語彙大系の日英翻訳パターンと我々の日中翻訳パターンとをパターン数の点から比較分析した.日英翻訳パターン数と日中翻訳パターン数の対比を表10に示す.

\begin{table}[b]
\caption{翻訳パターンと日中翻訳パターンの対比}
\input{03t10.txt}
\end{table}


表10を見ると,日英の「ある」と「ない」のパターンは日中翻訳のよりかなり多いことがわかる.この差異については以下のように分析できる.

\inHZ
1.一つの日中翻訳パターンに多数の日英翻訳のパターンが対応する.

主な原因は次の2点にある:

\inHZ
\UTF{2460}英語では形容詞で物の性質,素質と状況を説明するが,日本語では存在文(状態名詞存在主文,意味素質名詞存在主文,出現物名詞存在主文)で表現することが多い.そのために日英の翻訳では個々の形容詞を対応させるためにパターン数が増える.中国語では,日本語とほぼ同様に存在文で表現することが多いので,パターン数は少なくてすむ.

\begin{tabular}{rl>{\begin{簡体中文}}l<{\end{簡体中文}}}
例(1)& N1は暇がある.& \\
     & N1 have time  & N1\ul{有}\UTFC{65F6}\UTFC{95F4} \\
 (2) & N1は勇気がある& \\
     & N1 be courageous & N1\ul{有}勇气 \\
 (3) & N1は意義がある & \\
     & N1 be significant & N1\ul{有}意\UTFC{4E49} \\
 (4) & N1は教養がある & \\
     & N1 be educated & N1\ul{有}教\UTFC{517B} \\
 (5) & N1は価値がある & \\
     & N1 be valuable & N1\ul{有}价\UTFC{503C} \\
 (6) & N1は\ul{背長がある} & \\
     & N1 be tall & N1\ul{个高} \\
 (7) & N1は\ul{含蓄がある} & \\
     & N1 be pregnant & N1\ul{意味深\mbox{\UTFC{957F}}}
\end{tabular}

\inHZ
\UTF{2461}英語では中国語と比べると存在主に対して前置詞(介詞)を細かく区別して使い分ける.

\inHRei{(1)}
N1はN2に効果がある\\
N1 be effective \ul{in} N2\\
\begin{簡体中文}N1\ul{\mbox{\UTFC{5BF9}}}N2有效果\end{簡体中文}

\inHRei{(2)}
N1はN2に同情心がある\\
N1 be sympathetic \ul{to} N2\\
\begin{簡体中文}N1\ul{\mbox{\UTFC{5BF9}}}N2有同情心\end{簡体中文}

\inHRei{(3)}
N1はN2に責任がある\\
N1 be responsible \ul{for} N2\\
\begin{簡体中文}N1\ul{\mbox{\UTFC{5BF9}}}N2有\UTFC{8D23}任\end{簡体中文}

\inHRei{(4)}
N1はN2に偏見がある\\
N1 have prejudice \ul{\mbox{against}} N2\\
\begin{簡体中文}N1\ul{\mbox{\UTFC{5BF9}}}N2有偏\UTFC{89C1}\end{簡体中文}

上の例では,日本語パターン「N1はN2にN3がある」という1つのパターンに対して,英語では,すべて前置詞を使い分けたパターンとなるが,中国語では,「N1\UTFC{5BF9}N2有N3」のように日本語同じような1つのパターンで表現できる.

\inHZ
2.日英翻訳のパターン(41個)に対応する日中翻訳のパターンは不必要である.

例えば,日本語パターン「N1はN2のN3がある」の形の日英パターンが39個あるが,日中の翻訳パターンは「N1はN3がある」と「N2のN3」の二つのパターンで翻訳できるので,「N1はN2のN3がある」の日中パターンは不必要である.

\begin{tabular}{llll}
(1) & 日本語:N1はN2の気品がある & & \\
    & 英 語:N1 be dignified as N2 & & \\
    & 中国語:\begin{簡体中文}N1\ul{有}N2的品格\end{簡体中文} & & \\
    &(N1はN2のN3がある & & N1はN2の気品がある\\
    & N1はN3がある+N2のN3 & → & \begin{簡体中文}N1有品格+N2的品格\end{簡体中文} \\
    & \begin{簡体中文}N1有N3+N2的N3\end{簡体中文} & & \begin{簡体中文}N1\ul{有}N2的品格\end{簡体中文})\\
(2) & 日本語:N1はN2の傾向がある & & \\
    & 英 語:N1 have a tendency to N2 & & \\
    & 中国語:\begin{簡体中文}N1\ul{有}N2的\UTFC{503E}向\end{簡体中文} & & \\
    & (N1はN2のN3がある & & N1はN2の傾向がある\\
    & N1はN3がある+N2のN3 & → & \begin{簡体中文}N1有\UTFC{503E}向+N2的\UTFC{503E}向\end{簡体中文}\\
    & \begin{簡体中文}N1有N3+N2的N3\end{簡体中文} & & \begin{簡体中文}N1\ul{有}N2的\UTFC{503E}向\end{簡体中文})
\end{tabular}

\inHZ
3.あるパターンは日中翻訳パターンにはあるが,日英翻訳のパターンにない.この場合には日英翻訳パターンにおける文型の整理が完全ではないのが原因である.例えば,

  日本語:N1はN2に恩義がある

  英 語:N1 be indebted to/for N2

  中国語:\begin{簡体中文}N1欠N2的恩情\end{簡体中文}

なお,日本語語彙体系(池原悟他 1997)では,「だ」に関しては,日英の存在表現の「だ」のパターンを記述していないので,比較はできなかった.



\section{翻訳実験と評価}

3節で述べた翻訳規則を検証し,評価するために,手作業による翻訳実験を行った.また実際に機械翻訳システムに実装できる翻訳規則であることの検証の意味も含めて,我々が開発中の日中機械翻訳システムによる翻訳実験も行った.最後に,誤った翻訳例について誤訳の原因を分析した.



\subsection{手作業による翻訳実験と評価}

手作業による翻訳規則評価実験では3節で述べたように日英対訳コーパス(村上仁一 2002)の中で規則作成のための分析に用いた例文を除いた中から「ある」を含む300文,「いる」を含む200文,「ない」を含む200文を抽出して翻訳実験の対象データとした.

評価は,存在表現の翻訳に注目して,その訳語と語順と助詞の翻訳が合っているかどうかという観点から個人判断で評価を行った.評価の例を表11に示している(存在表現の部分に関する評価箇所に下線を引いている).同時にある市販機械翻訳ソフトを用いて翻訳し,その結果も評価した(表12).



評価は次の3種類で行った.

○=中国語として文法上で基本的に正確であり,自然な翻訳

△=文法上では少し不自然な感じるが文全体の意味としては通じる場合

×=文法上では明確に中国語の文法に反し,意味も通じないか不完全な場合

評価する時には,○は正訳として,△と×は誤訳として,及び○と△は正訳として,×は誤訳として二つの評価基準で統計した.

\begin{table}[p]
\caption{手作業での翻訳例文と評価の一部}
\input{03t11.txt}
\par\vspace{2\baselineskip}
\caption{翻訳結果(手作業:オープンテスト)}
\input{03t12.txt}
\end{table}


評価の際,我々の規則の評価については存在表現の部分以外の翻訳は全部正しく翻訳されたものとして,存在表現の翻訳に関する我々の規則を適用した結果のみを評価しており,一方市販ソフトに対する評価では,存在表現の部分以外の翻訳の正否は無視して,存在部分の正誤のみに注目して評価している.

手作業の場合には,機械翻訳の流れをまねて,評価対象文に対して,存在動詞に係る文節を抽出し,翻訳規則と照合し,翻訳結果が正しいかどうか判断した.以下に手作業による翻訳の手順の例を示す.

文:この\ul{金貨}は古銭を集めている\ul{人}に\ul{とても価値}が\ul{ある}.

step 1 構文解析して\footnote{
	手作業の例文は我々の研究室で開発している構文解析システムIBUKIで自動解析して使用した.存在表現の部分について解析誤りはほとんどない.}
,下記の存在動詞に係る各情報を得る.

存在動詞=「ある」.

「金貨」の意味属性は「貨幣」で,「は」が付いている.

「人」の意味属性は「人間」で,「に」が付いている.

「価値」の意味属性は「是非」で,「が」が付いている.

「とても」は程度副詞であり,「ある」を修飾している.

日本語文の構造は「N1はN2にN3がある」である.

Step 2 Step 1の結果と「ある」の翻訳規則を照合する.

表1の(12) と照合できる.

(12) の翻訳規則によれば,中国語文の構造は「主語=\UTFC{8D27}\UTFC{5E01},謂語=\begin{簡体中文}有\end{簡体中文},目的語=\begin{簡体中文}价\UTFC{503C}\end{簡体中文},対象者=\begin{簡体中文}人\end{簡体中文},対象者の修飾語=\UTFC{5BF9}」である.

Step 3 「とても」の翻訳結果と基本部分の訳とをあわせて生成し,通常の文の線状化の語順によって,下記の訳文を得る.

→中国語:\begin{簡体中文}\UTFC{94F6}\UTFC{5E01}\UTFC{5BF9}人很有价\UTFC{503C}.\end{簡体中文}

表12に示したように我々の規則ではと,全体として90\%以上の正訳率が得られており,市販の翻訳ソフトの現状と比較すると,我々の正訳率は各々34\%,62\%,48\%まさっており,我々の方法は十分な有効性が期待できると考えている.

誤り分析については4.3で述べる.



\subsection{jaw/Chineseによる機械翻訳実験}

4.1では手作業による翻訳規則の評価(翻訳実験)について述べたが,これらの翻訳規則が実際に機械翻訳システムのための規則として実装でき有効であることを検証する意味を含めて,開発中の日中機械翻訳システムJaw/Chineseによる翻訳実験を行った.



\subsubsection{翻訳規則の登録}

Jaw/Chineseシステムは機械翻訳エンジンJawを用いた日中機械翻訳システムである(宇野修一 2005,謝軍 2004).日本語文の解析,表現パターン変換辞書との照合による中国語の表現構造への変換,表現構造からの中国語の生成という手順によるパターン変換ルールベース・トランスファー方式のシステムである.

存在文に関する訳し分けの実験を行うために,表2,4,5,6の翻訳規則をJaw/Chineseに登録した.

Jaw/Chineseでは,翻訳規則は,各表現パターンごとにパターン変換辞書に記述されており,それを基にして翻訳規則関数を作成している.翻訳規則の記述方法は表現パターンの種類によってBase Type,AdditionCW Type,AdditionFW Typeと用言後接機能語に分かれている.存在表現の規則は次の三つのタイプの翻訳規則で記述することができる.表現パターンとは,日本語の表現の中から文の部分構造を取り出し,そのいくつかの部分を変数化して抽象化したものである.各表現パターンは,キーワードと呼ぶ単語を必ずひとつだけ持っている.入力文とパターンのデータベースを照合する際には,入力文中の単語をキーワードとして含むパターンの集合をまず抽出し,その後,その中の各パターンが要求する条件と入力文との詳細な照合検査が行われる.図1はキーワードの検索と詳細照合を示している.

\begin{figure}[t]
\begin{center}
\includegraphics{14-5ia3f0.eps}
\end{center}
 \caption{キーワードの検索と詳細照合}
\end{figure}
\setcounter{figure}{0}


日本語表現中のどの部分をキーワードとするかで,三つのタイプが区別される.

\noindent
\UTF{2460}Baseタイプ

Baseタイプはキーワードが内容語であり,キーワードを含む文節が受ける文節に条件を持つパターンである.ここでは,内容語である存在動詞「ある」,「いる」,「ない」,「存在する」,「だ」をキーワードとして,それを修飾するN1,N2,N3,X等に条件をつける翻訳規則である.例えば,

\begin{tabular}{lllll}
日本語パターン:& N1が & N2に/に対して & N3が & ある\\
	& N1(主体)& N2(体言)& N3(抽象)& \\
文節番号:&1 & 2 & 3 
\end{tabular}

\begin{tabular}{lccll}
例文:& \ul{喫煙は} & \ul{健康に} & &  \parbox{11zw}{\ul{悪影響が}あるといわれている.}\\
文節番号:& 1 & 2 & &  3 \\
訳文:\begin{簡体中文}一般\UTFC{8BA4}\UTFC{4E3A}\end{簡体中文} & \begin{簡体中文}吸烟\end{簡体中文} & \begin{簡体中文}\UTFC{5BF9}健康\end{簡体中文} & \begin{簡体中文}有\end{簡体中文} & \begin{簡体中文}不良影\UTFC{54CD}.\end{簡体中文}\\
 & (m\_subject) & (m\_nounModifier) & (m\_centerW) & (m\_directobject)
\end{tabular}

表13-(2) は,表13-(1) の日本語パターンに対応する翻訳規則である.Jawの翻訳規則は日本語の表現パターンに対応する目的言語の表現構造を作り出すプログラム(翻訳規則関数)に変換される.表現構造はC++オブジェクトであり,そのメンバ変数には目的言語の言語表現を生成するために必要なさまざまの言語構成上の情報が翻訳規則によって書き込まれる.目的言語の一次元の言語表現(文)は,各クラスに定義してあるメンバ関数(線状化関数)を実行することによって生成される.

\begin{table}[t]
\caption{Base Type翻訳規則記述の例}
\input{03t13.txt}
\end{table}


表13-(2) は,表13-(1) に対する中国語の表現構造がクラスCPropositionに属するオブジェクトであり,そのオブジェクトのいくつかのメンバ変数を設定していることを表わしている.線状化関数によって,各文節の順番は決める.具体的に言うと,通常の文の線状化の語順は以下のとおりです:

{\setlength{\leftskip}{2zw}
\noindent
文頭に来る接続詞 → 範囲を示す修飾語 → 主題 → \ul{主語} → 文中に来る接続語\\
→ 評注性副詞 → 関連副詞 → 時間 → 時間副詞 → 頻度副詞 → 場所(述語より前に置かれる場合) → 範囲副詞 → 否定副詞 → 助動詞 → 時間と空間の始点を表す修飾語 → 道具を表す修飾語 → 共同副詞 → \ul{対象者を示す修飾語} → 方向を示す修飾語 → 程度副詞 →重複副詞 → 重複副詞 → 描写性副詞\\
→ 副動詞 → \ul{述語} → 結果補語 → 可能補語 → アスペクト → 間接修飾語\\
→ \ul{直接目的語} → 方向補語 → 程度補語 → 時間と空間の終点を表す修飾語 → 場所(述語より後ろに置かれる場合) → 語気詞 → 句読点\par}

表13-(2) の場合では,「有」は述語であり,1は主語であり,3は直接目的語であり,2は対象者を示す修飾語である.だから,最終の順番は「1 2 有 3」として生成される.

\noindent
\UTF{2461}Addition-FWタイプ

Addition-FWタイプはキーワードが機能語であり,存在動詞はキーワードと同じ文節の自立語として,キーワードを含む文節が係る文節に条件を持つパターンである.例えば,

{\setlength{\tabcolsep}{0pt}
\begin{tabular}{lcl}
日本語パターン:& \ul{X}と & あって(ある/て),P(文)\\
 & X(用言)& \\
文節番号:& 1 \\
例文:&\ul{大型の台風が接近している}&とあって,どの家も対策におおわらわだ\\
文節番号:&1\\
訳文:&\multicolumn{2}{c}{\begin{簡体中文}因\UTFC{4E3A}\UTFC{8BF4}是 \ul{\mbox{大型台\UTFC{98CE}}靠近},所以不管\UTFC{54EA}家都\UTFC{4E3A}了\UTFC{5BF9}策忙得不可\UTFC{5F00}交.\end{簡体中文}}
\end{tabular}
}


表14-(2) は,表14-(1) の日本語パターンに対応する翻訳規則である.この翻訳規則は接続詞の訳語と接続詞の位置を表している.

\begin{table}[b]
\caption{Addition-FW Type翻訳規則記述の例}
\input{03t14.txt}
\end{table}


ここでは,「Xとあって」は「(Xと)+(ある+て)」と解析し,「て」をキーワードとするパターンとして登録しているが,「とあって」を一つ機能語と解析して,「とあって」をキーワードとするパターンとして翻訳規則を書くこともできる.

\noindent
\UTF{2462}用言後接機能語の翻訳規則

用言後接機能語には,\UTF{2461}で述べたような命題的内容の表現に関わる機能語のほかに,受身や使役,テンスやモダリティ,否定などのさまざまの機能的内容の表現に関わる機能語がある.Jawではこれらに関する翻訳規則は,それらを翻訳するために必要な訳し分けの条件や,目標言語の表現構造上に書き込む表現要素が翻訳規則表に記述してある.この規則表は表現構造に情報を書き込むプログラムに変換される.慣用句的に使われる存在動詞の翻訳規則には日本語用言後接機能語の翻訳規則として,それに対応する中国語の表現要素を用言後接機能語部の翻訳テーブルに記述しているものもある.表15は「〜ことがある」「〜たことがある」に対する翻訳規則の例である.

\begin{table}[b]
\vspace{-0.5\baselineskip}
\caption{用言後接機能語の翻訳規則の記述例}
\input{03t15.txt}
\end{table}
\begin{table}[b]
\caption{同形語の翻訳規則}
\input{03t16.txt}
\end{table}




例:xことがある;  x:動詞連体形現在式

例文1:群衆の中にいても寂しく感じる\ul{ことがある}

例:xことがある  x:動詞連体形過去式

例文2:彼は外国で生活した\ul{ことがある}.

また存在動詞「ある」と「ある」の同形語「或る」,「いる」と「いる」の同形語「要る」,「入る」を区別するために,同形語「ある(或る),」「いる(要る)」と「いる(入る)」の翻訳規則もまとめた(表16).

これは現在の日中市販翻訳ソフトで,存在文に関する誤りの中に存在動詞と同じ形式の単語とを混同することが観察されだので,それとの比較評価も行う必要があったからである.



\subsubsection{機械翻訳実験}

3節で述べた存在表現の翻訳規則を前節で述べたJaw/Chineseに組み込み,また前節で述べた同形語の翻訳規則についても登録して,機械翻訳実験を行った.

実験対象としたデータは3節でも述べたように日英対訳コーパス(村上仁一 2002)の中で存在表現の分析対象とした「ある」を含む72文と「いる」を含む18文(クローズドテスト),及び分析対象とはしなかった存在文100文(オープンテスト)である.

これらについては,存在表現に関する部分だけでなく,文全体を翻訳するための翻訳規則を記述し,翻訳実験を行った.評価は存在表現の部分の翻訳評価と文全体の翻訳の評価を,いづれも筆者の個人判断で行った.

クローズドテストでは,存在表現の部分の翻訳に関しては全て正訳と判断できたが,文全体としての翻訳に関しては,接続表現の翻訳規則やとりたて詞,否定詞の翻訳規則が未だ完全には実装されていないことなどjaw/Chineseの不充分さのために誤訳が生じた.

市販の翻訳システムでの結果と合わせて,翻訳結果の一部と評価結果を表17に示す(存在表現の部分に関する評価箇所に下線を引いている).

\begin{table}[t]
\caption{Jaw/Chineseでの翻訳例文と評価の一部(クローズテスト)}
\input{03t17.txt}
\end{table}


この実験により,存在表現に関する我々の翻訳規則は実際の翻訳システムに実装できて有効に働くという点については検証できたと考える.

同じようにオープンテストによる翻訳結果の一部と評価結果を表18に示す(存在表現の部分に関する評価箇所に下線を引いている).また100文全体の翻訳実験について,存在表現の部分の翻訳についての評価結果を表19に示す.

\begin{table}[p]
\caption{Jaw/Chineseでの翻訳例文と評価の一部(オープンテスト)}
\input{03t18.txt}
\end{table}


翻訳規則そのものの問題点については,手作業による翻訳実験の問題と合わせて,4.3で分析する.

Jaw/Chineseでの翻訳実験の正訳率は市販の翻訳ソフトより高い結果となった.しかし,手作業の正訳率より低い.その原因は現在のJaw/Chineseの機能は以下のような点でまだ十分ではないためである.

\noindent
(1) 機能語が省略されているため,照合できない場合があった.

例えは:彼はくつを脱いで身長6フィートある.

「身長」の後の「が」が省略されると,規則と照合できないので,翻訳できなかった.手作業で判断する場合には,機能語の省略を補完しているので,正しい翻訳結果を得ることになる.

\noindent
(2) 日本語側解析の問題.

手作業の場合には日本語の解析を誤ることはないという前提となるが,機械の場合には日本語解析を誤る場合がある.

\begin{table}[t]
\caption{例文翻訳結果(Jaw/Chinese:オープンテスト)}
\input{03t19.txt}
\end{table}


\subsection{誤り分析}

4.1節および4.2節で手作業による翻訳実験と機械翻訳実験について述べた.手作業による翻訳実験の誤訳は700文中52文であり,機械翻訳実験の誤訳は190文中12文である(△を含む,合計64文).本節では,手作業と機械の実験の誤りの問題点を整理して,誤訳原因を分析する.


\subsubsection{規則の不足の問題}

(1) :突然\ul{言いようのない}恐怖感に襲われた.

現訳:\begin{簡体中文}突然,\ul{没有\mbox{\UTFC{8BF4}}法}的恐怖\UTFC{88AD}来.\end{簡体中文}(×)

正訳:\begin{簡体中文}突然,\ul{无法形容}的恐怖\UTFC{88AD}来.\end{簡体中文}(○)

分析:「言いようのない」では,「无法形容」が正解だが「没有\UTFC{8BF4}法」に訳している.「言いようのない」は慣用句であり,中国語でも「无法形容」は慣用句である,「ない」の規則に加えて「言いよう+の/が+ない」のような慣用句パターンの追加が必要である.

(2) :寒くて足の指の感じがない.

現訳:\begin{簡体中文}由于冷,没有脚趾的感\UTFC{89C9}.\end{簡体中文}(×)

正訳:\begin{簡体中文}由于冷,脚趾没有感\UTFC{89C9}.\end{簡体中文}(○)

分析:この文では,「ない」の訳語の選択(没有)は正しいが,中訳文の意味は間違っている.現訳の意味は「寒くて足の指がないと感じる.」という意味になる.「N(動物部分)の感じがある→N\begin{簡体中文}有感\UTFC{89C9}\end{簡体中文}」というパターンを追加すれば,否定処理によって,「N(動物部分)の感じがない」は「N\begin{簡体中文}没有感\UTFC{89C9}\end{簡体中文}」になり,「足の指の感じがない」は「足の指には感じがない」という正しい意味になる.このパターンの追加が必要である.

「…つもりでいる」というパターンに対する誤訳.

(3) :親は彼を医者にする\ul{つもりでいた}.

正訳:\begin{簡体中文}父母本来\ul{打算}\UTFC{8BA9}他当医生的.\end{簡体中文}

(4) :彼女はまるで小説のヒロインにでもなった\ul{つもりでいる}.

正訳:\begin{簡体中文}\UTFC{5979}\UTFC{7B80}直\ul{\mbox{\UTFC{8BA4}\UTFC{4E3A}}}(自己)成了小\UTFC{8BF4}的女主角.\end{簡体中文}

(5) :きみには来年主将になってもらうからその\ul{つもりでいてくれ}.

正訳:\begin{簡体中文}(我\UTFC{4EEC})希望\UTFC{4F60}明年成\UTFC{4E3A}主将,\UTFC{4E3A}此(\UTFC{4F60})\ul{要做好准\mbox{\UTFC{5907}}}\UTFC{554A}!\end{簡体中文}

(6) :彼は自分がチームのエースの\ul{つもりでいる}.

正訳:\begin{簡体中文}他\ul{\mbox{\UTFC{8BA4}\UTFC{4E3A}}}自己是球\UTFC{961F}的主攻投手.\end{簡体中文}

分析:元の規則には「…つもりでいる」という構造に対するパターンが無かったために翻訳できなかった.「…つもりでいる」には対応する中国語の意味がいくつかあるので分析が必要である.今のところ

\begin{tabular}{ll}
\UTF{2460}「N1(人)は+X+つもりでいる」,& X:動詞連体形現在式 \\
\UTF{2461}「N1(人)は+X+つもりでいる」 & X:動詞連体形過去式 \\
\UTF{2462}「「Nのつもりでいる」& 
\end{tabular}

\noindent
という三つの規則を増やすことで文3,文4と文6に対応できると考えている.文5では「…そのつもりでいる」という字面照合パターンで翻訳するのが適切である.

このように「ある」,「いる」と「ない」に関しては慣用的/固定的な表現が多い(誤訳64文のうち48文(75\%)はそのような表現であった).さらに多くの文例を翻訳しながら,そのようなデータを集積,整理していくことが必要である.



\subsubsection{「いる」の翻訳と副詞の問}

(7) :私は彼に1日いてくださいと頼んだ.

現訳:\begin{簡体中文}我\UTFC{8BF7}求他\ul{在}一天.\end{簡体中文}(×)

正訳:\begin{簡体中文}我\UTFC{8BF7}求他\ul{呆}一天.\end{簡体中文}(○)

(8) :私は彼にもう1日いてくださいと頼んだ.

現訳:\begin{簡体中文}我\UTFC{8BF7}求他再\ul{在}一天.\end{簡体中文}(×)

正訳:\begin{簡体中文}我\UTFC{8BF7}求他再\ul{呆}一天.\end{簡体中文}(○)

(9) :彼は10年前はボストンにいた.

現訳:\begin{簡体中文}他10年前在波士\UTFC{987F}.\end{簡体中文}(○)

誤訳:\begin{簡体中文}他10年前呆波士\UTFC{987F}.\end{簡体中文}(×)

分析:文7の「1日」は「数量詞+時間名詞」の形をとり,時間量をあらわし,中国語では,補語になって,動詞の後に置く.「いる」の訳語「在」は補語をとるのは文法的に間違いであり,「呆」に翻訳される場合は自然である.「在」と「有」は,後に各種の補語を置くことはできない.また,文8の「もう」の「再」と「いる」の「在」は中国語の同音語であって,音の異なる「呆」を選択する方が自然な翻訳である.文7と逆に,文9の「10年前」は,中国語では,状語として動詞の前に置き動詞「在」を用いて表現するのが正しく,「呆」に翻訳するのは正しくない.このように「いる」の中訳は時間副詞と関連して訳し分けが必要である.時間表現の分類,時間副詞と動詞の位置関係などさらなる考察が必要である(誤訳64文のうち4文(6.3\%)).



\subsubsection{存在主省略の翻訳問題}

(10) :ここにいすが6つあり隣の部屋にはもっとある.

現訳:\begin{簡体中文}\UTFC{8FD9}里有6把椅子,在隔壁的屋子更多.\end{簡体中文}(△)

正訳:\begin{簡体中文}\UTFC{8FD9}里有6把椅子,隔壁的屋子有更多.\end{簡体中文}(○)

分析:文10では前半の文で,存在主を主語として言及しているが,後半の主文では省略している.現在の規則では前半の文「ここにいすが6つある」の「ある」は「有」に翻訳されるが,後半の文「隣の部屋にはもっとある」の「ある」は「在」に翻訳される.後半の文だけが独立してあるのならこの訳でもよいが,この場合,主文の動詞としては前半の動詞と同じ「有」を用いるのが適切である.このように,文脈上のことを考慮に入れた翻訳処理が必要になってくる場合がある(誤訳64文のうち3文(4.8\%)).



\subsubsection{機能語によって存在動詞の訳語が変わる問題}

(11) :一人でいる.

現訳:\begin{簡体中文}一个人在.\end{簡体中文}(○)

(12) :一人でいたい気分だった.

現訳:\begin{簡体中文}想一个人在.\end{簡体中文}(×)

正訳:\begin{簡体中文}想一个人呆着.\end{簡体中文}(○)

分析:文11の場合には,「いる」は「\begin{簡体中文}在\end{簡体中文}」に翻訳される.文12の場合には,用言後接機能語「たい」が付くと,「いる」の訳語は「\begin{簡体中文}在\end{簡体中文}」から「\begin{簡体中文}呆\end{簡体中文}」に変える必要がある.このように,機能語の影響も翻訳処理にとり入れる必要が出てくる(誤訳64文のうち2文(3.2\%)).



\section{存在文の日中機械翻訳に関連する今後の課題}

\subsection{テンス・アスペクトに関する問題}

中国語のアスペクト助詞は事柄のテンス・アスペクトを表し,助詞「\begin{簡体中文}了/着/\UTFC{8FC7}\end{簡体中文}」などがある.「\begin{簡体中文}了\end{簡体中文}」には(1) 過去を表す,(2) 完了または実現を表す,(3) 変化が生じた事を表す,(4) 語気の役を担うなどの意味用法がある.「\begin{簡体中文}着\end{簡体中文}」には(1) 動作の進行中を表す,(2) 動作・状態の持続を表す,(3) 語気を表すなどの意味用法がある.また「\begin{簡体中文}\UTFC{8FC7}(過)\end{簡体中文}」には動作が済んだことあるいは経験を述べる意味用法がある.

「ある」/「いる」の中訳語「\begin{簡体中文}有\end{簡体中文}」と「\begin{簡体中文}在\end{簡体中文}」は静態動詞であるので,状態や性質を表し,時間性と関係ないため,現在のJawのアルゴリズムによれば,過去形「あった,いった」の場合の翻訳には「\begin{簡体中文}了\end{簡体中文}」を使わない.しかし,ある場合は「\begin{簡体中文}了\end{簡体中文}」が必要で,ある場合は過去を表す時間副詞の補足が必要である.

 例:お父さんはもういません.(生死文と実在文)

 現訳:\begin{簡体中文}父\UTFC{4EB2}已\UTFC{7ECF}不在.\end{簡体中文}

 正訳:\begin{簡体中文}父\UTFC{4EB2}已\UTFC{7ECF}去世\ul{了}.\end{簡体中文}

 例:ここに山があっ\ul{た}.

 現訳:\begin{簡体中文}\UTFC{8FD9}里有山.\end{簡体中文}

 正訳:\begin{簡体中文}\UTFC{8FD9}里曾\UTFC{7ECF}有山.\end{簡体中文}

また,中国語におけるテンス・アスペクトを表現するのには,「\begin{簡体中文}了\end{簡体中文}」,「\begin{簡体中文}着\end{簡体中文}」,「\UTFC{8FC7}」などの助詞で表現だけではなく,時間副詞(「\begin{簡体中文}已\UTFC{7ECF}\end{簡体中文}」,「\begin{簡体中文}曾\UTFC{7ECF}\end{簡体中文}」,「\begin{簡体中文}就\end{簡体中文}」,「\begin{簡体中文}在\end{簡体中文}」など),趨向補助語(「\begin{簡体中文}去\end{簡体中文}」,「\begin{簡体中文}来\end{簡体中文}」,「\begin{簡体中文}起来\end{簡体中文}」など)および結果補助語(「\begin{簡体中文}完\end{簡体中文}」,「\begin{簡体中文}到\end{簡体中文}」,「\UTFC{89C1}」,「\begin{簡体中文}在…上\end{簡体中文}」など)を用いる場合がある.テンスとアスペクトの翻訳アルゴリズムの分析を深めることは今後の課題である.


\subsection{「ている」と「てある」}

日本語の存続体はある種の動作が終了しその結果が存続する状態にあることを表し,他動詞の連用形+「てある」(助詞「を」は「が」に変更する),あるいは自動詞の連用形+「ている」により構成される.例えば

\inHRei{(13)}
壁に絵を掛ける.\\
→\begin{簡体中文}把画挂在\UTFC{5899}上.\end{簡体中文}

\inHRei{(14)}
壁に絵が掛かって\ul{いる}.\\
→\begin{簡体中文}\UTFC{5899}上\ul{挂着}画.\end{簡体中文}

\inHRei{(15)}
壁に絵が\ul{掛けてある}.\\
→\begin{簡体中文}\UTFC{5899}上\ul{挂着}画.\end{簡体中文}

中国語の文法では文13は「把字句」という文型であるが,文14と文15は「存在句」という文型であり,異なる構造の文として表現される.

\inHRei{(16)}
犬は\ul{放してある}.\\
→\begin{簡体中文}狗\ul{被放\mbox{\UTFC{5F00}}}.\end{簡体中文}

\inHRei{(17)}
自動車は歩道に沿って\ul{駐車してある}.\\
→\begin{簡体中文}汽\UTFC{8F66}沿路\ul{停着}.\end{簡体中文}

\inHRei{(18)}
この家は左利きの人が住みやすいように\ul{工夫してある}.\\
→\begin{簡体中文}\UTFC{8FD9}家\UTFC{4E3A}了\UTFC{8BA9}左\UTFC{6487}子人住得方便,\ul{花了功夫}!\end{簡体中文}

文16は,中国語の受身文である「被字句」という文型に翻訳される.文17は助詞「\begin{簡体中文}着\end{簡体中文}」の補足が必要で,文18は助詞「\begin{簡体中文}了\end{簡体中文}」の補足が必要である.このように,「てある」の文は「存在句」,「被字句」,「\begin{簡体中文}着\end{簡体中文}」の補足と「\begin{簡体中文}了\end{簡体中文}」の補足などいろいろ訳し分けが必要である.

\inHRei{(19)}
実は,銀行には国際化と自由化の荒波が\ul{押し寄せている}.\\
→\begin{簡体中文}原来,国\UTFC{9645}化和自由化的\UTFC{6EDA}\UTFC{6EDA}浪潮\ul{正在}向\UTFC{94F6}行\ul{冲来}.\end{簡体中文}

\inHRei{(20)}
彼女は口をあんぐり開けたまま\ul{突っ立っている}.\\
→\begin{簡体中文}\UTFC{5979}\UTFC{5F20}着大嘴巴\ul{站着}.\end{簡体中文}

\inHRei{(21)}
彼は最近やせて\ul{いる}.\\
→\begin{簡体中文}他最近\UTFC{7626}\ul{了}!\end{簡体中文}

上例のように,「ている」も,「てある」と同様に,いろいろな文型に翻訳され,訳語の区別が必要である.

現在の日中市販翻訳ソフトを調べると,「ている」と「てある」に関する誤り(語順の問題や動詞の誤訳など)が多く見られる.「ている」と「てある」における翻訳規則の整理が必要である.



\section{おわりに}

本論文では日本語の基本表現の一つである存在文の中国語への翻訳処理について述べた.

存在文の日中翻訳に関しては,主として次のような問題が観察された.

\begin{itemize}
\item[(i)]
存在動詞の訳し分けの問題.存在動詞の対応が一対多であるが,日本語の基本的な存在動詞は「ある」あるいは「いる」であるが,「ある」は中国語に翻訳すると,意味と文法上の規定により,「\begin{簡体中文}在,有,在于,\UTFC{53D1}生,\UTFC{8FDB}行…\end{簡体中文}」の可能性がある.「いる」の場合も「\begin{簡体中文}在,有,\UTFC{5904}于…\end{簡体中文}」などの可能性がある.

\item[(ii)]
訳語の語順の問題.特に連体修飾存在文の語順の問題と存在動詞の翻訳位置の問題.

\item[(iii)]
介詞の訳し分けの問題.

\item[(iv)]
習慣用法の誤訳の問題.
\end{itemize}

これらの問題を解決するために,日中両言語の存在文における異同について考察し,日中機械翻訳のために,日本語文の構文特徴,対応名詞の属性,構文構造などを利用して存在文の翻訳規則をまとめた.中国語の構文上の組み合わせの制限を総合して考察し,判定条件を機械で処理の可能な形で示した.更に例文を用いて手作業でこの規則を検証し,手作業による評価では市販ソフトと比較して良好な結果を示した.これらの翻訳規則は我々の研究室で開発している日中機械翻訳システムJaw/Chineseに組み込んで,翻訳実験を行っている.

今後例文を更に増やして翻訳実験を進め,翻訳システムの改良を行い,さらに翻訳精度が上がるように分析を深めていく予定である.また,関連するテンス.アスペクトの問題や「ている」,「てある」の問題なども含めてさらに広く翻訳処理について検討を進めていく予定である.


\acknowledgment

本研究を進めるにあたって岐阜大学池田研究室jaw/Chineseのグループの皆様及びその研究室の他の皆様に感謝します.また,本論文に対して有益な御意見,御指摘を頂きました査読者の方に感謝致します.


\begin{thebibliography}{3}


\item
ト朝暉,池田尚志:日中機械翻訳における否定文の翻訳(2004).自然言語処Vol. 11, No. 3. July. 2004. p. 97--112.

\item
グループ・ジャマシイ(2001).日本語文型辞典.くろしお書店.

\item
飯田隆(2001).在と言語—存在文の意味論,http://phil.flet.keio.ac.jp/person/iida/papers/Sonzai.pdf.

\item
池原悟他(1997).日本語語彙大系—5構文体系.岩波書店.

\item
金田一春彦,林大,柴田武(1988).日本語百科大事典.大修館書店.

\item
金水敏(2006).存在表現の歴史.ひつじ書房.

\item
小池清治,小林賢次,細川英雄,山口佳也(2002).日本語表現・文型記事,朝倉書店.

\item
劉月華,潘文娯,故\UTFC{97E1}(1996).現代中国語文法総覧,くろしお出版.

\item
村上仁一(2002).日英対訳データーベースの状況.「言語,認識,表現」第7回年次研究会プログラム.

\item
西山佑司(2003).日本語名詞句の意味論と語用論,ひつじ書房.

\item
北京・対外経済貿易大学,北京・商務印書館/小学館.日中辞書.

\item
王軼謳,ト朝暉,宇野修一,浅井良信,池田尚志(2006).日中機械翻訳における存在文および関連する問題について,情報処理学会研究報告 2006-NL-171, pp. 95--102.

\item
王軼謳,ト朝暉,宇野修一,浅井良信,池田尚志(2006).日中機械翻訳における存在文の翻訳処理について.言語処理学会第12回年次大会発表論文集.pp. 660--663.

\item
謝軍,今井啓允,池田尚志(2004).日中機械翻訳システムjaw/Chineseにおける変換・生成の方法,自然言語処理Vol. 11, No. 1. p. 43--80.

\item
儲澤祥,劉精盛,龍国富,田輝(1997).「\UTFC{6C49}\UTFC{8BED}存在句的\UTFC{5386}\UTFC{65F6}性考察」.古\UTFC{6C49}\UTFC{8BED}研究.

\item
宇野修一,福本真哉,田中伸明,松本忠博,池田尚志(2005).日本語から多言語への翻訳エンジンjaw.言語処理学会第11回年次大会発表論文集.pp. 538--541.

\item
山口巌:存在文と存在否定文について(1979).言語研究75,pp. 1--30.


\end{thebibliography}




\clearpage

\section*{付録1 最適解の選択}

最適解の選択には,3節に述べたように,以下のような条件を設定している.

\noindent
・日本語表現木を構成するパターンの数や種類による条件

\noindent
・適用された制約条件の厳しさによる条件

\noindent
・適用された意味属性間の距離による条件


\subsection*{1.日本語表現木を構成するパターンの数や種類による条件}

パターンの種類としてはAdditionタイプ(節4.2.1)よりもBaseタイプ(節4.2.1)を優先し,パターンの数がより少ない方を優先する.これらのことは,図1に示すコスト表とコストの計算式に反映されている.以下に例文「彼は夜に仕事する」を例として,コストの計算例を示す

\begin{figure}[b]
\begin{center}
\includegraphics{14-5ia3f1.eps}
\end{center}
 \caption{パターンの数や種類による最適解の選択時のコスト計算}
\end{figure}


以下の(1) 〜(4) のように,入力例文を覆う日本語パターンが照合できたとすると,それぞれのコストは右側に示したように,計算される.

2)と4)の場合は,足りない文節をAdditionタイプで補ったり,必要ない文節を省略していることから,パターン数は4つであり,合計コストが高くなっている.また,1)と3)を比較した場合,パターンの数は3つと同じであるが,3)は字づらによる固定表現がされており,1)よりも適合していることになる.よって,合計コストに従って3)がこの4つのTTのうち最も適した解といえる.

このほか,パターン変形などの特殊処理を行って照合に成功したパターンよりも登録されているパターンをそのまま使用して照合に成功したパターンを優先する.例えば受身や使役表現などの場合,パターン辞書に登録してある基本能動態の表現パターンを受身や使役のパターンに変形した上で,入力文との照合を行っている.図3は受身・使役表現におけるパターン変形の処理例を示している.

\begin{figure}[t]
\begin{center}
\includegraphics{14-5ia3f2.eps}
\end{center}
 \caption{パターンの数や種類による最適解の選択の例}
\end{figure}

しかし「息を弾ませる」のように慣用的な表現は,使役形のままでパターンに登録されており,変形して生成されたパターンをの照合より,そのままの形でのパターンの方式コストは低くなるように設定してある.

\begin{figure}[t]
\begin{center}
\includegraphics{14-5ia3f3.eps}
\end{center}
 \caption{受身・使役表現におけるパターン変形の処理例}
\end{figure}

\subsection*{2 適用された制約条件の厳しさによる条件}

各パターンの条件には,自立語条件,字づら条件,機能語条件の3つの制約条件がある.この3つの条件を用いて各パターンの条件の厳しさを判定する.字づら条件が最も優先される.自立語条件は,意味属性がシソーラスで表現されているので,照合に適用された自立語条件の意味属性のシソーラスでの深さを自立語条件の得点として厳しさを判定する(得点の高い方は厳しいである).機能語条件は,例えば「が」の得点は1,「に対して」の得点は3.7というように,個々の機能語条件について得点が設定されている.自立語条件のシソーラス上での深さと機能語条件の得点の積を条件文節ごとに算出し,条件文節ごとの得点の和がパターンが持つ条件の厳しさとなる.図4に,条件の厳しさの計算式を示す.

\begin{figure}[t]
\begin{center}
\includegraphics{14-5ia3f4.eps}
\end{center}
 \caption{制約条件の厳しさによる最適解の選択時のコスト計算}
\end{figure}

例として,「彼が道を歩く」を照合して生成された三つのTTに対する計算例を示す(図5).

1)と2)を比較すると,主語の意味属性は〈人〉で両パターンとも同じであるが,目的言語の意味属性は〈場所〉と〈道路〉になっている.このとき,自立語のシソーラスを見ると意味属性の〈場所〉は,〈道路〉を包含することとなり,〈場所〉よりも〈道路〉の方が深い意味属性となる.

よって,2)は1)よりも深い意味を持つ自立語条件を持っていることとなる.3)の場合「N〈人〉がN〈乗り物〉でN〈道路〉を歩く」というパターンの条件文節「N〈乗り物〉で」が省略されて適用されたものである.このとき,3)は,3つの中で一番条件が厳しいパターンとして考えられるが,「N〈乗り物〉で」の文節が省略された場合は,2)より得点が低くなる.


\begin{figure}[p]
\begin{center}
\includegraphics{14-5ia3f5.eps}
\end{center}
 \caption{制約条件の厳しさによる最適解の選択の例}
\vspace{2\baselineskip}
\begin{center}
\includegraphics{14-5ia3f6.eps}
\end{center}
 \caption{意味属性同士の関連}
\end{figure}

\subsection*{3 適用された意味属性間の距離による条件}

意味属性条件に関してはパターン中で与えられている意味属性と入力文の語の意味属性とのシソーラス上の距離が短い方を優先する.

例文「彼が道を歩く」に対してBaseタイプのパターン「N〈人〉がN〈道路〉を歩く」が適用されたとき,文節「N〈人〉が」の自立語に「彼〈3人称〉」,「彼〈男〉」の2つのBaseタイプのパターン「彼」が適用されたものとする.

このとき,2つのパターンのどちらが最適なパターンかを判別する必要がある.それぞれのパターンが持つ意味属性は「〈3人称〉」,「〈男〉」である.これが適用されたということは,この2つのパターンは「歩く」の条件文節「〈人〉」の包含関係にあるといえる.意味属性同士の関連を図6に示す.

このとき,2)の意味属性〈男〉は〈3人称〉より自立語の条件である〈人〉に近い属性なので「歩く」が持つ文節「〈N〉が」に3)より適切な属性といえます,よって2)と3)を比較した場合2)のBaseタイプのパターン「彼〈男〉」が選択される.



\subsection*{4 各選択条件の優先順位}

最適解の選択は,「TTを構成するパターンの数や種類による条件(1) 」,「適用された制約条件の厳しさによる条件(2) 」,「適用された意味属性間の距離による条件(3) 」の3つの点数の合計で選択を行っている.この3つの条件を用いた最適解選択の計算式は以下のようになる.

\begin{figure}[h]
\begin{center}
\includegraphics{14-5ia3f7.eps}
\end{center}
 \caption{最適解選択の計算式}
\end{figure}







\clearpage
\section*{付録2 「ある」の字面照合翻訳規則}
\input{03app02.txt}






\clearpage
\section*{付録3 「ない」の字面照合翻訳規則}
\input{03app03.txt}


\begin{biography}

\bioauthor{王 軼謳}{
2001年中国大連理工大学化学工学科及び英語科卒.2004年中国大連理工大学応用情報研究科修士課程修了.工学修士.現在岐阜大学工学研究科博士後期課程在学中.日中機械翻訳,特に機械翻訳のための日中言語の分析,翻訳規則の作成,システムの改良に関する研究に従事.言語処理学会学生会員.}

\bioauthor{池田 尚志(正会員)}{
1968年東京大学教養学部基礎科学科卒.同年工業技術院電子技術総合研究所入所.制御部情報制御研究室,知能情報部自然言語研究室に所属.1991年岐阜大学工学部電子情報工学科教授.現在,同応用情報学科教授.工博.自然言語処理,人工知能の研究に従事.情報処理学会,電子情報通信学会,人工知能学会,言語処理学会,各会員.}

\end{biography}


\biodate


\end{document}

