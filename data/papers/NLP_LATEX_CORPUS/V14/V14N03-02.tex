    \documentclass[japanese]{jnlp_1.3c}
\usepackage{jnlpbbl_1.1}
    \usepackage[dvips]{graphicx}
    \usepackage{lingmacros}
    \usepackage{udline}

\setulminsep{1.2ex}{0.2ex}

\setcounter{secnumdepth}{3}


\Volume{14}
\Number{3}
\Month{Apr.}
\Year{2007}

\received{2006}{4}{18}
\revised{2006}{8}{21}
\rerevised{2006}{10}{10}
\accepted{2006}{10}{18}

\setcounter{page}{17}

\jtitle{共在性からみた「です・ます」の諸機能}
\jauthor{
 宮地 朝子\affiref{NagoyaU} 
 \and 北村 雅則\affiref{NIJLA}
 \and 加藤  淳\affiref{NagoyaU}
 \and 石川美紀子\affiref{NagoyaU}
 \and \\ 加藤 良徳\affiref{SEU}
 \and 東  弘子\affiref{APU}}

\jabstract{
「です・ます」は,丁寧語としての用法のみならず場面に応じてさ
まざまな感情・態度や役割の演出などの表示となる.これは「です・ます」が持
つ「話手と聞手の心的距離の表示」という本質と,伝達場面における話手/聞手
のあり方とその関係の変化によって生じるものと考えられる.本稿では「です・
ます」をはじめ聞手を必須とする言語形式を,コンテクストとは独立して話手/ 
聞手の〈共在〉の場を作り出す「共在マーカー」と位置づけ,コンテクストにお
ける聞手の条件による「共在性」と組み合わせることで伝達場面の構造をモデル
化した.コミュニケーションのプロトタイプとしての〈共在〉の場では,「です・
ます」の本質的な機能が働き心的距離「遠」の表示となる.これに対して〈非共
在〉の場では,典型的には「です・ます」は出現しない.しかし,〈非共在〉の
場合でも共在マーカーが使用されると話手のストラテジーとして疑似的な〈共在〉
の場が作り出される.この場合,共在マーカーとしての役割が前面に出ることに
よって聞手が顕在化し,話手/聞手の関係が生じて「親・近」のニュアンスが生
まれる.「です・ます」が表す「やさしい」「わかりやすい」「仲間意識」など
の「親・近」の感情・態度は〈非共在〉を〈共在〉にする共在マーカーの役割に
よって,「卑下」「皮肉」といった「疎・遠」の感情・態度は〈共在〉での心的
距離の操作による話手/聞手の関係変化によって説明できる.}

\jkeywords{です・ます,共在性,共在マーカー,感情の表示}

\etitle{Copresence and The Functions of ``-desu/-masu''}

\eauthor{
 Asako Miyachi\affiref{NagoyaU}
 \and Masanori Kitamura\affiref{NIJLA}
 \and Jun Kato\affiref{NagoyaU}
 \and Mikiko Ishikawa\affiref{NagoyaU}
 \and \\ Yoshinori Kato\affiref{SEU}
 \and Hiroko Azuma\affiref{APU}}

\eabstract{
The ``{\it -desu/-masu}'' form not only functions as a formal
language marker, it also indicates emotions/attitudes and characterizes
some roles in conversation. These attribute of ``{\it -desu/-masu}''
indicate ``psychological distance between speakers and addressees'', the
relationship between speakers and addressees in communication and its
change. This paper defines the language forms that require addressees,
such as ``{\it -desu/-masu}'', as ``copresence markers'' that create
copresent space independently of contexts. We model the structure of
communication with the notion of a copresence marker and the degree of
copresence determined by contexts. In copresence, ``{\it -desu/-masu}''
indicates psychological distance. In nonpresence, however ``{\it
-desu/-masu}'' does not typically appear, but a quasi-copresence sort of
virtual space can be created by a copresence marker. In this case, the
function of ``{\it -desu/-masu}'' as a copresence marker is foregrounded
and an addressee is elicited. This relationship between a speaker and an
addressee results in intimate emotion. Intimate emotions/attitudes can
be explained by the function of a copresence marker that changes
``nonpresence'' into ``copresence''. Distance can be explained by the
change of mental relationship between speakers and addressees with
psychological distance operating in ``copresence''.}

\ekeywords{{\it -desu/-masu}, Copresence, Copresence marker, Indication of emotion}

\headauthor{宮地,北村,加藤,石川,加藤,東}
\headtitle{共在性からみた「です・ます」の諸機能}

\affilabel{NagoyaU}{名古屋大学文学研究科}{Graduate School of Letters, Nagoya Univ.}
\affilabel{NIJLA}{独立行政法人 国立国語研究所}{The National Institute for Japanese Language}
\affilabel{SEU}{静岡英和学院大学人間社会学部}{Faculity of Humanities and Social Sciences, Shizuoka Eiwa Gakuin Univ.}
\affilabel{APU}{愛知県立大学外国語学部}{Faculty of Foreign Studies, Aichi Prefectural Univ.}
\begin{document}
\maketitle


\section{はじめに}

現代日本語の「です・ます」は,話手の感情・評価・態度に関わるさまざまな意
味用法を持つことが指摘されている.従来の研究では,敬語および待遇表現,話
し言葉/書き言葉の観点や,文体論あるいは位相論といった立場・領域から個別
に記述されてきたが,「です・ます」の諸用法を有機的に結びつけようとする視
点での説明はなされていない\footnote{
	「敬語」の一種であるという位置づけがなされている程度である.一
	例として,次のような記述がある.「「です・ます」は,一連の文章や話し言葉
	の中では,使うとすれば一貫して使うのが普通で,その意味で文体としての面を
	もちます.「です・ます」を一貫して使う文体を敬体,一貫して使わない文体を
	常体と呼びます.(中略)しかし,文体である以前に,「です・ます」はやはり
	まず敬語です」(菊地1996: 90--91)}.
本稿では「伝達場面の構造」を設定し,言語形式「です・ます」の諸用法を,
その本質的意味と伝達場面との関係によって導かれるものと説明する.こうした
分析は,「です・ます」個別の問題に留まらず,言語形式一般の記述を単純化し
ダイナミックに説明しうる汎用性の高いものと考える.

本稿の構成は以下の通りである.まず,\ref{youhou}. で従来指摘されている
「です・ます」の諸用法を確認し,\ref{model}. において「話手/聞手の「共在
性」」に注目しつつ伝達場面の構造をモデル化する.さらに「共在性」を表示す
る形式を「共在マーカー」と名付け,なかでも「です・ます」のような聞手を前
提とする言語形式の操作性に注目する.これを受けて\ref{meca}. では,「です・
ます」の「感情・評価・態度」の現れが,伝達場面の構造モデルと「です・ます」
の本質的機能および共在マーカーとしての性質から説明できることを述べ,
\ref{matome}. のまとめにおいて今後の課題と本稿のモデルの発展性を示す.

\section{「です・ます」の諸用法 --- 従来の指摘と本稿の立場}\label{youhou}

\subsection{従来の指摘}

まず「です・ます」の諸用法について,先行研究における指摘,記述を確認して
おこう.

\subsubsection{丁寧語としての「です・ます」}

(\ex{+1})は,クラスメートと先生に対しての小学生の発話である\footnote{
	以下,用例の下線は引用者による.また出典情報が長い場合は脚注と
	する.下線の種類は次の通りである.
	\begin{list}{}{}
	 \item  です・ます:\unami{   },非です・ます:\utensen{   },その他:\ul{   }.
	\end{list}}.

\enumsentence{(司=小学生,女子=クラスメート)
 \begin{description}
  \item[司:] (クラスメートに対して)「小坂先生に恋人がいることは本当の
	     ことだし,みんな,どんな人か興味\utensen{あるよね}」
  \item[女子:] 「\utensen{あった},すごく興味\utensen{あった}」(…中略…)
  \item[司:] (先生に対して)「ね.僕は先生にアドバイスされた通り,みん
	     なの知りたがってることを書いた\unami{だけです}」
\end{description}
 \hfill (シナリオ「うちの子にかぎって……」\unskip\footnote{
	BAN IS FOR BAN伴一彦オフィシャルサイト. 
	{\tt http://www.plala.or.jp/ban/index.html}(2006.2.5アクセス)})
}

クラスメートに対しては,「興味あるよね」と「非です・ます」の普通体を使う
のに対し,先生に対しては,「書いただけです」と「です・ます」を使う\footnote{
	ここでの「です・ます」,「非です・ます」が使われる場は,鈴木
	(1997)の「丁寧体世界」・「普通体世界」にそれぞれ相当する.}.

聞手が目上である場合だけでなく,初対面,ソトの人物,嫌いな人物の場合,あ
るいは公的な場,忌避すべき話題に言及する場合などに「です・ます」が用いら
れる.近年のポライトネス理論では敬意の表示よりむしろ,聞手との心的距離の
表示として説明される\footnote{
	滝浦(2002, 2005a, 2005b),また本稿\ref{Ikyori}節参照.}.

\subsubsection{文体の基調をなす「です・ます」}

新聞などの文字メディアでは,(\ex{+1}a)(\ex{+2}a)のように「非です・ます」
の「だ・である体」が一般的であるが,(\ex{+1}b)(\ex{+2}b)のように「です・ま
す体」が使われる場合もある.文末が全て「です・ます」の場合は,書き言葉に
おける「です・ます体」というスタイルの一つと捉えられている.

\eenumsentence{
 \item 財政再建にあたって国・地方の公務員の総人件費削減は\utensen{緊急課
 題だ}.高すぎる給料や余剰な定員を大胆に\utensen{削減する}.官僚の抵抗は
 強まろうが,小泉純一郎首相の言う構造改革の試金石\utensen{である}.(中
 日新聞社説,2005.10.24)
 \item 「ポスト郵政」の最大テーマに政府系金融機関の改革が\unami{浮上して
 います}.ここは官僚機構のいわば\unami{「聖域」です}.小泉純一郎首相はど
 こまで\unami{切り込めるでしょうか}.(中日新聞社説,2005.10.23)
}
\eenumsentence{\label{iraq}
 \item イラクに派遣された陸上自衛隊は,既に相応の責任を果たした.基本計
 画の修正により派遣期間は一年延長されたが,今から撤収の準備に着手
 \utensen{すべきだ}.(中日新聞社説,2005.12.9)
 \item イラクに駐留する自衛隊は,しばらく復興支援を続けることに\unami{な
 りました}.でも,いずれ治安がよくなれば政府開発援助 (ODA) などに
 \unami{出番が回るはずです}.(中日新聞社説,2005.12.11)}

(\ex{-1}b)(\ex{0}b)のように尊敬語や謙譲語とともに使用されていない「です・
ます」については,
永野(1966)が
「「です・ます体」は読者への ``敬意'' にもとづく敬体ではなく,相手意識の
相対的強さを感じさせる「文体」である」と,つとに指摘しているが,話手\footnote{
	本稿における「話手」と「聞手」は,「話す」「聞く」という行為の
	参与者に限定するものではなく,メディアにかかわらず言語の発信者とその受け
	手に相当する術語として用いる.}の「相手意識の相対的強さ」という漠とした基準は,文体の特徴を述べる指摘に
とどまっている.

\subsubsection{感情・態度の表示とされる「です・ます」}\label{hyoji}

(\ex{+1})は,スポーツ新聞のコラムである.「非です・ます」で書かれたコラ
ムの最後の一文に「です」が使用され,感情の表示となっている.

\enumsentence{福留が名カメラマンぶりを\utensen{披露した}.室内練習場で報
道陣から取材用のカメラを拝借.マシン打撃中の森野を\utensen{激写だ}.
そのうちの1枚がボールがバットに当たる,打つ瞬間をバッチリ\utensen{捉(とら)
えていた}.これには貸したカメラマンもびっくり.「おれ,職間違えた.カメラ
マンが合ってるよ」とは福留.いやいや,できる人は何をやらせてもできるとい
うこと\utensen{です}.(中日スポーツ,2006.2.4)}

会話でも,「非です・ます体」のくだけたやりとりの中で「です・ます」が出現
し,話手の感情・評価・態度を表示する効果が生まれることがある\footnote{
	例文(\ex{+1})〜(\ex{+3})は鑑定士・気象予報士・芸能人など「特定
	のキャラクターと結びついた,特徴のある言葉づかい」の「役割語」(金水2003) 
	として,単に感情を表示するのみならずある種の役割と結びついて態度の表示と
	なっていると捉えることもできる.}.

\enumsentence{(鑑定士ぶって)「\unami{いい仕事してますねえ}」}
\enumsentence{(気象予報士ぶって)「今日は花粉が\unami{多いようです}」}
\enumsentence{(記者会見の芸能人ぶって)「\unami{4カラットですの},ホホホ」\unskip\footnote{
	例文(\ex{0})は定延利之氏との個人談話による.}}

こうした例については,従来多くの論考で,普通体と丁寧体の「混在・混用」と
捉えられ,その「スタイルシフト」によって感情が表示される効果があると説明
されてきた (メイナード1991, 2001a, 2001b; 日高2004, 2005など).メイナード
は,「です・ます」は相手への配慮を表すとし,「相手意識の強さや相手に自分
をどのようにアピールしたいかという」話手の「感情が作用」(メイナード
2001b)してシフトが起こるとしている.しかし「です・ます」の感情表示機能発
現のメカニズムを,他の用法と関連づけようとはしていない.

以上のように「です・ます」の諸用法についてはさまざまな記述がある.しかし,
それぞれ離散的,個別的に記述されたものであり,「です・ます」がなぜこのよ
うな諸用法を持つのか,その用法はいかなる条件で現れるかといった包括的な説
明は試みられていない.

\subsection{本稿の立場}

本稿は,言語形式の現れ方を個別に記述説明するだけでなく,話手/聞手のあり方
を含めた伝達場面を設定することにより,言語形式の「文法」と「文体」を有機
的に統合し,諸用法に対する包括的説明を試みるものである.話手/聞手の関係や
条件から言語形式の用法を説明するという点においては,言語行為論(Austin
1962; Searl 1969),関連性理論(Grice 1975; Sperber and Wilson 1986),ポ
ライトネス理論(Brown and Levinson 1978)などの語用論的研究,さらに,語用
論的条件を組み入れた統合的な理論であるという点で談話管理理論(田窪,金水
1993など)などと軌を一にする.しかし,本稿のモデルでは,聞手は話手から見
た受け手として発話の場を条件づける要素であるに留まり,情報内容の伝達やコ
ミュニケーションの成否を問題にするものではない.つまり,コミュニケーショ
ンモデルの中で言語現象を説明するという目的から導き出されたものである一方,
話手の認識条件のみで言語形式の用法を説明できる汎用的な発話モデルとして提
案するものである.以下,「伝達場面の構造」の枠組みから「です・ます」の諸
用法を説明していく.

\section{「伝達場面の構造」モデル}\label{model}

言語研究において,コミュニケーションモデルや語用論的条件を考慮に入れた試
みは多いが,一方で理論上の限界と問題点も指摘されている.例えば談話管理理
論では,話手/聞手の「相互知識の無限遡及」(Clark and Marshall 1981)を回
避して知識の相互性から独立した言語形式の記述を提案したにもかかわらず,
「伝達モデルの残滓というようなア・プリオリに仮定された発話の意味や意図を
引きずって」いると指摘される(山森1997).また,感動詞など談話に特有の言
語形式の分析記述においても機能や目的を読み込む「狩人の智恵」式機能主義が
跋扈しているという憂慮もある(定延2005e).さらに,話手/聞手の知識・情報
差とその伝達の成否を前提としたコミュニケーションモデル(コード・モデル)
の前提そのものに関わるパラドックスも指摘されている\footnote{
	「共有知識(shared knowledge)のパラドックスに関する批判[Clark
	and Marshall 1981; Sperber and Wilson 1986]」(水谷1997)}.

以上の問題点をふまえ,本稿では日常の対面対話をプロトタイプとし,情報伝達
の成否を問題としない「伝達場面」の構造モデルを提示する.

\subsection{伝達場面の構造モデル}\label{submodel}

\subsubsection{〈共在〉/〈非共在〉\unskip\protect\footnote{
	本稿の〈共在〉/〈非共在〉は,定延(2003)でも引用される
	Tannen (1980, 1982)のinvolvementとその訳語「共在(性)」から示唆を得て設定
	したものであるが,Tannenのinvolvement(「相手と同じコミュニケーションの場
	に身を置き,その場の中で,(時には相手と一緒に)言語表現をおこなうという
	構図」
	)
	 / detachment(「解釈者とは切り離された構図」)とは異なる.また
	Goffman (1963)のcopresence,Clark and Carlson (1982)の言語的共在
	(linguistic copresence),物理的共在(physical copresence),木村(1996)
	の「共在」などの術語「共在/copresence」とも同一の概念ではない.}}

伝達場面の構造モデルでは,日常の対面対話すなわち話手から個別・具体・特定
の受け手(聞手とする)への発話をコミュニケーションのプロトタイプとし,こ
のような発話の場を「共在」とする.

「共在性」はさまざまな要因によって決定されうるが,最も重要な条件は聞手の
特定性,個別・具体性である.例えば伝達場面において,特定の聞手と対面して
いるなどの条件があれば共在性は高く,その場は〈共在〉といえる
(図\ref{kyozai}(I))
.〈共在〉の場においては,「話手から個別・具体・特定の受け手への発話とし
ての表示」がある.その表示,マーカーを「共在マーカー」と呼ぼう.共在マー
カーには言語形式と非言語形式がある.話手は,具体的な発話場面(時間・場)
を共有する特定の聞手に対し,共在マーカーとして,表情,視線,身振りなどと
ともに,あいづち,いいよどみ,といった談話の標識を自在に用いることができ
る.また聞手への働きかけ(質問・命令・勧誘),話手の視点に関わる表現で聞
手や場の位置づけを前提とする表現(ダイクシス・待遇・授受表現など)や文脈
情報の扱いを表示する言語形式(終助詞(山森1997))などが使用される.「で
す・ます」も話手による聞手の位置づけの表示であり,相手との心的距離を示す
ものとして,「聞手を必須とする要素」共在マーカーの一つにあたる.

\begin{figure}[htbp]
 \begin{center}
  \includegraphics[scale=0.8]{./zu1.eps}
  \caption{\label{kyozai}聞手条件による伝達場面の共在性}
 \end{center}
\end{figure}

\subsubsection{共在マーカーの操作性}

図\ref{kyozai}に示すように,聞手が不特定・多数で抽象的である場合の伝達場
面は〈非共在〉で,共在マーカーは現れない.個別・具体・特定の聞手が存在す
る場合,伝達場面は〈共在〉であり,言語形式としての共在マーカーが現れる.
伝達場面における「共在性」の高/低は聞手の個別・具体性によるもので,共在
マーカーの出現と連動する.

共在性の低い場,すなわち(II)〈非共在〉の場では共在マーカーは基本的に出現
しない.しかし,文字媒体のマス・メディアのような受信者が不特定多数の伝達
場面においても,「です・ます」や終助詞が使用されることがある\footnote{
	映像媒体のマス・メディアにおいても共在マーカーが使用される傾向
	にある.受信者が不特定多数という点では文字媒体のマス・メディアと同じであ
	るが,「カメラ」の存在によって,基本的に場の共在性が「高」の,対面のコミュ
	ニケーションと位置づけられる.すなわち〈共在〉(I)として共在マーカーが
	出現すると考えられ,次に述べる〈疑似共在〉を作り出すストラテジーとは区別
	する.}.
これは,プロトタイプの場で共在マーカーが共在の表示となるという関係から
説明できる.聞手が個別・具体・特定でない非共在の場において,あえて有標の
言語形式「共在マーカー」\unskip\footnote{
	話手にとって操作可能性・応用性の最も高いのが言語形式の共在マーカーである.
	共在マーカーは,形式として明示的であるという点でメディアの制約を受けにく
	い.記号(!・?・♪),絵文字やフェイスマーク,音声特徴を示す表記なども,
	「形を持つ」共在マーカーといえる.}を用いることで,
共在マーカーは「疑似的な〈共在〉の場」を作り出すストラテ
ジーとなる(図\ref{kyozai2}).

このように,共在マーカーを用いることで設定される疑似的な〈共在〉の場
(III)を〈疑似共在〉と呼ぼう.(I)の共在性と(II)の非共在性は,話手の聞手認
識が,個別・具体・特定か,不特定多数・抽象かといった対立によって決まるが,
(III)は,共在マーカーの使用以外に共在性を保証する要素はなく,共在マーカー
の出現によって〈共在〉の場が構築されると考えられる.以下ではこの伝達場面
の違いを確認した上で,共在マーカーの諸用法を動態的に説明していく.

\begin{figure}[htbp]
 \begin{center}
   \includegraphics[scale=0.8]{./zu2.eps}
  \caption{\label{kyozai2}共在マーカーの使用による(III)〈疑似共在〉の構築}
 \end{center}
\end{figure}

\subsection{伝達場面の諸相}

図\ref{kyozai2}では,(I)(II)(III)(IV)の4つの伝達場面を設定している.(I)は,
話手から特定の聞手への発話の場〈共在〉であり,通常の対面対話や特定の個人
宛の手紙・メールなどである.(II)は,聞手が不特定多数・抽象的である場〈非
共在〉であり,新聞・論文など「書くメディア」を典型とする.(I)(II)では,
共在マーカーの有無が〈共在〉/〈非共在〉に対応するが,(III)(IV)では一致し
ない.(I)(II)の関係を前提として,(III)は不特定多数の聞手を特定化する
〈疑似共在〉の場であり,(IV)は目の前に対面する聞手が具体的にありながら共
在マーカーを使用しない疑似的な〈非共在〉の場である.

\subsubsection{共在マーカー「無」の場面 --- (II)(IV)の非共在性}

(II)(IV)では共在マーカーが出現しないという意味でも,また伝達場面のプロト
タイプとしての(I)との対立においても,〈非共在〉性を持つ.ただし,それぞ
れの対立による効果は異なる.

\subsubsection{(II)の非共在性 --- 「論理性」の追求}\label{tuikyu}

(II)は聞手の不特定性によって「共在マーカー」の表示の根拠がないが,「共在
マーカーを使用しないこと」でプロトタイプを離れる効果が生じる.具体的な発
話場面(時間/場(イマ・ココ))を共有する話手/聞手関係からの乖離によって
場の抽象性が確立し,その抽象性に支えられて,論理性・客観性を追求する場と
なっている.

論文,レポートなど,客観的,普遍的記述を求めるタイプの文の書き方マニュア
ルを見てみよう.論文・レポートでは,明確さ,正確さに基づく客観的な情報の
伝達が目的であり,論理展開を明解に示すこと,情緒的,冗長な表現は避けるこ
とに加え,「具体的な話手や聞手の存在をにおわせない」ことが指向される.そ
れは例えば森山(2003)で「無私の文体」とされるもので,「私的な手紙やメール
と異なり,不特定多数の読者を想定する文章では,筆者の個人的立場に拠った情
報の提示の仕方は避けるべき」といった指導に反映する.このように「話手や聞
手の存在を示さない」ことが,とりわけ注意されている.

(\ex{+1})は,論文・レポートの書き方マニュアルの一例である.話手/聞手の存
在を示さず「無私の文体」を実現する注記とともに,「「です」「ます」は使わ
ず,普通体で書く」「「ね」「よ」のような終助詞も使わない」と「共在マー
カー」の使用を厳に禁じている.

\enumsentence{第一に,\ul{書き手自身の存在を強く感じさせる語・表現や,読
み手に話しかけるような語・表現は,あまり使わない}.例えば,「僕」「俺」
「あたし」のような語は,書き手の属性(どのような書き手であるか)を感じさ
せるので,論文・レポートでは基本的に使わない.\ul{一人称を指すことばはあ
まり使わずに書くのがよい}.「私は〜と思う」のような表現ではなく,「〜と思
われる」「〜と考えられる」「〜する」のような形にする.

また,\ul{「です」「ます」は使わず,普通体で書く}.名詞文では,「これは例外だ」
のような「〜だ」の形よりも,「これは例外である」のような「〜である」の形
を用いたほうがよい.また,\ul{「ね」「よ」のような終助詞も使わない}.\\
\hfill (ケース19 論文・レポートのことば
\footnote{ケーススタディ日本語のバラエティ.おうふう.2005: 114--119.(下線
	は引用者)})}

「書くメディア」に
おける文の「客観的記述」においては,「書き手・読み手の
存在を消す」「共在マーカーを用いない」といった,すなわち「非共在性」が欠
かせないものとなっている\footnote{
	この点歴史的には「言文一致体」の獲得において追求,腐心されたこ
	とであるという(清水1989:33).具体的な伝達場面から離れることで現代の我々
	の論理的文章が成立しているという経緯は興味深い.この点については別稿に譲
	る.}.〈非共在〉の場では,共在マーカーを用いないことにより,時間・場およびそ
れらを共有する聞手(といった〈共在〉の要素)から脱した「抽象的」な場が構
築され,故に「客観性」「記録性」「論理性」が指向されたモノローグ,すなわ
ち(本稿の定義におけるプロトタイプとしての)コミュニケーションでない発話
の場として成り立っていると考えられる.

この(II)の非共在性を共在マーカーの使用によって疑似的な〈共在〉の場にみな
したのが(III)である.「みなす」ことにより言語形式の意味用法が変化する.
その効果は後に確認することにしよう.

\subsubsection{(IV)の非共在性 --- 疑似的な非共在}

一方,(IV)は,「特定の聞手に向けての発話にも関わらず,共在マーカーが(あ
えて)使用されない」という伝達場面である.対面している聞手を意識しないか
のような発話場面は有り難いように思われるが,(IV)には例えば辞令交付などの
場面が相当する可能性がある.典型的には「証する」「命ずる」といった遂行動
詞述語文が現れ,聞手を目の前にしながら共在マーカーは現れない.

\enumsentence{(辞令交付) 4月1日付けで本社営業部への異動を命ずる.}
\enumsentence{(学位記授与) 博士(文学)の学位を授与する.}

「聞手の存在を前提としない言語要素」すなわち「非共在マーカー」というよう
なものは想定しにくいが,「共在マーカーをあえて使用」することがストラテジー
だとすれば,使うべき場面での「非使用」も,「一方的な伝達」としてのストラ
テジーとも考えられる.

\subsubsection{共在マーカー「有」の場面 --- (I)(III)の共在性}

(I)(III)は,ともにどちらも共在マーカーがあるという点では共通するが,その
共在マーカーが単に特定の聞手に向けられた発話であることの表示か,特定の聞
手が設定できない場面で聞手の特定化を指向したストラテジーとして使用される
かという大きな違いがある.(I)はコミュニケーションのプロトタイプで特定の聞
手に対して会話特有の要素としての共在マーカーが用いられる.これに対して
(III)では共在マーカーが聞手の特定化に伴う表現効果を持つ.ここでは(III)の
様相について例を挙げて具体的にみておこう.

新聞・ブログ・エッセイ・教科書など,本来具体的な聞手が存在しないにもかか
わらず「です・ます」「終助詞」「やりもらい」などの共在マーカーが使用され
ることがある.これが(III)の疑似的な〈共在〉である.

(\ex{+1})は大学生向けの教科書であるが,「です・ます体」が採用されている.
書名に「やさしい」とあり出版社のレビューにも「わかりやすく」と明記されて
いるように「わかりやすさ」を目指したものである.(\ex{+2})は新聞の署名記事
であり,書き手としての記者が明示されている.一般の記事には使われない「〜
てあげる」という授受表現が出現し聞手を顕在化する興味深い用例である\footnote{
	待遇表現の場合,聞手に関わりのない話題の人物を待遇すると,話手
	の視点を聞手に同化する作用がある(東他2005).授受表現においても,話手の
	話題の人物に対する視点:エンパシーを聞手に同化すると考えられ,そのことが
	共在マーカーとしての効果を持つと考えられる.}.

\enumsentence{「だろう」(丁寧な言い方では「でしょう」)という助動詞が
\unami{あります}.学校文法では推量を表すと\unami{されています}.よく似た
ものに「ようだ」(
話しことば
では「みたいだ」になることが多い)が\unami{あります}.こちらは推定と呼ば
れることが\unami{多いようです}.

推量と推定,よく\unami{似ていますね}.実際,この2つはどちらも使える場合が
よく\unami{あります}.例えば,次のような\unami{場合です}.\\
\hfill (庵功雄他2003. やさしい日本語のしくみ,くろしお出版.\unskip)}
\enumsentence{堀江容疑者は表舞台から\utensen{退場した}.だが,彼の「功」
の部分を\ul{認めてあげる}ことも\utensen{大切なのではないか}.社会の反発を
浴びつつも,既成概念や閉塞感を打破することは,いつの時代でも\utensen{必要
だからだ}.\hfill (毎日新聞,記者の目,2006.02.01\footnote{
	「堀江バッシングに違和感」=柴沼均(北海道報道部)})}

さらに(II)〈非共在〉であるはずの「書くメディア」において,共在マーカーの
使用によって〈疑似共在〉の場に移行していると考えられる例を見よう.
(\ex{+1})は幼児向けの絵本,(\ex{+2})はウェブ上の求人情報である.どちらも,
不特定多数の聞手を具体化・特定化するような表現となっている.終助詞,「ほ
ら」などの同一視点からの呼びかけ,問いかけ,「です・ます」などの共在マー
カーが使われている.さらに,「くださぁ〜い」といった,話し言葉で効果を持
つ
音声特徴
を記号化した書記の工夫も見られる\footnote{
	書き言葉では字体や飾り,文字色などの工夫も可能である(定延
	2005c).なお,(\ex{+1})(\ex{+2})の\unami{   }は共在マーカを示す.}.

\enumsentence{
 キラリンの はねは,いろだけじゃない\unami{よ}.\\
 もようも かわるんだ\unami{よ}.\\
 \unami{ほら}, くまちゃんの ふくと おんなじ.\\
 うさちゃんの ふくとおんなじ. (「まほうのはねのキラリン」\unskip\footnote{
	チャイルドブック ジュニア,4月号,2004. 4,チャイルド本社.})}

\enumsentence{ずっと愛されるお店になるため,名物スタッフが必要\unami{で
す}! 皆さんも親しみの持てる名物スタッフにならない\unami{?} キレイなお店
で仲間もたくさん\ul{出来ちゃう}\unami{よ}!みんな集合して\ul{くださぁ〜
い}! (フロムエー・ナビ)\unskip\footnote{
	{\tt http://www.froma.com/}(2006.3.10アクセス)}}

いずれも話し言葉的文体などといわれるものの,これらの言語要素が書き言葉で
出現する際の表現効果や用法,どのように発現するのかについては詳細に記述し
た研究はない.話し言葉的という観点だけでは,(\ex{0})のように「です・ます」
が「心的距離--遠」を示す敬語ではないばかりか,「くださぁ〜い」のようなく
だけた話し言葉的要素と連動して用いられ,親しみをも感じさせる事実を説明で
きない.(I)〈共在〉と(III)〈疑似共在〉の共通点と相違点を整理することで明
確に示すことができる.

本稿では,(I)(III)の〈共在〉が「場」としての「共在」だけでなく,共在性の
表示となる,形を持つものとしての共在マーカーの使用によって構築されるもの
と考える.ただし,(I)と(III)では「場」における共在性の違いによって言語形
式の「役割」が異なるのである.

\section{「です・ます」諸用法発現のメカニズム}\label{meca}

\ref{submodel}で提案したモデルに基づき,「です・ます」の諸用法を,場面の
設定と関連づけて分析する.

\subsection{(I)〈共在〉{\unskip}…心的距離の表示}\label{Ikyori}

まず,(I)〈共在〉の場における「です・ます」を見る.ここでは,聞手の条件
によって「共在性」が高であるので,共在マーカーはその本質的働きをそのまま
表すと考えられる.〈共在〉では次のような場合に用いられる.

\eenumsentence{
 \item (上司に:上)コピーを取ってき\unami{ます}.
 \item (初対面:疎)はじめまして,佐藤\unami{です}.
 \item (講演:公)今日はアンチエイジングについてお話をしたいと思い
 \unami{ます}.
 \item (夫婦喧嘩:遠)一体何を隠したというの\unami{です}か?
}

(\ex{0}a)は会社で上司に,(\ex{0}b)は初対面の人物に,そして(\ex{0}c)は「公」
の場面で聴衆に,(\ex{0}d)は喧嘩相手としての配偶者に対しての待遇表現である.
「丁寧語」と説明されてきたこれらの用法は,夫婦喧嘩の冷戦状態の際に使われ
る「です・ます」なども含め,本質的に心的な「距離」の表示と整理できる\footnote{
	本質的働きとした「心的距離」については滝浦(2005b)参照.この本質
	的働きは聞手の存在を何らか前提するという共在マーカーの定義とも矛盾しない.
	また,聞手との心的距離の関係が待遇表現の本質であるという考え方は東(2004)
	東他(2005)にも示されている.研究書ではないが橋本(2005)も同様である.}.
「心的距離の表示」は,敬語(丁寧語)に直結するものではあるが,敬意のみ
を表すのではなく\footnote{
	「です・ます」に限らず「敬語」が表すさまざまな運用上の意味につ
	いては,南(1987)に詳しい.},
忌避・疎遠・公などを統一的に表示する一つの軸である.

〈共在〉にあたる対面コミュニケーションの場で「です・ます」が採用されると,
その「話手/聞手の心的距離」が表示される.これを本稿のモデルでは「です・
ます」の本質的機能の発現としての用法と見る.

\enumsentence{(大河内=教授,財前=助教授,里見=財前と同期の助教授)
 \begin{description}
  \item[大河内:] 入りなさい.
  \item[財前:] 失礼\unami{いたします}.
  \item[財前:] (里見を見て) \unami{驚いたな},君も\unami{いたのか}.
  \item[大河内:] 困るかね?
  \item[財前:] いいえ.基礎講座でともに大河内先生に教わったころを
	     \unami{思い出します}.
 \end{description}
  \hfill(ドラマ「白い巨塔 第一部」\unskip\footnote{
	「白い巨塔 第一部 DVD-BOX」ポニーキャニオン.})}

財前助教授は,大河内教授に対しては「失礼いたします」「思い出します」のよ
うに「です・ます」を使い,大河内教授との心的距離を「遠」に置くが,同期の
里見に対しては,「驚いたな,君もいたのか」のように「非です・ます」を使い,
心的距離が「近」であることを示す.対面対話や特定の相手に向けた手紙文など
では「です・ます」が「心的距離」の表示となり,他の敬語(尊敬語・謙譲語)
などとも連動して用いられる.

\subsection{(III)〈疑似共在〉{\unskip}…{\unskip}「節度ある」聞手の特定化}

次に(III)〈疑似共在〉の場での「です・ます」を見る.書くメディアのような,
話手から見て聞手が特定できないコンテクストの場合,\ref{tuikyu}で述べた通
り,一般的には「だ・である体」が好まれ「です・ます体」は推奨されない選択
肢の一つとなっている.しかしそこで「です・ます」が用いられると,共在マー
カーは(III)〈疑似共在〉の場の構築を指向した話手のストラテジーとなり,疑
似的な聞手の顕在化という役割が前面に出ると予測できる.

さきに挙げた例をもう一度見よう.

\eenumsentence{
 \item イラクに派遣された陸上自衛隊は,既に相応の責任を果たした.基本計
 画の修正により派遣期間は一年延長されたが,今から撤収の準備に着手
 \utensen{すべきだ}.((\ref{iraq}a)再掲)
 \item イラクに駐留する自衛隊は,しばらく復興支援を続けることに\unami{な
 りました}.でも,いずれ治安がよくなれば政府開発援助(ODA)などに
 \unami{出番が回るはずです}.((\ref{iraq}b)再掲)}

例(\ex{0}b)のような,書くメディアでは推奨されないはずの「です・ます」が
採用された文章では,「わかりやすい」「やさしい」ニュアンスが生じている.
教科書や子供向けの文章,非日本語話者向けの文章で「です・ます体」が採用さ
れているのは,「です・ます体」の選択が「わかりやすい雰囲気」を演出する意
図に合った文体として確立しているためといえる\footnote{
	これについて
	東他(2006)
	では,メディア論(原田2005)を援用し,文
	字という媒体の制限によって表情や繰り返し確認,音声による強調などの手段が
	制限されるために「書くメディア」において採られる代替手段と位置づけている.}.
「です・ます」が,ストラテジーとして共在マーカーの機能を果たしているこ
とから生じる効果である.話手と聞手が顕在化し,元来は関係のなかったところ
に関係が生じることによって「近づく感じ」\unskip\footnote{
	「「近づく」感じがする」というのは,あくまで話手の感覚である.
	一般的には「です・ます」体について「柔らかい」「優しい」「わかりやすい」
	という解釈は聞手側においても成り立ちうる.ただしそれは,このストラテジー
	が(例えば「先生」口調,子ども向けのやさしい・わかりやすい解説のモードと
	して)社会的に成立していること(すなわち母語話者として,話手側のストラテ
	ジーを知っていること)によって成り立つ解釈と考える.「近づく感じ」は,話
	手側が装い,その装いがモードとして共有される場合に限られるのである.例え
	ば,外国人向けの災害時情報で使用されている「やさしい日本語」による表現で
	は「です・ます」体が採用されることが多いが,これはカタカナ語を含め難解な
	語彙を避け,単文で構成するといった配慮とともに,ほとんどの日本語初級教科
	書で導入されている「〜があります」「〜てください」「〜ないでください」な
	どの文型にあてはめて表現することで「わかりやすく」情報を伝えるためであり
	(佐藤(2000)参照)
	,「です・ます」が,非母語話者にとっての「やさしさ」に直結するものではな
	い.}が生まれる.
これが演出されたもの
としての「わかりやすさ」の正体であろう.相手を敬して遠ざけるはずの「です・
ます」が,共在性のない伝達場面で使用されると却って「近づく」感じの演出と
なる
,「です・ます」が共在のストラテジーとして機能していることから説明可能と
なる.

\def\boutenchar{}
\def\bou#1{}
 \def\getlength#1{}
 \def\dot#1{}

一方,「です・ます」を他の共在マーカーと比較したときには,\bou{相対的}に
「遠」なる聞手を顕在化させる.それぞれの共在マーカーによってどのような聞
手が顕在化するのか見てみよう.(\ex{+1}a)は新聞の社説,(\ex{+1}b)は幼児向
け絵本の例である.

\eenumsentence{
 \item 新聞の社説\\ 
 日本はことしから人口減少時代に\unami{入るかもしれません}.いまは悲観論ば
 かりが\unami{先行しています}が,{\unskip}総選挙では真の豊かさを実感できる社会にす
 る論争を\unami{期待します}.\\
 \hfill (中日新聞社説,2005.8.28)
 \item 絵本\\
 ここが どこだか \utensen{わかる?}\\
 ひまわりが こんなに いっぱい さいて いて,\\
 まるで ひまわりの うみみたいでしょ!\\
\hfill (『よいこのがくしゅう』第44巻第4号,学習研究社,2005.7)
}

「です・ます」は,心的距離が「遠」なる聞手を顕在化することで,話手と聞手
の「節度ある」関係を導くため,公的な場においての使用に堪える.対して,
(\ex{0}b)の,「わかる?」「うみみたいでしょ!」のような問いかけや終助詞は,
共在マーカーとして心的距離が「近」の話手/聞手関係を顕在化する.したがって,
これらの共在マーカーは,親密,私的な場において使用されている.これは「で
す・ます」と他の共在マーカーの相対的な関係の反映である.

共在マーカーにより聞手を顕在化する例は,ほかにも見られる.(\ex{+1})は求人
誌から引用したものである.求人誌はその性格上,不特定多数の読者に向けられ
るが,どの共在マーカーを使うかによって,求める人物像が浮かび上がってくる.

\eenumsentence{
 \item お客様からなどの電話対応やパソコンでの文書作成などを\unami{お願い
 します}.
 \item 閉店後のお店の清掃を\unami{お願いします}.広いお店なのでいい運動
 に\unami{なりますよ}.
 \item いろんなスポーツ用品やアメカジに囲まれて,楽しく\utensen{バイトし
 よう}!!\\
\hfill 
(『タウンワーク名古屋東部・瀬戸周辺版』12/22号vol.~1--2, 2005)
}

共在マーカーの使用が,話手/聞手を顕在化させ,関係を作り出すものであるこ
と,その「関係」「近づき方」が共在マーカーの使い分けによって異なって実現
されていることが見て取れる.この違いには,作り出された〈疑似共在〉(III)
であっても,(I)の場での各要素の本質的機能の差が反映していると考えられる.
一度〈共在〉の場が作り出されると,その場では,聞手が特定である(I)と同様,
「です・ます」か「非です・ます」かによる関係の違いが生じるのである.これ
はあくまで共在マーカー間の相対的な関係の反映であり,「です・ます」自体の
問題ではないと考える.

\subsection{感情・態度の表示となるメカニズム}

次に,「です・ます」の出現が,話手と聞手の関係のあり方の変化を表し,「感
情・態度の表示」となる場合を説明する.

本稿のモデルでは,話手と聞手の関係変化について,二通りの変化が想定できる.

一つは,聞手が不特定・多数・抽象的であるような共在性の低い場で,話手が
「です・ます」を使用し,そもそも存在しなかった,話手と聞手の関係を生じさ
せるという関係変化,すなわち〈非共在〉(II)から〈疑似共在〉(III)へのシフ
トである.

そしてもう一つは,〈共在〉(I)および〈疑似共在〉(III)の場において「非です・
ます」がスタイルとなっている中で「です・ます」を,また「です・ます」がス
タイルとなっている中で「非です・ます」を使用することによって,話手と聞手
の関係を変化させるものである.

\subsubsection{〈非共在〉から〈共在〉へのシフト}

まず,(II)の場において「です・ます」を使用し,〈非共在〉から〈疑似共在〉
へと変化させるタイプを見よう.(\ex{+1})は「だ・である体」で書かれた論文で
ある.最後の謝辞で「です・ます」が出現している.また(\ex{+2})は新聞所載の
「だ・である体」のエッセイの末尾に「〜ますよ」が出現している.

\enumsentence{付記\hspace{3mm}インフォーマントのお二人,またインフォーマ
ントを紹介してくださった
××氏\footnote{
	用例(\ex{0})の「××氏」については原典で個人名のため記号に置き換えた.}に
\unami{お礼申し上げます}.本稿は,
国語学会2000年度秋季大会で発表した内容を大幅に改訂したもの\utensen{であ
る}.発表の際,\unami{ご意見・ご教示くださった}皆様に\unami{お礼申し上げ
ます}.(国語学.52(3), p.~44)}
\enumsentence{では,その高揚感をあおってくれたものは\utensen{何だったろ
う}.そこで思い浮かぶのが各局の中継の\utensen{テーマ曲だ}.NHKなら古関裕
而作曲の「スポーツ・ショー行進曲」(♪チャンチャチャンチャチャンチャチャ
ララ……),(中略)TBSならばあの曲(ツッチャッチャッチャッ,チャッチャラ
ラッチャッチャーン).どれも今でも口ずさめるテーマ曲\utensen{ばかりだ}.
(中略)いかんせんわかりづらい表記になってしまったが,ツーだのチャラだの
口ずさんで\ul{みられよ}\footnote{
	用例(\ex{0})には末尾の「〜ますよ」の直前に命令形「みられよ」が出現してい
	る.命令形も,命令の動作内容を達成能力のある聞手の存在を前提とする点で共
	在マーカーの一つと考えられる.「だ・である体」には一般に命令形は出現しに
	くく,「〜たいものだ」「べきだ」などの義務表現で代用されている.}.
きっと\unami{思い出しますよ}.
\hfill (毎日新聞夕刊,2006.4.8\footnote{
	やくみつる「週間テレビ評「プロ野球中継」」})}

聞手が不特定多数の(II)〈非共在〉の発話においては,共在マーカーは現れない.
そのような場で,共在マーカー「です・ます」が出現すると,「です・ます」は
聞手を顕在化させ,異なる伝達場面を構築することとなる.すなわち,瞬間的に
〈非共在〉(II)を〈疑似共在〉(III)に変化させることで,話手の態度の表示とな
ると考えられる.(\ex{-1})のように
「××氏」
「皆様」という個別具体的かつ特定の聞手を顕在化する場合も,(\ex{0})のよう
に不特定のままで集合を絞り込むように具体化・顕在化する場合もある.興味深
いのは(\ex{-1})で「ご教示くださった」「申し上げる」といった敬語が用いられ
ていることである.論文の読者が不特定多数であることに変わりはなく,名指し
された
「××氏」
が読んでいるとは限らないが,共在マーカーを使用すること
で発話を(III)〈疑似共在〉とし,特定化した聞手に待遇表現を用いているの
だと考えられる.共在マーカーの使用は,話手の〈共在〉指向ストラテジーであ
る.そのストラテジーにおいて,敬語の使用も「です・ます」の出現も連動して
可能になっていると考えられる.〈疑似〉的に特定化した聞手に対しても,その
個別性・具体性・特定性が高ければ高いほど,〈共在〉での聞手同様に扱うこと
ができ,待遇的に位置づけることも可能になる.不特定多数の読者というコンテ
クストが公の場として,あるいは第三者の存在として敬語の出現に影響している
可能性もある\footnote{
	「第三者の存在」(バフチン1952--1953[1988])については,対話の大前
	提と考える.また,Brown and Levinson (1978, 1987), Bell (1984), Clark (1993)な
	どで,多様なaudienceの存在によるコミュニケーションへの影響を示唆したモデ
	ルが提案されている.本稿においては扱えないが,傍聴者を含めた伝達場面の構
	成要素とそのあり方,さらに共在性との関わりの整理は今後の課題である.}.

以上,共在マーカーの使用によって〈非共在〉から〈疑似共在〉へのシフトによ
り,話手の感情・態度の表示となる例を見た.

\subsubsection{〈共在〉における心的距離の変化}

次に,〈共在〉の場において話手と聞手の心的距離を変化させるタイプを見てみ
よう.話手と聞手が「共在」もしくは「疑似共在」している(I)および(III)の場
において,「非です・ます」スタイルとなっている中に「です・ます」が出現す
ると,\ref{hyoji}で見たように感情・態度の表示となる.これは,「です・ます」
の使用により聞手が「近」から「遠」になる,という関係変化が起こり,それに
よって感情,態度を表すことになるものと考える.

(\ex{+1})の\unami{  }部では「です・ます」の出現により,聞手である朋美と
の心的距離を瞬間的に遠くし,察しの悪い朋美に対する「皮肉」のような感情を
表す.(\ex{+2})では,(III)〈疑似共在〉の場で「起きたら雨だったよ〜,しか
たないねー」のように,「非です・ます」の「近」の関係で共在していた聞手
(ブログ読者)に対し,「夫婦揃ってお腹壊しました」と「です・ます」を出現
させることで,改まった態度の表示となり「照れ隠し」といった感情が示される.

\enumsentence{(次郎(32歳):父が営む児童養護施設に仮住まいの身,朋美
(27 歳):児童養護施設の保育士)
\begin{description}
 \item[次郎:] ねえねえ? どうして\utensen{朋美先生なの}?
 \item[朋美:] はっ?
 \item[次郎:] 何で保育士に\unami{なったんですか}?
 \item[朋美:] こんなときに語れるほど簡単じゃありません.
 \item[次郎:] \utensen{あっそう}.
 \item[朋美:] どうしてですか?
 \item[次郎:] いや\utensen{鈍いから}.(せきばらい)
\end{description}
\hfill (ドラマ「エンジン」\unskip\footnote{
	「エンジン DVD-BOX」ビクターエンタテインメント})}
\enumsentence{今日は,先週いけなかったゴルフ(ハーフだけ)に行こうとしてた
のに,起きたら\utensen{雨だったよ}〜 {\tt (T\_T)}
 \utensen{しかたないねー}.(中略)今日は実は結婚記念
日.何が食べたいか協議の結果,餃子の王将に決定.でも年取ったせいか(油が
きつかったのかな?)夫婦揃ってお腹\unami{壊しました}.来年は,もうちょっとヘルシー
なもの食べよう…\\
\hfill (「しろうと女房の厩舎日記」\unskip\footnote{
	{\tt http://blog.livedoor.jp/yukiko.miyamotol/}(2006.1.30アクセス)})}

逆に「です・ます」スタイルの中に,「非です・ます」を出現させると,聞手と
の関係が「遠」から「近」に変化する.いずれも,〈共在〉の場において「です・
ます」と「非です・ます」を瞬間的にスイッチすることで,「遠--近」の関係を
変化させ感情を示すものである.

\enumsentence{(里見=助教授,柳原=医局員,君子=看護師)
 \begin{description}
  \item[柳原:] 財前先生は?
  \item[君子:] オペに\unami{入りましたよー}.
  \item[柳原:] えっ?
  \item[君子:] (柳原に)今日みたいな大事なオペに\utensen{遅れるわけな
	     いじゃない}.
  \item[里見:] 大事なオペってー?
  \item[君子:] ご存じない\unami{んですか?} 患者,大阪府知事の鶴川幸三
	     \unami{なんです}.
 \end{description}
\hfill(ドラマ「白い巨塔 第一部」\unskip\footnote{
	「白い巨塔 DVD-BOX 第一部」ポニーキャニオン})}

君子と柳原には看護師と医師という立場の差があり,君子は通常,柳原に対して
「オペに入りましたよー」のように「です・ます」を使うが,「大事なオペに遅
れるわけないじゃない」と「非です・ます」を出現させると,心的距離が「近」
となり,職業上の立場差が示されなくなる.その結果,医師と看護師という立場
の差からは表れ得ない,柳原に対する「あきれ」といった感情が示されている.

また,「です・ます」が,特定のキャラクターと結びついた「役割語」として使
われる場合も,話手と聞手の関係変化によると考えられる.「です・ます」を使
うことで,普段の話手と聞手の関係とは異なる「遠」の関係の表示となり,話手
は,自分とは違うキャラクターに変身する.

\enumsentence{(のび太が思わぬ品物を手に入れて悦に入る場面)\\
のび太:「これは\unami{たいへんなものですよ}.」\\
\hfill (ドラえもん,7巻,小学館,(定延2005d:128))}

ある種の役割やキャラクターを表す場合,終助詞の類がその中心的な役割を果た
す.役割やキャラクターを表しうる共在マーカーの使い分けが,話手の変身の演
出,すなわち聞手との関係変化につながる\footnote{
	キャラ助詞については,音声特徴の多様性を利用することができない
	書くメディアでより多く使用される(定延2005a, 2005c).またウェブ上の匿名
	ブログではあるが,キャラ助詞を書き言葉で使用すると「わかりやすくなる」と
	いう指摘がある(「それだけは聞かんとってくれ」「第6回猫なんだニャ」{\tt
	http://www.sorekika.com/dame.jsp?idx=006} (2006.4.2アクセス)).}.

以上のような〈共在〉での心的距離の遠近の操作は,小説等のセリフで効果的に
利用され表現効果を発揮する.(\ex{+1})は,小説の中の嫁--舅の口喧嘩の中での
舅のセリフである.

\enumsentence{「……\ul{なんじゃい},二言めにはスジだの金だのといい
\utensen{くさって}.ああ,どうせわしは得手勝手な\ul{爺いじゃ}.震災からこ
のかた遊びくらした道楽もんの,憎まれものの,邪魔っけな\ul{爺いじゃ}.よう
わかってるよそんなことは.お前はえらい,\unami{しっかり者です}.立派な
\unami{女子さんです}.わしは\ul{バカじゃ}.失言もし物忘れも\utensen{する}.
この年だ,文句は年に\utensen{いうてくれよ}.……ふん,鬼みたいな顔で睨み
\utensen{くさって}.わしは,お前みたいなこわい女とはよう\unami{暮しませ
ん}.この年になってピリピリチリチリ暮すなんて,わしは\ul{ごめんじゃ}.あ
あ,\unami{まっぴらです}.わしを,Sへ
\utensen{やってください\\}
」
\hfill (山川方夫1975,海岸公園.新潮文庫.p.~23)\footnote{
	用例(\ex{0})は定延利之氏との個人談話による.}}

「わし」「〜じゃ」といった老人の役割語とともに「です・ます」と「非です・
ます」が混在して現れるセリフによって,「老人」が「感情むき出し」でいじけ
てみせていることが見事に表されている.

\subsubsection{感情・態度の表示となるメカニズム}

感情の表示については,従来一様にスタイルシフトの効果と説明されてきたが,
本稿の枠組みでは二通りに整理した.一つは伝達場面の転換というべき「〈非共
在〉から〈共在〉への変化」によって,話手/聞手が顕在化し,それと同時に共
在という関係が生じることから,「近づく」感じの感情・評価・態度の表示とな
るものである.もう一つは〈共在〉における,ある一定のスタイルの中で異なる
スタイルが出現する場合であり「聞手との関係変化」の表示,すなわち「遠ざか
る・改まる」といった感情・評価・態度の現れとなるものである.この整理によ
り,どのような場合に「親しみ」「わかりやすさ」「仲間意識」といった近づく
方向の感情・態度の表示になり,どのような場合に「卑下」「遠慮」「皮肉」
「専門家意識」「照れ隠し」といった遠ざかる方向の感情・態度の表示になるの
かといったことも説明可能となった.

\section{まとめと今後の課題}\label{matome}

本稿では,「です・ます」の諸用法を概観し,その分化を伝達場面の構造モデル
に照らし合わせることで包括的な説明を試みた.そして,「です・ます」の諸用
法とされてきたものは,「です・ます」が持つ「話手と聞手の心的距離の表示」
という本質と,伝達場面における〈共在〉/〈非共在〉性およびそれに基づいた
話手のストラテジーによって説明できることを明らかにした.

これは,コミュニケーションのあり方のさまざまを,伝達場面の構造として文法
記述に生かす立場であり,本稿はそのような立場の有効性・発展性を主張するも
のである.

言語形式が伝達場面の変化によって意味・機能を変えるという事実は,本稿のモ
デルの言語形式の機能変化を含む言語の動態を説明する枠組みとしての有用性を
も示唆するものである.「です・ます」に加え,個別の共在マーカーの分析を積
み重ねることでその有効性を補強すると同時に,伝達場面の構造モデルそのもの
の精緻化,他の語用論的条件やコミュニケーションモデルと本稿の伝達場面の構
造モデルとの関係づけなどは,全て今後の課題である.

\acknowledgment

本稿は,執筆者一同による次の発表論文および口頭発表に基づいて大幅に整理し
加筆修正したものである.これらに対する多くのご意見および査読者の貴重なご
指摘,照会に依るところが大きい.記して謝意を表する.

\begin{itemize}
\item 北村雅則他(2006).``伝達場面の構造と「です・ます」の諸機能''.言語処理学
      会第12回年次大会発表論文集,pp.~1139--1142.
\item 東弘子他.伝達場面の構造と言語形式—「です・ます」の諸機能と話手・
      聞手の共在性を手がかりに—.名古屋言語研究会第32回例会2006.3.18(於:名
      古屋大学)
\end{itemize}


本研究は平成17年度科学研究費16720108(若手研究B:研究代表者・東弘子)の
研究成果の一つである.



\begin{thebibliography}{3}

 \item 
東弘子(2004). ``「話題の人物」の待遇を決定するシステム.'' 名古屋大学国語国文学, \textbf{95}, pp.~103--192.
 \item 
東弘子,加藤淳,宮地朝子,江口正(2005). ``マスメディアにおける敬語使用の変異と聞手の感情に及ぼす効果.'' 言語処理学会第11回年次大会(NLP2005)発表論文集, pp.~458--461.
 \item 
東弘子,加藤良徳,北村雅則,石川美紀子,加藤淳,宮地朝子(2006).``「書くメディア」にあらわれる「です・ます体」のわかりやすさ.'' 言語処理学会第12回年次大会(NLP2006)発表論文集, pp.~24--27.
 \item 
菊地康人(1994). 敬語. 角川書店(講談社学術文庫より再刊1997).
 \item 
菊地康人(1996). 敬語再入門. 丸善ライブラリー.
 \item 
木村大治(1996). ボンガンドにおける共在感覚. 叢書・身体と文化2  コミュニケーションとしての身体. 大修館書店, pp.~316--344.
 \item 
金水敏(2003). ヴァーチャル日本語 役割語の謎. 岩波書店.
 \item 
金水敏,田窪行則(1996). ``複数の心的距離による談話管理.'' 認知科学, \textbf{3}(3), pp.~58--74.
 \item 
鈴木睦(1997). 日本語教育における丁寧体世界と普通体世界. 視点と言語行動, 田窪行則編, くろしお出版, pp.~45--76.
 \item 
定延利之(2003). ``体験と知識—コミュニカティブストラテジー.'' 国文学 解釈と教材の研究, \textbf{48}(12), pp.~54--64.
 \item 
定延利之(2005a). ささやく恋人, りきむレポーター—口の中の文化. 岩波書店.
 \item 
定延利之(2005b). ケース17 話しことばと書きことば(音声編){\unskip}. ケーススタディ 日本語のバラエティ, 上野智子・定延利之・佐藤和之・野田春美編, おうふう, pp.~102--107.
 \item 
定延利之(2005c). ケース18 話しことばと書きことば(文字編){\unskip}. ケーススタディ 日本語のバラエティ, 上野智子・定延利之・佐藤和之・野田春美編, おうふう, pp.~108--113.
 \item 
定延利之(2005d). ケース21 マンガ・雑誌のことば. ケーススタディ 日本語のバラエティ, 上野智子・定延利之・佐藤和之・野田春美編, おうふう, pp.~126--133.
 \item 
定延利之(2005e). ``「表す」感動詞から「する」感動詞へ.'' 言語, \textbf{34}(11), pp.~33--39.
 \item 
佐藤和之(2000). ``「災害時の外国人用日本語」マニュアルを考える—災害時情報と外国人居住者.'' 日本語学, \textbf{19}(2), pp.~34--45.
 \item 
清水康行(1989).文章語の性格. (講座日本語と日本語教育5)日本語の文法・文体(下){\unskip}, 山口佳紀編, 明治書院, pp.~26--45.
 \item 
滝浦真人(2002). ``敬語論の“出口”—視点と共感と距離の敬語論に向けて—.'' 言語, \textbf{31}(6), pp.~106--117.
 \item 
滝浦真人(2005a). ``日本社会と敬語像—「親愛の敬語」を超えて」{\unskip}.'' 言語, \textbf{34}(12), pp.~36--43.
 \item 
滝浦真人(2005b). 日本の敬語論—ポライトネス理論からの再検討. 大修館書店.
 \item 
谷泰編(1997). コミュニケーションの自然誌. 新曜社.
 \item 
永野賢(1966). ``「です」「ます」体の文章と敬語—敬体の文章における敬語.'' 国文学 解釈と教材の研究, \textbf{11}(8), pp.~108--113.
 \item 
橋本治(2005). ちゃんと話すための敬語の本. ちくまプリマー新書.
 \item 
バフチン,ミハイル(1952--1953). ことばの諸ジャンルの問題.(邦訳:ことばのジャンル.ことば・対話・テキスト, ミハイル・バフチン著作集8, 新谷敬三郎訳(1988), 新時代社, pp.~115--189.)
 \item 
原田悦子(2005).メディアと表現様式の変化;認知工学の立場から. 講座社会言語科学2 メディア, ひつじ書房, pp.~118--133.
 \item 
日高水穂(2004). ``普通体と丁寧体の混在による表現効果.'' 言語, \textbf{33}(11), pp.~118--119.
 \item 
日高水穂(2005). ケース11 ことばの切りかえ. ケーススタディ 日本語のバラエティ, 上野智子・定延利之・佐藤和之・野田春美編, おうふう, pp.~66--71.
 \item 
水谷雅彦(1997). 伝達・対話・会話—コミュニケーションのメタ自然誌へむけて—. コミュニケーションの自然誌, 谷泰編. 新曜社, pp.~5--30.
 \item 
南不二男(1987). 敬語. 岩波新書.
 \item 
三宅和子,岡本能里子,佐藤彰(2004).メディアとことば1. ひつじ書房.
 \item 
メイナード,K・泉子(1991). ``文体の意味—ダ体とデスマス体の混用について—.'' 言語, \textbf{20}(2), pp.~75--80.
 \item 
メイナード,K・泉子(2001a). ``心の変化と話しことばのスタイルシフト.'' 言語, \textbf{30}(7), pp.~38--45.
 \item 
メイナード,K・泉子(2001b). ``日本語文法と感情の接点—テレビドラマに会話分析を応用して—.'' 日本語文法, \textbf{1}(1), pp.~90--110.
 \item 
森山卓郎(2003). コミュニケーション力をみがく—日本語表現の戦略—. NHK出版.
 \item 
安川一(1991). ゴフマン世界の再構成—共在の技法と秩序. 世界思想社.
 \item 
山森良枝(1997). 終助詞の局所的情報処理機能. コミュニケーションの自然誌, 谷泰編. 新曜社, pp.~130--172.
 \item 
Austin, J. L.(1962). How to Do Things With Words. Oxford University Press: Oxford, England. 
 \item 
Bell, A. (1984). ``Language style as audience design.'' \textit{Language in Society}, \textbf{13}, pp.~145--203.
 \item 
Brown, P. and Levinson, S. (1987 [1978]). Politeness: Some Universals in Language Usage.  Cambridge University Press: Cambridge, England.
 \item 
Clark, Herbert H. and Carlson, Thomas B. (1982). ``Speech acts and hearers' beliefs.'' In N. V. Smith (Ed.), Mutual knowledge. New York: Academic Press, pp.~1--45.
 \item 
Clark, Herbert H. (1992). Arenas of Language Use. University of Chicago Press, Tx. 
 \item 
Clark, Herbert H. and Marshall, C. R. (1981). ``Definite Reference and Mutual Knowledge.'' In Joshi, A. K., Webber B. L. and Sag I. A. (Ed.), ``Elements of Discourse Understanding.'' Cambridge University Press: Cambridge, England. 
 \item 
Goffman, Eaving (1963). Behavior in Public Places: Notes on the Social Organization of Gathering. New York: The Free Press.
 \item 
Grice, H. P. (1975). ``Logic and conversation.'' In Cole, Peter, and Morgan J. L. (Ed.), Syntax and semantics: Speech acts. Vol.~3. New York: Academic. pp.~41--58.
 \item 
Searle, J. R. (1969). Speech Acts: An Essay in the Philosophy of Language. Cambridge University Press: Cambridge, England. 
 \item 
Sperber, D. and Wilson, D. (1986). Relevance: Communication and Cognition, Harvard University Press.
 \item 
Tannen, Deborah (1980). ``Spoken/Written Language and the Oral/Literate Continuum.'' \textit{Proceedings of The Sixth Annual Meeting of The Berkley Linguistics Society}, University of California, Berkley, pp.~207--218.
 \item 
Tannen, Deborah (1982). ``The Oral/Literate Continuum in Discourse.'' Spoken and Written Language: Exploring Orality and Literacy, Norwood, NJ: ABLEX Publishing Corp., pp.~1--16.

\end{thebibliography}

\begin{biography}
\bioauthor{宮地 朝子}{
2001年名古屋大学大学院文学研究科博士課程後期課程修了,博士(文学),現在
名古屋大学大学院文学研究科講師,日本言語学会,日本語学会,日本語文法学会,日
本語教育学会各会員.
}
\bioauthor{北村 雅則(正会員)\unskip}{
2005年名古屋大学大学院文学研究科博士課程後期課程満期退学,同年博士(文学)取得,現在
独立行政法人国立国語研究所特別奨励研究員,日本語学会,日本語文法学会各会員.
}
\bioauthor{加藤  淳}{
2006年愛知県立大学外国語学部英米学科卒業,
現在名古屋大学大学院文学研究科博士課程前期課程在学中.
}
\bioauthor{石川美紀子}{
2002年名古屋大学大学院文学研究科博士課程前期課程修了,
現在名古屋大学大学院文学研究科博士課程後期課程在学中,日本語学会会員.
}
\bioauthor{加藤 良徳}{
2002年名古屋大学大学院文学研究科博士課程後期課程満期退学,
博士(文学),現在静岡英和学院大学人間社会学部講師,日本語学会会員.
}
\bioauthor{東  弘子(正会員)\unskip}{
1997年名古屋大学大学院文学研究科博士課程後期満期退学,同年博士(文学)取得,現在愛知県立大学外国語学部准教授,日本言語学会,日本語学会,日本語文法学会各会員.
}
\end{biography}


\biodate

\end{document}
