    \documentclass[japanese]{jnlp_1.3c}
\usepackage{jnlpbbl_1.1}
\usepackage[dvips]{graphicx}

\Volume{14}
\Number{3}
\Month{Apr.}
\Year{2007}
\received{2006}{4}{20}
\revised{2006}{8}{11}
\accepted{2006}{9}{5}

\setcounter{page}{81}

\jtitle{語彙間の主観的な類似度による感動語の分類}
\jauthor{大出 訓史\affiref{Author_1} \and 今井  篤\affiref{Author_1}
	\and 安藤 彰男\affiref{Author_1} \and 谷口 高士\affiref{Author_2}}
\jabstract{
日常生活の様々な体験において,その体験の素晴らしさを表現する言葉として,『感動』という言葉がしばしば用いられる.感動とは,『美しいものや素晴らしいことに接して強い印象を受け,心を奪われること』(大辞林\cite{Book_103})であり,体験に対する肯定的な評価であると共に,記憶の定着や感情の喚起を伴った心理状態の大きな変化である.感動を喚起する対象としては,マスメディアが提供するドラマや映画,音楽などの割合が高いとされている\cite{Web_401}.しかし,感動という心理状態の定義については,研究者の中でも曖昧である.\par
我々の目的は,放送番組の品質評価,特に音の評価に,『感動』をキーワードとした評価指標を導入するために,感動という心理状態を明確にすることにある.まず,アンケート調査を実施し,感動という言葉で表現される体験と,感動を表現する言葉(以下,感動語)を収集した.次に,感動語同士の一対比較による主観評価実験を行い,感動語から連想される心理状態の類似度を求めた.他の感動語との類似度によって表現される類似度ベクトルの距離に基づいて,感動語の分類を行った.その結果,感情とは,特定の感情そのものではなく,大きく心が動かされたという体験に対して,肯定的な印象を持っているという個々の心理状態の総称であり,感動という心理状態が,感動の対象と感情の種類,感情の動きの組み合わせによって分類できることが分かった.
}
\jkeywords{感動,感情,評価,語彙分類}

\etitle{Classification of Words Expressing the Emotional Affects Based on the Subjective Similarity Measures}
\eauthor{Satoshi Oode\affiref{Author_1} \and Atsushi Imai\affiref{Author_1}
	\and Akio Ando\affiref{Author_1} \and Takashi Taniguchi\affiref{Author_2}} 
\eabstract{
Some wonderful experiences are expressed in Japanese by the word ``Kandoh'' which could be translated into English as ``emotional-affect''. The ``emotional-affect'' is defined by the dictionary as ``making people have strong feeling in facing beautiful or wonderful things''. According to the public-opinion poll by Mitsubishi Research Institute, Inc. in 2003, the mass media is one of the major subjects evoking emotional-affect. But there have been no studies that tried to define a mental state of emotional-affect.\par
The purpose of our study is to describe the emotional-affect for evaluation of broadcast programs. First, we addressed a questionnaire about emotional-affect and picked up words which expressed the situation of emotional-affect from the answers. Furthermore, we calculated a distance of the word by similarity measures based on the subjective evaluation. The obtained results are 1) the emotional affect could be classified into a few main groups, and 2) individual groups were classified by some factors including the object and kind of emotion, not the emotion itself. These results suggest that emotional-affect is the general term of the conditions of mind such as affirmative impressions and uncontrollable minds due to a very strong passive and compulsory stimulus.
}
\ekeywords{Emotional affect, Feeling, Evaluation, Vocabulary analysis}

\headauthor{大出,今井,安藤,谷口}
\headtitle{語彙間の主観的な類似度による感動語の分類}

\affilabel{Author_1}{日本放送協会放送技術研究所}{
	Science \& Technical Research Laboratories, Japan Broadcasting Corp.}
\affilabel{Author_2}{大阪学院大学情報学部情報学科}{
	Dept. of Informatics, Faculty of Informatics, Osaka Gakuin Univ.}



\begin{document}
\maketitle



\section{はじめに}

日常生活の様々な体験において,その体験の素晴らしさを表現する言葉として,『感動』という言葉がしばしば用いられる.感動とは,『美しいものや素晴らしいことに接して強い印象を受け,心を奪われること』(大辞林\cite{Book_103})とあるように,体験に対する肯定的な評価であると共に,記憶の定着や感情の喚起を伴った心理状態の大きな変化である.そして,感動するような体験には,人のやる気を高めたり,価値観を変えたりするなどの効果があるといわれている\cite{Article_007}.また,このような感動を引き起こす対象としては,マスメディアが提供するドラマや映画,音楽などの割合が高いとされる\cite{Web_401}.

本研究の目的は,放送番組の品質評価,とりわけ音の評価に,『感動』という言葉をキーワードとした評価指標を導入することにある.コンサートホールで演奏された音楽を聞くなど,音そのものに直接的に感動することもあれば,ドラマやスポーツ中継などのBGMや歓声,アナウンスなどの音が放送番組を盛り上げることで間接的に感動を喚起することもある.実際,音楽聴取における感情誘導効果や覚醒水準調整効果などの心理的な影響が,多くの実験によって確かめられており\cite{Book_101},音が引き起こす心理的な効果が,番組コンテンツの評価に与える影響は大きいと考えられる.


従来の研究では,音の評価を行う際,言葉を使ってその評価を表現することが多い.難波ら\cite{Book_105}は,音の物理特性と人が受ける印象評価との関連を調べるために,形容詞対を用いたSD法による音色や音質の評価や,それに基づく音の分類を行っている.また,音響システムの展覧会などで配布される広告では,システムの目的や想定される購入者によって,音を表現する言葉を使い分けている.たとえば,映画を対象としたサラウンドシステムにおいては,『迫力』や『臨場感』,『低音の響き』,『余韻』といった言葉が多く使われている.これに対して,ピュアオーディオの分野では,『音像』や『サウンドステージ』,『静寂』,『実在感』,『反応のよさ』といった言葉が使われている.これらの言葉は,従来の研究では使われない評価語であるが,音響の特徴を表す表現として日常的に用いられ,映画音楽とクラシック音楽などの各コンテンツがもつ音の良さを表現しているものと思われる.広告が,消費者ニーズを満たすために洗練された表現を使い分けていることを考えると,コンテンツによって要求される音の印象評価の内容が異なることも考えられる.


川上ら\cite{Inproc_201}は,感情語と『感動』を用いて音楽の印象評価を行ったが,印象評価としての『感動的な』音楽と,気分評価として実際に『感動した』音楽が異なることを指摘している.音楽の印象評価だけで音によって喚起される感動を一意に評価することは難しく,どういう人がどういう状況においてその音響特徴に良さを見出すのかを検討する必要がある.これは,ある状況において聴取者がその音をどのように聞きたいのかという価値観を調査することに他ならない.すなわち,現実の聴取場面を考えた場合,状況や音源,聴取者の心理状態や動機づけを無視して,物理的な音響特徴だけに焦点をあてて音の良さを論じることはナンセンスである.


2005年秋の音響学会研究発表会において開かれた「なぜ音楽が心に響くのか」というスペシャルセッションでは,音楽に音の良さを見出している時の心理状態は,『感動する』の他に,『心に響く』,『心を躍らせる』,『深く内省する』,『揺り動かす』,『至高感』,『一体感』,『理解』,『共感』,『興奮』,『楽しい』,『悲しい』などの様々な言葉を用いて表現されていた\cite{Inproc_202}\cite{Inproc_204}.しかし,これらの言葉の語義や言葉から連想される心理状態は,かなり異なる.音の素晴らしさを表現する際,『感動』という言葉でまとめて記述することは可能であるが,どのように感動するのかを言及しなければ,用いる言葉の曖昧性から,音に対する評価が評定者間で一致しないことも考えられる.実際,感動は単一の感情価ではないが,喜びや悲しみといった感情を伴う\cite{Article_006}ことや,感動は感情の質ではなく,複合情動の総合的強度と相関がある\cite{Inproc_203}と言われており,研究者の中でも感動という心理状態の定義は曖昧である.


そこで,我々は,『感動』という言葉で表現しようとしている心理状態を明確にするために,心理状態を言葉で評価するのではなく,言葉から心理状態を連想することで,『感動』という心理状態の分類を試みた.まず,アンケートを実施し,人が日常的にどういう対象に対して感動するのか,また,感動している心理状態をどういう言葉を用いて表現しているのかを調査した.さらに,アンケート結果から抽出した感動を表現する言葉(以下,感動語)を主観評価(一対比較)することによって,各々の感動語から連想される心理状態の類似度を求め,類似度ベクトルの距離に基づいて数学的に感動語を分類した.本稿では,感動を喚起した要因について考察するとともに,感動語間の類似度ベクトルに基づいて得られた感動語の分類結果について述べる.



\section{感動に関する従来の心理学研究}

感情心理学の分野では,感情の種類を幾つかに分類する研究が多く行われてきた.感情は,時間的な側面から比較的長い期間持続する気分(mood)と一過的で強烈な感情である情動(emotion)などに分類されている\cite{Book_104}.また,感情の質的な分類として,喜怒哀楽のような特定のカテゴリー\cite{Book_108}\cite{Book_301}\cite{Book_302}や,快—不快,興奮—沈静,睡眠—覚醒などの少数の次元による記述\cite{Book_109}\cite{Article_002},肯定的—否定的という最も基本的な2分法などが提唱されている.さらに,感情を表現する言葉の分類から感情を分類する研究も行われている\cite{Article_004}\cite{Article_003}\cite{Article_005}.


これらの感情の研究に対し,感動という心理状態の特異性が報告されている.戸梶\cite{Article_006}は,喜びを伴う感動はその対象を選ばないが,悲しみを伴う感動が喚起されるのは内容的に第三者の立場である場合に限定されることを指摘している.さらに,感動は複数の感情との間に密接な関係をもっており,従来の枠組みである単一感情価では捉えることができない上,感動に伴って喚起される感情は,喜びや悲しみ,驚きなどであり,恐怖や怒りといった感情は伴わないことを指摘している.同様に,中村ら\cite{Inproc_203}は,音楽聴取時の感動についても,基本的な情動理論を単純に当てはめることができず,種々の情動の強さの複合として議論する必要があると指摘している.


安田ら\cite{Inproc_205}は,音楽聴取時の情動評定項目を,高揚感群(高揚感を感じる,興奮を感じる)と切なさ群(涙が出る,切なさを感じる,胸が締め付けられる),鳥肌群(背筋がぞくぞくする,鳥肌が立つ)に分けて,Hevner\cite{Article_003}が考案した8つの形容詞による印象評価を考察している.その結果,鳥肌群の情動は,高揚感群と切なさ群の両方と相関の高い形容詞と相関が高く,音楽聴取時の感動体験に共通する要因であることが示唆されている.Bloodら\cite{Article_001}は,自分にとって素晴らしい音楽を聴取した際に感じるゾクッとするような体験 (``shivers-down-the-spine'' or ``chills'') において賦活する脳の部位が,脳の報酬系といわれる他の情動体験でも反応する部位と共通していることを示している.


一方で,語義的には『美しいものや素晴らしいことに強い印象を受ける』とあるように,感動はある体験の統括的な評価として肯定的な印象を示す言葉であるとも考えられる.何に感動するか,何を肯定的に評価するかは,個人や状況によって異なる.また,情動の変化の「強さ」をどの程度に捉えるかによって,ある体験を感動とみなすかどうかが個人よって異なる可能性がある.実際,感動体験に関する世論調査\cite{Web_401}では,感動を経験する頻度が人によってかなり異なることや,感動の事由が,『期待以上であった』,『自分にはできない』,『共感できる』,『心配していた』,『期待にこたえた』,『自分にも似た経験がある』など多岐にわたること,感動を表現する言葉として,『ジーン』,『ウルウル』,『ドキドキ』,『グッと』,『ワクワク』,『ゾクゾク』,『ウキウキ』など多様であることなどが報告されている.


このように,感動という心理状態は,感情以外にも実に様々な要素を含んでいるにも関わらず,言葉として『感動』というクラスがあるために,感動を一つの情動として捉えがちである.しかし,戸梶\cite{Article_006}が,感動に伴う感情の種類によって感動の分類を試みたように,むしろ幾つかの情動をまとめて感動と表現していると考えられる.確かに,感動とは,何かしらの感情を伴う心の動きではあるが,感動を喚起させる対象に依存することから,従来の感情そのものを対象とした心理現象ではなく,感動対象との関係も含めた複合的な心理状態を検討する必要がある.本稿では,以下,『感動』という言葉で表現している心理状態や感動対象をまとめて分類を試みる.



\section{日常生活における感動の抽出}

\subsection{感動に関するアンケート}

まず,日常生活で『感動』という言葉がどのように用いられているのかを調べるために,自由記述方式のアンケートを実施した.アンケートの主項目は,感動を表現する言葉(感動語),最近感動した体験,音(音楽・音声)を聴取して感動した体験とした.今回の調査は,感動を表現する言葉を抽出することを目的とした探索的な調査であり,アンケートの対象者は,身近な母集団として,40歳前後を中心とした20歳代から50歳代の技術研究者25名(内音響研究者21名,内女性2名)と情報科学を専門とする大学生25名(内女性7名)の計50名とした.



\subsection{感動体験の事例}

感動した体験について116件(技術研究者67件,大学生49件),感動した音について61件(技術研究者37件,大学生24件)の事例が挙がった.感動の対象としては,『映画』,『音楽』,『スポーツ』,『人の優しさ』,『自然の景観』など,従来の研究\cite{Web_401}\cite{Article_007}と同じような事象が並んだ.感動を喚起する要因としては,『美しさ』,『切なさ』,『楽しさ』,『懐かしさ』,『優しさ』などの感動の対象に関する印象から記述できる側面があった.これらは従来の形容詞や感情語による印象評価で評価できる部分である.一方,映画の『ラストシーン』や『生死にかかわるシーン』,『実話に基づいたドキュメント』,『大学に合格する』,『スポーツで優勝する』など,印象評価というよりは,具体的な内容を伴う対象も多く含まれた.さらに,『落胆していたとき』,『卒業式で聴いた』,『ずっと行きたかった』,『自分の気持ちにあった』,『初めて聴くのに』,『映画のあるシーンを連想する』,『思いもよらない』,『予想を上回る』など,体験時の心理的な状態や音を聴く状況に関する条件を述べている例も多く見られた.


このように,対象そのもの評価的印象以外に,日常の関心事項や自分の経験に基づいた知識や,その時の心理状態が,感動を喚起する要因として大きな影響を与えていることが示された.これが,音楽の印象評価としての『感動的な』音楽に,必ずしも聴取者が気分評定として『感動する』わけではない\cite{Inproc_201}理由と考えられる.



\subsection{母集団による感動対象の傾向}

\begin{table}[b]
\begin{center}
\caption{感動した体験}
\begin{tabular*}{125mm}{|p{75mm}|p{12mm}|p{12mm}|p{12mm}|}       \hline
\multicolumn{1}{|c|}{感動体験} & 合計 & 研究者 & 大学生 \\       \hline
映画・テレビ・ラジオ・ゲーム   & 26   & 14     & 12     \\       \hline
スポーツに関すること           & 16   & 10     & 6      \\       \hline
音楽                           & 13   & 9      & 4      \\       \hline
優しさ・親切にされて           & 10   & 3      & 7      \\       \hline
自然の景観                     &  9   & 6      & 3      \\       \hline
家族・子供・恋愛・愛情         &  8   & 5      & 3      \\       \hline
願い事がかなう・達成する       &  8   & 5      & 3      \\       \hline
アイディアや技術               &  8   & 7      & 1      \\       \hline
偶発的な事象                   &  6   & 0      & 6      \\       \hline
美味しい食事                   &  4   & 3      & 1      \\       \hline
その他                         &  8   & 5      & 3      \\       \hline
\end{tabular*}                                          \\
\par
\vspace{1\baselineskip}
\caption{感動した音}
\begin{tabular*}{125mm}{|p{75mm}|p{12mm}|p{12mm}|p{12mm}|}     \hline
\multicolumn{1}{|c|}{感動した音・音楽・音声} &合計&研究者&大学生 \\ \hline
コンサートなどの生演奏             & 14   & 13     & 1      \\ \hline
歌詞・優しい言葉・気の利いた台詞   & 10   &  2     & 8      \\ \hline
音の響き・素晴らしさ・音楽そのもの &  9   &  4     & 5      \\ \hline
その日の気分との相乗効果           &  7   &  4     & 3      \\ \hline
ドラマや映画,ゲームの挿入曲       &  5   &  2     & 3      \\ \hline
懐かしい・ある出来事を思い出す     &  4   &  2     & 2      \\ \hline
予想を超える                       &  4   &  3     & 1      \\ \hline
その他                             &  8   &  7     & 1      \\ \hline
\end{tabular*}                                              \\
\end{center}                                
\end{table}

感動した体験,音を聴いて感動した体験について,アンケートの回答事例を集計した.結果を表1, 2に示す.『スポーツに関すること』は,実際に自分が体を動かして運動をすること以外に,スポーツ観戦や,応援しているスポーツチームが優勝するなどの体験も含んでいる.『その他』の項目は,『絵画を見て』,『真剣さを目の当たりにして』などであった.また,『偶発的な事象』とは,『安売りセール』や『出かける直前に雨がやむ』,『目的地までの信号が全部青であった』などの偶然起きた出来事を指している.また,感動した音としては,『コンサートなどで聴いた生演奏』の他,『歌詞に共感した』,『その日の気分と一致した』,『ドラマなどで使われた』などの事例が挙げられた.『その他』の項目としては,『除夜の鐘』,『産声』などがあった.


アンケートに参加した多くの技術研究者が感動する事象は,知識やアイディアの意外性や技術の精巧さに関する項目であった.一方,大学生は偶発的な事象に感動する場合が多かった.また,感動した音についても,技術研究者が音楽の生演奏や音の響きを事例として挙げたのに対し,大学生は歌詞に共感した音楽や優しい言葉を挙げる例が多かった.このように,感動対象やその要因は,世代や職業,所属組織など母集団によってかなり割合に違いがあると考えられる.どういう母集団に分類するべきか,個人間のばらつきを超えてある程度共通する感動対象やその要因についても,多くの母集団を対象とした大規模で詳細な調査が必要であろう.以下の章では,大規模な感動対象の調査を目的として,評価語を選出するために,感動語をクラスに分類し,典型的な感動のパターンを得ることを試みた.



\section{感動の分類}

\subsection{感動語の抽出}

これまで我々は感動という心理状態を区別せずに議論を進めてきた.従来,感動の体験は,感動に伴って喚起される感情の種類によって分類が行われている\cite{Article_006}.しかし,同じ『嬉しい』という言葉を用いても,落胆していたときに優しくされて感じる心暖まるような嬉しさと,応援していた野球チームが優勝したときに感じる喜びを爆発させるような嬉しさ,駅から目的の場所までの信号がすべて青だったときに感じる嬉しさは,心理状態としては大きく異なっていると考えられる.本研究では,こうした感動における多様な心理状態を体系的に捉えるために,アンケートの回答から感動語を抽出し,分類することで,表現されている心理状態を分類することを試みた.


まず,アンケート調査において,感動を表現する言葉を記述させた.その結果,延べ170語(技術研究者84語,大学生86語)の感動語が得られた.アンケートより抽出した言葉は,なるべく言葉からイメージする心理状態をそのまま評価してもらうため,できる限りアンケートに書かれた言葉のまま使用した.ただし,『マジヤベェ』,『すげぇ』などの口語は,『やばい』,『すごい』と同じ感動語として扱った.その結果,感動を表現する言葉として,105語の感動語が得られた.また,アンケートの感動体験に記述されている用語で感動を表現する言葉として記述されていない表現(22語),その他感動に関する先行研究において用いられている言葉で今回抽出した感動語に含まれていない表現(10語),類語辞典から感動と似た語義の表現(13語)を選出した.その結果,得られた感動語は,150語になった.



\subsection{1対比較による感動の距離}

次に,前述の手続きで得られた150語について,ある感動語が別の感動語と同じ感動を表現しているか否かを一対比較する主観評価実験を行った.評価協力者は,アンケート調査を行ったどちらの母集団とも異なるように,感動に関するアンケートに参加していない20代から30代の女性11名(言語学,文学といった文系大学を卒業しており,心理実験には参加した経験がない)とした.実験は,3日間行い,1日の実験時間は5時間(途中,1時間と30分の休憩を取得)とし,実験時間中,自由に休憩を取らせた.評価協力者は,3グループに分けて召集され(5人,4人,2人),言葉の意味が近いかではなく,ある感動語 から感動している状態を連想し,その状況や心理状態が別の感動語 を用いて表現することが可能か否かを1/0の2件法で評価するよう集団で教示された.その後,各自,自分のペースで,50音順に並べてパソコンの画面に提示された150語の感動語を総当りで評価した.


同じ感動を表現できると評定された割合を一致率 とすると,ある感動語 は他の感動語との一致率 を用いて150次元のベクトルで表記することができる.
$$X_i = (R_{i1},R_{i2},\cdots,R_{ij},\cdots,R_{iN}) \eqno(1)$$ 
また,感動語 と の心理状態上の距離をユークリッド距離を用いて,
$$D_{ij} = \sqrt{\frac{1}{N}\sum_{k=1}^N (R_{ik}-R_{jk})^2} \eqno(2)$$
とした.



\subsection{感動の分類方法}

感情心理学では,感情語を用いて感情状態を分類する研究が多く行われているが,その中に少数の次元を仮定して感情を記述しようとする次元研究がある.次元の数や種類は研究間で必ずしも一致しているわけではないが,共通しているのが,快—不快の快楽次元と覚醒—眠気といった覚醒次元の2次元である.(たとえば\cite{Article_002}).そこで,今回は,感動に関しても2分岐で分類することを試みた.2分岐の分類には,LBGアルゴリズム\cite{Article_008}を用い,ベクトル距離$D_{ij}$に応じた感動語の分類を行った.


LBGアルゴリズムでは,2分岐を行う際の初期値として,あるクラス$k$の任意の2つの感動語$X_{ki}$,$X_{kj}$を用いる.残りの感動語が$X_{ki}$,$X_{kj}$のどちらに近いかを算出し,距離の近さで感動語を2つに分類する.これを初期クラス$k1$,$k2$とする.次に,各初期クラスの重心$C_{k1}$,$C_{k2}$を求め,各重心との距離が近い感動語を新しいクラス$k'1$,$k'2$とする.この新しいクラスの重心を求め,求めた重心との距離が近い感動語でさらに別のクラスを作成する.これを何度か繰り返すことで,感動語$X_{ki}$,$X_{kj}$を初期値としたクラス$k1ij$,$k2ij$が求まる.この分割において,両クラス内の感動語と両クラスの重心との距離の平均自乗誤差を歪$DS_{k1ij}$,$DS_{k2ij}$とする.クラス$k$内のすべての感動語を初期値として歪を求め,歪が最も小さくなる初期値を用いてクラス$k$を2つに分類した.


最初に全感動語を2つに分類した後は,両クラスの重心と歪を求め,歪が大きいクラスについて,LBGアルゴリズムを用いて,2分岐による分類を行いった.さらに,各クラスについて重心と歪を求め,最も大きい歪のクラスについて分類を行い,最終的に各クラスが閾値として設定された歪の大きさになるまで分類を進めた.



\subsection{感動の分類結果}

ここでは,各クラスの中心となる概念をクラスの重心に近い感動語4つを用いて表現する.まず,分類する前の全感動語の中心概念は,『しみる』,『心にしみる』,『余韻』,『心をわしづかみにする』であった.これを歪の大きさに応じて14のクラス(AからN)まで分類した.その結果を表3に示す.表中の数字は,歪の大きい順に分類した分岐番号である.また,各クラスの中心概念を表す感動語を星印で表現した.




\begin{table}[t]
\begin{center}
\caption{感動語の分類結果}
\includegraphics[scale=0.75]{pic.eps}
\par\vspace{1\baselineskip}
\begin{tabular*}{142mm}{p{6mm}p{128mm}}
A.& 胸がいっぱいになる*,思わず涙*,涙*,愛*,ああ,言葉にできない,
      よい,泣く \\
B.& 心が暖まる*,癒される*,安らぎ*,家族愛*,ありがとう,幸せ,
      安堵,なんか良い \\
C.& しみる*,黄昏*,ノスタルジー*,心にしみる*,落涙,泣けた,
      感涙,胸が詰まる,悲しい,感傷,寂しい,切なくなった,思い出,
      しみじみ,情緒,懐かしい,感じ入る,ジーンとする,心に残る,
      忘れられない,余韻,心に響く,ため息,ものあはれ \\
D.& 綺麗*,美しい*,素敵*,すばらしい*,憧れ,心を奪われる,
      しびれる,魅惑的,景色,感嘆,絶景,雄大,喜び \\
E.& 思わず無言*,無言* \\
F.& 胸を打つ*,グッとくる*,琴線に触れる*,心が熱くなる*,命,
      感銘,心が打たれる,感極まる,心が震える,こみあげる,感激,
      胸がキュンとなる \\
G.& うぉー*,うわぁ*,わぁ*,おー*,すごい,気持ちが高鳴る,
      興奮する,人に言いたくなる \\
H.& 共感*,経験*,自己陶酔*,満足*,甘い,美味しい,感心,最終回 \\
I.& 心が躍る*,ワクワクする*,わーい*,爽快*,おもしろい,たのしかった \\
J.& ヤッター*,歓喜*,優勝*,達成*,嬉しい,おっしゃー,キター,
      やっとの思い,認められる,チームワーク,めっちゃ楽しい \\
K.& 背筋がゾッとする*,パニック*,混乱*,あぜん,驚愕*,焦り,
      ありえない,怖い,息が詰まる,緊迫,ゾクッとする,絶対笑うって \\
L.& 無情*,いたたまれない*,つらい*,やりきれない*,怒り,不条理,
      打ち震える,号泣,同調 \\
M.& 鳥肌がたつ*,心をわしづかみにする*,やばい!*,身震い*,妖しい,
      畏敬,荘厳,心を射抜く,迫力がある,血が騒ぐ,臨場感がある,
      情動,震える,ドキドキする \\
N.& マジ*,意外*,目が覚める*,見たことがない*,聴いたことがない,
      へー,発見,スピードがある,大きい,でかい,うそぉ,驚き,
      口があく,呆然,衝撃を受ける \\
\end{tabular*}                      
\end{center}
\end{table}

最初の分類(分岐番号1)において,クラスA, B, C(中心概念『しみる』,『情緒』,『心にしみる』,『余韻』)とそれ以外(『心をわしづかみにする』,『しびれる』,『心が震える』,『聴いたことがない』)に分かれた. ここでは,感情の動き方という観点から,動きを表現する言葉に着目する.前者のクラスには,『ジーンとする』,『しみる』などのじわじわと状態が続く表現が多く含まれる.後者のクラスでは,『心が打たれる』,『心を射抜く』などのするどく急激な動きを表す言葉,『心が震える』,『混乱』などの非定常な状態を表す言葉があった.幸せや思い出にしみじみと感じ入るような比較的静かな感情の変化が含まれる感動と,激しく表出するような鋭く,強い感情の変化を示すものが多い感動に分かれている.クラスA, B, Cを『受容』,それ以外のクラスを『表出』と呼ぶ.強度と時間的な継続性という観点では,比較的緩やかな感情である『気分』と鋭く短い反応である『情動』に似た分類との対応が可能であり,感情の2次元モデル(快—不快や肯定—否定)といった質的な分類とはならなかった.


次に,『受容』と『表出』で歪が大きかったのは,『表出』の感動クラスであり,D〜J(中心概念『しびれる』,『心が熱くなる』,『心が震える』,『経験』)とK〜N(中心概念『驚愕』,『心をわしづかみにする』,『マジ』,『やばい!』)に分類された(分岐番号2).前者のクラスは,喜びや楽しみ,嬉しいといった正の感情を表現する感動語を多く含んでいる.これに対し,後者のクラスは,焦りや怒りといった負の感情(K, L)と驚きといった中立的な感情(M, N)を表現する感動語を多く含んでいる.従来の研究では,戸梶\cite{Article_006}が感動に伴う感情に着目し,『悲しみ』,『喜び』,『驚き』という感情で感動を分類している.『悲しみ』は,『受容』の感動であるクラスCに含まれており,感動に伴う3つの大きな感情は,今回分類した感動の大きな3つのクラス『受容』,『表出:正の感情』,『表出:負・中立の感情』のそれぞれに内包される.また,安田ら\cite{Inproc_205}が,音楽聴取時の感動として分類した『切なさ』,『高揚感』,『鳥肌』の3つの情動群を表現する言葉も,『受容』のクラスC,『表出:正の感情』のクラスG,『表出:負・中立の感情』のクラスMにそれぞれ内包される.


このように,従来研究の知見を包含しえたのは,今回抽出した感動語の抽出結果と評価尺度による感動語の距離の求め方の妥当性を示すものと考えられる.今回実験に参加した評価協力者には偏りがあるが,感動を大きく分類する範囲においては,普遍性があることが考えられる.


次に,『表出:正の感情』がクラスD, E, F, G(中心概念『しびれる』,『すばらしい』,『胸を打つ』,『琴線に触れる』)とクラスH, I, J(中心概念『認められる』,『心が躍る』,『ヤッター』,『たのしかった』)に別れた(分岐番号3).前者が,『美しい』『景色』が『胸を打つ』という比較的『受容』的な感動に近い,受動的な感動であるのに対し,後者は,『達成』して『認められる』ことで『心が躍る』と能動的,主体的な行為による感動を表現している.以下,『表出:受動的正の感情』,『表出:能動的正の感情』とする.


次に,4つクラスは各々2つに分けることができ,以下の計8つとなる.


『受容』の感動(分岐番号4)\par                               
\begin{tabular}{p{20mm}l}
A     & 『胸がいっぱいになる』,『涙』,『思わず涙』,『愛』\\
B,C  & 『しみる』,『しみじみ』,『心にしみる』,『情緒』  \\
\end{tabular}  \par                                            
『表出:受動的正の感情』の感動(分岐番号5)\par                 
\begin{tabular}{p{20mm}l}
D, E, F& 『琴線に触れる』,『胸を打つ』,『しびれる』,『心が打たれる』\\
G       & 『うぉー』,『うわぁ』,『わぁ』,『おー』                    \\
\end{tabular}  \par                                             
『表出:能動的正の感情』の感動(分岐番号7)\par                  
\begin{tabular}{p{20mm}l}
H, I & 『たのしかった』,『共感』,『心が躍る』,『わーい』\\
J     & 『ヤッター』,『歓喜』,『優勝』,『達成』          \\
\end{tabular}  \par                                             
『表出:負,中立の感情』の感動(分岐番号6)\par                  
\begin{tabular}{p{20mm}l}
K, L & 『不条理』,『混乱』,『息が詰まる』,『怒り』                      \\
M, N & 『見たことがない』,『心をわしづかみにする』,『マジ』,『やばい!』\\
\end{tabular}  \par                                              
さらに分類を進めると,最終的に分類された14のクラス(表3)のようになる.



\section{考察}

\subsection{類語辞典を用いた感動語の分類}

本稿では, LBGアルゴリズムを用いたベクトル距離によって収集した感動語を分類・検討してきた.ところで,語彙の分類を行う際の非数量的方法として,類語辞典を用いることが考えられる.日本語の代表的なシソーラスとしては,分類語彙表\cite{Book_102}や日本語語彙体系\cite{Book_106},類語新辞典\cite{Book_107}などがある.分類語彙表では,まず,用の類(名詞の仲間),体の類(動詞の仲間)などで大分類されている.感動語の分類では,これらの区別は必要ではない.一方,日本語語彙大系は,精密に作られたシソーラスであるが,感動語の分類を扱うにはあまりにも分類が詳細すぎる.そこで,今回は,類語新辞典を用いて収集した感動語を分類した結果について検討する.


類語新辞典では,言葉を自然,人事,文化に大きく3つに分類している.このうち,文化に分類される感動語は存在しなかった.さらに,感動語は,自然について,自然・性状・変動に分類され,人事について,行動・心情・性向に分類された.感動詞,間投詞などが語彙に含まれていなかったため,分類語彙表を参考としてその他という分類を設定した.分類結果を表4に示す.



\begin{table}[tb]
\begin{center}
\caption{類語新辞典による感動語の分類}
\begin{tabular*}{140mm}{|p{52mm}|p{8mm}|p{70mm}|}\hline
\multicolumn{1}{|c|}{分類} &
\multicolumn{1}{|c|}{語数} & 
\multicolumn{1}{|c|}{代表的な感動語}                           \\ \hline
自然(天文,景観,生理) & 10語 & 絶景,命,癒される,涙,など \\ \hline
性状(形状,刺激,価値,程度,数量,時間,状態) & 20語 & 大きい,美味しい,
素敵,ありえない,美しい,もののあはれ,すごい,など           \\ \hline
変動(動揺,情勢,関連) & 6語 & 迫力がある,混乱,緊迫,など  \\ \hline
行動(表情,見聞,陳述,労役) & 12語 & 泣けた,震える,
                                          言葉にできない,など \\ \hline
心情(感覚,思考,学習,要求,誘導,闘争,意向,栄辱,愛憎,悲喜)
 & 57語 & 共感,思い出,畏敬,情緒,感激,心が躍る,
                          胸を打つ,興奮する,安堵,驚き,など \\ \hline
性向(姿態,身振り,態度,境遇,心境) & 32語 & 懐かしい,ゾクッとする,
          切なくなった,たのしかった,心が暖まる,あぜん,など \\ \hline
その他(間投詞) & 13語 & ああ,うぉー,へー,わぁ,など       \\ \hline
\end{tabular*}
\end{center}
\end{table}

分類の結果,心情(悲喜)や性向(心境)を表す言葉が多かった.これは,感動が自分の心的な状態を表現する言葉であるためと考えられる.一方で,性状を表す言葉として,価値を表す言葉が多かった.これは,『素晴らしい』,『ありえない』などの感動対象に対する自分の価値を表現したためと思われる.『絶景』や『迫力がある』などの自然・変動に分類される言葉は,感動の対象を,『泣ける』や『鳥肌がたつ』といった行動に関する言葉は,感動した際に起こる身体的な変化を表している.


類語新辞典は,言葉の持つ意味で分類されており,その言葉を使う心理的な状態の類似性については配慮されているとは限らない.本稿では,感動の対象や身体的な反応,心理状態の区別無く,どの言葉が同じ心理状態を表現しているのかで感動語を分類した結果,類語新辞典を用いて分類した場合に比べ,対象の特徴と感動した心境・心情が混在した分類となった.これらのクラスは,感動しているという状況を表現するという観点において類義語であるといえる.つまり,『懐かしい』という心境は,『心にしみる』,『心に響く』という心情に近い心理状態を表しており,『ため息』という行為で感動を表現していると考えられる.これらの言葉は,その言葉が持つ意味としては異なるが,表現しようとした心理状態は近いのである.



\subsection{感動語のあいまい性・多義性}

感動語には,かなり抽象的な言葉が含まれていた.そのため,感動語そのものが広義な意味やイメージを持ち,言葉から連想される感動している状況が一意に決められないという可能性がある.そこで,一対比較実験において,評価協力者によって評価結果にばらつきがあった感動語と,ばらつきがなかった感動語をそれぞれ上位18語,リストアップした(表5).ただし,感動語$X_{i}$の評価のばらつき具合は,他の感動語との一致率$R_{ij}$を用いて$0.5-R_{ij}$の絶対値の和が大きいものをばらつきのなかった感動語とした.



\begin{table}[tb]
\begin{center}
\caption{感動語の1対比較おける評価結果のばらつき}
\begin{tabular*}{140mm}{|p{42mm}|p{91mm}|}        \hline
ばらつきがあった感動語 & 無言,経験,わあ,心に残る,よい,うぉー,思わず無言,言葉にできない,愛,胸がいっぱいになる,ああ,琴線に触れる,おー,忘れられない,こみあげる,すばらしい,憧れ,うわぁ         \\ \hline
ばらつきがなかった感動語 & 緊迫,安らぎ,背筋がゾッとする,絶対笑うって,混乱,安堵,わーい,壮快,いたたまれない,同調,不条理,関心,口があく,つらい,息が詰まる,無常,感傷,やりきれない               \\ \hline
\end{tabular*}                    
\end{center}
\end{table}

ばらつきがあった感動語に関しては,間投詞などの抽象的な言葉が多く含まれており,これらの言葉を用いて感動を表現する場合,評価者によって異なる心理状態を想定していた可能性がある.逆に,ばらつきがなかった『緊迫した状況』や『安らぎを感じる状況』において感動する条件というのは限られていると言える.ばらつきがあった感動語は,主にクラスA,Gに多く含まれていた.これらの言葉は,評価に個人差がありながらも,似たような評価を受けており,共通したイメージが連想できたものと思われる.


しかし,言葉が抽象的であるがゆえ,具体的な事象を述べた複数のクラスを包含している可能性が高い.たとえば,『ああ』は,同じ『受容』であれば,『安らぎ』も『懐かしい』も表現できる.これらの感動語は,評定協力者によって,思い描く感動が異なるため,今後,音の評価を行っていく評価語としては不適切であると思われる.



\subsection{感動の心理状態}
今回の感動語に含まれる感情としては,『受容』には,悲しみや切なさの他に,安らぎや幸せ,懐かしさなどがあった.また,『表出』に含まれる負の感情として,怒りや焦りといった感情も感動に関与することが示唆された.また,多面的感情状態尺度\cite{Article_005}の8つの感情群のうち,『活動的な快』,『非活動的な快』,『親和』,『驚愕』,『集中』,『抑鬱・不安』,『敵意』に相当する表現が含まれており,欠けている感情群は『倦怠』だけであった.特に,『活動的な快』,『非活動的な快』,『親和』,『驚愕』に関する感動語が多かった.感動は確かに複雑な感情の組み合わせによって成り立ってはいるが,基本的には体験に対する肯定的な評価であることを考えると当然といえる.


感動語には多くの感情価が含まれるが,『受容』の感動であるクラスB,Cには,安らぎや,幸せ,悲しみ,切なさなどの複数の感情価が混在している.また,クラスA,G,Hは,『言葉にできない』,『泣く』,『気持ちが高鳴る』,『興奮する』,『感心』,『満足』など感情の種類を特定できる表現が含まれていなかった.このように,感動語のクラスは,喜怒哀楽というようないわゆる感情の種類では分かれなかった.これは,従来の感情の分類ではうまく感動を記述できないという戸梶\cite{Article_006}や中村ら\cite{Inproc_203}の指摘を支持するものである.


また,肯定的な評価であるはずの感動に,怒りや悲しみが含まれるのは矛盾を感じるが,戸梶\cite{Article_006}は,悲しみを伴う場合,感動の対象に対して第三者的な立場である必要があると指摘している.つまり,負の感情を伴う場合,自分が体験するのではなく,サスペンスドラマなどを視聴し,主人公の死に悲しみを感じ,その犯人に怒りを覚え,その物語に切なさを感じるという体験が感動として認知されると考えられる.また,クラスDは,『景色』に『雄大』である,『美しい』,『すばらしい』と感じ,『心を奪われる』というような感動であり,クラスJは,『優勝』を『達成』して『嬉しい』という感動である.このように,感動とは,感情そのものではなく,感動の対象と心の動き,複数の感情が入り混じった状態の組み合わせによって分類することができる.


ここで,『感動』という言葉で表現されるこういった様々な心の状態の共通要素について検討する.まず,『すごい』,『衝撃を受ける』,『言葉にできない』,『忘れられない』という言葉に代表されるように,心の動きの程度が強いことを表す言葉が多くのクラスに点在した.また,『思わず涙』,『認められる』,『癒される』,『忘れられない』,『心が打たれる』などの強制的,受動的にその感情が喚起されたことを表現する言葉が多く含まれていた.


以上のことから,感動とは,ある対象から影響を受け,分類されたように心が動くが,その影響が自分では制御できないくらい強いという体験を表現する肯定的な評価の総称であると考えられる.



\subsection{感動語を用いた音の評価に向けて}

アンケート調査における感動を表現する言葉が多く含まれた感動語のクラスは,延べ23語のクラスC,延べ22語のクラスN,延べ22語のクラスG,延べ20語のクラスDであった.今回調査に参加した技術研究者の回答に多く含まれていた感動語のクラスは,N(技術研究者15語:大学生7語),F(6語:2語)であり,大学生の回答に多く含まれていた感動語のクラスは,J(大学生14語:技術研究者2語),K(7語:3語)であった.大学生で多く見られたクラスJの『嬉しい』という感情で表現される感動は,技術研究者では少なかった.アンケート調査における感動体験の記述においても同様に,大学生は,優しくされて嬉しい,雨が止んで嬉しいといった『嬉しい』に関係する感動が多かった.


このような母集団による傾向の違いが生じる理由として,まず,言葉の定義の違いが考えられる.つまり,感動とは何かしらの感情の変化の強度と関係があると思われるが,どの程度の強さ,どういった感情の変化を『感動』として定義しているかが,母集団によって異なっている可能性がある.また,別の理由として,同じ体験から喚起される感情そのもの質的な変化が考えられる.例えば,感動体験の記述において,『卒業式で聞いた曲』といった表現があったが,大学生にとって卒業というイベントは現実性が高く,楽しい,悲しいという感情が比較的新しい記憶と結びつきやすいのに対し,技術研究者にとって卒業とは過去を懐かしむ気持ちが強いことも考えられる.


今回実験に参加した評価協力者の150の感動語の重心は,『しみる』,『心にしみる』,『余韻』とクラスCが中心であった.しかし,母集団の年齢や知識,興味によって感動する対象に違いがあり,その結果喚起される感動の心理状態も異なる可能性がある.別の評価協力者で評価実験を行うと異なる結果となることも予想される.感動とは,いわゆる感情とは異なり,対象と強く結びついており,感情の強弱だけではなく,その対象との関わりや背景知識までも含めた議論が必要である.どういう人がどの対象に対してどういう感動をするのかという割合的な傾向については,大規模な調査研究が望まれる.


従来,音楽や音,音響システムの評価を行う際,聞こえるか聞こえないか,印象としてどういう形容詞で評価できるかで,その価値を記述しようとしていた.つまり,物理的な音の特徴の類似性を主観的に評価するという手法である.しかし,感動対象として音の評価を行う際,どういう観点で感動したのかを区別して評価する必要があると思われる.今回の分類結果は,具体的に音を評価する上で,どういう評価語が必要であるかを感動表現から絞込んだものである.例えば,クラスCの『心にしみる』音楽と,クラスNの『目が覚める』音楽はまったく異なる曲調を連想する.高揚感のある明るい音楽に『心が躍る』(クラスI)ように感動することもあれば,自然環境音に『癒される』(クラスB)ように感動することも考えられる.悲しい印象を与えるような音楽がドラマに挿入される場合でも,ドラマに共感している人には『胸を打つ』(クラスF)という感動を与えるが,共感を覚えない人には退屈な音楽かもしれない. どういうシチュエーションで,どういう音の良さを引き出すことが,放送番組における感動の質をより豊かにすることに繋がるのかを今後も検討していく必要がある.



\section{まとめ}

『感動』という心理状態をモデル化するために,アンケート調査と語彙分析を行い,感動という言葉で表現される心理状態の種類について検討した.アンケートの回答から抽出した感動語について,言葉の意味ではなく,言葉から連想した感動している状況を主観的に評価して類似度を求めた.そして,語彙間の類似度ベクトルの距離に応じて感動語の分類を行った.その結果,感動語は,いわゆる単一の感情では分類できなかった.しかし,感動に伴う感情や感動の対象,感情の動きの組み合わせによって典型的なパターンに分類できた.『感動』とは,情動そのものではなく,対象からの影響があまりにも強いために自分の感情がうまく制御できないという心理状態の肯定的な評価を表す総称であることがわかった.今後の課題として,どういう人がどういう対象にどういった感動するのかについては,大規模な調査研究が望まれる.







\bibliographystyle{jnlpbbl_1.2}
\begin{thebibliography}{}

\bibitem[\protect\BCAY{Blood \BBA\ Zatorre}{Blood \BBA\
  Zatorre}{2001}]{Article_001}
Blood, A.~J.\BBACOMMA\ \BBA\ Zatorre, R.~J. \BBOP 2001\BBCP.
\newblock \BBOQ Intensely Pleasurable Responses to Music Correlate with
  Activity in Brain Regions Implicated in Reward and Emotion\BBCQ\
\newblock {\Bem Proceedings of national academy of science U.S.A.}, {\Bbf 98}
  (20), \mbox{\BPGS\ 11818--11823}.

\bibitem[\protect\BCAY{Darwin}{Darwin}{1892}]{Book_108}
Darwin, C. \BBOP 1892\BBCP.
\newblock {\Bem The expression of the emotions in man and animals}.
\newblock D.appleton.

\bibitem[\protect\BCAY{Ekman}{Ekman}{1984}]{Book_301}
Ekman, P. \BBOP 1984\BBCP.
\newblock \BBOQ Expression and the nature of emotion\BBCQ\
\newblock In {\Bem In K.Scherer and P.Ekman(Eds.), Approaches to emotion},
  \mbox{\BPGS\ 319--343}. Hillsdale,NJ:Erlbaum.

\bibitem[\protect\BCAY{Hevner}{Hevner}{1936}]{Article_003}
Hevner, K. \BBOP 1936\BBCP.
\newblock \BBOQ Experimental studies of the elements of expression in
  music\BBCQ\
\newblock {\Bem American Journal of Psychology}, {\Bbf 48}, \mbox{\BPGS\
  246--268}.

\bibitem[\protect\BCAY{Linde, Buzo, \BBA\ Gray}{Linde
  et~al.}{1980}]{Article_008}
Linde, Y., Buzo, A., \BBA\ Gray, R.~M. \BBOP 1980\BBCP.
\newblock \BBOQ An Algorithm for Vector Quantizer Design\BBCQ\
\newblock {\Bem IEEE Transactions on Communications}, {\Bbf COM-28}  (1),
  \mbox{\BPGS\ 84--95}.

\bibitem[\protect\BCAY{Plutchik}{Plutchik}{1984}]{Book_302}
Plutchik, R. \BBOP 1984\BBCP.
\newblock \BBOQ Emotions: A general psychoevolutionary theory\BBCQ\
\newblock In {\Bem In K.Scherer and P.Ekman(Eds.), Approaches to emotion},
  \mbox{\BPGS\ 197--219}. Hillsdale, NJ: Erlbaum.

\bibitem[\protect\BCAY{Russell}{Russell}{1980}]{Article_002}
Russell, J.~A. \BBOP 1980\BBCP.
\newblock \BBOQ A circumplex model of affect\BBCQ\
\newblock {\Bem Journal of Personality and Social Psychology}, {\Bbf 39}  (6),
  \mbox{\BPGS\ 1161--1178}.

\bibitem[\protect\BCAY{Shaver, Schwartz, Kirson, \BBA\ O'Connor}{Shaver
  et~al.}{1987}]{Article_004}
Shaver, P.~R., Schwartz, J., Kirson, D., \BBA\ O'Connor, C. \BBOP 1987\BBCP.
\newblock \BBOQ Emotion knowledge: Further exploration of a prototype
  approach\BBCQ\
\newblock {\Bem Journal of Personality and Social Psychology}, {\Bbf 52},
  \mbox{\BPGS\ 1061--1086}.

\bibitem[\protect\BCAY{Wundt}{Wundt}{1910}]{Book_109}
Wundt, W. \BBOP 1910\BBCP.
\newblock {\Bem Grundzuge der physiologischen Psychologie. 6th ed.}
\newblock Leipzig:Wilhelm Engelmann.

\bibitem[\protect\BCAY{戸梶}{戸梶}{2001}]{Article_006}
戸梶亜紀彦 \BBOP 2001\BBCP.
\newblock \JBOQ 『感動』喚起のメカニズムについて\JBCQ\
\newblock \Jem{認知科学}, {\Bbf 8}  (4), \mbox{\BPGS\ 360--368}.

\bibitem[\protect\BCAY{戸梶}{戸梶}{2004}]{Article_007}
戸梶亜紀彦 \BBOP 2004\BBCP.
\newblock \JBOQ 『感動』体験の効果について—人が変化するメカニズム\JBCQ\
\newblock \Jem{広島大学マネジメント研究第4号}, \mbox{\BPGS\ 27--37}.

\bibitem[\protect\BCAY{川上\JBA 中村\JBA 河瀬\JBA 安田\JBA 片平\JBA
  堀中}{川上\Jetal }{2005}]{Inproc_201}
川上愛\JBA 中村敏枝\JBA 河瀬諭\JBA 安田晶子\JBA 片平建史\JBA 堀中康行 \BBOP
  2005\BBCP.
\newblock \JBOQ
  演奏音の印象と演奏音聴取後の気分の関係—``感動''の視点から—\JBCQ\
\newblock \Jem{ヒューマンインターフェイスシンポジウム2005論文集}, \mbox{\BPGS\
  627--630}.

\bibitem[\protect\BCAY{NTTコミュニケーション科学研究所\JBA 池原\JBA 宮崎\JBA
  白井\JBA 横尾\JBA 中岩\JBA 小倉\JBA 大山\JBA
  林}{NTTコミュニケーション科学研究所\Jetal }{1997}]{Book_106}
NTTコミュニケーション科学研究所監修\JBA 池原悟\JBA 宮崎正弘\JBA 白井諭\JBA
  横尾昭男\JBA 中岩浩巳\JBA 小倉健太郎\JBA 大山芳史\JBA 林良彦\JEDS\ \BBOP
  1997\BBCP.
\newblock \Jem{日本語語彙大系}.
\newblock 岩波書店.

\bibitem[\protect\BCAY{松山\JBA 浜}{松山\JBA 浜}{1974}]{Book_104}
松山義則\JBA 浜治世 \BBOP 1974\BBCP.
\newblock \Jem{感情心理学1—理論と臨床—}.
\newblock 誠信書房.

\bibitem[\protect\BCAY{谷口}{谷口}{2000}]{Book_101}
谷口高士\JED\ \BBOP 2000\BBCP.
\newblock \Jem{音は心の中で音楽になる}.
\newblock 北大路書房.

\bibitem[\protect\BCAY{谷口}{谷口}{2005}]{Inproc_204}
谷口高士 \BBOP 2005\BBCP.
\newblock \JBOQ なぜ音楽は心に響くのか(2)—心理学からのアプローチ—\JBCQ\
\newblock \Jem{日本音響学会講演論文集}, \mbox{\BPGS\ 733--736}.

\bibitem[\protect\BCAY{国立国語研究所}{国立国語研究所}{2004}]{Book_102}
国立国語研究所\JED\ \BBOP 2004\BBCP.
\newblock \Jem{分類語彙表}.
\newblock 大日本図書.

\bibitem[\protect\BCAY{三菱総合研究所}{三菱総合研究所}{2003}]{Web_401}
三菱総合研究所 \BBOP 2003\BBCP.
\newblock
\newblock \JBOQ 2003年の感動に関するアンケート\JBCQ.

\bibitem[\protect\BCAY{安田\JBA 中村\JBA 河瀬\JBA 川上\JBA 片平\JBA
  堀中}{安田\Jetal }{2005}]{Inproc_205}
安田晶子\JBA 中村敏枝\JBA 河瀬諭\JBA 川上愛\JBA 片平建史\JBA 堀中康 \BBOP
  2005\BBCP.
\newblock \JBOQ 聴取者の感動体験に伴う情動と演奏音の音響的特性の関係\JBCQ\
\newblock \Jem{ヒューマンインターフェイスシンポジウム2005論文集}, \mbox{\BPGS\
  575--580}.

\bibitem[\protect\BCAY{大野\JBA 浜西}{大野\JBA 浜西}{1981}]{Book_107}
大野晋\JBA 浜西正人 \BBOP 1981\BBCP.
\newblock \Jem{類語新辞典}.
\newblock 角川書店.

\bibitem[\protect\BCAY{寺崎\JBA 岸本\JBA 古賀}{寺崎\Jetal }{1992}]{Article_005}
寺崎正治\JBA 岸本陽一\JBA 古賀愛人 \BBOP 1992\BBCP.
\newblock \JBOQ 多面的感情状態尺度の作成\JBCQ\
\newblock \Jem{心理学研究}, {\Bbf 62}, \mbox{\BPGS\ 350--356}.

\bibitem[\protect\BCAY{難波\JBA 桑野}{難波\JBA 桑野}{1998}]{Book_105}
難波清一郎\JBA 桑野園子 \BBOP 1998\BBCP.
\newblock \Jem{音の評価のための心理学的測定法}.
\newblock コロナ社.

\bibitem[\protect\BCAY{永岡}{永岡}{2005}]{Inproc_202}
永岡都 \BBOP 2005\BBCP.
\newblock \JBOQ なぜ音楽は心に響くのか(2)—美学からのアプローチ—\JBCQ\
\newblock \Jem{日本音響学会講演論文集}, \mbox{\BPGS\ 729--732}.

\bibitem[\protect\BCAY{中村\JBA 結城\JBA 河瀬諭\JBA Maria~R.\JBA
  片岡}{中村\Jetal }{2004}]{Inproc_203}
中村敏枝\JBA 結城牧子\JBA 河瀬諭\JBA Maria~R.Draguna\JBA 片岡智嗣 \BBOP
  2004\BBCP.
\newblock \JBOQ 音楽聴取時の感動体験に関わる情動について\JBCQ\
\newblock \Jem{ヒューマンインターフェイスシンポジウム2004論文集}, \mbox{\BPGS\
  815--818}.

\bibitem[\protect\BCAY{松村}{松村}{1995}]{Book_103}
松村明\JED\ \BBOP 1995\BBCP.
\newblock \Jem{大辞林第二版}.
\newblock 三省堂.

\end{thebibliography}

\begin{biography}
\bioauthor{大出 訓史(非会員)\unskip}{
1997年上智大学理工学部物理学科卒業.1999年東京工業大学大学院修士課程修了.同年,NHK入局,現在,放送技術研究所(人間・情報)研究員.音響認知,音声合成等の研究に従事.電子情報通信学会,日本音響学会,映像情報メディア学会各会員.
}
\bioauthor{今井  篤(非会員)\unskip}{
1989年埼玉大学工学部電気工学科卒業.同年,NHK入局,現在,放送技術研究所(人間・情報)主任研究員.話速変換,音声知覚,音声合成等の研究に従事.電子情報通信学会,日本音響学会,映像情報メディア学会各会員.
}
\bioauthor{安藤 彰男(非会員)\unskip}{
1978年九州芸術工科大学音響設計学科卒業.1980年同大学院修士課程修了.同年,日本放送協会入社.現在,放送技術研究所(人間・情報)高臨場感音響研究リーダ.高臨場感音響システム,音響デバイス,音響認知科学,音響信号処理などの研究に従事.工学博士.電子情報通信学会論文賞,日本音響学会技術開発賞.映像情報メディア学会業績賞など受賞.IEEE,AES,電子情報通信学会,日本音響学会,映像情報メディア学会会員.
}
\bioauthor{谷口 高士(非会員)\unskip}{
1987年京都大学教育学部教育心理学科卒業.1992年同大学院博士課程退学.同年,大阪学院短期大学専任講師,現在,大阪学院大学情報学部教授.1995年博士(教育学).音楽と感情,感情と認知をテーマに研究.日本心理学会,日本音楽知覚認知学会,日本感情心理学会会員.
}
\end{biography}


\biodate





\end{document}
