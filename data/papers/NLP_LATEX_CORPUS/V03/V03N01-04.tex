



\documentstyle[epsf,jnlpbbl]{jnlp_j_b5}

\setcounter{page}{67}
\setcounter{巻数}{3}
\setcounter{号数}{1}
\setcounter{年}{1996}
\setcounter{月}{1}
\受付{1995}{5}{8}
\再受付{1995}{7}{6}
\採録{1995}{9}{8}

\setcounter{secnumdepth}{2}

\title{名詞の指示性を利用した日本語文章における名詞の指示対象の推定}
\author{村田 真樹\affiref{KUEE} \and 長尾 真\affiref{KUEE}}

\headauthor{村田 真樹・長尾 真}
\headtitle{名詞の指示性を利用した日本語文章における名詞の指示対象の推定}

\affilabel{KUEE}{京都大学工学研究科 電子通信工学}
{Department of Electronics and Communication, Kyoto University}

\jabstract{
日本語文章における名詞の指す対象が何であるかを把握することは,
対話システムや高品質の機械翻訳システムを実現するために必要である.
そこで,本研究では
名詞の指示性と修飾語と所有者の情報を用いて名詞の指示対象を推定する.
日本語には冠詞がないことから,
二つの名詞が照応関係にあるかどうかを判定することが困難である.
これに対して,我々は冠詞にほぼ相当する名詞の指示性を
表層表現から推定する研究を行なっており\cite{match},
この名詞の指示性を用いて名詞が照応するか否かを判定する.
例えば,名詞の指示性が定名詞ならば既出の名詞と照応する可能性があるが,
不定名詞ならば既出の名詞と照応しないと判定できる.
さらに,名詞の修飾語や所有者の情報を用い,
より確実に指示対象の推定を行なう.
この結果,
学習サンプルにおいて適合率82\%,再現率85\%の精度で,
テストサンプルにおいて適合率79\%,再現率77\%の精度で,
照応する名詞の指示対象の推定をすることができた.
また,対照実験を行なって
名詞の指示性や修飾語や所有者を用いることが
有効であることを示した.
}

\jkeywords{機械翻訳,指示対象,指示性,修飾語,所有者}

\etitle{An Estimate of Referent of Nouns in Japanese Sentences\\
 with Referential Property of Nouns}
\eauthor{Masaki Murata \affiref{KUEE} \and Makoto Nagao\affiref{KUEE}} 

\eabstract{
It is important to clarify 
referents of nouns in machine translation and \mbox{conversational} processing. 
We present a method for estimating referents of nouns in Japanese sentences 
using referential properties, modifiers, and owners of nouns.
Since there is no article in Japanese language, 
it is difficult 
to decide whether two nouns have the same referent in Japanese. 
To solve this difficult problem, 
we had made a research to estimate referential properties of nouns 
that correspond to articles using words in the sentences, 
and we estimate referents of nouns using these referential properties.
For example if the referential property of a noun is definite, 
the noun can refer to nouns that appear ahead, and
if the referential property of a noun is indefinite, 
the noun can not refer to  nouns that appear ahead. 
Furthermore we estimate referents of nouns 
using modifiers and owners of nouns more precisely.
As a result, we obtained a precision rate of 82\% 
and a recall rate of 85\% in the estimation 
of referent of nouns that have antecedents 
on training sentences, 
and obtained a precision rate of 79\% 
and a recall rate of 77\% 
on held-out test sentences. 
We verified that 
it is effective to use 
referential properties, modifiers, and owners of nouns 
through control experiments.
}

\ekeywords{machine translation, referent, referential property, modifier, owner}

\begin{document}
\maketitle


\section{はじめに}

日本語文章における名詞の指す対象が何であるかを把握することは,
高品質の機械翻訳システムを実現するために必要である.
例えば,以下の文章中の二つ目の「おじいさん」は
前文の「おじいさん」と同じなので翻訳する際には
代名詞化するのが望ましい.
\begin{equation}
  \begin{minipage}[h]{10cm}
\underline{おじいさん}は地面に腰を下ろしました.\\
\underline{The old man} sat down on the ground.\\[0.1cm]
やがて\underline{おじいさん}は眠ってしまいました.\\
\underline{He} soon fell asleep.
  \end{minipage}
\label{eqn:ojiisan_jimen_meishi}
\end{equation}
これを計算機で行なうには,
二つの「おじいさん」が同じ「おじいさん」を指示することが
わかる必要がある.
そこで,本研究では
名詞の指示性,修飾語,所有者などの情報を用いて名詞の指示対象を推定する.
このとき,
指示詞や代名詞やゼロ代名詞の指示対象も推定する.

英語のように冠詞がある言語の場合は,
それを手がかりにして
前方の同一名詞と照応するか否かを判定することができるが,
日本語のように冠詞がない言語では
二つの名詞が照応関係にあるかどうかを判定することが困難である.
これに対して,我々は
冠詞に代わるものとして名詞の指示性\cite{match}を
研究しており,
これを用いて名詞が照応するか否かを判定する.
名詞の指示性とは名詞の対象への指示の仕方のことであり,
総称名詞,定名詞,不定名詞がある.
定名詞,不定名詞はそれぞれ定冠詞,不定冠詞がつく名詞に対応する.
総称名詞には定冠詞,不定冠詞のどちらがつくときもある.
名詞の指示性が定名詞ならば既出の名詞と照応する可能性があるが,
不定名詞ならば既出の名詞と照応しないと判定できる.

以上で述べた名詞の指示性の情報だけでは
指示対象が異なる二つの名詞の指示対象が同一であると誤る場合がある.
この誤りを正すために
名詞の修飾語や所有者の情報を用い,
より確実に名詞の指示対象の推定を行なう.


\section{名詞の指示性}


名詞による照応現象の例として以下のものがある.
\begin{equation}
  \begin{minipage}[h]{10cm}
昔,昔,\underline{おじいさん}と\underline{おばあさん}が住んでおりました.\\
\underline{おじいさん}は山へ柴刈りに,
\underline{おばあさん}は川に洗濯に行きました.
  \end{minipage}
\label{eqn:ojiisan_obaasan_meishi}
\end{equation}
第一文目の「おじいさん」と第二文目の「おじいさん」は
同じおじいさんを指示し,これらは照応関係にある.

このような名詞による照応現象を解析する際に
名詞の指示性が重要になる.
指示性とは名詞の対象への指示の仕方のことである.
上の例文(\ref{eqn:ojiisan_obaasan_meishi})の
第二文目の「おじいさん」は,
定名詞句に相当する文脈上唯一の事物を指示するという指示性を持っていることから,
第一文目の「おじいさん」を先行詞として照応することがわかる.
このように名詞の指示性は
照応関係を明らかにするために重要な役割を持つ.

以下に名詞の指示性
\footnote{
名詞の指示性についてはわれわれはすでに研究している\cite{match}.
}
について説明する.
指示性の観点から,
まず名詞句は,その名詞句の類の成員すべてか
類自体を指示対象とする
{\bf 総称名詞句}と,
類の成員の一部を指示対象とする
{\bf 非総称名詞句}に分ける.
次に,非総称名詞句を指示対象が確定しているか否かで,
{\bf 定名詞句}と{\bf 不定名詞句}に分ける.
さらに,不定名詞句を
実際にその名詞の指示するものが存在しているか否かによって
特定性不定名詞句と不特定性不定名詞句に分ける
(図\ref{fig:sijisei_bunrui})
\footnote{この分類は文献\cite{meishi}を参考にした.
}
.

\begin{figure}[t]
\begin{center}
\fbox{
  \begin{minipage}[c]{270pt}
\begin{center}
{
\small
\[
\mbox{\normalsize 名詞句}
\left\{
\begin{array}[h]{ccc}
\mbox{\normalsize 総称名詞句} & \\
 & \\
\mbox{\normalsize 非総称名詞句} &
\left\{
\begin{array}[h]{cc}
\mbox{\normalsize 定名詞句}\\
 \\
\mbox{\normalsize 不定名詞句} &
\left\{
\begin{array}[h]{cc}
\mbox{\normalsize 特定性}\\
 \\
\mbox{\normalsize 不特定性} 
\end{array}
\right.
\end{array}
\right.
\end{array}
\right.
\]
}
\end{center}
  \caption{名詞の指示性の分類}
  \label{fig:sijisei_bunrui}
  \end{minipage}
}
\end{center}
\end{figure}

\paragraph{総称名詞句}

総称名詞句は,その名詞句が意味する類に属する任意の成員のすべて,
もしくはその名詞句が意味する類それ自身を指示する.
例えば,次の文(\ref{eqn:doguse})の「犬」は総称名詞句である.
\begin{equation}
\underline{犬}は役に立つ動物です.
  \label{eqn:doguse}
\end{equation}
ここでの「犬」は
「犬」という類に属する成員のすべてを指示対象としている.

総称名詞句であると判断できれば,
その名詞句は他の名詞句と照応することは
ないと判断できる
\footnote{
二つの総称名詞は照応関係になりえるが,
本研究ではこの総称名詞の照応は照応関係に含めていない.
これは,総称名詞の場合,意味が同一ならば必ず同一のものを指示するため,
多義性の問題を解消すれば照応処理が行なえたと言って良いからである.
}
.

\paragraph{定名詞句}

定名詞句は,その名詞句が意味する類に属する文脈上唯一の成員
を指示する.
例えば,次の文(\ref{eqn:thedoguse})の「犬」は定名詞句である.
\begin{equation}
その\underline{犬}は役に立ちます.
  \label{eqn:thedoguse}
\end{equation}
ここでの「犬」は,
「犬」という類に属する文脈上唯一の成員を指示対象としている.

定名詞であると判断できれば,
その名詞は既出の名詞と照応すると予想され,
先行詞を探すことになる.

\paragraph{不定名詞句}

不定名詞句は,その名詞句が意味する類に属するある不特定の成員
を指示する.
不特定の成員を指示するというのは,現時点での聞き手の情報ではその名詞句が
成員のどれを指し示すのか確定していないという意味である.
不定名詞句は総称名詞句とは異なり,その名詞句の意味する類の成員のすべてを
指示するのではなくて,その名詞句の意味する類の成員の一部を指示する.
次の文の「犬」は不定名詞句である.
\begin{equation}
\underline{犬}が三匹います.
  \label{eqn:dog3}
\end{equation}
ここでの「犬」は
犬という類に属する任意の三匹の成員を指示対象として持ちえる.
これは
どんな犬でも三匹いればこの文が使えるということである.

不定名詞句には,更に,{\bf 特定性不定名詞句}と{\bf 不特定性不定名詞句}がある.
特定性不定名詞句とは,実際にその名詞の指示するものが存在し,
それを指すことが話者に認識されている不定名詞句のことである.
不特定性不定名詞句とは,
実際にその名詞の指示するものが存在するかどうかわからず,
それを指すことが話者に認識されていない不定名詞句のことである.
例えば,上の文(\ref{eqn:dog3})の「犬」は特定性不定名詞句であり,
次の文の「犬」は不特定性不定名詞句である
\begin{equation}
\underline{犬}を飼っていらっしゃいますか?
  \label{eqn:dog_kau}
\end{equation}

不特定性の不定名詞句は,
照応関係になることはなく,
後に出てくる名詞によって指示されることはない.

それに対し,特定性の不定名詞句は,
既に出ている名詞を指示することはないが,
後に出てくる名詞によって指示される可能性がある.


\section{名詞の指示対象の推定方法}

本研究では名詞の指示対象の推定する際に,以下の三つの点に注目する.

\begin{itemize}
\item 
名詞の指示性

\item 
名詞の修飾語

\item 
名詞の所有者

\end{itemize}

これらの三つの観点から作成した条件を以下で説明する.
本研究ではこれらの条件をすべて満足するときのみ照応すると解析する.

\subsection{名詞の指示性の利用}

表層表現から名詞の指示性を推定し
\footnote{
\label{foot:sijisei}
総称名詞,定名詞,不定名詞の判定は
文献\cite{match}で行ない,
不定名詞における特定性・不特定性の判定は
``「(名詞A)が存在しない」ならば名詞Aは不特定性''という
表層表現を利用した決定的な規則により行なっている.
文献\cite{match}の利用においては次の二点の変更を行なった.
(1)名詞の意味素性\cite{imiso-in-BGH}
がPAR(動物の一部)の場合定名詞に得点を加えるという規則を追加した.
これは,動物の一部を表わす名詞は
ほとんどの場合その所有者が近くに存在しそれに限定されるために
定名詞であるからである.
(2)同一名詞が前方にある場合定名詞などに
得点を与える規則があったがこれを省いた.
これは,
同一名詞が前方にあるだけでは
それと照応するかどうかが明らかでないためである.
}
これを利用して指示対象を推定する.
推定した指示性が定名詞句の場合は前方にある同一名詞を指示対象とする.
例えば,以下の例文の二文目の「おじいさん」は,
助詞「は」がつき係る動詞が過去形であるという表層表現から
名詞の指示性が定名詞句であると推定でき,
最も近い前方の同一名詞の「おじいさん」と照応すると解析できる.
\begin{equation}
  \begin{minipage}[h]{10cm}
昔,昔,\underline{おじいさん}と{おばあさん}が住んでおりました.\\
\underline{おじいさん}は山へ柴刈りに,
{おばあさん}は川に洗濯に行きました.
  \end{minipage}
\label{eqn:ojiisan_obasan_kotai}
\end{equation}

推定した指示性が定名詞句以外の場合は,
前方の主題と焦点
\footnote{
本研究での主題と焦点はそれぞれ
表\ref{fig:shudai_omomi},表\ref{fig:shouten_omomi}により定義する.
ノ格が先行詞である場合もあるが,
主題や焦点にノ格を含めていない.
これは,
ノ格を主題や焦点に加えると全体としての解析誤りが増えるからである.
}
から指示対象を探し,
以下の三つの情報を組み合わせることにより照応するか否かを判定する.
\begin{itemize}
\item
指示性の推定における定名詞句でない度合
\footnote{
文献\cite{match}による指示性の推定では
得点を用いており,
定名詞句でないと推定した場合
得点から定名詞句でない度合が得られる.
その度合が大きいほど照応しにくくする.
この度合は,具体的には
後で述べる表\ref{tab:teimeishidenai_doai}により計算する.
}

\item
主題・焦点の重み

\item
指示対象との距離

\end{itemize}
本来は定名詞句以外の場合は
既出の名詞を指示することはないが,
名詞の指示性の推定を誤ることがあり
実際には定名詞句の可能性があるため
このような処理を行なう.

\subsection{名詞の修飾語の利用}

名詞の指示性の情報だけでは,
指示対象が異なる二つの名詞の指示対象が同一であると誤る場合がある.
例えば,以下の例文中の「左の頬」は指示性が定名詞であるので,
指示性の情報だけでは
前方の「右の頬」と照応すると解析してしまう.
\begin{equation}
  \begin{minipage}[h]{10cm}
さて,隣の家に瘤のあるおじいさんがもう一人住んでおりました.\\
このおじいさんの瘤は\underline{右の頬}にありました.\\
(中略)\\
天狗達は,前の晩に来たおじいさんから取った瘤をそのおじいさんの\underline{左の頬}に付けてしまいました.
  \end{minipage}
\label{eqn:mouhitori_ojiisan_hoho_kobu}
\end{equation}
このような誤りを生じないようにするために,
本研究では修飾語を持つ名詞については,
同じ修飾語を持つ同一名詞であることを照応する条件とする.

\subsection{名詞の所有者の利用}

修飾語のない名詞の場合でも
所有者を推定できる場合は
所有者を修飾語と同じように用いることで
適正な指示対象の推定を行なう.
すなわち,所有者を推定できる名詞の場合は,
同じ所有者を持つ同一名詞であることを照応する条件とする.
例えば,以下の文章中の「頬」は所有者が
同じ「おじいさん」であることから照応する.
\begin{equation}
  \begin{minipage}[h]{10cm}
さて,おじいさんには\underline{(おじいさんの)
  左の頬}に瘤がありました.\\
それは人の拳ほどもある瘤でした.\\
まるで\underline{(おじいさんの)頬}を(おじいさんが)膨らませているかの様に見えるのでありました.
  \end{minipage}
\label{eqn:ojiisan_hoho_huku}
\end{equation}
所有者の推定は,
意味素性\footnote{
本研究では名詞意味素性辞書\cite{imiso-in-BGH}を用いる.
}
が動物の一部を意味するPARである名詞に対してのみ行なう.
その名詞が存在する文の主語かそれまでの主題
の中から
意味素性がHUM(人間)かANI(動物)のものを探し出して,
それを所有者とする.

\subsection{補足}

前節までの説明では同一名詞の場合に限っていたが,
以下の例のようにある名詞を末尾に含む名詞が
その名詞の指示対象となることがある.
\begin{equation}
  \begin{minipage}[h]{10cm}
オーストリアの\underline{レルヒ少佐}が、日本の陸軍将校に、本格的にスキーを教えはじめたのは、明治の末のことである。
\underline{少佐}は日本の軍人たちを前にしてまず「メテレスキー」と号令した。
  \end{minipage}
\label{eqn:bubun_mojiretu}
\end{equation}
このような場合も解析できるように,
本研究では
前節までの説明での名詞が同一であるという条件を
末尾に含むという条件に変更する.
ただし,
末尾に含むという条件の場合は
同一名詞の場合に比べ照応しにくいので
これを考慮した解析を行なう.

\section{代名詞等の指示対象の推定方法}

指示詞については種類が多く,
それぞれに対して詳細に規則を作ることによって
指示対象を推定する.
代名詞は会話文章中によく現れるので,
会話文章の話し手や聞き手を把握することで
\footnote{
\label{footnote:kaiwa}
会話文章の話し手や聞き手の推定は,
その会話文章の発話動作を表す用言のガ格とニ格をそれぞれ
話し手,聞き手とすることによって行なう.
会話文章の発話動作を表す用言は,
その会話文章に「と言った.」などがつけば
それとし,そうでない場合は前文の文末の用言とする.
}
,
指示対象を推定する.
ゼロ代名詞は,
主題や焦点と格フレームによる選択制限によって
推定する
\cite{Murata_ipal95}
.
\footnote{
指示詞・代名詞・ゼロ代名詞の指示対象の推定方法については
別の機会に詳しく述べる.
}

\section{名詞等の指示対象を推定する枠組}

\begin{figure}[t]
  \leavevmode
  \begin{center}
\fbox{
    \begin{minipage}[c]{8cm}
      \hspace*{0.7cm}条件部 $\Rightarrow$ \{提案 提案 ..\}\\[-0.1cm]
      \hspace*{0.7cm}提案 := ( 指示対象の候補 \, 得点 )

    \caption{列挙判定規則の表現}
    \label{fig:kouho_rekkyo}
    \end{minipage}
}
  \end{center}
\end{figure}

\begin{figure}[t]
  \leavevmode
  \begin{center}
\fbox{
    \begin{minipage}[c]{8cm}
      \hspace*{1.5cm}条件部 $\Rightarrow$ ( 得点 )
    \caption{判定規則の表現}
    \label{fig:kouho_hantei}
    \end{minipage}
}
  \end{center}
\end{figure}


\subsection{推定の手順}
\label{wakugumi}

本研究での名詞の指示対象の推定は,
名詞の解析の手がかりとなる複数の情報をそれぞれ規則にし,
これらの規則を用いて指示対象の候補に得点を与えて,
合計点が最も高い候補を指示対象とすることによって実現する.

まず,解析する文章を構文解析・格解析する\cite{csan2_ieice}.
その結果に対して
文頭から順に文節ごとに
すべての規則を適用して指示対象を推定する.
規則には,
指示対象の候補をあげながら
その候補の良さを判定する列挙判定規則と
その列挙された複数の候補すべてに対して適用する判定規則の二種類がある.
列挙判定規則は図\ref{fig:kouho_rekkyo},
判定規則は図\ref{fig:kouho_hantei}の構造をしている.

図中の「条件部」には文章中のあらゆる語,
その分類語彙表\cite{bgh}の分類番号,
IPALの格フレーム\cite{ipal}の情報,
名詞の指示性の情報,
構文解析・格解析の結果の情報などを条件として書くことができる.
「指示対象の候補」には
指示対象の候補とする名詞の位置もしくは
「特定指示
として導入」などを書くことができる.
「特定指示として導入」のときは,
個体を特定指示として新たに導入する.
これは特定性不定名詞句
$^{\ref{foot:sijisei}}$
などのように,既出の個体を指示せず
談話に新たに特定指示として個体を導入する場合に利用される.
「得点」は指示対象としての適切さの度合を表している.

指示対象の推定は条件を満足した規則により与えられる
得点の合計点で行なう.
まずすべての列挙判定規則を適用し
得点のついた指示対象の候補を列挙する.
このとき同じ候補を列挙する規則が複数あれば得点は加えてまとめる
.
次に列挙された指示対象の各候補に対して
すべての判定規則を適用して,各候補ごとに得点を合計する.
最も合計点の高い指示対象の候補を指示対象と判定する.
最も合計点の高い指示対象の候補が複数個ある場合は,
一番初めに出された指示対象の候補を指示対象とする.

\subsection{指示対象の推定に用いる規則}

{\bf 名詞の解析のための規則}

名詞の解析は列挙判定規則のみで行なった.
列挙判定規則は9個作成したが,
それらをすべて適用順序に従って以下に示す.
すべての規則をあげたのは,
本研究のような規則を用いて解析する方法では
規則が一番重要なものであると考えるからである.



{ {\bf 名詞の解析のための列挙判定規則}
\begin{enumerate}
\item 
  「以下」「後述」の名詞や
  「次のような/次のように/次の〜点」における「次」の場合\\
  \{(次の文 \,$50$)\}
  \footnote{列挙判定規則の提案のリストを表わす.図\ref{fig:kouho_rekkyo}参照.}
  ~ \footnote{
    \label{foot:matsuoka}
    この規則は,松岡\cite{matsuoka_nl}を利用している.}

  \begin{itemize}
  \item[(使用例)]

  \underline{以下}は,中国を旅行した同僚記者の取材メモによる.\\
  北京内燃機総廠(しょう)という有力なエンジンメーカーがある.\\
  日本の会社と異なるのは重役陣が,
  住宅,医療,老後など従業員と家族の民生面まで責任を担っていることだろう.

  \end{itemize}

\item 
  「それぞれの」「各々の」「各」などに修飾された名詞の場合\\
  \{(特定指示として個体導入 \,$25$)\}

  \begin{itemize}
  \item[(使用例)]

  今回のG7では,\.そ\.れ\.ぞ\.れ\.の\hspace{-0.2mm}\underline{国}\hspace{-0.2mm}が
  経済政策の基本的な考え方を示し
  理解を求めておくべきだ.

  \end{itemize}

\item 
  「自分」の場合\\
  \{(「自分」が存在する文の主格,「自分」が主格の場合は
  「自分」を含む文の主節の主格 \,$25$)\}

  \begin{itemize}
  \item[(使用例)]

  一時は,\underline{自分(=重原勇治さん)}が経営していた
  薬品会社の社長の座を\\
  \.(\.重\.原\.勇\.治\.さ\.ん\.が\.)退いた.」

  \end{itemize}

  この例文中の「\.(\.重\.原\.勇\.治\.さ\.ん\.が\.)」は
  前方の文章から復元されたものである.

\item 
  \label{enum:定名詞探索}
  推定した名詞の指示性が定名詞の場合で,
  その名詞を末尾に含み修飾語や所有者が同じ
  名詞Aが前方にある場合
  (ただし固有名詞の場合は,修飾語や所有者の条件を無視し,
  また,末尾に限らず含まれればよいとする
  \footnote{
    修飾語や末尾の文字が異なることで異なる事物を
    指す場合があるが,
    修飾語や末尾の文字が異なっていても
    照応すると解析した方が精度が良くなるためである.
    }
  .) 
  \\ \{(名詞A 
  \,$20$)\}

\item 
  \label{enum:総称名詞導入}
  名詞の指示性が総称名詞の場合\\ 
  \{(総称指示として個体導入 \,$10$)\} 
  \footnote{
    \label{foot:soushou_dounyu}
    総称指示もしくは不特定指示として導入された個体は,
    他の名詞から指示されないようにしている.}


\item 名詞の指示性が不特定性の不定名詞の場合\\ 
  \{(不特定指示として個体導入 \,$10$)\}
  $^{\ref{foot:soushou_dounyu}}$

\item
  名詞の指示性が総称名詞でも不特定性の不定名詞でもない場合\\ 
  \{(特定指示として個体導入 \,$10$)\}

\item 
  「普通」「様」「大部」「一緒」「本当」「何」などの指示対象を持たない名詞の場合\\
  \{(指示対象なし \,$30$)\}

  \begin{itemize}
  \item[(使用例)]

  天狗達は\underline{一緒}に笑い出しました.

  \end{itemize}

  \begin{table}[t]
    \begin{center}
      \leavevmode
    \caption{定名詞でない度合}
    \label{tab:teimeishidenai_doai}
  \begin{tabular}[h]{|l|r|}\hline
  指示性の推定における得点の状況                           & 定名詞でない度合$d$\\\hline
  定名詞の得点を越える得点を総称名詞と不定名詞が持たない時 & 0\\
  定名詞の得点より1点高い得点を総称名詞か不定名詞が持つ時  & $3$\\
  定名詞の得点より2点高い得点を総称名詞か不定名詞が持つ時  & $6$\\
  定名詞の得点より3点以上高い得点を総称名詞か不定名詞が持つ時 &  規則は適用されない \\\hline
  \end{tabular}
    \end{center}
  \end{table}

\item 
  \label{enum:定名詞以外探索}
  この規則は名詞の指示性が定名詞以外の場合に適用される.
  以下の得点の説明で用いるdとwとnの説明をする.
  dは,文献\cite{match}によって推定した指示性に基づいて
  表\ref{tab:teimeishidenai_doai}から定まる定名詞でない度合である.
  wは,表\ref{fig:shudai_omomi},表\ref{fig:shouten_omomi}
  から定まる主題と焦点の重みである.
  nは,今解析している名詞と指示対象の候補とする名詞との間の距離を
  反映した数字である.

\begin{table}[t]
  \caption{主題の重み}
  \label{fig:shudai_omomi}
\begin{center}
    \newcommand{\mn}[1]{}
\begin{tabular}[c]{|l|l|r|}\hline
  \multicolumn{1}{|c|}{表層表現} & \multicolumn{1}{|c|}{例} & 重み
  \\\hline
  {ガ格の指示詞・代名詞・ゼロ代名詞} &
  (\underline{太郎}が)した.&21 \\\hline
名詞 は/には        &  \underline{太郎}はした.  &20 \\\hline
\end{tabular}
\end{center}
\end{table}

\begin{table}[t]
  \caption{焦点の重み}
  \label{fig:shouten_omomi}
\begin{center}
    \newcommand{\mn}[1]{}
\begin{tabular}[c]{|l|l|r|}\hline
\multicolumn{1}{|l|}{
{表層表現(「は」がつかないもので)}}  & \multicolumn{1}{|c|}{例}   & 重み \\\hline
{ガ格以外の指示詞・代名詞・ゼロ代名詞} & (\underline{太郎}に)した.& 16 \\\hline
{名詞 が/も/だ/なら/こそ} & \underline{太郎}がした.  & 15 \\\hline
名詞 を/に/,/.        & \underline{太郎}にした.  & 14 \\\hline
名詞 へ/で/から/より    & \underline{学校}へ行く.  & 13 \\\hline
\end{tabular}
\end{center}
\end{table}

  \{
  (修飾語や所有者が同じで
  重みが$w$で$n$個前
  \footnote{
    主題が何個前かを調べる方法は,
    主題だけを数えることによって行なう.
    主題がかかる用言の位置が
    今解析している文節よりも前の場合は
    その用言の位置にその主題があるとして数える.
    そうでない場合はそのままの位置で数える.} 
  の同一名詞の主題 \,$w-n-d+4$)\\ 
  (修飾語や所有者が同じで,
  今解析している名詞を末尾に含む
  重みが$w$で$n$個前の主題 \,$w-n-d+4-5$)\\ 
  (修飾語や所有者が同じで重みが$w$で$n$個前の同一名詞の焦点 \,$w-n-d+4$)\\ 
  (修飾語や所有者が同じで,
  今解析している名詞を末尾に含む
  重みが$w$で$n$個前の焦点 \,$w-n-d+4-5$)\}
  \footnote{\label{foot:meishi_shouou}
  提案リスト中の式 $w-n-d+4$ の一つ目の項は主題・焦点の重みが大きいほど
  指示対象として適正であることを意味する項である.
  二つ目の項は照応詞と先行詞との距離が離れているほど照応しにくいことを
  意味する項であり,
  三つ目の項は定名詞でない度合が大きいほど
  照応しにくいことを
  意味する項である.
  四つ目の項は他の規則との整合性から定めたものである.
  また,今解析している名詞を末尾に含む名詞の場合は,
  さらに $-5$ を加えているが,
  これは同一名詞に比べて照応しにくくするためである.
  }

\end{enumerate}
}

{\bf 指示詞や代名詞やゼロ代名詞の解析のための規則}

指示詞や代名詞やゼロ代名詞を解析するために,
列挙判定規則,判定規則をそれぞれ
79個,21個作成したが,そのうち主要なものを以下に示す.

{
{\bf 指示詞や代名詞やゼロ代名詞の解析のための列挙判定規則}
\begin{enumerate}
\item 「それ/あれ/これ」や連体詞形態指示詞の場合で,
  その指示詞の直前の文節に
  用言の基本形か「〜とか」などの例を列挙するような表現がある場合\\ 
  \{(例を列挙するような表現 \,$40$)\}
\item 「それ/あれ/これ」や連体詞形態指示詞の場合\\ 
  \{(前文,もしくは,指示詞の前方の同一文内に逆接接続助詞か条件形を含む
  用言がある場合はその用言 \,$15$)\}

  \begin{itemize}
  \item[(使用例)]
    おじいさんは一所懸命に歌い,そして\underline{おじいさんは}\\
    \underline{一所懸命に踊りました}\.が,
    \underline{それ}は言葉では言い表せないほど下手糞でありました.
  \end{itemize}

\item 名詞形態指示詞か「その/この/あの」の場合\\ 
  \{(重みが$w$で主題と焦点を合わせて数えて$n$個前にある,
  同一文中か前文の主題 \,$w-n-2$)\\ 
  (重みが$w$で主題と焦点を合わせて数えて$n$個前にある,
  同一文中か前文の焦点 \,$w-n+4$)\}
  \footnote{
  提案リスト中の式 $w-n-2$ ,$w-n+4$ の
  一つ目の項と二つ目の項は
  注\ref{foot:meishi_shouou}と同様である.
  最後に係数$-2$,$4$をつけたのは,
  指示詞の場合
  既知情報の主題と照応するよりも未知情報の焦点と
  照応しやすいと考えたためである.
}

\item 一人称の代名詞の場合 \{(話し手 \,$25$)\}

\item ガ格の省略の場合のデフォルト規則\\
  \{(重みが$w$で$n$個前の主題 \,$w-n*2$+1)\\
  (重みが$w$で$n$個前の焦点 \,$w-n$+1)\\
  (今解析している節と並列の節の主格 \,25)\\
  (今解析している節の従属節か主節の主格 \,23)\\
  (今解析している節が埋め込み文の場合で主節の主格 \,22)\}
  \footnote{
  提案リスト中の式 $w-n*2$+1 ,$w-n$+1 の
  一つ目の項と二つ目の項は
  注\ref{foot:meishi_shouou}とほぼ同様である.
  係数は他の規則との整合性から定めたものである.
  式$w-n*2$+1の二つ目の項は2倍されているが,
  これは主題が焦点よりも出現する割合が小さいからである.
}

  主題や焦点の定義と重みは表\ref{fig:shudai_omomi},
  表\ref{fig:shouten_omomi}のとおりである.

\end{enumerate}

以上の他に表層表現から指示対象を推定する規則がある.
}

{
{\bf 指示詞や代名詞やゼロ代名詞の解析のための判定規則}
\begin{enumerate}

\item 「ここ/そこ/あそこ」であって,
  指示対象の候補となった名詞が場所を意味する
  意味素性LOCを満足する時, $10$点を与える.

\item ソ系の連体詞形態指示詞の場合に,
  それが係る名詞Bの用例「名詞Aの名詞B」
  \footnote{
    この用例にはEDRの共起辞書\cite{edr_kyouki_2.1}を用いる. } を検索し,
  名詞Aと指示対象の候補となった名詞の類似レベルにより得点を与える.

\item 代名詞の場合に,
  指示対象の候補となった名詞が意味素性HUMを満足する時,
  $10$点を与える.

\item 指示対象の候補となった名詞と
  格フレームの格要素の用例の名詞との類似レベルにより得点を与える
  \cite{Murata_ipal95}.


\end{enumerate}

以上の他に,名詞形態指示詞の場合は
人を指示対象としにくくするための判定規則などがある.

}

\begin{figure}[t]
  \begin{center}
\fbox{
\begin{minipage}[h]{13cm}

その時お爺さんはあまり遠くない所にある空き地に
火が燃えているのに気が付きました。

赤い顔をして、鼻の青い、恐ろしい目付きの五六人の男が、
\underline{火}の周りに立っているのを見ました。

\vspace{0.5cm}

\begin{tabular}[h]{|l|r|r|r|}\hline
候補                                  & 総称指示として導入 & 前文の「火」  \\\hline
\ref{enum:総称名詞導入}番目の規則     &    10              &               \\
\ref{enum:定名詞以外探索}番目の規則   &                    &    12         \\\hline
合計                                  &    10              &    12         \\\hline
\end{tabular}

\vspace{0.2cm}

指示性の推定結果

\vspace{0.2cm}

\begin{tabular}[h]{|l|r|r|r|}\hline
指示性     &  不定名詞   & 定名詞 &  総称名詞\\\hline
得点       &  1          &  2     &  3       \\\hline
\end{tabular}

\begin{tabular}[h]{lcl}
\ref{enum:定名詞以外探索}番目の規則      & $=$ & $ w - n - d + 4 $\\
                 & $=$ & $15 - 4 - 3 + 4 = 12$\\
\end{tabular}

\caption{名詞の指示対象の推定例}
\label{tab:dousarei}
\end{minipage}
}
\end{center}
\end{figure}


\subsection{名詞の指示対象の推定例}

名詞の指示対象を推定した例を
図\ref{tab:dousarei}に示す.
これは図中の下線部の「火」の解析を
正しく行なったことを示している.
これを以下で説明する.

まず,下線部の「火」の指示性を推定するが,
助詞「の」がつく時に適用される規則
\footnote{
文献\cite{match}の研究では
助詞「の」がつく場合で
他に手がかり語がない場合は総称名詞と推定する.
「の」は旧情報と結び付きやすい性質を持っており,
ほとんどの場合は定名詞か総称名詞のいずれかである.
定名詞の場合は他の情報により推定可能になると考え,
他に情報がない場合は総称名詞と推定する.
}
により図中の
下の表の推定結果のように総称名詞と誤る.
推定結果が総称名詞であることから,
\ref{enum:総称名詞導入}番目の規則により
「総称指示として導入」という候補があげられ,
それに10点が与えられる.
また,
指示性の推定を誤った場合のことを
考慮する\ref{enum:定名詞以外探索}番目の規則によって
``前文の「火」''が候補としてあげられる.
この候補に与えられる得点は,式$w - n - d + 4$
によって与えられる.
式中の重み$w$は,``前文の「火」''につく助詞が「が」であることから
表\ref{fig:shouten_omomi}より15である.
また,今解析している下線部の「火」から見て``前文の「火」''までに
焦点が「男」「顔」「気」``前文の「火」\hspace{-1.4mm}''と4個あるので,
``前文の「火」\hspace{-1.4mm}''までの距離$n$は$4$である.
また,指示性の推定に\mbox{おける}総称名詞と定名詞の得点差が1により
表\ref{tab:teimeishidenai_doai}から定名詞でない度合$d=3$が得られる.
よって,表中の式の計算の結果,12点となる.
候補の中で合計点がもっとも高い``前文の「火」\hspace{-1.4mm}''が\mbox{指示対}象と正しく解析される.


\begin{table*}[t]
\fbox{
  \begin{minipage}[h]{14cm}
    \caption{本研究の実験結果}
    \label{tab:sougoukekka}
  \begin{center}
\begin{tabular}[c]{|p{2.4cm}|r|r@{}c|r@{}c|r@{}c|r@{}c|}\hline
\multicolumn{1}{|p{2cm}|}{テキスト}&\multicolumn{1}{|l|}{文数}&
\multicolumn{2}{l|}{名詞}&
\multicolumn{2}{l|}{指示詞}&
\multicolumn{2}{l|}{代名詞}&
\multicolumn{2}{l|}{ゼロ代名詞}\\\hline
学習サンプル                 &204 &  85\% & (130/153) &  87\% &  ( 41/ 47)  & 100\% &   ( 9/ 9)   &  86\% & (177/205) \\\hline
テストサンプル               &184 &  77\% & ( 89/115) &  86\% &  ( 42/ 49)  &  82\% &   ( 9/11)   &  76\% & (159/208) \\\hline
\end{tabular}
\end{center}
各規則で与える得点は学習サンプルにおいて人手で調節した.\\
{\small
学習サンプル\{例文(43文),童話「こぶとりじいさん」全文(93文)\cite{kobu},天声人語一日分(26文),社説半日分(26文),サイエンス(16文)\}\\
テストサンプル\{童話「つるのおんがえし」前から91文抜粋\cite{kobu},天声人語二日分(50文),社説半日分(30文),サイエンス(13文)\}
}
  \end{minipage}
}
\end{table*}

\section{実験と考察}

指示対象の推定を行なう前に構文解析・格解析を行なうが,
そこでの誤りは人手で修正した.
格フレームはIPALの辞書のものを用いたが,
IPALの辞書にない用言に対しては人手で格フレームを作成した.
本研究による方法で名詞,指示詞,代名詞,ゼロ代名詞の
指示対象を解析した実験結果を
表\ref{tab:sougoukekka}に示す.
名詞の解析精度は文中に指示対象
が存在する名詞についてのものである.
これは照応する名詞に注目したためである.
ゼロ代名詞の解析精度は
指示対象が存在するか否かがあらかじめ
わかっていると仮定して解析した時の精度である.



また,本稿であげた各手法の有効性を確かめるために
指示性の利用の仕方をかえて
表\ref{tab:sijisei_taishou}の対照実験を行なった.
表のように,
本研究の規則による方法では適合率と再現率がともに均等に良かった.
これは本研究の規則が指示性を適切に利用していることを意味している.
また,本研究の方法は
指示性を用いない方法に比べ適合率と再現率がともに良いので,
指示性を用いることが有効であることがわかる.
指示性が定名詞句と推定された名詞句のみが
照応するとした方法では再現率が悪い.
これは指示性の推定の時に
定名詞句であるのに他の名詞句と誤って推定し,
照応しないとシステムが解析したためである.
また,修飾語・所有者の条件を用いない方法では適合率が悪い.
これは,修飾語・所有者の条件を用いなければ,
指示性が定名詞句であっても
「左の頬」と「右の頬」のように照応しないものが多く,
これらを誤って照応すると解析してしまうからである.
また,末尾に含む名詞とすべて照応するとした方法ではさらに適合率が悪くなる.
これは,修飾語・所有者の条件の他に
指示性の条件を用いないため,
不定名詞句であると判定された名詞も
前方の名詞と照応すると解析するからである.

\begin{table*}[t]
  \leavevmode
\fbox{
  \begin{minipage}[h]{14cm}
\hspace*{0.5cm}
    \caption{名詞の解析における対照実験の結果}
    \label{tab:sijisei_taishou}
  \begin{center}
    \newcommand{\mn}[1]{}
\begin{tabular}[c]{|r@{}c|r@{}c|r@{}c|r@{}c|r@{}c|}\hline
 \multicolumn{2}{|p{2.3cm}|}{\mn{指示性が定名詞句と推定された名詞句のみ照応する}}& \multicolumn{2}{|p{2.3cm}|}{本研究の規則}& \multicolumn{2}{|p{2.3cm}|}{\mn{指示性を用いない}}& \multicolumn{2}{|p{2.3cm}|}{\mn{修飾語・所有者の条件を用いない}}& \multicolumn{2}{|p{2.3cm}|}{\mn{末尾に含む名詞とすべて照応する}}\\\hline
\multicolumn{10}{|l|}{学習サンプル}\\\hline
  92\% & (117/127) &  82\% & (130/159) &  72\% & (123/170) &  65\% & (138/213) & 52\% & (134/260) \\
  76\% & (117/153) &  85\% & (130/153) &  80\% & (123/153) &  90\% & (138/153) & 88\% & (134/153) \\\hline
\multicolumn{10}{|l|}{テストサンプル}\\\hline
  92\% & ( 78/ 85) &  79\% & ( 89/113) &  69\% & ( 79/114) &  58\% & ( 92/159) & 47\% & (102/218) \\
  68\% & ( 78/115) &  77\% & ( 89/115) &  69\% & ( 79/115) &  80\% & ( 92/115) & 89\% & (102/115) \\\hline
\end{tabular}
\end{center}
\small

表中の上段と下段はそれぞれ適合率と再現率を表す.
評価に適合率と再現率を用いたのは,
先行詞がない名詞をシステムが誤って先行詞があると解析することがあり,
この誤りを適切に調べるためである.

適合率は先行詞を持つ名詞のうち
正解した名詞の個数を,システムが先行詞を持つと
解析した名詞の個数で割ったもので,
再現率は先行詞を持つ名詞のうち
正解した名詞の個数を,先行詞を持つ名詞の個数で割ったものである.

「指示性が定名詞句と推定された名詞句のみ照応する」は
名詞の解析のための\ref{enum:定名詞以外探索}番目の規則をけずったものに相当し,
「指示性を用いない」は
\ref{enum:定名詞探索}番目の規則をけずり,
\ref{enum:定名詞以外探索}番目の規則のdをけずり
この規則がいずれの指示性の時でも適用されるようにしたものに相当する.
\end{minipage}
}
\end{table*}

\subsection*{誤り例}

修飾語句や所有者を利用して
指示対象の絞り込みを行なったが,
これが有効に働いた.
しかし,以下の例のように所有者の推定を誤ったために,
指示対象の推定を誤ったものがあった.
\begin{equation}
  \begin{minipage}[h]{10cm}
おじいさんは(おじいさんの)\.背\.中から柴の束を下ろして,一休みするために地面に腰を下ろしました.\\
(中略)\\
初めのうちは,おじいさんは男達を人間だと思っていましたが,間もなく天狗である事が分かりました.\\
\underline{(天狗の)背中}には大きな翼があるのです.
  \end{minipage}
\label{eqn:ojiisan_senaka_ayamari}
\end{equation}
この例の下線部の「背中」は動物の一部なので
所有者を求めるが,
正しい所有者は「天狗」であるが
主題が前文の「おじいさん」と判断されて
「おじいさん」が下線部の「背中」の所有者と誤って解析された.
このため,
この「背中」は文章のかなり前にあった
「おじいさんは(おじいさんの)\.背\.中から柴の束を下ろして」
の部分の「(おじいさんの)\.背\.中」を下線部の「背中」の指示対象と
誤って解析された.

また,修飾語句が異なっていても照応する場合があり,
このような場合は解析を誤った.
\begin{equation}
  \begin{minipage}[h]{10cm}
そこでおじいさんは\underline{近くの大きな杉の木の
根元にある穴}で雨宿りをすることにしました.\\
(中略)\\
次の日,このおじいさんは山へ行って,\\
\underline{杉の木の根元の穴}を見つけました.
  \end{minipage}
\label{eqn:ojiisan_ana_shouou}
\end{equation}
この例の下線部の「穴」は同一の穴であり照応するが,
修飾語の文字列が異なっているため照応しないと誤って解析された.
このような場合についても解析できるようにするには,
異なる表現であっても同じ意味であることを
把握できるようにする必要がある.



\section{おわりに}

本研究での手法は主に
名詞の指示対象の推定に名詞の指示性や修飾語や所有者を用いることであった.
対照実験を通じて,これらを用いることの有効性を示した.

名詞の指示性の推定精度が向上すると
名詞の指示対象の推定精度が向上すると考えている.
そこで,名詞の指示性の推定精度を向上させる研究を行なう必要がある.

\acknowledgment

本研究および実験に関して援助して下さった
松岡正男氏をはじめとする長尾研究室の皆様に感謝します.


\bibliographystyle{jnlpbbl}
\begin{thebibliography}{}

\bibitem[\protect\BCAY{井上和子, 山田洋, 河野武, 成田一}{井上和子\Jetal
  }{1985}]{meishi}
井上和子, 山田洋, 河野武, 成田一 \BBOP 1985\BBCP.
\newblock \Jem{名詞}, \Jem{現代の英文法}, 6\JVOL.
\newblock 研究社.

\bibitem[\protect\BCAY{国立国語研究所}{国立国語研究所}{1964}]{bgh}
国立国語研究所 \BBOP 1964\BBCP.
\newblock \Jem{分類語彙表}.
\newblock 秀英出版.

\bibitem[\protect\BCAY{Kurohashi \BBA\ Nagao}{Kurohashi \BBA\
  Nagao}{1994}]{csan2_ieice}
Kurohashi, S.\BBACOMMA\  \BBA\ Nagao, M. \BBOP 1994\BBCP.
\newblock \BBOQ A Method of Case Structure Analysis for Japanese Sentences
  based on Examples in Case Frame Dictionary\BBCQ.
\newblock {\Bbf E77--D}  (2), 227--239.

\bibitem[\protect\BCAY{情報処理振興事業協会技術センター}{情報処理振興事業協会
技術センター}{1987}]{ipal}
情報処理振興事業協会技術センター \BBOP 1987\BBCP.
\newblock \JBOQ 計算機用日本語基本動詞辞書{IPAL}({Basic Verbs})説明書\JBCQ.

\bibitem[\protect\BCAY{松岡, 村田, 黒橋, 長尾}{松岡正男\Jetal
  }{1995}]{matsuoka_nl}
松岡正男, 村田真樹, 黒橋禎夫, 長尾眞 \BBOP 1995\BBCP.
\newblock \JBOQ
  表層表現を利用した日本語文章における後方照応表現の自動抽出\JBCQ\
\newblock \Jem{情報処理学会自然言語処理研究会 95-NL-108}.

\bibitem[\protect\BCAY{{M. Murata and M. Nagao}}{{M. Murata and M.
  Nagao}}{1993}]{match}
{M. Murata and M. Nagao} \BBOP 1993\BBCP.
\newblock \BBOQ {Determination of referential property and number of nouns in
  Japanese sentences for machine translation into English}\BBCQ\
\newblock In {\Bem Proceedings of the 5th TMI}, \BPGS\ 218--225.

\bibitem[\protect\BCAY{村田真樹 $\cdot$ 長尾眞}{村田真樹\JBA
  長尾眞}{1995}]{Murata_ipal95}
村田真樹\BBACOMMA\ $\cdot$ 長尾眞 \BBOP 1995\BBCP.
\newblock \JBOQ 用例を用いた日本語文章におけるゼロ代名詞の指示対象の推定\JBCQ\
\newblock \Jem{「IPALシンポジウム'95」論文集}, \BPGS\ 63--66.
  情報処理振興事業協会 技術センター.

\bibitem[\protect\BCAY{中尾}{中尾}{1985}]{kobu}
中尾清秋 \BBOP 1985\BBCP.
\newblock \Jem{こぶとりじいさん 他 鶴の恩がえし、きき耳ずきん},
  \Jem{英訳「日本むかしばなし」シリーズ}, 7\JVOL.
\newblock 日本英語教育協会.

\bibitem[\protect\BCAY{(株)日本電子化辞書研究所}{(株)日本電子化辞書研究所}{199
4}]{edr_kyouki_2.1}
(株)日本電子化辞書研究所 \BBOP 1994\BBCP.
\newblock \JBOQ {EDR}電子化辞書 日本語共起辞書評価版第2.1版\JBCQ.

\bibitem[\protect\BCAY{渡辺, 黒橋, 長尾}{渡辺\Jetal }{1992}]{imiso-in-BGH}
渡辺靖彦, 黒橋禎夫, 長尾眞 \BBOP 1992\BBCP.
\newblock \JBOQ {IPAL}辞書と分類語彙表を用いた単語意味辞書の作成\JBCQ\
\newblock \Jem{情報処理学会第45回全国大会予稿集,6F-8}.

\end{thebibliography}

\begin{biography}
\biotitle{略歴}
\bioauthor{村田 真樹}{
1993年京都大学工学部電気工学第二学科卒業.
1995年同大学院修士課程修了.
同年,同大学院博士課程進学,現在に至る.
自然言語処理,機械翻訳の研究に従事.}
\bioauthor{長尾 真}{
1959年京都大学工学部電子工学科卒業.工学博士.京都大学工学部助手,
助教授を経て,1973年より京都大学工学部教授.1976年より国立
民族学博物館教授を兼任.京都大学大型計算機センター長(1986.4 -- 1990.3),
日本認知科学会会長(1989.1 -- 1990.12),パターン認識
国際学会副会長(1982 -- 1984),日本機械翻訳協会初代会長(1991.3 -- ),
機械翻訳国際連盟初代会長(1991.7 -- 1993.7).
電子情報通信学会副会長(1993.5 -- 1995.4).
情報処理学会副会長(1994.5 -- ).
京都大学附属図書館長(1995 -- ).
パターン認識,画像処理,機械翻訳,自然言語処理等の分野を並行して研究.}

\bioreceived{受付}
\biorevised{再受付}
\bioaccepted{採録}

\end{biography}

\end{document}
