



\documentstyle[epsf,jnlpbbl]{jnlp_j_b5}

\newcounter{sentence}
\newcommand{\sent}[1]{}

\setcounter{page}{49}
\setcounter{巻数}{3}
\setcounter{号数}{4}
\setcounter{年}{1996}
\setcounter{月}{10}
\受付{1995}{9}{14}
\再受付{1995}{12}{6}
\採録{1996}{2}{20}

\setcounter{secnumdepth}{2}

\title{語用論的・意味論的制約を用いた\\ 日本語ゼロ代名詞の文内照応解析}
\author{中岩 浩巳\affiref{NTTCS} \and 池原 悟\affiref{TU}}

\headauthor{中岩 浩巳・池原 悟}
\headtitle{語用論的・意味論的制約を用いた日本語ゼロ代名詞の文内照応解析}

\affilabel{NTTCS}{NTTコミュニケーション科学研究所}
{NTT Communication Science Laboratiries, NTT}

\affilabel{TU}{鳥取大学工学部知能情報工学科}
{Department of Information and Knowledge Engineering, Faculty of Engineering,
Tottori University}

\jabstract{\vspace*{-2.18mm}照応要素が同一文内に現れる日本語ゼロ代名詞に対する,語用論
的・意味論的制約を用いた照応解析の手法を提案する.本手法は,接続語のタ
イプ,用言意味属性,様相表現のタイプの3種類の語用論的・意味論的制約に
着目して,同一文中に照応要素を持つゼロ代名詞の照応要素を決定するものであ
る.本手法を日英翻訳システムALT-J/E上に実現して,日英翻訳システム評価
用例文(3718文)中に含まれる文内照応のゼロ代名詞139件を対象に,解析ルー
ルを整備し,解析精度の評価実験を行なった.その結果,上記3種類の制約条
件を用いた場合,それぞれの条件が文内照応解析に有効に働き,対象としたゼ
ロ代名詞が再現率98%,適合率100%の精度で正しく照応要素を決定できるこ
とが分かった.本手法を,従来の代表的な手法であるCenteringアルゴリズム
(再現率74%,適合率89%)と比べると,再現率,適合率共に十分高い.特に,
適合率100%と,認定した照応関係に誤りがないことから,本手法が機械翻訳シ
ステムでの実現に適することがわかった.以上の結果,提案した方式の有効性
が実証された.今後,さらに多くの文を対象に解析ルールの整備を進めること
により,同一文内照応要素を持つゼロ代名詞の大半を復元し,補完できる見通
しとなった.} 

\jkeywords{照応解析,ゼロ代名詞,文脈処理,機械翻訳,語用論,意味論}

\etitle{Intrasentential Resolution of Japanese Zero Pronouns \\
using Pragmatic and Semantic Constraints}
\eauthor{Hiromi Nakaiwa\affiref{NTTCS} \and Satoru Ikehara\affiref{TU}} 

\eabstract{\vspace*{-2.18mm}This paper proposes a method to resolve intrasentential references of 
Japanese zero\mbox{} pronouns suitable for application in widely used and practical 
machine translation systems.  This method focuses on semantic and pragmatic 
constraints such as conjunctions, verbal semantic attributes and modal 
expressions to determine intrasentential antecedents of Japanese zero 
pronouns. 
This method is highly effective because the volume of knowledge that must be 
prepared beforehand is not so large and its precision of resolution is good.
This method was realized in Japanese to English machine translation system, 
ALT-J/E. To evaluate the performance of our method, we conducted a 
windowed test for 139 zero pronouns with intrasentential antecedents in a 
sentence set for the evaluation of the performance of Japanese to English 
machine translation systems (3718 sentences). According to the evaluation, 
intrasentential antecedents could be resolved correctly for 98\% of the zero 
pronouns\mbox{} examined using rules consistent for intersentential and 
extrasentential resolution. The accuracy was higher than the accuracy of 
the centering algorithm which is a conventional method to resolve zero 
pronouns. By the further examination of the evaluation, we found that 
this method can achieve high accuracy using relatively simple rules. }

\ekeywords{Anaphora Resolution, Zero pronouns, Context processing, 
Machine Translation, Pragmatics, Semantics}

\begin{document}
\maketitle


\section{はじめに}

自然言語では通常,相手(読み手もしくは聞き手)に容易に判断
できる要素は,文章上表現しない場合が多い.この現象は,機械翻
訳システムや対話処理システム等の自然言語処理システムにおいて
大きな問題となる.例えば,機械翻訳システムにおいては,原言語
では陽に示されていない要素が目的言語で必須要素になる場合,
陽に示されていない要素の同定が必要となる.特に日英機械翻訳シ
ステムにおいては,日本語の格要素が省略される傾向が強いのに対
し,英語では訳出上必須要素となるため,この省略された格要素
(ゼロ代名詞と呼ばれる)の照応解析技術は重要となる.

従来からこのゼロ代名詞の照応解析に関して,様々な手法が提案
されている.KameyamaやWalkerらは,Centeringアルゴリズムに
基づき助詞の種類や共感動詞の有無により文章中に現われる照応要
素を決定する手法を提案した\cite{Kameyama1986,WalkerIidaCote1990}.
また,Yoshimotoは,対話文に対して文章中にあらわれる照応要素に
ついては主題をベースとして照応要素を同定し,文章中に現われな
いゼロ代名詞については敬語表現やspeech actに基づき照応要素を
同定する手法を提案した\cite{Yoshimoto1988}.
堂坂は,日本語対話における対話登場人物間の待遇関係,話者の視
点,情報のなわばりに関わる言語外情報の発話環境を用いて,ゼロ
代名詞が照応する対話登場人物を同定するモデルを提案した
\cite{Dousaka1994}.Nakagawaらは,複文中にあらわれるゼロ代名
詞の照応解析に,動機保持者という新たに定義した語用論的役割を
導入して,従属節と主節それぞれの意味的役割と語用論的役割の間
の関係を制約として用いることで解析するモデルを提案した
\cite{NakagawaNishizawa1994}.これらの手法は,翻訳対象分野を
限定しない機械翻訳システムに応用することを考えると,解析精度
の点や対象とする言語現象が限られる点,また,必要となる知識量
が膨大となる点で問題があり,実現は困難である.

ところで,照応される側の要素から見ると,機械翻訳システムで
解析が必要となるゼロ代名詞は次のような3種類に分類できる.

\begin{enumerate}
\item[(a)] 照応要素が同一文内に存在するゼロ代名詞(文内照応) 
\item[(b)] 照応要素が文章中の他の文に存在するゼロ代名詞(文間照応)
\item[(c)] 照応要素が文章中に存在しないゼロ代名詞(文章外照応)
\end{enumerate}

\noindent
これら3種類のゼロ代名詞を精度良く解析するためには,個々の
ゼロ代名詞の種類に応じた照応解析条件を用いる必要がある.また,
これら3種類のゼロ代名詞を解析するための解析ルールは,相互矛
盾が起きないように,ルールの適用順序を考慮する必要がある.こ
の3種類のうち,(b)タイプに関しては,既に,知識量の爆発を避
けるための手段として,用言のもつ意味を分類して,その語のもつ
代表的属性値によって,語と語や文と文の意味的関係を決定し,文
章中の他の文内に現われる照応要素を決定する手法をが提案されて
いる\cite{NakaiwaIkehara1993}.また,(c)タイプに関しては,語
用論的・意味論的制約を用いることによって,文章中に存在しない
照応要素を決定する手法が提案されている
\cite{NakaiwaShiraiIkehara1994,NakaiwaShiraiIkeharaKawaoka1995}

本稿では,照応要素が同一文内に存在するゼロ代名詞((a)タイプ)
に対して,接続語のタイプや用言意味属性や様相表現の語用論的・
意味論的制約を用いた照応解析を行なう汎用的な手法を提案する.

\section{日英機械翻訳システム評価用例文でのゼロ代名詞の出現傾向}
\subsection{調査対象文}
照応要素が同一文内に存在するゼロ代名詞の傾向を掴むために,
本章では独立した文(文間文脈情報が得られない文)におけるゼロ
代名詞を調査した.調査対象は,日英機械翻訳システム評価用例文
3718文\cite{IkeharaShirai1990}である.この評価用例文は,日本語
の性質と表現の種類,及び日本語と英語との相違に基づき体系化さ
れた約500種類の試験項目を評価するために,実用文中の表現を
抽出して作成された日本語例文である.個々の文には模範となる英
訳が付与されており,そのほとんどの文は文脈の情報無しに(一文
単独で)翻訳が可能である(3718文中3704文)ため,個々
の文を日英機械翻訳システムで翻訳し,予め用意された英訳と比較
することでシステムの翻訳機能の評価が行なえる.また,個々の例
文は自然な日本語文であり広範囲な表現が含まれているため,これ
らの例文におけるゼロ代名詞とその照応要素の出現傾向を調査する
ことによって,同一文内に照応要素がある場合や,文中に現われな
い照応要素の傾向を把握することが可能と期待できる.

\subsection{出現傾向}

上記の試験文に対して照応解析が必要となるゼロ代名詞とその照
応要素の出現傾向を調査した結果を表1に示す.照応要素の出現場
所からみて,同一文内に存在する場合と,同一文内に存在しない場
合に分かれる.

調査結果によれば,全ゼロ代名詞512件に対して照応要素が同
一文内に存在するゼロ代名詞が139件(27%)であった.また,
照応要素が文中に存在しない場合が373件(73%)存在した.
この373件の詳細については既に報告しているので
\cite{NakaiwaShiraiIkehara1994,NakaiwaShiraiIkeharaKawaoka1995},
ここでは照応要素が同一文内に存在するゼロ代名詞の詳細について
述べる.

照応要素が同一文内に存在するゼロ代名詞のうちでは,ガ格がゼ
ロ代名詞化され照応要素がハ格の要素となる場合が102件と最も
多い.この102件を詳細に分析してみると,このなかにはゼロ代
名詞が照応要素より文中の前の部分に存在する後方照応表現が8件
含まれることが分かった.この現象は,助詞の種類に基づく前方照
応解析手法では解析することが出来ず,接続語のタイプ等に基づく
照応解析が必要となる.

ゼロ代名詞化されたものと同じ格の要素が照応要素となる(例え
ば,ガ格がゼロ代名詞化され同一文内のガ格が照応要素となる)場
合が10件(ハ格ゼロ代名詞がハ格を照応する場合が1件,ガ格ゼ
ロ代名詞がガ格を照応する場合が6件,ヲ格ゼロ代名詞がヲ格を照
応する場合が3件)存在した.これは,接続語のタイプにより,同
一文内の格要素が共有できるかが決まるという特質を用いることで
解析可能となることが予想される.

また,照応要素が同一文内に存在するゼロ代名詞の中で,埋め込
み文又は引用文内の格要素がゼロ代名詞化されている場合が9件
(ガ格の埋め込み文内が4件,ガ格の引用文内が4件,ヲ格の埋め
込み文内が1件),照応要素が埋め込み文又は引用文内に存在する
場合が4件(ハ格の引用文内が2件,ガ格の埋め込み文内が2件)
あった.これらのゼロ代名詞を正しく解析するためには,埋め込み
文や引用文と同一文内のそれ以外の表現との意味的関係を用言意味
属性や様相表現,埋め込み文が修飾する名詞のタイプ等の情報を用
いて決定することが必要となる.

この結果から,照応要素が同一文内に存在するゼロ代名詞を解析
するためには接続語のタイプや様相や用言意味属性を用いることが
有効と推定できる.

\begin{table}[htbp]
  \begin{center}
    \caption{ゼロ代名詞とその照応要素の出現傾向}
   {\footnotesize 
(調査対象文:日英機械翻訳システム評価用例文3718文,照 \\
 応解析を要するゼロ代名詞が存在する文は463文,512件)} \\
    \vspace{2mm}
    \leavevmode
    \footnotesize
    \begin{tabular}{||c|c||c|c|c|c|c|c|c||c|c|c|c|c|c||c||} \hline \hline
      \multicolumn{2}{||c||}{ゼロ} & \multicolumn{13}{c||}{照応要素の出現場所} & \\ \cline{3-15}
      \multicolumn{2}{||c||}{代名詞} & \multicolumn{7}{c||}{同一文内} & \multicolumn{6}{c||}{文章中になし} & 小  \\ \cline{3-15}
      \multicolumn{2}{||c||}{出現} & \multicolumn{2}{c|}{は} & \multicolumn{2}{c|}{が} & & & & 受 & I & & 人 & & & 計 \\ \cline{4-4} \cline{6-6}
      \multicolumn{2}{||c||}{場所} & & 引用 & & 埋込 & を & に & 他 & & か & you & & it &  他 & [件] \\  
      \multicolumn{2}{||c||}{}     & & 文内 & & 文内 &    &    &    & 身 & we &     & 間 & &     & \\  \hline \hline
      \multicolumn{2}{||c||}{は} & 1   & 0 & 0 & 0 & 0 & 0 & 0 & 5   & 0  & 0  & 0  & 2  & 0 & 8   \\ \hline
      \multicolumn{2}{||c||}{が} & 102 & 2 & 6 & 2 & 0 & 1 & 7 & 151 & 69 & 28 & 23 & 50 & 3 & 444 \\ \cline{2-16}
       & 埋込文内                      & 3   & 0 & 1 & 0 & 0 & 0 & 0 & 15  & 0  & 0  & 2  & 0  & 0 & 21  \\ \cline{2-16}
       & 引用文内                      & 4   & 0 & 0 & 0 & 0 & 0 & 0 & 0   & 0  & 0  & 0  & 0  & 0 & 4   \\ \hline
      \multicolumn{2}{||c||}{を} & 3   & 0 & 0 & 0 & 3 & 1 & 0 & 0   & 0  & 0  & 0  & 11 & 0 & 18  \\ \cline{2-16}
       & 埋込文内                      & 1   & 0 & 0 & 0 & 0 & 0 & 0 & 0   & 0  & 0  & 0  & 0  & 0 & 1   \\ \hline
      \multicolumn{2}{||c||}{に} & 1   & 0 & 0 & 0 & 0 & 0 & 0 & 2   & 2  & 5  & 0  & 0  & 2 & 12  \\ \hline
      \multicolumn{2}{||c||}{他}     & 0   & 0 & 0 & 0 & 1 & 0 & 0 & 0   & 1  & 1  & 0  & 1  & 0 & 4 \\ \hline \hline
      \multicolumn{2}{||c||}{小計[件]} & \multicolumn{7}{c||}{139} & \multicolumn{6}{c||}{373} & 512 \\ \hline \hline 
    \end{tabular} \\ 
  \end{center}
  \label{tab:dist}
\end{table}

\section{ゼロ代名詞の同一文内照応解析}

2章で得られた結果を元に,照応要素が同一文内に存在するゼロ
代名詞の解析手法について提案する.

\subsection{助詞の種類を用いた文内照応解析}

日本語ゼロ代名詞の文章中に存在する照応要素を決定する解析手
法としては,助詞のタイプや共感動詞の有無に基づき格要素の
Centerをランク付けし,単文間の話題の継承性を認定することに
よってゼロ代名詞の照応要素を決定するCenteringアルゴリズムが
知られている\cite{Kameyama1986,WalkerIidaCote1990}.文内照
応ゼロ代名詞に対しては,文中に含まれる個々の文を単文に分割す
ることによって解析を行う.例えば,

\begin{quote}
(\sent{sent:1}) 彼は方程式を解いて(φが)答えを出した.
\end{quote}

\noindent
という文では,文を「彼は方程式を解く」と「(φが)答えを出す」
に分割し,動詞「出す」のガ格のゼロ代名詞の照応要素として助詞
「は」で主題化された「彼」が認定される.また,埋め込み文を伴
う

\begin{quote}
(\sent{sent:2}) 太郎はキムに[(φが)(φを)弁護する]ことを話した.
\end{quote}

\noindent
という文では,文を「太郎はキムに話した」と「(φが)(φを)
弁護する」に分割し,動詞「弁護する」のガ格のゼロ代名詞の照応
要素として助詞「は」で主題化された「太郎」が,ヲ格のゼロ代名
詞の照応要素としてニ格の「キム」が認定される.

本手法は,アルゴリズムが極めて簡単であるため,実現が容易で
あるが,本アルゴリズムでは解析できないゼロ代名詞が存在する.
例えば,本手法は前方照応指示を解析対象としているため,次に様
な後方照応指示表現は解析不可能である.

\begin{quote}
(\sent{sent:3}) (φが)縄を枝から枝にかけて,子供達は遊んだ.
\end{quote}

\noindent
この例ではゼロ代名詞の照応要素は後半の用言のハ格である「子供
達」となるが,ゼロ代名詞が照応要素より前の単文にあるため,こ
のアルゴリズムでは解析不可能となる.

さらに,本来文章外照応解析が必要であったり,翻訳する際に受
け身変形することによって照応解析が不要であるゼロ代名詞に対し
ても,このアルゴリズムでは文内照応とみなし解析してしまうとい
う問題も存在する.例えば次の例文を見ると,

\begin{quote}
(\sent{sent:4}) たとえ海が荒れても(φが)船を出す.
\end{quote}

\noindent
この文では「出す」のガ格がゼロ代名詞となっているが,この照応
要素は,この文のみでは推測不可能なため決定できない.しかし,
Centeringアルゴリズムでは,従文のガ格である「海」を誤って照
応要素と認定する.

2章での調査結果を検討すると,助詞の種類に基づく制約だけで
なく,接続語のタイプや用言意味属性,様相表現をゼロ代名詞の同
一文内照応格要素の推定に用いれば,より正確に照応要素が決定で
きると予想される.例えば,(3)の表現においても,接続語が様態
を示す「て」であり,ゼロ代名詞を含む用言が動作を示すことから,
子供達が遊ぶ様子を示す表現であることが推測され,ゼロ代名詞の
照応要素は「子供達」であると判断できる.

\newpage

\subsection{論用論的・意味論的制約を用いた文内照応解析}

2章の調査結果の分析によると,接続語のタイプや用言意味属性
や様相表現のタイプがゼロ代名詞の文内照応要素を決定するのに有
効であることが分かった.本節では,2章で調査した例文を詳細に
検討することにより得られた,接続語,用言意味属性,様相表現の
3種類による文内照応要素を決定するための語用論的・意味論的制
約について述べる.

\subsubsection{接続語による制約}

接続語は,ゼロ代名詞の文内照応要素を決定するうえで最も強力
な制約となることが期待される.これは,接続語のタイプに応じた
格の共有に関する制約にもとづくものである.南\cite{Minami1974}や田窪
\cite{Takubo1987}らが提案しているとおり,日本語の接続語は,格要素
の影響範囲を制限するものがある.例えば,南は,日本語接続語を
A,B,Cの3種類に分類し,主題「は」格も助詞「が」格も共有
するような「つつ」や「ながら」のような継続を示す接続語をA類,
「は」格は共有するが「が」格は共有しないような「ので」や「た
ら」のような条件を示す接続語をB類,「は」格も「が」格も共有
しないような「けれど」や「けど」のような接続語C類と呼んだ.
この分類によるとA類の接続語を伴う複文において,この接続語を
はさんだ片方の単文のガ格がゼロ代名詞化されもう片方の単文にガ
格が存在する場合には,このゼロ代名詞の照応要素はもう片方の単
文のガ格の要素となる.このような,接続詞の種類による格要素の
特徴を,照応要素の決定に活用することができる.

表2に,実際に本手法で用いる接続語による文内照応解析条件の
1部を示す.これは,南による日本語接続語の分類を,南の記述に
ない接続語や英語に翻訳すると接続語となるような日本語表現にも
拡張し,実際の文に現れた文内照応ゼロ代名詞の分析に基づいて,
接続語前後の単文内の格の共有の特性を整理しルール化したもので
ある.

\begin{table}[htbp]
  \begin{center}
    \caption{接続語によるゼロ代名詞の文内照応解析条件}
    \vspace{2mm}
    \leavevmode
    \footnotesize
    \begin{tabular}{||c|c|c||} \hline \hline
    接続語の例 & ゼロ代名詞の条件 & 照応要素との関係{\scriptsize 1)}\\ \hline \hline
    から,し,ば & ハ格 & 従文 \( \rightarrow \) 
 主文 \\ \hline
    ため         & ハ格 & 従文 \( \leftarrow \) 
主文 \\ \hline
    まま         & ハ格,ガ格 & 従文 \( \rightarrow \)
主文 \\ \hline
    たり,て     & ハ格,ガ格 & 従文 \( \leftrightarrow \)
主文 \\ \hline
    と           & ハ格,ヲ格 & 従文 \( \rightarrow \) 
主文 \\ \hline
    つつ,ながら{\scriptsize 2)} & ハ格,ガ格,ヲ格 & 従文 \( \leftrightarrow \) 
主文 \\ \hline \hline
    \end{tabular} \\ 
   {\scriptsize 
    1) この矢印は,照応要素の含む文から,これと照応可能なゼロ代名詞を含む文への方向を示す\\
    2) ``つつ'',``ながら''の場合,ヲ格はその接続関係が「逆接確定」の場合のみ補完対象となる }
  \end{center}
  \label{tab:conj}
\end{table}

\subsubsection{用言意味属性による制約}

用言意味属性による制約は大きく以下の2種類に分かれる.

\vspace{2mm}
\noindent
{\bf (a)用言意味属性による制約}

\vspace{1mm}
ゼロ代名詞又はその照応要素が,埋め込み文又は引用文内にある場
合の文内照応解析は,埋め込み文又は引用文の表現と文内のその他
の表現との意味的関係を認定ことが必要となる.この意味解析に,
文中の用言の用言意味属性が有効となると期待される.例えば,

\begin{quote}
(\sent{sent:5}) 港区は資本参加すると(φが)言った.
\end{quote}

\noindent
と言う表現においては,用言「言う」のガ格がゼロ代名詞化されて
おり,その照応要素は引用文「港区は資本参加する」のハ格となる.
この表現においては,引用文の用言「参加する」はガ格の属性変化
を示す用言意味属性を持ち,用言「言う」はガ格の要素が引用文の
内容を精神的移動させるという用言意味属性を持つ.このような用
言意味属性の属性対によって,この文は「助詞ハで主題化された要
素が属性変化を起こすという情報を同じハ格の要素が伝える」とい
う意味をもつことが認定でき,このゼロ代名詞の照応要素が「港区」
となると決定できる.このように,接続語のタイプで照応要素が
決定できない場合でも,用言の意味属性を利用することで用言間の
意味的関係が認定でき,文内照応が可能となる.

\vspace{2mm}
\noindent
{\bf (b)用言意味属性と接続語による制約}

\vspace{1mm}
3.2.1で示した接続語のタイプによる格の共有による制約は,
文内照応解析に極めて有効であると期待される.しかし,接続語の
種類によっては1種類の接続語が複数の接続語タイプの曖昧性を持
つ場合がある.例えば,接続語「て」は,接続語の前後の文の関係
に関して,動作の様態を示すA類の意味,時間関係や原因を示すB
類の意味,並列表現を示すC類の意味がある. 例えば,(3)の文
における接続詞「て」は,子供達の遊び方を示し,様態を示すAタ
イプの意味となる.これに対して,次の文では,

\begin{quote}
(\sent{sent:6}) 彼は成長して(φが)立派な紳士になった.
\end{quote}

\noindent
接続詞「て」は原因理由を示し,B類の意味となる.よって,この
ような接続語に対しては,接続語の前後の用言意味属性,様相表現
の種類の共起によって接続語の意味を特定し接続語のタイプを決定
する必要がある.例えば,(3)の文では,用言「かける」用言「遊
ぶ」の用言意味属性はともに身体動作であり,ハ格の要素が用言
「かける」の後ろにあることから,様態の関係を示すことが決まる
.同様にして,(6)の文では,用言「成長する」用言「成る」の
用言意味属性はともに属性変化であることから,属性変化することに
よって属性変化するという関係を示す文であると認定でき,原因理
由の関係を示すことが決まる.これにより,用言「成る」のガ格の
ゼロ代名詞の照応要素を用言「成長する」のハ格である「彼」と決
めることができる.このように,意味の曖昧性を持つ接続語の解
析に,用言意味属性は有効であり,これによって文内照応要素が正
確に決定できる.さらに,これは,接続語を正確に機械翻訳するた
めにも必要な処理となる.

\subsubsection{様相表現による制約}

様相表現は,ゼロ代名詞の文章外照応要素を決定するうえで最も
強力な制約となる
\cite{NakaiwaShiraiIkehara1994,NakaiwaShiraiIkeharaKawaoka1995}.
例えば,ガ格がゼロ代名詞化して
いる場合には,様相表現「〜したい(φ want to〜)」(希望)や
「〜してほしい(φwant φ to〜)」(三人称希望・使役)を伴う
と,照応要素は``I''になり,様相表現「〜してはいけない(φmust 
not〜)」(禁止)や「〜するべきだ(φshould)」(義務)を伴う
と,照応要素は``you''になると言える.このような特長は,文内照
応解析にも有効となる.例えば,
 
\begin{quote}
(\sent{sent:7}) 彼が天文クラブ員なので (φが)あの星を知っているだろう.
\end{quote}

\noindent
という表現では,接続語「なので」がB類であるので,同一文内で
ガ格の要素が必ず共有するとはいえないため,接続語のタイプのみ
では照応解析できない.しかし,推量を示す様相表現「だろう」が
用言「知る」に伴っているので,ガ格は"I"以外の要素が来ると予
測され,「彼」が照応要素となる.このように,文内照応解
析においても様相表現により照応要素を決定することが出来る.
\vspace*{-5mm}
\subsection{アルゴリズム}

3.1と3.2で示した議論を元に,同一文内に照応要素を持つ
ゼロ代名詞の解析アルゴリズムについて提案する.但し,日本文を
英訳するうえで必須要素となるゼロ代名詞のみを解析対象とする.
前述の条件をアルゴリズム化する際には,照応要素が文章中に現わ
れない場合や,他の文に現われる場合の解析精度も考慮にいれて,
全体的にゼロ代名詞の解析精度が良くなるように実現する必要があ
る.処理アルゴリズムは,以下のとおりである.なお,各ステップ
において文章内外の照応要素を決定する際には,アルゴリズム中に
記した条件だけではなく,用言がゼロ代名詞に課す意味的制約を満
たすかも検証する.
\vspace*{2.5mm}
\begin{enumerate}
\item[Step-1] ゼロ代名詞を検出する.(例えば,\cite{NakaiwaIkehara1993}で提案し
た手法で検出する)\\
もし検出されれば,現在解析中の文のタイプによって処理を分類
する.複文・重文の場合はStep-2へ,単文の場合はStep-3へ.
\item[Step-2] 複文・重文におけるゼロ代名詞の照応解析を下記の順で行う.
\begin{enumerate}
\item[1)]用言意味属性,様相表現および接続語の種類による文内照応
解析.(条件:3.2節・用言意味属性による制約(b),3.2節・様相表現による制約)
\item[2)]接続語のタイプによる文内照応解析.(条件:3.2節・接続語による制約)
\end{enumerate}
照応要素が決定すれば解析終了,決定しなければStep-3へ.
\item[Step-3] 現在解析中の文に埋込文や引用文が含まれている場合,用言意味
属性の制約を用いた文内照応解析.(条件:3.2節・用言意味属性による制約(a))\\
照応要素が決定すれば解析終了,決定しなければStep-4へ.
\item[Step-4] 他の文に照応格要素が存在するか調査する.(例えば,
\cite{NakaiwaIkehara1993}の手法で調査する) \\
照応要素が決定すれば解析終了,決定しなければStep-5へ.
\item[Step-5] ゼロ代名詞が支配する用言の用言意味属性,様相表現および接続
語の種類による文章外照応解析
\cite{NakaiwaShiraiIkehara1994,NakaiwaShiraiIkeharaKawaoka1995}.\\
照応要素が決定すれば解析終了.
\item[Step-6] 照応要素が決定できない場合,用言がゼロ代名詞に課す意味的制
約により照応要素を推測.また受け身変形可能な場合は受け身化
し解析終了.
\end{enumerate}

\section{評価}

\subsection{評価の方法}

3章で提案した文内照応解析の方法を,日英機械翻訳システム
ALT-J/Eの上に実現し,作成したルールが整合性を保ちつつ正しい
解析結果を与えるか,ルールは容易に作成できるかの2点を中心に,
評価を行った.
実験条件は以下の通りである.

\subsubsection{解析対象}

日英機械翻訳システム評価用例文3718文中のゼロ代名詞512
文の内,文内照応解析が必要なゼロ代名詞139件を解析対象とし
た.

\subsubsection{照応解析ルール}

上記139件のゼロ代名詞に対して,3章の方法で作成した70件
の規則を使用した\footnote{現状では,構文解析等の段階で失敗す
る文を本技術の評価に使用するのは困難である.そこで,ここでは,
提案した手法の技術的限界を見極めるため,ルールの整合性の検証
を評価の第1の目的とし,ルール作成に使用した標本(デバッグさ
れた文)を,評価に使用(ウインドウテストと)した.今後,シス
テム全体のデバッグを待って,ブラインドテストによる評価も行っ
ていく予定である.}.また,本手法の解析精度を客観的に調査する
ため,Centeringアルゴリズムを用いた場合の解析精度も調査した.

\subsubsection{用言意味属性体系}

図1に示すような用言の意味属性(107分類)を,日英機械翻訳
システムALT-J/Eの日英構造変換用パターン対辞書(約15,00
0パターン)に付与し,それを利用した\cite{NakaiwaYokooIkehara1994}.

\begin{figure}[htbp]
\begin{center}
\fbox{\epsfile{file=VSA.epsf,scale=0.65}}

\end{center}
\caption{用言意味属性体系}
\label{fig:vsa}
\end{figure}

\subsubsection{ルール適用文}

文内照応解析のための上記70ルールを,文内照応解析が必要な1
39件に加えて,文献
\cite{NakaiwaShiraiIkehara1994,NakaiwaShiraiIkeharaKawaoka1995}
の手法で文章外照応解析が
可能となるゼロ代名詞(ガ格-I or we, ガ格-you,ガ格-人, ガ格
-it, ニ格-youの5種類,175件)および,受け身変形することによ
り照応解析が不要となるゼロ代名詞173件の計487件にも適用
した.これにより,上記139件が正しく解析出来るかに加えて,
本来,文章外照応が必要であったり,照応解析が不要であるゼロ代
名詞に,誤って文内照応解析ルールが適用されないかを調査した.

\subsubsection{評価項目}
解析規則の種類と解析精度の関係を調べるため,以下の2項目に分
けて評価した.

\begin{itemize}
\item 照応解析のための制約条件と解析精度の関係\\ 
接続語,用言意味属性,接続表現に関する3種類の制約条件と解析
精度の関係を評価した.
\item 照応解析ルールの複雑さと解析精度の関係\\
照応解析ルールの複雑さを定式化し,それと解析精度の関係を求め
た.
\end{itemize}
\vspace*{-3mm}
\subsubsection{評価尺度}

本評価では,以下の2種類の解析精度を示す評価尺度を用いた.

\begin{itemize}
\item 再現率\\ 
文内照応解析が必要となるゼロ代名詞139件なかで正しい照応要
素が決定できたゼロ代名詞の割合である.
\item 適合率\\
上記70件の文内照応解析ルールを用いて文内照応であると認定さ
れ文内照応要素が決定されたゼロ代名詞のなかで,正しい照応要素
が決定できたゼロ代名詞の割合である.
\end{itemize}

ゼロ代名詞の照応解析手法を機械翻訳システム上に実現する場合
を考えると,再現率の低下と適合率の低下は異なった影響を訳文
品質に与える.まず,誤って照応解析した要素の修正(後編集)
という観点から考えると,再現率の低下に影響する解析誤りは,
補えなかった照応要素発見しそれを補う後編集作業を必要とし,
適合率の低下に影響する解析誤りは,誤って補った要素を発見し
それを正しい照応要素に置き換える後編集作業を必要とする.
両者の後編集の作業量を比較すると,前者は,訳語表現が受身表
現に変換されたり照応要素未定のマークが訳文中に示されるのに
対し,後者は,文の流れや原言語表現を詳しく調査することが必
要になるので,適合率の低下の方が再現率の低下より悪影響が大
きいと言える.また,得られた訳文品質の優劣の観点から見ると,
適合率の低下に影響する解析誤りは,誤って補った要素により決
定的な内容の誤解が生じる恐れが大きいのに対して,再現率の低
下に影響する解析誤りは,補われなかった要素の解釈が読者に任
されるため,訳文品質への被害が少ないと言える.以上のことか
ら,再現率が低い場合に比べ適合率が低い場合の方が大きく影響
するため,機械翻訳システム上で実現する際には,適合率の方が
再現率より重要であると言える.

\subsection{評価結果}
\subsubsection{照応解析の条件と解析精度の関係}

照応解析条件と解析精度の関係を調べるために,3章で提案した
手法の再現率と適合率を次の4種類の条件で評価した.

\begin{itemize}
\item 接続語による制約のみを用いた場合
\item 接続語と用言意味属性による制約を用いた場合
\item 接続語と様相表現による制約を用いた場合
\item 接続語と用言意味属性と様相表現による制約を用いた場合
\end{itemize}

また,本評価では,本手法の解析精度を客観的に計るため,3.
1で示したCenteringアルゴリズムを適用した場合の解析精度も評
価した.

表3に照応解析条件と解析精度の関係を示す.この表から,3.
2に示す条件をすべて用いることにより,139件中136件のゼ
ロ代名詞の文内照応解析がルールの不整合なしに正しく行なわれ,
Centeringアルゴリズムより再現率,適合率とも精度が高いことが
分かる.また,接続語による条件に用言意味属性による条件を追加
することによって,再現率,適合率ともにCenteringアルゴリズム
より高い値が得られることから,用言意味属性導入の効果が分かる.
また,用いた条件にかかわらず本手法はCenteringアルゴリズムより
高い適合率を得ている.

次に,本評価結果の中から,Centeringアルゴリズムでは正しく
解析できないが,提案手法では正しく解析できた例を示す.例文
(4) では,照応要素がこの文だけでは決まらず文内照応解析処理
としては決定する必要がないが,Centeringアルゴリズムでは,従
文の「海」を照応要素と認定してしまい,適合率を低下させてしまっ
た.しかし,我々の手法では,接続語「ても」がB類でありガ格は
照応要素のならないため,文内に照応要素は存在しないと認定し,
適合率の低下をまねかなかった.さらに,

\begin{quote}
(\sent{sent:8}) 土地が転売される中でどんどん(φが)値上がりする.
\end{quote}

\noindent
では,「土地」が用言「転売される」を修飾すると構文解析される
ので,用言「値上がりする」のガ格がゼロ代名詞となる.この解析
結果をもとに単文に分割すると「(φが)〜中でどんどん値上がり
する」と「土地が転売される」になり,本表現はもともとは後方照
応指示表現ではないが,この解析結果をもとにCenteringアルゴリ
ズムを適用しようとすると,後方照応指示表現と等価になり,この
アルゴリズムでは解析不可能となる.しかし,我々の手法では,2
種類の用言の用言意味属性と様相表現より,「転売される」と「値
上がりする」の意味的関係が認定され,「土地」が照応要素として
正しく認定される.

上記の結果より,本手法で用いたそれぞれの条件が文内照応解析
に有効に働いていており,本手法は機械翻訳システム上での実現に
適した手法であることが言える.

\begin{table}[htbp]
  \begin{center}
    \caption{照応解析条件と解析精度の関係}
    \vspace{2mm}
    \leavevmode
    \footnotesize
    \begin{tabular}{||c|c|c||} \hline \hline
    ゼロ代名詞 & \multicolumn{2}{c||}{解析精度} \\ \cline{2-3}
    照応要素   & 再現率 & 適合率 \\ 
    解析条件   & & \\ \hline \hline
    接続語     & 71\% (98/139) & 96\% (98/102) \\ \hline
    接続語+用言意味属性 & 88\% (123/139) & 98\% (123/125) \\ \hline
    接続語+様相表現     & 73\% (101/139) & 97\% (101/104) \\ \hline
    接続語+用言意味属性+様相表現 & 98\% (136/139) & 100\% (136/136) \\ \hline \hline
    Centering アルゴリズム & 74\% (103/139) & 89\% (103/116) \\ \hline \hline
    \end{tabular} \\ 
  \end{center}
  \label{tab:eval-cond}
\end{table}

\subsubsection{照応解析ルールの複雑さと解析精度の関係}

照応解析ルールの複雑さに対する解析精度を検討するために,3.
2で提案した手法の解析精度をルールの複雑さに応じて評価した.
ここでルールの複雑さCは,接続語,様相表現,用言意味属性に対
する条件が1箇所あるとそれぞれ1点とし,その積算を複雑さとし
た.

\noindent
C = 接続語による制約の数 + 用言意味属性による制約の数 + 接続語による制約の数 

\noindent
この計算によると,例えば,接続語の条件があり,主文に用言意味
属性,従文に対して様相表現と用言意味属性の条件がある場合には,
様相表現1点 \( \times \) 1 + 用言意味属性1点 \( \times \)  2 + 
接続語1点 \( \times \) 1 の計算により複雑さは4となる.

表4に照応解析ルールの複雑さと解析精度の関係を示す.この結
果によると,139種類中136種類のゼロ代名詞を照応解析する
ために用いられたルール数は70種類であった.また,複雑さが3
以下のルール(58種類)のみを用いた場合の解析精度は再現率が
88%,適合率が99%,4以下のルール(66種類)のみを用い
た場合は再現率が94%,適合率が100%と,簡単なルールだけ
でもルール間の不整合を起こさずに高い解析精度が得られることが
分かった.この結果から,接続語,用言意味属性,様相表現を用い
て文内照応解析をおこなうことにより,比較的単純なルールにより
高い解析精度が得られることが分かった.

\begin{table}[htbp]
  \begin{center}
  \caption{照応解析ルールの複雑さと解析精度の関係}
    \hspace{2mm}
    \footnotesize
    \begin{tabular}{||c|c|c|c|c||c|c||} \hline \hline
      \multicolumn{3}{||c|}{解析条件} & ルールの & ルール数 & \multicolumn{2}{c||}{解析精度}\\ \cline{1-3}\cline{6-7}
      接続語 & 用言意味属性 & 様相表現 & 複雑さ & & 再現率 & 適合率\\ \hline \hline
      1 & 0 & 0 & 1 & 38 & 71\% (98) & 96\% (98) \\ \hline
      1 & 0 & 1 & 2 & 40(+2) & 72\%(+1\%) (100(+2)) & 97\% (100/103) \\ \hline
      1 & 2 & 0 & 3 & 57(+17) & 87\%(+15\%) (121(+21)) & 99\% (121/122) \\ \hline
      1 & 0 & 2 & 3 & 58(+1)  & 88\%(+1\%) (122(+1)) & 99\% (122/123) \\ \hline
      1 & 2 & 1 & 4 & 65(+7)  & 94\%(+6\%) (130(+8)) & 100\% (130/130) \\ \hline
      2 & 2 & 0 & 4 & 66(+1)  & 94\%(+1\%) (131(+1)) & 100\% (131/131) \\ \hline
      2 & 3 & 0 & 5 & 68(+2)  & 96\%(+2\%) (134(+3)) & 100\% (134/134) \\ \hline
      2 & 2 & 2 & 6 & 70(+2)  & 98\%(+2\%) (136(+2)) & 100\% (136/136) \\ \hline \hline
    \end{tabular} \\
  \end{center}
  \label{tab:eval-comp}
\end{table}

\section{まとめ}

本論文では,接続語,用言意味属性,様相表現の語用論的・意味
論的制約を用いた日本語ゼロ代名詞の文内照応解析手法を提案した.
本手法は,接続語のタイプ,用言意味属性,様相表現のタイプによ
りゼロ代名詞の同一文内に存在する照応要素が決まるという語用論
的・意味論的な制約に着目し,文の表現のタイプに応じて文の意味
を決定し,照応要素を決定するものである.本手法を日英翻訳シス
テムALT-J/E上に実現して,日英翻訳システム評価用例文(371
8文)中のゼロ代名詞を有する文を対象に,照応解析ルールを整備
した状態で性能評価を行った.本標本実験によると,英語表現で訳
出が必要な同一文内に照応要素を持つゼロ代名詞(139件)が再
現率98%,適合率100%の精度で正しい照応要素を決定でき,
従来の代表的な手法であるCenteringアルゴリズムを用いた場合
(再現率74%,適合率89%)より高い精度が得られることがわ
かった.特に,適合率100%と認定した照応関係に誤りがないこ
とから,本手法が機械翻訳システムでの実現に適することがわかっ
た.また,照応解析条件と解析精度の関係から,それぞれの条件が
文内照応解析に有効に働いていることが分かった.さらに,使用し
た照応解析ルールの複雑さと解析精度の関係から,高い解析精度が
比較的簡単なルールを記述することで得られる(複雑さ3以下のルー
ルで再現率87%,適合率99%)ことが分かった.以上の結果,
語用論的・意味論的制約を用いた本手法の有効性が実証され,これ
らの制約を用いたルールを蓄積することによって,補完すべき要素
が同一文内に存在する省略格要素の大半が復元できるという見通し
を得た.

今回の実験では,日英機械翻訳システム評価用例文を対象にウイ
ンドウテストで本手法を評価したが,今後は,本手法で提案した4
種類の制約を用いた照応解析ルールをより多く蓄積し,様々な文種
の例文を用いてブラインドテストによる評価を行っていきたい.ま
た,現在,この照応解析ルールは人間による分析結果をもとに人手
で作成しているが,この照応解析ルールの自動獲得に関する検討も
行っていく予定である.
 

\acknowledgment

本研究を進めるにあたりご指導いただいた河岡司同志社大
学教授,松田晃一NTTコミュニケーション科学研究所所長に感
謝致します.また,日頃熱心に討論していただくNTTコミュニケー
ション科学研究所翻訳処理研究Gの皆様に感謝いたします.


\bibliographystyle{jnlpbbl}
\bibliography{jpaper}

\newpage
\appendix
\noindent
{\bf 具体的な照応解析ルール}

本文中で説明に利用した例文中のゼロ代名詞の照応解析に利用する具体的な文内
照応解析ルールを下表にまとめる.なおここでは,文内照応解析における語用論
的・意味論的制約の有効性の説明に利用した4文のルールのみについて示す.

\begin{table}[htbp]
\label{tab:standard1}
\hbox to\hsize{\hfil
    \leavevmode
    \footnotesize
    \begin{tabular}{||c|c|l|l||} \hline \hline
    利用する制約 & 文番号 & 照応解析条件 & 照応要素 \\ \hline \hline
    用言意味属性 & (5) & 引用文の主語=「ハ格」                  & 主文の主語の \\
                 &     & 引用文の用言意味属性=「ガ格の属性変化」& ゼロ代名詞は \\
                 &     & 主文の主語=「ゼロ代名詞」              & 引用文の主語 \\
                 &     & 主文の用言意味属性=                    & のハ格を照応 \\
                 &     & 「ガ格が引用文の内容を精神的移動」      & \\ \hline
    用言意味属性 & (3) & 接続語=「て」                          & 従文の主語の \\
         +      &     & 主文の主語=「ハ格」                    & ゼロ代名詞は \\
       接続語    &     & 主文の用言意味属性=従文の用言意味属性  & 主文の主語の \\
                 &     & =「ガ格の身体動作」                    & ハ格を照応   \\
                 &     & 従文の主語=「ゼロ代名詞」              & \\ \cline{2-4}
                 & (6) & 接続語=「て」                          & 主文の主語の \\
                 &     & 従文の主語=「ハ格」                    & ゼロ代名詞は \\
                 &     & 主文の用言意味属性=従文の用言意味属性  & 従文の主語の \\
                 &     & =「ガ格の属性変化」                    & ハ格を照応   \\
                 &     & 主文の主語=「ゼロ代名詞」              & \\ \hline
    用言意味属性 & (7) & 接続語=「なので」                      & 主文の主語の \\
         +      &     & 従文の主語=「ガ格」&``I''以外         & ゼロ代名詞は \\
       接続語    &     & 従文の用言意味属性=「ガ格がヲ格を知覚する」& 従文の主語の \\
         +      &     & 主文の主語=「ゼロ代名詞」              & ガ格を照応   \\
      様相表現   &     & 主文の様相表現=「だろう」              & \\ \hline \hline
    \end{tabular} \hfil}
\end{table}

    
\begin{biography}
  \biotitle{略歴} \bioauthor{中岩 浩巳}{
    1985年法政大学工学部電気工学科卒業.
    1987年名古屋大学大学院工学研究科電気系専攻博士前期課程終了.
    同年,日本電信電話(株). 以来,日英機械翻訳技術の研究に従事.
    現在,NTT コミュニケーション科学研究所主任研究員.
    1995年9月より1年間マンチェスタ工科大学(UMIST)客員研究員.
    言語処理学会,情報処理学会,人工知能学会,ACL各会員.}
  \bioauthor{池原 悟}{
    1967年大阪大学工学部電気工学科卒業.1969年同大学大学院修士課程終了.
    同年,日本電信電話公社に入社.
    以来,電気通信研究所において数式処理,トラヒック理論,自然言語処理の研究に従事.
    1996年より鳥取大学工学部知能情報工学科教授,スタンフォード大学客員教授.
    工学博士.
    1982年情報処理学会論文賞,1993年情報処理学会研究賞,
    1995年日本科学技術センター賞(学術賞),1995年日本人工知能学会論文賞受賞.
    言語処理学会,電子情報通信学会,情報処理学会,人工知能学会各会員.}

\bioreceived{受付}
\biorevised{再受付}
\bioaccepted{採録}

\end{biography}

\end{document}
