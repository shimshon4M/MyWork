






























\documentstyle[epsf,jnlpbbl]{jnlp_j_b5}




\setcounter{page}{87}

\setcounter{巻数}{3}

\setcounter{号数}{4}

\setcounter{年}{1996}

\setcounter{月}{10}

\受付{1995}{11}{13}

\再受付{1996}{1}{29}

\採録{1996}{4}{19}




\setcounter{secnumdepth}{2}





\title{縮退型共起関係を用いた学習機能付き係り受け解析システム}

\author{安原 宏\affiref{OKI} }





\headauthor{安原 宏}

\headtitle{縮退型共起関係を用いた学習機能付き係り受け解析システム}




\affilabel{OKI}{沖電気工業(株)研究開発本部 (現在,財団法人
イメージ情報科学研究所)}
{Media Laboratory, OKI Electric Industry Co.,Ltd. ~(Now working in
Laboratories of Image Information Science and Technology)}




\jabstract{
実用的な自然言語処理を開発するには大規模な言語資源が必要になる.語彙解析では
辞書が共通の言語資源である.一方,構文解析では文法規則が主流になってくる.規
則ベースは抽象的で解析時の挙動を理解することは困難であり,規則の規模が大きく
なると保守改良が困難になるという課題がある.本論文では実際の文章から縮退型共
起関係という2文節間の係り受け関係を品詞と付属語列で表現するデータを抽出し,
係り受け解析の唯一の言語資源として利用したシステムを示す.本方式を用いて40
00文から抽出した8000の縮退型共起関係データを用いたプロトタイプシステム
を構築し,50文の新聞社説で評価実験を行ったところ,80%の係り受けが成功し
た.本システムの特徴として,規則駆動では困難な構文規則を学習したり拡張するこ
とが共起関係によるデータ駆動の良さから実現可能になっていることが挙げられる.}




\jkeywords{係り受け,依存文法,構文解析,共起関係,連続句}





\etitle{Kakari-Uke Dependency Analysis\\ With Learning Function\\
Based On Reduced Type Cooccurrence Relation}

\eauthor{Hiroshi Yasuhara \affiref{OKI}}




\eabstract{
Large scale language resources are key materials for practical natural
language processing. The most common language resource is a dictionary
which plays an important role in lexical processing. On the other hand many
syntactic processing systems are based on context free grammars of phrase
structure. CFG rules take complementary position of lexical data resource.
In general the rules are absolute and difficult to get exact image in the
analysis system. These properties make the syntactic analysis difficult to
understand the total behavior when the size of rules grows. In this paper 
reduced type cooccurrence relations are collected from real text as a
unique language resource of the Japanese kakari-uke dependency analysis.
The data is a simple binary relation format of phrase dependency. It is
extracted automatically using a syntactic analysis. The prototype system
with eight thousand of the reduced cooccurrence relations showed eighty
percent accuracy in kakari-uke dependency analysis of editorial articles.
The system provides learning and incremental facility for the cooccurrency
relation database. }




\ekeywords{Kakari-Uke, Dependensy Grammar, Syntax Analysis, Cooccurrence
Relation, Continuous Phrase}



\begin{document}

\maketitle










\section{まえがき}


自然言語処理のための言語リソースとして語彙辞書が最も基本となるが,構文構造の
基本となる構成要素は,2文節間あるいは2単語間の係り受け構造である.係り受け
関係は,CFG規則の最も単純な形式であるチョムスキ標準形と見なすことができる.
通常この関係は共起関係と呼ばれている.本論文は,文法規則というよりは言語デー
タの一種と見なせる共起関係を用いて日本語の係り受け解析を行い,かつ更新,学習
機能を取り入れることにより,カナ漢字変換に見られるような操作性の良さを有する
簡便な日本語係り受け解析エンジンを提示することを目的とする.

これまで共起関係による自然言語解析には,\cite[など]{Yoshida1972,Shirai1986,TsutsumiAndTsutsumi1988,Matsumoto1992}の研究がある.\cite{Yoshida1972}は本論
文に最も関係するもので係り受けによる日本語解析の基礎を与えるものである.
\cite{Shirai1986}は日本語の共起関係の記述単位として品詞と個別単語との中間に位置
すると見なせるクラスター分類で与えるとともに半自動的にインクリメンタルに共起
辞書を拡大することを述べている.\cite{TsutsumiAndTsutsumi1988}は英語に関して
動詞の格ペアーとして共起関係を捉えている.\cite{Matsumoto1992}は英語構文解析
の規則に共起関係を抽出する補強項を付け加へ,2項以上の多項関係を解析時に自動
的に抽出している.しかし,いずれのシステムも共起関係だけから実用規模の係り受
け解析を構築したものはない.

一般に共起関係は\cite{Yoshida1972}を除き係り側の自立語と付属語(機能語)列お
よび受け側の自立語(終止形)で論じられることが多い.その際,係り側の付属語は
両方の自立語の表層格(関係子)として考えられている.\cite{Yoshida1972}は二文
節間の関係に着目して受け側も自立語と付属語列として考察した.さらに機械処理の
観点から,付属語・補助用言・副詞などの語は個々の単語で記述し,他の語は品詞水
準で扱った.これを準品詞水準と称している.本論文では,副詞も含めてすべて自立
語は品詞で記述し,付属語列はリテラルで表現することにする.品詞に縮退させてい
るためこれを縮退型共起関係あるいは省略して単に共起関係と呼ぶ.

本論文では,実際の文章から機械的に抽出した係り受け関係を共起データとし,いわ
ゆる文法規則の類を一切使用せずに係り受け解析システムを構築する.その際,共起
関係の構文情報の中に連続性の概念を導入して,これまで文法的には曖昧であるとさ
れていた構造も本質的に曖昧性が解消出来ているのではないが,実際の文章では出現
頻度が少ないとか,分野を限定すれば同一文体が続く傾向があるために係り受けパタ
ーンを絞り込めるのではないかと予想して開発した.これは最近研究の盛んなコーパ
スに基づく統計的言語処理の一つの試みにもなる.また単純な形式の共起関係のみを
用いて解析を行うため,日本語の係り受け解析で一文ごとに規則に相当する共起関係
を学習する機能を持たせることができ,共起関係の更新機能と併用することで従来の
ものと比較して,柔軟性,拡張性に富んだシステムが得られる.

以下,\ref{data-str}章では,構文構造と共起関係のデータ構造を定義する.
\ref{new-ana}章では本共起関係を用いた学習機能付き日本語係り受け解析システムを
説明する.\ref{eval}章では解析システムの実験結果を示し,評価を行う.

\section{データ構造}
\label{data-str}

構文解析で使用する言語リソースのデータ構造についてその定義を与える.第1は,
係り受け関係の表現ための構文木,第2は,係り受け関係の基本データ構造である縮
退型共起関係である.

\subsection{構文木}

日本語の係り受けを解析することをここでは構文解析ともいう.解析結果を与
える構文は,工学的観点にたって見やすく工夫している.
図\ref{構文解析結果の表示例}にその例を示す.
各文節は順序番号を付与して係り先の文節番号
と係り受け関係を明示しグラフで表示している.文節は長い文節にも対応でき
るように縦に配置している.文節の先頭の品詞も明示している.詳細は省くが,
このグラフ表示に対して形態素,構文レベルの修正が可能である.修正機能は
100%正解が困難な現状の自然言語処理にとって有用な支援機能である.グ
ラフの上段は形態素解析の結果を示している.

\begin{figure}[htb]
\begin{center}
\epsfile{file=fig1.eps,width=119mm}

\caption{構文解析結果の表示例} \label{構文解析結果の表示例}
\end{center}
\end{figure}

\subsection{係り受けの連続性}

文中で連続した文節が係り受けの関係になっている場合,その係り受けの2文節は「
連続」あるいは「連続句」であるという.文中で離れた文節間で係り受け関係があれ
ば「不連続」という.日本語においては特定の係り受け関係は連続して生ずるケース
が見られ,これに注目するため連続,不連続を区別している.

連続,不連続を付与した係り受け関係においては,連続して用いられる度合の強い係
り受け関係は解析において単語に次ぐ重要な言語リソースで,辞書の見出しを単語か
ら句に拡大するとき最初にエントリの候補となるものである.

\subsection{縮退型共起関係}

縮退型共起関係では,文節の自立語列はヘッドとなる最終の自立語の品詞で,付
属語列はリテラルとしてそのままの文字列で表現される.句読点も
付属語列に含める.係り受け関係は例えば\cite{YamagamiAndYasuhara1993}に
示したような係り受け関係名で記述する.単に係り受けの関係だけを解析する
のであれば関係名は係り側の付属語列で代用するか,あるいはまったく無視し
ても本処理系には影響を与えない.図\ref{データ構造}に縮退型共起関係のデー
タ構造とその例を示す.
\vspace*{5mm}

\begin{figure} [htb]
\begin{center}
\epsfile{file=fig2.eps,width=131mm}
\caption{縮退型共起関係のデータ構造とその例} \label{データ構造}
\end{center}
\end{figure}

大文字A,Bは自立語品詞,Fは付属語品詞列,小文字a,bは自立語リテラル,f
は付属語列のリテラルを表わすとすると,一般に共起関係を構文規則として見て,
抽象化の強いものから並べると次の4種類に分類される.1)AFBF型,2)Af
Bf型,3)Afbf型あるいはafBf型,4)afbf型である.2),3),
4)のように具体的に記述するにしたがって共起関係の数は増大する.通常の構文規
則は1)のレベルで記述するのが一般的である.

縮退型共起関係は2)の型に属し,その特徴は文節の自立語を品詞で代用しているこ
とと受け側の付属語も添付していることである.付属語は付属語相当語を含めて現在
650語を用いているがオーダ的には組み合わせや句読点を考慮して1000程度で
あり,リテラルとした.自立語は10万のオーダになりそれらをすべて記憶しておく
ことは2項関係としては動詞が1万とすると名詞と動詞の組み合わせだけでギガのオ
ーダであり,付属語パターンも含めると現状の機械処理の立場からは困難である.こ
れは\cite{Yoshida1972}でも示された考え方であった.

本論文では,付属語列の末尾に句読点情報も付与している.なぜなら,現実の文章に
おいては,句読点が係り受け関係の決定に意味を持っているからである.これの有効
性は従属節の述語間の係り受けに対して\cite{Shirai1995}でも示されている.

係り受け関係で受け側の付属語列を伴ったAfBf型が重要なことは,AfBでは次
にどのような付属語列が来るかはBのみの品詞情報で決定され,文節Aの付属語列は
関与しないことによる.しかし文法的には等価であるが慣用的な用法が多い自然言語
では,特定の付属語を好む場合がある.例えば,文「\underline{欧米においては}電
車で寝ているような光景は決して\underline{見られないが,}」で,「欧米において
は」は動詞に係るというだけの情報では,この文の場合「寝ている」に係るとしてし
まうことになる.AfBf型にするとBに付属語「〜ないが,」が付くパターンを優
先して使用することができる.もっと単純な場合では,「N(名詞)はA(形容詞)
」の規則で,「彼は若い.」と「彼は長い(トンネルを)」を同じと見てしまう.し
かしBにfを付与すると「.」が効力を持つ.もちろん組み合わせの数はAfBf型
がAfB型より3桁多くなるが,係り受けの絞り込みは良くなる.

付属語列をリテラルにしておくメリットは,「男は車に\underline{近づいた.}」の
縮退型共起関係である「NはV\underline{た.}」\hspace*{-1mm}により「男は近づいた車を見た.
」\hspace*{-1mm}の文で「男は」\hspace*{-1mm}は\hspace*{-1mm}「近づい\underline{た}」\hspace*{-1mm}ではなく\hspace*{-1mm}「見\underline{た.}」\\に係
る.また「女は赤い服を着た.」で,「女は」は「赤い」に係らないで「着た.」に
係っている.「N\underline{助詞}A」といったAFB型の共起記述では「女は」は
「赤い」に係ってしまう.もちろん自立語で品詞を使用することによって「髪が長い
少女に会った.」の類に関しては「NがA」の規則が生まれて上記の文の解析を誤ら
しめて良くないがこれは頻度情報や後述する学習機能によって避けられると考えてい
る.別の解決法は名詞の品詞を細分類したり,\cite{Shirai1986}のようにクラスタ
ー分割することが一つの解決策であるが,本論文では単純な方法を採用した.

縮退型共起関係と連続句の概念によって解析の曖昧性は次のように解消できる.
\cite{Yoshida1972}では,「彼は山に登って景色を見た.」に対して3つの解析例が
ある
としているが,連続句を優先すれば「山に」は「登って」に係り,縮退型共起関係「
NはVて」が無いか,あるいは在っても「NはVた.」に比較して頻度が少なければ
「彼は」は「見た.」に係かる.

\subsection{共起データベース}

共起データベースCOODBとは,係り受け解析の出力から得られる縮退型共起関
係を蓄積したデータベースである.但し,頻度は1であるので記憶していない.
各レコードは,縮退型共起関係からなっている.それらを文に対応させて記憶
することにより,任意の縮退型共起関係から逆にそれが使用された原文を参照
することが可能になる.これにより与えられた縮退型共起関係が正しいかどう
かの判断を実例文で確認することができる.図\ref{COODBの例}に共起データ
ベースの例を示す.共起データベース中の縮退型共起関係をソートして頻度を
付与したものをソート済み共起データベースSTCDBと呼ぶ.
\vspace*{3mm}

\begin{figure} [htb]
\begin{center}
\epsfile{file=fig3.eps,width=101mm}
\caption{縮退型共起関係データベースCOODBの例}
\label{COODBの例}
\end{center}
\end{figure}

\section{新解析系}
\label{new-ana}

\subsection{係り受け文法の定式化}

AfBf型の係り受け関係を形式化するとCFGと等価なことが分かる.さらに
図\ref{データ構造}でも示したように係り受け関係には頻度や確率が付与されるた
め確率付
きCFGと見なすことも可能である.文節の文法カテゴリをγBf,δBf等で表わすと
一般の係り受け関係は,
\begin{eqnarray}
δBg  =     γBf     γBg
\label{cc}
\\
δBg  =     δBf     γBg
\label{dc}
\\
δBg  =     γBf     δBg
\label{cd}
\\
δBg  =     δBf     δBg
\label{dd}
\end{eqnarray}
で表現することが出来る.ここで,γは規則が最初に適用される文節カテゴリに付与
しており,δは1度以上規則を適用してできたカテゴリである.カテゴリの添え字γ
,δは,意味的には連続,不連続と関係させたものである.連続のものは規則が適用
されると文節間にギャップができるため左辺のカテゴリにはγは現われない.またB
は品詞,fは付属語リテラルに相当する.~(\ref{cc})の右辺の”γBf γBg”は
文節カテゴリγBfがγBgに連続して係ることを示す.
(例.机を=>運ぶ.)~(\ref{dc})の意味はδBfがγBgに連続して係ることを表
わしている.(例.(大きな
=>)少年の=>頭には)~(\ref{cd})はγBfが不連続にδBgに係る.(例.少年
が〜>(机を=>)運ぶ)~(\ref{dd})はδBfが不連続にδBgに係ることを示して
いる.(例.(大きな=>)少年が〜>(机を=>)運ぶ.)ここで=>は連続した
係り受け関係を表わし,〜>は不連続の係り受け関係を意味している.
以上の~(\ref{cc}),~(\ref{dc})の規則は連続フラグが立っている縮退型共起関係
から生成され,
~(\ref{cd}),~(\ref{dd})
は不連続な縮退型共起関係から生成したものである.もう少し緩い規則として連続,
不連続のいずれも上記4つの規則に展開しておくことを考えてもよい.なぜなら,一
般に日本語においては連続した共起関係は不連続でも発生し得るし,逆も可能である
からである.いずれにしてもSTCDBをチョムスキー標準形のCFG規則に変換すること
が出来る.しかしこれは従来のCFGの規則数と比較してオーダの違った規則群になる
.解析するのは原理的にCYK法のようなボトムアップ解析によれば可能である.その
場合,係り受けの交差も自動的に回避できる.ここでは全解パージングではなくコス
ト等を導入して最尤解を求めるために\ref{ana-sys}に述べるような独自な系を作成し
た.

\subsection{解析系}
\label{ana-sys}

以下で縮退型共起関係を用いた解析系を記述する.
\begin{itemize}
\item[(1)] システム構成

縮退型共起関係を用いた解析システムの構成を図\ref{システム構成}に示す.

\begin{figure} [htb]
\begin{center}
\epsfile{file=fig4.eps,width=113mm}
\caption{係り受け解析システム構成}
\label{システム構成}
\end{center}
\end{figure}

本システムで使用する特徴的な言語リソースを以下に示す.いわゆる解析規則はなく
,共起関係データベースがその代りになっている.共起関係データベースはフィード
バック系になっている.

\clearpage

\begin{itemize}
\item[(a)] 係り受けマトリックス

\begin{figure} [htb]
\begin{center}
\epsfile{file=fig5.eps,width=106mm}
\caption{係り受けマトリックス}
\label{マトリックス}
\end{center}
\end{figure}
\vspace*{-2mm}

これは,\cite{Yoshida1972}でも与えられているが,
本論文では図\ref{マトリックス}に示すように行列要素の値を係り受けの可否では
なくコストで与え
る点とさらにそれらの関係名を記述する点で相違する.

\item[(b)]ソート済み縮退型共起関係データベースの作成

COODB,STCDBともに初期状態は空である.先ず実際の文章からCOODBを収集し,ソー
トして重複頻度付きの形式でSTCDBを作成する.係り受け解析の結果は共起関係で表
現できるから,先ず何らかの手段で解析データを収集する.人手でやるのも可能だが
量的に限界がある.解析の結果に対して機械的に収集すればよい.すなわち解析その
ものに共起関係を利用することが最終目標であるが,そのためのデータ収集は,人手
で解析されたデータや別の解析ツールを使用することができる.これによって本解析
エンジンで最低限度の解析ができる程度の容量になるまで蓄積する.今回は,本解析
方式(図\ref{システム構成}中の日本語係り受け解析2)とは別の解析システム(日
本語係り受け解析1)を用いてCOODBの自動抽出ツールを作成した.STCDBのデータ構
造はB木や配列として記憶しておく.解析エンジンが始動してそこからデータを収集
すればブートストラップになる.
\end{itemize}
\item[(2)]解析ステップ

以下,解析のステップを順に述べる.
先ず,システムは形態素解析を行い,文節に区切られた結果を構文解析に渡す.構文
解析では係り受けマトリックスをSTCDBを用いて作成する.その後,文頭の文節から
順にその係り先を係り受けマトリックスを用いて非交差条件を守りながらコストの低
いものを優先して決定する.したがって係り先の文節はコストが同じなら距離が近い
ものほど優先することになる.連続する共起関係は,学習で選んだものを除いて最も
コストを低くしている.最終的に最小コストの係り受け関係を一つだけ出力する.

コストを例えば1から6に設定すると,1は最小コストで6が最大コストになる.後
述する学習した共起関係のコストは1とし,連続する共起関係は2,不連続は頻度に
よって3から5とし,疑わしい共起関係は6とする.データ量が多くなり飽和してく
れば頻度の代りに確率を使うことも考えられるが,現状は1文づつの更新で常時頻度
が変化しているため確率や頻度計算をすることは避け,連続,不連続だけで選択して
いる.

\item[(3)]更新機能および学習機能

本方式による係り受け解析は100%の正確さではない.従ってユーザには,
失敗に対して,係り先や係り受け関係を修正する機能が提供されている.
\ref{data-str}章で述べたように係り受け解析の結果はグラフ表示が出来るた
めユーザは任意の係り受け関係を画面上で係り元,係り先および関係名を番号
で指示することにより修正することができる.

係り受け関係を修正すると,縮退型共起関係の4種類の更新機能を聞いてくる.ユー
ザはいずれかを選択する.具体的には,1)何もしない.2)現在の共起関係自身が
間違っている疑いがある.3)修正結果は新規の共起関係としてSTCDBに登録する.
および4)学習機能である.2)の場合は直接削除することはやめて,疑問符を付け
ておき,コストも最大にする.後日,COODB等を用いて適否を検討するようにしてい
る.4)の学習機能は係り先を変更したり,古い係り受け関係を新しい関係で置き換
えることによって起動し,古い関係はコストを高めることにより優先度を下げ,当該
共起関係の選択を抑止するようにする.同時に新しく指定した共起関係に対しては~
図\ref{データ構造}(a)に示した学習フラグをセットして最小のコストを付与す
る.頻度を高めていく方法もあるが,学習効果を即時に得るためこの方法を採用した
.1文の解析が終了するとこの文で指示されたSTCDBに対する上述の共起関係の追加
,修正および学習が実行される.

以上のステップから,更新機能によってインクリメンタルにSTCDBが拡大して
いくとともに学習機能によって優先順位が更新され最近の選択結果を優先する
ことが可能になる.同じ構文をこの後に実行すると優先順位に逆転が起こり,
正しい係り受け解析が得られる.

\begin{figure} [htb]
\begin{center}
\epsfile{file=fig6.eps,width=137mm,height=90.5mm}
\caption{学習の例}
\label{学習}
\end{center}
\end{figure}

図\ref{学習}に学習例を示す.構文解析が学習によって適応規則を変更してい
くため,インタラクティブな環境ではワープロのカナ漢字変換に似た学習効果
が得られる.学習・更新効果の評価は別の機会にゆずるが,一般的に述べると
本方式の特徴は付属語リテラルのパターンを用いることにある.「ですます調」
とか「だ文」とかの文体あるいは「〜ですか」のような会話文独特の表現はい
ずれも付属語が文体を代表しており効果が現われやすい.
\end{itemize}

\section{解析実験と評価}
\label{eval}

\subsection{共起データの統計的情報}

縮退型の共起関係が係り受け解析にどの程度有効かを調査するために,先ず一つの文
章を順に110文だけ解析した.これは本解析エンジンとは別の解析を用いた.1つ
の文章から順々に文を解析しているから言い回しが似ていて共起関係が重複して出現
することが期待できる.110文を10文ずつに分割して解析し,その中に重複して
含まれる共起関係を抽出した.その結果を表\ref{10sent}に示す.

\begin{table}[htb] \caption{10文毎の共起関係の重複度}
\label{10sent}
\begin{center}
\begin{tabular}{|l|l|l|l|l|l|l|l|l|l|l|l|}
\hline
   &000&010 &020 &030&040&050&060&070&080&090&100\\
文番号& - & - & - & - & - & - & - & - & - & - & - \\
   &009&019&029&039&049&059&069&079&089&099&109\\ \hline
共起デ& & & & & & & & & & &\\
ータ数&79&89&86&67&78&67&90&79&71&28&69\\ \hline
ソート& & & & & & & & & & &\\
共起デ&76&76&75&60&68&59&81&66&70&16&65\\
ータ数& & & & & & & & & & &\\ \hline
重複数&3&13&11&7&10&8&9&13&1&12 &4\\
重複率&4\%&15\%&13\%&10\%&13\%&12\%&10\%&16\%&1\%&43\%&6\%\\ \hline
累積重複 & &24 &24&28&24 &23&24 &25&10&20&13\\
(重複率)& &27\%&28\%&42\%&31\%&34\%&27\%&32\%&14\%&71\%&19\%\\ \hline
累積数&79&168&254&321&399&466&556&635&706&734&803\\
ソート&76&141&203&242&296&340&406&460&521&529&585\\
収縮率&96\%&84\%&80\%&75\%&74\%&72\%&73\%&72\%&74\%&72\%&73\%\\ \hline
\end{tabular}
\end{center}
\end{table}

共起データ数は文の文節数によって変化するが,これらを累積すると次第にそれ以前
に解析した共起関係と重複することが多くなる.10文毎の重複率は10文内限定と
それ以前のものも利用するのとでは平均約2倍に高まっている.重複したも
のを一つにカウントしたソート共起データ数と重複を別々にカウントした出現個数の
比を収縮率と呼ぶと,文数が増大すると収縮の度合が拡大する傾向が出ている.すな
わち収縮率が小さいことは,新規の共起関係の出現率が少ないことを示す.文数をさ
らに3000文にまで増大した例が表\ref{3000sent}である.100文では70%前
半であった収縮率は,1000文では50%,3000文では40%前半になった.

\begin{table}[htb] \caption{3千文による共起関係の収縮率の変化}
\label{3000sent}
\begin{center}
\begin{tabular}{|c|c|c|c|}  
\hline
&1000文&1000文&1000文\\ \hline
共起データ数&3467&6900&6453\\
ソート数&1878&3241&3335\\
収縮率&54\%&47\%&52\%\\ \hline
累積共起データ数&3467&10367&16820\\
累積ソート数&1878&4588&7190\\
収縮率&54\%&44\%&43\%\\ \hline
\end{tabular}
\end{center}
\end{table}

表\ref{freq}はSTCDB中の8413種類の共起関係の頻度別の分布を示したものであ
る.1回しか出現しない共起関係が80%を占めており,これは,まだ共起関係が収
束していないことを意味している.この程度ではまだ共起パターンが拡散するのであ
ろう.ちなみに最高出現頻度509回のものから順に上位7つを示すと,「NのNを
」(509回),「NをV(連体形)」(291回),「NのNが」(231回),
「NのNの」(230回),「NはV.」(223回),「NをV.」(220回)
,「NのNは」(195回)である.

\begin{table}[htb] \caption{頻度別の共起関係の要素数}
\label{freq}
\begin{center}
\begin{tabular}{|r|r||r|r||r|r||r|r||r|r||r|r|} \hline   
頻度 &要素数&頻度 &要素数&頻度 &要素数&頻度 &要素数&頻度 &要素数&頻度
&要素数\\ \hline \hline
1&6671&2&805&3&298&4&153&5&93&6&61\\ \hline
7&52&8&38&9&29&10&22&11&19&12&11\\ \hline
13&8&14&11&15&7&16&11&17&9&18&11\\ \hline
19&11&20&6&21&3&22&5&23&4&24&2\\ \hline
25&1&26&4&27&6&28&2&29&1&30&2\\ \hline
31&3&32&1&34&1&36&5&38&1&39&4\\ \hline
40&2&42&3&44&2&46&2&47&1&49&3\\ \hline
50&2&55&1&56&1&58&1&60&1&62&1\\ \hline
78&1&82&2&85&1&87&1&89&1&90&1\\ \hline
91&1&96&1&101&1&102&1&119&1&121&1\\ \hline
124&1&152&1&195&1&220&1&223&1&230&1\\ \hline
231&1&291&1&509&1&&&&&合計&8413\\ \hline
\end{tabular}
\end{center}
\end{table}


\subsection{解析実験}

本方式をWS上にインプリメントした.実験で使用したSTCDBのレコード数は上記の共
起関係の収集で使用した約4000文から得られた約8000種である.この程度の
蓄積では必ずしも日本語の共起関係をカバーしていないことは,~表\ref{3000sent}
で示したように収縮率が4割程度であることからも明らかである.つまり6つの文節
からなる文では5つの係り受け関係が生じるのであるが,その内2つ程度を新規に追
加していく必要が残っていることになる.収縮率を裏返せば成功率は60%以下とい
うことになる.

具体的な実験で説明すると,本システムを利用して新たに新聞社説から100文を入
力し,係り受け解析を行った.その結果を~表\ref{100sent}及び~表\ref{50sent}に
示す.~表\ref{50sent}の50文は,~表\ref{100sent}の100文中で形態素分割が
正しく出力されているものに限定して選択した.構文規則が全く白紙の状態から,4
000文の共起関係を記憶することにより未登録の係り受けを除けば新聞社説の50
文に対して係り受けは成功率79%である.未登録を失敗とすれば59%の成功率に
なる.
\begin{table}[htb] \caption{100文の係り受け解析実験結果} \label{100sent}
\begin{center}
\begin{tabular}{|c|c|c|} \hline   
係り受け総数&1文当り平均係受数&未登録共起関係数\\ \hline
853&8.53&276(32\%)\\ \hline
\end{tabular}
\end{center}
\end{table}


\begin{table}[htb] \caption{50文に対する係り受け成功率} \label{50sent}
\begin{center}
\begin{tabular}{|c|c|c||c|c|} \hline   
係り受け総数&平均文節数&未登録共起関係数&係り受け成功数&係り受け失敗数\\ \hline
389&8.73&100(26\%)&228(79\%)&61(21\%)\\ \hline
\end{tabular}
\end{center}
\end{table}



共起関係を大規模に収集すれば,収縮率が漸近的に小さくなっていくことは上記のデ
ータからも予想できるが,収縮率が10%を切る(すなわち成功率が90%を超える
)にはどの程度の共起関係を収集すればよいかは現時点のデータでは不十分である.
成功率は文の長さにも依存する.5文節程度の短い文に限れば成功率を90%以上に
することは不可能ではない.

上記の実験データは,一般文章の係り受けを解析した場合であった.もし解析範囲を
データベースのフロントエンドや質問応答システム等の自然言語インタフェースある
いは天気予報といった特定分野に限定すると,文節数は少なくなり,使用するパター
ンも制限されたものが使われるものと思われ,本方式の特徴である規則作成の容易性
及びインタラクティブな学習更新機能が即効的に作用することが期待できる.例えば
天気予報文でよく出現する「Na(日中)はNb(南)のNc(風)で,」や
「Na(明日)はNb(冬型)のNc(気圧配置)で,」といった表現では
「Naは〜>Ncで,」の係り受け規則が適用できる.ところが一般の文章で
は「Na(最近)はNb(計算機)のNc(おかげ)で,」や「Na(大統領)
はNb(ボストン)のNc(ホテル)で,」のように「Naは〜>Ncで,」
の係り受け規則を適用できないものもある.係り受けの曖昧性の解消が分野を限定
すればどの程度の効果を持つかは今後実験を重ねて行く必要がある.

\subsection{課題}

評価実験の解析失敗事例からいくつかの課題が出てきた.以下それらを列挙する.共
通事項として,品詞を細分類することや意味情報を用いることが課題解決に必要である.

\begin{itemize}
\item[(1)] 品詞の見直し

「この」「あの」と「小さな」は品詞としてはどちらも連体詞であるが,次のように
異なる構文役割を持っている.これは品詞体系の課題である.
「身体(が|の)=>小さな人」は良いが,「N (が|の)=>(この|あの)」
は成立しない.「一番=>大きな」は良いが,「ADV=>(この|あの)」は成立
しない.

\item[(2)] 「NをNに」型

「産業を中心に」,「技術等を対象に」,「今春を目標に」等がこのタイプになるが
,受け側の語彙をサブの品詞カテゴリによって何らかの形でグループ化することが必
要である.

\item[(3)] 並列句

並列句は構文情報だけでは解釈困難であるから本方式の限界でもある.

\hspace*{-1mm}「私は新聞\underline{と}本\underline{を}読む.」\hspace*{-2mm}と\hspace*{-1mm}「私は弟\underline{と}
本\underline{を}読む.」\hspace*{-2mm}の区別には意味情報が必要となる.
並列構造の解析を\cite{nagao1994}のように形態素解析と係り受け解析の間に挟む
のが妥当であろう.

\item[(4)] 省略(文脈)

「私は(t)赤いのがよい.」(t)には文脈によって「服は」,「車は」,「ワイン
は」などが入る.これは一文だけでは「本は面白いのがよい.」のと品詞レベ
ルでは区別不可能で,係り受けが曖昧になる.これも意味情報あるいは品詞細分類が
必要である.
\end{itemize}

\section{むすび}

2文節間の係り受け解析において,共起関係を実際のテキストから抽出し,頻度情報
を付与したり,学習機能を利用すれば,構文解析規則に代わる言語リソースに成り得
ることを示した.これは規則駆動による解析から事例駆動,データ駆動による解析へ
の一つの例になる.共起関係は見方を変えれば,文脈自由文法になるが,共起関係は
文法規則と言うよりはどちらかといえば共起辞書に近い.その理由は,形式が単純で
あり個数が10万や百万のオーダになるからである.

現在,8500程度の共起関係で解析システムが動作しているが,これほど単純な規
則を用いて一定水準の動作確認が出来たことは自然言語インタフェース等への応用の
可能性を示唆している.また,法律,経済,医学等の個別分野における共起関係の分
布パターンを特徴抽出することも興味ある研究課題である.

解析性能の向上には,本論文で述べた原理的な方法以外に,AfBfCf等の3文節
以上の多文節間の係り受け関係の導入,並列句を含めた係り受けの曖昧性解消策とし
て従来から研究が進んでいる意味情報の利用,afbf,Afbf等の自立語も含め
た文節リテラルによる例外規則としての付与,規則利用による冗長な付属語列を持つ
共起関係のコンパクト化など様々な付加手続きを利用して,データ収集と性能評価を
行う必要がある.縮退型共起関係を用いた構文解析が実際の自然言語処理の場で使用
されるようにするには,たとえばタスクを限定して共起関係の登録数を増大させ新規
の縮退型共起関係の登録回数を減らすことが第一の目標になる.もちろん量の増大に
伴う副作用も検討していかなければならない.

係り受け解析が,2項関係あるいはチョムスキ標準形のような単純なデータ構造を基
本として,そのデータ上でのいくつかの条件付与で可能であるとすると,係り受け解
析が複雑で抽象的な句構造規則を指向したものではなく,語彙辞書と単純なパターンで
ある縮退型共起関係といったデータ指向の延長線上でかつ推論規則よりも単純なパタ
ーンマッチングによって実現可能になることが期待できることから本方式は人間の言
語習得に関する研究においても検討材料を提供する可能性があるものと考えられる.










\bibliographystyle{jnlpbbl}


\bibliography{coop}







\begin{biography}

\biotitle{略歴}

\bioauthor{安原 宏}{

1969年京都大学 理学部卒業.
1972年同大学院修士課程修了.
同年,沖電気工業株式会社入社.1982年〜1992年第五世代コンピュータプロジェクト
に従事.1986年〜1995年(株)日本電子化辞書研究所.1987年〜1995年(財)国際情
報化協力センター機械翻訳システム研究所.1996年〜(財)イメージ情報科学
研究所勤務.自然言語処理、概念をベースとした情報処理に興味を持っている.} 



\bioreceived{受付}

\biorevised{再受付}

\bioaccepted{採録}



\end{biography}



\end{document}

