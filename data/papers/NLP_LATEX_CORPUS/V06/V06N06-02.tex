



\documentstyle[epsf,jnlpbbl]{jnlp_j_b5}

\setcounter{page}{27}
\setcounter{巻数}{6}
\setcounter{号数}{6}
\setcounter{年}{1999}
\setcounter{月}{7}
\受付{1998}{9}{28}
\再受付{1998}{12}{8}
\採録{1999}{1}{14}

\setcounter{secnumdepth}{2}

\title{5W1H情報抽出・分類によるテキスト要約}
\author{奥村 明俊\affiref{NEC} \and 池田 崇博\affiref{NEC}
\and 村木 一至\affiref{NEC2}}
\headauthor{奥村, 池田, 村木}
\headtitle{5W1H情報抽出・分類によるテキスト要約}

\affilabel{NEC}{日本電気(株)C\&Cメディア研究所}
{NEC Corp., C\&C Media Research Laboratories}
\affilabel{NEC2}{日本電気(株)パーソナルソフトウエア事業部}
{NEC Corp., Personal Software Diviosn }


\jabstract{オフィス業務においては,大量の関連情報から,特定のイベント
  についての経過や状況を把握するために,要約や抄録の生成が求めらている.
  本論文では,複数の文書から抄録や要約をロバストに生成する手法として,
  あるイベントに関する時間的経緯を抄録として生成するエピソード抄録と,
  大量の情報を大局的に把握するための要約文を生成する鳥瞰要約を提案する.
  エピソード抄録では,あるイベントを表す 5W1H (だれが,なにを,いつ,
  どこで,どうした) が含まれる文書を検索し,そのイベントに関する時間的
  経緯を抄録として生成する.鳥瞰要約は,文章中の 5W1H 要素を,シソーラ
  スを用いてそれらの上位概念で置き換えることで,要約文を生成する.新聞
  記事 10,000 件とセールスレポート 2,500 件を対象として適用し,その効
  果を確認した.}

\jkeywords{抄録,要約,エピソード,鳥瞰,5W1H}

\etitle{Text Summarization based on\\ Information Extraction and
  Categorization\\ Using 5W1H}
\eauthor{Akitoshi Okumura \affiref{NEC} \and Takahiro Ikeda \affiref{NEC}
\and Kazunori Muraki \affiref{NEC2}}

\eabstract{ In an office, it is necessary for understanding the
  temporal transition and the \mbox{overall} situation on an event from
  various information to extract and abstract a large number of
  documents.  This paper proposes two robust methods for generating an
  extract and an abstract from documents: an episodic extraction
  method which generates an extract on the temporal transition of an
  event and an overall abstraction method which generates an abstract
  of overall documents for survey.  The episodic extraction method
  retrieves documents including the 5W1H (who, when, where, what, why,
  how and predicates) information which specifies an event and
  generates an extract on the temporal transition of the event.
  The overall abstraction method abstracts documents by replacing
  5W1H elements in each document with their upper categories in a
  thesaurus.  These methods proved to be effective for office work from 
  an application to 10 000 news articles and 2 500 sales reports.}

\ekeywords{extract, abstract, episode, survey, 5W1H}

\begin{document}
\maketitle



\section{はじめに}

インターネット,イントラネットが急速に拡大し,情報洪水と呼ばれる程,多
くの情報が氾濫している.氾濫する情報を効率良く入手する技術として,従来
から,要約や抄録に関する研究が行なわれている
\cite{tamura,hara,yamamoto}.これらの多くは,主に一つのドキュメントの
内容を要約することに重点を置いているため,新聞やニュースのようなイベン
トに対して複数のドキュメントが存在する場合,時間的なイベントの変化のよ
うなエピソード的な情報を構造化しにくいという問題がある\cite{yoshida}.
情報を構造化して要約する手法としては,ドキュメントに対する重要項目をテ
ンプレートとして準備し,テンプレートを用いて抽出した情報から要約を行な
う手法がある\cite{mckeown,ando}.また,要約のためのテンプレートを,与
えられた話題に関するドキュメント集合から自動的に抽出する手法も提案され,
重要度を考慮したテンプレートの抽出が可能となっている\cite{yoshida}.

実際のオフィス業務においては,イベントの経過情報や状況把握など分析的な
情報選別のために,要約や抄録情報が求められる.この場合,ユーザの関心や
意図が多岐に渡り,かつ,対象とする関連情報が大量に存在する.そのため,
重要度を考慮するだけでなく,ユーザの様々な視点や観点から抄録情報をロバ
ストに生成するフレームワークと,大量にある情報を大局的に把握するための
要約が必要となる.

一般に,要約 (Abstract)とは,文書の中心的な話題を簡潔にまとめたもので
あり,抄録 (Extract)とは文書から何らかの基準で文を抜き出しだしたもので
ある.要約は,(1) indicative: 読むか読まないか (2) informative: 内容の
要約(3) critical: 要約+批評 (4) comparative: サーベイ というレベルに分
けることができるが,内容の理解が必要となり,現在の技術では困難なものが
多い\cite{paice}.

抄録は,何らかの手かがりを元に重要な文を抜き出すことで,各文にスコアを
付けてスコアの高いものを抜き出すことが多い.手法としては,キーワードの
頻度によるもの,タイトルのキーワードを用いるもの,文の位置情報を用いる
もの,構文関係を用いるもの,手がかりとなるキーワードを用いるもの,文の
関係に着目するもの,などの方法が提案されている. エピソード抄録は,係
り受け関係と固有名詞やパターン表現を手がかりとして, 情報の要素をより
詳細にインデクスして,時間的または位置的に情報をアレンジして抄録を作成
する手法である.

本稿では,時間表現,固有名詞,動作表現,動詞の格フレームに着目して,テ
キストに含まれる Who (だれが)・When (いつ)・Where (どこで)・What (なに
を)・Why (なぜ)・How (どのように)・Predicate(どうした)といういわゆる
5W1H情報を抽出して,時間や場所をソートキーとしたエピソード抄録,5W1H項
目にシソーラスを適用して上位概念で要約する鳥瞰要約を提案する.5W1H情報
は,日常の出来事を理解するためのキーとなっている概念であり,出来事の内
容の核心部分を表現する.5W1H情報に着目することによって,オフィス業務に
おける有効な抄録情報と大局情報を要約として生成することが可能となる.

本報告では,まず,オフィス業務で求められる抄録と要約について説明し,次
に,5W1H情報を用いたエピソード抄録と鳥瞰要約について説明する.そして,
5W1H情報抽出の手法について述べ,エピソード抄録を新聞記事とセールスレポー
トに適用した事例と,鳥瞰要約を新聞記事情報に適用した事例を報告する.

\section{オフィス業務で求められる要約・抄録}

ユーザの情報要求(information need)には,ユーザの情報に対する認識や理解
度によって異なるレベルの要求が存在する\cite{taylor}.ユーザが,ある情報
の集合に対して,まったく情報がなければ,そもそも,その情報の集合は何を
話題としたものかといった大局的な情報把握の要求が生じるであろうし,情報
要求がある程度具現化されていていれば,その情報をトリガーとして,詳細情
報や関連情報に対する要求が生じる.最近のWWWやインターネットの発達によっ
て,オフィス業務においては,まさに,異なるレベルの情報要求が進化的に変
遷する状況が生じている\cite{bates}.例えば,インターネットやフィルタリ
ングサービスなどを通じて,大量の情報が提供されているが,提供された情報
および情報の集合に対して,以下の3種類の情報要求が存在する.

\noindent\underline{\bf 鳥瞰情報:  情報全体像の把握}

提供された情報が数量的に膨大な場合,全体像を鳥瞰的に把握する必要がある.
例えば,製品開発情報の集合が得られた場合,いつ頃,どのような種類の組織
が,どの分野に関して,何を発表しているのかといった集合全体を把握したい
という情報要求がある.

\noindent\underline{\bf 時間的経緯: 縦方向への展開} 

提供された情報に含まれるイベントが,発生するに至る経緯情報が必要となる.
例えば,「A社が〜ギガのメモリを開発した」という1次情報に対して,A社が
今までメモリの開発に関してどのような時期にどのような技術を開発してきた
のかという,いわば,1次情報からの垂直方向,縦方向への情報要求がある.

\noindent\underline{\bf 類似情報:  横方向への展開}

提供された情報を構成する要素を変数として,類似情報を比較的に獲得する.
例えば,技術調査レポートを作成するユーザが,新聞記事検索を行なって,
「A社が低価格のX製品を開発した」という情報を検索した場合,詳細情報は新
聞記事本文で得ることができる.さらに,「他社はどうなっているか」,「A
社のY製品はどうなっているか」などの情報を得るため2次的検索として行なう.
いわば,1次情報からの水平方向,横方向への情報要求がある.

現状の情報検索技術は,これらの情報要求に対して必ずしも有効な手段を提供
していない.現在,一般的に行なわれている複数のキーワードの組み合わせに
よる情報検索では,キーワードが存在すれば,その論理的関連性の有無に係わら
ず文書が検索されノイズとなることが多い.例えば,{\bf NEC\&半導体\&生産} 
というキーワードで新聞記事を検索すると,「NECが〜と技術提携し,〜が半
導体を生産する.」というように3つのキーワードの間に直接的なつながりが
存在しない文も検索されることになる.このようなノイズは,いわゆる係り受
け検索によって有効に絞り込まれることが報告されている\cite{mine}.

我々は,これらの問題点を解決するために,係り受け情報だけでなく,テキス
トに含まれる Who (だれが)・When (いつ)・Where (どこで)・What (なにを)・
Why (なぜ)・How (どのように)・Predicate(どうした)といういわゆる5W1H情
報を用いた分類・ナビゲーション手法を提案してきた\cite{IAM97,ikeda98}.
次章では,時間的経緯と鳥瞰情報に関する情報要求に応えるために開発した,
5W1H情報を用いたエピソード抄録と鳥瞰要約について述べる.

\section{エピソード抄録・鳥瞰要約}

\subsection{エピソード抄録} 
\vspace{-0.0cm}
情報の中から特定のイベントに着目し,そのイベントの時間的な経過をエピソー
ドのように読める形の抄録として提示する.例えば,図\ref{episode1}のよう
な電機業界に関する技術情報情報が与えられた時に,NECのPDPに関する取り組
みという観点で,情報を時間的な経過とともに拾い読みすることができれば,
一連のエピソードのように情報を獲得することができる.

\begin{figure}[tbp]
  \vspace*{-6pt}
 \begin{center}
    \leavevmode
    \hbox to -20pt{\hss}
    \epsfile{file=episode1.prn,scale=0.40}
    \hbox to -20pt{\hss}
    \caption{エピソードの拾い読み}
 
    \label{episode1}
  \end{center}
\end{figure}

この場合,単純な論理式によるキーワード検索では,関連性の低いものも混在
してエピソードのように読みとれなくなってしまう.そこで,図
\ref{episode2}のように,情報の中から5W1H情報を抽出して,ある出来事に関
して,predicate と argument の関連のある情報だけを抽出し,時間順に並べ
て抄録情報を生成する.

\begin{figure}[tbp]
  \vspace*{-6pt}
  \begin{center}
    \leavevmode
    \hbox to -20pt{\hss}
    \epsfile{file=episode2.prn,scale=0.40}
    \hbox to -20pt{\hss}
    \vspace*{-7pt}
    \caption{エピソード抽出}
    \vspace*{-6pt}
    \label{episode2}
  \end{center}
\end{figure}

この結果,その出来事に関するこれまでの経緯をエピソード的に読むことがで
きるようになる.例えば,Who要素に NEC,What要素にPDPを含む文書を検索し,
時間と対応づけて順に並べることで,NECのPDPの開発に関するエピソードとし
て,「95年10月にNECがパソコン対応PDPを開発.96年10月にNECがPDP量産工場
を建設.97年2月にNECがカラーPDPテレビを量産.97年5月にNECが50インチPDP
を発表.」という抄録を生成することが可能となる.

エピソード抄録としては,時間軸とともに場所に着目することによって,位置
的に展開したエピソードを提示することが可能である.図\ref{episomap}は,
技術関連ニュースヘッドラインから,Where 要素として場所の名前を抽出し,
地図上にマップして表示した例である.対応する場所にマップして表示するこ
とで,エピソードと場所との関連を視覚的にとらえることが可能となる.エピ
ソードの位置展開は,特に,政治家の遊説,犯人の逃亡記事,台風など自然災
害情報のように,移動を伴うイベントのエピソード抄録として有効だと思われ
る.しかしながら,文中の Where 要素が,必ずしもイベントの発生地点を表
しているのではないため,Where 要素を含むすべての文を,同等に地図上にマッ
プすべきかどうかは議論の余地がある.例えば,「北海道に製品工場を建設」
と「北海道向け製品工場を建設」とでは,Where 要素「北海道」が表している
事柄が異なっている.このような場合の位置情報の効果的な提示方法について
は,今後も検討が必要である.

\begin{figure}[tbp]
  \vspace*{-6pt}
  \begin{center}
    \leavevmode
    \hbox to -20pt{\hss}
    \epsfile{file=episomap.prn,scale=0.40}
    \hbox to -20pt{\hss}
    \vspace*{-7pt}
    \caption{エピソードの位置展開}
    \vspace*{-6pt}
    \label{episomap}
  \end{center}
\end{figure}

\subsection{鳥瞰要約} 

5W1H抽出した情報の集合に対して,シソーラスを用いることによって,各要素
の情報をそれらの上位概念で代表させて,鳥瞰的に要約を生成することが可能
となる.図\ref{chokan}は,「NECがPC98NXを発売」「××通信機が次世代パ
ソコンを開発」「××電気が携帯パソコンを発売」「××通信が仮想商店街
を開設」「××電話がインターネットサービスを強化」という5つのニュース
からWho要素とWhat要素を抽出し,それぞれの要素に対してシソーラスの上位
概念を照合したものである.

\begin{figure}[tbp]
  \vspace*{-6pt}
  \begin{center}
    \leavevmode
    \hbox to -20pt{\hss}
    \epsfile{file=chokan.prn,scale=0.40}
    \hbox to -20pt{\hss}
    \vspace*{-7pt}
    \caption{鳥瞰要約}
    \vspace*{-6pt}
    \label{chokan}
  \end{center}
\end{figure}

この結果, これら5つのニュースに対して,例えば,「電機企業3社がパソコ
ンを開発・販売した.」あるいは,「通信企業2社がネットワークサービスを
開設・強化した.」という鳥瞰的な要約を生成することができる.

鳥瞰要約に対しては,要約の対象となる母集団の情報に対して要約率と要約レ
ベルを定義することができる.まず,ある 1 つの 5W1H 要素 $wh$ (Who,
When, Where, What, Why, How, Predicate のどれか) に対して抽出された単
語の集合 ${\bf W}_{wh}$ をある概念 $C_{wh}$ で代表させるときの要約率
$R_{wh}$ を,以下のように定義する.
\[
R_{wh} =  \frac{{\rm Num}({\bf W}_{wh} \cap {\rm Word}(C_{wh}))}
{{\rm Num}({\bf W}_{wh})}
\]
ただし,
\[
\begin{array}{l}
{\rm Num}(S): \mbox{集合$S$の要素数}\\
{\rm Word}(C): \mbox{概念$C$に含まれるすべての単語の集合}
\end{array}
\]
とする.このとき,全体の要約率 $R_{all}$ を,
\[
R_{all} = \prod_{wh \in \mbox{\{5W1H要素\}}}R_{wh}
\]
と定義する.これは,各 5W1H 要素を構成する単語のうち,それらを置き換え
る上位概念に含まれるものの割合を,すべての5W1H要素に関して乗じたものと
なっている.例えば,図 \ref{chokan} の 5 つの文について,Who 要素を
「電機企業」で代表させ,What 要素を「パソコン」で代表させ,Predicate
要素を「発売」で代表させる場合,各 5W1H 要素に対する要約率は,$R_{Who} 
= 0.6, R_{What} = 0.6, R_{Predicate} = 0.4$ となり,全体の要約率は,
$R_{all} = 0.144$ となる.

要約レベルは,5W1H要素のそれぞれについて,独立に定義される値で,置き換
える上位概念が,シソーラス上で何レベル上位の概念であるかを示す.要約レ
ベルをあげていくと要約率は高いものとなる.例えば,日本の各分野の100社
が,様々な製品を開発したニュースが100件与えられた場合,「日本企業100社
が製品を開発した.」という要約文は,要約率100\%ということになる.しか
しながら,このような要約文が,ユーザにとって価値があるとは言いにくい.
要約率・要約レベルという概念を導入することで,ユーザが指示する要約率・
要約レベルに合わせるようにシソーラスの上位概念を探索して,鳥瞰要約を生
成することが可能となっている.実際の要約では,適切な要約率になる上位概
念を選択できないことがあるが,シソーラスで適切にカバーできない要素に関
しては,頻度の多かった代表的なものを列挙して「など」という表現を加えて
代表させることで,上位概念の代わりに用いることで解決できる.

\subsection{5W1H要素の抽出}
\label{sec:chushutsu}

5W1H要素の抽出は,形態素解析結果と辞書情報を基にした浅い解析手法,表層
格指向パーシング CBSP(Case-Based Shallow Parsing)によって行なう.
CBSPによって,頑健で効率的な5W1H抽出が実現できる.CBSPは,形態素解析を
行い各単語に品詞情報を付与したテキストに対し,語彙情報,字句のパターン,
助詞の情報を用いて5W1H解析を行うモデルで,浅い解析により,頑健で効率的
な解析を実現している.基本的には,以下の3ステップから構成される.

\begin{enumerate}
\setlength{\itemindent}{8pt}
\item 固有名詞の抽出

固有名詞のうち,人名・組織名をWho要素として,地名はWhere要素として抽出
する.これにより,例えば,「NECが中国で半導体を生産する.」という
文からは,Who要素としてNECが,Where要素として中国が抽出される.現在,
約6万語の固有名詞辞書を利用している.

\item 特徴的表現のパターンマッチ

特徴的なパターンに着目して,人名・組織名(Who要素),日時(When要素)
を抽出する.例えば,「株式会社××」,「××大学」などのように「株式会
社」が頭に付く語や大学が後に続く語は,会社名や大学名と考えられるため,
これらをWho要素として抽出する.また,「平成××年×月」,「××/××/
××」のようなパターンに当てはまる語は日時を示していると考えられるため,
これらはWhen要素として抽出する.現在,人名・組織名のパターンを約100種
類,日時のパターンを約20種類用意して,解析に利用している.

\item 表層格解析

上記 1,2 のステップで抽出されなかった名詞は,その名詞に続く助詞
等の情報を基に,どの5W1H要素に対応するのかを決定する.例えば,「が」お
よび「は」が後に続く語はWho要素とし,「を」および「に」が後に続く語は
What要素とする.動詞はPredicate要素として抽出する.

1 文中に複数の Predicate 要素がある場合には,他の 5W1H 要素が,もっと
も近い Predicate 要素に係るものとして処理する.ただし,このとき,同種
の 5W1H 要素は,同じ Predicate 要素には係らないものとしている.

\end{enumerate}

詳細なアルゴリズムを図\ref{fig:cbsp}に示す.

\begin{figure}[ptb]
  \begin{center}
    \begin{tabbing}
      \hspace*{1em}\=\hspace{1em}\=\hspace{1em}\=\hspace{1em}\=
      \hspace{1em}\=\hspace{1em}\=\hspace{1em}\=\kill
      {\bf procedure} CBSP;\\
      {\bf begin}\+\\
      入力文を形態素解析;\\
      {\bf foreach} 文中の単語 {\bf do begin}\+\\
      {\bf if} その単語が人名または組織名である {\bf then}\+\\
      その単語を Who 要素としてマークし,スタックに積む;\-\\
      {\bf else if} その単語が地名である {\bf then}\+\\
      その単語を Where 要素としてマークし,スタックに積む;\-\\
      {\bf else if} その単語が人名・組織名のパターンに適合する {\bf then}\+\\
      その単語を Who 要素としてマークし,スタックに積む;\-\\
      {\bf else if} その単語が日時のパターンに適合する {\bf then}\+\\
      その単語を Where 要素としてマークし,スタックに積む;\-\\
      {\bf else if} その単語が名詞である {\bf then}\+\\
      {\bf if} その次の語が「が」または「は」である {\bf then}\+\\
      その単語と,未定の要素を Who 要素としてマークし,スタックに積む;\-\\
      {\bf if} その次の語が「を」または「に」である {\bf then}\+\\
      その単語と,未定の要素を What 要素としてマークし,スタックに積む;\-\\
      {\bf else}\+\\
      その単語を,未定の要素として保持する;\-\-\\
      {\bf else if} その単語が動詞である {\bf then begin}\+\\
      その単語を 5W1H セットの Predicate 要素として確定する;\\
      {\bf repeat}\+\\
      スタックから 1 語取り出す;\\
      {\bf if} その単語に付けられたマークと同じ種類の 5W1H 要素がまだ
      未確定である {\bf then}\+\\
      その単語を付けられたマークと同じ種類の 5W1H 要素として確定する;\-\\
      {\bf else}\+\\
      repeat ループを脱出する;\-\-\\
      {\bf until} スタックが空である;\-\\
      {\bf end}\-\\
      {\bf end}\-\\
      {\bf end}
    \end{tabbing}
    \caption{The algorithm for CBSP}
    \label{fig:cbsp}
  \end{center}
\end{figure}

CBSPによる解析で,約6400件の新聞記事ヘッドラインから,実際にWho,What,
Predicate要素を抽出した結果について分析した結果を表\ref{tab:5w1heval}に示す.
\begin{table*}[tbp]
  \begin{center}
    \caption{Who, What, Predicateの各要素および全体での抽出結果の評価}
    \vspace*{10pt}
    \small
    \begin{tabular}{lrrrrrrrrrr}
      \hline
      & \multicolumn{3}{l}{Who要素}
      & \multicolumn{3}{l}{What要素}
      & \multicolumn{3}{l}{Predicate要素} & \\ \cline{2-10}
      & 存在 & 非存在 & 計
      & 存在 & 非存在 & 計
      & 存在 & 非存在 & 計 & 全体 \\ \hline
      正解 & 5423 & 71 & 5494 & 5653 & 50 & 5703
      & 6042 & 5 & 6047 & 5270 \\
      誤り & 414 & 490 & 904 & 681 & 14 & 695
      & 55 & 296 & 351 & 1128 \\ \hline
      合計 & 5837 & 561 & 6398 & 6334 & 64 & 6398
      & 6097 & 301 & 6398 & 6396 \\ \hline
      精度 & 92.9\% & 12.7\% & 85.9\% & 89.2\% & 78.1\% & 89.1\%
      & 99.1\% & 1.7\% & 94.5\% & 82.4\% \\ \hline
    \end{tabular}
    \label{tab:5w1heval}
  \end{center}
\end{table*}

この表では,新聞記事ヘッドラインにおいて,Who,What,Predicateの各要素
が実際に存在している場合と存在していない場合のそれぞれについて,それら
の要素が正しく抽出された文の数と正しく抽出されなかった文の数をまとめて
いる.これによると,Who,What,Predicateの各要素が実際に存在している場
合には,ほぼ90\%以上の文から各要素が正しく抽出できている.しかしながら,
各要素が実際に存在していない場合には,高い精度が得られていない.これは,
要素が実際に存在していない場合でも,別な語をその要素として抽出してしま
う傾向があるためである.これにより,関係のない語が5W1H要素として抽出さ
れることになるが,正しい5W1H要素が落ちるわけではないので,5W1Hを利用し
た検索において適合率を下げることはなく,実際には,大きな問題とはならな
い.要素がある場合とない場合との平均では,85\%から95\%の精度となってお
り,全体でも82.4\%の精度が得られている.その結果,エピソード抄録や鳥瞰
要約として,ほぼ妥当な結果を得ることができる.

\section{新聞記事とセールスレポートへの適用}

新聞記事10000件を対象として,5W1H に基づいて分類ナビゲーションを行なう
情報活用プラットフォームを構築してきた\cite{okumura97}.図
\ref{fig:plat}は,構築したプラットフォームの構成図である.指定された情
報源から情報収集ロボットによって情報が収集され情報DBに毎日格納されてい
く.5W1H情報抽出モジュールは,収集した情報から,\ref{sec:chushutsu} 節
のアルゴリズムにより,1 文ごとに5W1H情報を抽出し,5W1Hインデクスを生成
する.ここでは,新聞記事ヘッドラインと本文記事からWho・What・Predicate
の3種類について,5W1H要素としてキーワードを抽出し,5W1Hの種類とキーワー
ドから文書を引くための5W1Hインデックスを作成している.抽出したキーワー
ドをそのまま用いるため,同義語は別の語として扱うことになるが,要約文の
生成時には,シソーラスによって同義語が統合される.また,すべての文を対
象として抽出を行うため,Who, What, Predicate の 3 要素の一部が存在しな
い文でも,存在する要素からの検索が可能である.

5W1Hインデクス情報を基に,情報分類モジュール,エピソード抽出モジュール,
鳥瞰情報生成モジュールが,WWWサーバーを介してユーザに分類・ナビゲーショ
ン機能を提供する.フィルタリングサービス部は,ユーザプロファイルを参照
して情報収集ロボットによって集められた情報からフィルタリングしてユーザ
に配信する.ユーザは配信された情報から,5W1H分類・ナビゲーションによっ
て自分の読み進みたい方向へと情報を獲得していくことができる.今回,エピ
ソード抽出モジュールにエピソード抄録生成機能を,鳥瞰情報生成モジュール
に鳥瞰要約を生成する機能を実装した.

\begin{figure}[tbp]
\vspace{-0.45cm}
  \vspace*{-6pt}
  \begin{center}
    \leavevmode
    \hbox to -20pt{\hss}
    \epsfile{file=platform.prn,scale=0.40}
    \hbox to -20pt{\hss}
\vspace*{-17pt}
    \caption{情報活用プラットフォーム}
   \vspace*{-6pt}
    \label{fig:plat}
  \end{center}
\end{figure}
\vspace{0.5cm}

\subsection{新聞記事に対するエピソード抄録}

エピソード抽出では,指定されたヘッドラインからWho・What・Predicate要素
を抽出し,それと同じWho・What・Predicate要素の組を記事中に含む文書を関
連エピソードとして検索して,ヘッドラインを時間順に並べて抄録として生成
する.図\ref{fig:episode}は,「NEC半導体部門の生産予測を18\%増と発表」
というヘッドラインから抽出した5W1H要素のうち,NEC・半導体・生産という
Who・What・Predicate要素を本文中にもつ記事を検索し,ヘッドラインを時間
順に整理してエピソード抄録として提示した例である.利用者は,エピソード
抄録を読むことで,96年10月はNECの半導体生産が下方修正され,半導体各社
が投資を減らし,世界規模で市場が減となり,12月には他の会社が工場建設を
中止するなど,市場が冷え込んでいくが,97年になると,次世代DRAMで好転し
て,各社,計画を増額していくエピソードを読みとることができ,NECの半導
体の生産が18\%増に至るまでの経緯を知ることができる.

\begin{figure}[tbp]
  \vspace*{-6pt}
  \begin{center}
    \leavevmode
    \epsfile{file=episode.ps,scale=0.4}
    \vspace*{-2pt}
 
    \caption{5W1Hによるエピソード抽出}
    \vspace*{-6pt}
    \label{fig:episode}
  \end{center}
\end{figure}

\vspace{-0.cm}

\subsection{新聞記事に対する鳥瞰要約}

情報鳥瞰では,Who要素に出現する約2800の企業を業種別に分類したシソーラ
スと,What要素に出現する約2000のキーワードを技術分野別に分類したシソー
ラスを利用して,Who・What 要素の各キーワードを統合し,鳥瞰的な分類を生
成する.Predicate 要素については,キーワードの種類が少ないことから,高
頻度で出現するキーワード8語で分類している.図\ref{fig:bview2}に97年4月
の約400件のヘッドラインに対する情報鳥瞰の結果を示す.Who・What 要素の
キーワードを階層的なシソーラス構造として扱うことで,最初は荒い分類を提
示し,必要な部分だけ展開して細かい分類を見せることができる(図
\ref{fig:bview3}).

\begin{figure}[tb]
  \vspace*{-20pt}
  \begin{center}
    \leavevmode
    \hbox to -5pt{\hss}
    \epsfile{file=bview2.eps,scale=0.5}
    \hbox to -5pt{\hss}
    \vspace*{-2pt}
    \caption{シソーラスを利用した情報鳥瞰}
    \vspace*{-6pt}
    \label{fig:bview2}
  \end{center}
\end{figure}


\begin{figure}[tb]
  \vspace*{-25pt}
  \begin{center}
    \leavevmode
    \hbox to -5pt{\hss}
    \epsfile{file=bview3.eps,scale=0.5}
    \hbox to -5pt{\hss}
    \vspace*{-10pt}
    \caption{シソーラスを利用した情報鳥瞰:細分類}
    
    \label{fig:bview3}
  \end{center}
\end{figure}

これらの鳥瞰分類結果をもとに,各月の記事の鳥瞰要約を生成する.まず,もっ
とも要約率の高い要約を生成する部分,すなわち,分類された記事がもっとも
多い部分について,Who, What, Predicate のシソーラス上の概念または代表
的なキーワードをパターンに当てはめて要約文を生成し,ユーザに提示する.
図\ref{fig:bview2}の例では,「97年4月は,電機企業30社がマルチメディア
総合技術に関する開発・発売などを131件発表している.」という要約文が生
成される.この場合,Who 要素と What 要素については,キーワードをシソー
ラス上の 1 つ上の上位概念で置き換えているため,要約レベルは 1 となり,
Predicate要素については,キーワードそのままの表現を用いているため,要
約レベルは 0 となる.ユーザには,この後,要約率の高い順に要約文が順々
に提示される.

\vspace{0.5cm}

\subsection{セールスレポートに対するエピソード抄録}

ある会社の約2500件のセールスレポートに対して5W1H情報抽出を行ない,エピ
ソード抄録を生成した.ここでは,実際のセールスレポートを実験
用に変換した例を用いて説明する.セールスレポートは,営業マンが担当する
顧客の要望などを報告したものである.このレポートは,メッセージアベニュー
という配信サービスとして関係者に配信される.セールスレポートは,5W1H情
報抽出によって,日時,顧客名,対応支店名,機種名,用件が抽出される.

図\ref{fig:mavenue}は,97年4月の一ヶ月分のセールスレポートの一覧
である.画面上部のフレームにセールスレポートのリストが,画面下部のフレー
ムに選択されたセールスレポートの内容が表示される.図\ref{fig:mavenue}
の例では,リストの3番目のレポートを表示している.このレポートにおいて,
○×建築が顧客名,若葉台支店が対応支店名,XX282ソフトが機種名,
2000年問題が用件に対応\break
する.この画面から,例えば,○×建築殿という部分
をクリックすることで顧客,○×建築に関するレポートを顧客名,対応支
店名,機種名の視点からブラウズすることができる.
\begin{figure}[tb]
  \vspace*{-6pt}
  \begin{center}
    \leavevmode
    \hbox to -20pt{\hss}
    \epsfile{file=mavenue.eps,scale=0.5}
    \hbox to -20pt{\hss}
    \vspace*{-7pt}
    \caption{セールスレポートの一覧}
    \vspace*{-6pt}
    \label{fig:mavenue}
  \end{center}
\end{figure}

顧客名,対応支店名,機種名,用件の中の任意の2要素に関して,情報を共有
するレポートからエピソード抄録を生成する.「97/4/16,A銀行殿が,新宿
営業所に,空調システムBの保守対応」という記事をクリックするとレポート
本文が表示される.本文には,「4/16 昨日発生した空調システムBの故障の
件でA銀行殿に報告に伺った.『障害対応に関し弊社の対応はまだ手ぬるい.
全体的にレスポンスが悪い』とのご指摘が有り,担当拠点を集めてオリエンテー
ションを開催する事にした.」と記述されている.このレポートに対して,顧
客と機種名をキーとしてエピソード抄録機能を適用すると,図
\ref{episomap2}のように,今までの経過が時間順に提示される.この抄録情
報によって,A銀行では空調システムBに障害・保守対応が続発しており,顧
客が不満を持っているというエピソードを読みとることができる.
\begin{figure}[tb]
 \vspace*{-20pt}
  \begin{center}
    \leavevmode
    \hbox to -5pt{\hss}
    \epsfile{file=episomap2.prn,scale=0.42}
    \hbox to -5pt{\hss}
    \vspace*{-35pt}
    \caption{顧客と機種に関するエピソード}
    \vspace*{-6pt}
    \label{episomap2}
  \end{center}
\end{figure}


\vspace{1.0cm}

\section{おわりに}
本稿では,5W1H情報によるエピソード抄録と鳥瞰要約を提案した.これらの機
能は,ある出来事に至るまでの経緯という時間的経過に関する情報要求,多量
の情報の中から主たる話題・内容を大局的に把握するという情報要求に応える
ことを目的としている.エピソード抄録と鳥瞰要約を新聞記事10000件とセー
ルスレポート2500件を対象として適用したところ,エピソード抄録は,ユーザ
の様々な視点から抄録情報をロバストに生成できること,鳥瞰要約は,大量に
ある情報を大局的に把握するための要約となることを確認した.さらに,鳥瞰
要約では,要約率と要約レベルを定義することができ,ユーザの指定にしたがっ
て要約を生成することができる.今後,ユーザにとってより満足度の高い抄録・
要約機能の実現するため,以下の改良を行なう.

\begin{itemize}
  \item エピソード抄録:

5W1Hの各要素をどの程度共有するものをエピソードとするかに関しては,いく
つかのレベルが考えられる\cite{ikeda97}.ユーザの目的にあわせて,そのレ
ベルをコントロールして,抄録を生成するべきである.今回,新聞記事では,
Who・What・Predicate要素をキーとしてエピソードを生成し,セールスレポー
トでは,顧客名,対応支店名,機種名,用件の中の任意の2要素をキーとして
エピソード抄録を生成した.今後は,エピソードに関するレベルコントロール
機能を実装して評価を行なう\cite{ikeda97}.また,エピソード抄録の構成方
法としては,新聞記事への適用のように,本文中に関連情報を含むヘッドライ
ンを列挙する方法と,セールスレポートのように関連情報を含む文そのもの(ヘッ
ドライン)を列挙する方法を実装したが,長い抄録となる場合もある.同種類
の内容のものは繰り返さない,鳥瞰要約と組み合わせて情報を圧縮するなど,
簡潔な抄録を得るための工夫を行なう.

  \item 鳥瞰要約:

鳥瞰要約では,シソーラスの選択および適用方法が重要である.Who要素に関
しては,比較的ユニークなシソーラスを構築しやすいが,What要素はシソーラ
スは一通りではなく,複合的なオントロジとなる.オントロジを用いていくつ
かの観点から鳥瞰要約を生成可能として,ユーザにとって必要な視点の要約
を提示する必要がある.また,鳥瞰要約では,どのようにシソーラスの上位概
念を置き換えて要約するのが効果的なのかユーザの目的によって異なる.要約
率,要約レベル,選択するシソーラスを組み合わせてより効果的な鳥瞰
要約の生成方法を検討する.

\end{itemize}



\vspace{17mm}

\bibliographystyle{jnlpbbl}
\bibliography{v06n6_01}

\begin{biography}
\biotitle{略歴}
\bioauthor{奥村 明俊}{
1984年京都大学工学部精密工学科卒業.
1986年同大学院工学研究科修士課程修了.同年,NEC入社.
1992年10月南カリフォルニア大学客員研究員 (DARPA MTプロジェクト1年半参加).
1999年東京工業大学情報理工学研究科博士課程修了.
現在,C\&Cメディア研究所,主任研究員.
自然言語処理の研究に従事.工学博士.
情報処理学会,ヒューマンインタフェース学会などの各会員.}

\bioauthor{池田 崇博}{
1994年東京大学理学部情報科学科卒業.
1996年同大学院理学系研究科修士課程修了.同年,NEC入社.
現在,C\&Cメディア研究所,研究員.
自然言語処理,情報分類の研究に従事.
情報処理学会会員.}

\bioauthor{村木 一至}{
1974年京都大学工学部情報工学科卒業.
1976年同大学院工学研究科修士課程修了.
1979年NEC入社.
現在,パーソナルソフトウェア事業部,グループマネジャ.
自然言語理解,アイディアプロセッシングの研究に従事.
情報処理学会,人工知能学会,ACL各会員.
Natural Language Engineering 編集委員.}

\bioreceived{受付}
\biorevised{再受付}
\bioaccepted{採録}

\end{biography}

\end{document}
