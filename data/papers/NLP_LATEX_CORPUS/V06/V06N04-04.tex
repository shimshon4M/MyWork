



\documentstyle[epsf,jnlpbbl,lingmacros]{jnlp_j_b5}

\setcounter{page}{67}
\setcounter{巻数}{6}
\setcounter{号数}{4}
\setcounter{年}{1999}
\setcounter{月}{7}
\受付{1997}{10}{8}
\再受付{1998}{3}{30}
\採録{1998}{12}{24}

\setcounter{secnumdepth}{2}

\title{日本語の指示詞における\\直示用法と非直示用法の関係について}
\author{金水 敏\affiref{osaka}}

\headauthor{金水}
\headtitle{日本語の指示詞における直示用法と非直示用法の関係について}

\affilabel{osaka}{大阪大学大学院文学研究科}
{Graduate School of Letters, Osaka University}

\jabstract{
日本語の指示詞の3系列(コソア)は,いずれも直示用法とともに非直示用法を
持つ.本稿では「直示」の本質を「談話に先立って話し手がその存在を認識
している対象を,話し手が直接指し示すこと」ととらえ,ア系列およびコ系列
では直示・非直示用法にわたってこの直示の本質が認められるのに対し,
ソ系列はそうではないことを示す.
本稿では,ア系列の非直示用法は「記憶指示」,すなわち話し手の
出来事記憶内
の要素を指し示すものであり,コ系列の非直示用法は「談話主題指示」,すなわち
先行文脈の内容を中心的に代表する要素または概念を指し示すものと考える.
「記憶指示」も「談話主題指示」も上記の直示の本質を備えている上に,
ア系列およびコ系列の狭義直示用法において特徴的な話し手からの遠近の対立
も備えているという点は,ア系列およびコ系列の非直示用法がともに
直示用法の拡張であることを示唆している.
さらにさまざまなソ系列の非直示用法を検討した上で,ソはコ・アとは異なって,
本質的に直示の性格が認められないことを論じる.非直示用法のソ系列は
話し手が談話に先立って存在を認めている要素を直接指すためには用いられず,
主に言語的な表現によって談話に導入された要素を指し示すために用いられる.
またソが,「直示」によっては表現できない,分配的解釈や,いわゆる代行用法
等の用法を持つことも,ソがアやコと違って非「直示」的であるという主張と
合致する.
}

\jkeywords{指示詞,日本語,直示,出来事記憶,談話主題,文脈照応}

\etitle{On the Relation between \\
the Deictic Use and the Non-deictic Use\\ of the Japanese Demonstratives}
\eauthor{Satoshi Kinsui \affiref{osaka}} 

\eabstract{
This paper addresses the status of deixis and anaphoricity in Japanese
demonstrative system. It has traditionally been claimed that the Japanese
demostrative prefixes {\it ko}, {\it so}, and {\it a} all have both deictic 
and anaphoric
usages.  I shall argue that the core notion of {\it ko}, and {\it a} are 
deictic, in
the sense that the expressions with these demonstrative prefixes make direct
reference to an entity whose existence is recognized by the speaker prior to
the discourse session in question, while that of {\it so} is not. I will show
that even in cases in which they are treated as anaphoric both {\it ko} and
{\it a} are `deictic', in the sense given. In the case of {\it a-}series,
the entity referred to is located in the `episodic memory', whereas in the
case of {\it ko-}series, the relevant entity is a `discourse topic'. The
contrast between the `episodic memory' and the `discourse topic' can be
characterized in terms of `remote' vs. `close', which are the key notions in
characterizing the deictic usages of {\it a} and {\it ko}, thereby
suggesting that the non-deictic usages of {\it a-}/{\it ko-}series are
derived from their deictic usages. I further argue that the various usages
of the {\it so-}series demonstratives can be best described by assuming that
{\it so}, unlike {\it a} and {\it ko}, does not refer an object by direct
reference. I demonstrate in particular that the {\it so-}series cannot be
used to directly refer to an entity whose existence is assumed by the
speaker, while it can be used to refer to an entity that is introduced into
the discourse by some linguistic expressions.  The claim that {\it so} is
non-`deictic' in nature is further supported by the observation that the
{\it so-}series, but not {\it a-}/{\it ko-}series, allows a usage that
cannot be expressed by `deixis', such as the distributive interpretation.
}

\ekeywords{demonstratives, Japanese, deixis, discourse topic, anaphoric use}

   
    \def\susumu{}
    \def\Susumu{}
\unitlength=1mm

\let\enums
\let\eenums
\let\fn
\let\ul
\newcommand{\gl}[2]{}
\def\rubyii#1#2{}
\newcommand{\sitatuki}[1]{}

\def\sqbr#1{}

\widelabel=20pt

\begin{document}
\maketitle



\vspace{-3mm}
\section{はじめに}
日本語の指示詞については豊富な研究史が存在するが,それは日本語が豊かな
指示詞の体系を持ち,さまざまな興味深い振る舞いを見せてくれるからである.
談話構造との関わりから眺めた場合にも,未だ充分解決されていない,重要な
問題が多数浮かび上がってくる.本稿は,指示詞の分析から談話構造の理論的研究へ
と向かう一つの切り口を提示することを目指す.

本稿で特に問題としたいのは,日本語の指示詞の体系
における,指示の機能の上での等質性と異質性である.
結論の一部を先取りして言えば,指示詞の3系列(コ・ソ・ア)
の内,ソ系列はコ系列およびア系列に対して異質な性格を多く持っている.
この異質性を突き詰めていく過程で,指示の構造に対する根底的な理解が
要請されてくるのである.
指示詞をコ・ア対ソという対立関係で捉えようとする見方は\citeA{horiguti},
\citeA{kuroda}を先駆とするが,本稿ではこの見方を談話処理モデルの
中に位置づけることを試みる.その要点は次の通りである.

コ・アはあらゆる用法において基本的に直示の性質を保持している.
狭義の直示とは,文字どおり眼前の対象を直接指し示すことであるが,
本稿では特にこの直示の本質を,次のようにとらえたい\cite{takukin97}.

\enums{
{\bf 直示の定義} :\\
談話に先立って,言語外世界にあらかじめ存在すると話し手が認める対象を
直接指し示し,言語的文脈に取り込むことである.}

この定義から,次のような事柄が帰結として導かれる.まず,指示対象は
言語的文脈とは独立に,言語外世界に存在するので,先行する言語的文脈に対しては
基本的に自由である.まして,言語的文脈によって概念的に設定された
対象を指し示すことはない.

次に,直示に用いられた指示詞表現(例「この犬」)には,指示対象の
カテゴリーを表す表現が付いていることがある(例「犬」)が,
直示の場合,指示対象が(多くは眼前に)現に存在するということを
示すことが意味の中心であるので,カテゴリーは副次的にしか機能しない.
極端な例として,眼前にいるカラスを指さして「この犬」と言っても,
指示は成立しているのである.

加えて,直示における指示対象が基本的に確定的・唯一的であることも前提される.
指示対象自体が変域を持ったり,未定・不定であるということはあり得ない.
言葉を換えると,直示表現は使用された段階では変項ではありえない
\fn{指示詞自体は,一般的に,それが用いられた文脈によって指示対象が変わるので,
変項として扱われるのが普通である.
ここでは,文脈の中で直示なり,照応なりによって指示詞の機能が確定した
あとの機能について述べている}.

コ系列が「近称」,ア系列が「遠称」と捉えられるように,話し手からの
距離によってこれらの指示詞が特徴づけられることも,直示の本質にとって
重要な点である.すなわち,対象があらかじめ言語外世界に存在するが故に,
話し手はそれを「近い」とか「遠い」とか判定できるのである.

コ系列・ア系列指示詞には非直示用法も存在するが,
非直示用法においても上に述べた直示の性質が保持されることを本稿に
おいて示す.

一方,ソはいわゆる直示用法を持つ一方で非直示用法も持つが,ソの
非直示用法は直示とはまったく異なる性質を持っている.すなわち,
ソのいわゆる照応用法は,言語外世界とは関係なく,先行文脈によって概念的に
設定された対象を指し示す.その場合,指示対象の概念が検索の重要なキーと
なるので,カテゴリーを変えると指示対象の同定が困難になる.また,
指示対象が未定・不定・曖昧であったり,束縛変項のように,指示対象に変域が
生じる場合がある.以上の点から,ソの非直示用法は,本来の直示とは
全く異なるものであるということを主張する.

本稿では,このようなコ・アとソの違いが,指示詞表現が指定する
心的な領域の違いから生じるものと考える.
本稿が依って立つ談話処理のモデルは,\citeA{fauconnier85}に始まる
「メンタル・スペース」理論の流れを組み,\citeA{jcss},\citeA{sakahara96}
等に受け継がれた談話管理に関する理論である.これらの理論に共通するのは,
言語表現と外的世界とをつなぐ位置に中間構造としての心的表示を仮定する
点である.

\enums{
言語表現 $\longrightarrow$ 心的表示 $\longrightarrow$ 外的世界}

ここで言語表現は,心的表示への操作(登録,検索,マッピング等)の
指令,あるいはモニター装置として機能する.このモニター機能によって
聞き手は話し手の心の働きをある程度知ることができ,コミュニケーション
の効率化を助けるのである.

指示詞の研究は,この心的表示の構造や指示詞がモニターする操作の
実態を明らかにすることを目標とする.このような見方のもとで,
本稿は次のように議論を進めていく.まず次節で,指示詞表現一般の
談話的な機能を概括し,以下の議論の準備とする.3節から5節では,
主にいわゆる文脈照応用法を中心として,コ・アとソを対比する形で
実際の用法を検討し,コ・アを用いる表現には常に直示の性質が備わっ
ていること,逆に
ソを用いる表現には直示とは相容れない特徴が認められることを明らかにしていく.
6節では直示用法について簡単に触れ,ソ系列の特異性を指摘する.
最終節では,以上の議論をまとめ,併せて文脈照応用法と直示用法
の関係についての課題を提示する.ソの非直示用法が本来の直示と異なるものである
と考える場合,ソにおける直示と非直示の相関関係が問題になるが,この点については
本稿では結論を出すことができないので,今後の方向性を最終節で
提示するにとどめる.

\section{指示詞表現の意味論的・談話理論的機能}
指示詞表現は,一種の談話管理標識である.すなわち,指示対象の同定に
あたって話し手が行っている心的な処理の状態をモニターしている訳である.
それによって知られる話し手の心的処理に基づいて,聞き手も指示対象の
同定を行うことができる.

指示詞表現がモニターする心的処理を「このN」(Nは名詞)という
形式を例にとって説明しよう.「このN」が表す心的処理は,「``コ''として
マークされた領域と関連づけられ,かつNによって
示されるカテゴリーに属する対象を,値として指示せよ」というものである.
これを\citeA{fauconnier85}が示した役割関数の考え方に基づいて
\citeA{sakahara96}が用いた簡略的な記法によってしめせば,

\enums{
N(こ)}

ということになる.ここで,「``コ''としてマークされた領域」とは,
典型的な直示用法を例にとると,話し手の指差しや眼差し等の行動によって
焦点化された,比較的話し手に近い眼前の空間の一部である.「領域と関連
づけられる」とは,典型的には値としての個体がその領域に
存在することを表す.例えば,犬を指差しながら「この犬」と言った
場合,指された個体としての犬そのものが指示の値となる\fn{
なお,上の例ではNは普通名詞を想定しているが,「\ul{あの}松田聖子が……」
のように固有名詞など指示的な要素が置かれる場合もあるであろう.
その場合は,指示詞によってマークされた領域と,Nが指示する対象とを
結びつけるということになる.即ち,指定された領域におけるNの属性を強調する
表現になるのである.}.

しかし,時にはもう少し複雑な場合が存在する.例えば
写真を見ながら「\ul{この人}は誰ですか」と言ったり,
バーのマッチ箱を取り出して「\ul{この店}に行こう」と言ったり,仕出し弁当を
食べながら「\ul{ここ}($=$今食べている弁当を作った店)も味が落ちたねえ」
などという場合である.このような用法を「間接直示」と呼ぶことにし,
4.1節で詳しく述べることにする.さらに,例えば犬を指さしながら「\ul{この犬}は
シベリアが原産地だ」「\ul{この犬},うちでも飼ってる」
などと言う場合がある.これは,関数「犬(この)」
の値が個体ではなく「ハスキー犬」のような下位カテゴリーになっているのである.
通常の,個体を値とする用法を「個体読み」,下位カテゴリーを値とする
用法を「種類読み」と呼ぼう.
個体読み/種類読みの分化は,指示詞表現だけでなく「同じ犬」
「きのう見た犬」等の表現でも見られる,ごくあたりまえの現象である.
個体読み/種類読みの差異が問題と
なる現象の詳細については\citeA{kinsui90}に譲り,以下では,カテゴリーも個体と
同様のオブジェクトとみなし得る範囲で,個体と区別せずに扱うことにする.
さらに問題となるのは,「会員と\ul{その家族}」などという場合に見られる
「代行指示」用法\cite{syooho81}である.この用法も結局は「家族(その)」の
ような関数適用のヴァリエーションとして統一的に説明できると考えるが,ソと
コ・アでかなり差が出る用法であるので,5.1節で再び考察したい.

以上の分析では,すべて「このN」の形式を持った指示詞表現を扱ってきたが,
他の形式でも同様に,コソアの部分は指示の領域を,コソアの後ろの部分は
表現の意味的・統語的カテゴリーを表すと考えられる.
例えば「ここ」は領域マーカー「こ」と,カテゴリー``場所''を表す「こ」が複合した
形式であり,「こんな」は,領域マーカー「こ」と,領域からなんらかの静的な
属性を抽出し,形容詞相当の語句を形成する「んな」が複合した形式であるという
具合に分析できる.本稿で問題となるのは,コソアの部分が示す領域の
違いであるので,後半部分の形態素が示すカテゴリーの違いについての言及は
最小限にとどめる.

さて,上に述べたように,日本語の指示詞表現は形態的に指示の領域と表現の
カテゴリーの組み合わせとして説明できるわけであるが,後に詳しく述べるように,
コ・アとソでは,領域指定の部分とカテゴリー指定の部分との
重みにおいて違いがある.コ・アの指示詞表現にとっては,コ・アが指定する領域に
その指示対象が
あらかじめ存在する,という「直示」に関わる部分が重要なのであり,カテゴリーの関数的な機能は
あまり重要でない.4.4節で述べるように,実はカテゴリーは先行詞とまったく
別のものになってしまっても,領域内に指示対象が存在するという含意に支えられて
指示は成立する.一方,ソ系列,特にその文脈照応用法では,カテゴリーの関数とし
ての機能,すなわちカテゴリーに合致する値を返す,という機能が重要であり,ソが領域として指定する言語的文脈は指示対象の存在を何ら保証しないのである.

次節ではまず,ア系列とソ系列の文脈照応用法を比較する.

\vspace{-3mm}
\section{ア系列とソ系列}
\subsection{ア系列の原型と拡張}
ア系列の原型的用法は,狭義の直示用法に求めることができる.直示における
アの領域は,
眼前の空間において,コと対立する形で,話し手が直接操作できない
遠方の空間を指差し,眼差し等の行為によって焦点化することによって
形成される.いわゆる遠称である.

さらに,ア系列は話し手の出来事記憶中の場面を領域として焦点化する
用法を持つ.これを仮に「記憶指示用法」としておこう.
ここで言う「出来事記憶中の場面」とは,次のような事柄を表す.
「今・ここ」において
眼前に広がる状況は,指示詞表現(コ・ソ・ア)と指差し・眼差し等の行為によって
焦点化され,区分される.これが,狭い意味での直示用法である.
時間が経過し,眼前の状況が変化するとともに,以前の眼前の状況は
出来事記憶に格納される.格納された状況とは,時間・場所のインデックスが
ついて何らかの出来事がそこで生起する「場面」である.この,過去の
記憶中の場面は,回想によって焦点化された時点でアの領域となる\fn{
\citeA{jcss}では,現場の状況および出来事記憶中の要素に至るポインタ
を格納するデータベースを仮定し,D--領域と名付け,ア系列はD--領域の
要素を指し示すと述べた.本稿では,指示詞の機能に即して分かりやすく
するためにD--領域の概念は用いず,直接的に現場の状況や出来事記憶
を参照するモデルにしたが,ア系列の指示詞の説明における基本的な
アイディアは\citeA{jcss}で示したものと同等である.}
.記憶指示用法は,この回想された場面と関連づけられた対象を値とする.
場面は特定の時間・場所である場合もあるし,不特定である場合も
あるが,ア系列の場合は,必ず話し手が直接体験した場面でなければならない.

記憶指示一般について言えば,現に眼前に存在する対象ではないので,
直示的である必然性はないが,記憶内の場面を眼前の状況と同等に扱えば,
直示と同じ方法によって記憶内の要素を指示する
ことも可能なはずである.アの記憶指示用法とは,このような拡張的な
直示によるものと考えることができる.
\citeA{mikami70}が,アは空間的遠方だけでなく
時間的遠方も指せる,と述べた通りである.

一般に,アの文脈照応用法と呼ばれるものは,すべてこの記憶指示用法である.
先行文脈によって喚起された場面に,指示対象が存在することが示されている
のである.一見,先行文脈に依存した表現のように見えるが,言語的文脈は
聞き手の便利のために
アによって焦点化される場面を限定しているに過ぎない.

\enums{
きのう,山田さんに会いました.\ul{あの人},変わった人ですね.}

また,先行文脈を持たない記憶指示用法も存在する.それは,(\ex{1})のような独り言
や,(\ex{2})の夫婦の会話のように,生活時間を濃密に共有していて,話し手が
焦点化した場面を聞き手が容易に推測できる対話者の会話においてよく現れる.

\enums{
神戸で食べた\ul{あの肉饅頭},おいしかったなあ.}

\enums{
(夫が長年連れ添った妻に)\\
おい,\ul{あれ}を出してくれ.}

\subsection{ソの指示対象}
ソ系列がマークする領域とはどのようなものであろうか.ソ系列の典型的な
用法である文脈照応用法について,本稿の仮説を述べる.言語的文脈は,
話し手にとっての外的世界とは独立に,それだけで状況を形成することができる.
この状況は,語彙の概念的意味,フレーム的意味と最小限の推論等によって
形成される.この言語的文脈がつくる状況を,\citeA{jcss}その他に従って
I--領域と呼んでおこう.ソの照応が単に語句の一致ではなく,推論によって
形成される状況を領域とすることを示す例として,次のようなものが
挙げられる.

\enums{
小麦粉と牛乳をよく混ぜ,\ul{それ}をフライパンに注ぐ.}

上記の「それ」が指し示す値は,前文に現れた名詞句「小麦粉」「牛乳」のいずれ
でもなく,「小麦粉と牛乳を混ぜた混合物」とでも言うべき対象である.これは
言語的には現れないが,フレーム的知識と推論によって状況に導入された
ものである\cite{yosimoto86}.

上のようなソ系列の文脈指示用法の指示の値は,自立的なものではなく,
言語的な先行文脈にのみ依存している.

\subsection{久野説と黒田説について}
いわゆる文脈照応用法のソ系列とア系列の違いについて述べた重要な
論文として,\citeA{kuno73b}がある.久野は次のような一般化をしている.

\enums{
\begin{description}
\item[ア--系列:]その代名詞の実世界における指示対象を,話し手,聞き手
ともによく知っている場合にのみ用いられる.
\item[ソ--系列:]話し手自身は指示対象をよく知っているが,聞き手が指示対象
をよく知っていないだろうと想定した場合,あるいは,話し手自身が指示対象を
よく知らない場合に用いられる.(185頁)
\end{description}
}

この久野の一般化は,次のようなソとアの分布を一見きれいに説明する.

\eenums{
\item[A :]昨日,山田さんにはじめて会いました.\{あの/??その\}人随分変わった
人ですね
\item[B :]ええ,\{あの/??その\}人は変人ですよ}

\eenums{
\item[A :] こんど,新しい先生がくるそうよ
\item[B :] \{その/*あの\}先生,独身?}

\enums{\label{soko}
神戸にいいイタリア料理店があるんですが,こんど\{そこ/??あそこ\}に行って
みませんか?
}

(\ex{-2})は話し手も聞き手も対象をよく知っている場合で,アしか許されない.
(\ex{-1})のBは話し手が,(\ex{0})は聞き手が対象をよく知っていない場合で,
逆にアは許されず,ソを使わなければならない.

この久野の一般化に対し,\citeA{kuroda}はソとアの独立用法の観察に基づいて,
次のような修正が必要であるとした.

\enums{
指示詞ソ・アの選択に真に本質的な要因は,話し手及び聞き手が対象を
「よく知っているかいないか」ということではなく,話し手が,指示詞使用の
場面において,対象を概念的知識の対象として指向するか直接的知識の
対象として指向するか,ということにあるのである.\cite{kuroda}}

すなわち,聞き手の知識はソとアの選択にとって本質的ではなく,対象の
意味論的な差異が重要だとするのである.黒田の主張を最も端的に裏付ける
のは次のような例文である.

\enums{\label{kotodakara}
昨日神田で火事があったよ.\{?あの/*その\}火事のことだから,人が
何人も死んだと思うよ.}

この文脈では,話し手はその火事のことをよく知っているが,聞き手は
よく知らないので,久野の一般化はアを排除しソが適切であると予測するが,
事実は逆である.
ここで用いられている「〜のことだから」という表現は,それに続く判断
(この場合は「人が何人も死んだ」)のための根拠を示す.
アによって示される火事は話し手が直接知っている
火事なので,後件の判断の根拠として適切であると感じさせるが,
ソによって指し示される火事は,話し手が発話によって差しだした文字どおりの情報
「昨日神田で火事があった」およびそれから最小の推論によって
導かれる情報以上の情報を持たないので,「人が何人も死んだ」ことを
導く根拠とはなりえないのである.この現象は,ア系列指示詞表現の
直示的な性質を示すものと思われるが,それを示すために,
黒田の議論を若干補っておこう.

(\ex{0})で「その火事」が不適切であるのは,先行文脈に示される
「その火事」の属性が「人が何人も死んだ」ことの根拠とはならないから
と述べたが,次のような対比を考えてみよう.

\enums{
昨日神田で火事があったよ.すごい炎と煙が上がっていて,
消防車が何台も来ていた.\{そんな/*その\}火事のことだから,人が何人も
死んだと思うよ}

(\ex{0})では,「そんな火事」は適切であるが,「その火事」は依然として
不適格である.これは次のような理由によると考えられる.
「NPのことだから」という
表現における「NPのこと」とは,「NPの指示対象の持つ属性のすべて」を指示すると
考えることができる\fn{
田窪行則氏の個人的談話による.}
.そしてコトダカラ文の適格性は,その後件の命題が「NPの指示対象の持つ属性の
すべて」に含まれるか否か,という点で測られる.例えば

\enums{
田中さんのことだから,きっと時間どおりやってくるよ.}
のようにNPに固有名詞が充てられる場合,未知の将来の時点における属性
「時間どおりにやってくる」を「田中さん」の属性と認めうるが故に
この表現は適格と認められる.すなわち,固有名詞のような
固定指示的な表現の指示対象は,言語的文脈や話し手の知識以外
の状況で,同一の対象が言語的文脈や話し手の知識にない属性を持ちうるのである.
一方,「その火事」のような表現の持つ属性とは,言語的文脈に依存し,
そこから出ることはないので,語られていない属性を後件に持つと
不適格となる.これに対し,「そんな火事」は,言語的文脈に
依存しているのは「そ」の部分だけであり,「そんな火事」全体の意味は
「文脈に示されたような属性を持つ火事(一般)」という範疇的・総称的
な表現である.範疇的・総称的な対象は,言語的文脈に示された
属性以外の属性を当然持ってよいので,その適格性は,前件から後件を
導く推論の妥当性のみに依存するのである.

黒田が示したこの事実は,我々が指示対象に関する知識を,少なくとも直示的・
直接的なものと概念的・間接的なものに分類して持っていることを表している
と考えられる.ではなぜ,(\ref{soko})では,話し手は対象を直接的知識として
持っているにも関わらず,アの使用が不適切に感じられるのか.それは,
\citeA{jcss}が指摘しているように,聞き手にとって
新規な情報を提示する際の唯一許された方法が
概念的知識をベースとした伝達であるからである.この点について,
次節で詳しく述べる.

既に明かなように,本稿の立場は\citeA{kuroda}に近い.黒田の言う「概念的
知識/直接的知識として志向する」というアイディアを,本稿では
言語的に形成された状況(I--領域),または直接的に体験した出来事記憶内の
場面を介在させて指示対象へ至るという形式によって表現している.

\enums{
\evnup{
\unitlength=1zw
\begin{picture}(30,5)
\put(0,4){ア}\put(2,4.5){\vector(1,0){2}}\put(5,4){出来事記憶}
\put(0,0){ソ}\put(2,0.5){\vector(1,0){2}}\put(6,0){I--領域}
\put(10.5,4.5){\vector(3,-1){3}}\put(15,2){(現実世界の火事)}
\put(10.5,0.5){\vector(3,1){3}}
\end{picture}}
}


\subsection{久野と黒田の再解釈}
本稿の仮説に従えば,直接に経験した過去の場面に関連づけられた対象であることが
ア系列を使用するための必要条件である.従って,話し手が対象をよく知らない,
即ち直接に経験した過去の場面とは無関係な対象(例「新しい先生」)はア系列で
指し示すことができない.一方,言語的文脈によって形成された状況であるので,
ソ系列には適している.

\eenums{
\item[A :] こんど,新しい先生がくるそうよ
\item[B :] \{その/*あの\}先生,独身?}

次に,話し手に知識があるが聞き手には充分な知識がない場合はどうであるか.
アを使用する必要条件は満たされている.また言語的文脈を提示すれば,ソ系列
の使用の条件も満たされる.このように意味論的な条件ではソもアも使用可能である
から,どちらを用いるかは語用論的な理由によって決定される.次のように
説明的・提示的な文脈では,話し手は知識を持たない聞き手に合わせて,I--領域即ち
言語的文脈をベースにしてコミュニケーションを行うことがいわば義務づけられている
と見ることができる.

\enums{
神戸にいいイタリア料理店があるんですが,こんど\{そこ/??あそこ\}に行って
みませんか?
}

これは,基本的に相手の知らないことを新規に知らせるには言語的文脈に基づいて
概念的に提示する方法しかないからで,そうでない方法を用いると,聞き手は
高度な推論を起動する必要がある.これは聞き手にとって負荷のかかる処理と
なる.ア系列を用いて指示する方法がまさにそれで,
自分にない知識に基づいて相手が話しているわけであるから,
自分の知識と対立的な相手の知識を推論しなければならず,強い負荷がかかる.
従って,説明的・提示的な文脈でソを使用するということは,次のような
聞き手の負荷に関する語用論的な要請に基づく選択であると見ることが
できる.

\enums{
{\bf 聞き手負荷制約:}\\
聞き手が発話を処理する際にかかる負荷を最小にせよ.}

但しこれはあくまで語用論的な制約なので,聞き手への負荷を無視して
よい文脈ではアが用いられることが知られている.例えば次のような,
叱責,勧め,思い出語り等の文脈である\cite{kuroda,yosimoto86}.

\eenums{
\item[A :]先生,この山田って誰ですか.
\item[B :]君は\ul{あの}先生も知らないのか.}

\enums{
君は大阪に行ったら山田先生に会うといい.きっと,\ul{あの}独特な魅力にきっと
魅せられると思うよ.}

\eenums{
\item[A :]先生がこの世界に入られた頃はどんな様子だったでしょうか.
\item[B :]\ul{あの}頃は物資が窮乏していて,大変でした.}

では,話し手・聞き手がともに対象についてよく知っている場合はどうか.
この場合も,言語的文脈があれば,アもソも必要条件が満たされるので,
両者とも使用可能である.にも関わらず,\citeA{kuno73b}によれば,アしか
使用できないとされる.

\eenums{
\item[A :]昨日,山田さんにはじめて会いました.\{あの/*その\}人随分変わった
人ですね
\item[B :]ええ,\{あの/*その\}人は変人ですよ}

このような場合の語彙選択も,本稿では語用論的制約に基づくものであると
考える.この文脈のように聞き手負荷制約がかからない場面では,次のような
原則が有効となるとするのである.

\enums{
{\bf 直示優先の原則:}\\
直示を優先せよ.}

即ち,直示または直示に近い手段(拡張直示)によって対象を指示できるならば,直示
を使うことが優先される.日本語の場合,それが話者にとって最も負荷の低い
指示であると見ることができる.
この原則は,\citeA{ktninti}で「指示トリガー・ハイアラーキー」と呼んだもの
である.同論文では,この原則に言語差があることも述べている.このことに
ついて7節で再びふれる.

この原則は,やはり語用論的なものなので,キャンセルされる場合がある.
聞き手負荷制約が競合する場合に,聞き手負荷制約が優先されることは既に
見た.また次のように,対立的な視点が有効である場合にもやはりソが使用可能に
なる.

\eenums{
\item[A :]『言語学原論』買いました.
\item[B :]\ul{その本}よりも,『言語学のすべて』の方がいいぞ.}

この対立的なソの使用はソの直示とも関係する.6節で再び述べる.



\section{コ系列とソ系列}
\subsection{コ系列の原型と拡張}
コ系列の原型もまたア系列と同様に狭義の直示用法が原型となると考えられる.
コの直示は2節ですでに述べたように,眼前の状況において
指差しや眼差しによって焦点化された話し手の近傍の領域と関連づけられた
要素を値とするものである.コ系列には,いわゆるコの文脈照応用法と
言われるものがあるが,これは直示用法の拡張であろう.コの文脈照応用法は,
\citeA{sakahara96}で述べられているように言語的文脈に明示的に導入された要素
を専ら指し示し,フレーム的知識等によって間接的に導入された要素は指し示せ
ないので,一見ソと同様に言語的文脈に依存した表現のようにも見える.
しかし一方で,\citeA{kuno73b}に触れられているようにソとは異なって
あたかも対象が眼前に生き生きと存在するような直示的な語感がある.
本稿ではコの文脈照応用法を,言語的表現を介した,間接直示
の一種と考える.

間接直示とは,2節で少し触れたように,
写真を見ながら「\ul{この人}は誰ですか」と言ったり,
バーのマッチ箱を取り出して「\ul{この店}に行こう」と言ったり,仕出し弁当を
食べながら「\ul{ここ}($=$今食べている弁当を作った店)も味が落ちたねえ」
などという用法である.
これらのケースでは,「写真$\rightarrow$本人」
「マッチ$\rightarrow$店」「弁当$\rightarrow$弁当屋」のような一種の
語用論的関数,あるいはマッピング・ルールを介在させることによって値にたど
りつくことができる.
直示された対象(写真,マッチ,弁当)は,最終的な真の指示対象(本人,店,
弁当屋)を代表し,指示対象そのものと臨時的に同一視されている訳である.
真の指示対象を代表する対象が眼前に存在することによって,真の指示対象
の存在が臨時に保証されるのである.

コの文脈照応用法では,
マッチ箱で店を代表して直示するように,言語的表現を手がかりとして
対象を直示していると考える.口頭の談話で
話し手がコ系列を用いるとき,しばしば対象を指差したり
手に持ったりするような身振りが見られることもこのことを裏付けるであろう
\fn{
ただしこの観察は実証されていない.}.
言語で対象を導入することによって,あたかもその対象が目の前に存在している
かのように振る舞うのである.
直示の一種であるということは,言語外世界に指示対象があらかじめ存在する
ということであると述べたが,それは
指示対象が現実世界に「実在」するか否かという
こととは独立した事象である.次のように架空の対象であってもコ系列で指示
できるが,対象は指差すように外在的なものとして取り扱われている.これは,
ペガサスの絵を指しながら「このペガサス」と称する現象と似ている.

\enums{
今から,お話を作ります.主人公は医者にしましょう.
名前を仮に,田中さんとします.
\ul{この男}はとても腕のいい心臓外科医です.}

文脈照応用法のコ系列はしばしば「談話主題」を指し示すと言われるが
\cite{syooho81,iori95a},談話の主題であると言うことは,談話に先だって
それについて述べるべき指示対象が存在しなければならないということであり,
先にみた直示的な性質とも一致する.

以下の節では,さらに3つの証拠を挙げて,コの文脈照応が直示の一種であるという
仮説を検証する.

\subsection{非対称性}
\cite{kuno73b}では,話し手は自分で導入した要素をコで指し示すことは
できるが,相手が導入した要素はコを用いて指すコとができないとしている.
次のような例である.

\eenums{\label{asymmetry}
\item[A :] 僕の友達に山田という人がいるんですが,\ul{この男}はなかなかの
理論家で……
\item[B :] \{その/??この\}人は何歳くらいの人ですか?
}

(\ex{0})では,話者Aにとって「山田という人」は自分が導入した要素であるので
「この男」で指し示せる.ところがBにとっては相手が導入した未知の要素で
あるので,ソで指し示すことはできるがコは使いにくい.このような特徴を,
コ系列の文脈照応における話し手と聞き手の``非対称性''と呼んでおこう.

この非対称性は,話し手が導入した要素であるが故にまさに話し手の近傍にあると
感じられるためと説明できる.情報の受け手がコを使えないのは,対立的な
視点のもとで対象が捉えられているからである.
これは一種の語用論的な効果であり,次の
会議の例のように,情報の受け取り手であっても,対象について自分自身の
主題でもあると捉えることができれば,コの使用は可能であろう.

\eenums{
\item[A :]……以上で,ファッション・シティ・プロジェクトの概要の説明を
終わります.
\item[B :]\ul{このプロジェクト}は,いつから開始するのかね.}

後者の例では,「プロジェクト」に関する資料が発話者の手元にあれば
より現れやすいが,その場合はもちろんまったくの間接直示である.
ともあれ,情報の与え手も受け手もコが使えるのは,対象が抱合的な視点のもと
で捉えられているからである.視点が対立的か,抱合的かによって
コの分布が変わることは,直示用法でしばしば見られることで
\cite{mikami55,ktninti},
コの文脈照応用法が直示用法と近いものであることを示す証拠と考えられる.


\subsection{先行詞との距離}
\citeA{iori95a}によれば,
ソ系列の先行詞は,基本的にその文中かまたはその直前にあるのが
普通で,それより前にあると,同一指示の解釈が非常に難しくなる.
これに対し,コの先行詞は1文以上前にあってもよい.次の例では,
先行詞「ウラル地方の鉱山都市」と指示詞表現の間に4つの文が
存在する.この場合,コ系列はまったく問題なく用いられるが,
ソ系列は不適切である.

\enums{
「私は帝政ロシアの皇女アナスタシア」.そういい続けたアンナ・アンダーソン
さんが,不遇のまま八十二歳で死んでもう八年以上たつ.イングリッド・バーグマン
主演の映画『追想』のモデルにもなり,晩年は米国に住んだ.英紙サンデー・タイムズ
が先ごろ,{\bf ウラル地方の鉱山都市}
で見つかった十一体の遺骨について,
最後の皇帝ニコライ二世と家族全員などであることが確実になったと
報じた.遺骨に残る傷跡などが一家のものと一致したという.ところが
最近になって,AP通信が「四女のアナスタシアとアレクセイ皇太子の遺骨は
含まれていなかった」という米国の法医学者の分析結果を伝えた.英紙が本当なら,
アンナさんは完全に偽物だし,APの報道通りなら,「兵士に助けられ,脱出した」
という数奇な話が多少とも真実味を帯びてくる.ロマノフ王朝の最期は,
いまだになぞめいている.一家は\{この/\#ソノ\}{\bf 鉱山都市}
のイバチョフ館と呼ばれる屋敷に幽閉されていた.\\
(「天声人語」1992.8.10,\citeA{iori95a}より)}

ところが,次のように先行詞を含む文と指示詞表現を含む文を隣接させると,
ソ系列によって同一指示の解釈が可能になる.

\enums{
……英紙サンデー・タイムズ
が先ごろ,{\bf ウラル地方の鉱山都市}で見つかった十一体の遺骨について,
最後の皇帝ニコライ二世と家族全員などであることが確実になったと
報じた.一家は\{この/ソノ\}{\bf 鉱山都市}のイバチョフ館と呼ばれる屋敷に
幽閉されていた.}

この現象について,本稿の立場から次のように解釈できる.ソ系列が
マークする領域とは,言語的文脈によって作られる状況であるが,状況の
焦点化は,まさしくその文脈を発話することによって行われる.従って,
状況が転換すればそれ以前の状況は領域から外れる.(\ex{-1})の例では,
先行詞「ウラル地方の鉱山都市」を含む状況が提示されたあと,
AP通信の報道に関する状況,および「ロマノフ王朝の最期は,いまだになぞめい
ている.」という背景的評価が提示される.そのことによって,
「ウラル地方の鉱山都市」は焦点から外されるのである.一方コ系列の
場合は,先行詞が談話主題として保持されている間は使用可能である.
談話主題とは,先に述べたように,話し手が言語表現をその代理として,
眼前にあるとみなしている対象である.
この点から明らかになるのは,ソ系列指示詞があくまで言語的文脈に
依存しているのに対し,コ系列の使用は,談話主題の指定という,談話に先だって
決定される言語外的な話し手の行為に依存しているということで,これも
コ系列の直示的な性質を表していると言える.

では,コで指し示される談話主題はどのような条件によって決定されるのであろう
か.\citeA{iori97}によれば,談話主題とは,現在話されている話題を構成する
キー概念,キー状況のネットワークのようなものとして捉えられている.
つまり,このネットワークが生きている間は,それに関連づけられた個々の
状況が文脈の中で移り変わったとしても,ネットワーク内の概念や状況を
コで指し示すことができるのである.(\ex{-1})の例で言えば,この段落全体
にわたって,「帝政ロシアの皇女アナスタシア,アンナ・アンダーソン,
最後の皇帝ニコライ{II}世,ウラル地方の鉱山都市,…」
といったキー概念のネットワークが焦点化されていると考えられる.
このネットワークは眼前にあるのと同じで,それが焦点化されている限り,
いつコで指し示してもよいのである.以上が本稿における談話主題について
の仮説であるが,より具体的な検証は今後の課題である.


\subsection{カテゴリー転換}
\citeA{sakahara96}は,日本語・英語・フランス語ともに,指示詞表現
では先行詞のカテゴリーをまったく別のカテゴリーに転換できる
ことを示している.このような現象は,英語・フランス語の定冠詞句や
それに相当する日本語の裸名詞句では起こらない.次の例では,先行詞
のカテゴリー「{\bf 熱帯林}」が指示詞表現では「この{\bf ゆりかご}」に
転換されている.

\enums{
{\bf 熱帯林}は,恐竜の時代から氷河期を超えて生き残ってきた地球最古の森林だ.
時の流れの中で,熱帯林にすむ生物は多くの種に分かれ,針葉樹林や温帯林
とは異なる複雑な生態系を作った.そして無数の未知のウィルスも,
\{この/??$\phi$\}{\bf ゆりかご}の中に身を隠している.(朝日新聞,1995.8.3)}

\citeA{sakahara96}では,このようなカテゴリー転換は,対象が属する
フレームを転換する合図となっていると分析している.

しかし,このような大きなカテゴリー転換が可能になるのは,日本語の場合,指示詞の
中でもコ系列およびア系列であって,ソ系列ではむしろカテゴリー転換は
むずかしい.このことを次の例によって確かめよう.

\eenums{
\item 五歳の誕生日に真智子は両親に熊の{\bf ぬいぐるみ}を買ってもらった.
\ul{この}{\bf 友人}を,真智子は一生大切にした.
\item 五歳の誕生日に真智子は両親に熊の{\bf ぬいぐるみ}を買ってもらった.
??\ul{その}{\bf 友人}を,真智子は一生大切にした.
}

aでは,「熊のぬいぐるみ」と「この友人」の同一指示を認識することは
たやすいが,bで「熊のぬいぐるみ」と「その友人」を同一とみなすことは
かなり難しい.これは,次のような理由によると考えられる.

ソ系列は言語的文脈によって形成され,発話によって焦点化された状況を領域
とする.そして各要素が持つ属性は,言語的文脈の語彙的意味,フレーム的意味
および最小限の推論によってもたらされる意味に限定される.それゆえに,
大きなカテゴリー転換があると指示詞表現と先行詞を結びつけることが
困難になる.
一方コ系列の場合は,指示対象は言語的文脈によって提示されは
するが,言語的文脈は指示対象を代表しているだけであって,
言語表現とは独立に,あらかじめ存在するものとして捉えられている.
それ故に,対象の同一性を保ちながら,
異なるカテゴリー付けを行う,すなわち複数の異なるフレームに属するものとして
扱うことができるのである. (\ex{0})に促して言えば,第1文は一般的な
フレームに基づいて「ぬいぐるみ」というカテゴリーを与えられた対象が,
第2文では「真智子」から見た世界のフレームに基づいて,「友人」という
カテゴリーを与えられている.ソ系列では一般的なフレームの中での
「ぬいぐるみ」しか参照できないのに対し,コ系列では,たまたま第1文
で「ぬいぐるみ」であるような対象を,眼前に存在するかのように指し示す
ので,別のフレームで別のカテゴリー「友人」が与えられても指示が成立するのである
\fn{
本稿とは観点が異なるが,\citeA{iori96a}の「言い換え」の議論も参照に
なる.次のような例文である.
\begin{quote}
(i){\bf エリザベス・テーラー}がまた結婚した.\{この/\#その
/??$\phi$\}{\bf 女優}が結婚するのはこれで7回目だそうだ.
\end{quote}
この文でソ系列が使いにくいのは,やはり先行文脈に「女優」という
属性を持つ対象が存在しないからである.}.

ついで,ア系列でもカテゴリー転換が容易であることを次の例で確かめておこう.

\enums{
山田は{\bf 主任教授の一人娘}と結婚したわけだけど,\{あの/??その\}{\bf お荷物}
を背負ってよくがんばれるもんだと感心するよ.}

指示対象は異なるフレームで「主任教授の一人娘」,「お荷物」という異なった
カテゴリーを与えられているわけであるが,ア系列の表現はその直示性の
故に同一性を表現することができる.ところがソ系列は言語的文脈で示された
フレームしか参照できないので,「お荷物」を対象と結び付けることが
できない.

\section{ソ系列特有の用法}
以下の節では,コ系列,ア系列にはないソ系列特有の用法を分析し,コ・アとは
異なる非直示的なソ系列の特質をさらに明らかにしていく.

\subsection{代行指示用法}
代行指示用法とは,「会員と\ul{その家族}($=$会員の家族)」のように
「そのN」の解釈が
「[先行詞]のN」と一致するものである.次の対比を見て分かるように,
代行指示ができるのはソ系列だけである.ア,コは当該の文脈には現れない
他の特定の対象を指示しているとしか解釈できない.

\eenums{
\item 会員は\ul{その家族}とともに宿泊することができる.(代行指示可能)
\item 会員は\ul{この家族}とともに宿泊することができる.(代行指示不可能)
\item 会員は\ul{あの家族}とともに宿泊することができる.(代行指示不可能)}

ただしソ系列でも,次のような状況を見て分かるように,直示用法では
代行指示的解釈が難しい.

\enums{
(相手が持っている蓋のない瓶を指しながら)\\
\{??その蓋/それの蓋\},どこへ行った?}

すなわち,代行指示用法はソ系列の文脈照応用法にのみ成立する用法であると
見ることができる.本稿の立場から,その理由を考えていきたい.

まず,代行指示の「そのN」は「[ [ そ ]\sitatuki{NP} の ] N」という統語構造を
持ち,「それのN」
と同等であると分析する考え方がある.しかしこれは,次のように「それの」が
ゼロ代名詞を許すのに対し,「その」が許さないという点で否定される.

\enums{
象の心臓はとても大きい.一方,鼠にも心臓はあるが,\{それの/*その\}$\phi$
($=$鼠の心臓)はとても小さい.}

これに対し,統語構造ではなく,意味的に「そ」が主名詞とは独立に指示を担うと
考えることは
特に否定されないが,なぜコやアはだめでソにだけそのような解釈が許されるのか
という疑問に対して説明が与えられなければならない.

本稿の立場では,代行指示用法も他の指示詞表現と同様に,「N(そ)」
のようなカテゴリー関数に対する領域パラメータの適用として統一的に
捉える.すなわち,「その」がマークする領域と関連づけられたカテゴリーN
である対象を値とするのである.「会員と\ul{その家族}」であれば,目下の
発話において焦点化された言語的文脈に「家族」が関連づけられるわけで,
実質的には「家族(会員)」という関数適用が成立するのである.ではなぜ
コ・アではそれができないか.直示用法の場合,「領域と関連づける」という
のは,基本的に「両域内に指示対象が存在する」ということを表す.
間接直示や種類読みのように,真の指示対象が眼前にない
場合でも,
指示対象を代表するもの,指示対象の``現れ''が眼前にあるという点で事情は
同じである.つまり,
「この家族」「あの家族」等の表現では,現在焦点化されている領域に
指示される「家族」またはその代表物が存在しなければならないのである.
ところが
ソ系列の文脈照応の場合は,領域とは言語的文脈によってのみ形成されるので,
そもそも対象の存在は前提されていない.それ故に,「その」が指定する
文脈に「そのN」の指示対象が無くても,「そのN」という表現それ自体に
よって,Nである対象を新規導入できるのである.
この点においても,ソ系列指示詞表現の指示対象の非直示性が
現れていると言えよう.

\subsection{分配的解釈}
ソ系列の指示の値の特徴をよく表す現象として,分配的
解釈の可否の問題がある.分配的解釈とは(\ex{1})のようなも
のである.

\eenums{
\item  55\%の自動車会社が\ul{そこ}の弁護士を解雇したらしい.
\item どの会社が\ul{そこ}の弁護士を解雇したのですか.}

上の例では,「そこの弁護士」は特定の会社の弁護士ではなく,
トヨタはトヨタの,日産は日産の弁護士のように分配的に解釈
される.このような解釈は指示詞表現の中ではソ系列のみで
可能である.次のコ系列,ア系列では特定・確定的な解釈しか
できない.

\eenums{
\item 55\%の自動車会社が\{ここ/あそこ\}の弁護士を解雇したらしい.
\item どの会社が\{ここ/あそこ\}の弁護士を解雇したのですか.}

(\ex{-1})の例は,いずれも「55\%の自動車会社」「どの会社」の
ような数量詞表現を含むものである.\citeA{ueyama97}は,分配的
解釈の統語的基盤をなす依存関係には FD(Formal Dependency),
ID(Indexical Dependency)の2種類があり,その分配的解釈に
かかわる表現の種類や統語構造によって FD か ID のどちらが
使用可能であるかが決まると主張している\fn{
\citeA{ueyama97}の分析は,分配的解釈以外の照応関係も考慮に入れた上で,
\citeA{hoji95}の洞察をとらえなおすことを目的としたものである.
\citeA{hoji95}では分配的解釈がどのような条件のもとで可能に
なるかが日本語と英語の観察をもとに記述されており,構造的
条件に加えて,数量詞や指示詞表現のタイプに関する言及が
必要であることが指摘された.\citeA{hoji95}では,依存関係と
してはFDのみを仮定して説明を試みているのに対して,
\citeA{ueyama97}では依存関係に2種類あると仮定した方が文法の
全体像がより明解になると論じている.FDという概念につい
ては\citeA{hoji98}も参照のこと.}.たとえば,そこでの主張
によると,(\ex{-1}a)のタイプの分配的解釈はFDにもとづくこと
しかできないのに対して,(\ex{-1}b)における分配的解釈はFDに
もとづく必要はなく,IDの例であると考えることも可能と
いうことになる.

FDは狭い意味での束縛変項照応であり,ソ系列指示詞表現が
数量詞に統語的に束縛されていなければならないが,IDでは
統語的な束縛は必要なく,線条的な先行が成立していれば,
(\ex{1})のように文境界を越える照応でも分配的解釈が生じる \fn{
(\ex{1})が IDの例であるというのは厳密には不正確な
表現であるが,FDとは別の統語的基盤にもとづいていると
いう点は変わりない.くわしくは\citeA{ueyama97}を参照のこ
と.}.

\enums{
学生たちは一生懸命論文を書いた.しかし結局だれも
\ul{その論文}を教授に提出しなかった.}

(\ex{0})では「その論文」は「学生たちがおのおの書いた自分の
論文」という分配的な解釈ができるが,コやアの表現では
やはりそのような読みは出ない.

\enums{
学生たちは一生懸命論文を書いた.しかし結局だれも
\{この/あの\}論文を教授に提出しなかった.}

(\ex{-1})の「その論文」はいわゆるEタイプ代名詞であるが,本稿で
今まで扱ってきたソの文脈照応とまったく同じものと考えられる.
本稿でUeyama説の詳細について検証することはできないが,
とりあえずこれらの例において分配的解釈が可能となる理由に
ついて本稿の立場から考えてみたい.

上の例においてソ系列指示詞表現が参照する状況とは,

\eenums{
\item どの会社(x)( xがxの顧問弁護士を解雇した )
\item 学生たちのだれも(x)( xがxの論文を教授に提出しなかった )}
のようなもので,数量詞表現によって状況自体が変域をなして
いるのである.この変域付き状況を「N(そ)」という指示詞表現に
関数適用すると,変域に応じて値域が生じる.つまり値は確定的
でなく,可変的なのである.言うまでもなく,このように可変的な
状況や値が生じるのは,それが言語的文脈によって形成されたもの
であるからに他ならない.直示される状況では,要素はすべて
確定的で,そのような変域はありえない.コ系列やア系列に
分配的な解釈が生じ得ないのはこのことから明かであろう.

上の説明はIDに関するものであったが,多少不正確な言い方を
すると,FDについても,ソ系列指示詞の性質そのものは,IDの
それとまったく同じであり,FDはIDよりも統語的な制約がさらに
厳しい特殊なケースであると考えてよいであろう.いずれにして
も,直示的なコ系列やア系列では,FDであれIDであれ,分配的解釈
を表すことができないのである.

なお,このソ系列の分配的解釈に関連して,「その日暮らし」
「その場しのぎ」等の慣用句に触れる必要がある\cite{hoji91}.
これらにおいて「その日」「その場」は名詞句の中に閉じこめら
れているので,直接的に関数としては機能しないが,意味としては
分配的な解釈が前提となっている.「\{この/あの\}日暮らし」
「\{この/あの\}場しのぎ」等の表現がないこともそれを示している.
さらに,分配的な意味を表す副詞「それぞれ」も,指示詞「それ」を
語源としていることを付け加えておこう.

\subsection{「その気」「そのつもり」「そのはず」}
次のような「その気」「そのつもり」は,動作主の意図を表すものであるが,
コ系列やア系列の表現が存在しない.

\enums{
彼を\{その気/*この気/*あの気\}にさせるのは大変だ.}
\enums{
わたしは\{その/*この/*あの\}つもりがなかったのに,行きがかりで父の
買い物につき合う事になった.}

「その意志」「その意図」等でも同様の現象が見られる.この種の表現に
コ・アが適しない理由として,「意図」というものが本来極めて概念的なもの
であり,直示しにくいものであるということが挙げられる.

次のような「そのはず」もまた,ア系列やコ系列は生じ得ない.

\eenums{
\item[A :]きのう田中さんは確かに10時前には帰ったんですね.
\item[B :]ええ,\{その/*あの/*この\}はずです.}

「〜はずだ」という表現は,言語的表現によって示された命題について論理的
妥当性を保証することを表す形式であるから,アやコで指し示されるような,
すでに存在が前提されている状況とは適合しないのである.

\subsection{「そんなに」「それほど」}
程度を表す「そんなに」「それほど」は,「こんなに」「これほど」「あんなに」
「あれほど」とは異なった用法を持っている.それは,否定対極表現として,
最終的に否定されるような量を表すという点である\cite{hattoritadasi}.
これには,(\ex{1})のような照応的な場合もある一方で,(\ex{2})のように
照応すべき言語的文脈を持たない場合もある.

\eenums{
\item[A :]荷物,重そうですね.
\item[B :]いえ,\{そんなに/それほど\}重くないです.}

\enums{
荷物は\{そんなに/それほど\}重くなかったので手に下げて歩いて帰りました.}

後者の先行詞のない例は,次に述べる曖昧指示表現の一種とも見られる.
重要なのは,コ系列・ア系列の表現では必ずマークされる領域が固定的に
存在するので,それによって指示される程度は決して否定できないことである.
例えば

\enums{
荷物は\{こんなに/これほど\}重くなかったので手に下げて歩いて帰りました.}
では,実際に指示詞で指し示される「こんなに/これほど」重い荷物が
存在するのであり,その``程度''は否定されていない.一方「そんなに/それほど」
で指し示される状況は,それがあるとするならば,最終的に否定されるため
だけに存在する,仮定・架空の状況である.そのような状況を指し示すことは,
ソには出来ても,コやアにはできないのである.

\subsection{曖昧指示のソ}
ソ系列の指示詞を用いた慣用表現がいくつかあるが,それらの中には,
照応すべき言語的文脈をもたない上に,指示対象がはっきり特定できないような
ものがある.むしろ,特定できないことを表現効果として生かしているのである.
これを仮に「曖昧指示」と呼んでおこう.曖昧指示表現におけるソ系列を
コ系列やア系列に変えると,確定的な指示対象が
あるように解釈されてしまうので,曖昧指示表現にはならない.

たとえば,次のような会話がある.

\eenums{
\item[A :]お出かけですか
\item[B :]ええ,ちょっと\ul{そこ}まで}

この表現では,話し手にとって自分の目的地はあらかじめ確定的に存在する
訳であるが,上の表現は目的地を直示するように指し示してはいないであろう.
例えば,次のようにア系列に変えると,聞き手は目的地がどこであるのか,
推論を開始せずにはいられない.それは,聞き手に指示の値が現実世界に存在する
確定的な場所として明示されているからである.

\eenums{
\item[A :]お出かけですか
\item[B :]ええ,ちょっと\ul{あそこ}まで}

「そこ」の場合は,わざと指示対象を非確定的に示しており,そのことに
「目的地を言いたくない」という話し手の意図が暗然と表現されている
ので,聞き手もそれ以上は推論を押し進めない.話し手の立場から言えば,
「行き先をそれ以上聞かないでほしい」という意志を表明しているのである.

場所に関する曖昧指示にはこのほかに,「(どこか)その辺($=$不定,未詳の
場所)」「そこらの/その辺の($=$ありふれた,平凡な)」等がある.

\eenums{
\item[A :]私のメガネ知らない?
\item[B :]\ul{その辺}に置いてるんじゃないの}

\enums{
彼は\ul{そこらの}学者とは比較にならない学識と経験を持っている.}

それぞれの意味が形成されるに至った経路については別に考えるとして,
ここで注意したいのは,これらの曖昧指示にソ系列が用いられる理由である.
それは,ソ系列が言語外世界に確定的な値をあらかじめ持たない,ということに
他ならないであろう.コ系列やア系列に変えると,必ず特定の場所と結びつけた
解釈が生じ,曖昧指示とはならないことから,それが確認できるのである.

その外に,曖昧指示といえるものとして,次のような慣用表現がある.

\enums{
(しばらくぶりで再会した人に)\\
\ul{その節}はどうもありがとうございました}

この例では,話し手・聞き手が共有しているはずの過去の経験を
ソ系列で指しているように見える.しかし,(\ex{1})のようなア系列を用いた
表現が特定の場面を指示しているのに対し,(\ex{0})はいささか具体性を欠いた
文字どおり``挨拶''的な表現との印象を与えるのである.

\enums{
\ul{あの時}はほんとに助かりました.どうもお世話になりました.}

ただし(\ex{-1})については,「その節」という表現の定形性に注意する必要が
ある.次のように「節」ではなく「時」というより一般的な表現に変えると,
ソ系列の表現は先行文脈無しには解釈できなくなる.

\enums{
??\ul{その時}はどうもありがとうございました.}

これは,「その節」という表現がソ系列の古い特徴を残す表現として残存して
いるのではないか,という可能性を示唆している.この点については,後に7節で
一つの見通しを提示したい.

\section{ソ系列の直示用法}
コ系列,ア系列については,既に直示用法がその原型的用法であり,
その他の用法が直示用法の拡張であるとの見方を示した.一方でソ系列にも
直示用法があるが,ソの直示はコやアとはたいへん異なった,複雑で微妙な性質を
持っている.眼前の状況から部分的な領域を何らかの方法で焦点化し,
その領域と関連づけられた対象,典型的にはその領域に存在する対象を値として
指し示す,という点はコやアとまったく同じであるが,領域の性質や指定の
仕方がコやアと違っているのである.

ソの直示用法には,大きく分けて人称区分的な聞き手領域指示の用法と,
距離区分的な中距離指示の用法がある.(\ex{1})は聞き手領域指示,
(\ex{2})は中距離指示の典型的な例である.

\eenums{
\item[A :](相手の着けているネクタイを見て)\ul{その}ネクタイ,いいですね.
\item[B :]ああ,これはイタリアで買ってきたんです.}

\enums{
(タクシーの客が,運転手に)\\
\ul{そこ}の信号の手前で降ろしてください.}

コやアは基本的に距離区分であり,
眼前の空間の認識と指差し・眼差し等のプリミティブな動作のみによって
領域が焦点化される.従って話し手のみによって指示は成立するので,
独り言でもコとアは用いられる.ところが聞き手領域指示は「聞き手」という
空間認識とは異質なコミュニケーション上の要素が関与しているので,
独り言にすると聞き手領域指示のソは現れない\cite{kuroda}.また
聞き手があったとしても,空間や要素が話し手と対立的に認識される場合に
のみソが使用可能となり,話し手と聞き手が接近するなどして包合的関係になると,
聞き手領域のソは消失するのである\cite{mikami55}.このことは,聞き手
によって指定される領域がコミュニケーション上有用になる場合に限り
最小限に適用されることを示している.

次に中距離指示のソであるが,``中距離''という領域は近距離,遠距離
が確定した後に寄生的に成立する領域で,プリミティブとは言えない.また
中距離のソはやはり独り言では現れにくく,聞き手への教導,注意喚起等
に多く用いられる\fn{
\citeA{sakata}が示した中距離のソの多くの用例からもそのことが伺える.}
.さらに中距離のソは,「そこ」「その辺」等の場所表現
がほとんどを占めるのであるが,これは空間が本来,連続的・一体的で境界を持たない
ために,話し手の積極的な焦点化をまって初めて区分が成立する,
ということと関連するようである.場所以外の要素は,(時間は別の問題がある
ので措くとして)基本的に空間内に分離的・離散的に存在する.そのような
対象はそれ自身が図として,地から区分されて存在しているので,それを
まず近称や遠称,あるいは聞き手領域に配置させれば指示は事足りるので,
あえて中距離という領域に当てはめる必要はないのである.
以上の点を勘案すると,中距離のソも
,聞き手領域指示のソと似て,コミュニケーション上の要請によって
隙間を埋めるように適用された用法と見ることができるのである.

\section{まとめと今後の課題}
以上の考察から,次のようなことが明らかになったと考える.

コ系列とア系列は,直示用法が原型的な用法であり,文脈照応用法
と見られるものでも,直示的な性質を色濃く残している.ア系列の文脈照応
用法と見えるものは,実際には話し手が直接経験した場面が領域となり,直接経
験という点で眼前の直示と共通する性質を持つ.コ系列の文脈照応用法
は,言語表現を指示対象の代表物として取扱い,あたかも対象が眼前にある
かのように指し示すものであった.コ系列にしてもア系列にしても,その
指示の値は確定的・唯一的であり,発話に先だって指示対象の存在が非言語的に
決定されている.一方ソ系列には,現場における直示用法,文脈照応用法および
慣用的な曖昧指示用法がある.
文脈指示用法は,言語的文脈によって形成される状況を指示の領域とする.
指示の値は言語的文脈に依存し,その指示対象の言語外の世界における
存在が直接保証される訳ではない.
曖昧指示もまた,言語外世界との間接性という点で,文脈照応用法と共通する性質を
持つのである.つまりソ系列の非直示用法は,アやコとは違って,直示としての
性質をいっさい持たないことによって特徴づけられるのである.ところが,
ソ系列には直示用法も
ある.直示である以上,その指示の値は当然眼前に唯一的に存在する訳であるから,
非直示用法とは相反する性質を持つということに
なる.

上記の点で,現代語のソ系列の指示詞は多義であると言わざるを得ない.
問題は,このような多義性が,単一の原型からの派生として説明できるのか,
できるとすればどのような派生の経路がありうるのか,という点である.
ここで,ソ系列の直示の特殊性,複雑性が考慮されなければ
ならない.ソは直示においてもさらに人称区分と距離区分に分かれるのである
が,いずれもプリミティブな用法というより,コミュニケーション上の要請に
より隙間を埋める充填剤的な手段として適用されているのである.このことは,
直示用法がソ系列の原型的用法であるという可能性の低いことを指し示して
いるようである.

しかしこれだけの証拠だけでは,ソ系列が直示から非直示へ拡張したのか,はたまた
その逆かを決定することはできない.なぜなら,語義の変化においては,
原型的な用法から派生的な用法が生じたあとで,原型的なものが
衰弱して,派生的なものがかえって本来的なものに見えることがしばしば
あるからである.
ここから先は,歴史的資料に
証拠を求め,また対照的研究をも援用しなければならないであろう.それは
すでに本稿の守備範囲を超えるが,ここで,とりあえず大まかな見通しだけを
述べて締めくくりとしたい.

筆者の予備的調査および\citeA{okazaki97}で収集されたデータを見ると,
古代語のコ系指示詞は現代語とほとんど違いがないが,中称(ソ,シ,サ,シカ等を
含める)および遠称(カ,ア)の分布は現代語と異なる.要点をまとめると
次の通りである.

\eenums{
\item 中称は現代語と同様に文脈照応によく用いられ,分配的解釈もみられる
一方で,文脈照応と独立した純然たる直示用法は中世末まで見られない.
\item 直示優先の原則が弱い.すなわち,目に見えているものでも,
2度目以降に言及される場合は直示ではなく文脈照応の中称が用いられる
場合がある.
\item 眼前になく,話し手の記憶内にある対象は,古くは中称で指し示された.
時代が下るに従って,記憶指示の遠称の用法が拡大していく.}

上記の特徴から考えると,日本語の中称とはもともと,直接的に,今ここで知覚に
捉えられない対象を指し示す指示詞であったと捉えることができる.
記憶指示に相当する表現に遠称ではなく中称が充てられる点にその特徴が
端的に表れている.また,文脈照応用法とは本稿で言うI--領域指示であり,
これも言語的文脈を介した間接的な指示であるので,やはりこの規定に合致する
のである.ただし,
直接的に知覚に捉えられる対象であっても,文脈照応的に指される場合は,
「知覚に捉えられない」方の扱いになる.また直示優先ではないという点も
重要である.このような体系から,現代語へといたる変化は次のようなものと
考えられる.

一つには遠称の直示が拡張されて記憶指示用法を
発達させる過程である.また一つは,それとともに,直示優位の原則が生じ,
非直示用法においてソ系列の使用範
囲が狭まっていく過程である.そしてもう一つは,おそらく文脈照応用法を足がかりに
して,直示用法においてソ系列の領域が確定していく過程である.
記憶指示用法が発達し,直示優先の原則が成立すると,ソ系列は自然に
自分の知識とは対立する情報を指し示すことが多くなる.これが,ソ系列と
聞き手を結びつける契機となるのである.中距離指示の用法も,聞き手への
教導,注意喚起という,
自分の知覚とは対立する相手の知覚に関わる文脈に現れるという点で,
聞き手領域指示と共通する基盤を持っている.

ここで大いに参考になるのが,朝鮮語の状況である.
朝鮮語には日本語に似た3系列の指示詞が存在し,中称が文脈照応に用いられる
点も共通するが,次の3点で現代日本語と異なる.1つは,遠称指示詞(アに相当)の
記憶指示用法がなく,過去の経験
はすべて中称指示詞(ソに相当)で表されるという点である.たとえば,
「その節はお世話になりました」は実は朝鮮語にほぼ直訳できる\fn{
「\ul{ku}ttaynun komawesseyo(\ul{その}時はありがとうございました.)」など.}.

第2点は,「直示優先の原則」あるいは指示トリガー・
ハイアラーキーが朝鮮語では弱く,連続した文脈において同一指示を
行う場合は,直示できる対象であっても,文脈照応用法の中距離指示詞を
用いることができるという点である\fn{ただし,直示優先が働かないのは
遠称の場合のみで,近称と文脈照応では近称が優先される.}.

第3点は,上のこととも深く関係するが,中称指示詞の現場指示的な用法が
日本語よりもはるかに弱いという点である.中称による聞き手領域指示は,聞き手が
実際に手に触れられるもの,身に着けているものぐらいに限定されていて,
聞き手から少しでも離れると使えない.また中距離指示もない.

以上の点で,朝鮮語は,現代日本語よりも古代日本語に近く,同じような出発点から
やや異なる方向に発展しつつある体系であると見ることができる.

上記の仮説をこれ以上検証することは既に本稿では果たせないので,
今後の課題としたい.
古代日本語の用法を視野に入れ,それと現代語との連続性・非連続性を
明らかにしつつ,さらに朝鮮語のような外国語の状況も参照することに
よって,現代語日本語の体系はいっそう明らかになるものと考える.


\section*{付記}
本稿のドラフトに対して懇切なるご助言とご教示を賜った田窪行則氏,傍士元氏,
上山あゆみ氏に対し,心から感謝の意を表します.
また,朝鮮語のデータは 美庚さん(慶尚北道出身,30歳,女性)の面接調査
に基づいています.ご協力に感謝いたします.

\bibliographystyle{jnlpbbl}
\bibliography{v06n4_04}

\begin{biography}
\biotitle{略歴}
\bioauthor{金水 敏}{
1979年東京大学文学部国語学専修課程卒業.文学修士.
東京大学文学部助手,神戸大学教養部講師,大阪女子大学助教授
,神戸大学文学部助教授を経て,1998年より大阪大学大学院文学研究科
助教授,現在に至る.

専門は日本語の文法史であるが,田窪行則氏との共同研究で,
現代日本語の談話管理に関する研究も行っている.
}

\bioreceived{受付}
\biorevised{再受付}
\bioaccepted{採録}

\end{biography}

\end{document}

