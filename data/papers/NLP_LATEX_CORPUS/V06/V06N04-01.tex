\documentstyle[epsf,jnlpbbl]{jnlp_j_b5}

\setcounter{page}{3}
\setcounter{巻数}{6}
\setcounter{号数}{4}
\setcounter{年}{1999}
\setcounter{月}{7}
\受付{1997}{10}{1}
\再受付{1998}{1}{28}
\採録{1998}{12}{24}

\setcounter{secnumdepth}{2}

\unitlength 1mm

\title{日本語の照応関係理解に関する一考察\\—「主題」(Topic)が果た
す役割を中心に—}
\author{横川 博一 \affiref{kyoto-edu}\affiref{kyoto-foreign}}
\headauthor{横川}
\headtitle{日本語の照応関係理解に関する一考察—「主題」(Topic)が果
たす役割を中心に—}
\affilabel{kyoto-edu}{京都教育大学}
{Kyoto University of Education}
\affilabel{kyoto-foreign}{京都外国語大学}
{Kyoto University of Foreign Studies}

\jabstract{
 日本語の照応関係理解のプロセスにおいて,どのようなストラテジーが関与
しているのかについて,言語心理学的実験を通して考察した.実験1では,自
己のペースによる読解課題およびプローブ認識課題を用いて,日本語の主語を
表す「が」と主題を表す「は」の違いが照応関係理解に影響を及ぼすかどうか
について調査した.その結果,「は」でマークされた名詞句で読解時間がかか
る傾向が見られ,それを照応表現の指示対象として優先する傾向が見られた.
また,プローブ認識課題では,主題を表す「は」の影響が見られ,目的語名詞
句よりも主語名詞句をプローブ語として呈示した場合の方が判断時間が速い傾
向が見られた.このように,主題の影響が見られたことから,「主題割当方略」
とでも言うべきストラテジーが利用されていることが分かった.実験2では,
英語の実験に基づいて提案されている「主語割当方略」や「平行機能方略」
と呼ばれるストラテジーが日本語の照応理解にも利用されるのかどうかにつ
いて調査した結果,parallelな構造をもつ文では,平行機能方略が用いられ
ることが分かった.さらに実験3では,これら2つのストラテジーおよびそ
の他のストラテジーと主題割当方略との相互関係について調査を行った.そ
の結果,日本語の照応関係理解のプロセスでは,これらのストラテジー競合
する場合,主題割当方略が優先的に利用されることが分かった.このことは,
日本語が「主題卓立言語」としての性質を持っていることを示している.
}

\jkeywords{照応解決,発見的ストラテジー,主題割当方略,主語割当方略,
平行機能方略}

\etitle{Anaphoric Resolution in Japanese Sentences: \\
Topic Assignment Strategy}
\eauthor{Hirokazu Yokokawa \affiref{kyoto-edu}\affiref{kyoto-foreign}} 

\eabstract{
 This paper describes three psycholinguistic experiments on
anaphoric resolution during sentence comprehension in Japanese. 
   In Japanese language, there is a postposition ``wa'' which signifies
``topic'' and ``ga'' which signifies ``(grammatical) subject''. Experiment 1 
investigated the influence of the difference between ``wa'' and ``ga'' to
assign ambiguous pronouns. In a self-paced reading paradigm, reading
times were longer subject noun phrase than object noun phrase,
irrespective of the difference of postposition, and there were more
assignment to the topic noun phrase(Topic-NP: ``NP+wa'') than subject
noun phase(Subject-NP: ``NP+ga''). In a probe recognition task, reaction 
times (RTs) were faster subject(S-probe) than object(O-probe), and RTs 
for S-probe were faster Topic-NP sentence than Subject-NP sentence.
Overall, the influence of topic was observed, thus suggesting that the 
topic assignment strategy (TAS) were utilized during the assignment of 
pronouns in Japanese.
   Experiment 2 investigated the influence of another heuristic
strategies which have been proposed to account for the assignment of
pronouns in sentences in English: the subject assignment strategy
(SAS) and the parallel function strategy (PFS). Furthermore,
Experiment 3 investigated to distinguish between the heuristic
strategies: TAS, SAS, and PFS.  In both Experiment 2 and 3, there was
a strong preference assigning a pronoun to the preceding Topic-NP,
thus showing TAS was predominantly used over other heuristic
strategies. These findings support that Japanese has a linguistic
nature of a topic prominent language. 
}

\ekeywords{anaphoric resolution, heuristic strategies, 
Topic Assignment Strategy,  Subject Assignment Strategy,
Parallel Function Strategy}

\begin{document}
\maketitle

\section{はじめに}
 照応関係の理解は,統語的・意味的レベルの問題であるとともに,談話レベ
ルの問題でもあり,照応表現とその先行詞をどのように同定するかは,言語
理論にとっても~\cite{sag}~\cite{tsujimura:1996}~\cite{imanishi:1990},
工学的な談話理解システムを構築する上でも重要な課題である
~\cite{nakaiwa:1996}~\cite{murata:1997a}~\cite{tanaka:1979}.
 本稿では,日本語の照応表現について,発見的ストラテジー(heuristic
strategy)が照応関係理解のプロセスでどのように関与するのかについて心理
言語学的実験を通して考察する.

\section{人間の照応関係理解}

\subsection{照応理解モデル}
 英語の照応関係理解に関して行われてきているさまざまな実験結果に基づい
て,~\cite{abe:1994}は照応関係理解過程のモデル化を試みている.
彼らは,代名詞および名詞句による照応,ゼロ代名詞に対するモデルを提案し
ているが,それらに共通したプロセスは,概ね,次の図~\ref{fig:process}に
示す通りである.

\begin{figure}[t]
  \begin{center}
   \epsfile{file=5.eps,height=88mm}
  \end{center}
  \caption{照応関係理解のプロセス}
  \label{fig:process}
\end{figure}

 彼らは,現在のモデルはまだまだ不十分であるとしているが,筆者は,こ
うしたモデルをさらに精緻化するという意味で,特に次のような問題を解決
する必要があると考える.

\begin{itemize}
  \item 制約条件の設定がどのようなメカニズムで行われるのか.\\
  先行文脈(照応表現を取り巻く文脈)から得られる統語的情報や意味的情報
  が,照応理解のプロセスにおいて,いつどのように利用されるのか,また,
  先行文脈が照応関係の候補決定および照応解決に対して,いつ,どのような
  影響を及ぼすのかについて,さらに調査する必要がある.
  \item 照応関係の決定がいつ行われるのか.\\
  指示対象があいまいな照応表現が認定された時点で,即時的に指示対象が仮
  に決定されるのか,文末(あるいは以後の処理)まで留保されて決定が遅れ
  るのかを明らかにする必要がある.
  \item 指示候補の選択はどのようなメカニズムで行われるのか.\\
  照応表現の指示対象候補が複数ある場合,その候補からの選択に働く原理を
  明確にする必要がある.
\end{itemize}

\footnotetext[1]{ここでいう「意味処理」とは,照応表現をとりまく文脈
の意味処理を指す.}

\subsection{研究課題}
 本稿では,照応表現の指示対象候補が複数ある場合に,上記モデルの「制約
 条件の設定」および「指示候補の選択」がどのようなメカニズムで展開する
 のかに焦点を当てて,以下の点について考察する.

\begin{itemize}
  \item 文脈によって規定される「主題」(topic)という概念が,指示対象の候
  補絞り込みに,どのような影響を及ぼすのか.
  \item 日本語の代名詞の指示対象を同定するプロセスにおいても,英語に対
  して提案されている発見的ストラテジーである「主語割当方略」(Subject
  Assignment Strategy)や「平行機能方略」(Parallel Function Strategy)
  が用いられるのか.
\end{itemize}

\section{主題割当方略 (Topic Assignment Strategy)}

\subsection{英語の照応関係における「主題」の役割}
 代名詞の照応表現に関して,文文法レベルにおける統語的制約だけでは,代
名詞とその指示対象(先行詞)との間にある照応関係を説明するのに十分では
ない.そこで,主題などの機能構文論的な概念を導入した説明がなされること
がある.理論的にも照応関係を決定する一つの要因として,主題という概念が
大きな役割を果たしていることが主張されている~
\cite{takami:1997}~\cite{takami:1987}~\cite{kanzaki:1994}.例えば,~
\cite{takami:1997}は,主題の概念を用いて,次のような文の適格性を説明す
ることができるとしている.


\vspace{3mm}
\begin{tabular}{cl}
(1)& a. John $i$ took out Mary to dinner when he $i$ went to Boston.\\

   & b. $*$He $i$ took out Mary to dinner when John $i$ went to Boston.\\
(2)& a. When he  $i$ went to Boston, John $i$ took out Mary to dinner.\\
   & b. When John $i$ went to Boston, he $i$ took out Mary to  dinner.\\
\end{tabular}
\vspace{-2mm}
\begin{flushright}
\cite{takami:1997}
\end{flushright}

 (1a)が適格であるのは,主節の主語であるJohnが同時に主題であるからであ
る.それに対して,(1b)では主節の主語が主題ではあるが,代名詞になってい
るため不適格となる.言い換えると,主節が従属節に先行する場合,代名詞が
主節の主語になることはできない.ところが,(2b)が適格であるのは,従属節
と主節が意味上,独立した等位節として解釈され,Johnが意味上,この文全体
の主題として解釈されるためであるとしている.

\subsection{日本語の照応関係における「主題」の役割}
 言語類型論的な観点から見ると,英語などの言語は「主語卓立言語」
(subject prominent language) であり,中国語などは「主題卓立言語」
(topic prominent language)
であり,日本語はどちらの要素も多く持っていると言われている 
~\cite{huang:1984}~\cite{li:1976}~\cite{kanzaki:1994}.日本語には,主
題を示す標識である助詞「は」があり,もし,助詞の情報が利用されるとすれ
ば,日本語などの主題卓立の要素をもった言語には,(3)に示す「主題割当方
略」とでも言うべきストラテジーが利用されると仮定することができる.

\vspace{3mm}
\begin{tabular}{cl}
(3)& 主題割当方略(Topic Assignment Strategy; TAS)\\
   & 代名詞には,先行する文脈の中で「主題」である指示対象を割り当てよ.\\
\end{tabular}
\vspace{3mm}

次の例文を見られたい.

\vspace{3mm}
\begin{tabular}{cl}
 (4)& a. 太郎が先週の金曜日健太を殴り,そして彼は慌てて逃げた.\\
   & b. 太郎は先週の金曜日健太を殴り,そして彼は慌てて逃げた.\\
\end{tabular}
\vspace{3mm}

日本語では,格助詞「が」は主語を表すのに対して,「は」は主題を表すと言
われている.(4a)では,文頭名詞句「太郎」に「が」が付与されていることか
ら,主語であることが示され,後続の「彼」と同一指示的となっている.また,
(4b)では,文頭名詞句「太郎」に主題を示す「は」が付与されており,後続の
「彼」と照応が可能である.このような現象などから,\cite{kanzaki:1994}
は,(5)のような一般化を行っている.

\vspace{3mm}
\begin{tabular}{cl}
(5)& 1文中の代名詞の照応\\
   & 代名詞は話題\footnotemark[2]になっている句(節)の中の同一名詞句と照応が可能である.
\end{tabular}
\vspace{3mm}

\footnotetext[2]{「主題」,「話題」とは,\cite{halliday:1985}の定義に従え
ば,主題が文頭にくる要素であるのに対して,話題は必ずしも文頭にある必
要はない.したがって,主題は話題の一部とみなすことができ,この点を除
いてはほぼ同義と考えることができる.しかし,特にここでは厳密な区別は
必要ないので,本稿では,「主題」という用語を用いることにする.}

主要部後置型言語 (head-final language)
である日本語では,文理解プロセスにおいて「助詞」が重要な役割を果たし,
それが担う情報が利用されている.人間の文理解プロセスにおいては,主題を
示す標識である「は」と主語を示す「が」の相違が影響を及ぼすと考えられる.
 また,ゼロ代名詞の照応関係についても,主題という概念が重要な役割を果
たしていると考えられる.\cite{takami:1997}は(6a)と(6b)に見ら
れる文の適格性の違いを,主題という概念を用いて説明している.


\vspace{3mm}
\begin{tabular}{cl}
 (6)& a. $\phi i$会社から帰ったとき,父 $i$は青い顔をしていた.\\
    & b. $*\phi i$ 会社から帰ったとき,父$i$が青い顔をしていた.\\
\end{tabular}
\vspace{3mm}

(6a)が適格であるのは,「父」が主題を表す「は」でマークされているため,
従属節のゼロ代名詞の先行詞とも解釈されるからである.それに対して,(6b)
は,「父」が中立叙述を表す「が」でマークされており,この文には主題が明
示されていない.このような場合,従属節のゼロ代名詞の先行詞は,通例,話
し手と解釈される.

\subsection{Nagata (1991,1995)の実験}

 日本語の照応理解に関する心理言語学的実験はそれほど多くないが,ここで
は日本語の照応関係理解の「主語の優位性」を主張している
~\cite{nagata:1991}~\cite{nagata:1995}を取り上げ,そ
の成果と問題点について簡単に述べることにする.

 ~\cite{nagata:1991}では,日本語の再帰代名詞「自分」が,
主語名詞句と間接目的語のいずれも先行詞としうる統語的曖昧文を用いて,
Probe Recognition Taskを行った.プローブ語は,再帰代名詞の直後か文末で
呈示される.

\vspace{3mm}
\begin{tabular}{cl}
 (7) & 太郎$i$は 花子$j$に 自分$i/j$の 家族の 話ばかり された.\\
\end{tabular}
\vspace{3mm}

(7)では,統語的には「自分」は主語の「太郎」も目的語の「花子」も先行
詞としてとり得る.しかし,プローブ語を呈示する位置に関係なく,文の主
語名詞句が呈示された場合の方が間接目的語を呈示した場合よりも判断時間
が短いことから,主語の優位性を主張している.


  さらに,~\cite{nagata:1995}では,日本語の再帰代名詞「自分」が,主
  節主語と従属節主語のいずれも先行詞としうる統語的曖昧文を作成し,ど
  ちらを先行詞とするかについて,(8)のようなlogophoric文と
  non-logophoric文を用いて,同様の実験を行った.(8a)では,主節の動詞
  は,個人の視点や思考,感情,意識の状態を反映する動詞であり,再帰代
  名詞の指示対象は曖昧である.このようなタイプの束縛特性を含む文は
  logophoric文と呼ばれる.それに対して,(8b)では指示対象の曖昧さは見
  られない.このような文はnon-logophoric文と呼ばれる.


\vspace{3mm}
\begin{tabular}{cl}
(8)& a. 先生$i$は 生徒$j$が 自分$i/j$の 誤りを 見つけたと すぐに 察した.\\
   &   b.先生$i$は 生徒$j$が 自分$*i/j$の 誤りを 見つけたとき すぐにほめた.\\
\end{tabular}
\vspace{3mm}

このような文を用いた場合は,プローブ語を呈示する位置に関係なく,従属節
主語よりも主節主語が呈示された場合の方が,判断時間が速い傾向が見られた.
この傾向は,(8b)のように主節主語が先行詞として不適切な場合にも見られた
ことから,主節主語の優位性を主張した.

 このように,英語における先行研究で指摘された「主語の優位性」が,日本
語でも確認されたと主張されているが,~
\cite{nagata:1991}~\cite{nagata:1995}で用いられた実験刺激文は,主語名
詞句が「は」でマークされており,指示対象候補を設定する段階で「主題の優
位性」が影響している可能性がある.そこで,主語の優位性と主題の優位性と
を区別する実験を行った.

\subsection{心理言語学実験}
 本稿の以下の節では,次に示すような心理言語学実験の結果を報告する.

\vspace{-4mm}
\subsubsection{被験者}
 日本語を母語とする大学生18名.
\vspace{-4mm}

\subsubsection{実験方法}
\vspace{-2mm}
  実験課題は,以下に述べるSelf-Paced Reading TaskおよびProbe
Recognition Taskの2種類を用意した.被験者は,実験の目的についての説明
を受けた後,Self-Paced Reading Taskの練習(3試行)を行った後,テスト
問題を行い,続いてProbe Recognition Taskの練習(3試行)を行った後,テ
スト問題をそれぞれ行った.以下にそれぞれのTaskについて詳述する.

\begin{itemize}
  \item Self-Paced Reading Task\\
     実験文および内容理解を確認するQ\&Aはすべてコンピュータの画面
     上に文節ごとに呈示した.最初に実験手順を説明した後,実験に慣れる
     ための練習を行った.

      実験では,初期画面には「準備ができたら何かキーを押して下さい」
     という指示が数秒出ており,被験者がスペース・キーを押すと,画面中
     央に*****マークが数秒間呈示され,この位置に実験文が呈示されること
     を示す.実験文はすべて文節ごとに呈示され,被験者の自己のペースで
     読み進めるにつれて,前の文節は消えていく.

      1文の呈示が終わると,内容理解を確認するQ\&Aが現れ,被験者は選択
     肢の数字キー(1または2)を押し,最後にリターン・キーを押すと1つ
     の試行が終了する.Q\&Aは照応表現の先行詞を尋ねる問題で,たとえば,
     「〜したのは誰ですか?1)太郎 2)健太」というような形式になっ
     ている.被験者はなるべく速く,正確に読むように,かつ内容を理解す
     ることがもっとも大切であると言われた.
 
       コンピュータには,各被験者の文節ごとの読解時間,Q\&Aの選択した解
     答および反応時間が自動的に記録された.
  \item Probe Recognition Task\\
    実験文の呈示方法は,Self-Paced Reading Taskと同じく,文節ごとに自
     己のペースで読み進めるが,文の途中で(本実験では,代名詞の直後)
     同じ画面にプローブ語(ターゲットへの手がかり)が呈示され,その語
     が文中に存在していたかどうかを判断する.プローブ語は,代名詞が指
     示対象とする人名である.心理学実験における反応結果(ここではプロー
     ブ語の判断時間)の一方がもう一方に比べて速い場合,速い方の人名が
     脳内で活性化されているのではないかと解釈することができる~
     \cite{abe:1994}.

\end{itemize}

\subsection{実験1:照応理解における主題割当方略}
 本実験では,3.2節の(3)に示した「主題割当方略」という発見的ストラテジー
が,日本語の照応理解のプロセスで用いられるのかどうかについて検証する.

\subsubsection{実験刺激文}

 表\ref{table:exam1s}に実験1で用いられる刺激文の例を示した.それぞれ(a)の文は,主語名詞句に助詞「が」が付与されており,(b)の文は,主語名詞句に助詞「は」が
付与されている.このような基準で6セット作成した.なお,文中の▲は
Probe Recognition Taskにおけるプローブ語の現れる位置を表す.

\begin{table}[htbp]
\begin{center}
\caption{実験1で用いる刺激文}
\label{table:exam1s}
\begin{tabular}{cl}\hline
(9)& a. 太郎$i$が 花子に [健太$j$が 自分$i/j$を ▲ 批判した] と言った.\\
   & b. 太郎$i$は 花子に [健太$j$が 自分$i/j$を ▲ 批判した] と言った.\\
(10)& a. 太郎$i$が 花子に [健太$j$が 自分自身$i/j$を ▲批判した] と言った.\\
    & b. 太郎$i$は 花子に [健太$j$が 自分自身$i/j$を ▲ 批判した] と言った.\\
(11)& a. 太郎$i$が [健太$j$が 自分$i/j$の車で ▲ 東京へ行った] と思っている.\\
    & b. 太郎$i$は [健太$j$が 自分$i/j$の車で ▲ 東京へ 行った] と思っている.\\
(12)& a. 太郎$i$が [健太$j$が 自分自身$i/j$の車で ▲ 東京へ行った]と 思っている.\\
    & b. 太郎$i$は [健太$j$が 自分自身$i/j$の車で ▲ 東京へ 行った]と 思っている.\\
(13)& a. 太郎$i$が 先週の金曜日 健太$j$を 殴り,そして彼$i/j$は ▲ 次郎を 殴った.\\
    & b. 太郎$i$は 先週の金曜日 健太$j$を 殴り,そして 彼$i/j$は ▲ 次郎を 殴った.\\
(14)& a. 太郎$i$が 先週の金曜日 健太$j$を 殴り,そして 次郎は 彼$i/j$を ▲ 殴った.\\
    & b. 太郎$i$は 先週の金曜日 健太$j$を 殴り,そして 次郎は 彼
   $i/j$を ▲ 殴った.\\ \hline
\end{tabular}
\end{center}
\end{table}


\subsubsection{結果の予測}
 それぞれの刺激文において,「が」を用いた(a)の文よりも「は」を用いた(b)の文で,Q\&A反応において「太郎」を照応表現の先行詞として割り当てる反応が多ければ,TASが用いられていることになる.
 また,「が」を用いた(a)の文よりも「は」を用いた(b)の文で,Probe
Recognitionの判断時間が短かければ,TASが用いられていることになる.

\subsubsection{結果と考察}
 表\ref{table:exam1r}にSelf-Paced Reading Taskにおける読解時間・Q\&A反
応結果,およびProbe Recognition Taskにおける判断時間(ms)(PRT反応と略
記)を示す.

\begin{itemize}
  \item Self-Paced Reading Task\\
 照応表現の先行詞を判断するQ\&Aでは,「は」が付与された場
合,その反応率が増加する傾向が見られた.特に,(10)のように
先行詞として同一節内の名詞句をとる傾向の強い「自分自身」
に対しても「は」による逆転が見られた.ただ,(12)においては
むしろ「太郎(は)」を先行詞とする反応が減少したが,これは
「自分自身」がさらに深く名詞句の中にある(「自分自身の車」)
ことが影響した可能性がある.
「が」が付与された名詞句よりも「は」が付与された名詞句で
読解時間がかかる傾向が見られ,分散分析を行った結果,それぞ
れの刺激文において両者の間に有意差が見られた((10a-b):
$F(1,17)=11.214, p<0.05$; (11a-b): $F(1,17)=14.461, p<0.01$;
(12a-b): $F(1,17)=143.831, p<0.01$; (13a-b): $F(1,17)=17.235, 
p<0.01$; (14a-b): $F(1,17)=5.612, p<0.05$).
この傾向は,助詞「は」のもつ情報が利用されていることを示
していると考えられる.

  \item Probe Recognition Task\\
 「が」を含む(a)の文では,文によってばらつきが見られ,「健太」の方が
判断時間が速い場合もある.それに対して,(a)の文に比べて「は」を含む(b)
の文では一貫して,プローブ語として「太郎」が呈示された場合の方が判断時
間が速かった.これらの結果は,主題を表す「は」の情報が利用され,照応理
解のプロセスでTASが利用されていることを示している.
\end{itemize}


\begin{table}
\begin{center}
\caption{実験1の結果}
\label{table:exam1r}
\begin{tabular}{|rl|r|r|r|r|r|r|} \hline
 \multicolumn{2}{|c|}{}&
 \multicolumn{2}{c|}{読解時間} &
 \multicolumn{2}{c|}{Q\&A反応} &
 \multicolumn{2}{c|}{PRT反応(ms)} \\ \hline
 \multicolumn{2}{|c|}{実験文} & 
  太郎が/は & 照応表現 & 太郎 & 健太 & 太郎 & 健太 \\ \hline
 (9)a  & 太郎が\ldots 自分を \ldots & 14.77 & 21.60 & 13 & 5 & 1507 & 2272\\
    b  & 太郎は\ldots 自分を \ldots & 17.37 & 17.15 & 17 & 1 & 1372 & 1547\\
 (10)a & 太郎が\ldots 自分自身を \ldots &13.23 & 16.07 & 7 & 11 & 1598 &1488\\
     b & 太郎は\ldots 自分自身を \ldots &17.77 & 21.89 & 10& 8 & 1367 &1482\\
 (11)a & 太郎が\ldots 自分の \ldots & 13.07 & 28.65 & 3 & 15 & 1477 & 1488\\
     b & 太郎は\ldots 自分の \ldots & 17.07 & 19.65 & 8 & 10 & 1303 & 1627\\
 (12)a & 太郎が\ldots 自分自身の \ldots & 14.43 &	18.68 & 5 & 13
 & 1513 & 1528\\
     b & 太郎は\ldots 自分自身の \ldots & 19.15 & 22.38 & 4 & 14 &
 1340 & 1485\\
 (13)a & 太郎が\ldots 彼は \ldots & 13.36	& 12.48	& 12 & 6 &
 1397	& 1888\\

     b & 太郎は\ldots 彼は \ldots & 17.34	& 12.19	& 13 & 5 &
 1360	& 1400\\

 (14)a & 太郎が\ldots 彼を \ldots & 12.84	& 10.98	& 11 & 7 &
 1445	& 1817\\

     b & 太郎は\ldots 彼を \ldots & 15.21	& 14.66	& 14 & 4 &
 1323	& 1898\\ \hline
\end{tabular}
\end{center}

\end{table}

\section{主語割当方略および平行機能方略}

  実験1では照応表現の解釈に主題の効果があることを見た.次に,英語に
  おける心理言語学的実験に基づいて提案されている他の発見的ストラテジー
  が日本語の照応関係理解のプロセスで利用されるかどうかについて考察す
  る.

\subsection{照応理解における主語割当方略および平行機能方略}

  英語における心理言語学的実験で「発見的ストラテジー」として提案されて
  いるものに,主語割当方略(Subject Assignment Strategy)と平行機能方
  略(Parallel Function Strategy)と呼ばれるものがある.これらのストラ
  テジーは概略次のようなものである.

\vspace{3mm}
\begin{tabular}{cl}
(15)&主語割当方略(Subject Assignment Strategy; SAS)\\
  & 代名詞がいかなる文法的位置にあっても,それには,\\
    & 先行する節中の主語位置にある名詞句の解釈を付与  \\
    & せよ.\\
(16)&平行機能方略(Parallel Function Strategy; PFS)\\
    & 代名詞には,先行する節中で代名詞と同じ文法的位 \\
    & 置にある名詞句の解釈を付与せよ.\\
\end{tabular}
\vspace{3mm}

  これらのストラテジーをめぐって様々な心理言語学的実験が行われてきてい
る~\cite{crawly:1990}.Crawlyらは,SASとPFSのどちらが利用さ
れているかについて実験を行った.

\vspace{3mm}
\begin{tabular}{cl}
(17)& a. John hit Bill and he ran away.\\
    & b. John hit Bill and Mary kicked him.\\
\end{tabular}
\vspace{3mm}

彼らは,(17a)の代名詞のみならず,(17b)のような文においても,目的語代名
詞には前節の主語名詞句を割り当てるという結果から,SASを支持する結果が
得られたとした.また,性別の手がかりがある場合には,両者のストラテジー
間には差が見られないことから,発見的ストラテジーの使用は他に決定的な手
がかりが存在しない場合に限られるとした.

 一方,~\cite{smyth:1994}は,parallelな統語構造を持っている文では,
PFSの使用が促進され,それ以外の構造においては,SASが促進されることを
示した.また,これらのストラテジーは他の手がかりが存在しないときに活
用されることも示した.このように,英語に関する先行研究では矛盾する結
果が得られているので,日本語の場合について実験を行うことにした.

\subsection{実験2}

 本実験では,指示対象が曖昧な代名詞を用いて,日本語の照応理解のプロセ
スで,SASあるいはPFSといったストラテジーが用いられるのか,また用いられ
るとしたらSASか,それともPFSなのかについて調査する.

\subsubsection{実験刺激文}
 表\ref{table:exam2as}に実験2で用いられる刺激文を示した.実験文に
は,等位接続詞「そして」によって接続された第2節の主語が代名詞の場合
(主語代名詞条件)と目的語が代名詞の場合(目的語代名詞条件)を設定し
た.第1節には同姓の2人の人物が登場し,第2節中の代名詞は2人の人物
のいずれも指示することが可能である.なお,文中の▲はProbe
Recognition Taskにおけるプローブ語の現れる位置を表す.

\vspace{-3mm}
\begin{table}[htbp]
\begin{center}
\caption{実験2で用いる刺激文}
\label{table:exam2as}
\begin{tabular}{cl}\hline
(18)&「が」:主語代名詞\\
    & a. 太郎が  健太を 殴り そして 彼は ▲ 次郎を 殴った.\\
    &「が」:目的語代名詞\\
    & b. 太郎が  健太を 殴り そして 次郎は 彼を ▲ 殴った.\\ \hline
\end{tabular}
\end{center}
\end{table}
\vspace{-3mm}

\subsubsection{結果の予測}
 いずれのストラテジーが用いられても,(18a)の「彼」は先行節の「太郎」
が指示対象として優先されることが予測される.それに対して,(18b)では,
SASが用いられれば「太郎」が,PFSが用いられば「健太」を指示対象として選
択することが予測される.
 これらは,Self-Paced Reading TaskのQ\&A反応およびProbe Recognition
Taskの反応時間から判断することができる.PRT反応では,(18a)ではいずれの
ストラテジーを用いた場合でも,「太郎」の方が「健太」に比べて反応時間が
速いはずである.しかし,(18b)ではSASが用いられれば,プローブ語として
「太郎」が呈示された方が,PFSが用いられれば「健太」が呈示された方が反
応時間が速いことが予測される.

\subsubsection{結果と考察}
  表\ref{table:exam2ar}に,Self-Paced Reading Taskにおける読解時間(
  Q\&A反応結果,およびProbe Recognition Taskにおける判断時間(ms)(PRT
  反応)を示す.

\begin{itemize}
  \item Self-Paced Reading Task\\
    刺激文(18a-b)では,前節の主語を先行詞とする傾向が見られた.この傾
    向は,主語代名詞の場合だけでなく,目的語代名詞の場合にも見られたこ
    とから,SASを支持する結果が得られた.
 \item Probe Recognition Task\\
    まず,(18a)では,「健太」(1032ms)よりも「太郎」(930ms)がプローブ語
    として呈示された場合の方が判断時間が速く,両者の間には有意差が見ら
    れた($F(1,17)=5.318, p<0.05$).しかし,(18b)では,これとは逆の傾向
    が見られ,「太郎」(770ms)よりも「健太」(684ms)がプローブ語として呈
    示された場合の方が判断時間が速く,両者の間には有意差が見られた($F
    (1,17)=16.878, p<0.01$).
\end{itemize}

\begin{table}
\begin{center}
\caption{実験2の結果}
\label{table:exam2ar}
\begin{tabular}{|rl|r|r|r|r|r|r|} \hline
 \multicolumn{2}{|c|}{}&
 \multicolumn{2}{c|}{読解時間} &
 \multicolumn{2}{c|}{Q\&A反応} &
 \multicolumn{2}{c|}{PRT反応(ms)} \\ \hline
 \multicolumn{2}{|c|}{実験文} & 
  太郎が/は & 照応表現 & 太郎 & 健太 & 太郎 & 健太 \\ \hline
 (18)a & 太郎が\ldots 彼は\ldots & 13.58 & 12.71 & 15 & 3 & 930 & 1032\\
     b & 太郎が\ldots 彼を\ldots & 12.71 & 12.83 & 17 & 1 & 770 &
 684 \\ \hline
\end{tabular}
\end{center}
\end{table}

  以上のことをまとめると,まず,Self-Paced Reading TaskにおけるQ\&A反応
  では,照応表現の先行詞を前節の主語とする反応が多いことから,SASを利
  用しているものと考えられる.

   しかし,(18b)については,実験課題による結果の相違が見られた.
  Self-Paced Reading TaskにおけるQ\&A反応では,「太郎」を先行詞とする反
  応が多いが,Probe Recognition Taskでは,「太郎」に比べて「健太」の方
  が判断時間が速く,照応表現が現れた時点では,被験者の脳内で「健太」が
  活性化され,「健太」を先行詞としていることが分かる.つまり,後者の実
  験課題では,PFSを支持する結果が得られた.

 こうした不一致は,次のような理由によるものと考えられる.Self-Paced
Reading TaskにおけるQ\&Aは一文を読み終えた後に行われるため,文末以降の
文処理を反映しており,意味処理や文脈処理が行われた結果が反映していると
考えることができる.したがって,Self-Paced Reading Taskの結果からは,
つねにSASが使用されると断定することはできず,提示される文の意味内容に
よってはPFSを支持する結果が得られる可能性もあるため,どちらのストラテ
ジーが用いられるかを決定するのに有効な実験課題ではなかった.一方,
Probe Recognition Taskは,照応表現が現れた時点で先行詞として何が活性化
されているかというリアルタイムでの処理を反映している.そうすると,照応
表現の位置では,(18b)のような目的語代名詞の場合には,前節の目的語が暫
定的に先行詞として割り当てられていると考えることができ,オンラインでは
PFSが用いられていることになる.

\section{発見的ストラテジー間の相互関係}
 実験1では,日本語の照応理解プロセスで「主題役割方略」が用いられるこ
とを,また実験2では,parallelな構造をもつ文では「平行機能方略」が用い
られることを示す結果が得られた.しかし,両者のストラテジーの関係は明ら
かになっていない.つまり,この2つのストラテジーが競合する場合,どちら
か一方が優先的に利用されるということがあるのだろうか.
 また,これまで得られた結果について,他のストラテジーが関与している可
能性はないのだろうか.

\subsection{その他の発見的ストラテジー}
 英語の照応理解では,前節で触れた「主語割当方略」,「平行機能方略」
以外にも,さまざまなストラテジーが提案されている.例えば次のような,
いわゆる知覚上のストラテジーがある.

\vspace{3mm}

\begin{flushleft}
\begin{tabular}{cl}
(19)& ファーストメンション効果(first mention effect; FM)\\
    & 文の最初に言及された人物を代名詞の指示対象として割り当てよ.
\end{tabular}
\end{flushleft}
\begin{flushright}
\vspace{-2mm}
~\cite{gernsbacher:1988}
\end{flushright}

\vspace{-2mm}
\begin{flushleft}
\begin{tabular}{cl}
(20)& 近接効果(recency effect; RE)\\
    & 代名詞に最も近い位置に現れた人物をその代名詞の指示対象として割
    り当てよ.\\
\end{tabular}
\end{flushleft}
\vspace{3mm}

 これまでの実験で,主題を表す「は」でマークされた語が代名詞の先行詞と
して解釈される場合は,(19)のストラテジーが利用されて,最初に言及した人
物を割り当てているとも考えられる.また,parallelな構造をもつ文で目的語
代名詞の場合には,先行する節の目的語が代名詞の先行詞として解釈される現
象は,(20)のストラテジーが利用されている可能性もある.
 日本語の照応理解においても以上のようなストラテジーが働く可能性がある
ので,これらの知覚上のストラテジーとの関係について検討する必要がある.

\subsection{予備実験}
 複数のストラテジーが競合する場合,利用されるストラテジーに優先度が
あるのかどうかについて詳細に調査する前に予備実験を行った.

\subsubsection{実験刺激文}
 実験2で取り上げた刺激文(18)は,第1節には主語を示す「が」を用いた文
であったが,それを主題を表す「は」に置き換えた刺激文(19)を用いて比較実
験を行った.予備実験で用いた例文を表\ref{table:exam2bs}に示す.なお,参
照の煩雑さを避けるため,刺激文(18)を再度掲載する.

\vspace{-3mm}
\begin{table}[htbp]
\begin{center}
\caption{実験2で用いる刺激文}
\label{table:exam2bs}
\vspace{-2mm}
\begin{tabular}{cl}\hline
(18) &「が」:主語代名詞\\
     & a. 太郎が  健太を 殴り,そして 彼は ▲ 次郎を 殴った.\\
     &「が」:目的語代名詞\\
     & b. 太郎が  健太を 殴り,そして 次郎は 彼を ▲ 殴った.\\
(19) &「は」:主語代名詞\\
     & a. 太郎は  健太を 殴り,そして 彼は ▲ 次郎を 殴った.\\
     &「が」:目的語代名詞\\
     & b. 太郎は  健太を 殴り,そして 次郎は 彼を ▲ 殴った.\\ \hline
\end{tabular}
\end{center}
\end{table}
\vspace{-3mm}

\subsubsection{結果の予測}
刺激文(18)では,PRT反応時間を測定した結果,主語代名詞の場合には第1節
の主語を,目的語代名詞の場合には第1節の目的語を提示した方が反応時間が
速かったことから,オンラインではPFSが利用されていることが分かった.

 これに対して,刺激文(19)では次のような予測が成り立つ.主語代名詞の
(19a)では,PFSが用いられてもTASが用いられても,プローブ語として提示さ
れる「健太」に比べて「太郎」の反応時間が速いことが予測される.しかし,
目的語代名詞の(19b)では,「健太」の反応時間が速ければPFSが用いられたこ
とになり,「太郎」の反応時間が速ければTASが用いられたことになる.


\subsubsection{結果と考察}
 実験結果を表~\ref{table:exam2br}に示す.主題を示す「は」を用いた刺激文
(19a)では,「健太」(1192ms)よりも「太郎」(747ms)がプローブ語として
呈示された場合の方が判断時間が速く,両者の間には有意差が見られた
($F(1,17)=18.635,p<0.01$).また,(19b)でも同様の傾向が見られ,(18b)
とは逆に,「健太」(1062ms)よりも「太郎」(888ms)がプローブ語として
呈示された場合の方が判断時間が速く,両者の間には有意差が見られた
($F(1,17)=6.231,p<0.05$).

 すなわち,(18a-b)と(19a-b)の間には利用されるストラテジーに違いが見ら
れたが,これはPFSとTASが競合する場合,TASが優先的に利用されたと考えら
れる.

 予備実験の結果,いくつかのストラテジーが競合する場合,いずれかのス
トラテジーが優先して利用される可能性があることが分かった.

\begin{table}
\begin{center}
\caption{実験2の結果}
\label{table:exam2br}
\begin{tabular}{|rl|r|r|r|r|r|r|} \hline
 \multicolumn{2}{|c|}{}&
 \multicolumn{2}{c|}{読解時間} &
 \multicolumn{2}{c|}{Q\&A反応} &
 \multicolumn{2}{c|}{PRT反応(ms)} \\ \hline
 \multicolumn{2}{|c|}{実験文} & 
  太郎が/は & 照応表現 & 太郎 & 健太 & 太郎 & 健太 \\ \hline
 (18)a&	太郎が\ldots 彼は\ldots & 13.58 &	12.71 & 15 & 3 & 930 & 1032\\
     b&	太郎が\ldots 彼を\ldots & 12.71 & 12.83 & 17 & 1 & 770 & 684 \\
 (19)a&	太郎は\ldots 彼は\ldots & 12.42 & 17.24 & 16 & 2	& 747 & 1192\\
     b&	太郎は\ldots 彼を\ldots & 12.82 & 15.00 & 18 & 1	& 888 & 1062\\  \hline
\end{tabular}
\end{center}
\end{table}

\subsection{実験3}
 実験3では,SASあるいはPFSとTASが競合する場合,どれが優先的に利用さ
れるかついて,人間の知覚上のストラテジーであるFM, REの影響とあわせて調
査する.

\subsubsection{実験刺激文}
 表\ref{table:exam3s}に実験3で用いられる刺激文を示した.前節と後節がパ
ラレルな構造をもつように実験文を作成した.(a)を基本形として,語順のか
き混ぜ(scrambling)によって文法項の位置を操作し,前節の目的語を前置した文
(b),さらに,前置した目的語を主題化した文(c)を作成した(実験文
(22),(23)).さらに,(22a-b),(22a-b)の「が」を「は」に置き換えて,
(24a-b), (25a-b)を作成した.なお,文中の▲はProbe Recognition Taskにお
けるプローブ語の現れる位置を表す.

\begin{table}[htbp]

\caption{実験3で用いる刺激文}
\label{table:exam3s}
\begin{center}
\begin{tabular}{cl} \hline
(22) &主語代名詞\\
     & a. 太郎が 健太を 殴った.そして 彼は ▲ 逃げていった.\\
     &主語代名詞:目的語前置\\
     & b. 健太を 太郎が 殴った.そして 彼は ▲ 逃げていった.\\
     &目的語前置:主題化\\
     & c. 健太は 太郎が 殴った.そして 彼は ▲ 逃げていった.\\
(23) &目的語代名詞\\
     & a. 太郎が 健太を 殴った.そして 次郎は 彼を ▲ 殴った.\\
     &目的語代名詞:目的語前置\\
     & b. 健太を 太郎が 殴った.そして 次郎は 彼を ▲ 殴った.\\
     &目的語前置:主題化\\
     & c. 健太は 太が 殴った.そして 次郎は 彼を ▲ 殴った.\\
(24) &主語代名詞\\
     & a. 太郎は 健太を 殴った.そして 彼は ▲ 逃げていった.\\
     &主語代名詞:目的語前置\\
     & b. 健太を 太郎は 殴った.そして 彼は ▲ 逃げていった.\\
(25) &目的語代名詞\\
     & a. 太郎は 健太を 殴った.そして 次郎は 彼を ▲ 殴った.\\
     &目的語代名詞:目的語前置\\
     & b. 健太を 太郎は 殴った.そして 次郎は 彼を ▲ 殴った.\\ \hline
\end{tabular}
\end{center}
\end{table}

\subsubsection{結果の予測}
 主語代名詞を含む(22)では,SASに従えばいずれの場合も代名詞の直後では
「健太」に比べて「太郎」の反応時間が速いことが予測される.しかし,TAS
に従えば(22b),(22c)では「太郎」に比べて「健太」の反応時間が速くなるこ
とが予測される.
 目的語代名詞を含む(23)では,SASに従えばいずれの場合も代名詞の直後で
は「健太」に比べて「太郎」の反応時間が速いことが予測されるが,PFSに従
えば,(21a-b)では「健太」の反応時間が速いことが予測される.しかし,
(22a)と(23b),(24a)と(25b)の間にそれぞれ反応時間の差が見られないはず
である.

\subsubsection{結果と考察}
  表\ref{table:exam3r}にSelf-Paced Reading Taskにおける読解時間・
  Q\&A反応結果,およびProbe Recognition Taskにおける判断時間
  (msec)(PRT反応)を示す.
\vspace{0.5cm}
\begin{itemize}
  \item Self-Paced Reading Task\\
 「が」を用いた文のうち,主語代名詞の刺激文(22)では,(22a-c)のいずれ
の場合にも代名詞には「太郎」を割り当てる傾向が見られた.目的語代名詞の
刺激文(23)でも,同様の傾向が見られ,(23a-c)のいずれの場合にも代名詞に
は「太郎」を割り当てる傾向が見られた.また,「は」に置き換えた文
(24),(25)でも同様の傾向が見られた.このことは,SASを支持する結果といえ
る.
  \item Probe Recognition Task\\
 しかし,Probe Recognition Taskの結果では,(22a)ではプローブ語として
「太郎」(940ms)が呈示された場合の方が「健太」(981ms)が呈示された場合よ
りも判断時間が速いが,(22b)ではプローブ語として「健太」(945ms)が呈示さ
れた場合の方が「太郎」(1063ms)が呈示された場合よりも判断時間が速く,
(22c)でも同様の傾向が見られた.
 目的語代名詞の(23)の場合,(23a)ではプローブ語として「健太」(1094ms)
が呈示された場合の方が「太郎」(1238ms)が呈示された場合よりも判断時間が
速く,(23b),(23c)でも同様の傾向が見られた.(23a)に見られた傾向から,
照応表現が現れた位置ではPFSが利用されている可能性がある.
\end{itemize}

\begin{table}
\vspace{-6mm}
\begin{center}
\caption{実験3の結果}
\label{table:exam3r}
\begin{tabular}{|rl|r|r|r|r|r|r|} \hline
 \multicolumn{2}{|c|}{}&
 \multicolumn{2}{c|}{読解時間} &
 \multicolumn{2}{c|}{Q\&A反応} &
 \multicolumn{2}{c|}{PRT反応(ms)} \\ \hline
 \multicolumn{2}{|c|}{実験文} & 
  太郎が/は & 照応表現 & 太郎 & 健太 & 太郎 & 健太 \\ \hline
(22)a & 太郎が健太を\ldots 彼は & 15.17 & 19.87 & 12	& 6 & 940 & 981\\
    b & 健太を太郎が\ldots 彼は & 12.08 & 24.24 & 11	& 7 & 1630 & 945\\
    c & 健太は太郎が\ldots 彼は & 12.62 & 16.48 & 11	& 7 & 1234 & 840\\
(23)a & 太郎が健太を\ldots 彼を & 17.04 & 24.51 & 14	& 4 & 1238 & 1094\\
    b & 健太を太郎が\ldots 彼を & 14.22 & 13.75 & 16	& 2 & 1448 & 1280\\
    c & 健太は太郎が\ldots 彼を & 18.55 & 16.65 & 18	& 0 & 1020 & 727\\
(24)a & 太郎は健太を\ldots 彼は & 15.34 & 15.78 & 18	& 0 & 990 & 1006\\
    b &	健太を太郎は\ldots 彼は & 14.05 & 15.30 & 15	& 3 & 1030 & 1417\\
(25)a & 太郎は健太を\ldots 彼を & 17.58 & 16.46 & 16	& 2 & 1013 & 1188\\
    b & 健太を太郎は\ldots 彼を & 15.57 & 15.10 & 18	& 0 & 904 & 1106\\ \hline
\end{tabular}
\end{center}


\end{table}

\vspace{0.5cm}

 TASの影響やRE,FMの影響などの関係について,さらに詳細に比較検討をおこ
なう.表9〜表14には,比較検討する文のそれぞれについて,PRT反応(ms)お
よびそれぞれの差が示されている.なお,PRT反応の下に示したストラテジー
は,当該の反応時間がもう一方に比べて速い場合に,そのストラテジーが機能
していると考えられることを示している.

\vspace{-0.3cm}

\begin{itemize}
  \item REとSASが競合する場合\\
 まず,REとSASのいずれのストラテジーが用いらるか検討する.REは,(22a)でも(24a)でもより近い位置にある「健太」の反応時間が速いことを」予測するが,SASは,いずれの場合も「太郎」の反応時間が速いことを予測する.
 表\ref{table:ptr}にその結果を示した.PRT反応時間(ms)を比較した結果,(22a)では「健
太」よりも「太郎」の判断時間が速く,REよりもSASが優位であることを示し
ている.(24a)でも同様の傾向が見られたが,「太郎は」は主語でもあり主題
でもあるので,REよりもSAS, TASが優位であることを示している.全体として
は,SASがREよりも優位なストラテジーであると言える.
\end{itemize}

\begin{table}
\begin{center}
\caption{Probe Recognition Taskにおける平均判断時間}
\label{table:ptr}
\begin{tabular}{|r|l|c|c|c|} \hline
    \multicolumn{2}{|c|}{} & \multicolumn{2}{c|}{PRT反応(ms)} &   \\ \hline
      & 実験文 & 太郎 & 健太 & 差 \\ \hline
(22)a & 太郎が健太を\ldots 彼は & 940 &  981	& +41 \\
      &                    & SAS & RE   & \\ \hline
(24)a & 太郎は健太を\ldots 彼は & 990 & 1006	& +16 \\
      &                    & SAS,TAS & RE & \\ \hline
\end{tabular}
\end{center}
\end{table}

\begin{itemize}
  \item FMとTASが競合する場合\\
 次にFMとTASが競合する場合について検討しよう.「健太を」を文頭に移動
した文については,(22b)では「太郎」よりも「健太」の判断時間が速く,SAS
よりもFM効果が優位であることを示している(表\ref{table:fmtas}).同様の語順
で「太郎は」に置き換えた(22b)では,逆に「健太」よりも「太郎」の判断時
間が速くなったことから,FMよりもSAS, TASが優位であると考えられる.これ
らの間にはそれぞれ有意差が見られた((22b): $F(1,17)=209.503,
p<0.01;(22b):F(1,17)=35.338,p<0.01$).(22b)と(22b)を比較すると,「太
郎」と「健太」の判断時間に逆転現象が見られたことから,TASがFMよりも優
位なストラテジーであると言える.(22b)と(22b)の間には「太郎」および「健
太」の判断時間に有意差が見られた(「太郎」: $F(1,17)=108.532, p<0.01;$ 
「健太」:$F(1,17)=70.503,p<0.01$).
\end{itemize}

\begin{table}
\vspace{-5mm}
\begin{center}
\caption{Probe Recognition Taskにおける平均判断時間}
\label{table:fmtas}
\begin{tabular}{|r|l|c|c|c|} \hline
    \multicolumn{2}{|c|}{} & \multicolumn{2}{c|}{PRT反応(ms)} &   \\ \hline
      & 実験文 & 太郎 & 健太 & 差 \\ \hline
(22)b & 健太を太郎が\ldots 彼は & 1630 & 945& -685 \\
      &                    & SAS & FM & \\ \hline
(22)b & 健太を太郎は\ldots 彼は & 1030 & 1417 & +38 \\
      &                    & SAS,TAS & FM & \\ \hline
  差  &                    & -600 & +472 & +1072 \\ \hline
\end{tabular}
\end{center}
\end{table}

\begin{itemize}
  \item PFSとTASが競合する場合\\
 次に目的語が代名詞となっている場合の「が」と「は」を比較する(表~\ref{table:pfstas}).
「が」を用いた(23a)では,「太郎」に比べて「健太」の方が判断時間が速く,
SASよりもPFSが用いられている可能性がある.両者の間には有意差が見られた
($F(1,17)=6.801, p<0.01$).しかし,同様の語順で「太郎は」に置き換え
た(25a)では,逆に「健太」よりも「太郎」の方が判断時間が速くなったこと
から,PFSよりもTASが優位であると考えられる.両者の間には有意差が見られ
た($F(1,17)=5.389, p<0.01$).
\end{itemize}

\begin{table}
\begin{center}
\caption{Probe Recognition Taskにおける平均判断時間}
\label{table:pfstas}
\begin{tabular}{|r|l|c|c|c|} \hline
    \multicolumn{2}{|c|}{} & \multicolumn{2}{c|}{PRT反応(ms)} &   \\ \hline
      & 実験文 & 太郎 & 健太 & 差 \\ \hline
(23)a & 太郎が健太を\ldots 彼を & 1238 &1094 & -168 \\
      &                         & SAS & PFS & \\ \hline
(23)a & 太郎は健太を\ldots 彼を & 1013 & 1188 & +175 \\
      &                    & SAS,TAS & PFS & \\ \hline
  差  &                    & -225 & +94 & +319\\ \hline
\end{tabular}
\end{center}
\end{table}

\begin{itemize}
  \item PFS・SASとTASが競合する場合\\
 (23b)では,「太郎」よりも「健太」の判断時間が速く,SASよりもPFSが優
位である可能性がある(表\ref{table:pfssastas}).ただし,「健太」は文法機能
(目的語)においては「彼」とparallelであるが,語順はparallelではないの
で,純粋なPFSが働いているかどうかは検討の余地がある.「太郎は」に置き
換えた(23b)では,「健太」よりも「太郎」の判断時間が速くなったことから,
PFSよりもSAS, TASが優位であると考えられる.(23b)では,「太郎」と「健太」
の判断時間の間に有意差が見られた($F(1,17)=11.611, p<0.01$).(23b)に
比べて(25b)では,プローブ語が「太郎」と「健太」のいずれの場合にも判断
時間が速くなる傾向が見られたことから,TASが影響していると言える.(25b)
と(25b)の間では,「太郎」の判断時間に有意差が見られた
($F(1,17)=34.414,p<0.01$). 
\end{itemize}

\begin{table}
\begin{center}
\caption{Probe Recognition Taskにおける平均判断時間}
\label{table:pfssastas}
\begin{tabular}{|r|l|c|c|c|} \hline
    \multicolumn{2}{|c|}{} & \multicolumn{2}{c|}{PRT反応(ms)} &   \\ \hline
      & 実験文 & 太郎 & 健太 & 差 \\ \hline
(23)a & 健太を太郎が\ldots 彼を & 1448 & 1280 & -168 \\
      &                    & SAS & PFS? & \\ \hline
(25)a & 健太を太郎は\ldots 彼を & 904 & 1106 & +202\\
      &                    & SAS,TAS & PFS? & \\ \hline
  差  &                    & -544 & -174 & +370\\ \hline
\end{tabular}
\end{center}
\end{table}

\begin{itemize}
  \item FMとTASが競合する場合\\
 目的語「健太を」を文頭に移動した(22b)は,上述した通り,SAS効果よりも
FMが優位であることを示す結果が得られた.さらに主題化して「健太は」とし
た(22c)でも同様の傾向が見られたことから,SASよりもFM, TASが優位である
と考えられる.(22c)では,「太郎」と「健太」の判断時間の間に有意差が見
られた($F(1,17)=12.191, p<0.01$).プローブ語が「健太」の場合,(22b)
よりも(22c)の方で判断時間が短くなったことから,TASがその差を縮めたと言
える(表~\ref{table:fmtas2}).
\end{itemize}

\begin{table}
\begin{center}
\caption{Probe Recognition Taskにおける平均判断時間}
\label{table:fmtas2}
\begin{tabular}{|r|l|c|c|c|} \hline
    \multicolumn{2}{|c|}{} & \multicolumn{2}{c|}{PRT反応(ms)} &   \\ \hline
      & 実験文 & 太郎 & 健太 & 差 \\ \hline
(22)b & 健太を太郎が\ldots 彼は  & 1630 & 945 &-685 \\
      &                    & SAS & FM & \\ \hline
(22)c & 健太は太郎が\ldots 彼は & 1234 & 840 & -396\\
      &                    & SAS & FM,TAS & \\ \hline
  差  &                    & -396 & -105 & +289\\ \hline
\end{tabular}
\end{center}
\end{table}

\begin{itemize}
  \item FMとPFSとTASが競合する場合\\
  最後に,(23b)では,上述した通り,SASよりもPFSが優位である可能性を示
 す結果が得られた.(23c)でも同様の傾向が見られ,このことは,SASよりも
 FM, PFS,TASが優位であることを示している(表\ref{table:fmpfstas}).(23c)では
 「太郎」と「健太」の判断時間の間に有意差が見られた($F(1,17)=9.592,
 p<0.01$).
 プローブ語が「健太」の場合,(23b)よりも(23c)の方で判断時間が短くなっ
 たことから,TASがその差を縮めたと言える.これらの間には有意差が見られ
 た($F(1,17)=41.179, p<0.01$)
\end{itemize}

\begin{table}
\begin{center}
\caption{Probe Recognition Taskにおける平均判断時間}
\label{table:fmpfstas}
\begin{tabular}{|r|l|c|c|c|} \hline
    \multicolumn{2}{|c|}{} & \multicolumn{2}{c|}{PRT反応(ms)} &   \\ \hline
      & 実験文 & 太郎 & 健太 & 差 \\ \hline
(23)b & 健太を太郎が\ldots 彼を & 1448 & 1280 & -168 \\
      &                    & SAS & FM,PFS & \\ \hline
(25)c & 太郎は健太が\ldots 彼を & 1020 & 727 & -293\\
      &                    & SAS & FM,TAS & \\ \hline
  差  &                    & -428 & -553 & -125\\ \hline
\end{tabular}
\end{center}
\end{table}


 以上の結果から,これらの結果から,SAS,PFSなどのストラテジーとTASが競合する場合,TASが優先的に利用されると言える.また,知覚上のストラテジーであるREやFMなどの効果よりもTASが優位なストラテジーであると言える.

\section{まとめと今後の課題}

 本稿では,日本語の照応理解のプロセスにおいて,いくつかの発見的ストラテジーが関与していることを見てきた.日本語の実験では,parallelな構造を持つ文では,PFSが利用されることが分かった.
 また,英語のような「主語卓立言語」とは異なり,日本語では,SAS
あるいは
PFSとTASが競合する場合,むしろTASが利用されることが明らかになった.つまり,このことは,TASが他の発見的ストラテジーよりもより優位な立場にあるストラテジーであることを示唆するものである.これは,日本語が「主題卓立言語」の性質をもっていることを示している.
 今後は,文脈情報との相互作用など照応関係の理解に影響を及ぼすと思われ
る要因を考慮し,TASの優位性をさらに調査したい.また,統語的制約,意味
的制約と発見的ストラテジーとの相互関係についても考察したい.

\acknowledgment

 本稿の執筆にあたり,多くの貴重なご助言を賜った本稿査読者および,大
阪大学の郡司隆男教授(現在,神戸松蔭女子学院大学),三藤博助教授に深謝
いたします.また,有益なコメントを下さった大阪大学大学院言語文化研究
科の言語工学研究会の諸氏にも感謝いたします.なお,本稿に残る誤植や間
違いはすべて筆者の責任である.

\bibliographystyle{jnlpbbl}
\bibliography{v06n4_01}

\begin{biography}
\biotitle{略歴}
\bioauthor{横川博一}{
1994年京都教育大学大学院教育学研究科修士課程修了,1996年大阪大学大学
院言語文化研究科博士前期課程修了,1999年同大学院言語文化研究科博士後
期課程単位取得満期退学.現在,京都教育大学,京都外国語大学非常勤講師.
心理言語学,応用言語学の研究に従事.関西英語教育学会事務局長,全国英
語教育学会理事.言語処理学会,日本認知科学会,語学ラボラトリー学会,
大学英語教育学会,日本児童英語教育学会などの会員.
}

\bioreceived{受付}
\vspace{-2mm}
\biorevised{再受付}
\vspace{-2mm}
\bioaccepted{採録}

\end{biography}

\end{document}
