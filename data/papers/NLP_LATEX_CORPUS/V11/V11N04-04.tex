\documentstyle[epsbox,jnlpbbl]{jnlp_j}

\setcounter{page}{3}
\setcounter{巻数}{2}
\setcounter{号数}{3}
\setcounter{年}{1995}
\setcounter{月}{7}
\受付{1995}{5}{6}
\再受付{1995}{7}{8}
\採録{1995}{9}{10}

\newenvironment{LIST}{}{}

\newcounter{sentcounter}
\newenvironment{SENT2}{}{}

\setcounter{secnumdepth}{2}

\title{複合語の内部情報・外部情報を統合的に利用した訳語対の抽出}

\author{吉見 毅彦\affiref{Ryukoku}\affiref{NICT} \and
九津見 毅\affiref{SHARP} \and 
小谷 克則\affiref{NICT} \and 佐田 いち子\affiref{SHARP} \and 
井佐原 均\affiref{NICT}}

\headauthor{吉見,九津見,小谷,佐田,井佐原}
\headtitle{複合語の内部情報・外部情報を統合的に利用した訳語対の抽出}

\affilabel{Ryukoku}{龍谷大学}{Ryukoku University}
\affilabel{SHARP}{シャープ(株)}{SHARP Corporation}
\affilabel{NICT}{情報通信研究機構}{National Institute of Information and 
Communications Technology}

\jabstract{本稿では,機械翻訳システムの辞書に登録されておらず,かつ(対応
付け誤りを含む)対訳コーパスにおいて出現頻度が低い複合語を対象として,
その訳語を抽出する方法を提案する. 
提案方法は,複合語あるいはその訳語候補の内部から得られる情報と,複合語あ
るいはその訳語候補の外部から得られる情報とを統合的に利用して訳語対候補に
全体スコアを付ける.
全体スコアは,複合語あるいはその訳語候補の二種類の内部情報と
二種類の外部情報に基づく各スコアの加重和を計算することによって求めるが,
各スコアに対する重みを回帰分析によって決定する.
読売新聞とThe Daily Yomiuriの対訳コーパスを用いた実験では,全体スコアが
最も高い訳語対(のうちの一つ)が正解である割合が86.36\%,全体スコアの
上位二位までに正解が含まれる割合が95.08\%という結果が得られ,提案手法の
有効性が示された.}

\jkeywords{訳語対抽出,対訳コーパス,低出現頻度語,辞書未登録語,
回帰分析}

\etitle{Integrated Use of Internal and External Evidence \\
in the Alignment of Compound Words}

\eauthor{Takehiko Yoshimi\affiref{Ryukoku}\affiref{NICT} \and 
Takeshi Kutsumi\affiref{SHARP} \and 
Katsunori Kotani\affiref{NICT} \and 
Ichiko Sata\affiref{SHARP} \and
Hitoshi Isahara\affiref{NICT}} 

\eabstract{This paper proposes a method of extracting English compound 
words and their Japanese equivalents from a parallel corpus.
The aim of our research is to extract compound words which
are not listed in a dictionary of an English-to-Japanese MT
system and appear infrequently in a parallel corpus.
Our method makes its alignment on the basis of two kinds of external 
evidence provided by the context in which a bilingual pair appears, as 
well as two kinds of internal evidence within the pair.
Each kind of evidence is accompanied by a score, and the aggregate score 
is computed as a weighted sum of the scores.
The appropriate weights are estimated with the logistic 
regression analysis.
An experiment using a parallel corpus of Yomiuri Shimbun and The Daily 
Yomiuri satisfactorily found that 86.36\% of the extracted 
bilingual pairs with the highest scores and 95.08\% with the top two 
scores were judged to be correct.}

\ekeywords{Bilingual Lexicon Extraction, Parallel Corpus, Low Frequency 
Word, Unknown Word, Regression Analysis}

\begin{document}

\maketitle


\section{はじめに}
\label{sec:intro}

機械翻訳システムの辞書は質,量ともに拡充が進み,最近では200万見出し以上
の辞書を持つシステムも実用化されている.
ただし,このような大規模辞書にも登録されていない語が現実のテキストに出現
することも皆無ではない.
辞書がこのように大規模化していることから,辞書に登録されていない語は,コ
ーパスにおいても出現頻度が低い語である可能性が高い. 

ところで,文同士が対応付けられた対訳コーパスから訳語対を抽出する研究はこ
れまでに数多く行なわれ\cite{Eijk93,Kupiec93,Dekai94,Smadja96,Ker97,Le99},
抽出方法がほぼ確立されたかのように考えられている.
しかし,コーパスにおける出現頻度が低い語とその訳語の対を抽出することを目
的とした場合,語の出現頻度などの統計情報に基づく方法では抽出が困難である
ことが指摘されている\cite{Tsuji00}. 

以上のような状況を考えると,対訳コーパスからの訳語対抽出においては,
機械翻訳システムの辞書に登録されていない,出現頻度の低い語を対象とした
方法の開発が重要な課題の一つである.
しかしながら,現状では,低出現頻度語を対象とした方法の先行研究としては
文献\cite{Tsuji01b}などがあるが,
検討すべき余地は残されている.
すなわち,利用可能な言語情報のうちどのような情報に着目し,それらをどのよ
うに組み合わせて利用すれば低出現頻度語の抽出に有効に働くのかを明らかにす
る必要がある.

本研究では,実用化されている英日機械翻訳システムの辞書に登録されていない
と考えられ,かつ対訳コーパス
\footnote{本研究で用いたコーパスは,文対応の付いた対訳コーパスであるが,
機械処理により対応付けられたものであるため,対応付けの誤りが含まれている
可能性がある.}
において出現頻度が低い複合語とその訳語との対を抽出する方法を提案する.
提案方法は,複合語あるいはその訳語候補の内部の情報と,
複合語あるいはその訳語候補の外部の情報とを統合的に利用して訳語対候補に
スコアを付け,全体スコアが最も高いものから順に必要なだけ訳語対候補を出力
する.
全体スコアは,複合語あるいはその訳語候補の内部情報と外部情報に基づく
各スコアの加重和を計算することによって求めるが,各スコアに対する重みを
回帰分析によって決定する
\footnote{回帰分析を自然言語処理で利用した研究としては,重要文抽出への
適用例\cite{Watanabe96}などがある}.

本稿では,英日機械翻訳システムの辞書に登録されていないと考えられる複合語
とその訳語候補のうち,機械翻訳文コーパス(後述)における出現頻度,それに
対応する和文コーパスにおける出現頻度,訳文対における同時出現頻度がすべて
1であるものを対象として行なった訳語対抽出実験の結果に基づいて,
複合語あるいはその訳語候補の内部情報,外部情報に基づく各条件の有効性と,
加重和計算式における重みを回帰分析によって決定する方法の有効性を検証する.


\section{訳語対抽出処理の概要}
\label{sec:outline}

本稿で提案する方法による訳語対抽出処理の概要は次の通りである.
\begin{enumerate}
\item \label{enum:mt}
対訳コーパスのうち英文コーパスを機械翻訳システムで翻訳する.
翻訳には「翻訳これ一本2003」
\footnote{ http://www.sharp.co.jp/ej/}
を利用した.
\item \label{enum:sel_unknown}
翻訳結果に原語のまま現れた二単語以上の単語列のうち,大文字で始まる語のみ
から構成される単語列を対象とする.
小文字で始まる語を含む単語列を対象外とする理由は,予備調査の結果,小文字
始まり語を含む単語列が辞書に登録されていない原因がつづりの誤りであること
と,「kokumin nenkin」のような日本語のローマ字表記であることが多かったか
らである. 
大文字始まり語のみから構成される単語列(複合語)を含む文を抽出し,その集合
を機械翻訳文コーパスとする.
以下では,大文字始まり語のみから構成される辞書未登録の複合語を単に未登録
語と呼ぶ.
\item \label{enum:morph}
機械翻訳文コーパスとそれに対応する和文コーパスそれぞれに対して形態素解析
を行なう.
解析には「茶筌」
\footnote{http://chasen.aist-nara.ac.jp/chasen/}
を利用した.
\item \label{enum:sel_cand}
未登録語に対応する訳語候補を,各機械翻訳文に対応する和文から抽出する.
どのような語を訳語候補として抽出するかについては
\ref{sec:conditions:pos}\,節で述べる. 
\item \label{enum:sel_low_freq}
上記の処理(\ref{enum:sel_unknown})と(\ref{enum:sel_cand})で得られた
訳語対候補から,機械翻訳文コーパスにおける出現頻度,それに対応する和文コ
ーパスにおける出現頻度,訳文対における同時出現頻度がすべて1であるものを
抽出する.
\item \label{enum:sel_local}
各訳語対候補に対して次の各観点からスコアを付与する.
\begin{itemize}
\item 未登録語の構成単語の訳語を単純に合成した訳語と訳語候補との類似性
\item 未登録語のローマ字読みと訳語候補の読みとの類似性
\item 未登録語の近傍に現れる名詞の集合と訳語候補の近傍に現れる名詞の集合
との類似性
\item 訳語候補と同一の語が機械翻訳文にも存在するか否か
\end{itemize}
これらの詳細については,それぞれ
\ref{sec:conditions:element}\,節,\ref{sec:conditions:romaji}\,節,
\ref{sec:conditions:neighbor}\,節,\ref{sec:conditions:same_noun}\,節で
述べる.
\item \label{enum:sel_global}
処理(\ref{enum:sel_local})で付与された各スコアを統合して全体でのスコア
を決定し,全体スコアが最も高いものから順に必要なだけ訳語対候補を出力する.
スコアの統合方法については,\ref{sec:conditions:integration}\,節で述べる.
\end{enumerate}


\section{訳語対抽出に用いる制約条件と優先条件}
\label{sec:conditions}

本節では,\ref{sec:outline}\,節で概要を述べた訳語対抽出処理で用いる
言語情報(制約条件と優先条件)について説明する.


\subsection{品詞指定による訳語候補の絞込み}
\label{sec:conditions:pos}

辞書未登録の複合語は名詞であることが多い.このため,それに対する訳語候補
の品詞も名詞であるとする.
ただし,名詞すべてを訳語候補とするのではなく,「茶筌」の細分類品詞のうち,
原則として,名詞-一般,名詞-サ変接続,名詞-形容動詞語幹,名詞-副詞可能,
名詞-ナイ形容詞語幹,名詞-固有名詞を訳語候補とする.
また,「茶筌」で未知語とされた語も,原則として,名詞とみなして訳語候補と
する.

これらの原則に従わない主な例外は次の二つの場合である.
一つ目は,「する」か「できる」が名詞に後接しているとき,全体をサ変動詞と
する場合である.
二つ目は,半角の括弧のように記号とみなすべきものが「茶筌」の未知語になっ
ている場合である.

名詞あるいは未知語が連続している場合,全体を複合語とみなして一つの訳語候
補とする. 
なお,名詞あるいは未知語の連続には接辞が含まれていてもよい.
以下では,複合語を構成する単語の区切りを`/'で表わす. 


\subsection{未登録語の構成単語の単純合成訳と訳語候補の類似性による優先順位付け}
\label{sec:conditions:element}

未登録語の訳語すなわち複合語全体としての訳語は,複合語を構成する個々の
単語の訳語を単純に合成したものであるとは限らない.
しかし,複合語全体としての訳語と,複合語の構成単語の訳語を単純に合成し
た訳語との間にはある程度の類似性が見られることもある
\cite{Kumano94,Takao96}. 
そこで,未登録語の訳語候補と,未登録語の構成単語の訳語を単純に合成した
訳語との類似性に応じて未登録語と訳語候補の対にスコアを付けることを考える.
以下,未登録語の構成単語の訳語の単純な合成を単純合成訳と呼ぶ.

ここでは,訳語候補と単純合成訳の類似性を表わす尺度としてジャッカード係数
\cite{Romesburg92}を用いる.
ジャッカード係数は,この場合,未登録語の訳語候補と未登録語の単純合成訳
の両方に現れる単語の数を,少なくとも一方に現れる単語の数で割った値である
と定義できる.
すなわち,未登録語$E$の訳語候補$J$を構成する単語の集合を$X$とし,
未登録語$E$の単純合成訳を構成する単語の集合を$Y$としたとき,
未登録語$E$と訳語候補$J$の対に対する単純合成訳の類似性スコア$S_1$は次の
式(\ref{eq:jaccard})で求められる.
\begin{equation}
S_1(E,J) = \frac{|X \cap Y|}{|X \cup Y|}
\label{eq:jaccard}
\end{equation}
なお,本稿では複合語における構成単語の出現順序は考慮しない.

実験に用いた対訳コーパスでは,例えば「Disaster/Prevention/Law」と
いう未登録語の訳語候補として「災害/対策/基本/法」が挙げられる. 
他方,実験に用いた機械翻訳システムで「Disaster」,「Prevention」,
「Law」を個別に翻訳するとそれぞれ「災害」,「防止」,「法」という訳語が
得られる.
このとき,訳語候補を構成する単語の集合と,単純合成訳を構成する単語の集合
の積集合に属するものは「災害」と「法」の2語であり, 
和集合に属するものは「災害」,「対策」,「基本」,「法」,「防止」の5語
である.
従って,「Disaster/Prevention/Law」と「災害/対策/基本/法」が対訳である可
能性に対して,複合語全体としての訳語と単純合成訳の類似性の観点から
$S_1(\mbox{Disaster Prevention Law},災害対策基本法)=2/5$という
スコアが与えられる. 


\subsection{未登録語のローマ字読みと訳語候補の読みとの類似性による優先順位付け}
\label{sec:conditions:romaji}

\ref{sec:conditions:element}\,節で述べた条件は,複合語全体としては辞書に登
録されていないが,複合語を構成する個々の単語は辞書に登録されている場合に有
効である.
しかし,地名や人名,組織名などを表わす固有名詞は辞書に登録されていないこ
とも少なくなく,このような場合には有効に働かない.

今回の実験では読売新聞とThe Daily Yomiuriをコーパスとして用いたが,この
ように日本に関する事柄について述べた記事とその対訳記事を多く含むであろう
コーパスを処理対象とする場合には地名や人名,組織名も日本のものであること
が多い.
このような日本に関する固有表現には日本語をローマ字表記した単語が多く含ま
れる可能性が高い.
実際,The Daily Yomiuriコーパスのうち今回の実験対象とした文
を機械翻訳システムで翻訳して得られた訳文に含まれる未登録語から100語を
単純無作為抽出し,それらが例えば「Tsukuba/Circuit」のように日本語を
ローマ字表記した単語を含むかどうかを調べたところ,50語にローマ字表記語が
含まれていた.  

未登録語が日本語をローマ字表記したものである場合,五十音表を用意すれば,
比較的容易にかつある程度の精度でその読みを得ることができると考えられる.
また,未登録語の訳語候補の読みも「茶筌」で得ることができる.

これらの点に着目して,未登録語に対して得られるローマ字読みと訳語候補の
読みを照合し,スコアを付けることにした.
読みのスコアには,\ref{sec:conditions:element}\,節と同じく,ここでも
ジャッカード係数を用いる.
すなわち,未登録語を構成する単語のローマ字読みの集合を$X$とし,
訳語候補を構成する単語の読みの集合を$Y$としたとき,
読みの類似性スコア$S_2$を式(\ref{eq:jaccard})と同様の式で求める.

訳語候補については,それを構成する全ての単語の読みが得られるが,未登録語
については,ローマ字読みが得られる単語とそうでない単語がある. 
未登録語を構成する単語のローマ字読みが得られない場合,便宜的にその単語そ
のものを読みとする.
例えば未登録語「Tsukuba/Circuit」の場合,「ツクバ」と「Circuit」を読みと
する.
従って,「Tsukuba/Circuit」の読みの集合と訳語候補「筑波/サーキット」の読
みの集合の積集合に属するものは「ツクバ」となり,和集合に属するものは
「ツクバ」,「Circuit」,「サーキット」となるため,
未登録語「Tsukuba/Circuit」と訳語候補「筑波/サーキット」に対する
読みの類似性スコア$S_2(\mbox{Tsukuba Circuit},筑波サーキット)$は $1/3$
となる.

実験に用いたコーパスでは,未登録語「Tsukuba/Circuit」の訳語候補として,
「筑波/サーキット」の他に,「レース/用/バイク」,「パーツ/販売」が挙がっ 
てくるが,「レース/用/バイク」と「パーツ/販売」の場合は共に読みの類似性
スコアが0となるので,未登録語「Tsukuba Circuit」に対する訳語としては
「筑波/サーキット」が優先される.


\subsection{未登録語の近傍名詞集合と訳語候補の近傍名詞集合との類似性による優先順位付け}
\label{sec:conditions:neighbor}

訳語候補が未登録語の正しい訳語である場合,和文において訳語候補の前後に現
れる語の集合と,機械翻訳文において未登録語の前後に現れる語の集合とは
類似性が高いと考えられる
\cite{Fung98,Kaji01}.
そこで,未登録語の近傍に現れる名詞の集合と,訳語候補の近傍に現れる名詞の
集合との類似性を訳語としての確からしさとして考慮する. 

ある語の近傍に現れる語を近傍名詞と呼び,近傍名詞になる可能性がある
名詞を近傍名詞候補と呼ぶ
\footnote{近傍名詞候補のうちどれを近傍名詞にするかは,どのくらいの距離を
近傍とみなすかによる.}.
近傍名詞候補には,未登録語の訳語候補(\ref{sec:conditions:pos}\,節参照)
の他に,「茶筌」品詞の名詞-数も含める.
これは,例えば「九十二/年」のような数表現は,訳語候補としては適切ではない
が,未登録語の訳語候補の中から正しいものを選び出すのには有効な情報を提供
すると考えられるからである. 

ある語の近傍の範囲は,その語が現れる文に含まれる近傍名詞候補の総数に比
例するものとする.
今回の実験では,未登録語が現れる機械翻訳文に含まれる近傍名詞候補の総数を
$N$としたとき,未登録語の前方の近傍名詞候補を最大$N/4$語まで,後方の近傍
名詞候補を最大$N/4$語まで近傍名詞として集合に加えた.
なお,近傍名詞集合において,近傍名詞の出現位置が未登録語の前方か後方かは
区別しない.
また,近傍名詞候補数の上限値の小数点以下は切り捨てる.
訳語候補についての近傍名詞集合についても和文において同様に求める.
複合語の語数の計測では,個々の単語に分解せず,複合語全体で一語と数える.

未登録語の近傍名詞集合と訳語候補の近傍名詞集合との類似性スコアも
ジャッカード係数で表わす.
すなわち,未登録語$E$の近傍名詞集合を$X$とし,訳語候補$J$の近傍名詞集合
を$Y$としたとき,近傍名詞集合の類似性スコア$S_3$を式(\ref{eq:jaccard})と
同様の式で求める.

例えば,次の和文(H\ref{SENT:maastrich})と機械翻訳文(M\ref{SENT:maastrich})
から成る組では,未登録語「Maastrich/Treaty」の訳語候補は,
「マーストリヒト/条約」と「弾み」である.
\begin{SENT2}
\sentE With France, Germany provided a powerful new impetus to European 
integration, terminating in the Maastrich Treaty of 1992.
\sentH フランスとともにドイツは,九二年のマーストリヒト条約として結実す
る欧州統合に新たな,そして強力な弾みを与えた.
\sentM フランスに関して,1992年のMaastrich Treatyで終了して,ドイツは,
欧州統合に強力な新しい刺激をした. 
\label{SENT:maastrich}
\end{SENT2}

機械翻訳文(M\ref{SENT:maastrich})には7語の近傍名詞候補が現れる
\footnote{機械翻訳文(M\ref{SENT:maastrich})に現れる近傍名詞候補は,
「フランス」,「1992/年」,「Maastrich/Treaty」,「ドイツ」,「欧州統合」,
「強力」,「刺激」である.} 
ので,「Maastrich/Treaty」の前後それぞれ1語ずつ「1992/年」と「ドイツ」が
「Maastrich/Treaty」の近傍名詞となる. 
他方,和文(H\ref{SENT:maastrich})には8語の近傍名詞候補が現れる
\footnote{和文(H\ref{SENT:maastrich})に現れる近傍名詞候補は,
「フランス」,「ドイツ」,「九二/年」,「マーストリヒト/条約」,
「欧州統合」,「新た」,「強力」,「弾み」である.} 
ので,「マーストリヒト/条約」の前方の2語「ドイツ」と「九二/年」,および
後方の2語「欧州統合」と「新た」が「マーストリヒト/条約」の近傍名詞となる. 
訳語候補「弾み」の場合は,後方に近傍名詞候補が存在しないので,前方の2語
「新た」と「強力」だけが近傍名詞となる.
表\ref{tab:neighbor}\,に近傍名詞をまとめて示す.
\begin{table}[htbp]
\caption{近傍名詞集合の例}
\label{tab:neighbor}
\begin{center}
\begin{tabular}{|l||l|}\hline
\multicolumn{1}{|c||}{未登録語,訳語候補} & 
\multicolumn{1}{c|}{近傍名詞集合} \\\hline\hline
Maastrich/Treaty    &	1992/年,ドイツ \\
マーストリヒト/条約 &	ドイツ,九二/年,欧州統合,新た \\
弾み		    &	強力,新た \\\hline
\end{tabular}
\end{center}
\end{table}

未登録語「Maastrich/Treaty」の近傍名詞集合と訳語候補
「マーストリヒト/条約」の近傍名詞集合との積集合に属するものは「ドイツ」
の1語となり,和集合に属するものは,「1992/年」,「ドイツ」,「九二/年」,
「欧州統合」,「新た」の5語となるので,
$S_3(\mbox{Maastrich Treaty},マーストリヒト条約)=1/5$という
近傍名詞集合の類似性スコアが与えられる.
他方,「Maastrich/Treaty」と訳語候補「弾み」の場合は,両者の近傍名詞集合
の積集合に属する単語は存在しないので,
近傍名詞集合の類似性スコア$S_3(\mbox{Maastrich Treaty},弾み)$は0となる.
従って,近傍名詞集合のスコアの観点からは,未登録語「Maastrich/Treaty」の
訳語として「マーストリヒト/条約」が「弾み」よりも優先される. 


\subsection{訳語候補と同一語の存在/非存在による優先順位付け}
\label{sec:conditions:same_noun}

和文に現れている訳語候補が機械翻訳文にも現れている場合,それらは対応関係
にある可能性が高く,訳語候補と未登録語が対応関係にある可能性は低いのでは
ないかと考えられる
\footnote{同様の考え方が文献\cite{Ishimoto94}に示されている.}.
例えば,次の和文(H\ref{SENT:foodstuff})と機械翻訳文(M\ref{SENT:foodstuff})
から成る組では,未登録語「Foodstuff/Sanitation/Law」の訳語候補として,
「食品/衛生/法」と「施行/規則」が挙げられる.
このうち後者は,機械翻訳文(M\ref{SENT:foodstuff})に「規則」という単語が
存在するため,「Foodstuff/Sanitation/Law」よりも「規則」に対応する可能性
が高いと考えるのが自然であろう.
\begin{SENT2}
\sentE First, they plan to coordinate views with their counterparts at 
the Agriculture, Forestry and Fisheries Ministry and then revise 
regulations of the Foodstuff Sanitation Law by as early as this fall. 
\sentH 同省は、農水省と調整し、この秋にも食品衛生法の施行規則などを改正
し、来年四月一日の施行を目指す。 
\sentM 最初に、それらは、農林水産省でそれらの相対物を持つビューを統合し、
その後、Foodstuff Sanitation Lawの規則をこの秋と同じくらい早く改正するつ
もりである。 
\label{SENT:foodstuff}
\end{SENT2}

このため,機械翻訳文に同じ語が現れる訳語候補と未登録語の対には,
機械翻訳文に同じ語が現れない訳語候補と未登録語の対に与えるスコアよりも
低いスコアを与える.
なお,訳語候補と機械翻訳文に現れる語との照合は,複合語単位ではなく単語単
位で行なう.
すなわち,訳語候補と機械翻訳文に現れる語の両方あるいは一方が複合語である
場合,それらを構成する単語のうち少なくとも一つが一致すれば,両者は対応関
係にあるとみなす.
実験では,訳語候補を構成する単語と同じ単語が機械翻訳文に現れる場合,
同一語の存在/非存在に関するスコア$S_4$を0とし,現れない場合0.5とした.
\[
S_4 = \left\{
\begin{array}{lp{0.65\columnwidth}}
0   & 訳語候補の構成単語と同じ単語が機械翻訳文に現れる場合 \\
0.5 & 現れない場合
\end{array}
\right.
\]


\subsection{総合的評価}
\label{sec:conditions:integration}

提案方法では,
次の式(\ref{eq:weight})のような総合評価式に基づいて,
未登録語と訳語候補の内部情報
(未登録語の構成単語の単純合成訳と訳語候補との類似性,
未登録語のローマ字読みと訳語候補の読みとの類似性)と
未登録語と訳語候補の外部情報
(未登録語の近傍名詞集合と訳語候補の近傍名詞集合との類似性,
訳語候補と同一語の存在/非存在)
を組み合わせた評価を行ない,全体スコア$S$が最も高い訳語対から順に出力す
る.
\begin{equation}
S = C + \displaystyle \sum_{i=1}^{4} W_i \times S_i
\label{eq:weight}
\end{equation}
$C$は定数であり,重み$W_i$は各観点からのスコア$S_i$の相対的重要度を表わす.

訳語対内外の情報を併用するという考え方は,文献\cite{Kaji01}で示唆
されているが,提案の段階に留まっており実験結果などは示されていない.

$C$や$W_i$の決定方法として,直感的にあるいは予備実験を通じて経験的に決定
する方法(以下,経験的な決定法と呼ぶ)と,回帰分析によって決定する方法が
考えられる. 
本稿では,両者の場合について訳語対抽出の正解率を比較する.


\section{実験と考察}
\label{sec:experiment}


\subsection{実験方法}
\label{sec:experiment:method}

実験には,内山ら\cite{Uchiyama03}によって文対応付けが行なわれた読売新聞
とThe Daily Yomiuriの対訳コーパスのうち1989年から1996年7月中旬までの記事
で構成される部分を用いた.
さらに,内山らの文対応スコアの上位10\%の訳文対に対象を限定した.
このコーパスに対して訳語対抽出処理を行ない,各未登録語ごとに全体スコアが
高いものから順に訳語候補を出力した.

得られた訳語対データから標本抽出を行ない,標本中の各未登録語とその訳語候
補の対に対して次のような「正解」か「不正解」の評価値を与えた.
この評価値は,抽出された訳語対を辞書に登録する際の作業量の観点に立ったも
のである.
\begin{LIST}
\item[\bf 正解]
訳語の追加や削除,置換を行なわなくても,そのまま辞書に登録できる.\\
例:Comprehensive/Security/Board $\Longleftrightarrow$ 総合/安全/保障/審議/会
\item[\bf 不正解]
辞書に登録するためには,訳語の追加や削除,置換が必要である.\\
例:Liquor/Tax/Law $\Longleftrightarrow$ 逆手
\end{LIST}
なお,上記の評価を行なう際,次のような場合は対象外とした.
この措置は,訳語対抽出に用いた各条件の有効性と統合方法の有効性の検証に重
点を置きたいことなどによる.
\begin{itemize}
\item
未登録語の抽出が不適切である(未登録語が一つの名詞句を構成していない)場合.
原因は\ref{sec:outline}\,節で述べた処理(\ref{enum:sel_unknown})が失敗
したことにある.\\
例:Kita/Ward/Tuesday $\Longleftrightarrow$ 大阪/市/北/区
\item
正解の一部分しか訳語候補になっていない場合.
処理(\ref{enum:sel_cand})の失敗によるものである.
なお,この場合は,訳語の追加を行なえば辞書に登録でき
る.例えば次の例では「法」を追加すればよい.
例:Administrative/Procedures/Law $\Longleftrightarrow$ 行政/手続
\item
訳語候補に正解が含まれていない場合.
これは,機械翻訳文コーパスにおける出現頻度,それに対応する和文コーパスに
おける出現頻度,訳文対における同時出現頻度がすべて1であるものを対象とし
ているため,
処理(\ref{enum:sel_low_freq})で訳語候補から除外されることによる.
また,元々,未登録語を含む機械翻訳文に対応する和文に正解が含まれていない
ことにもよる.
\item
未登録語に対する訳語候補が一つしかない場合.
訳語候補が正解であっても対象外とした.
\end{itemize}

総合評価式(\ref{eq:weight})における定数$C$と重み$W_i$としては,
それぞれ表\ref{tab:weight}\,に示す値を用いた.
\begin{table}[htbp]
\caption{重みの値}
\label{tab:weight}
\begin{center}
\begin{tabular}{|l||r|r|}\hline
& \multicolumn{1}{c|}{経験的決定法} & \multicolumn{1}{c|}{回帰分析} \\\hline\hline
定数$C$           & 0 & -4.58 \\
単純合成訳$W_1$   & 2 & 20.75 \\
ローマ字読み$W_2$ & 3 & 15.04 \\
近傍名詞集合$W_3$ & 1 &  3.58 \\
同一語$W_4$       & 2 &  2.81 \\\hline
\end{tabular}
\end{center}
\end{table}

経験的な決定法による値は,予備実験で得られた訳語対データから未登録語を
100語無作為抽出し,各未登録語とその全訳語候補との対に与えられた各条件に
よるスコアを観察した経験から,訳語対抽出に用いる各条件の信頼性が次の順で
高くなっていくと判断したことによるものである.
\begin{enumerate}
\item 未登録語の近傍名詞集合と訳語候補の近傍名詞集合との類似性($S_3$)
\item 未登録語の構成単語の単純合成訳と訳語候補との類似性($S_1$)と,
訳語候補と同一語の存在/非存在($S_4$)
\item 未登録語のローマ字読みと訳語候補の読みとの類似性($S_2$)
\end{enumerate}

回帰分析による方法で決定した値は,個々の条件に基づくスコアを説明変数と
し,正解か不正解か(1か0か)を目的変数としてロジスティック回帰分析を行なっ
て求めたものである.
訓練データの規模は1734件であり,このうち正解が148件,不正解が1586件であ
る.

表\ref{tab:weight}\,で
経験的な決定法による重みの値と回帰分析による方法で決定した重みの値
を比較すると,次のような違いがある.
\begin{itemize}
\item 経験的な決定法の場合,
未登録語と訳語候補の内部情報
に基づくスコアに対する重みと
未登録語と訳語候補の外部情報
に基づくスコアに対する重みの間の差は小さいが,回帰分析による方法の場合は
両者の差が比較的大きい. 
\item 経験的な決定法では各条件の信頼性が
$S_3$$< $$S_1$$=$$S_4$$<$$S_2$のように高くなっていくと考えたが,
回帰分析による方法では
$S_4$$<$$S_3$$<$$S_2$$<$$S_1$の順で高くなっていくとみなされている.
\end{itemize}


\subsection{実験結果}
\label{sec:experiment:result}

抽出された標本は,264語の未登録語と1086語の訳語候補から成る.
すなわち,未登録語に対する訳語候補数は平均で4.11語(1086/264)であった.

経験的な決定法による重みを用いた場合と
回帰分析による方法で決定した重みを用いた場合のそれぞれの評価結果を
表\ref{tab:result}\,に示す.
表\ref{tab:result}\,を見ると,
単独一位での正解率,同点一位を含めた場合の正解率,上位二位まででの正解率
のいずれにおいても,回帰分析による方法のほうが経験的な決定法よりも高い
正解率が得られている.
\begin{table}[htbp]
\caption{訳語対抽出の正解率}
\label{tab:result}
\begin{center}
\begin{tabular}{|l||c|r|c|}\hline
         & 単独一位         & \multicolumn{1}{c|}{第一位(同点一位を含む)} & 上位二位 \\\hline\hline
経験的決定法   & 74.24\%(196/264) & 83.71\%(221/264)       & 92.80\%(245/264)\\\hline
回帰分析 & 77.65\%(205/264) & 86.36\%(228/264)       & 95.08\%(251/264)\\\hline
\end{tabular}
\end{center}
\end{table}

経験的な決定法と回帰分析による方法とで,正解が出力された順位がどのよう
に変化したかを表わす分布を表\ref{tab:rank_change}\,に示す. 
表\ref{tab:rank_change}\,によれば,経験的な決定法より回帰分析による
方法のほうが下がったものが5語である. 
逆に回帰分析による方法のほうが順位が上がったものは15語あり,
その内訳は,同点一位から単独一位へ上がったものが2語,
二位から単独一位へ上がったものが7語,三位以下から単独一位へ上がったものが5語,
三位以下から二位へ上がったものが1語である.
\begin{table}[htbp]
\caption{正解が出力された順位の変動}
\label{tab:rank_change}
\begin{center}
\begin{tabular}{|l||r|r|r|r|r|}\hline
\multicolumn{1}{|c||}{経験的決定法$\backslash$回帰分析}
& \multicolumn{1}{c|}{単独一位} & \multicolumn{1}{c|}{同点一位}
& \multicolumn{1}{c|}{二位} & \multicolumn{1}{c|}{三位以下} 
& \multicolumn{1}{c|}{合計} \\\hline\hline
単独一位 & 191 &  0 &  5 &  0 & 196 \\
同点一位 &   2 & 23 &  0 &  0 &  25 \\
二位     &   7 &  0 & 17 &  0 &  24 \\
三位以下 &   5 &  0 &  1 & 13 &  19 \\\hline
合計     & 205 & 23 & 23 & 13 & 264 \\\hline
\end{tabular}
\end{center}
\end{table}


\subsection{失敗原因の分析}
\label{sec:experiment:failure}

正解が出力された順位が第二位以下であったものについて,その原因ごとに分類
した結果を表\ref{tab:cause_of_failure}\,に示す.
表\ref{tab:cause_of_failure}\,を見ると,
未登録語のローマ字読みと訳語候補の読みとの類似性による優先順位付けの誤り
によるもの(ローマ字読み)と
訳語候補と同一語の存在/非存在による優先順位付けの誤りによるもの
(同一語)の件数が他の原因に比べて多い.
この二つの原因について分析する.
なお,複合的原因は複数の原因が絡んでいると考えられるものである.
\begin{table}[htbp]
\caption{失敗原因の分類}
\label{tab:cause_of_failure}
\begin{center}
\begin{tabular}{|l||r|r|}\hline
\multicolumn{1}{|c||}{原因} & \multicolumn{1}{c|}{経験的決定法} 
& \multicolumn{1}{c|}{回帰分析} \\\hline\hline
単純合成訳   &  2 &  6 \\
ローマ字読み & 12 & 11 \\
近傍名詞集合 &  7 &  7 \\
同一語       & 19 &  9 \\
複合的原因   &  3 &  3 \\\hline
合計         & 43 & 36 \\\hline
\end{tabular}
\end{center}
\end{table}

未登録語のローマ字読みと訳語候補の読みとの類似性による優先順位付けの誤り
によるものは,次のように細分できる.
\begin{itemize}
\item 
表記からローマ字読みを得る処理の不備によるもの.
実装した処理では,未登録語「Nihon/Shimbun/Kyokai」と正解「日本/新聞/協会」
の対応関係が認識できなかった.
この原因は,両唇音の直前の閉鎖音は両唇音に変わる音韻規則を考慮していなか
ったため,「shimbun」の読みを得ることができなかったことにある.
また,「キョウ」を「kyo」と表記する書記規則を考慮していなかったため,
「kyokai」の読みが「キョウカイ」ではなく「キョカイ」になってしまったこと
にも原因がある.
\item 
読みの曖昧さによるもの.
「Nihon/Shimbun/Kyokai」と「日本/新聞/協会」の対応関係が認識できなかった
もう一つの原因は,「日本」の読みが「ニホン」ではなく「ニッポン」になって
いたことにある.
この問題は,「茶筌」の設定が読みの第一候補だけを得るようになっていたため
に生じたものであり,全候補を得る設定にすれば解決可能である.
\item
「茶筌」辞書の未登録語によるもの.
「茶筌」の辞書に「熱川」という固有名詞が登録されていなかったために,
「Atagawa」と「熱/川」の対応関係が認識できなかった.
\end{itemize}

訳語候補と同一語の存在/非存在による優先順位付けの誤りによるものは,
機械翻訳文中の訳語候補(の構成要素)が和文にも現れているにもかかわらず,
その訳語候補が正解である場合である.
例えば,次の機械翻訳文(M\ref{SENT:price})に現れる未登録語
「Price/Control/Ordinance」の正解訳語である「物価/統制/令」は,
機械翻訳文(M\ref{SENT:price})に現れる「物価/上昇」と対応していると誤認識
されてしまう.
\begin{SENT2}
\sentE The Price Control Ordinance was imposed in March 1946 to curb 
price hikes and ease the supply and demand of products. 
\sentH 物価統制令は一九四六年三月、物価暴騰を抑え、物資需給の円滑化を
目的にポツダム命令として公布。 
\sentM Price Control Ordinanceは、物価上昇を抑制し、そして、製品の需要と
供給を緩和するために、1946年3月に課された。 
\label{SENT:price}
\end{SENT2}
このような誤りを防ぐには,「物価/上昇」が「物価/暴騰」に対応していること
を認識する必要がある.


\subsection{条件間の独立性}
\label{sec:experiment:corel}

複数の条件を組み合わせて訳語対抽出を行なう方法では,条件間の独立性が高い
ほうが望ましい.
そこで,各条件間の独立性を調べるために,各素性間でのスピアマンの順位相関
係数\cite{Siegel83}を求めた.
その結果を表\ref{tab:corel}\,に示す.
\begin{table}[htbp]
\caption{各条件間でのスピアマンの順位相関係数}
\label{tab:corel}
\begin{center}
\begin{tabular}{|l||r|r|r|r|}\hline
& \multicolumn{1}{c|}{単純合成訳} & \multicolumn{1}{c|}{ローマ字読み} 
& \multicolumn{1}{c|}{近傍名詞集合} & \multicolumn{1}{c|}{同一語}
\\\hline\hline  
単純合成訳   & \multicolumn{1}{c|}{$-$} & 0.057 & 0.048 & 0.039 \\ 
ローマ字読み & \multicolumn{1}{c|}{$-$} & \multicolumn{1}{c|}{$-$} & 0.099 & 0.133 \\
近傍名詞集合 & \multicolumn{1}{c|}{$-$} & \multicolumn{1}{c|}{$-$} & \multicolumn{1}{c|}{$-$} & -0.034 \\
同一語       & \multicolumn{1}{c|}{$-$} & \multicolumn{1}{c|}{$-$} & \multicolumn{1}{c|}{$-$} & \multicolumn{1}{c|}{$-$} \\\hline
\end{tabular}
\end{center}
\end{table}

表\ref{tab:corel}\,によれば,どの条件の間でも相関係数の値は小さく,独立
性が高いことが分かる.


\subsection{各条件の有効性}
\label{sec:experiment:effect}

本節では,各条件が正解率の向上にどの程度寄与しているかを調べる.
個々の条件を課さない場合の重みの値は,各条件を除いた状態で1734件の
訓練データに対してロジスティック回帰分析を行なうことによって求めた.
各条件を課さない場合の正解率を表\ref{tab:feats-effect}\,に示す.
括弧内の数字は正解数である.
\begin{table}[htbp]
\caption{各条件の有効性}
\label{tab:feats-effect}
\begin{center}
\begin{tabular}{|l||c|r|c|}\hline
                 & 単独一位     & \multicolumn{1}{c|}{第一位(同点一位を含む)} & 上位二位 \\\hline\hline
全条件           & 77.65\%(205) & 86.36\%(228) & 95.08\%(251) \\
単純合成訳なし   & 63.64\%(168) & 81.44\%(215) & 90.15\%(238) \\
ローマ字読みなし & 58.33\%(154) & 79.92\%(211) & 90.90\%(240) \\
近傍名詞集合なし & 74.24\%(196) & 91.67\%(242) & 95.45\%(252) \\
同一語なし       & 76.89\%(203) & 86.74\%(229) & 94.70\%(250) \\\hline
\end{tabular}
\end{center}
\end{table}

未登録語の構成単語の単純合成訳と訳語候補との類似性に関する条件を課さな
い場合と,未登録語のローマ字読みと訳語候補の読みとの類似性に関する条件を
課さない場合は,全ての条件を課した場合に比べて,
単独一位での正解率,同点一位を含めた場合の正解率,上位二位まででの正解率
のいずれもが低くなっている.
特に単独一位での正解率の低下が大きい.
従って,これら二つの条件は正解率の向上に寄与していると言える.

他方,未登録語の近傍名詞集合と訳語候補の近傍名詞集合との類似性に関する条
件を課さない場合と,訳語候補と同一語の存在/非存在に関する条件を課さない
場合は,全ての条件を課した場合に比べて,同点一位を含めた場合の正解率は高
くなっており,また,単独一位での正解率と上位二位まででの正解率も若干低く
なっている程度である.
従って,これら二つの条件は正解率の向上に寄与していないと言える.

正解率の向上に寄与している二つの条件は複合語あるいはその訳語候補の内部情
報に関するものであり,寄与していない二つの条件は外部情報に関するものであ
る.
訳語対抽出の正解率向上に有効に働く外部情報を探っていくことが今後の課題で
ある.


\section{おわりに}

本稿では,実用化されている機械翻訳システムの辞書に登録されておらず,かつ,
(対応付け誤りを含む)対訳コーパスにおいて出現頻度が低い複合語を対象として,
その訳語を抽出する方法を示した. 
提案方法では,複合語あるいはその訳語候補の内部の情報とそれらの外部の情報
を統合的に利用して訳語対候補に全体スコアを付ける.
全体スコアは四種類の情報に基づく各スコアの加重和を計算することによって
求めたが,各スコアに対する重みをロジスティック回帰分析によって決定する方
法を採った.
読売新聞とThe Daily Yomiuriの対訳コーパスを用いた実験では,加重和による
総合評価式において各スコアに対する重みをロジスティック回帰分析により決定
した場合,全体スコアが最も高い訳語対(のうちの一つ)が正解である割合が
86.36\%,上位二位までに正解が含まれる割合が95.08\%という結果が得られた.
この結果は,直感的にあるいは予備実験を通じて経験的に決定する方法による結
果を上回るものである. 

\bibliographystyle{jnlpbbl}
\begin{thebibliography}{}

\bibitem[\protect\BCAY{Dekai \BBA\ Xia}{Dekai \BBA\ Xia}{1994}]{Dekai94}
Dekai, W.\BBACOMMA\  \BBA\ Xia, X. \BBOP 1994\BBCP.
\newblock \BBOQ {Learning an English-Chinese Lexicon from a Parallel
  Corpus}\BBCQ\
\newblock In {\Bem Proceedings of the Annual Conference of Association for
  Machine Translation of America}, \BPGS\ 206--213.

\bibitem[\protect\BCAY{Eijk}{Eijk}{1993}]{Eijk93}
Eijk, P. \BBOP 1993\BBCP.
\newblock \BBOQ {Automating the Acquisition of Bilingual Terminology}\BBCQ\
\newblock In {\Bem Proceedings of the 6th Conference of the European Chapter of
  the Association for Computational Linguistics}, \BPGS\ 113--119.

\bibitem[\protect\BCAY{Fung}{Fung}{1998}]{Fung98}
Fung, P. \BBOP 1998\BBCP.
\newblock \BBOQ {Statistical View on Bilingual Lexicon Extraction: From
  Parallel Corpora to Non-parallel Corpora}\BBCQ\
\newblock {\Bem Lecture Notes in Artificial Intelligence}, {\Bbf 1529}, 1--17.

\bibitem[\protect\BCAY{石本浩之\JBA 長尾眞}{石本浩之\JBA
  長尾眞}{1994}]{Ishimoto94}
石本浩之\JBA  長尾眞 \BBOP 1994\BBCP.
\newblock \JBOQ
  対訳文章を利用した専門用語対訳辞書の自動作成---訳語対応における両立不可能性
を考慮した手法について---\JBCQ\
\newblock 研究報告{NL}102-11, 情報処理学会.

\bibitem[\protect\BCAY{梶博行\JBA 相薗敏子}{梶博行\JBA 相薗敏子}{2001}]{Kaji01}
梶博行\JBA  相薗敏子 \BBOP 2001\BBCP.
\newblock \JBOQ 共起語集合の類似度に基づく対訳コーパスからの対訳語抽出\JBCQ\
\newblock \Jem{情報処理学会論文誌}, {\Bbf 42}  (9), 2248--2258.

\bibitem[\protect\BCAY{Ker \BBA\ Chang}{Ker \BBA\ Chang}{1997}]{Ker97}
Ker, S.\BBACOMMA\  \BBA\ Chang, J. \BBOP 1997\BBCP.
\newblock \BBOQ {A Class-based Approach to Word Alignment}\BBCQ\
\newblock {\Bem Computational Linguistics}, {\Bbf 23}  (2), 312--343.

\bibitem[\protect\BCAY{熊野明\JBA 平川秀樹}{熊野明\JBA
  平川秀樹}{1994}]{Kumano94}
熊野明\JBA  平川秀樹 \BBOP 1994\BBCP.
\newblock \JBOQ 対訳文書からの機械翻訳専門用語辞書作成\JBCQ\
\newblock \Jem{情報処理学会論文誌}, {\Bbf 35}  (11), 2283--2290.

\bibitem[\protect\BCAY{Kupiec}{Kupiec}{1993}]{Kupiec93}
Kupiec, J. \BBOP 1993\BBCP.
\newblock \BBOQ {An Algorithm for Finding Noun Phrase Correspondences in
  Bilingual Corpora}\BBCQ\
\newblock In {\Bem Proceedings of the 31th Annual Meeting of the Association
  for Computational Linguistics}, \BPGS\ 17--22.

\bibitem[\protect\BCAY{Le\JBA Youbing\JBA Lin \BBA\ Yufang}{Le
  et~al.}{1999}]{Le99}
Le, S.\JBA Youbing, J.\JBA Lin, D.\JBA  \BBA\ Yufang, S. \BBOP 1999\BBCP.
\newblock \BBOQ {Word Alignment of English-Chinese Bilingual Corpus Based on
  Chunks}\BBCQ\
\newblock In {\Bem Proceedings of the Joint SIGDAT Conference on Empirical
  Methods in Natural Language Processing and Very Large Corpora}, \BPGS\
  110--116.

\bibitem[\protect\BCAY{Romesburg}{Romesburg}{1992}]{Romesburg92}
Romesburg, H.~C. \BBOP 1992\BBCP.
\newblock \Jem{実例クラスター分析}.
\newblock 内田老鶴圃, 東京.
\newblock 西田英郎,佐藤嗣二 訳.

\bibitem[\protect\BCAY{Siegel}{Siegel}{1983}]{Siegel83}
Siegel, S. \BBOP 1983\BBCP.
\newblock \Jem{ノンパラメトリック統計学---行動科学のために---}.
\newblock マグロウヒルブック, 東京.
\newblock 藤本煕 監訳.

\bibitem[\protect\BCAY{Smadja \BBA\ McKeown}{Smadja \BBA\
  McKeown}{1996}]{Smadja96}
Smadja, F.\BBACOMMA\  \BBA\ McKeown, K. \BBOP 1996\BBCP.
\newblock \BBOQ {Translating Collocations for Bilingual Lexicons: A Statistical
  Approach}\BBCQ\
\newblock {\Bem Computational Linguistics}, {\Bbf 22}  (1), 1--38.

\bibitem[\protect\BCAY{高尾哲康\JBA 富士秀\JBA 松井くにお}{高尾哲康\Jetal
  }{1996}]{Takao96}
高尾哲康\JBA 富士秀\JBA  松井くにお \BBOP 1996\BBCP.
\newblock \JBOQ 対訳テキストコーパスからの対訳語情報の自動抽出\JBCQ\
\newblock 研究報告{NL}115-8, 情報処理学会.

\bibitem[\protect\BCAY{辻慶太}{辻慶太}{2001}]{Tsuji01b}
辻慶太 \BBOP 2001\BBCP.
\newblock \JBOQ
  対訳コーパスからの低頻度訳語対の抽出:翻字・頻度情報の統合的利用\JBCQ\
\newblock \Jem{第49回日本図書館情報学会研究大会発表要綱}, \BPGS\ 59--62.

\bibitem[\protect\BCAY{辻慶太\JBA 芳鐘冬樹\JBA 影浦峡}{辻慶太\Jetal
  }{2000}]{Tsuji00}
辻慶太\JBA 芳鐘冬樹\JBA  影浦峡 \BBOP 2000\BBCP.
\newblock \JBOQ
  対訳コーパスにおける低頻度語の性質:訳語対自動抽出に向けた基礎研究\JBCQ\
\newblock 研究報告{NL}138-7, 情報処理学会.

\bibitem[\protect\BCAY{内山将夫\JBA 井佐原均}{内山将夫\JBA
  井佐原均}{2003}]{Uchiyama03}
内山将夫\JBA  井佐原均 \BBOP 2003\BBCP.
\newblock \JBOQ 日英新聞の記事および文を対応付けるための高信頼性尺度\JBCQ\
\newblock \Jem{自然言語処理}, {\Bbf 10}  (4), 201--220.

\bibitem[\protect\BCAY{Watanabe}{Watanabe}{1996}]{Watanabe96}
Watanabe, H. \BBOP 1996\BBCP.
\newblock \BBOQ {A Method for Abstracting Newspaper Articles by Using Surface
  Clues}\BBCQ\
\newblock In {\Bem Proceedings of the 16th International Conference on
  Computational Linguistics (COLING)}, \BPGS\ 974--979.

\end{thebibliography}


\begin{biography}
\biotitle{略歴}
\bioauthor{吉見毅彦}
{1987年電気通信大学大学院計算機科学専攻修士課程修了.
1999年神戸大学大学院自然科学研究科博士課程修了.
(財)計量計画研究所(非常勤),シャープ(株)を経て,
2003年より龍谷大学理工学部情報メディア学科勤務.
}
\bioauthor{九津見毅}
{1965年生まれ.
1990年,大阪大学大学院工学研究科修士課程修了(精密工学—計算機制御).
同年,シャープ株式会社に入社.
以来,英日機械翻訳システムの翻訳エンジンプログラムの開発に従事.}
\bioauthor{小谷克則}
{1974年生まれ.
2002年より情報通信研究機構受託研究員.
2004年,関西外国語大学より英語学博士取得.}
\bioauthor{佐田いち子}
{1984年北九州大学文学部英文学科卒業.
同年シャープ(株) に入社.
現在,同社情報通信事業本部情報商品開発センター技術企画室副参事.
1985年より機械翻訳システムの研究開発に従事.}
\bioauthor{井佐原均}
{1978年京都大学工学部卒業.
1980年同大学院修士課程修了.
博士(工学).
同年通商産業省電子技術総合研究所入所.
1995年郵政省通信総合研究所関西支所知的機能研究室室長.
2001年情報通信研究機構(旧:通信総合研究所)けいはんな情報通信融合研究セン
ター自然言語グループリーダー.
自然言語処理,機械翻訳の研究に従事.
言語処理学会,情報処理学会,人工知能学会,日本認知科学会,ACL,各会員.}

\bioreceived{受付}
\biorevised{再受付}
\bioaccepted{採録}
\end{biography}

\end{document}
