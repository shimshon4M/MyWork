\documentclass{nlp}

\usepackage{graphicx}
\usepackage{latexsym}
\usepackage{jnlpbbl}

\begin{document}

\setcounter{page}{1}
\setcounter{Volume}{9}
\setcounter{Number}{1}
\setcounter{Month}{1}
\received{2002}{1}{1}
\revised{2002}{1}{1}
\accepted{2002}{1}{1}

\setcounter{secnumdepth}{2}

\title{自動構築した格フレーム辞書と先行詞の位置選好順序を用いた\\省略解
析}
\author{河原 大輔\affiref{Nishida-Kurohashi-Lab} \and 黒橋 禎夫
\affiref{Nishida-Kurohashi-Lab}}

\headauthor{河原 大輔, 黒橋 禎夫}
\headtitle{自動構築した格フレーム辞書と先行詞の位置選好順序を用いた省略
解析}

\affilabel{Nishida-Kurohashi-Lab}{東京大学 大学院情報理工学系研究科,
Graduate School of Information Science and Technology, The University of
Tokyo}

\begin{abstract}
 本稿では、日本語文章中における格要素の省略(ゼロ代名詞)を検出し、その先
 行詞を同定する手法を提案する。本手法は、自動構築した格フレーム辞書に基
 づく格解析によってゼロ代名詞を検出し、同辞書による正確な選択制限を用い
 てゼロ代名詞の先行詞を同定する。また、先行詞はゼロ代名詞から近いところ
 に存在しやすいという傾向を正確にモデル化するために、文・文章中の構造を
 考慮した先行詞の位置選好順序をコーパスから学習し、これを解析で利用する。
 格フレーム辞書、先行詞の位置選好順序、さらに機械学習を統合した省略解析
 システムを作成し、100記事の大規模解析実験を行った結果、ゼロ代名詞検出が
 適合率87.1\%、再現率74.8\%、ゼロ代名詞の先行詞同定が61.8\%の精度であっ
 た。
\end{abstract}

\keywords{格フレーム, ゼロ代名詞, 省略解析}

\etitle{Zero Pronoun Resolution based on Automatically Constructed Case
Frames and Structural Preference of Antecedents}

\eauthor{Daisuke Kawahara\affiref{Nishida-Kurohashi-Lab} \and Sadao
Kurohashi\affiref{Nishida-Kurohashi-Lab}}

\begin{eabstract}
This paper describes a method to detect and resolve zero pronouns in
Japanese text. We detect zero pronouns by case analysis based on
automatically constructed case frames, and rank candidate antecedents of
a zero pronoun based on similarity to examples in the case frames. We
also introduce an order of antecedent location preference to precisely
capture the tendency that a zero pronoun has its antecedent in its close
position. Large experimental results on 100 articles indicate that the
precision and recall of zero pronoun detection are 87.1\% and 74.8\%
respectively and the accuracy of antecedent estimation is 61.8\%.
\end{eabstract}

\ekeywords{case frame, zero pronoun, anaphora resolution}


\newcommand{\sm}[1]{}

\newcounter{ex}
\newcommand{\itemex}{}
\def\ex#1{}

\maketitle

\section{はじめに}
\label{はじめに}

省略・照応を高精度に解析する技術は、自動要約、機械翻訳、質問応答などの言
語処理アプリケーションを高度化するために必要である。日本語における照応詞
は省略されることがほとんどであるので、本論文では、省略された照応詞(ゼロ
代名詞と呼ばれる)に注目し、その解析手法を提案する。省略解析における大き
な手がかりとしては、次の2つが考えられる。

\begin{description}
 \item[選択制限] 先行詞は、ゼロ代名詞の文脈、特に関係する述語からの意味
	    的制約をうける。
 \item[近距離特性] 先行詞は、ゼロ代名詞から近いところにある傾向がある。
\end{description}

選択制限のような意味的手がかりとしては、これまで、人、具体物、抽象物のよ
うな粗い意味属性しか利用されていなかった。これは、それ以上に詳細に記述さ
れたリソースが存在しなかったからである。我々は、大規模な格フレーム辞書を
自動的に構築しており、そこには個々の単語レベルで詳細な選択制限が記述され
ている。本研究では、この格フレーム辞書を選択制限に用いる。

また、格フレーム辞書はゼロ代名詞の検出にも不可欠である。格フレームには、
用言のとりうる格が記述されているので、入力文の述語項構造と格フレームとの
マッチングの結果、格フレームに対応づけられていない格があれば、それがゼロ
代名詞であると認識できる。従来の研究では、ゼロ代名詞の検出ができたと仮定
し研究対象としていないか、人手で作成した格フレーム辞書を用いてゼロ代名詞
を検出している。これと比べて、本研究では、自動構築した大規模格フレーム辞
書を用いており、一般の文章に対して実用的に使える省略解析システムを作成す
ることができる。

省略解析のもうひとつの大きな手がかりは近距離特性であり、これを解析に反映
しようという試みがこれまでもなされてきた
\cite{Aone1995,Yoshino2001,Seki2002b}。それらは、ゼロ代名詞と先行詞候補
間の距離を機械学習の素性や確率モデルのパラメータとするものである。しかし、
これらの手法の大きな問題は、距離をflatな尺度でしか計っていないということ
である。つまり、ゼロ代名詞と先行詞候補の間の文数や単語数を距離としており、
文・文章に当然存在する構造を考慮するようなことを行っていない。本研究では、
近距離特性を正確にモデル化するために、まず、従属節、主節、埋め込み文など
といった文・文章中の構造を考慮して、ゼロ代名詞に対する先行詞候補の位置を
分類する。そして、どの位置にある候補が先行詞となりやすいかを学習コーパス
から取得し、それを順に並べることによって先行詞の位置選好順序を得るという
ことを行う。

本研究で提案する省略解析手法は、格フレーム辞書と先行詞の位置選好順序に加
えて、先行詞の決定に関与するその他の要因を考慮するために機械学習による分
類器を用いる。本手法は、この3つを統合的に用いるもので、先行詞の位置選好
順に候補を調べ、分類器が正例と分類し、かつ、格フレームによる選択制限を満
たす候補を先行詞として決定する。


\section{関連研究}

省略・照応解析の研究は、人手で作成した規則による手法とコーパスを用いた統
計的手法に大別できる。

人手で作成した規則による省略・照応解析手法
\cite{Nakaiwa1993,Nakaiwa1996,Murata1997}は、照応詞と先行詞候補間の統語
的および意味的な制約・選好に着目したルールを作成し、それを適用することに
より省略・照応解析を行っている。これらは高い精度を実現しているが、中岩ら
は新聞記事のリード文や1文単位で独立した文、村田らは物語文や新聞記事のコ
ラムなどを対象としているため、一般の文章で利用するには規則の修正が必要で
あると思われる。

一方、機械学習や確率モデルを用いたコーパスベースの手法が提案されている。
機械学習による手法は、照応詞と先行詞候補の間の文数や単語数を距離尺度とし
て機械学習の素性のひとつにしている \cite{Aone1995,Yoshino2001}。これは前
節で述べたように、文・文章中の構造を考慮していないことが問題である。また、
これらの手法は、ある範囲の候補の中から、機械学習によって作成した分類器の
出力スコアがもっともよいものを先行詞として選択している。しかし、分類器の
出力スコアは、候補を独立に見た分類クラスの信頼度であり、候補間の比較に用
いるのは直接的でなく根拠が薄いと思われる。

確率モデルによる省略解析手法 \cite{Seki2002b}では、ゼロ代名詞と先行詞と
の間の文数をパラメータに組み込んでいる。しかし、パラメータ数の爆発を避け
るために、距離属性が他の属性と独立であることの仮定を用いており、この妥当
性は不明である。また、距離を文数のみで表しているので、同一文内照応の場合
は距離による選好が働かないという欠点がある。

日本語以外の言語でも照応解析は盛んに行われており、コーパスベースの手法が
一定の成功を収めている\cite{Ge1998,Soon2001,Muller2002,Ng2002a}。しかし、
これらの手法もまた、距離尺度として、照応詞と先行詞候補の間の文数や単語数
などしか用いていない。照応詞と先行詞候補の間の位置関係を構造的に捉えた手
法として、Hobbsによるものがある\cite{Hobbs1978,Ge1998}。この手法は、構文
木中を横断しながら先行詞を探索するモデルである。このモデルは、1文内の構
造によって探索順序を制御しているが、複数文にわたって探索する場合の優先順
序は扱っていない。


\section{格フレーム辞書の自動構築とそれに基づく省略解析}

本研究では、ゼロ代名詞の検出と、ゼロ代名詞の先行詞が満たすべき選択制限に、
自動的に構築した格フレーム辞書\cite{Kawahara2002e}を用いる。本章では、ま
ず、この格フレーム辞書の構築方法について概略を述べる。次に、格フレーム辞
書を用いた格解析、そして、その格解析結果を用いたゼロ代名詞検出手法につい
て説明する。


\subsection{格フレーム辞書の自動構築}
\label{Section::格フレーム辞書の自動構築}

大規模テキストから格フレーム辞書を自動構築する際の最大の問題は、用言の用
法の曖昧性である。つまり、同じ表記の用言でも複数の意味、用法をもち、とり
うる格や用例が異なる。例えば、「トラックに 荷物を 積む」と「経験を 積む」
は、用言は「積む」で同じであるが用法が異なっている。用法が異なる格フレー
ムを別々につくるために、我々は、格フレーム収集の単位を用言とその直前の格
要素の組とした。「積む」の例では、「荷物を積む」「経験を積む」を単位とし
て格フレームを収集する。さらに、「荷物を積む」「物資を 積む」などかなり
類似している格フレームをマージするためにクラスタリングを行う。

この格フレーム辞書構築の手順を以下に示す。

\begin{enumerate}
 \item テキストを構文解析する。
 \item 構文解析結果から信頼度の高い述語項構造を抽出する。
 \item 抽出した述語項構造を用言とその直前の格要素ごとにまとめ、(最初の) 
       格フレームをつくる。以後、用言の直前の格要素を「直前格要素」、そ
       の格を「直前格」と呼ぶ。
 \item 3でつくった格フレームをシソーラスに基づいてクラスタリングし、類似
       しているものをマージする。
 \item 格フレームごとに必須格を選択する。直前格の用例数に対して、閾値以
       上の用例をもつ格を必須格とする。ただし、ガ格は常に必須格とする。
\end{enumerate}

\vspace*{2ex}

4のクラスタリングでは、シソーラスとして日本語語彙大系\cite{NTT}を用い、
2用例間の類似度を定義し利用している。
2つの用例$e_1, e_2$間の類似度$sim(e_1, e_2)$は以下のように定義する。
\begin{eqnarray} \label{Formula::Similarity}
 & sim(e_1, e_2) = max_{x \in s_1, y \in s_2} \, sim(x, y) \\[5pt]
 & sim(x, y) = \frac{2L}{l_{x}+l_{y}} \nonumber
\end{eqnarray}
ここで、$x, y$は意味属性であり、$s_1, s_2$はそれぞれ$e_1, e_2$の日本語語
彙大系における意味属性の集合である(日本語語彙大系では、単語に複数の意味
属性が与えられている場合が多い)。$sim(x, y)$は意味属性$x, y$間の類似度で
あり、$l_{x}, l_{y}$は$x, y$のシソーラスの根からの階層の深さ、$L$ は$x$ 
と$y$の意味属性で一致している階層の深さを表す。この類似度$sim(x, y)$は0
から1の値をとる。

本研究では、新聞20年分のテキストから自動構築した格フレーム辞書を用いる。
この辞書には、約23,000個の用言が含まれており、1用言あたりの格フレーム数
は約14.5個である。構築した格フレームの例を表\ref{例::格フレーム}に示す。
表において、\sm{主体}は意味マーカの主体を表し、それをもつ格は人、組織と
いった主体的要素をとることを示す。格ごとにそれぞれの用例がシソーラス上で
主体属性以下にあるかどうかをチェックし、半数以上の用例が主体に属す場合に
\sm{主体}を付与している。

\begin{table}
 \begin{center}
  \caption{格フレームの例}
  \label{例::格フレーム}
  \begin{tabular}{c|l|l||c|l|l} \hline
   & 格 & 用例 & & 格 & 用例 \\ \hline\hline
   & ガ & \sm{主体}, 政府, 委員会, $\cdots$ & & ガ & \sm{主体}, 派, 政党,
   $\cdots$ \\
   \raisebox{-1.7ex}[0pt][0pt]{決定(1)} & ヲ & 方針, 案, 計画, $\cdots$
   & 擁立(1) & ヲ & \sm{主体}, 候補, 候補者 \\
   & デ & 会, 総会, 閣議, $\cdots$ & & ニ & \sm{主体}, 選挙区, 選,
   $\cdots$ \\ \cline{4-6} 
   & 時間 & \sm{時間} & & ガ & \sm{主体} \\ \cline{1-3}
   \raisebox{-1.7ex}[0pt][0pt]{決定(2)} & ガ & \sm{主体}, 地裁, 家裁,
   $\cdots$ & 擁立(2) & ヲ & \sm{主体}, 議員, 外相, $\cdots$ \\
   & ヲ & 処分, 措置, 扱い, $\cdots$ & & ニ & \sm{主体}, 候補, 後継, $\cdots$ \\ \hline
   \multicolumn{1}{c|}{$\vdots$} & \multicolumn{1}{c|}{$\vdots$} &
   \multicolumn{1}{c||}{$\vdots$} & \multicolumn{1}{c|}{$\vdots$} & \multicolumn{1}{c|}{$\vdots$} &
   \multicolumn{1}{c}{$\vdots$} \\ \hline
  \end{tabular}
 \end{center}
\end{table}


\subsection{格フレーム辞書を用いた省略解析}
\label{Section::格フレーム辞書を用いた解析}

本論文で提案する省略解析手法の概略は以下のとおりである。
\begin{enumerate} \itemsep = 0mm \renewcommand{\labelenumii}{}
 \item 入力文を構文解析する\footnote{格・省略解析と同時に構文構造を決め
ることもできるが、本研究では最初に構文構造を決定した。}。
 \item 入力文中の各用言について、文頭から順番に以下の処理を行う。\label{省略解析アルゴリズム::用言ごと}
 \begin{enumerate}
  \item 入力用言の述語項構造に合致する格フレームを選択する。\label{省略
	解析アルゴリズム::格フレーム選択}
  \item 格フレームと入力側の格要素との対応をとる。\label{省略解析アルゴ
	リズム::格要素対応}
  \item 格フレーム中で対応づけられていない格をゼロ代名詞と認識する。
	\label{省略解析アルゴリズム::ゼロ代名詞認識}
  \item ゼロ代名詞の先行詞を同定する。\label{省略解析アルゴリズム::先行詞同定}
 \end{enumerate}
\end{enumerate}

\vspace*{2ex}

手順2.1、2.2の処理は格解析であり、どちらにも以下のような難しさがある。

格フレーム辞書には、用言ごとに複数の格フレームが存在するので、入力文の用
言の用法にもっとも合致する格フレームを選択する。格フレームの選択を行うた
めには、入力の述語項構造が用言の用法を決定するのに十分な情報をもつ必要が
ある。十分な情報がなく、格フレームを選択できない場合は、入力用言のすべて
の格フレームについて、ゼロ代名詞の先行詞同定までの処理を行い、最後にもっ
ともスコアが高かった格フレームに決定する(\ref{Section::先行詞同定処理}章
参照)。

格要素の格フレームの格スロットへの対応付けは、基本的には一致している格同
士を対応付ければよいが、係助詞句や非連体修飾詞の場合には格が明示されてい
ないので問題となる。

以下では、手順2.1から2.3の処理を説明する。2.4の先行詞同定処理については
\ref{Section::先行詞同定処理}章で詳細に述べる。


\subsubsection{格フレームの選択} 

前節で述べたように、用言の用法の決定に対して、用言の直前格要素が重要な役
割を果たす。特に、直前格がヲ格、ニ格の場合はその傾向が強い。また、直前格
要素が\sm{主体}の場合、例えば「\sm{主体}が \ 求める」という表現からは、
用言の用法が決まらず、格フレームを選択することができない。これらの点を考
慮して、格フレームの選択を行う条件を以下のように設定する。
\begin{enumerate}
 \item 入力側の対象用言が直前格要素$C$をもつ。
 \item 直前格要素$C$と直前格$cm$が以下のいずれかの条件を満たす。\label
       {格フレーム選択アルゴリズム::直前条件}
 \begin{itemize}
  \item $cm$がヲ格、ニ格のいずれかである。
  \item $cm$がヲ格、ニ格以外で、$C$が意味属性\sm{主体}をもたない。
 \end{itemize}
 \item $cm$をもつ格フレームが存在し、$cm$の用例群と$C$の類似
       度が閾値以上ある。\label{格フレーム選択アルゴリズム::類似している格フレーム}
\end{enumerate}

条件\ref{格フレーム選択アルゴリズム::類似している格フレーム}を満たす格フ
レームのなかで、もっとも類似度が高い格フレームを選択する。もっとも類似度
が高い格フレームが複数存在するときは、格フレームを選択できない場合と同様
に、それらの格フレームそれぞれについて後続の処理を行った後に格フレームを
決定する。ここで用いる類似度は、直前格要素と格フレームの直前格の各用例と
の類似度のうちもっとも高いものとする。用例間の類似度は
(\ref{Formula::Similarity})式を用いて計算する。

例として、図\ref{図::記事の例}の2文目の「擁立する」を考える。「擁立」に
対して表\ref{例::格フレーム}のような格フレームがある。入力側の表現「候補
を \ 擁立する」は上記の条件1, 2を満たし、格フレーム「擁立(1)」が条件3を
満たすので、格フレーム「擁立(1)」が選択される。

\begin{figure}[t]
 \begin{center}
  \begin{tabular}{|l|} \hline
   \begin{minipage}{12cm}

    \vspace*{1ex}

    $\cdots$

    \vspace*{1ex}

    石原知事が再選を目指して、知事選への立候補を表明した。

    \vspace*{2ex}

    自民党は支持する方針を決定したが、民主党は独自候補を\underline{擁立
    する}ことを検討している。

    \vspace*{1ex}

    $\cdots$

    \vspace*{1ex}

   \end{minipage} \\ \hline
  \end{tabular}
 \caption{記事の例}
 \label{図::記事の例}
 \end{center}
\end{figure}


\subsubsection{入力側格要素と格フレームの格との対応付け}

選択された格フレームについて、入力側の格要素と格フレームの格との対応づけ
を行う\cite{Kuro-IEICE1994}。格要素に格助詞が付属している場合は、その格
助詞の格に対応する格フレーム側の格に対応づける。被連体修飾詞や係助詞句の
ように、文中から格がわからない場合は、次表の格それぞれに対応させ、対応づ
け全体のスコアがもっともよい対応を選択し、格を決定する。対応づけ全体のス
コアは、各格の対応の類似度を足したものとする。各格の対応の類似度は
「格フレームの選択」で述べた類似度と同様である。
\begin{center}
 \begin{tabular}{l@{ : }l} \hline
  係助詞句 & ガ, ヲ, ガ2\footnotemark[2] \\ 
  被連体修飾詞 & ガ, ヲ, 外の関係 \\ \hline
 \end{tabular}
\end{center}
\footnotetext[2]{二重主語構文の外のガ格を「ガ2」格と呼ぶ。}


\subsubsection{ゼロ代名詞の検出}

格解析が終わったときに、格フレームに入力文の格要素と対応づけられていない
格があり、それがガ格、ヲ格、ニ格のいずれかであれば、ゼロ代名詞であると認
識する。
図\ref{図::記事の例}の「候補を \ 擁立する」では、格フレーム「擁立(1)」が
選択されており、格フレームのガ格とニ格に対応する入力側の格要素がないこと
がわかる。従って、システムはガ格とニ格をゼロ代名詞として検出する。

検出されたゼロ代名詞に対する先行詞同定処理は\ref{Section::先行詞同定処理}
章で述べる。


\section{位置カテゴリの設定とその順序づけ}

格フレームの選択制限によって、先行詞候補は、先行詞として適格なもののみに
絞ることができるが、それでも複数残る場合が多い。一般に、ゼロ代名詞の先行
詞は、ゼロ代名詞から距離が近いところにある傾向(近距離特性)があるので、こ
の傾向を先行詞同定に利用することを考える。従来の研究では、ゼロ代名詞と先
行詞の間にある文数や単語数などflatな距離でその近さを捉えており、従属節、
主節、埋め込み文など文・文章中の構造を用いていない。本研究では、近距離特
性を正確にモデル化するために、まず、文・文章中の構造を考慮して、照応詞に
対する先行詞候補の位置を分類する。そして、どの位置にある候補が先行詞とな
りやすいかを学習コーパスから取得し、それを順に並べることによって先行詞の
位置選好順序を得る。

本章では、まず、学習コーパスにおける先行詞の捉え方を述べる。次に、ゼロ代
名詞と先行詞候補の位置関係の分類(\textbf{位置カテゴリ}と呼ぶ)を導入し、
最後にその順序づけ、つまり先行詞の位置選好順序について述べる。


\begin{table}[t]
 \begin{center}
  \caption{ゼロ代名詞の格分布 (上位8個まで)}
  \label{Table::格分布}
  \begin{tabular}{c|r@{ }r||c|r@{ }r}\hline
   格 & \multicolumn{1}{c}{回数} & \multicolumn{1}{c||}{(割合)} & 格 &
   回数 & \multicolumn{1}{c}{(割合)} \\ \hline
   ガ & 4,295 & (65.1\%) & ガ2 & 118 & (1.8\%) \\
   ニ & 1,026 & (15.5\%) & 外の関係 & 80 & (1.2\%) \\
   ヲ &   460 &  (7.0\%) & ト & 71 & (1.1\%) \\
   デ &   163 &  (2.5\%) & カラ & 47 & (0.7\%) \\ \hline
  \end{tabular}
 \end{center}
\end{table}

\begin{table}[t]
 \begin{center}
  \caption{先行詞の出現位置分布}
  \label{Table::出現位置分布}
  \begin{tabular}{c|r@{ }r||c|r@{ }r}\hline
   位置 & \multicolumn{1}{c}{回数} & \multicolumn{1}{c||}{(割合)} &位置
   & 回数 & \multicolumn{1}{c}{(割合)} \\ \hline
   同一文 & 3,354 & (67.3\%) & 2文前     & 301 & (6.0\%) \\
   1文前  &   990 & (19.9\%) & 3文前以前 & 341 & (6.8\%) \\ \hline
  \end{tabular}
 \end{center}
\end{table}


\subsection{関係コーパスにおける先行詞の捉え方}

本研究では、関係コーパス\cite{Kawahara2002ec}を用いて位置カテゴリの順序
の学習や解析実験などを行う。このコーパスは、新聞記事に対して文章中の用言・
サ変名詞に対する格関係、名詞間の関係、および共参照をタグづけしたものであ
る。用言に対する格関係には、ゼロ代名詞の先行詞が含まれている。

本研究で対象とするのは、関係コーパス中の379記事、3,695文である。用言(動
詞、形容詞、名詞+判定詞)は11,149個出現し、そのうちゼロ代名詞を格要素とす
る用言は5,530個あり、ゼロ代名詞は6,602個であった。ゼロ代名詞となっている
格の分布を表\ref{Table::格分布}に示す。ゼロ代名詞のうち先行詞が記事中に
存在するものは4,986個、先行詞が記事中に存在しないものは1,616個であった。
記事中に存在しない先行詞は不特定の人々である場合が多かった。

先行詞が記事中に存在するゼロ代名詞に対して、コーパスに直接タグづけされて
いるもの以外にも先行詞が存在することがある。それは、タグづけされている先
行詞と共参照タグで関係づけられているものや、同じ対象を指示する他のゼロ代
名詞であり、それらも先行詞として捉える。例えば、図
\ref{Figure::ZeroRecognition}において、「甘い」のガ格はゼロ代名詞であり、
その先行詞は前文主節の「とちおとめ」であるとタグづけされている。この「と
ちおとめ」と、前文従属節にある「とちおとめ」は共参照関係にあるので、前文
従属節の「とちおとめ」も先行詞とみなす。また、対象文の「大きく」のガ格も
ゼロ代名詞であり、これも「とちおとめ」を指しているため、このゼロ代名詞も
先行詞とみなす。

先行詞が出現する文の位置を同一文、1文前、2文前、$\cdots$、n文前、
$\cdots$のように分類したときの分布を表\ref{Table::出現位置分布}に示す。
この表において、先行詞が複数存在する場合は、nの値がもっとも小さい先行詞
の位置としている。次節で導入する位置カテゴリは、先行詞が対象文から2文前
までの間にある場合を対象としている。これによって93.2\%のゼロ代名詞がカバー
される。

\begin{figure}[t]
 \begin{center}
  \includegraphics[width=9cm]{antecedents.eps} \caption{先行詞の捉え方}
  \label{Figure::ZeroRecognition}
 \end{center} 
\end{figure}

以下で述べる位置カテゴリの順序づけの学習、\ref{Section::分類器の利用}章
で述べる分類器の学習は、関係コーパス中の279記事を学習コーパスとして用い
る。\ref{Section::実験}章の実験では、残りの100記事をテストコーパスとして
用いる。


\subsection{位置カテゴリの設定}

近距離特性を正確にモデル化するために、まず、ゼロ代名詞に対する先行詞候補
の位置を分類した位置カテゴリを導入する。これは、従属節、主節、埋め込み文
など文・文章中の節がもつ構造に着目して表\ref{表::位置カテゴリ}のように設
定した。
表\ref{表::位置カテゴリ}において、$V_z$はゼロ代名詞をもつ用言を示す。こ
こで、表\ref{表::位置カテゴリ}において「」で囲まれた用言を$V_a$と呼ぶこ
とにする($V_a$の格要素が先行詞候補である)。``並列''とは、$V_z$ と$V_a$が
並列関係にあることを示し、``主節''とは$V_a$がその文の主節用言であること
を示す。位置カテゴリはそれぞれ独立であり、``主節用言''は、主節属性をもつ
ほかの位置カテゴリに含まれない主節の用言を表す。

\begin{table}[t]
 \begin{center}
  \caption{設定した位置カテゴリ}
  \label{表::位置カテゴリ}
  \begin{tabular}{l@{: }lcc} \hline
   \multicolumn{4}{l}{対象文} \\ \hline
   $L_1$ & 「$V_z$の親用言」の格要素 & & 主節 \\
   $L_2$ & 「$V_z$の親用言」の格要素 & & \\
   $L_3$ & 「$V_z$の親用言」の格要素 & 並列 & 主節 \\
   $L_4$ & 「$V_z$の親用言」の格要素 & 並列 & \\
   $L_5$ & 「$V_z$の子用言」の格要素 & & \\
   $L_6$ & 「$V_z$の子用言」の格要素 & 並列 & \\
   $L_7$ & 「$V_z$の親体言の親用言」の格要素 & & 主節 \\
   $L_8$ & 「$V_z$の親体言の親用言」の格要素 & & \\
   $L_9$ & 「$V_z$の親用言の親用言」の格要素 & & 主節 \\
   $L_{10}$ & 「$V_z$の親用言の親用言」の格要素 & & \\
   $L_{11}$ & 「主節用言」の格要素 & & 主節 \\
   $L_{12}$ & 「主節に係る従属節用言」 & & \\
   $L_{13}$ & その他の節の格要素($V_z$より後) & & \\
   $L_{14}$ & その他の節の格要素($V_z$より前) & & \\ \hline
   \multicolumn{4}{l}{1文前} \\ \hline
   $L_{15}$ & 「主節用言」の格要素 & & 主節 \\
   $L_{16}$ & 「主節に係る従属節用言」の格要素 & & \\
   $L_{17}$ & その他の節の格要素 & & \\ \hline
   \multicolumn{4}{l}{2文前} \\ \hline
   $L_{18}$ & 「主節用言」の格要素 & & 主節 \\
   $L_{19}$ & 「主節に係る従属節用言」の格要素 & & \\
   $L_{20}$ & その他の節の格要素 & & \\ \hline
  \end{tabular}
 \end{center}
\end{table}

例えば、図\ref{Figure::ZeroRecognition}の「甘い」については、ガ格がゼロ
代名詞である。「移っている」は前文の主節であるので、その格要素である「中
心」「栽培」「とちおとめ」は$L_{15}$、「大きく」は子用言なので「とちおと
め」を指すゼロ代名詞と「女峰」は$L_5$などとなる。

\begin{figure}
 \begin{tabular}{@{}c@{}c@{}}
  \begin{minipage}{0.5\hsize}
   \begin{center}
    \includegraphics[width=7.1cm]{ga.eps}
    \caption{位置カテゴリの順序 (ガ格)}
    \label{図::位置カテゴリの順序-ガ格}
   \end{center}
  \end{minipage}
  & 
  \begin{minipage}{0.5\hsize}
   \begin{center}
    \includegraphics[width=7.1cm]{wo.eps}
    \caption{位置カテゴリの順序 (ヲ格)}
    \label{図::位置カテゴリの順序-ヲ格}
   \end{center}
  \end{minipage} \\
 \end{tabular}
\end{figure}

\begin{figure}
 \begin{center}
  \includegraphics[width=7.1cm]{ni.eps}
  \caption{位置カテゴリの順序 (ニ格)}
  \label{図::位置カテゴリの順序-ニ格}
 \end{center}
\end{figure}

\begin{figure}
 \begin{center}
  \includegraphics[width=9cm]{location-class.eps}
  \caption{$V_z$ = 「擁立する」のときの位置カテゴリ}
  \label{Figure::LocationClass}
 \end{center} 
\end{figure}


\subsection{位置カテゴリの順序づけ}

それぞれの位置カテゴリが、どれくらい先行詞をとりやすいかを関係コーパスか
ら取得し、その順に位置カテゴリを並べることによって、先行詞の位置選好順序
を得る。位置カテゴリ$L$の先行詞のとりやすさを次式のようにスコア化する。
\[
 \frac{\mbox{先行詞が}L\mbox{にある回数}}{L\mbox{にある先行詞候補の数}}
\]
例えば、図\ref{Figure::ZeroRecognition}の「甘い」のゼロ代名詞に対して、
前文の主節$L_{15}$に含まれる先行詞候補は「中心」「栽培」「とちおとめ」の
3つであり、「とちおとめ」は先行詞であるので、この位置カテゴリのスコアは
$1/3$となる。子用言の節$L_{5}$については、候補が「とちおとめ」を指示する
ゼロ代名詞と「女峰」の2つなので、スコアは$1/2$となる。

学習コーパス279記事を用いて、スコアをゼロ代名詞の格ごとに集計し、スコア
順に位置カテゴリを並べた。ガ格、ヲ格、ニ格の順序をそれぞれ、図\ref{図::
位置カテゴリの順序-ガ格}、\ref{図::位置カテゴリの順序-ヲ格}、\ref{図::位
置カテゴリの順序-ニ格}に示す。ガ格のゼロ代名詞は、主節の親用言の格要素を
先行詞としやすく、ヲ格のゼロ代名詞は、子用言の格要素を先行詞とすることが
多いことなどがわかる。

図\ref{図::記事の例}の記事中の「擁立する」のガ格とニ格のゼロ代名詞に対す
る位置カテゴリの順序を図\ref{Figure::LocationClass}に示す。ガ格では$L_7$ 
が1位になるが、ニ格では$L_{14}$が1位となり、格ごとに位置カテゴリの順序が
異なっている。また、$L_{12}$のように、構造を考慮しない距離尺度でゼロ代名
詞に近い位置であっても、位置カテゴリ順の上位にくるわけではないことがわか
る。


\begin{table*}
 \begin{center}
  \caption{分類器に用いる素性}
  \label{Table::Features}
 \begin{tabular}{ll} \hline
  \multicolumn{2}{@{ }l}{ゼロ代名詞と先行詞候補の両方に関わる素性} \\ \hline
  \textit{similarity} & 先行詞候補と格フレームの用例との類似度 \ \ (0 - 1) \\
  \textit{loc\_category}& 先行詞候補の位置カテゴリ ($L_1, \cdots, L_{20}$) \\
  \textit{before\_ana} & 先行詞候補がゼロ代名詞より前にある \ \ (yes, no) \\
  \textit{depend\_over\_ana} & 先行詞候補がゼロ代名詞を越えて係る \ \ (yes, no) \\ \hline
  \multicolumn{2}{@{ }l}{先行詞候補の素性} \\ \hline
  \textit{ante\_CM} & 先行詞候補の格 \ \ (が, を, に, $\cdots$) \\
  \textit{ante\_pred\_nm} & 先行詞候補が係る用言が連体修飾節を構成してい
  る \ \ (yes, no) \\
  \textit{ante\_fs} & 先行詞候補がその記事の先頭文にある \ \ (yes, no) \\
  \textit{ante\_depend\_mc} & 先行詞候補がその文の主節に係る
  \ \ (yes, no) \\
  \textit{ante\_tm} & 先行詞候補が後に係助詞「は」を伴っている
  \ \ (yes, no) \\
  \textit{ante\_agent} & 先行詞候補が意味属性\sm{主体}に属する \ \ (yes,
  no) \\
  \textit{ante\_pred\_level} & 先行詞候補の係る用言の節の強さ \ \ (0, 1, $\cdots$, 6) \\ \hline
  \multicolumn{2}{@{ }l}{ゼロ代名詞の素性} \\ \hline
  \textit{ana\_CM} & ゼロ代名詞の格 \ \ (が, を, に) \\
  \textit{ana\_pred\_nm} & ゼロ代名詞をもつ用言が連体修飾節を構成してい
  る \ \ (yes, no) \\
  \textit{ana\_pred\_voice} & ゼロ代名詞をもつ用言の態 \ \ (能動,
  受動, 使役) \\
  \textit{ana\_pred\_type} & ゼロ代名詞をもつ用言の種類 \ \ (動詞, 形容詞, 名詞+判定詞) \\
  \textit{ana\_head\_verbal} & ゼロ代名詞をもつ用言がサ変名詞である \ \ (yes, no) \\
  \textit{ana\_cf\_agent} & 格フレームの用例が意味属性\sm{主体}に属する \ \ (yes, no) \\ \hline
 \end{tabular}
 \end{center}
\end{table*}


\section{分類器の利用}
\label{Section::分類器の利用}

本研究では、先行詞の決定に関与する様々な要因を考慮するために機械学習によ
る分類器を用いる。この分類器は、先行詞候補が先行詞として適格かどうかを判
定する2値分類器である。これは、表\ref{Table::Features}に示した素性を用い
て作成する。素性は、類似度や位置カテゴリなど、先行詞候補とゼロ代名詞の両
方に関係するもの、格や意味属性など、先行詞候補とゼロ代名詞のどちらか一方
にのみ関係するものからなる。これらは自動的に生成されるのでノイズが含まれ
る。類似度\textit{similarity}は、格フレームの選択において格フレームが決
定できない場合には計算できない。このような場合には、対象となっている用言
のすべての格フレームに含まれる用例群と先行詞候補との類似度をとることにす
る。

分類器としてはSupport Vector Machines (SVM) \cite{SVM}を用いる。訓練デー
タは関係コーパスを用いて作成する。正例としては、ゼロ代名詞からもっとも近
く、ゼロ化していない先行詞が存在する文と解析対象の文との間にある先行詞と
し、負例は、その間にある先行詞以外の候補とする。もっとも近い先行詞が解析
対象の文の2文前までに存在しないゼロ代名詞については、訓練対象から除くこ
とにする。図\ref{Figure::ZeroRecognition}の場合は、「甘い」のゼロ代名詞
に対して、「大きく」のゼロ代名詞と前文の2つの「とちおとめ」を正例とし、
それ以外の名詞を負例とする。


\section{先行詞同定処理}
\label{Section::先行詞同定処理}

本研究で作成した省略解析システム(\ref{Section::格フレーム辞書を用いた解
析}節)の先行詞同定処理は、位置カテゴリの順序、格フレーム辞書、分類器を統
合的に用いる。本手法は、格ごとに設定された位置カテゴリの順序に従って先行
詞候補を調べ、格フレームの用例との類似度が閾値を越え、かつ、分類器によっ
て正例と分類される候補を先行詞に決定する。先行詞としての適格さを計るため
に分類器のみを用いる手法も考えられるが、本研究では格フレームとの類似度を
重要視し、分類器と類似度のand条件とした。これは、次節で述べる実験によっ
ても有効性が示されている。

本手法の先行詞探索範囲は、対象のゼロ代名詞がある文、1文前、2文前の3文と
し、そこに含まれる形式名詞、副詞的名詞以外の名詞を先行詞の候補とする。本
手法は、文頭に近い用言から文末に向かって順番に解析しており、すでに解析さ
れたゼロ代名詞も候補に含める。先行詞探索範囲内に条件を満たす先行詞候補が
みつからない場合において、格フレームに意味マーカ\sm{主体}があれば、先行
詞として「不特定:人」を出力する。これは文章中に存在しない不特定の人々を
指すという意味である。候補と格フレームの用例との類似度は、格フレームの対
象としている格に属する用例群と候補との類似度で、「格フレームの選択」の節
で述べたものと同じである。

手順2.1「格フレームの選択」において、格フレームが選択されなかった場合は、
それぞれの格フレームについて先行詞同定までの処理を行う。省略の指示対象も
含む入力述語項構造と格フレームとの対応づけ全体のスコアを格フレームのスコ
アとし、それがもっとも高かった格フレームに決定する。対象となっている用言
の省略解析の結果は、その格フレームを用いた場合のものとする。

類似度の閾値は0.60に設定した。ヲ格、ニ格に関しては、位置カテゴリのうち、
スコアが0.20より低いものは使わないことにした。これは、ある程度よりスコア
が低い位置カテゴリは、ゼロ代名詞を余分に認識させることにつながり、適合率
を悪化させるためである。これらの閾値は、学習コーパスを用いて交差検定を行
うことにより決定した。

図\ref{図::記事の例}の2文目の「擁立」では、ガ格とニ格がゼロ代名詞として
認識されている。例えば、ガ格の候補は、位置カテゴリ順に、$L_{7}$:民主党,
$L_{14}$:自民党($\phi$ガ), $L_{14}$:石原知事($\phi$ヲ), $\cdots$となって
いる。1番目の「民主党(類似度: 0.73)」は分類器によって正例と分類され、格
フレームの用例との類似度が0.73と閾値を越えているので先行詞に決定される。


\section{実験}
\label{Section::実験}

関係コーパスを用いて、省略解析の実験を行った。コーパス中の各記事の長さを
ある程度そろえるために、各記事の先頭10文までを用いることにした。学習コー
パス279記事を分類器の学習、テストコーパス100記事をテストに用いた。分類器
としては、SVMパッケージ\textit{TinySVM} \cite{TinySVM}を用い、2次の多項
式カーネルで学習を行った。テストとしては、正解の構文構造をもつ100記事を
システムに入力し、実験・評価を行った。

\begin{figure}[t]
 \begin{center}

  \vspace*{4ex}

  \includegraphics[width=13cm]{ex-config-ja.eps}
  \caption{実験設定}
  \label{Figure::Configurations}
 \end{center}
\end{figure}

\begin{table}[t]
 \begin{center}
  \caption{省略解析結果}
  \label{Table::Result}
  \begin{tabular}{c||r@{ }r|r@{ }r|r} \hline
   & \multicolumn{2}{c|}{適合率} & \multicolumn{2}{c|}{再現率} &
   \multicolumn{1}{c}{F値} \\ \hline
   ex-1 & 443/908  & (48.8\%) & 443/1072 & (41.3\%) & 44.7\% \\
   ex-2 & 449/841  & (53.4\%) & 449/1072 & (41.9\%) & 46.9\% \\
   ex-3 & 472/964  & (49.0\%) & 472/1072 & (44.0\%) & 46.4\% \\
   ex-4 & 480/933  & (51.5\%) & 480/1072 & (44.8\%) & 47.9\% \\
   ex-5 & 494/976  & (50.6\%) & 494/1072 & (46.1\%) & 48.2\% \\
   ex-6 & 239/1149 & (20.8\%) & 239/1072 & (22.3\%) & 21.5\% \\
   ex-7 & 496/921  & (53.9\%) & 496/1072 & (46.3\%) & 49.8\% \\ \hline
  \end{tabular}
 \end{center}
\end{table}

その他の手法を比較するために、図\ref{Figure::Configurations}のように7つ
の設定について実験を行った。パラメータは、「先行詞の探索戦略」、「距離の
尺度」、「スコア付け」の3つである。「先行詞の探索戦略」とは、どのように
先行詞を探索し決定するかで、``best''と``closest''の2種類である。``best'' 
とは、先行詞候補のなかでもっともスコアのよい候補を選択する手法で、
``closest'' とは、近い候補から順に調べていき、スコアがある閾値を越える候
補を先行詞に決定する手法である。「距離の尺度」とは、照応詞と先行詞候補間
の距離の近さを計る尺度であり、``structure''と``flat''の2種類である。
``structure''とは、文・文章を構造的にみる手法であり、``flat''とは、間に
ある文節数を距離とする手法である。「スコア付け」は、先行詞候補を比較する
スコアとして何を用いるかで、「分類器」、「類似度」、「分類器\&類似度」の
3種類である。「分類器」は分類器の出力する信頼度をそのまま用いる手法で、
「類似度」は先行詞候補と格フレームの用例との類似度を用いる手法であり、
「分類器\&類似度」は本手法のように分類器と類似度の両方を用いる手法である。
図のex-7が本手法であり、ex-1が\cite{Aone1995,Yoshino2001,Ng2002a}による
手法に類似しており、ex-3が\cite{Soon2001,Muller2002}の手法に類似している。

実験結果を表\ref{Table::Result}に示す。表の適合率、再現率はゼロ代名詞検
出、先行詞の同定処理の双方を併せて評価したものである。先行詞同定の評価は、
システムの出力する先行詞が正解コーパスの先行詞(群)に含まれていれば正解と
する。システムの出力が「不特定:人」の場合も評価に含め、正解コーパスも
「不特定:人」であれば正解とする。表\ref{Table::Result}によると、ex-7の精
度が他の手法を上回っており、本手法の有効性が示されている。また、ex-7につ
いて、ゼロ代名詞の検出と先行詞の同定の評価を別々に行った。ゼロ代名詞検出
については、適合率87.1\%、再現率74.8\%、F値80.5\%であり、先行詞同定につ
いては61.8\%の精度であった。

本手法の精度を、実験環境が類似している関らの手法 \cite{Seki2002b}と比較
する。関らは、新聞の社説記事30件と報道記事30件に対して、ゼロ代名詞検出と
先行詞の同定処理を行っている。我々が用いたコーパスには社説記事は含まれて
いなかったので、報道記事についての精度をここに挙げると、ゼロ代名詞検出は
適合率48.9\%、再現率88.2\%、F値62.9\%、先行詞同定については54.0\%の精度
となっている。コーパスのサイズの違いなどがあるため一概に比較するのは難し
いが、本手法の精度は関らの手法より、ゼロ代名詞検出のF値が17.6\%、先行詞
同定の精度が7.8\%よい。特に、ゼロ代名詞検出の精度は大幅に向上しており、
関らの用いた人手構築の格フレーム辞書より、自動構築した格フレーム辞書が有
効であったと思われる。

主な誤り原因を以下に示す。

\vspace*{-2ex}

\paragraph{ゼロ代名詞の検出誤り} \ \\[5pt]
ゼロ代名詞を余分に検出する傾向がある。

\ex{一方、ドゥダエフ政権側の首都防衛司令官は同日夕、テレビを通じ、首都防
衛はうまくいっており、ロシア軍の戦車五十両を破壊したと\fbox{発表}。}

\noindent
この例では、「発表」の格フレームに対応づけられていないニ格があり、ニ格を
ゼロ代名詞と認識してしまう。これは格フレームが悪いのではなく、文脈によっ
てその格をとる場合ととらない場合があるためである。これに対処するためには、
文脈に関する素性を機械学習に入れる必要がある。


\paragraph{省略解析の対象を限定していることによる誤り} \ \\[5pt]
現在の省略解析システムは用言のみしか対象としていない。

\ex{村山富市\underline{首相}$_{×}$は年頭にあたり首相官邸で内閣記者会と
二十八日会見し、\underline{社会党}$_{○}$の新民主連合所属議員の離党問題
について「政権に影響を及ぼすことにはならない。離党者がいても、その範囲に
とどまると思う」と述べ、大量離党には\fbox{至らない}との見通しを示した。}

\noindent
この例の「至らない」のガ格は「社会党」(○印)であるが「首相」(×印)と誤っ
て解析される。これは、「社会党」は位置カテゴリ順では上位に来ないためであ
る。この誤りは、省略解析の対象をサ変名詞など用言以外にも広げることによっ
て解決できると思われる。つまり、「離党(問題)」「(大量)離党」のガ格に「社
会党」が補われ、「(大量)離党」は「至らない」に対して位置カテゴリの上位な
ので、「至らない」のガ格に「社会党」が補われる。このように用言だけでなく
文章中のより多くの関係を用いれば、省略解析の精度向上が期待できる。


\section{おわりに}

本稿では、格フレーム辞書、先行詞の位置選好順序、機械学習の3つを統合した
省略解析手法を提案した。本手法は、ゼロ代名詞検出とその先行詞同定の2つの
処理からなる。ゼロ代名詞検出は、格フレーム辞書に基づく格解析によって行う。
先行詞同定は、先行詞の位置選好順に候補を調べ、機械学習による分類器が正例
と分類し、かつ、格フレームによる選択制限を満たす候補を先行詞として決定す
る。大規模解析実験を行った結果、ゼロ代名詞検出が適合率87.1\%、再現率
74.8\%、ゼロ代名詞の先行詞同定が61.8\%の精度であった。特に、ゼロ代名詞検
出の精度は、先行研究と比べてかなり向上しており、自動構築した格フレーム辞
書の有効性が示された。今後は、解析対象にサ変名詞を入れるなど文章中の関係
をさらに捉えることを目指す予定である。


\bibliographystyle{jnlpbbl}
\begin{thebibliography}{}

\bibitem[\protect\BCAY{Aone \BBA\ Bennett}{Aone \BBA\ Bennett}{1995}]{Aone1995}
Aone, C.\BBACOMMA\  \BBA\ Bennett, S.~W. \BBOP 1995\BBCP.
\newblock \BBOQ Evaluating Automated and Manual Acquisition of Anaphora
  Resolution Strategies\BBCQ\
\newblock In {\Bem Proceedings of the 33rd Annual Meeting of the Association
  for Computational Linguistics}, \BPGS\ 122--129.

\bibitem[\protect\BCAY{Ge, Hale, \BBA\ Charniak}{Ge et~al.}{1998}]{Ge1998}
Ge, N., Hale, J., \BBA\ Charniak, E. \BBOP 1998\BBCP.
\newblock \BBOQ A Statistical Approach to Anaphora Resolution\BBCQ\
\newblock In {\Bem Proceedings of the 6th Workshop on Very Large Corpora},
  \BPGS\ 161--170.

\bibitem[\protect\BCAY{Hobbs}{Hobbs}{1978}]{Hobbs1978}
Hobbs, J. \BBOP 1978\BBCP.
\newblock \BBOQ Resolving Pronoun References\BBCQ\
\newblock {\Bem Lingua}, {\Bbf 44}, 339--352.

\bibitem[\protect\BCAY{Kawahara \BBA\ Kurohashi}{Kawahara \BBA\
  Kurohashi}{2002}]{Kawahara2002e}
Kawahara, D.\BBACOMMA\  \BBA\ Kurohashi, S. \BBOP 2002\BBCP.
\newblock \BBOQ Fertilization of Case Frame Dictionary for Robust {J}apanese
  Case Analysis\BBCQ\
\newblock In {\Bem Proceedings of the 19th International Conference on
  Computational Linguistics}, \BPGS\ 425--431.

\bibitem[\protect\BCAY{Kawahara, Kurohashi, \BBA\ Hasida}{Kawahara
  et~al.}{2002}]{Kawahara2002ec}
Kawahara, D., Kurohashi, S., \BBA\ Hasida, K. \BBOP 2002\BBCP.
\newblock \BBOQ Construction of a {J}apanese Relevance-tagged Corpus\BBCQ\
\newblock In {\Bem Proceedings of the 3rd International Conference on Language
  Resources and Evaluation}, \BPGS\ 2008--2013.

\bibitem[\protect\BCAY{Kudo}{Kudo}{2002}]{TinySVM}
Kudo, T. \BBOP 2002\BBCP.
\newblock {\Bem TinySVM: Support Vector Machines}.
\newblock http://cl.aist-nara.ac.jp/{\tt\symbol{"7E}}taku-ku/software/TinySVM/.

\bibitem[\protect\BCAY{Kurohashi \BBA\ Nagao}{Kurohashi \BBA\
  Nagao}{1994}]{Kuro-IEICE1994}
Kurohashi, S.\BBACOMMA\  \BBA\ Nagao, M. \BBOP 1994\BBCP.
\newblock \BBOQ A Method of Case Structure Analysis for {J}apanese Sentences
  based on Examples in Case Frame Dictionary\BBCQ\
\newblock In {\Bem IEICE Transactions on Information and Systems},
  \lowercase{\BVOL}\ E77-D No.2.

\bibitem[\protect\BCAY{{M\"uller}, Rapp, \BBA\ Strube}{{M\"uller}
  et~al.}{2002}]{Muller2002}
{M\"uller}, C., Rapp, S., \BBA\ Strube, M. \BBOP 2002\BBCP.
\newblock \BBOQ Applying Co-Training to Reference Resolution\BBCQ\
\newblock In {\Bem Proceedings of the 40th Annual Meeting of the Association
  for Computational Linguistics}, \BPGS\ 352--359.

\bibitem[\protect\BCAY{Ng \BBA\ Cardie}{Ng \BBA\ Cardie}{2002}]{Ng2002a}
Ng, V.\BBACOMMA\  \BBA\ Cardie, C. \BBOP 2002\BBCP.
\newblock \BBOQ Improving Machine Learning Approaches to Coreference
  Resolution\BBCQ\
\newblock In {\Bem Proceedings of the 40th Annual Meeting of the Association
  for Computational Linguistics}, \BPGS\ 104--111.

\bibitem[\protect\BCAY{Soon, Ng, \BBA\ Lim}{Soon et~al.}{2001}]{Soon2001}
Soon, W.~M., Ng, H.~T., \BBA\ Lim, D. C.~Y. \BBOP 2001\BBCP.
\newblock \BBOQ A Machine Learning Approach to Coreference Resolution of Noun
  Phrases\BBCQ\
\newblock {\Bem Computational Linguistics}, {\Bbf 27}  (4), 521--544.

\bibitem[\protect\BCAY{Vapnik}{Vapnik}{1995}]{SVM}
Vapnik, V. \BBOP 1995\BBCP.
\newblock {\Bem The Nature of Statistical Learning Theory}.
\newblock Springer.

\bibitem[\protect\BCAY{NTT}{NTT}{1997}]{NTT}
NTTコミュニケーション科学研究所 \BBOP 1997\BBCP.
\newblock \Jem{日本語語彙大系}.
\newblock 岩波書店.

\bibitem[\protect\BCAY{関和広, 藤井敦, 石川徹也}{関和広\Jetal
  }{2002}]{Seki2002b}
関和広, 藤井敦, 石川徹也 \BBOP 2002\BBCP.
\newblock \JBOQ 確率モデルを用いた日本語ゼロ代名詞の照応解析\JBCQ\
\newblock \Jem{自然言語処理}, {\Bbf 9}  (3), 63--85.

\bibitem[\protect\BCAY{吉野圭一, 竹内和広, 松本裕治}{吉野圭一\Jetal
  }{2001}]{Yoshino2001}
吉野圭一, 竹内和広, 松本裕治 \BBOP 2001\BBCP.
\newblock \JBOQ 機械学習を用いた日本語ゼロ代名詞照応関係の同定\JBCQ\
\newblock \Jem{言語処理学会 第7回年次大会発表論文集}, \BPGS\ 506--509.

\bibitem[\protect\BCAY{村田真樹 長尾眞}{村田真樹\JBA 長尾眞}{1997}]{Murata1997}
村田真樹\BBACOMMA\  長尾眞 \BBOP 1997\BBCP.
\newblock \JBOQ
  用例や表層表現を用いた日本語文章中の指示詞・代名詞・ゼロ代名詞の指示対象の推
定\JBCQ\
\newblock \Jem{自然言語処理}, {\Bbf 4}  (1), 87--109.

\bibitem[\protect\BCAY{中岩浩巳 池原悟}{中岩浩巳\JBA
  池原悟}{1993}]{Nakaiwa1993}
中岩浩巳\BBACOMMA\  池原悟 \BBOP 1993\BBCP.
\newblock \JBOQ
  日英翻訳システムにおける用言意味属性を用いたゼロ代名詞照応解析\JBCQ\
\newblock \Jem{情報処理学会論文誌}, {\Bbf 34}  (8), 1705--1715.

\bibitem[\protect\BCAY{中岩浩巳 池原悟}{中岩浩巳\JBA
  池原悟}{1996}]{Nakaiwa1996}
中岩浩巳\BBACOMMA\  池原悟 \BBOP 1996\BBCP.
\newblock \JBOQ
  語用論的・意味論的制約を用いた日本語ゼロ代名詞の文内照応解析\JBCQ\
\newblock \Jem{自然言語処理}, {\Bbf 3}  (4), 49--65.

\end{thebibliography}

\begin{biography}
\bioauthor{河原 大輔}{
1997年京都大学工学部電気工学第二学科卒業。
1999年同大学院修士課程修了。2002年同大学院博士課程単位取得認定退学。
現在、東京大学大学院情報理工学系研究科 学術研究支援員。
構文解析、省略解析の研究に従事。}
\bioauthor{黒橋 禎夫}{
1989年京都大学工学部電気工学第二学科卒業。
1994年同大学院博士課程修了。
京都大学工学部助手、京都大学情報学研究科講師を経て、2001年東京大学大学院
情報理工学系研究科助教授、現在に至る。
自然言語処理、知識情報処理の研究に従事。}
\end{biography}

\end{document}
