    \documentclass[english]{jnlp_1.4_rep}

\usepackage{jnlpbbl_1.3}
\usepackage[dvipdfm]{graphicx}
\usepackage{hangcaption_jnlp}
\usepackage{udline}
\setulminsep{1.2ex}{0.2ex}
\usepackage{amsmath}

\usepackage{theorem} 
\theorembodyfont{\normalfont}
\theoremstyle{break}
\newtheorem{theorem}{}[] 
\theoremstyle{plain}
\newtheorem{lemma}{}[] 
\newtheorem{syoumei}{}
\renewcommand{\thesyoumei}{}
\newcommand{\proof}[1]{}
\theoremstyle{plain} 
\newtheorem{definition}{}


\Volume{21}
\Number{4}
\Month{September}
\Year{2014}

\received{2005}{10}{31}
\accepted{2006}{2}{4}

\setcounter{page}{733}

\etitle{Preference Dependency Grammar and its Packed Shared Data Structure ``Dependency Forest''\footnotetext{\llap{*~}This article has been partially revised for better understanding of overseas readers.}}
\eauthor{Hideki Hirakawa\affiref{TOSHIBA}} 
\eabstract{
Preference dependency grammar (PDG) is a framework for integrating
morphological, syntactic, and semantic analyses. PDG provides packed
shared data structures that can efficiently encompass all possible
interpretations at each level of sentence analyses with preference
scores. Using the structure, PDG can calculate a globally optimized
interpretation for the target sentence. This paper first gives an
overview of the PDG framework by describing the base model of PDG,
which is a sentence analysis model, called a ``multi-level packed
shared data connection model.'' Then this paper describes packed
shared data structures, e.g., headed parse forests and dependency
forests, adopted in PDG. Finally, the completeness and soundness of
the mapping between the parse forest and the dependency forest are
revealed.
}
\ekeywords{Sentence analysis, Phrase structure, Dependency structure, Packed shared data structure}

\headauthor{Hirakawa}
\headtitle{PDG and its Packed Shared Data Structure ``Dependency Forest''}

\affilabel{TOSHIBA}{}{Toshiba Corporation}

\Reprint[T]{Vol.~13, No.~3, pp.~37--90}

\begin{document}

\maketitle

\vspace{2pt}
\section{Introduction}

Natural language processing systems have very complicated and
hard-to-formalize design 
\linebreak
issues because they have to treat some
specific features of natural languages, such as variety and 
\linebreak
ambiguity,
regularity and exceptionality, extensiveness and size, and time
variance. A natural language analysis (NLA) system has to manage these
features properly, according to the requirement level specified by its
application. One of the most important challenges in the design of NLA
systems is the treatment of varieties and ambiguities in natural
languages; i.e., how to find the correct interpretation of a sentence
from among the myriad of ambiguities at the analysis
levels of morphology, syntax, semantics, discourse, and pragmatics.

More precisely, an NLA system computes structures for a sentence by
generating a set of its possible interpretations (application of
interpretation generation knowledge), rejecting impossible
interpretations (application of constraint knowledge), and obtaining
the preference order of possible interpretations (application of
preference knowledge). 
\pagebreak
Figure \ref{fig:NLAnalysisModel} shows this
sentence analysis model.\footnote{Constraint knowledge can be defined
  as a type of preference knowledge that does not provide any
  possibility. However, the application of constraint knowledge
  implies pruning in the computation, which is in clear contrast with
  the application of preference knowledge.} A set of interpretations
of a sentence exists in the interpretation space prescribed by the
interpretation description scheme. Each interpretation is either
correct (◎), plausible (○), or implausible (×) with respect
to the real-world situation. The interpretation generation knowledge
generates a set of possible interpretations, constraint knowledge
filters out any impossible interpretation, and preference knowledge
provides the ordering of interpretations in the interpretation
space. The output of an NLA system is the optimum interpretation
extracted from among the remaining interpretations, according to the
order of preference. All three knowledge types exist at each level of
sentence analysis, i.e., morphology, syntax, semantics, etc. The
comprehensive ability of an NLA system is determined by the total of
all knowledge at each language analysis level. Preference knowledge
sometimes gives different suggestions for a sentence interpretation at
different levels \cite{Hirakawa89}.Such inconsistencies must be
resolved appropriately in order to construct an accurate NLA system.

\begin{figure}[t]
\begin{center}
\includegraphics{21-4ia6f1.eps}
\end{center}
\caption{Natural language analysis system model}
\label{fig:NLAnalysisModel}
\vspace{-0.7\Cvs}
\end{figure}

Interpretation generation knowledge and constraint knowledge
define the sentence coverage, i.e., a set of acceptable sentences in
an NLA system. Therefore, these knowledge types correspond to a grammar in
linguistics from the Chomskyan viewpoint. Many computational grammar
frameworks have been proposed and studied, in which a variety of linguistic
knowledge has been incorporated. Grammar frameworks are based on
interpretation description formalisms that prescribe interpretation
spaces such as phrase structure, dependency structure, semantic graph
structure, and logical formula.

In addition, considerable research on preference knowledge, such as
preference semantics \cite{Wilks75}, has been conducted in
linguistics. Preference knowledge has been widely adopted in NLA
systems through the use of a statistical method adapted from speech
technology. As statistical methods extend their application scope from
the N-gram model (word sequence) to context-free and
dependency grammar, more NLA systems can benefit from the
statistical power obtained using large-scale corpora.

For example, statistical models are introduced to
phrase-structure-based grammar frameworks such as CFG and lexical
functional grammar (LFG) \cite{Kaplan89,Riezler02},head-driven
phrase structure grammar (HPSG) \cite{Pollard94,Tsuruoka04},
combinatory categorical grammar (CCG)
\cite{Steedman00,Clark03},\footnote{Some systems produce deep
  structures as an output. The classification here is based on the
  interpretation structure during the parsing.} and to
dependency-structure-based grammar frameworks such as probabilistic
dependency grammar \cite{Lee97}, kakari-uke analysis
\cite{Shudo80,Ozeki94,Hirakawa01,Kudo05_e,Kawahara05_e}, 
constraint dependency grammar (CDG) \cite{Maruyama90,Wang04}, and link grammar
\cite{Sleator91,Lafferty92}.\footnote{Link grammar is not considered as
an instance of dependency grammar by its creators, and it departs from
the traditional view of dependency by using undirected links; however,
the representations used in link grammar parsing are similar to the
dependency representations in that they consist of words linked by
binary relations \cite{nivre2005dga}.}

From the viewpoint of integrating multilevel knowledge, grammar
frameworks are classified as either single-space or multi-space
models. Simple CFG and simple dependency grammar are typical examples
of a single-space framework without preference knowledge treatment,
based on phrase structure and dependency structure schemes,
respectively. DCG \cite{Pereira80} and BUP \cite{Matsumoto83} have
developed a mechanism to extend the CFG framework to incorporate
arbitrary extra conditions using Prolog codes, which, for example, can
be used for introducing semantic constraints. This is a type of
integration of multilevel constraint knowledge in the single-space
model. CDG is a single-space grammar framework based on the dependency
scheme with unary or binary constraints. These constraints are used
for incorporating multilevel constraint knowledge \cite{Maruyama90}.
On the other hand, LFG has at least two types of description schemes
(or data structures), i.e., c-structure (phrase structure) and
f-structure (functional structure).\footnote{Kaplan identified an idea
  to extend the LFG configuration to include more structures such as
  semantic and anaphoric structures.} The functional scheme provides
the constraints on the functional structures \cite{Kaplan89}.  PCFG
\cite{jelinek92} is an example of a single-space model with preference
knowledge. A certain number of statistical CFG-based parsers
incorporate statistical models that integrate the phrase structure
information and the phrase head and dependency information; the
integration of this information was proved to be effective in
improving parse accuracy
\cite{Carroll92,Eisner96b,Collins99,Charniak00,Bikel04}. Probabilistic
CDG \cite{Wang04} is an example of a single-space model based on the
dependency structure scheme. Preference dependency grammar (PDG)
adopts a multi-space approach and integrates multilevel knowledge
based on the packed shared data structures in multiple interpretation
spaces, as described below.

After providing PDG's basic sentence analysis model and overview,
this paper describes methods for obtaining two main packed shared data
structures of PDG, i.e., the ``headed phrase structure forest'' and
the ``dependency forest,'' and then gives proof of the completeness and
soundness of the dependency forest. This paper also provides an
experiment for analyzing prototypical ambiguous sentences, and
discusses the mapping relations between the phrase structure tree(s),
and the dependency tree(s), as well as the treatment of non-projective
dependency structures in PDG.


\section{Overview of PDG and Packed Shared Data Structure}

\subsection{Multilevel Packed Shared Data Connection Model}

PDG is a framework for the morphological, syntactic, and semantic
analysis of natural language sentences, and is mainly designed for
manipulating ambiguities in sentence interpretations \cite{Hirakawa07}. The targeted
issues are as follows:

\begin{itemize}
\item[(a)] suppression of combinatorial explosion
\item[(b)] proper treatment of generation, preference, and constraint knowledge
\item[(c)] modularity and integration of multilevel knowledge 
\end{itemize}

\begin{figure}[b]
\vspace{-0.7\Cvs}
 \begin{center}
\includegraphics{21-4ia6f2.eps}
 \end{center}
\caption{Multilevel packed shared data connection model}
\label{fig:MultilevelPSDataConnectionModel}
\end{figure}

\noindent
PDG adopts the ``multilevel packed shared data connection
\pagebreak
(MPDC) model'' shown in Figure \ref{fig:MultilevelPSDataConnectionModel}
to achieve these targets. Each analysis level encompasses all possible
sentence interpretations of its interpretation space on packed
shared data structure. The input sentence side is called the lower
level and the output side is called the upper level. Each
interpretation at each level has a link called the ``interpretation
linkage'' with its counterparts in the lower and, possibly, the upper
levels (represented by the dotted line labeled ``correspondence''). In
the MPCD model, the interpretations at the lower level involve the
corresponding interpretations at the upper level. The input sentence
involves all interpretations at all levels.  Each interpretation of a
sentence in some interpretation space should have an interpretation
linkage to its counterpart in the lower adjacent level of the
interpretation space; however, the inverse is not necessarily
true. For example, there can be a morphological interpretation of a
sentence having no corresponding syntactic interpretations, but there
cannot be a syntactic interpretation of a sentence having no
corresponding morphological interpretations. Each level has its
generation, preference, and constraint knowledge. The generation
knowledge generates all possible interpretations from a set of its
lower-level interpretations ($\longmapsto$ arrow in
Figure \ref{fig:MultilevelPSDataConnectionModel}). This is called
externalization. The constraint knowledge restricts or rejects
interpretations in the interpretation space. The preference knowledge
gives the order of preference of the interpretations in the
interpretation space.

PDG aims to achieve the following effects by adopting this model.

\begin{itemize}
\item[(a)] Utilizing multiple types of linguistic data structures, such as phrase and dependency 
\linebreak
structures, to treat morphological, syntactic, and semantic knowledge with high module independence.
\item[(b)] Utilizing packed shared data structures encompassing all ambiguities at each level of sentence analysis to suppress the combinatorial explosion without pruning.
\item[(c)] Interpretation linkage from the input sentence to the
uppermost level interpretations provides a basis for the integration
of multilevel preference knowledge (preference scores).
\item[(d)] Incorporating an optimum tree search method for a packed shared dependency structure with arc co-{\nobreak}occurrence constraints to compute the most plausible interpretation (dependency tree) of a sentence.
\end{itemize}

\noindent
Interpretations not rejected by the constraint knowledge
are called well-formed interpretations of the sentence. The optimum
well-formed interpretation is defined in each interpretation space,
and is computed by the optimum solution procedure on the basis of
preference and constraint knowledge. As described above, the
application determines the requirement level for the NLA system. For
example, machine translation systems between languages in the same
family may require phrase structure output, whereas those between
languages in different families may require dependency structure
output. The interpretation linkage can be used to search the optimum
interpretation, based on upper-level information and not current-level
information. This can be done by tracing back the linkage from the
optimum interpretation at the upper level to the corresponding
interpretation at the current level. For example, the tagger based on
the optimum semantic analysis result is constructed naturally.


\subsection{Implementation Model of the Preference Dependency Grammar}
\label{sec:OverviewPDG}

Figure \ref{fig:PDGAnalysisModel} shows an overview of the implementation
model of PDG. PDG has three basic linguistic levels, i.e., morphology,
syntax, and semantics with the syntax level further divided into two
levels. In total, PDG has four levels of interpretation space,
description scheme, and packed shared data structure. Four packed
shared data structures, the ``WPP trellis,''\footnote{Word pos pair
 (WPP) is a pair of a word and a part of speech (POS). The word
  ``time'' has WPPs such as ``time/n'' and ``time/v.''} ``headed
phrase structure forest,'' ``functional dependency forest,'' and
``semantic dependency forest'' encompass sets of sentence
interpretations, i.e., WPP sequences, phrase structure trees,
functional dependency trees, and semantic dependency trees,
respectively.  In this paper, the headed phrase structure forest/tree
and the functional dependency forest/tree are also called the phrase
structure forest/tree and the dependency forest/tree,
respectively. The processes for obtaining these data structures are
called morphological analysis, syntactic analysis, dependency forest
generation, and semantic expansion, respectively.

\begin{figure}[b]
\includegraphics{21-4ia6f3.eps}
\caption{PDG implementation model}
\label{fig:PDGAnalysisModel}
\vspace*{-0.3\Cvs}
\end{figure}

\begin{figure}[t]
 \begin{center}
\includegraphics{21-4ia6f4.eps}
 \end{center}
\caption{Packed shared data structures in PDG}
\label{fig:PDGPackedSharedDataStructures}
\end{figure}

Figure \ref{fig:PDGPackedSharedDataStructures} briefly explains the data
structures and examples of preference and constraint knowledge at
each level.  WPP trellis is a packed shared data structure
encompassing all WPP sequences for a sentence. Morphological
interpretations for a sentence are represented by \mbox{sequences} (or
strings) of WPP nodes, which represent the adjacency relations between
words.  The headed phrase structure forest encompasses a set of phrase
structure trees that represent the sub-categorization (or adjacency)
relations of phrases. Syntactic preference knowledge (e.g., phrase
frequency) and constraint knowledge (e.g., number agreements) are
described on top of the phrase structure.\footnote{Constraints such as
number agreements can be described as constraints at another
level or can be described in more than one level in parallel. This is
a design issue in actual grammar development. In general, constraint
knowledge should be described at the lowest level possible, because it
will improve the system efficiency by suppressing the generation of
useless interpretations.} The functional dependency structure is
another data structure at the syntax level of PDG. A functional
dependency tree consists of WPP nodes and arcs labeled with functional
dependency relations, such as subject and object. A set of functional
dependency trees representing the syntactic interpretations of a
sentence is represented by a functional dependency forest. The
dependency forest is a packed shared data structure that utilizes a
dependency graph with a framework for describing the preference and
constraint information for the arcs in the graph. Semantic
interpretations for a sentence are represented by semantic dependency
trees that consist of word POS concept triple (WPCT) nodes and arcs
labeled with semantic dependency relations. A semantic dependency
forest is used to represent a set of semantic dependency trees that
show the semantic interpretations of a sentence. In the PDG model,
preference scores provided by preference knowledge at each level are
integrated into the preference scores in the uppermost level data
structure through the interpretation linkage from the input sentence
to the uppermost level interpretations. The most plausible
interpretation (dependency tree) of a sentence is calculated by an
optimum tree search method \cite{Hirakawa05d_e}. Currently, an
experimental syntactic analysis version of the PDG system has been
implemented \cite{Hirakawa07}. This paper focuses on the syntactic
level data structures in the PDG model.

As described above, a characteristic feature of PDG is the integrated
use of phrase and dependency structures as the intermediate syntactic
data structures, which provides the advantage of a multilevel
construction framework. For example, CDG, which adopts a single-level
packed shared data model based on the dependency structure, generates
all possible dependency relations for all word pairs in a sentence and
applies constraints to obtain a set of well-formed interpretations for
a sentence. This is called eliminative parsing. This method has a
superior feature in that it does not eliminate possible
interpretations, but has a demerit in terms of efficiency in that it
produces many useless hypotheses. Harper et al. proposed techniques
for improving CDG, including the interpretation generation method
\cite{Harper99}. However, PDG produces the dependency forest
encompassing all possible dependency trees, which have their
counterparts in the phrase structure forest, for a sentence. This
indicates that generation and constraint knowledge at the phrase
structure forest level, i.e., knowledge based on phrase structure
representation,\footnote{Extended CFG is used in PDG.} can suppress
interpretations at the dependency forest level. Furthermore, as
described in Section \ref{sec:NonProjectivity}, using the description
capability of the phrase structure enables us to incorporate the
controlled non-projectivity in the dependency forest
\cite{Hirakawa06b_e,Hirakawa07}. This indicates that the description
capability of the dependency forest (dependency formalism) is enhanced
using phrase structure formalism.


\subsection{Prerequisites for Packed Shared Data Structures}
\label{sec:PrerequisitesForPackedSharedDataStructure}

The following are prerequisites for the data structure at each level of the MPDC model.

\begin{itemize}
\item[(a)] no combinatorial explosion
\item[(b)] a set of proportionate interpretations
\item[(c)] proper basis for knowledge description
\item[(d)] existence of interpretation linkages between the lower- and upper-level data structures
\end{itemize}

\noindent
(a) is a key issue with regard to constructing practical
NLA systems. Enumerative treatment of interpretations leads to lack of
time and space, or degrades the analytical capability due to
overpruning. (b) implies that the packed shared data structure at each
level encompasses all possible solutions correctly, i.e., it assures
no pruning of existing interpretations and no generation of
nonexistent interpretations originating from the packed shared data
structures. If this requirement is assured, it is beneficial for an
NLA system to be capable of introducing possible pruning (application
of constraint knowledge) in the early stage of sentence analysis from
the viewpoint of system performance. (c) is a prerequisite for the
efficient and precise knowledge development an NLA system. The
interpretation description scheme at each level should be an essential
basis for generation, constraint, and preference knowledge
\cite{Hirakawa02_e}.


\subsection{Traditional Methods for Packed Shared Data Structures}

\subsubsection{WPP Trellis and Packed Shared Phrase Structure Forest}

PDG utilizes WPP trellis as the basis for the morphological analysis
level. WPP trellis is a packed shared data structure encompassing all
WPP sequences for a sentence. The packed shared phrase structure
forest, or simply, phrase structure forest, is a well-known packed
shared data structure for encompassing all phrase structure trees
\cite{Tomita87}. WPP trellis and packed shared phrase structure forest
satisfy the interpretation linkage condition of the MPDC model because
each phrase structure tree in the packed shared phrase structure
forest corresponds to a WPP sequence in WPP trellis.


\subsubsection{Semantic Dependency Graph}
\label{sec:SemDdpGrp}

The ``semantic dependency graph'' (SDG) is a dependency graph
representing ambiguities in word dependencies (structural ambiguities)
and their semantic relations (semantic ambiguities)
\cite{Hirakawa02_e}. SDG is generated from a kakari-uke structure
produced by an ATN parser by expanding kakari-uke relations (dependency
relations) to semantic dependency relations. The SDG method lacks
in terms of generality in that it cannot handle backward dependency
and multiple WPPs because it depends on some linguistic features
peculiar to Japanese. PDG employs a dependency forest instead of an
SDC.


\subsubsection{Syntactic Graph}

Seo proposed the ``syntactic graph,'' which encompasses all dependency
trees corresponding to phrase structure trees in the phrase structure
forest for a sentence \cite{Seo89,Rim90}. The syntactic graph is a
promising candidate for a packed shared data structure in PDG, but
cannot be adopted as it is due to its difficulty in satisfying
prerequisite (d) in Section
\ref{sec:PrerequisitesForPackedSharedDataStructure}.

The syntactic graph is a directed graph that consists of nodes
representing WPPs and labeled arcs representing the syntactic
relations between nodes. It defines a set of dependency trees
(interpretations) for a sentence in combination with the ``exclusion
matrix,'' which represents exclusive co-occurrence relations between
arcs. The syntactic graph is a set of Triples, containing arc name and
two nodes (containing WPP, surface position,
etc.). Figure \ref{fig:KoubunGraph} shows the syntactic graph and
exclusion matrix of the sentence ``Time flies like an arrow.'' The
numbers in arcs are arc-IDs. Multiple arcs targeting one node
represent modification ambiguities,\footnote{The direction of
  dependency arc obeys the convention of the Japanese kakari-uke
  grammar (dependency grammar). The dependent node of an arc is the
  node located at the source of the arc. This is contrary to the
  convention in the syntactic graph, but not substantially
  different.} and S corresponds to the starting symbol.

\begin{figure}[b]
 \begin{center}
\includegraphics{21-4ia6f5.eps}
 \end{center}
\caption{Syntactic graph and exclusion matrix for the example sentence}
\label{fig:KoubunGraph}
\end{figure}

The exclusion matrix is one whose rows and columns are a set of
arcs in the syntactic graph that prescribes the co-occurrence relation
between arcs. Setting ``1'' at position (i, j) of the exclusion
matrix indicates that the i-th and j-th arcs must not co-occur in any
dependency tree (interpretation) obtained from the syntactic graph.

The syntactic graph and exclusion matrix are generated from a kind
of packed shared phrase structure forest. PDG adopts the same data
structure, called a headed phrase structure forest, described in Section
\ref{sec:datastructure}. In the rest of this paper, phrase structure
forest means headed phrase structure forest. The traditional phrase
structure forest (packed shared phrase structure forest) is called a
headless phrase structure forest.

Seo discussed the completeness and soundness of the correspondence
between the phrase structure forest and the syntactic
graph \cite{Seo89}. The completeness is satisfied if each phrase
structure tree in each phrase structure forest has its counterpart(s)
in a syntactic graph. The soundness is satisfied if each dependency
tree in each syntactic graph has its counterpart(s) in a phrase
structure forest. The completeness of a syntactic graph is shown in
\cite{Seo89}, but the soundness is not assured. All exclusion matrix
cells are initially set to 1 (this means no two triples
co-occur). Then, the cells for all triple pairs in the dependency
tree generated from phrase structure trees are set to 0. Since the
exclusion matrix prescribes the co-occurrence relations for all
dependency trees in the dependency graph,\footnote{The constraint in
  the exclusion matrix is, in a sense, global.} the allowance of a
co-occurrence of two triples (set 1 to the cell for two triples) is
safe if and only if the restriction of these two triples is not
necessary for all other interpretations (dependency trees). Appendix~A
shows an example in which the syntactic graph cannot satisfy the
soundness.


\section{Packed Shared Data Structures in PDG}
\label{sec:datastructure}

PDG adopts a phrase structure forest and dependency forest for
the packed shared data structures for phrase structure and dependency
structure representations, respectively.


\subsection{Headed Phrase Structure Forest}

The headed phrase structure forest is a type of packed shared parse forest,
and consists of edges corresponding to rewriting rules in CFG. The
sub-trees, which satisfy the following conditions, are packed and
shared. Sub-trees have
\begin{itemize}
\item[(a)] the same nonterminal symbol (category) 
\item[(b)] the same coverage (phrase boundary)
\item[(c)] the same phrase head\footnote{Phrase head is a WPP in PDG.} (head constituent)
\end{itemize}

\noindent
Conditions (a) and (b) constitute the headless phrase
structure forest \cite{Schiehlen96}. The phrase structure trees in the
headed phrase structure forest have mapping to the phrase structure
trees in the headless phrase structure forest. An example of the edges
and the phrase structure forest in PDG, along with the parsing
algorithm, is shown in Section \ref{sec:construction}.


\subsection{Dependency Forest}

The dependency forest (DF) consists of a ``dependency graph'' (DG) and
a ``co-occurrence matrix'' (CM, C-Matrix), and is expressed as $\mathrm{DF}=\langle\text{DG,CM}\rangle$.


\subsubsection{Dependency Graph and Co-occurrence Matrix}

\begin{figure}[b]
\begin{center}
\includegraphics{21-4ia6f6.eps}
 \end{center}
\caption{Initial dependency forest for the example sentence}
\label{fig:IDF}
\end{figure}

Figure \ref{fig:IDF} shows a DG for the example sentence ``Time flies
like an arrow.'' The DG consists of nodes and directed arcs,
representing WPPs and the dependency relation between nodes,
respectively. An arc has its ID and preference score,\footnote{
  Preference score represents the plausibility of the arc and is used
  for the computation of the optimum interpretation.} which are hidden
in the figure. The dependency graph has one special node, called the
``top node'' (or sometimes called ``root node,'' for convenience),
which is the root of all dependency trees in the DG. In practice, a DG
is represented by a set of ``dependency pieces.'' A dependency piece
consists of one arc and its ``dependent (or modifier) node'' and
``governor (or modificand) node.'' Since the dependency piece and the
arc have a one-to-one correspondence, a dependency piece is referred
to as an arc in this paper. The number of arcs in a DG is called
the ``size of the dependency forest.'' A ``dependency tree'' is a subset
of the DG that forms a tree. Dependency trees represent
interpretations of sentences or phrases at dependency relation level.

A CM is a matrix whose rows and columns are a set of arcs in a DG that
prescribes the co-occurrence relation between arcs. Only when CM(i, j)
is ○, $arc_i$ and $arc_j$ are co-occurable in one dependency
tree. The co-occurrence relation is symmetric, and CM is a symmetric
matrix.


\subsubsection{Well-formed Dependency Tree}

The ``well-formed dependency tree'' is a dependency tree DT in the DF
that satisfies the following conditions, called the ``well-formed
dependency tree constraint.''

\begin{definition} \label{def:WellFormedDepTree} [Well-formed dependency tree]

\noindent
well-formed dependency tree constraint
\begin{itemize}
\item[(a)] Each input word has a corresponding node in the DT ({\bf coverage constraint}).
\item[(b)] No two nodes in the DT occupy the same input position ({\bf single role constraint}).
\item[(c)] Each arc pair in the DT has a co-occurrence relation in the CM ({\bf arc co-occurrence constraint}).
\end{itemize}
\end{definition}

\noindent
(a) and (b) are collectively referred to as the
``covering constraint.'' A dependency tree satisfying the covering
constraint is called a ``well-covered dependency tree.'' A dependency
tree satisfying (c) is called a ``well-co-occurred dependency tree.''
A set of well-formed dependency trees is the set of possible
interpretations for an input sentence. The DF in Figure \ref{fig:IDF} has
four well-formed dependency trees. In PDG, a set of one WPP node is
considered to be a special case of a dependency tree with no arcs, which
satisfies the well-formed dependency tree constraint.


\subsubsection{Initial Dependency Forest and Reduced Dependency Forest}

There can be more than one different-sized DF encompassing the
equivalent set of dependency trees with respect to the degree of arc
sharing. PDG treats the ``initial dependency forest'' and the
``reduced dependency forest,'' which is obtained from the initial
dependency forest. The initial dependency forest consists of the
``initial dependency graph'' and the ``initial C-matrix.'' The reduced
dependency forest is simply called a dependency forest in this
paper. The dependency graph of the initial dependency forest in
Figure \ref{fig:IDF} is different from the syntactic graph in
Figure~\ref{fig:KoubunGraph}, in terms of the number of arcs between
``fly/n'' and ``time/v.''


\clearpage
\section{Generation of the Phrase Structure Forest and the Initial \\\hspace{19pt}Dependency Forest}
\label{sec:construction}

PDG generates a dependency forest from an input sentence through four
processes, i.e., 
\linebreak
morphological analysis, syntactic analysis, phrase
structure/dependency forest generation, and dependency forest
reduction. The explanation of the morphological analysis process is
beyond the scope of this study.


\subsection{Grammar Rules}
\label{sec:bunpoukisoku}

Grammar rules in PDG are extended CFG rules, which define possible
phrase structures and mapping from the phrase structures to the
corresponding dependency structures. Grammar rules are written in the
following format.

y/$Y \rightarrow$ x$_1$/$X_1$, $\ldots$ , x$_n$/$X_n$ : [arc($arcname_1$, $X_i$, $X_j$), $\ldots$ , arc($arcname_{n-1}$, $X_k$, $X_l$)] \\
\quad (0$<i$, $j$, $k$, $l{\leq}n$)

e.g., vp/$V \rightarrow$ v$/V$, np$/NP$, pp$/PP$ : [arc(obj, $NP$, $V$), arc(vpp, $PP$, $V$)]

\noindent
A grammar rule consists of two parts separated by ``:'',
the rewriting rule part and the dependency structure part. The left
side of the rewriting rule ``y/$Y$'' and constituent ``x$_i$/$X_i$''
mean ``syntactic category/variable.'' $Y$ is a head constituent,
called the ``phrase head,'' and is the same as one of the variables
``$X_1 {\ldots} X_n$'' in the ``rule body.'' The dependency structure
part is a set of arcs in the form ``arc(arcname, variable1,
variable2).''\footnote{The dependency structure is a set of arcs, but
is represented by a list format using [ ]. In this paper, sets are
sometimes represented by [ ] in program codes.} A variable is bound
to a WPP node, which is a phrase head of a constituent in the
rewriting rule. In the example above, the dependency structure shows where
dependents $NP$ and $PP$ are connected to the governor phrase head $V$
by means of the obj arc and the vpp arc, respectively. The dependency
structure part constitutes a partial dependency tree, which satisfies
the following ``partial dependency structure conditions.''

\begin{definition} \label{def:DSCondition}[Partial dependency structure condition]

\begin{itemize}
\item[(a)] A partial dependency structure constitutes a tree structure whose top node is a phrase head of the head constituent $Y$. Phrase heads of non-head constituents are the dependents of phrase heads of other constituents.
\item[(b)] The phrase heads of the constituents in the rule body have one-to-one correspondence with variables in the partial dependency structure.
\end{itemize}
\end{definition}

Figure \ref{fig:ExampleGrammar} shows the grammar rules and lexicons for
analyzing the example sentence: ``Time flies like an arrow.'' Rule (R0),
whose rule head and rule body are ``root'' (predefined special symbol)
and ``s''(starting symbol) as rule body, is a special rule for creating
a ``root edge'' of the phrase structure forest and a ``top node''
[root]-x of the dependency forest.\footnote{This rule is introduced for
  convenience in the treatment of data structures.}

\begin{figure}[t]
 \begin{center}
\includegraphics{21-4ia6f7.eps}
 \end{center}
\caption{Grammar and lexicon for the example sentence}
\label{fig:ExampleGrammar}
\end{figure}


\subsection{Parsing Algorithm}
\label{sec:ParsingAlgorithm}

The syntactic analysis of PDG is implemented by extending the
bottom-up chart-parsing algorithm to generate a dependency structure.


\subsubsection{The Structure of Edge}
\label{sec:EdgeConstruction}

An ordinal chart parser utilizes edges composed of five elements
$\langle\text{FP, TP, C, FCS, RCS}\rangle$, i.e., the from-position (FP), the
to-position (TP), the category (C), the found constituent sequence
(FCS), and the remaining constituent sequence (RCS). The head of the
grammar rule corresponds to C. The body of the grammar rule
corresponds to both FCS and RCS and is partitioned by a dot (・),
which shows the boundary of FCS and RCS, as shown in the following edge
written in diagrammatic form.

$\langle \text{0, 1, s} \rightarrow \text{np ・ vp pp} \rangle$

\noindent
This edge is generated from the grammar rule ``s
$\rightarrow$ np vp pp'' and has elements FP=0, TP=1, C=s, FCS=[np], and
RCS=[vp, pp]. The result of the dictionary look-up for an input word is
an inactive edge whose category is the POS of the word, and whose found
constituent sequence is a word list, as follows:

$\langle \text{0, 1, n} \rightarrow \text{[time]・} \rangle$

\noindent
The parsing algorithm of PDG has two extensions, i.e., the treatment for the dependency
structure part in a grammar rule and the construction of the packed
shared data structure. The edge for PDG parsing has two additional
elements, i.e., the phrase head (PH) and the dependency structure (DS),
as follows:

Standard edge: $\langle \text{0, 1, s} \rightarrow \text{np ・ vp pp}\rangle $

PDG edge: $\langle \text{0, 1, s/PH} \rightarrow \text{np/n1 ・ vp/PH pp/PP : DS} \rangle $

\noindent
As described in Section \ref{sec:bunpoukisoku}, PH and DS
represent a phrase head (node) and dependency structure (a set of
arcs), respectively. The symbol n1 shows a node (WPP) that is the head
of the np phrase. PDG utilizes another data structure called ``packed
edge,'' which is obtained by packing inactive edges into one. The
packed edge has the list of FCS and of DS, instead of the FCS
and the DS in the PDG edge. The PDG edge with FSC and DS is called
``single edge,'' in contrast to packed edge. The packed edge is
equivalent to a set of single edges. The following shows the relation
between single edge and packed edge, diagrammatically.

Single edge: $\langle \text{0, 5, s/n2} \rightarrow \text{np/n1 vp/n2 pp/n3・: DS1}\rangle $

$\phantom{\text{Single edge: }}\langle \text{0, 5, s/n2} \rightarrow \text{np/n1 vp/n2 ・: DS2} \rangle $

Packed edge: $\langle \text{0, 5, s/n2} \rightarrow \text{[[np/n1 vp/n2 pp/n3], [np/n1 vp/n2]] ・: [DS1, DS2]}\rangle $

\noindent
n1 to n3 are nodes (WPPs) and n2 is a phrase head. [np/n1
  vp/n2 pp/n3] and [np/n1 vp/n2] are constituent sequences with their
phrase head (nodes). DS1 and DS2 are dependency structures (partial
dependency trees). For convenience, a packed edge is represented in
the form ``E,'' ``$\langle \mathrm{E} \ldots \rangle$,'' or ``edge E,'' and a single
edge is represented as ``e,'' ``$\langle \mathrm{e} \ldots \rangle$,'' or ``edge e.''
When it is not ambiguous, ``edge'' is used for representing ``packed
edge'' or ``single edge.'' Inactive edges are represented by adding
``*'' at the top of the edge symbol. Edges *E and *e are an inactive packed
edge and an inactive single edge, respectively.

The syntactic parsing of PDG described below utilizes packed
edges. Figure \ref{fig:ArcStructure} shows the formal constitution of a
packed edge. A packed edge consists of eight elements. FCSL and DSL
are lists (or sequences) with the same length. The pair
(FCS$_i$, DS$_i$) obtained by extracting the i-th elements of FCSL and
DSL is called the ``CSDS pair.'' The CSDS pair corresponds to the single edge
described above.

The edges E1 to *E3 in Figure \ref{fig:ArcStructure} show the growth of
the edge generated from a grammar rule for a noun phrase. Edge *E3 is
the inactive edge showing the interpretation that the input words ``an
arrow'' constitute a noun phrase, and its dependency structure is
\{arc(det-14, [an]-det-3, [arrow]-n-4)\}.  The complex expression
[arrow]-n-4 consists of three parts: the first part indicates a node
for the word ``arrow'', the second pos ``n'', and the third word
position 4. Edge *E4 is an edge with more than one
interpretation. Each of the two elements in FCSL and DSL have
correspondence and constitute two CSDS pairs, i.e., ([103, 169]
\{obj-25\})\footnote{The obj-25 is the abbreviation of arc(obj-25,
  [flies]-n-1, [time]-v-0).} and ([103, 119, 165] \{obj-4,
vpp-20\}). Edge @E5 is an example of an edge generated from
a dictionary look-up operation, called ``lexical edge.'' The data
structure of a lexical edge is a set of one node corresponding to the
consulted word. The lexical edge is explicitly represented by adding
``@.''

\begin{figure}[t]
 \begin{center}
\includegraphics{21-4ia6f8.eps}
 \end{center}
\caption{Structure of edge}
\label{fig:ArcStructure}
\end{figure}


\subsubsection{Parsing Algorithm}

Figure \ref{fig:ChartAlgotithm} shows the parsing algorithm of PDG, which
is basically a standard, bottom-up, chart-parsing algorithm using an
agenda \cite{Winograd83}. This algorithm inputs words from left to
right, one by one, adds lexical edges generated from the input words
to the agenda (Figure \ref{fig:ChartAlgotithm} (a), (b)), and combines
the edges in the agenda with inactive edges in the chart or in the
grammar rules until the agenda becomes empty
(Figure \ref{fig:ChartAlgotithm} (e), (f)). Packed edges are generated by
checking whether each edge in the agenda is mergeable ((c), (j)), and
then merging it to the existing edge if possible. A detailed
explanation of the algorithm is omitted. The following part explains
the construction of the data structure, which is peculiar to PDG
parsing.

\begin{figure}[t]
 \begin{center}
\includegraphics{21-4ia6f9.eps}
 \end{center}
\caption{PDG bottom-up chart parsing algorithm}
\label{fig:ChartAlgotithm}
\vspace{-0.5\Cvs}
\end{figure}

The PDG parser creates dependency structures in parallel with edge
generation. This is done by binding variables in the dependency
structures in an edge. Variable binding is performed by $bind\_var$,
which binds the phrase head (node) of the inactive edge to the
variable in the first constituent of the remaining constituent
sequence of the edge, when a new edge is created, either from a
grammar rule (Figure \ref{fig:ChartAlgotithm} (g)) or an active edge in
the agenda (Figure \ref{fig:ChartAlgotithm} (h)). If this binding
generates an arc whose dependent and governor are bound, $add\_arcid$
generates a unique arc-ID and is attached to the arc
(Figure \ref{fig:ChartAlgotithm} (i)). The arc with a bound dependent and
governor is called a ``fixed arc.'' Edge *E3 in
Figure \ref{fig:ArcStructure} is generated by binding the variable \$2 in
E2 to the node [arrow]-n-4.

Edges are associated through edge-IDs. The lower edge can be traced
from the upper edge. Edge *E3 (edge\#160) in
Figure \ref{fig:ArcStructure} is an edge generated from the grammar rule
``np $\rightarrow$ det n.'' The edges in the constituent sequence
[153, 156] in edge\#160, i.e., edge\#153 and edge\#156, have the
phrase category ``det'' and ``n,'' respectively. Edge\#153 and
edge\#156 are called reachable from edge\#160. This ``reachable''
relation is associative. Edges with more than one CSDS pair, like *E4
in Figure \ref{fig:ArcStructure}, are generated by
$merge\_csds$(Figure \ref{fig:ChartAlgotithm} (d)). Since only inactive
edges are merged, no active edge has more than one CSDS pair in this
algorithm. If the whole sentence is parsed successfully, the chart has
one inactive edge with phrase head [root]-x covering the whole
sentence. This edge is called ``root edge,'' and is described as
*E$_{root}$.


\subsection{Generation of Phrase Structure Forest and Initial Dependency \mbox{Forest}}
\label{sec:PFandDFseisei}

\begin{figure}[b]
 \begin{center}
\includegraphics{21-4ia6f10.eps}
 \end{center}
\caption{Algorithm for computing phrase structure forest and initial dependency forest}
\label{fig:HPF_IDF_Algorithm}
\end{figure}

When parsing is finished, the chart has active and inactive edges. The
phrase structure forest for an inactive edge *E, hpf(*E), is defined as a
set of edges reachable from edge *E. The phrase structure forest PF is
defined as hpf(*E$_\mathit{root}$). PF is a subset of the inactive edges in
the chart, because there are inactive edges that are unreachable from
*E$_\mathit{root}$. The initial dependency graph IDG is a set of arcs in the
dependency structures of the edges in
PF. Figure \ref{fig:HPF_IDF_Algorithm} shows the algorithm to compute PF,
IDG, and the initial C-Matrix (ICM) from
*E$_\mathit{root}$. Figure \ref{fig:PDGParseForest} shows PF for the example
sentence computed by the algorithm using the grammar in
Figure \ref{fig:ExampleGrammar}. All RCSLs of the edges in PF are [~], and
are not shown in Figure \ref{fig:PDGParseForest}. The number of edges in the
phrase structure forest is called the ``size of the phrase structure forest.'' The size of the
headed phrase structure forest is more than or equal to that of the headless
phrase structure forest, because the edge merge condition of the headed phrase structure
forest (Figure \ref{fig:ChartAlgotithm} (j)) is more strict compared with
that of the headless phrase structure forest.

\begin{figure}[b]
 \begin{center}
\includegraphics{21-4ia6f11.eps}
 \end{center}
\caption{Phrase structure forest for ``time flies like an arrow''}
\label{fig:PDGParseForest}
\end{figure}

The algorithm in Figure \ref{fig:HPF_IDF_Algorithm} traverses the chart
by using three mutually recursive functions, \textit{try\_edge}, \textit{try\_FCSL}, and
\textit{try\_CS} which compute PF, IDG, and ICM for their arguments, i.e., the
packed edge, the constituent sequence list, and the constituent,
respectively. \textit{try\_edge} calls \textit{try\_FCSL}
(Figure \ref{fig:HPF_IDF_Algorithm} (d)), \textit{try\_FCSL} calls \textit{try\_CS}
(Figure \ref{fig:HPF_IDF_Algorithm} (h)), and \textit{try\_CS} calls \textit{try\_edge}
\linebreak
(Figure~\ref{fig:HPF_IDF_Algorithm} (j)). The arc in the arc sets
returned from \textit{try\_edge(E)}, \textit{try\_FCSL(FCSL)}, and \textit{try\_CS(CS)} are called
the arc governed by $E$, \textit{FCSL}, and \textit{CS}, respectively.

The algorithm starts from Figure \ref{fig:HPF_IDF_Algorithm} (a) by
calling \textit{try\_edge}(*E$_\mathit{root}$). \textit{try\_edge} judges whether the
argument is already computed at (b). If it has already been computed,
\textit{try\_edge} simply returns the set of arcs recorded in \textit{TER}. The
registration of a set of arcs is performed in (g) when a new result is
obtained. At (c) and (e), new edges are added to \textit{PF}. As shown at
(f), the arcs governed by edge E are the union of the DSL in E and
those governed by FCSL.

\textit{try\_FCSL} processes a set of CSDS pairs, and \textit{try\_CS} processes one CSDS
pair in it. As shown at (i), the set of arcs governed by FCSL is the
union of the arcs governed by the CSs in the FCSL. As shown at (k),
the set of arcs governed by CS is the union of the arcs governed by
the packed edges in the CS.

Figure \ref{fig:TryEdgeFunctionExecution} shows the execution process of
E\#170 in Figure \ref{fig:PDGParseForest}. The (c\#) shows a function
call and (r\#) shows the return value, i.e., the set of arcs of the
function, and (c1) to (c4) correspond to the function calls (j), (d),
and (h) in Figure \ref{fig:HPF_IDF_Algorithm}, respectively. The (c4)
returns \{\} at (r4) because E\#103 is a lexical edge. The function
call (c4) returns result (r4). Then, the second CSDS pair ([103, 119,
  165], \{obj-4, vpp-20\}) is processed by function call (c5). The
second time execution for ``\textit{try\_edge}(E\#103)'' occurs at
(c6). This time, the execution result stored in \textit{TER} at
Figure \ref{fig:HPF_IDF_Algorithm} (b) is searched and returned. Finally,
the set of arcs at (r1) is obtained.

\begin{figure}[b]
\vspace{-0.5\Cvs}
 \begin{center}
\includegraphics{21-4ia6f12.eps}
 \end{center}
\caption{Example of algorithm execution}
\label{fig:TryEdgeFunctionExecution}
\end{figure}

The generation of the CM is performed on the basis of the following C-Matrix
setting conditions, which work to allow co-occurrence between all arcs
in the set of edges constituting a phrase structure tree in the phrase
structure forest.

\begin{definition} \label{def:CoocCondition} [C-Matrix setting conditions] 

The ``C-Matrix setting condition'' is any of the following three conditions:

\noindent
(CM1) The arcs in the same DS co-occur with each other.\\
(CM2) Given a CSDS pair (CS, DS), the arcs in the DS co-occur with those governed by the CS.\\
(CM3) The arcs governed by one CS co-occur with one another.

\end{definition}

\noindent
(CM1) to (CM3) correspond to the CM processing (1) to (3)
in Figure \ref{fig:HPF_IDF_Algorithm}. In processing E\#170,
\textit{set\_CM}(\{obj-4, vpp-20\}, \{obj-4, vpp-20\}) (\textit{set\_CM} is
defined in Figure \ref{fig:HPF_IDF_Algorithm}) is executed at the CM
processing (1) because the arcs in the second CSDS pair ([103, 119,
  165], \{obj-4, vpp-20\}) satisfy the co-occurrence setting condition
(CM1). At CM processing (2), $A\_CS$ has been set to the set of arcs
shown at (r5) in Figure \ref{fig:TryEdgeFunctionExecution} and
\textit{set\_CM}(\{obj-4, vpp-20\}, \{pre-15, det-14\}) is executed, due to
(CM2). In processing \textit{try\_CS}([103, 119, 165]), the CM processing
(3) set CM from among the arcs governed by E\#103, E\#119, and E\#165,
due to (CM3). The outputs PF and IDF are shown in
Figure \ref{fig:PDGParseForest} and \ref{fig:IDF},
respectively. E\#181, E\#176, and E\#174 have the same category ``s''
and the same coverage (from 0 to 5), but are not shared because their
phrase heads are different.


\subsection{Generation of the Reduced Dependency Forest}

The two arcs obj4 and obj25 in the IDF in Figure \ref{fig:IDF} have the
same structure except for their arc-IDs. IDFs may contain arcs of this
kind, called ``equivalent arcs,'' which are sometimes generated, from
one grammar rule and sometimes from different grammar rules. For
example, obj4 and obj25 are generated from the constituent sequence
``vp np'' in (R9) and (R10) in Figure~\ref{fig:ExampleGrammar}. In fact,
(R9) and (R10) differ in terms of the existence of the prepositional
phrase, but the interpretation of the ``obj'' relations in (R9) and
(R10) is considered to be the same.\footnote{(R9) and (R10) are merged
into one grammar rule vp/V$\rightarrow$ v/V, np/\textit{NP},
{pp/\textit{PP}}:[arc(obj, \textit{NP}, V),arc(vpp, \textit{PP}, V)] if the description
schema \{\} for optional elements is introduced. This grammar rule
does not generate the equivalent arcs for obj relation.}

Here, some definitions are given for treating equivalent arcs. The
``generalized arc'' is an arc with arc-ID `?'. Arcs with number IDs
are called IDed arcs. The generalized arc for an IDed arc is obtained
by simply replacing the arc-ID in the IDed arc with `?'. A dependency
tree consisting of generalized arcs is called a ``generalized
dependency tree.'' A dependency tree consisting of IDed arcs is called
an ``IDed dependency tree.'' The generalized arc for an IDed arc X is
written as ?X. The generalized tree for an IDed tree DT is written as
?DT. Two dependency trees that have the same generalized tree are
called ``equivalent.''

The reduced dependency forest is obtained by reducing the initial
dependency forest. The reduction of the dependency forest is an
operation in that more than one equivalent arc is merged into another arc
without increasing the number of generalized dependency trees in
the dependency forests. The reduced dependency has a smaller size,
compared with the original dependency forest before the merge
operation.


\subsubsection{Merge Operation for Equivalent Arcs}

The merge operation for the equivalent arcs $X$, $Y$ (written in
equiv($X$, $Y$)) is defined as follows:

\begin{definition}\label{def:ArcMrgOpe}[Arc merge operation]

\begin{itemize}
\item[(1)] Compute a new dependency graph DG' by removing $Y$ from DG. (DG'=DG$-\{Y\}$)
\item[(2)] Generate a new C-Matrix CM' from CM by applying $set\_CM(X, I)$ for arc $I$($I \in \mathrm{DG}$, $I \neq X$, $I \neq Y$, CM($Y$, $I$)=○)
\end{itemize}
\end{definition}

\begin{figure}[b]
 \begin{center}
\includegraphics{21-4ia6f13.eps}
 \end{center}
\caption{Merge operation for the equivalent arc pair (X, Y)}
\label{fig:ArcMerge}
\end{figure}

The merge operation generates a new dependency forest $\langle \text{DG', CM'}\rangle$. 
Figure \ref{fig:ArcMerge} shows a diagrammatic example of a merge
operation. In the following sections, changes in various values are
discussed; for example, the number of generalized dependency trees in
the dependency forest before and after the merge operation. To make
this distinction, the expression ``wrt $\langle \text{DG, CM}\rangle $'' or ``wrt DF''
(wrt: with respect to) is used. For example, the set of arcs that
co-occurs with an arc A is defined as co(A). Then, ``co($A$) wrt
$\langle \text{DG, CM}\rangle$'' and ``co($A$) wrt $\langle \text{DG', CM'}\rangle$'' represent the set of
arcs before and after the merge operation. ``co($A$) wrt $\langle \text{DG, CM}\rangle 
= \mathrm{co}(A)$ wrt $\langle \text{DG', CM'}\rangle$'' means the set of arcs is not changed by
the merge operation. In order to simplify the description, ``wrt
$\langle \text{DG, CM}\rangle$'' is not shown in default. Merging all equivalent arcs
in CM\footnote{This is equivalent to sharing all equivalent arcs
  during the parsing process} makes the structure equivalent to a
syntactic graph. This operation possibly impairs the soundness of the
data structure.


\subsubsection{Merge Condition for Equivalent Arcs}

From the definition, the condition of the dependency forest reduction is to preserve the soundness, i.e., no new generalized dependency tree (interpretation) increase by the merge operation.

\noindent
\textbf{[Merge Condition for Equivalent Arcs]}
\begin{quote}
When the dependency forest DF' is generated from DF by
merging arc Y with X in the dependency graph of DF, following is the condition for
forest reduction: 

``a set of generalized dependency trees in DF = a set of generalized
dependency trees in DF'.''
\end{quote}

\noindent
This condition is verifiable by searching a new
generalized dependency tree in DF'. The condition for the existence of
a generalized dependency tree not in DF but in DF' is called an
``increase condition for generalized dependency trees'' (ICG) in this
paper. The merge condition is represented as ``ICG is not
satisfied.''

A dependency forest is a set of IDed arcs, and prescribes the set of
IDed dependency trees and that of generalized dependency trees. The
condition for the existence of an IDed dependency tree not in DF but
in DF' (this kind of tree is called a ``new'' dependency tree) is
called an ``increase condition for IDed dependency trees''
(ICI). Obviously, if no IDed dependency tree increases by the merge
operation, no generalized dependency tree increases. Moreover, even if
there exist new IDed dependency trees in DF', no generalized
dependency trees increase when the generalized dependency trees of the
new IDed dependency trees are equivalent to those in DF. This indicates
that ICI is a necessary condition for ICG. In the following, ICI is
discussed first, and then ICG is verified to obtain a detailed merge
condition for equivalent arcs.


\subsubsection{Increase Condition for IDed Dependency Trees}
\label{subsec:IDDTZoukaJuoken}

Increase of a new IDed dependency tree is caused by allowing a new
co-occurrence relation between arcs caused by a merge of equivalent
arcs. The allowance of the co-occurrence relation in CM, i.e., the
change from CM(U, V)${\neq}$○ to CM(U, V)$=$○ for arcs U, V, is
called an ``allowance of the arc pair (U, V).'' The following lemma is
established.

\begin{lemma}[The allowance of arc pair and the increase of dependency trees]
\label{lem:ArcPairAndNewTree}
If a 
\linebreak
new well-formed dependency tree increases by the allowance of the arc pair (U, V), the dependency tree includes U and V.

\proof{This lemma is obvious because well-formed dependency trees in DG', which include neither U nor V, exist in DG.\\
}
\end{lemma}

\noindent
Here, uniq and diff are the sets of arcs defined for the
equivalent arcs X and Y, as follows:

uniq(X, Y)=\{$I$ ${\mid}$ CM(X, $I$)=○, CM(Y, $I$)${\neq}$○, $I{\in}$DG\}

diff(X, Y)=\{$(I, J) \mid I \in$ uniq(X, Y), $J \in$ uniq(Y, X)\}

\noindent
For arcs X, Y in Figure \ref{fig:ArcMerge},
uniq(X, Y)=\{j, n\}, uniq(Y, X)=\{k\}, and diff(X, Y)=\{(j, k), (n, k)\}. The
following lemma is established.

\begin{lemma} [Arcs in a new well-formed dependency tree]
\label{lem:ArcsInNewTree}
In the case that a well-formed dependency tree is generated by the
merge of equivalent arcs X, Y, the new tree includes at least two
arcs A, B, such that (A, B) ${\in}$ diff(X, Y).

\proof{
Let DF and DF' be the dependency forests before and after the merge of
X and Y. Assume that a new dependency tree DT$_x$ is obtained by the
allowance of the arc pair (X, B$_i$) caused by the merge of X and Y.
X and B$_i$ are included in DT$_x$, according to Lemma
\ref{lem:ArcPairAndNewTree}. Here, let
R$=$DT$_x-$\{X, B$_i$\}. CM(X, U)$=$○ wrt DF', CM(B$_i$, U)=○ wrt DF'
for U${\in}$R, because DT$_x$ is a well-formed dependency tree.

Assuming that there is no arc U such that CM(Y, U)${\neq}$○ wrt DF
(i.e., CM(Y, U)$=$○ wrt DF, U${\in}$R), DT$_y$=\{Y, B$_i$\}+R is a
well-formed dependency tree in DF. DT$_x$ is not a new generalized
dependency tree because DT$_x$ and DT$_y$ differ only in equivalent
arc X and Y, i.e., DT$_x$ and DT$_y$ are equivalent. Therefore, DT$_x$
must include at least one U$_i$, such that CM(Y, U$_i$)${\neq}$○ wrt
DF is a new generalized dependency tree. This lemma is established
because (B$_i$, U$_i$)${\in}$diff(X, Y).   }
\end{lemma}

\noindent
The following theorem is derived from Lemma \ref{lem:ArcsInNewTree}.

\begin{theorem}[The increase condition for IDed dependency tree]
\label{the:IDedDepTreeIncreaseCond}
Let arc pair (A, B)${\in}$diff(X, Y) for equivalent arcs X, Y in DG of
the dependency forest $\langle \text{DG, CM}\rangle$. The IDed dependency trees
increase if and only if $\langle \text{DG', CM'}\rangle$ obtained by the merge of Y to
X has an IDed dependency tree NDT that includes \{X, A, B\}.

\proof{This theorem is proved by showing that a new well-formed IDed
  dependency tree includes \{X, A, B\} and a well-formed IDed
  dependency tree that includes \{X, A, B\} is a new well-formed IDed
  dependency tree. Assuming that NDT is a new IDed dependency tree
  existing in $\langle \text{DG', CM'}\rangle$, there exists at least one arc pair
  (A$_i$, B$_i$)${\in}$diff(X, Y), A$_i \in$NDT,
  B$_i\in$NDT. On the other hand, X${\in}$NDT is true, due to Lemma
  \ref{lem:ArcPairAndNewTree}. Therefore, a new well-formed IDed
  dependency tree includes \{X, A, B\}. Moreover, no IDed well-formed
  dependency trees exist in $\langle \text{DG, CM}\rangle$ because (A, B)${\in}$diff(X,
  Y). Therefore, an IDed well-formed dependency tree that includes
  \{X, A, B\} is a new well-formed IDed dependency tree.

}
\end{theorem}

\noindent
Some functions and notations are introduced for discussion on the increase condition of dependency trees.

\vspace{0.5\Cvs}
\noindent
\fbox{
\begin{minipage}{405pt}
\hangafter=1\hangindent=2.5zw\noindent
same\_position($U$, $V$): The positions of dependent nodes of $U$ and $V$ are the same.
\par
\hangafter=1\hangindent=2.5zw\noindent
dts($S$) wrt $\langle\mathit{DG}, \mathit{CM}\rangle$ : a set of IDed well-formed dependency trees that consist of arcs in arc set $S{\subset}DG$ and satisfy the arc co-occurrence constraint
\par
\hangafter=1\hangindent=2.5zw\noindent
co($U$) wrt $\langle\mathit{DG}, \mathit{CM}\rangle$ : a set of arcs that co-occur with arc U including U, i.e., \{$X$ ${\mid}$ $X$=$U$ or CM($X$, $U$)=○, $X{\in}DG$\}
\par
\hangafter=1\hangindent=2.5zw\noindent
dts\_with\_arcs($A_1$, $A_2$, ${\ldots}$ , $A_n$) wrt $\langle\mathit{DG}, \mathit{CM}\rangle$ : a set of well-formed dependency trees in $\langle\mathit{DG}, \mathit{CM}\rangle$ that include arcs $A_1$, $A_2$, ${\ldots}$ , $A_n$, i.e., dts(co($A_1$)${\cup} {\cdots} {\cup}$co($A_n$)) wrt $\langle\mathit{DG}, \mathit{CM}\rangle$
\par
\end{minipage}}
\vspace{0.5\Cvs}

\noindent
ICI with respect to the arc pair (A, B) ${\in}$ diff(X, Y),
equiv(X, Y) can be checked by searching the existence of a well-formed
dependency tree including \{X, A, B\} in $\langle\text{DG', CM'}\rangle$, according to
Theorem \ref{the:IDedDepTreeIncreaseCond}. To make this search process
more efficient, the following three cases with respect to arcs X, A, and
B are considered.

\begin{itemize}
\item[(RC1)] Any of same\_position(A, B), same\_position(X, A), or same\_position(X, B) is true
\item[(RC2)] CM(A, B)${\neq}$○ is true
\item[(RC3)] except for (RC1) and (RC2)
\end{itemize}

\noindent
In cases (RC1) and (RC2), no well-formed dependency trees that
include \{X, A, B\} exist in $\langle\text{DG', CM'}\rangle$ because of the existence of
the single role constraint and the co-occurrence constraint,
respectively. In the case of (RC3), the existence of a well-formed
dependency tree that includes \{X, A, B\}, i.e., dts\_with\_arcs(X, A, B)
wrt $\langle\text{DG', CM'}\rangle$=\{\}, should be checked for ICI.


\subsubsection{Increase Condition for Generalized Dependency Trees}

As described above, ICI is a necessary condition for ICG. Therefore, ICG is defined as follows:

\noindent
\textbf{[The increase condition for generalized dependency tree]}

\begin{quote}
Let DF' be a dependency forest generated from dependency forest DF by
merging arc Y to X, where X and Y are equivalent arcs, and let
DT$_{new}$ be the set of IDed dependency trees not in DF but in
DF'. There exists at least one IDed dependency tree DT 
${\in}$
DT$_{new}$, such that the generalized dependency tree ?DT is not
included in DF.
\end{quote}

\noindent
The merge condition for equivalent arcs is the negation of ICG for the arcs.


\subsubsection{Dependency Forest Reduction Algorithm}

\begin{figure}[b]
 \begin{center}
\includegraphics{21-4ia6f14.eps}
 \end{center}
\caption{Algorithm for reduction of dependency forest}
\label{fig:DFReductionAlgorithm}
\end{figure}

Figure \ref{fig:DFReductionAlgorithm} shows the dependency forest
reduction algorithm for $\langle\text{DG, CM}\rangle$, based on the merge condition for
equivalent arcs (i.e., the condition for forest reduction) described
in the previous section. In this algorithm, CM is represented as a set
of co-occurable arc pairs. Arcs X and Y are co-occurable if $\langle\text{X,Y}\rangle \in \mathrm{CM}$.

Figure \ref{fig:DFReductionAlgorithm} (a) selects a pair of equivalent
arcs X, Y in DG, and the part from (b) to (h) checks if a new
generalized dependency tree is generated in the dependency forest when
arc pair (A, B) is allowed. If all arc pairs in \textit{diff}(X, Y) fail to
generate any new generalized dependency tree, the forest reduction is
performed at (i).

The availability of the allowance of (A, B) is determined by checking
ICG after checking ICI. At Figure \ref{fig:DFReductionAlgorithm} (b), the
conditions (RC1) and (RC2) in Section \ref{subsec:IDDTZoukaJuoken} are
checked. If either of the conditions is satisfied, the processing
proceeds to the next arc pair in \textit{diff}(X, Y) because the allowance
of (A, B) generates no new IDed dependency trees. If not, the
processing proceeds to the check of ICI. At (c), $\langle\text{DG', CM'}\rangle$ is
generated by merging Y to X. Since the existence check of a new IDed
tree is basically performed by tree search for $\langle\text{DG', CM'}\rangle$, the
reduction of search space improves the efficiency. Based on Theorem
\ref{the:IDedDepTreeIncreaseCond}, the search space is reduced from
DG' to $DG\_XAB$(=$co(X){\cap}co(A){\cap}co(B)$) at (d). Then,
\textit{search\_dt} at (e) searches a new IDed dependency tree for
DG\_XAB. If no dependency tree is obtained, the allowance of (A, B)
satisfies the merge condition for equivalent arcs.  If a dependency
tree DT is obtained, DT is an IDed dependency tree that includes \{X,
A, B\}. \textit{new\_generalized\_dt}(DT, CM, DG) at (f) checks whether the
generalized dependency tree ?DT is new by searching ?DT for $\langle\text{DG,CM}\rangle$. 
The detailed explanation of \textit{new\_generalized\_dt} is omitted
here, but it realizes the search for the generalized dependency tree
by limiting the arc set DG\_X so that it has only the equivalent arcs
of the arcs in DT by \textit{add\_equiv\_arcs} at (q). When ?DT is a new
generalized dependency tree, the merging of X and Y is not
available. Processing for arcs X and Y is terminated by (g) and (h),
and proceeds to the check of the next equivalent arc pair. If ?DT is
not new, the merging of X and Y, i.e., the forest reduction, is
performed at (i). Furthermore, when ?DT is proved to be not new at
(f), the search of other dependency trees for DG\_XAB is performed at
(e). \textit{search\_dt} searches a dependency tree that satisfies the
co-occurrence condition in depth-first manner with respect to input
position P. \textit{search\_dt} selects one arc from the arc set
\textit{arcs\_at}(DG, P) that is a set of arcs with position
P. \textit{search\_dt} searches all possible dependency trees by selecting
another arc at (k) when there are no solutions for arcs, from $P+1$ to
the end position at (m).


\subsubsection{Execution Example of the Dependency Forest Reduction Algorithm}

This section explains the execution process of the algorithm in
Figure \ref{fig:DFReductionAlgorithm} for the example sentence ``Tokyo
taxi driver call center'' in Appendix~A. The
reduced dependency forest for this example has equivalent
arcs. Figure \ref{fig:IDFandRDFexample} (a) shows the initial dependency
forest for the example sentence. It has four sets of equivalent arcs,
(1, 2), (5, 7), (13, 15), (25, 26, 27), which are surrounded with double
lines.

\begin{figure}[t]
 \begin{center}
\includegraphics{21-4ia6f15.eps}
 \end{center}
\caption{Initial dependency forest and its reduction}
\label{fig:IDFandRDFexample}
\end{figure}

The forest reduction is performed along with the algorithm in
Figure \ref{fig:DFReductionAlgorithm}. The first equivalent arc pair (1,
2) is selected to set $X=1$, $Y=2$. \textit{diff}($X$, $Y$) is computed as
\{(5, 14), (5, 15), (5, 27), ${\ldots}$ \} by combining the elements
in $\mathit{uniq}(X, Y)=\{5, 24, 25\}$ and $\mathit{uniq}(Y, X)=\{14, 15, 27\}$. The
first arc pair (5, 14) is skipped by the condition check for (RC1) at
Figure \ref{fig:DFReductionAlgorithm} (b) because $\mathit{same\_position}(5,
14)$ is true. The second arc pair (5, 15) is tried. (5, 15) does not
match the conditions at (b), so CM' and DG' are generated at
(c). $\mathrm{CM'}=\mathrm{CM}+\{\langle\text{1, 14}\rangle$, $\langle\text{1, 15}\rangle$, $\langle\text{1, 27}\rangle\}$, as shown in
Figure \ref{fig:IDFandRDFexample} (b). Then, $DG\_XAB$ is computed at
(d). $X=1$, $A=5$, and $B=15$ result in
$\mathit{DG\_XAB}=\mathrm{co}(1) \cap \mathrm{co}(5) \cap \mathrm{co}(15)$ wrt $\langle\text{DG', CM'}\rangle=\{\text{1,
28}\}$. The allowance of (5, 15) generates no new dependency trees
because the search of the dependency tree for $DG\_XAB$ by
$search\_dt$ at (e) fails. The processing proceeds to the check of the
next arc pair (5, 27). Similarly, all arc pairs in \textit{diff}(1, 2) are
assured to generate no new dependency trees, and then DG' and CM' are
set to DG and CM at (i), respectively.

Figure \ref{fig:IDFandRDFexample} (c) shows the reduced dependency forest
finally obtained by the algorithm. It has three equivalent arcs 25, 16,
and 27. The processing of the algorithm for this dependency forest is
described. Let $X=25$, $Y=26$. Then,
$\textit{uniq}(X, Y)=\{1, 24\}$, $\textit{uniq}(Y, X)=\{6, 13\}$, $\mathit{diff}(X, Y)=\{(1, 6), (1, 13), (24, 6), (24, 23)\}$. The
arc pair (1, 6) is skipped by the condition check for (RC1). The arc
pair (1, 13) is not skipped by the condition check for (RC1) and
(RC2). Then, \textit{DG\_XAB} is computed at
Figure \ref{fig:DFReductionAlgorithm} (d). $X=25$, $A=1$, and $B=13$ result
in $\mathit{DG\_XAB}=\mathrm{co}(25) \cap \mathrm{co}(1) \cap \mathrm{co}(13)$ wrt $\langle\text{DG', CM'}\rangle
=\{\text{25, 1, 13, 5, 28}\}$. A new IDed dependency tree \{25, 1, 13, 5, 28\} is
obtained by \textit{search\_dt} at (e) for \textit{DG\_XAB}.\footnote{This tree
corresponds to the tree in Figure 26 (d).} Then, \textit{new\_generalized\_dt} at
(f) is called, and \textit{add\_equiv\_arcs} at (q) computes a set of arcs
\textit{DG\_X}, where the equivalent arcs of the arcs in DT are added. In the
case of Figure \ref{fig:IDFandRDFexample} (c), arc 25 has equivalent arcs
26 and 27. Addition of these arcs results in
$\mathit{DG\_X}=\{\text{25, 26, 27, 1, 13, 5, 28}\}$. \textit{search\_dt} at
Figure \ref{fig:DFReductionAlgorithm} (r) tries to get a dependency tree
for \textit{DG\_X}, but fails because all equivalent arcs 25, 26, and 27 have
inconsistent arcs in $DG\_X$, i.e., $\langle 25, 13\rangle$, $\langle 26, 1\rangle$, and
$\langle\text{27, 5}\rangle$ are not in CM. As a result, \textit{new\_generalized\_dt} at (f)
becomes true, that is, an increase of the generalized dependency tree
occurs. Therefore, the merging of $X=25$ and $Y=26$ is not
performed. The dependency forest in Figure \ref{fig:IDFandRDFexample} (c)
includes the dependency trees in Figure \ref{fig:SynGraphBadExample} (a)
to (c), and retains the soundness.

The above algorithm does not assure generation of one reduced
dependency forest. The output dependency forest may vary by the
application order of the merge operations for equivalent arcs. There
exist different dependency forests containing the same three
generalized dependency trees for the above example. The algorithm in
Figure \ref{fig:DFReductionAlgorithm} does not assure that it generates
the minimum dependency forest. In fact, there exists a dependency
forest smaller than that in Figure \ref{fig:IDFandRDFexample}
(c). Moreover, there is room for improving the computational amount in
the above algorithm. The construction of the smallest reduced
dependency forest and improvement in algorithm performance are future
tasks.\footnote{PDG allows arbitrary mapping between the constituent
sequences and the partial dependency trees defined in grammar
rules. Therefore, any dependency structure can be assigned for any
constituent sequence, provided that they satisfy the partial
dependency structure condition. This feature suggests that, not only
the optimization techniques in the general algorithm, but also the
techniques based on the structural analysis of the grammar rules are
effective.}


\subsection{Correspondence between Phrase Structure Forest and Dependency Forest}
\label{sec:MappingBetweenPTAndDT}

Appendix~B shows that the initial dependency forest satisfies the
completeness and soundness with respect to the phrase structure
forest. Since the reduced dependency forest has the same set of
generalized dependency trees as the initial dependency forest, the
soundness and completeness between the (reduced) dependency forest and
the phrase structure forest are assured. The correspondences between
the phrase structure trees (phrase structure trees) in the phrase
structure forest and the dependency trees in the dependency forest are
not necessarily simple one-to-one relations. One phrase structure tree
may correspond to more than one dependency tree, whereas more than one
phrase structure tree may correspond to one dependency
tree. Considering the variety (one meaning can be expressed by more
than one expression) and ambiguity (one expression expresses more than
one meaning) encoded in natural languages, these
multiple correspondences may be natural. The correspondences between
phrase structure trees and dependency trees are discussed in the next
section by referring to the experiments for sentence analysis using a
PDG prototype system.


\section{Experiment for Analysis of Example Sentences}

One of the design targets of PDG is the suppression of the
combinatorial explosion caused by a variety of ambiguities using
packed shared data structures. This section describes the experiment
for analyzing typical ambiguous sentences, which contain various kinds
of ambiguities, using PDG grammar rules. This section also discusses
the relation between the phrase structure forest and the dependency
forest, and the generation of non-projective dependency trees on the
basis of real analysis examples. The performance of the algorithm is
an important factor from a practical viewpoint. The algorithms for
parsing, generation of phrase structure forest, and initial dependency
forest and dependency forest reduction described in this paper are
implemented for verifying the PDG analysis. The practical
implementation and its evaluation are subjects for future work. The
following experiment utilizes a prototype PDG system, implemented in
Prolog.


\subsection{PDG Grammar Rules for Example Sentences}

Figure \ref{fig:GrammarForExamples} shows PDG grammar rules used for
analyzing the example sentences containing various prototypical
ambiguities. The POSs in the grammar rules are determiner(det),
n(noun), be-verb(be), present particle of verb (ving), verb (v),
preposition (pre), adverb (adv), and relational phrase (relp). The
grammar rules are not for linguistic analysis, but for experiments with
respect to the PDG framework and algorithms. The grammar rules include
the following types of ambiguities.

\hangafter=1\hangindent=1em
\noindent
PP attachment: Two kinds of attachment ambiguities in R6 (noun attachment) and R10, R17 (verb attachment) 

\hangafter=1\hangindent=1em
\noindent
Coordination scope:]R11 (and) and R12 (or) are coordination noun phrase rules

\hangafter=1\hangindent=1em
\noindent
Be-verb interpretation: Two structural ambiguities caused by be-verb interpretations, i.e., R15 (present progressive) and R16 (copula)

\hangafter=1\hangindent=1em
\noindent
Interpretation of the present progressive form of verb: the following three ambiguities of the present progressive form of verb are described

\begin{itemize}
\item[(a)] adjective usage where the modified noun occupies the subject role (R7)
\item[(b)] adjective usage where the modified noun occupies the object role (R8)
\item[(c)] gerund usage (R9, R10)
\end{itemize}

\begin{figure}[t]
 \begin{center}
\includegraphics{21-4ia6f16.eps}
 \end{center}
\caption{Grammar for the example sentences}
\label{fig:GrammarForExamples}
\end{figure}

\noindent
R8 and R9 are similar since both rules prescribe the
relation between a noun and verb as an object. However, they have different
interpretations in the phrase head and generate different structures
of dependency trees. The grammar contains declarative form (R1) and
imperative sentence form (R2) to produce structural ambiguities in
combination with POS ambiguities of words for sentences like ``Time
flies like an arrow.'' (R19) is a rule for generating a non-projective
dependency structure.


\subsection{Analysis of Prototypical Ambiguous Sentences}

The example for an ambiguous sentence with POS ambiguities has
already been shown in the previous sections in detail. The examples in
the following sections show prototypical syntactic ambiguities, i.e.,
PP attachment ambiguity, coordination scope ambiguity, and ambiguities
in structural interpretations.


\subsubsection{PP attachment Ambiguity}
\label{sec:PP-attachment}

Figure \ref{fig:DFForISawSec4} shows the dependency forest for the
sentence ``I saw a girl with a telescope in the forest,'' which has PP
attachment ambiguities. Each arc in the dependency graph has an arc
name attached by arc-ID and preference score.\footnote{Preference
  scores show the degree of preference of arcs
  \cite{Hirakawa06b_e,Hirakawa07}. The preference score is not used in
  this paper.} The table in Figure \ref{fig:DFForISawSec4} shows the
POS and position information of each node. This sentence has no POS
ambiguities, but has PP attachment ambiguities for prepositions
``with'' (two ambiguities: \textit{npp}13, \textit{vpp}14) and ``in'' (three
ambiguities: \textit{npp}23, \textit{npp}25, \textit{vpp}26). CM in
Figure \ref{fig:DFForISawSec4} inhibits some of the combinations of these
ambiguous arcs. \textit{npp}13 and \textit{vpp}14 (or \textit{npp}23, \textit{npp}25, and \textit{vpp}26)
have no co-occurrence relation because they have the same position
(the single role constraint). The co-occurrence between $vpp14$ and
\textit{npp}25 is also inhibited. If this co-occurrence constraint does not
exist, the dependency forest has six interpretations caused by two PP
attachment ambiguities ($2*3=6$). CM(14, 25)${\neq}$ ○, which
corresponds to the projectivity constraint, excludes a non-projective
dependency tree from a set of well-formed dependency trees in the
dependency forest. This dependency forest has five well-formed
dependency trees, which are possible interpretations for the example
sentence.

\begin{figure}[b]
\begin{center}
\includegraphics{21-4ia6f17.eps}
\end{center}
\caption{DF for the example sentence including PP attachments}
\label{fig:DFForISawSec4}
\end{figure}

The sizes of the phrase structure forest, the initial dependency
forest, and the reduced dependency forest for this example are 25, 18,
and 13, respectively. The phrase structure forest contains five phrase
structure trees\footnote{In judging the equivalence of the phrase
  structure trees, phase heads (head nodes) are taken into
  consideration.} corresponding to the five interpretations of the
sentence. The initial dependency forest and the reduced dependency
forest have five IDed dependency trees, which correspond to the five
generalized dependency trees. \textit{obj}5, \textit{npp}13, \textit{vpp}14, and \textit{pre}11
have 2, 1, 1, and 1 equivalent arcs in the initial dependency forest,
respectively. For example, \textit{obj}5 and its equivalent arcs are
generated from the edges, shown below in diagrammatic form, originated
in the grammar rule (R14).

$\langle\text{1, 4, vp/([saw]-v-1)} \rightarrow \text{v(ID:109) np(ID:126) ・, \{arc(obj-5, [girl]-n-3, [saw]-v-1)\}}\rangle$

$\langle\text{1, 7, vp/([saw]-v-1)} \rightarrow \text{v(ID:109) np(ID:163) ・, \{arc(obj-15, [girl]-n-3, [saw]-v-1)\}}\rangle$

$\langle\text{1, 10, vp/([saw]-v-1)} \rightarrow \text{v(ID:109) np(ID:203) ・, \{arc(obj-28, [girl]-n-3, [saw]-v-1)\}}\rangle$

\noindent
The first edge has coverage from 1 to 4 (``saw a girl''),
phrase head [saw]-v-1, constituent sequence v(ID:109)\footnote{The
  packed edge whose category is v and edge-ID is 109.} np(ID:126), and
the obj arc with arc-ID 5. The above equivalent arcs are generated
from the combination with edges np(ID:126), np(ID:163), and np(ID:203),
which correspond to noun phrases with different coverage. The reduced
dependency forest has no equivalent arcs because all equivalent arcs
in the initial dependency forest are merged.


\subsubsection{Coordination Scope Ambiguity}

\begin{figure}[b]
\begin{center}
\includegraphics{21-4ia6f18.eps}
\end{center}
\caption{DF for the example sentence, including conjunctions}
\label{fig:DFForEarthAndMoonSec4}
\end{figure}

Figure \ref{fig:DFForEarthAndMoonSec4} shows the dependency forest for
the sentence ``Earth and Moon or Jupiter and Ganymede,'' which has
coordination scope ambiguities. ``Earth'' and ``Moon'' have two and
three outgoing arcs, respectively, corresponding to coordination scope
ambiguities. CM(22, 12)${\neq}$ ○, which corresponds to the
projectivity constraint, excludes a non-projective dependency tree
from a set of well-formed dependency trees in the dependency
forest. This dependency forest has five well-formed dependency trees,
which are possible interpretations for the example sentence.

The sizes of the phrase structure forest, the initial dependency
forest, and the reduced dependency forest for this example are 18, 17,
and 10, respectively. The phrase structure forest contains five phrase
structure trees corresponding to the five interpretations of the
sentence. The initial dependency forest and the reduced dependency
forest have five IDed dependency trees, which correspond to the five
generalized dependency trees. \textit{or}22, \textit{or}9, \textit{cnj}6, \textit{and}18, and \textit{cnj}14
have 1, 1, 1, 2, and 2 equivalent arcs in the initial dependency
forest, respectively. The reduced dependency forest has no equivalent
arcs because all equivalent arcs in the initial dependency forest are
merged. The coordination scope ambiguity is similar to the
PP attachment ambiguity in the previous section, but is different from
the PP attachment ambiguity because it has the modification scope
problem described below.


\subsubsection{Ambiguity in Structural Interpretation}
\label{sec:AmbiguityInStructuralInterpertation}

\begin{figure}[b]
\begin{center}
\includegraphics{21-4ia6f19.eps}
\end{center}
\caption{DF for the example sentence, including structural ambiguities}
\label{fig:DFForMyHobbyIsSec4}
\end{figure}

Figure \ref{fig:DFForMyHobbyIsSec4} shows the dependency forest for the
sentence ``My hobby is watching birds with a telescope,'' which has
ambiguities such as the interpretation of be-verb (present progressive
form or copula), the interpretation of ``watching birds''
(\textit{adjs}3, \textit{adjo}4, \textit{obj}5), and PP attachment
(\textit{npp}21, \textit{vpp}22, \textit{npp}24, \textit{vpp}25). This sentence has 10
interpretations.

The sizes of the phrase structure forest, the initial dependency
forest, and the reduced dependency forest for this example are 23, 24,
and 16, respectively. The phrase structure forest contains eight
phrase structure trees, corresponding to 10 interpretations of the
sentence. The initial dependency forest and the reduced dependency
forest have 10 IDed dependency trees, which correspond to the 10
generalized dependency trees. \textit{dsc}9, \textit{dsc}8, \textit{obj}5, \textit{npp}21, and
\textit{vpp}22 have 2, 2, 5, 2, and 2 equivalent arcs in the initial
dependency forest, respectively. In contrast to the example in Section
\ref{sec:PP-attachment}, the equivalent arcs are generated from more
than one grammar rule. For example, the equivalent arcs of \textit{obj}5 are
generated from the edges, shown below, originated in grammar rules
(R9), (R10), and (R15)

(R9) ⇒ $\langle \text{3, 5, np/([watching]-ving-3)} \rightarrow \text{ving(ID:121) np(ID:130)・, }$

\quad $\text{\{arc(obj-5, [birds]-n-4, [watching]-ving-3)\}}\rangle$

(R10) ⇒ $\langle \text{3, 7, np/([watching]-ving-3)} \rightarrow \text{ving(ID:121) np(ID:130) pp(ID:176)・, }$

\quad $\text{\{arc(obj-6, [birds]-n-4, [watching]-ving-3), arc(vpp-22, [with]-pre-5, [watching]-ving-3)\}}\rangle$

(R15) ⇒ $\langle\text{2, 5, vp/([watching]-ving-3) $\rightarrow$ be(ID:117) ving(ID:121) np(ID:130)・, }$

\quad $\text{\{arc(prg-2, [is]-be-2, [watching]-ving-3), arc(obj-7, [birds]-n-4, [watching]-ving-3)\}}\rangle$

\noindent
The reduced dependency forest has no equivalent arcs. In
this example, the number of phrase structure trees in the phrase
structure forest is smaller than that of generalized dependency trees
in the dependency forest. One phrase structure tree corresponds to
more than one dependency tree. The following section discusses the
correspondence between phrase structure trees and dependency trees.


\subsection{1 to N / N to 1 Correspondences between Phrase Structure Trees and Dependency Trees}

The correspondence between a phrase structure tree and a dependency
tree is assured in PDG, but sometimes, one phrase structure tree has
more than one corresponding dependency tree, and more than one phrase
structure tree has one corresponding dependency tree.


\subsubsection{1 to N Correspondence from Phrase Structure Tree to Dependency Trees}

When one phrase structure has more than one interpretation, one phrase
structure tree may correspond to more than one dependency tree. For
example, when ``watching birds'' is assigned to one phrasal structure
where the verb in present particle form modifies the noun, two
dependency structures, ``$\text{watching} \xrightarrow{\mathrm{subj}} \text{birds}$'' 
and ``$\text{watching} \xrightarrow{\mathrm{obj}} \text{birds}$,'' are
assigned to the phrasal structure. This happens when there exist more
than one grammar rule having the same rewriting rule but different
dependency structure parts. This is the case for (R7) and (R8) in
Figure \ref{fig:GrammarForExamples}, which are arbitrary rules,
introduced for verifying the dependency forest. Two kinds of
ambiguities in 1 to N mapping from phrase structure trees to
dependency trees are considered, i.e., the ambiguities in functional
relation and semantic relation. The former means the ambiguities in
functional assignments (subject, object, etc.) for phrase
structures. The functional structures and phrase structures have close
relation, and the difference in functional structures can be reflected
by the difference in phrase structure.\footnote{For example, the
  category in the rewriting rule is segmented into more detailed
  categories, reflecting the difference in functional
  assignments. This segmentation works to remove ambiguous mapping
  from one phrase structure to more than one functional
  structure.} Therefore, it seems to be unusual for more than one
functional structure to be assigned to one rewriting rule. In
contrast, the assignment of more than one semantic structure seems to
be quite a general phenomenon. However, introduction of 1 to N mapping
(from phrase structure to dependency structures) into grammar rules
may cause problems in terms of system performance and grammar rule
maintenance. In general, sentence analysis approaches, which treat
syntactic analysis and semantic analysis independently, are widely
proposed and utilized. PDG introduces a semantic analysis level
utilizing an SDG, which represents lexical concepts and semantic
relations between them. The framework itself, for mapping one phrase
structure tree to more than one dependency tree, is independent of the
linguistic discussion here. Rules with mappings to semantic structures
can be utilized properly with respect to the requirements from the
design and development of the grammar.

The example in Section \ref{sec:AmbiguityInStructuralInterpertation}
(Figure \ref{fig:DFForMyHobbyIsSec4}) contains a phrase structure tree
that corresponds to two dependency structures generated from (R7) and
(R8). Figure \ref{fig:MapFromOnePTToTwoDTs} shows this phrase structure
tree and dependency trees.

\begin{figure}[b]
\begin{center}
\includegraphics{21-4ia6f20.eps}
\end{center}
\caption{Mapping from one phrase structure tree to two dependency trees}
\label{fig:MapFromOnePTToTwoDTs}
\end{figure}


\subsubsection{N to 1 Correspondence from Phrase Structure Trees to One Dependency Tree}

Spurious ambiguity \cite{Noro02_e} is an example of N to 1 mapping
from phrase structures to dependency structure. A real ambiguity is
one where the difference in syntactic structures represents the
difference in semantic interpretations. Spurious ambiguity is one
where the difference in syntactic structures does not represent the
difference in semantic interpretations, or is an ambiguity caused by
linguistically illegal syntactic structures, generated by incomplete
grammar rules. Spurious ambiguity is an important issue in grammar
development from corpora \cite{Noro05_e}. Although it is not CFG, CCG
has many spurious ambiguities because of the flexibility of rule
application. A method for obtaining one normal form tree is proposed
\cite{Eisner96b}, which ensures that one phrase structure tree
among the phrase structure trees in one semantic class is obtained
on the basis of the definition that phrase structure trees having the
same set of leaf CCG category have the same meaning. In PDG, phrase
structure trees corresponding to the same generalized dependency tree
(interpretation of a sentence) can be classified into one semantic
class.

\begin{figure}[b]
\begin{center}
\includegraphics{21-4ia6f21.eps}
\end{center}
\hangcaption{DF containing the mapping from N phrase structure trees to 1 dependency tree (spurious ambiguity)}
\label{fig:DFContainingNtoOneMapping}
\end{figure}

Figure \ref{fig:DFContainingNtoOneMapping} shows the dependency forest
for the example ``She curiously saw a cat in the forest'' using the
example grammar having a spurious ambiguity. There is only one
co-occurrence constraint between \textit{npp}17 and \textit{vpp}18 that corresponds
to the single role constraint. The dependency forest has two
dependency trees, which have different governors for the part ``in the
forest.'' The phrase structure forest has three phrase structure
trees. The initial dependency forest has three IDed dependency trees
and two generalized dependency trees and the reduced dependency forest
has two IDed dependency trees and two generalized dependency
trees. The spurious ambiguities are generated from the difference in
rule application order of (R17) and (R18) for attaching a modification
phrase to a verb phrase. Figure \ref{fig:SuriousNtoOneMapping} shows the
phrase structure trees and the dependency tree.

\begin{figure}[b]
\begin{center}
\includegraphics{21-4ia6f22.eps}
\end{center}
\caption{Example of mapping from N phrase structure trees to 1 dependency tree (spurious ambiguity)}
\label{fig:SuriousNtoOneMapping}
\end{figure}

Obviously, the method of identifying the semantic class based on the
dependency tree does not capture all semantic aspects in natural
languages. For example, the subtle semantic difference
\cite{Eisner96b}\footnote{For example, ``softly knock twice'' has two
  equivalent semantic interpretations softly(twice(knock)) and
  twice(softly(knock)), whereas ``intentionally knock twice'' has two
  different semantic interpretations intentionally(twice(knock)) and
  twice(intentionally(knock)).} and the ambiguities in
number/quantifier scope\footnote{The model theoretic ambiguities as
  observed in, ``Three men bought ten cups,'' cannot be distinguished by
  standard phrase structure and dependency structure representations.}
have to be considered in discussing the equivalent semantic class. The
treatment of differences in semantic interpretation requires further
study. Mel'cuk discussed some linguistic structures where an ordinary
dependency structure fails to express the interpretations of a
sentence \cite{Melcuk88}. These structures are classified into two
categories: the structures, which cannot be treated by phrase
structure properly, and the others. The former includes the model
theoretic interpretation of a sentence. The latter is observed when a
dependency structure has a head word that has dependents located at
the right- and left-hand sides of the headword. In this case, the
dependency structure has ambiguities in modification scope, i.e., the
right-hand modifier modifies only the headword or the phrase including
the left-hand modifier. This problem is called the ``modification scope
problem'' in this paper.

Figure \ref{fig:DFContainingNtoOneMappingRealAmbiguity} shows the
dependency forest for ``Earth and Jupiter in Solar System.'' This
sentence has two interpretations, i.e., the prepositional phrase
modifies only the headword ``Jupiter'' or the phrase ``Earth and
Jupiter.'' The phrase structure forest has two phrase structure trees
corresponding to these two interpretations. On the other hand, the
initial dependency forest has two IDed dependency trees and one
generalized dependency tree, and the reduced dependency forest has one
IDed dependency tree corresponding to one generalized dependency
tree. The two phrase structure trees and one dependency tree are shown
in Figure \ref{fig:RealNtoOneMapping}.

\begin{figure}[b]
\begin{center}
\includegraphics{21-4ia6f23.eps}
\end{center}
\caption{DF containing the mapping from N phrase structure trees to 1 dependency tree (real ambiguity)}
\label{fig:DFContainingNtoOneMappingRealAmbiguity}
\end{figure}

\begin{figure}[t]
\begin{center}
\includegraphics{21-4ia6f24.eps}
\end{center}
\caption{Example of mapping from N phrase structure trees to 1 dependency tree (real ambiguity)}
\label{fig:RealNtoOneMapping}
\end{figure}

Mel'cuk proposes introducing a concept called Grouping into the
dependency structure framework to solve the modification scope problem
\cite{Melcuk88}. Grouping is theoretically equivalent to phrase, in
the sense that it specifies the word coverage information. However,
grouping information is not attached to every part of the structure,
but to some specific structures including the ``conjoined structure''
and ``operator word,'' such as ``not'' and ``only.'' The grammar
framework for a machine translation system \cite{Amano89} incorporates
a mechanism similar to the grouping.\footnote{A special node called a
  ``scoping node'' is introduced to specify the scope of a dependency
  modification as required.} In this grammar development for a
real-world application, the scope nodes are used only for conjoined
structures.\footnote{This is the case for an English-to-Japanese
  system. The requirement level may differ, according to the language
  pairs. For example, translation between languages in the same family
  may not require a grouping mechanism because the modification scope
  ambiguities are avoided by bypassing, i.e., an ambiguous source
  language structure is mapped to the corresponding target language
  structure without disambiguation.}  This experience suggests the
limitation of the application scope of grouping proposed by Mel'cuk is
reasonable.  Moreover, the treatment of the modification scope
ambiguity differs from language to language. According to
\cite{Melcuk88}, some modification scope ambiguities are
distinguishable by lexical or syntactic marking in Russian. Japanese
does not inherently have the modification scope problem because it has
the basic grammatical constraint that modifiers should be located at
the left-hand side of their modificand. In the PDG framework,
equivalent arcs represent the difference of modification scope, as
shown in the previous section. Therefore, the modification scope
problem may be avoided by introducing grouping into the treatment of
equivalent arcs. This is a future task.


\subsection{Generation of Non-projective Dependency Tree}
\label{sec:NonProjectivity}

The projectivity constraint\footnote{The projectivity condition
  consists of two conditions, i.e., ``no cross dependency exits'' and
  ``no dependency covers the top node.'' The second condition is
  unnecessary when a special top node is introduced at the top or end
  of a sentence.} is a basic constraint adopted by many dependency
analysis systems, and these parsers are called projective
parsers. Projective parsers fail to analyze sentences with
non-projective structures. Almost all sentences in many languages are
projective, but some types of non-projective sentences exist
\cite{Melcuk88}. For example, ``John saw a dog yesterday which was a
Yorkshire Terrier'' in English and ``私は本を東京に買いに昨日行きまし
た。'' (I went to Tokyo to buy a book yesterday) in Japanese have
projective dependency structures. McDonald reported that the
non-projective parser outperformed the projective parser in overall
accuracy for the analysis of Czech, which has a high degree of word
order freedom compared with English \cite{McDonald05}. However, the
simple introduction of non-projective interpretation may deteriorate
the accuracy of the parsers for other languages. This seems to be the
reason why the majority of dependency parsers adopt the projectivity
constraint.

As described in Section \ref{sec:bunpoukisoku}, the mapping between
the constituent sequence (the body of grammar rule) and the partial
dependency tree (the dependency structure of the grammar rule) is
defined in the grammar rule in PDG. This framework, in combination
with the description ability of the C-Matrix, enables a controlled
non-projectivity instead of all-or-nothing
non-projectivity. Controlled non-projectivity means that the
non-projective structures are defined by some rules, which prescribe
the well-formedness conditions. The (R19) in
Figure \ref{fig:GrammarForExamples} is a grammar rule for a phrase
pattern where an adverb is inserted before a relative clause and
produces a well-formed non-projective dependency
structure. Figure \ref{fig:Non-projectiveDT} shows the dependency tree
for ``She saw the cat curiously which was Persian,''\footnote{This is
  an artificial example only for showing the rule applicability.}
obtained by the example PDG grammar. The dependency forest has one
non-projective dependency tree.

\begin{figure}[b]
 \begin{center}
\includegraphics{21-4ia6f25.eps}
 \end{center}
\caption{Example of non-projective dependency tree generation}
\label{fig:Non-projectiveDT}
\end{figure}


\section{Concluding Remarks}

This paper describes the MPDC model, which is the basic analysis model
adopted by PDG, and explains two packed shared data structures of PDG,
i.e., the phrase structure forest and the dependency forest. The
completeness and soundness of the correspondence between the phrase
structure forest and the dependency forest are assured. This means
that the sentence interpretations represented in a packed shared phrase
structure and the sentence interpretations represented in a packed
shared dependency structure have linkages. This paper also describes
the experimental results for analyzing some typical ambiguous
sentences using an example PDG grammar.

The current implementation of the PDG system focuses on a feasibility
study of the PDG framework. The practical PDG system and its
performance evaluation are future tasks. Extension of the PDG grammar
formalism (such as the introduction of optional element specification
and feature conditions) and improvement in performance by efficient
codes and optimizing methods based on grammar analysis, should be
studied to realize a practical system.

\acknowledgment

I would like to thank Associate Professor Ko Sakai from the School of
Institute of Mathematics, University of Tsukuba, for his suggestions
on the proof of the completeness and soundness of the dependency
forest. I appreciate the comments from the referees of my related
papers submitted to the Transaction of Information Processing Society
of Japan and the Journal of Natural Language Processing.


\bibliographystyle{jnlpbbl_1.5}
\begin{thebibliography}{}

\bibitem[\protect\BCAY{Amano, Hirakawa, Nogami, \BBA\ Kumano}{Amano
  et~al.}{1989}]{Amano89}
Amano, S., Hirakawa, H., Nogami, H., \BBA\ Kumano, A. \BBOP 1989\BBCP.
\newblock \BBOQ The Toshiba Machine Translation System.\BBCQ\
\newblock {\Bem Future Computing Systems}, {\Bbf 2}  (3), \mbox{\BPGS\ 15--23}.

\bibitem[\protect\BCAY{Bikel}{Bikel}{2004}]{Bikel04}
Bikel, D.~M. \BBOP 2004\BBCP.
\newblock \BBOQ Intricacies of Collins' Parsing Model.\BBCQ\
\newblock {\Bem Computational Linguistics}, {\Bbf 30(4)}, \mbox{\BPGS\
  479--511}.

\bibitem[\protect\BCAY{Carroll \BBA\ Charniak}{Carroll \BBA\
  Charniak}{1992}]{Carroll92}
Carroll, G.\BBACOMMA\ \BBA\ Charniak, E. \BBOP 1992\BBCP.
\newblock \BBOQ Two Experiments on Learning Probablistic Dependency Grammars
  form Corpora.\BBCQ\
\newblock Technical report, Department of Computer Science, Brown University.

\bibitem[\protect\BCAY{Charniak}{Charniak}{2000}]{Charniak00}
Charniak, E. \BBOP 2000\BBCP.
\newblock \BBOQ A Maximum-Entropy-Inspired Parser.\BBCQ\
\newblock In {\Bem Proceedings of the 1st Conference of the North American
  Chapter of the Association for Computational Linguistics}, \mbox{\BPGS\
  132--139}.

\bibitem[\protect\BCAY{Clark \BBA\ Curran}{Clark \BBA\ Curran}{2003}]{Clark03}
Clark, S.\BBACOMMA\ \BBA\ Curran, J.~R. \BBOP 2003\BBCP.
\newblock \BBOQ Log-Linear Models for Wide-Coverage CCG Parsing.\BBCQ\
\newblock In {\Bem Proceedings of the SIGDAT Conference on Empirical Methods in
  Natural Language Processing (EMNLP '03)}, \mbox{\BPGS\ 97--104}.

\bibitem[\protect\BCAY{Collins}{Collins}{1999}]{Collins99}
Collins, M. \BBOP 1999\BBCP.
\newblock {\Bem Head-Driven Statistical Models for Natural Language Parsing}.
\newblock Ph.D.\ thesis, University of Pennsylvania.

\bibitem[\protect\BCAY{Eisner}{Eisner}{1996}]{Eisner96b}
Eisner, J. \BBOP 1996\BBCP.
\newblock \BBOQ Efficient Normal Form Parsing for Combinatory Categorial
  Grammar.\BBCQ\
\newblock In {\Bem Proceedings of the 34th Annual Meeting of the ACL},
  \mbox{\BPGS\ 79--86}.

\bibitem[\protect\BCAY{Harper, Hockema, \BBA\ White}{Harper
  et~al.}{1999}]{Harper99}
Harper, M.~P., Hockema, S.~A., \BBA\ White, C.~M. \BBOP 1999\BBCP.
\newblock \BBOQ Enhanced Constraint Dependency Grammar Parsers.\BBCQ\
\newblock In {\Bem Proceedings of the IASTED International Conference on
  Artificial Intelligence and Soft Computing}.

\bibitem[\protect\BCAY{Hirakawa}{Hirakawa}{2001}]{Hirakawa01}
Hirakawa, H. \BBOP 2001\BBCP.
\newblock \BBOQ Semantic Dependency Analysis Method for Japanese Based on
  Optimum Tree Search Algorithm.\BBCQ\
\newblock In {\Bem Proceedings of the PACLING2001}, \mbox{\BPGS\ 117--126}.

\bibitem[\protect\BCAY{Hirakawa}{Hirakawa}{2002}]{Hirakawa02_e}
Hirakawa, H. \BBOP 2002\BBCP.
\newblock \BBOQ Semantic Dependency Analysis Method for Japanese Based on
  Optimum Tree Search Algorithm (in Japanese).\BBCQ\
\newblock {\Bem IPSJ Journal}, {\Bbf 43}  (3), \mbox{\BPGS\ 696--707}.

\bibitem[\protect\BCAY{Hirakawa}{Hirakawa}{2005}]{Hirakawa05d_e}
Hirakawa, H. \BBOP 2005\BBCP.
\newblock \BBOQ Graph Branch Algotithm: An Optimum Tree Search Method for
  Scored Dependency Graph with Arc Co-occurrence Constraints (in
  Japanese).\BBCQ\
\newblock In {\Bem Natural Language Processing NL-169-16,IPSJ}, \mbox{\BPGS\
  101--108}.

\bibitem[\protect\BCAY{Hirakawa}{Hirakawa}{2006}]{Hirakawa06b_e}
Hirakawa, H. \BBOP 2006\BBCP.
\newblock \BBOQ {Graph Branch Algorithm: An Optimum Tree Search Method for
  Scored Dependency Graph with Arc Co-occurrence Constraints (in
  Japanese)}.\BBCQ\
\newblock {\Bem Journal of Natural Language Processing}, {\Bbf 13}  (4),
  \mbox{\BPGS\ 3--32}.

\bibitem[\protect\BCAY{Hirakawa}{Hirakawa}{2007}]{Hirakawa07}
Hirakawa, H. \BBOP 2007\BBCP.
\newblock {\Bem Preference Dependency Grammar (PDG): Sentence Analysis Method
  based on Integrated Multilevel Preference and Constraint}.
\newblock Ph.D.\ thesis, University of Tokyo.

\bibitem[\protect\BCAY{Hirakawa \BBA\ Amano}{Hirakawa \BBA\
  Amano}{1989}]{Hirakawa89}
Hirakawa, H.\BBACOMMA\ \BBA\ Amano, S. \BBOP 1989\BBCP.
\newblock \BBOQ Japanese Sentence Analysis using Syntactic/Semantic Preference
  (in Japanese).\BBCQ\
\newblock In {\Bem Proceedings of the 3rd National Conference of JSAI},
  \mbox{\BPGS\ 363--366}.

\bibitem[\protect\BCAY{Jelinek, Lafferty, \BBA\ Mercer}{Jelinek
  et~al.}{1992}]{jelinek92}
Jelinek, F., Lafferty, J., \BBA\ Mercer, R. \BBOP 1992\BBCP.
\newblock {\Bem {Speech Recognition and Understanding: Recent Advances, Trends,
  and Applications, Series F: Computer and Systems Sciences, Volume 75}}, \BCH\
  {Basic Methods of Probabilistic Context-free Grammars}, \mbox{\BPGS\
  1689--1690}.
\newblock Springer Verlag, New York.

\bibitem[\protect\BCAY{Kaplan}{Kaplan}{1989}]{Kaplan89}
Kaplan, R. \BBOP 1989\BBCP.
\newblock \BBOQ The Formal Architecture of Lexical-Functional Grammar.\BBCQ\
\newblock {\Bem Journal of Information Science and Engineering}, {\Bbf 5},
  \mbox{\BPGS\ 305--322}.

\bibitem[\protect\BCAY{Kawahara \BBA\ Kurohashi}{Kawahara \BBA\
  Kurohashi}{2005}]{Kawahara05_e}
Kawahara, D.\BBACOMMA\ \BBA\ Kurohashi, S. \BBOP 2005\BBCP.
\newblock 大規模格フレームに基づく構文・格解析の統合的確率モデル (in
  Japanese).\
\newblock In {\Bem Proceeding of the 11th Annual Meeting of the Association for
  Natural Language Processing}, \mbox{\BPGS\ 923--926}.

\bibitem[\protect\BCAY{Kudo \BBA\ Matsumoto}{Kudo \BBA\
  Matsumoto}{2005}]{Kudo05_e}
Kudo, T.\BBACOMMA\ \BBA\ Matsumoto, Y. \BBOP 2005\BBCP.
\newblock \BBOQ Japanese Dependency Parsing Using Relative Preference of
  Dependency (in Japanese).\BBCQ\
\newblock {\Bem IPSJ Journal}, {\Bbf 46}  (4), \mbox{\BPGS\ 1082--1092}.

\bibitem[\protect\BCAY{Lafferty, Sleator, \BBA\ Temperley}{Lafferty
  et~al.}{1992}]{Lafferty92}
Lafferty, J., Sleator, D., \BBA\ Temperley, D. \BBOP 1992\BBCP.
\newblock \BBOQ Grammatical Trigrams: a Probabilistic Model of Link
  Grammar.\BBCQ\
\newblock In {\Bem Probabilistic Approaches to Natural Language}.

\bibitem[\protect\BCAY{Lee \BBA\ Choi}{Lee \BBA\ Choi}{1997}]{Lee97}
Lee, S.\BBACOMMA\ \BBA\ Choi, K.~S. \BBOP 1997\BBCP.
\newblock \BBOQ Reestimation and Best-First Parsing Algorithm for Probablistic
  Dependency Grammars.\BBCQ\
\newblock In {\Bem Proceedings of the Fifth Workshop on Very Large Corpora},
  \mbox{\BPGS\ 41--55}.

\bibitem[\protect\BCAY{Maruyama}{Maruyama}{1990}]{Maruyama90}
Maruyama, H. \BBOP 1990\BBCP.
\newblock \BBOQ Constraint Dependency Grammar and its Weak Generative
  Capacity.\BBCQ\ Computer Software.

\bibitem[\protect\BCAY{Matsumoto, Tanaka, Hirakawa, Miyoshi, \BBA\
  Yasukawa}{Matsumoto et~al.}{1983}]{Matsumoto83}
Matsumoto, Y., Tanaka, H., Hirakawa, H., Miyoshi, H., \BBA\ Yasukawa, H. \BBOP
  1983\BBCP.
\newblock \BBOQ BUP: A Bottom-Up Parser Embedded in Prolog.\BBCQ\
\newblock {\Bem New Generation Computing}, {\Bbf 1}  (2), \mbox{\BPGS\
  145--158}.

\bibitem[\protect\BCAY{McDonald, Crammer, \BBA\ Pereira}{McDonald
  et~al.}{2005}]{McDonald05}
McDonald, R., Crammer, K., \BBA\ Pereira, F. \BBOP 2005\BBCP.
\newblock \BBOQ Spanning Tree Methods for Discriminative Training of Dependency
  Parsers.\BBCQ\
\newblock Technical report, UPenn CIS.

\bibitem[\protect\BCAY{Mel'cuk}{Mel'cuk}{1988}]{Melcuk88}
Mel'cuk, I. \BBOP 1988\BBCP.
\newblock {\Bem {Dependency Syntax: Theory and Practice}}.
\newblock State University of New York Press, Albany, New York.

\bibitem[\protect\BCAY{Nivre}{Nivre}{2005}]{nivre2005dga}
Nivre, J. \BBOP 2005\BBCP.
\newblock \BBOQ {Dependency Grammar and Dependency Parsing}.\BBCQ\
\newblock \BTR, MSI report 05133, Vaxjo University: School of Mathematics and
  Systems Engineering.

\bibitem[\protect\BCAY{Noro, Hashimoto, Tokunaga, \BBA\ Tanaka}{Noro
  et~al.}{2005}]{Noro05_e}
Noro, T., Hashimoto, T., Tokunaga, T., \BBA\ Tanaka, H. \BBOP 2005\BBCP.
\newblock \BBOQ Building a Large-Scale Japanese Grammar (in Japanese).\BBCQ\
\newblock {\Bem Journal of Natural Language Processing}, {\Bbf 12}  (1),
  \mbox{\BPGS\ 3--32}.

\bibitem[\protect\BCAY{Noro, Okazaki, Tokunaga, \BBA\ Tanaka}{Noro
  et~al.}{2002}]{Noro02_e}
Noro, T., Okazaki, A., Tokunaga, T., \BBA\ Tanaka, H. \BBOP 2002\BBCP.
\newblock 大規模日本語文法構築に関する一考察 (in Japanese).\
\newblock In {\Bem Proceeding of the Eighth Annual Meeting of the Association
  for Natural Language Processing}, \mbox{\BPGS\ 387--390}.

\bibitem[\protect\BCAY{Ozeki}{Ozeki}{1994}]{Ozeki94}
Ozeki, K. \BBOP 1994\BBCP.
\newblock \BBOQ Dependency Structure Analysis as Combinatorial
  Optimization.\BBCQ\
\newblock {\Bem Information Sciences}, {\Bbf 78(1-2)}, \mbox{\BPGS\ 77--99}.

\bibitem[\protect\BCAY{Pereira \BBA\ Warren}{Pereira \BBA\
  Warren}{1980}]{Pereira80}
Pereira, F.\BBACOMMA\ \BBA\ Warren, D. \BBOP 1980\BBCP.
\newblock \BBOQ Definite Clause Grammars for Language Analysis---A Survey of
  the Formalism and a Comparison with Augmented Transition Networks.\BBCQ\
\newblock {\Bem Artificial Intelligence}, {\Bbf 13}  (3), \mbox{\BPGS\
  231--278}.

\bibitem[\protect\BCAY{Pollard \BBA\ Sag}{Pollard \BBA\ Sag}{1994}]{Pollard94}
Pollard, C.\BBACOMMA\ \BBA\ Sag, I.~A. \BBOP 1994\BBCP.
\newblock {\Bem Head-Driven Phrase Structure Grammar}.
\newblock University of Chicago Press, Chicago.

\bibitem[\protect\BCAY{Riezler, King, Kaplan, Crouch, Maxwell~III, \BBA\
  Johnson}{Riezler et~al.}{2002}]{Riezler02}
Riezler, S., King, T.~H., Kaplan, R.~M., Crouch, R., Maxwell~III, J.~T., \BBA\
  Johnson, M. \BBOP 2002\BBCP.
\newblock \BBOQ Parsing the Wall Street Journal using a Lexical-Functional
  Grammar and Discriminative Estimation Techniques.\BBCQ\
\newblock In {\Bem Proceedings of the 40th Annual Meeting of the ACL (ACL-02)},
  \mbox{\BPGS\ 271--278}.

\bibitem[\protect\BCAY{Rim, Seo, \BBA\ Simmons}{Rim et~al.}{1990}]{Rim90}
Rim, H.~C., Seo, J., \BBA\ Simmons, R.~F. \BBOP 1990\BBCP.
\newblock \BBOQ Transforming Syntactic Graphs into Semantic Graphs.\BBCQ\
\newblock Technical report, Artificial Intelligence Lab, Texas at Austin.

\bibitem[\protect\BCAY{Schiehlen}{Schiehlen}{1996}]{Schiehlen96}
Schiehlen, M. \BBOP 1996\BBCP.
\newblock \BBOQ Semantic Construction from Parse Forests.\BBCQ\
\newblock In {\Bem Proceedings of COLING'96}, \mbox{\BPGS\ 907--912}.

\bibitem[\protect\BCAY{Seo \BBA\ Simmons}{Seo \BBA\ Simmons}{1989}]{Seo89}
Seo, J.\BBACOMMA\ \BBA\ Simmons, R.~F. \BBOP 1989\BBCP.
\newblock \BBOQ A Syntactic Graphs: A Representation for the Union of All
  Ambiguous Parse Trees.\BBCQ\
\newblock {\Bem Computational Linguistics}, {\Bbf 15}, \mbox{\BPGS\ 19--32}.

\bibitem[\protect\BCAY{Shudo, Narahara, \BBA\ Yoshida}{Shudo
  et~al.}{1980}]{Shudo80}
Shudo, K., Narahara, T., \BBA\ Yoshida, S. \BBOP 1980\BBCP.
\newblock \BBOQ Morphological Aspects of Japanese Language Processing.\BBCQ\
\newblock In {\Bem Proceedings of COLING'80}, \mbox{\BPGS\ 1--8}.

\bibitem[\protect\BCAY{Sleator \BBA\ Temperley}{Sleator \BBA\
  Temperley}{1991}]{Sleator91}
Sleator, D.\BBACOMMA\ \BBA\ Temperley, D. \BBOP 1991\BBCP.
\newblock \BBOQ Parsing English with a Link Grammar.\BBCQ\
\newblock Technical report, Department of Computer Science, Carnegie Mellon
  University.

\bibitem[\protect\BCAY{Steedman}{Steedman}{2000}]{Steedman00}
Steedman, M. \BBOP 2000\BBCP.
\newblock {\Bem The Syntactic Process}.
\newblock The MIT Press, Cambridge, MA.

\bibitem[\protect\BCAY{Tomita}{Tomita}{1987}]{Tomita87}
Tomita, M. \BBOP 1987\BBCP.
\newblock \BBOQ An Efficient Augmented-Context-Free Parsing Algorithm.\BBCQ\
\newblock {\Bem Computational Linguistics}, {\Bbf 13}  (1-2), \mbox{\BPGS\
  31--46}.

\bibitem[\protect\BCAY{Tsuruoka, Miyao, \BBA\ Tsujii}{Tsuruoka
  et~al.}{2004}]{Tsuruoka04}
Tsuruoka, Y., Miyao, Y., \BBA\ Tsujii, J. \BBOP 2004\BBCP.
\newblock \BBOQ Towards Efficient Probabilistic HPSG Parsing: Integrating
  Semantic and Syntactic Preference to Guide the Parsing.\BBCQ\
\newblock In {\Bem IJCNLP-04 Workshop: Beyond Shallow Analyses---Formalisms and
  Statistical Modeling for Deep Analyses}.

\bibitem[\protect\BCAY{Wang \BBA\ Harper}{Wang \BBA\ Harper}{2004}]{Wang04}
Wang, W.\BBACOMMA\ \BBA\ Harper, M.~P. \BBOP 2004\BBCP.
\newblock \BBOQ A Statistical Constraint Dependency Grammar (CDG) Parser.\BBCQ\
\newblock In {\Bem Workshop on Incremental Parsing: Bringing Engineering and
  Cognition Together (ACL)}, \mbox{\BPGS\ 42--49}.

\bibitem[\protect\BCAY{Wilks}{Wilks}{1975}]{Wilks75}
Wilks, Y.~A. \BBOP 1975\BBCP.
\newblock \BBOQ An Intelligent Analyzer and Under-stander of English.\BBCQ\
\newblock {\Bem Communications of the A.C.M.}, {\Bbf 18}, \mbox{\BPGS\
  264--274}.

\bibitem[\protect\BCAY{Winograd}{Winograd}{1983}]{Winograd83}
Winograd, T. \BBOP 1983\BBCP.
\newblock {\Bem Language as a Cognitive Process: Volume 1 Syntax}.
\newblock Addison-Wesley Publishing.

\end{thebibliography}


\clearpage

\appendix

\section{Problem in the Syntactic Graph}

Consider the parsing of ``Tokyo taxi driver call center'' using the
following grammar rules and lexicons.

\noindent
\texttt{[Grammar Rules]}
\begin{tabbing} 
\hspace{1em}\=\texttt{np3/Nd}~\= $\rightarrow$~\=\texttt{n/Na,n/Nb,n/Nc,n/Nd}\quad\=\texttt{:}~\=\texttt{[arc(nc,Na,Nb),arc(nj,Nb,Nd),arc(nc,Nc,Nd)]}\kill
\>\texttt{np/NP}\>$\rightarrow$\>{\tt npc/NP} \>{\tt :}\>{\tt []}\\
\>{\tt npc/Nb}\>$\rightarrow$\>{\tt np1/NP1,n/Na,n/Nb}\>{\tt :}\>{\tt [arc(nj,NP1,Nb),arc(nc,Na,Nb)]}\\
\>{\tt npc/Na}\>$\rightarrow$\>{\tt np2/NP2,n/Na}\>{\tt :}\>{\tt [arc(nc,NP2,Na)]}\\
\>{\tt npc/Na}\>$\rightarrow$\>{\tt np3/NP3,n/Na}\>{\tt :}\>{\tt [arc(nc,NP3,Na)]}\\
\>{\tt np1/Nc}\>$\rightarrow$\>{\tt n/Na,n/Nb,n/Nc}\>{\tt :}\>{\tt [arc(nc,Na,Nb),arc(nc,Nb,Nc)]}\\
\>{\tt np2/Nd}\>$\rightarrow$\>{\tt n/Na,n/Nb,n/Nc,n/Nd}\>{\tt :}\>{\tt [arc(nj,Na,Nc),arc(nc,Nb,Nc),arc(nc,Nc,Nd)]}\\
\>{\tt np3/Nd}\>$\rightarrow$\>{\tt n/Na,n/Nb,n/Nc,n/Nd}\>{\tt :}\>{\tt [arc(nc,Na,Nb),arc(nj,Nb,Nd),arc(nc,Nc,Nd)]}
\end{tabbing} 
\texttt{[Lexicon]}\\
\hspace{1em}{\tt word(n,[Tokyo]). word(n,[taxi]). word(n,[driver]).}\\
\hspace{1em}{\tt word(n,[call]). word(n,[center]).}

\begin{figure}[b]
 \begin{center}
\includegraphics{21-4ia6f26.eps}
 \end{center}
\caption{Problem in the syntactic graph/exclusion matrix}
\label{fig:SynGraphBadExample}
\end{figure}

This example sentence has three well-formed dependency trees shown in
Figure \ref{fig:SynGraphBadExample} (a), (b), and (c). The boxes np1, np2,
and np3 in the dependency trees are given only for showing the
correspondences between the phrase structures and dependency structures.

Since nc-1 and nc-2 in (a), nc-2 and nc-3 in (b), and nc-3 and nc-1 in
(c) have co-occurrence relations, respectively, the values of the
exclusion matrix for these three pairs are 0 (`` '' in the
figure). This allows the existence of dependency tree (d), which
has no corresponding phrase structure tree in the phrase structure
forest in the syntactic graph/exclusion matrix. Therefore, the
syntactic graph violates the soundness condition.


\section{Proof of the Completeness and Soundness of the Dependency Forest}

The phrase structure forest PF and the dependency forest
$\mathrm{DF}=\langle \mathrm{DG}, \mathrm{CM}\rangle$ is assumed in the following proof. Before showing the
proof, some relations between the phrase structure forest and the
dependency forest generated from the algorithms explained in Section
\ref{sec:ParsingAlgorithm}, and some lemmas required for the proof, are
described.


\subsection*{[Packed Edge and Single Edge]}

The phrase structure forest is a set of packed edges. As described in
Section \ref{sec:EdgeConstruction}, a packed edge is equivalent to a
set of single edges. In this proof, packed edges are treated as a set
of single edges. The following packed edge is shown in
Figure \ref{fig:ArcStructure}.

Packed edge: $\langle\text{ID, FP, TP, C, PH, FCSL, RCS, DSL}\rangle$ \\
\hspace{7em}where FCSL=[CS$_1$, ${\ldots}$ , CS$_n$], DSL=[DS$_1$, ${\ldots}$ , DS$_n$]

\noindent
is equal to the following set of single edges.

*e$_1$: $\langle\text{ID-1, FP, TP, C, PH, (CS$_1$ DS$_1$), RCS}\rangle$

      $\vdots$

*e$_n$: $\langle\text{ID-$n$, FP, TP, C, PH, (CS$_n$ DS$_n$), RCS}\rangle$

\noindent
For example, *E4 in Figure \ref{fig:ArcStructure} is a set
of arcs *e$_1$, *e$_2$\footnote{The partial dependency tree is
  represented in \{\} because it is a set of arcs.}

Single edge *e$_1$: $\langle$170-1, 0, 5, vp, [time]-v-0, [103, 169], [],
 \{arc(obj-25, [flies]-n-1, [time]-v-0)\}$\rangle$

Single edge *e$_2$: $\langle$170-2, 0, 5, vp, [time]-v-0, [103, 119, 165], [],\\
\space{8em}\{arc(obj-4, [flies]-n-1, [time]-v-0), arc(vpp-20, [like]-pre-2, [time]-v-0)\}$\rangle$

\noindent
Each edge is identified in the phrase structure forest by
the packed edge-ID and the position in the CSDS pair of the packed
edge. For example, *e$_1$ is identified by 170-1. The lexical edge is
treated as a set consisting of a single lexical edge. @E5 in
Figure \ref{fig:ArcStructure} is equal to the set {@e$_3$}.

Single edge @e$_3$: $\langle$156-1, 4, 5, n, [arrow]-n-4, [lex([arrow]-n)], \{[[arrow]-n-4]\}$\rangle$

\noindent
Various elements included in a packed edge and a single
edge have correspondences with one another. The following shows the
definitions of terms and relations.

\vspace{0.5\Cvs}
\noindent\fbox{
\begin{minipage}{405pt}
\setlength{\parindent}{0pt}
Edge and its elements

\quad \textbf{cs($X$)}: The constituent sequence of single edge $X$.

\qquad For example, cs(*e$_1$) =[103, 169] where 103 and 169 are packed edge-IDs.

\hangafter=1\hangindent=2em
\quad \textbf{ds($X$)}: The dependency structure DS of single edge $X$ or the node of single lexical \linebreak edge $X$.

\qquad For example, ds(*e$_1$)=\{arc(obj-25, [flies]-n-1, [time]-v-0)\}, ds(@e$_3$)=\{[arrow]-n-4\}

\quad \textbf{Arcs in the dependency forest and edge}: The arcs in a single edge $X$ mean $a{\in}$ds($X$). 

{\addtolength{\leftskip}{2em}
An arc in a packed edge Y means $a{\in}$ds($X$), $X{\in}Y$. An arc in the phrase structure forest means $a{\in}$ds($X$), $X{\in}Y$, $Y{\in}$PF. 
\par}

Relations between arcs and nodes

\quad \textbf{gov($X$)}: The governor node of arc $X$. 

\qquad For example, gov(arc(obj-25, [flies]-n-1, [time]-v-0)) = [time]-v-0

\quad \textbf{dep($X$)}: The dependent node of arc $X$.

\qquad For example, dep(arc(obj-25, [flies]-n-1, [time]-v-0) = [flies]-n-1

\noindent\hangafter=1\hangindent=2em
\quad \textbf{top\_node($X$)}: The top node of dependency tree $X$ (the node that is not a dependent of any arcs in $X$).

Relations between arcs $X$, $Y$ in dependency tree DT

\quad \textbf{sib($X$, $Y$)}: gov($X$)=gov($Y$). $X$ and $Y$ are called sibling arcs.

\noindent\hangafter=1\hangindent=2em
\quad \textbf{$X \xrightarrow[{\rm DT}]{1} Y$}: dep($X$)=gov($Y$). $X$ is a parent of $Y$ and $Y$ is a child of $X$. This relation is called a parent relation.

\noindent\hangafter=1\hangindent=2em
\quad \textbf{$X \xrightarrow[{\rm DT}]{+} Y$}: There is a parent relation chain from $X$ to $Y$. $X$ is an ancestor arc of $Y$ and $Y$ is a descendant arc of $X$.

\noindent\hangafter=1\hangindent=2em
\quad \textbf{$X \xrightarrow[{\rm DT}]{*} Y$}: $X=Y$ or $X{\xrightarrow[{\rm DT}]{+}}Y$

\end{minipage}
}


\subsection*{[Edges and Phrase Structure Trees in the Phrase Structure Forest]}

The phrase structure forest PF is a directed acyclic graph consisting
of packed edges, where the root is the root edge *E$_{root}$ and the
leaves are lexical edges. The ``path in PF'' is defined as follows:

\clearpage
\begin{definition}\label{def:PathInPF}[Path in the phrase structure forest]

A path in the phrase structure forest is a sequence consisting of
packed edges and single edges obtained by tracing a packed edge and a
single edge one after another by selecting one single edge from a
packed edge (a set of single edges) and selecting one packed edge from
the constituent sequence (a sequence of packed edges) of a single
edge.
\end{definition}

\noindent
Now, let *E$_0$, *E$_1$, *E$_2$ ${\cdots}$ in the phrase structure forest as follows:

*E$_0=$\{*e$_1$, *e$_2$\}, *E$_1=$\{*e$_3$\}, *E$_2=$\{*e$_4$, *e$_5$\}, *E$_3=$\{*e$_6$, *e$_7$\} ${\ldots}$

cs(*e$_1$)=[*E$_1$, *E$_2$], cs(*e$_2$)=[*E$_3$], cs(*e$_3$)=[*E$_4$, *E$_5$] ${\ldots}$

\noindent
The following are examples of paths.

[*E$_0$, *e$_1$, *E$_2$, *e$_5$], [*E$_0$, *e$_1$, *E$_2$], [*e$_1$, *E$_1$, *e$_3$, *E$_5$], [*e$_1$, *E$_1$, *e$_3$]

\noindent
The following shows the definitions of terms and relations used in the latter part.

\vspace{0.5\Cvs}
\noindent
\fbox{
\begin{minipage}{405pt}
\noindent
Terms and relations related to the phrase structure forest

\hangafter=1\hangindent=2em
\quad $X \xrightarrow[\mathrm{PF}]{+} Y$: There is a path [$X, {\ldots}, Y$] from a packed or single edge $X$ to a packed or single edge $Y$. $X$ is an ancestor of $Y$.

\hangafter=1\hangindent=2em
\quad $X \xrightarrow[\mathrm{PF}]{*} Y$: $X=Y$ or $X \xrightarrow[\mathrm{PF}]{+} Y$ is true for single or packed edges $X$ and $Y$. $Y$ is called ``reachable'' from $X$.

\hangafter=1\hangindent=2em
\quad $X \swarrow_{[PF]}\searrow Y$: $X \neq Y$, $\neg (X \xrightarrow[\mathrm{PF}]{*}Y)$, $\neg (Y \xrightarrow[\mathrm{PF}]{*}X)$ are true for single or packed edges $X$ and $Y$, and there exists at least one single or packed edge $Z$ in the phrase structure forest PF such that $Z \xrightarrow[\mathrm{PF}]{*}X$ and $Z \xrightarrow[\mathrm{PF}]{*}Y$.

\hangafter=1\hangindent=2em
\quad Arcs governed by an edge: Arc $X$ is governed by packed edge *E if $X \in$ds(*e), *E$ \xrightarrow[\mathrm{PF}]{*}$*e is true.

\end{minipage}
}
\vspace{0.5\Cvs}

\noindent
From the definition of the phrase structure forest, there
exists a path from the root packed edge *E$_\mathrm{root}$ to every single or
packed edge in the phrase structure forest. Using the definition
above, the C-Matrix setting conditions in Section
\ref{sec:PFandDFseisei} are defined as follows:

\begin{definition}\label{def:CMSetCond}[The C-Matrix setting conditions] 

Arcs $X$, $Y$ are co-occurable if any of the following conditions is satisfied.
\begin{itemize}
\item[(C1)] There exists a single edge *e, such that $X$, $Y{\in}$ds(*e), *e${\in}$*E, *E${\in}$PF
\item[(C2)] There exist *e$_x$ and *e$_y$, such that $X{\in}$ds(*e$_x$), $Y{\in}$ds(*e$_y$), *e$_x \xrightarrow[{\rm PF}]{+}$*e$_y$ or *e$_y \xrightarrow[{\rm PF}]{+}$*e$_x$.
\item[(C3)] There exist *e$_x$ and *e$_y$, such that $X{\in}$ds(*e$_x$), $Y{\in}$ds(*e$_y$), *e$_x \swarrow_{[PF]} \searrow$*e$_y$.
\end{itemize}
\end{definition}

\noindent
A phrase structure tree is defined as follows:

\clearpage
\begin{definition}\label{def:ParseTree}[Phrase structure tree] 

A phrase structure tree for a packed edge *E is a set of single arcs obtained by a recursive procedure get\_tree(*E) defined in Figure \ref{fig:get_parse_tree}.
\end{definition}

\begin{figure}[t]
 \begin{center}
\includegraphics{21-4ia6f27.eps}
 \end{center}
\caption{Algorithm for obtaining a phrase structure tree}
\label{fig:get_parse_tree}
\end{figure}

In the figure, select(RPE) selects one arbitrary arc included in a
packed edge RPE. If SE is a lexical edge, lexical\_edge(SE) is true. A
phrase structure tree covers the words from the from-position to the
to-position of edge *E.

\begin{definition}\label{AllParseTrees}[All phrase structure trees] 

ps\_trees(*E) is a set of all phrase structure trees for *E.
\end{definition}


\subsection*{[Relations between Edges and Arcs/Partial Dependency Trees]}

Although the parsing algorithm is constructed using packed edges as
basic data structures, the packing of edges is performed only when
inactive edges are generated (Figure \ref{fig:ChartAlgotithm}
(c), (d)). Therefore, every active packed edge has one single
edge,\footnote{The parsing algorithm utilizes packed edges because
  another algorithm, which shares active edges, can be constructed.}
and one active packed edge corresponds to one single active edge and
vice versa. In the following discussion, the word ``edge'' is used for
representing a packed edge.

Parsing proceeds by generating new edges by combining an inactive edge
to an active edge. Using a diagrammatic expression, as described in
Section \ref{sec:EdgeConstruction}, a combination of two arcs
generates a new edge by moving `・' in the active edge to the right
neighbor position, and binding the variable for the constituent at `・'
to the phrase head (node) of the inactive edge to
combine. Figure \ref{fig:EdgeCombinationTree} is a tree called an ``edge
combination tree,'' which represents the generation process of
inactive edges from a grammar rule by edge combinations. The inactive
edges located at the leaf of the edge combination tree are generated
from a grammar rule located at the root of the tree\footnote{The
  grammar rule is written in edge form in the edge combination
  tree. This edge is not generated in the real parsing process, but is
  introduced for convenience of explanation.} through the active edges
in between. The grammar rule is as follows:
\[
\mathrm{y}/X_h \rightarrow \mathrm{x}_1/X_1 \cdots \mathrm{x}_h/X_h \cdots \mathrm{x}_n/X_n : \{A_1, A_2, \ldots, A_{n-1}\}
\]

\begin{figure}[b]
 \begin{center}
\includegraphics{21-4ia6f28.eps}
 \end{center}
\caption{Edge combination tree}
\label{fig:EdgeCombinationTree}
\end{figure}

$A_i$ is an arc in the form of arc($a_i$, $X_k$, $X_l$), where $a_i$ is
an arcname, 1${\leq}k{\leq}n$, 1${\leq}l{\leq}n$, $k{\neq}l$. A set of
arcs \{$A_1, \ldots, A_{n-1}$\} satisfies the partial dependency
structure condition in Section \ref{sec:bunpoukisoku}.

The edges in the edge combination tree are expressed in diagrammatic
form using `・', neglecting the from and to positions. The arrows
between edges represent the edge combinations where the edge at the
source of the arrow is combined with the inactive edge attached to the
arrow to generate the resulting edge at the target position of the
arrow. For example, E$_{11}$ (Figure \ref{fig:EdgeCombinationTree} (a))
is combined with $\langle$*E x$_2$/n$_{21} \rightarrow \ldots \rangle$
((b)) to generate E$_{21}$ ((c)). Since `・' moves to the right
neighbor position, the depth of the tree, i.e., the number of arcs
from E$_0$ to each inactive edge, is equal to $n$, i.e., the number of
elements in the rule body in the grammar rule.

The phrase head (node) is bound to a variable in the active edge
during edge combination. The variable bindings are shown at the right
of the edge by \{ \}. For example, the combination of E$_0$ and the
edge (d) whose phrase head is n$_{11}$ generates E$_{11}$ which is
then combined with the edge (b), whose phrase head is n$_{21}$ to
produce E$_{21}$. As a result, the variable bindings of E$_{21}$ are
\{$X_1$:=n$_{11}$, $X_2$:=n$_{21}$\}.\footnote{Scope of a variable is
  within edge. } An arc whose governor and dependent are bound is
called a ``fixed arc'' and has a new unique arc-ID generated by
add\_arcid at Figure \ref{fig:ChartAlgotithm} (i).  Fixed arcs are
represented by a small letter such as a$_1$ in
Figure \ref{fig:EdgeCombinationTree} (e). One variable binding may
generate zero or more fixed arcs. The generated fixed arc has one
unique variable binding corresponding to one edge combination. This
edge combination generates one unique edge. This unique edge is called
the edge, which generated the fixed arc, or simply ``source edge'' of
the fixed arc, and is referred to as src\_Edge(a), where 'a' is a fixed
arc. For example, in the edge combination between (e) and (f) in
Figure \ref{fig:EdgeCombinationTree}, provided that the binding of the
node n$_{im}$ (let it [like]-pre-3) to the variable $X_i$ generates
the fixed arc a$_i$ (let it arc(pre-28, [like]-pre-3, [time]-v-0))
from the unfixed arc $A_i$ (arc(pre, $X_i$, [time]-v-0)), the edge
that generated the fixed arc a$_i$, i.e., src\_Edge(a$_i$) is
E$_{im}$ in Figure \ref{fig:EdgeCombinationTree} (g).

Each inactive edge (leaf of the combination tree)
(e.g., Figure \ref{fig:EdgeCombinationTree} (h)) has only fixed arcs
because all variables including phrase head variable in the edge are
bound, due to the partial dependency structure condition. An inactive edge
represents a result of a sequence of variable bindings, caused by the
edge bindings from the root to the leaf of the edge combination
tree. The following shows the definitions of terms and relations
related to the edge combination tree.

\vspace{0.5\Cvs}
\noindent
\fbox{
\begin{minipage}{405pt}
\hangafter=1\hangindent=1em
\textbf{fixed arc}: Arc whose governor and dependent nodes are fixed by the variable bindings caused by edge combinations.

\hangafter=1\hangindent=1em
\textbf{src\_Edge($a$) (source edge)}: The active or inactive edge that generated a fixed arc $a$. Mapping from a fixed arc to its corresponding edge is one to one, whereas the reverse is 1 to 0 - many.

\hangafter=1\hangindent=1em
\textbf{$X \xrightarrow[{\rm CT}]{*} Y$ (origin)}: Edge $X$ is located on a route from the top node to edge $Y$ or $X=Y$. $X$ is called an origin of $Y$.

\hangafter=1\hangindent=1em
\textbf{origin relation}: Edges $X$, $Y$ in an edge combination tree CT are said to be in origin relation if $X {\xrightarrow[{\rm CT}]{*}} Y$ or $Y {\xrightarrow[{\rm CT}]{*}} X$ is true.

\hangafter=1\hangindent=1em
\textbf{edge($a$, $DT$) (corresponding edge)}: The corresponding edge for a fixed arc $a$ and a well-formed dependency tree $DT$ is a single edge $e$ that satisfies the following condition (defined in Lemma \ref{lem:app2-2}):\\
\qquad $DT {\supseteq} {\rm ds(}e{\rm )}, {\rm a}{\in}{\rm ds(}e{\rm )}$
\end{minipage}
}
\vspace{0.5\Cvs}

\noindent
According to the structure of the edge combination tree
described above, two fixed arcs a$_i$ and a$_j$ have the following
relations.

\begin{lemma}[Relation between arcs in one partial dependency tree]
\label{lem:TwoArcsInOneDS}
Let a$_i$ and a$_j$ be fixed arcs a$_i$, a$_j \in$ds($e$), where $e$ is a single edge. Their source edges src\_Edge(a$_i$) and src\_Edge(a$_j$) are in origin relation.
\end{lemma}

\noindent
A fixed arc in edge $e$ in the edge combination tree is
in the edges that have $e$ as their origins. For example, fixed arc
a$_i$ generated at (g) is contained in (h).

\begin{lemma}[Relation between an arc and the edge that generated the arc]
\label{lem:app2-1}
\hspace{-0.5em}Suppose 
\linebreak
that fixed arcs a$_i$, a$_j$ satisfy
src\_Edge(a$_i$)${\xrightarrow[{\rm CT}]{*}}$src\_Edge(a$_j$).
Then a$_j \in$ds(*e) implies a$_i \in$ ds(*e) for every single arc *e
in PF (*e${\in}$*E, *E${\in}$PF).
\end{lemma}

Since a unique arc-ID is assigned to each arc, all inactive single
edges in the inactive packed edges in an arc combination tree, i.e.,
the leaves of the tree *E$_{n1}$ ${\cdots}$ *E$_{no}$ ${\ldots}$
*E$_{nw}$ in 
\linebreak
Figure~\ref{fig:EdgeCombinationTree}, have different partial
dependency trees. Therefore, ds(*e$_i$)${\neq}$ds(*e$_j$) for
arbitrary arcs *e$_i$, *e$_j$(*e$_i \neq$*e$_j$) in PF. A single
inactive edge and a partial dependency tree have one-to-one mapping.
In the parsing process, inactive edges that satisfy the conditions
shown in Figure \ref{fig:ChartAlgotithm} (j) are merged into one, and
this merged edge becomes an element of the phrase structure
forest. The one-to-one mapping relation between a single edge and a
partial dependency tree is assured in the phrase structure forest
because this merge operation does not change the dependency structures
in the single edges. Moreover, the following lemma related to the
phrase structure forest is established with respect to the C-Matrix
setting condition (C2).

\begin{lemma}[Constraints with respect to the arcs in the edges included in a path]
\label{lem:ArcConstraintOfArcsOnOnePath}
Suppose arcs a$_i \in$ds(e$_i$), a$_j\in$ds(e$_j$). If e$_i \xrightarrow[{\rm PF}]{+}$e$_j$, then dep(a$_i$)${\neq}$dep(a$_j$) is true. Inversely, if dep(a$_i$)$=$dep(a$_j$), then ${\neg}$(e$_i \xrightarrow[{\rm PF}]{+}$e$_j$) is true.

\proof{
This lemma is established because data structures in the single edges are trees whose top nodes are phrase heads, according to the partial dependency structure condition in \ref{sec:bunpoukisoku}.\\
}
\end{lemma}

\begin{lemma}[Existence of corresponding edge]
\label{lem:app2-2}
Suppose an arc a$_i$ in a well-formed dependency tree DT
(a$_i \in$DT). There exists only one packed edge E${\in}$PF
and single edge e${\in}$E that satisfy the following condition:

$\mathrm{DT} \supseteq \mathrm{ds}(e)$, $\mathrm{a}_i \in \mathrm{ds}(e)$

\proof{
Let the number of nodes in DT be $n$ (Number of arcs is $n-1$). Divide arc
set DT into the following two arc sets IN\_ARCS, OUT\_ARCS with respect
to a$_i$.

IN\_ARCS = \{ $a_j$ ${\mid}$ src\_Edge($a_i$)${\xrightarrow[{\rm CT}]{*}}$src\_Edge($a_j$) or src\_Edge($a_j$)${\xrightarrow[{\rm CT}]{*}}$src\_Edge($a_i$)\}

OUT\_ARCS = DT$-$IN\_ARCS

\noindent
Let SRC\_EDGES be a set of source edges corresponding to the arcs in IN\_ARCS.

SRC\_EDGES = \{$E$ ${\mid}$ $E$=src\_Edge($a$), $a{\in}$IN\_ARCS\}

\noindent
First, the following statement is to be established.
\[
\text{``Arbitrary edges in SRC\_EDGES are in origin relation''} \eqno{(A)}
\]
Suppose edges $U$, $V{\in}$SRC\_EDGES. $U$ and $V$ are in origin
relation by definition and in one of the following three cases.

\begin{itemize}
\item[(a)] One arc is an origin of E$_i$, and E$_i$ is an origin of the other arc. $U{\xrightarrow[{\rm CT}]{*}}$ E$_i$, E$_i \xrightarrow[{\rm CT}]{*} V$
\item[(b)] Both arcs are origins of E$_i$. $U{\xrightarrow[{\rm CT}]{*}}$ E$_i$, $V{\xrightarrow[{\rm CT}]{*}}$E$_i$
\item[(c)] E$_i$ is an origin of both arcs. E$_i \xrightarrow[{\rm CT}]{*}U$, E$_i \xrightarrow[{\rm CT}]{*}V$
\end{itemize}
 
\noindent
In cases (a) and (b), statement (A) obviously holds,
according to the structure of the edge combination tree. The following
shows that a contradiction is derived from the assumption that $U$ and
$V$ are not in origin relation in case (c).

Suppose that $U$ and $V$ are not in origin relation. There exist arcs
a$_u$, a$_v \in$DT such that $U$=src\_Edge(a$_u$),
$V$=src\_Edge(a$_v$), according to the premise. Since all arcs in DT
are in the packed shared forest, there exist inactive edges that have
$U$ and $V$ as their origin, respectively. Let *E$_u$ and *E$_v$ be
origins of $U$ and $V$, respectively. Let e$_u$ in *E$_u$ and *e$_v$
in *E$_v$ be edges such that a$_u \in$*e$_u$,
a$_v \in$*e$_v$. From Lemma \ref{lem:app2-1}, both *e$_u$ and
*e$_v$ contain a$_i$. Arcs a$_u$ and a$_v$ in the well-formed
dependency tree DT satisfy any of the C-Matrix setting conditions
(C1) to (C3). However, a$_u$ and a$_v$ do not satisfy (C1) because the
contradiction for the assumption that $U$ and $V$ are not in origin
relation is deduced from (C1), i.e., the existence of $e$ such that
a$_u$, a$_v \in$ds($e$), according to Lemma
\ref{lem:TwoArcsInOneDS}. (C2) indicates that *e$_u \xrightarrow[{\rm
      PF}]{+}$*e$_v$ is true (The reverse case is shown in the same
way). Nodes included in arcs in ds(*e$_u$) are phrase heads of the
constituents in cs(*e$_u$), according to the partial dependency
structure condition. This implies that either dep(a$_i$) or gov(a$_i$)
of a$_i \in$ds(*e$_u$) is a node that locates outside the
coverage of *e$_v$. On the other hand, both dep(a$_i$) and gov(a$_i$)
must be in the coverage of *e$_v$ because a$_i \in$ds(*e$_v$) is
true. From this contradiction, a$_u$ and a$_v$ do not satisfy
(C2). The (C3), i.e., *e$_u \swarrow_{[PF]} \searrow$*e$_v$ ,
is not satisfied by a$_u$ and a$_v$ because the coverage of *e$_u$ and
*e$_v$ have to be overlapped, due to the premise that a$_i$ is in both
*e$_u$ and *e$_v$. From the above, the supposition that $U$ and $V$
are not in origin relation contradicts the C-Matrix setting conditions
between a$_u$ and a$_v$. Therefore, statement (A) is true.

\noindent
Now, let E$_{last}$ be the last edge connected from
E$_i$, i.e., the edge that satisfies the following conditions:

E$_i \xrightarrow[{\rm CT}]{*}$E$_{last}$ 

E$_{last}$ is the only edge E$_j$ such that
E$_{last}{\xrightarrow[{\rm CT}]{*}}$E$_j$ (E$_j \in$SRC\_EDGES)

\noindent
Figure \ref{fig:CorrespondenceDT2Edge} shows the relation between
IN\_ARCS and SRC\_EDGES diagrammatically. E$_{start}$ corresponds to a
grammar rule. The grammar rule is as follows:

\begin{center}
y/$X_h$ $\rightarrow$ x$_1$/$X_1$ ${\cdots}$ x$_z$/$X_z$ :\{$A_1$, ${\ldots}$, $A_{z-1}$\}
\end{center}

\noindent
SRC\_EDGES constitutes a route on the edge combination tree CT with
root E$_{start}$, containing E$_1$, ${\ldots}$, E$_{last}$. The source
edge E$_i$ for a$_i$ exists somewhere on this route. There exists at
least one arc a$_{last}$ that is generated by E$_{last}$ in IN\_ARCS.

\begin{figure}[b]
 \begin{center}
\includegraphics{21-4ia6f29.eps}
 \end{center}
\caption{Edges corresponding to arcs in DT}
\label{fig:CorrespondenceDT2Edge}
\end{figure}

E$_{last}$ is either an active edge or an inactive
edge. Figure \ref{fig:CorrespondenceDT2Edge} shows a case where
E$_{last}$ is an active edge. E$_{last}$ is proved to be an inactive
edge as follows:

Suppose that E$_{last}$ is an active edge. As shown in
Figure \ref{fig:ActiveEdgeLast}, E$_{last}$ (Figure \ref{fig:ActiveEdgeLast}
(a)) has at least one remaining constituent x$_{u+1}$
(Figure \ref{fig:ActiveEdgeLast} (b). Variable for the constituent is not
shown). Let s1 and t1 be a from-position and a to-position of
E$_{last}$, respectively.

From the premise a$_{last}{\in}$DT, there exists at least one inactive
edge *E$_x$ (Figure \ref{fig:MapFromOnePTToTwoDTs} (c)) that has
E$_{last}$, the source edge of a$_{last}$, as its origin in the phrase
structure forest PF. As shown in the figure, *E$_x$ has the
from-position equal to the from-position s1 of E$_{last}$ and the
to-position greater than the to-position t1 of E$_{last}$.

\begin{figure}[t]
 \begin{center}
\includegraphics{21-4ia6f30.eps}
 \end{center}
\caption{Existence of corresponding edge}
\label{fig:ActiveEdgeLast}
\end{figure}

Consider the node n$_{t1+1}$ at the position t1+1
(Figure \ref{fig:ActiveEdgeLast} (d)).\footnote{One n$_{t1+1}$ exists due
  to the well-formedness condition of DT.} DT has one arc
\linebreak
$\mathrm{a}_\mathit{next}$(dep($\mathrm{a}_\mathit{next})=\mathrm{n}_{t1+1}$ (Figure \ref{fig:ActiveEdgeLast}
(e)). a$_{next}$ is an element of OUT\_ARCS, as per the definition of
a$_{last}$. a$_{next}{\in}$DT implies that at least one inactive edge
*E$_y$ whose origin is E$_{next}$, the source edge of a$_{next}$,
exists in the phrase structure forest PF. Since E$_{next}$ has
a$_{next}$ within its coverage, E$_{next}$ has the from-position s1
less than or equal to t1 (Figure \ref{fig:ActiveEdgeLast} (f)), and the
to-position t2 is greater than or equal to t1+1
(Figure \ref{fig:ActiveEdgeLast} (g)). Therefore, the from-position of
*E$_y$ is less than or equal to t1. From the above, *E$_x$, which has
its origin E$_{last}$, overlaps *E$_y$, which has its origin E$_{next}$
at the t1 position.

Now, no co-occurrence setting conditions for a$_{last}$ and a$_{next}$ hold:

Arcs a$_{last}$ and a$_{next}$ do not satisfy (C3) because *E$_x$ and
*E$_y$ overlap, as explained above. (C1) is not satisfied from the
premise *E$_x{\neq}$*E$_y$. Consider (C2) meaning that
*E$_x \xrightarrow[{\rm PF}]{+}$*E$_y$ is true (the reverse case is
shown in the same way). Let a$_m$ be the arc in *E$_x$ whose
dependent node is n$_{t1+1}$. a$_m\neq$a$_{next}$.\footnote{If
  a$_m =$a$_{next}$, src\_Edge(a$_{next}$) and E$_i$ are in origin
  relation. This contradicts the premise
  a$_{next}{\in}$OUT\_ARCS.} According to Lemma
\ref{lem:ArcConstraintOfArcsOnOnePath}, if *E$_x \xrightarrow[{\rm
      PF}]{+}$*E$_y$ is not true, then (C2) is not satisfied. From
the above, a$_{last}$ and a$_{next}$ satisfy no co-occurrence setting
conditions. This contradicts the premise that DT is a well-formed
dependency tree. Therefore, E$_{last}$ is not an active edge.

\noindent
Let E$_{last}$ be an inactive edge *E$_{last}$.

\begin{itemize}
\item[(a)] If *E$_{last}$, the source edge of a$_{last}$, is an inactive edge (a leaf of the edge combination tree), then no other packed edges contain a$_{last}$. Moreover, there exists only one *e$_{last}$ such that a$_{last}{\in}$ds(*e$_{last}$), *e$_{last}{\in}$*E$_{last}$.
\item[(b)] *E$_{last}{\in}$PF is true, due to a$_{last}{\in}$DT.
\item[(c)] According to the premise a$_{last}{\in}$DT and Lemma \ref{lem:app2-1}, DT$⊇$ds(*e$_{last}$) is true.
\end{itemize}

\noindent
Lemma \ref{lem:app2-1} is true due to (a) to (c). 

}

\end{lemma}



\subsection*{[Relation between Connected Arcs and their Corresponding Edges]}

Arcs a$_i$, a$_j$(a$_i\xrightarrow[{\rm DT}]{1}$a$_j$ or
sib(a$_i$, a$_j$)) are called ``connected arcs.'' The following two
lemmas are established with respect to connected arcs in a well-formed
dependency tree DT.

\begin{lemma}[Connected arcs and their corresponding edges]
\label{lem:app2-3}
Suppose *e$_i=\mathrm{edge}(\mathrm{a}_i, \mathrm{DT})$,*e$_j=\mathrm{edge}(\mathrm{a}_j, \mathrm{DT})$ for connected
arcs a$_i$, a$_j$ in DT. At least one of (a), (b), (c) is true.
\begin{itemize}
\item[(a)]*e$_i=$*e$_j$
\item[(b)]*e$_i{\xrightarrow[{\rm PF}]{+}}$*e$_j$
\item[(c)]*e$_j{\xrightarrow[{\rm PF}]{+}}$*e$_i$
\end{itemize}

Lemma \ref{lem:app2-3} means that if two arcs in DT are connected, one
of their corresponding edges is reachable from another edge in PF.

\proof{

*e$_i$ and *e$_j$ satisfy at least one of the C-Matrix setting
conditions (r1) to (r3), because *e$_i$ and *e$_j$ co-occur in DT.

\begin{itemize}
\item[(r1)] *e$_i$=*e$_j$
\item[(r2)] *e$_i \xrightarrow[\mathrm{PF}]{+}$*e$_j$ or *e$_j \xrightarrow[\mathrm{PF}]{+}$*e$_i$
\item[(r3)] *e$_i \swarrow_{[PF]} \searrow$ *e$_j$
\end{itemize}

Let $n$ be a node shared by the connected arcs a$_i$ and a$_j$. Both
*e$_i$ and *e$_j$ cover $n$. Therefore, (r3) is not satisfied by
*e$_i$ and *e$_j$. *e$_i$ and *e$_j$ have to satisfy (r1) or (r2).  }
\end{lemma}

\begin{lemma}[Ancestor-descendant arcs and their corresponding arcs]
\label{lem:app2-4}
Suppose *e$_i=\mathrm{edge}(\mathrm{a}_i, \mathrm{DT})$,*e$_j=\mathrm{edge}(\mathrm{a}_j, \mathrm{DT})$ for a$_i$, a$_j$(a$_i\xrightarrow[{\rm DT}]{+}$a$_j$). 

*e$_i{\xrightarrow[{\rm PF}]{*}}$*e$_j$

\proof{
In the case that dep(a$_i$)=gov(a$_j$), one of (a), (b), or (c) in
Lemma \ref{lem:app2-3} is true. From the node positioning relation
prescribed by the partial dependency structure condition in Section
\ref{sec:bunpoukisoku}, (c) is not satisfied by a$_i$ and a$_j$
because ${\neg}$(*e$_j \xrightarrow[{\rm PF}]{+}$*e$_i$) is true
for *e$_j$ and *e$_i$. Therefore, parent-child arcs satisfy either (a)
or (b). Lemma \ref{lem:app2-4} is established for
a$_i \xrightarrow[{\rm DT}]{*}$a$_j$, due to the associativity of
the relation ${\xrightarrow[{\rm PF}]{*}}$.

}
\end{lemma}


\subsection*{[Top single edge top\_edge(DT)]}

\begin{definition}\label{def:TopSingEdge}[Top single edge]

A ``top single edge'' for DT top\_edge(DT) is the single edge that
locates at the topmost position in PF among the edges that correspond
to the arc just under the top node of DT. That is, top\_edge(DT) is
edge a$_i$ satisfying the following conditions.

top\_node(DT)$=$gov(a$_i$) 

edge(a$_i$)${\xrightarrow[{\rm PF}]{*}}$edge(a$_j$) for all a$_j$ such that top\_node(DT)$=$gov(a$_j$)

\end{definition}

If DT is a tree consisting of one node, top\_node(DT) is the single
lexical edge corresponding to the node.

\begin{lemma}[Relation between top\_edge(DT) and edge(a$_j$, DT)]
\label{lem:RelBetweenTopedgeAndEdge}

*e$_t \xrightarrow[{\rm PF}]{*}$*e$_j$ is true for *e$_t$=top\_edge(DT) and *e$_j$=edge(a$_j$, DT) (a$_j \in$DT).

\proof{
If a$_j$ is an arc just under the top node of DT, i.e.,
gov(a$_j$)$=$top\_node(DT), *e$_t\xrightarrow[{\rm PF}]{*}$*e$_j$,
is true, according to Lemma \ref{lem:app2-3} and the definition of
top\_edge. If not, a$_j$ is a descendant of one of the arcs just under
the top node of DT. *e$_t\xrightarrow[{\rm PF}]{*}$*e$_j$ is true,
according to Lemma \ref{lem:app2-4}.\\
}
\end{lemma}


\subsection*{[Division of Well-formed Dependency Tree]}

The ``division of a well-formed dependency tree DT'' means the
creation of a set of partial dependency trees DT$_1$, ${\ldots}$, DT$_m$
by removing a set of arcs in ds(top\_edge(DT)) from DT, where $m$ is the
number of nodes in ds(top\_edge(DT)). Nodes isolated from all other
nodes by this operation are dependency trees that consist of one
node. For example, suppose that
ds(top\_edge(DT))=\{a$_s$, a$_t$, a$_u$, a$_w$\} in
Figure \ref{fig:DTGraphDivision}, DT is divided into partial dependency
trees DT$_s$, DT$_t$, DT$_u$, DT$_v$, DT$_w$ whose top nodes are
n$_s$, n$_t$, n$_u$, n$_v$, n$_w$, respectively. Since nodes n$_s$, n$_w$
are isolated from other nodes, DT$_s$ and DT$_w$ are dependency trees
consisting of a single node, i.e., \{n$_s$\} and \{n$_w$\},
respectively. The phrase heads of the packed edges in cs($e$) of a
single edge $e$ have one-to-one correspondence with the nodes in the
partial dependency tree df($e$), due to the partial dependency
structure condition in Section \ref{sec:bunpoukisoku}. Therefore,
there exists one packed edge *E$_i$ (1${\leq}i{\leq}m$) whose phrase
head is a top node n$_i$ of DT$_i$(1${\leq}i{\leq}m$). *E$_i$ is
called a ``root packed edge'' for DT$_i$ and is referred to as
root\_Edge(DT$_i$).

\begin{figure}[b]
 \begin{center}
\includegraphics{21-4ia6f31.eps}
 \end{center}
\caption{Division of well-formed DT}
\label{fig:DTGraphDivision}
\end{figure}

\begin{definition}\label{def:RootPackedEdge}[Root packed edge] 

Suppose DT$_i$ and its top node n$_i$ (1${\leq}i{\leq}m$) is obtained
by the division of DT. A root\_Edge(DT$_i$) is a packed edge that is
an element of cs(top\_edge(DT)) and whose phrase head is
top\_node(DT$_i$).
\end{definition}

For example, in Figure \ref{fig:DTGraphDivision}, *e$_v$ is top\_edge(DT)
and has the constituent sequence *E$_s$, *E$_t$, *E$_u$, *E$_v$, *E$_w$,
whose phrase heads are n$_s$, n$_t$, n$_u$, n$_v$, n$_w$,
respectively. Then, root\_Edge(DT$_t$) is *E$_t$. The following two
lemmas are explained with reference to
Figure \ref{fig:DTiObtainedFromDT}.

\begin{figure}[b]
\begin{center}
\includegraphics{21-4ia6f32.eps}
 \end{center}
\caption{Well-formed DT$_i$ obtained from division of DT}
\label{fig:DTiObtainedFromDT}
\end{figure}

\begin{lemma}[Relation between root packed edge and top single edge]
\label{lem:app2RootEdgeAndTopEdge}
\hspace{-0.5em}Suppose 
partial dependency trees DT$_i$, obtained from the division of
the well-formed dependency tree DT. *E$_i \xrightarrow[{\rm
      PF}]{+}$*e$_{o}$ is true for the root packed edge
*E$_i=$root\_Edge(DT$_i$) and the top single edge
*e$_{o}=$top\_edge(DT$_i$).

\proof{
Let *e$_t$ be top\_edge(DT) and n$_i$ be the top node of DT$_i$
(Figure \ref{fig:DTiObtainedFromDT}). Consider the case where DT is a
tree consisting of a single node, i.e., DT$i=$\{n$_i$\}. *e$_{o}$ is
the single lexical edge corresponding to
n$_i$. *E$_i\xrightarrow[{\rm PF}]{+}$*e$_{o}$ is true because the
phrase head of *E$_i$ is n$_i$. Consider another case where DT is a
tree consisting of arcs. *e$_t\xrightarrow[{\rm PF}]{*}$*e$_{o}$
is true, according to Lemma \ref{lem:app2-4}. Therefore,
X${\xrightarrow[{\rm PF}]{*}}$*e$_{o}$ is true for some packed edge X
in cs(*e$_t$). X$=$*E$_i$ is true because *E$_i \in$cs(*e$_t$) from
the definition and the phrase heads of *E$_i$ and *e$_{o}$ are the
same.

}
\end{lemma}

\begin{lemma}[Well-formedness of partial trees obtained by division of DT]
\label{lem:app2-6}
Suppose 
\linebreak
partial dependency trees DT$_i$ and its root packed edge
*E$_i$ with the from-position sp$_i$ and to-position tp$_i$ are obtained
from the division of the well-formed dependency tree DT. DT$_i$ is a
well-formed dependency tree that covers from sp$_i$ to tp$_i$.

\proof{
DT$_i$ is a well-formed dependency tree if it satisfies the
co-occurrence constraint and the well-covering constraint. Obviously
DT$_i$ satisfies the co-occurrence constraint because DT satisfies the
co-occurrence constraint. The following part shows the well-covering
constraint.

In the case that DT is a tree consisting of a single node, it
satisfies the well-covering constraint from the definition. Consider
the case where DT is a tree consisting of arcs. Let n$_i$ and n$_j$ be
the top node of DT and one of any other nodes in DT
(n$_j \neq$n$_i$). There exists arc a$_j$ such that
n$_j=dep(a_j$) in DT. Moreover, there exists a$_k\in$DT$_i$ such
that gov(a$_k$)$=$n$_i$, a$_k \xrightarrow[{\rm DT}]{*}$a$_j$ for
a$_j \in$DT$_i$. Let *e$_j$ and *e$_k$ be *e$_j=$edge(a$_j$, DT),
*e$_k=$edge(a$_k$, DT). *e$_k\xrightarrow[{\rm PF}]{*}$*e$_j$ is
true due to Lemma \ref{lem:app2-4}, because a$_k$ is equal to a$_j$ or
a$_k$ is an ancestor of a$_j$. Now, let *e$_t$ be the top single node
of DT, *e$_t \xrightarrow[{\rm PF}]{*}$*e$_k$ is true according to
Lemma \ref{lem:RelBetweenTopedgeAndEdge}. *e$_k$ is reachable from one
of the packed edges in cs(*e$_t$). *E$_i \xrightarrow[{\rm
      PF}]{*}$*e$_k$ is true because the phrase head of *e$_k$ is
n$_i$.

From the above, *E$_i\xrightarrow[{\rm
      PF}]{*}$*e$_k \xrightarrow[{\rm PF}]{*}$*e$_j$ is true and
n$_j$ is in the coverage of *E$_i$; that is, all nodes in DT$_i$ are
in the coverage from sp$_i$ to tp$_i$. Furthermore, the nodes in
DT$_k$($k{\neq}i$) are not in the coverage from sp$_i$ to
tp$_i$. Since DT satisfies the well-covering constraint, all nodes in
DT$_i$ occupy whole positions from sp$_i$ to tp$_i$.

}
\end{lemma}


\subsection*{[Proof of the Completeness and Soundness of the Dependency Forest]}

A corresponding dependency tree dependency\_tree(PT) for a phrase
structure tree PT = \{*e$_1$, ${\ldots}$, *e$_m$\} is defined as
follows:

\begin{definition}\label{def:DTforPT}[Dependency tree for a phrase structure tree PT]
dependency\_tree(PT) = ds(*e$_1$) ${\uplus}$ ${\cdots}$ ${\uplus}$ ds(*e$_m$) 

\noindent
The operator ${\uplus}$ is similar to the union operator
${\cup}$ , which is introduced to manage the union of dependency
structures that may be either a set of arcs or a set of a
node. ${\uplus}$ removes nodes from the union of dependency structures
if it has at least one arc. The following are examples of ${\uplus}$,
where n$_i$ and a$_i$ represent node and arc, respectively.

\begin{itemize}
\item[] \{n$_1$\}${\uplus}$\{a$_1$, a$_2$\} = \{a$_1$, a$_2$\}
\item[] \{a$_1$\}${\uplus}$\{a$_2$, a$_3$\} = \{a$_1$, a$_2$, a$_3$\}
\item[] \{n$_1$\}${\uplus}$\{\} = \{n$_1$\}
\end{itemize}

\end{definition}

\noindent
dependency\_tree(PT) is a tree because it is consructed
by combining each partial dependency tree ds(*e$_i$).

\begin{theorem}[The completeness of the dependency forest]
\label{the:CompletenessOfPF}
Let PT be a phrase structure tree in the phrase structure forest
PF. DT=dependency\_tree(PT) is a well-formed dependency tree in the
dependency forest DF.

\proof{
From the definition of the dependency forest, DT is included in
DG. Nodes contained in DT and PT have a one-to-one relation, according
to the partial dependency structure condition. Since PT covers the
whole sentence, DT is a well-covered dependency tree. According to the
C-Matrix setting conditions, every two arcs in DT satisfy the
co-occurrence constraint. Therefore, dependency\_tree(PT) is a
well-covering and well-co-occurred dependency tree in DF.\\
}
\end{theorem}

\begin{theorem}[The soundness of the dependency forest]
\label{the:SoundnessOfPF}
Let DT be a well-formed dependency tree in the dependency forest
DT. There exists a phrase structure tree PT in the phrase structure
forest PF such that DT=dependency\_tree(PT).

\proof{
The existence of a phrase structure tree PT that satisfies
PT${\in}$ps\_trees(*E$_{root}$) and dependency\_tree(PT)=DT is shown
below.

Let $n$ be a number of input words. The following is an algorithm,
called the phrase structure tree generation algorithm, which generates
a phrase structure tree from a packed edge *E$_r$ with from-position
sp$_r$ and to-position tp$_r$ (1${\leq}$sp$_r <$tp$_r{\leq}$n) and a
well-formed dependency tree DT, which covers from sp$_r$ to
tp$_r$. The proof that the phrase structure tree generation algorithm
generates a phrase structure tree that satisfies the above
conditions is shown below using mathematical induction for the number
of arcs in the dependency tree.

\vspace{0.5\Cvs}
\noindent
\fbox{
\begin{minipage}{405pt}
\setlength{\parindent}{0pt}
[Phrase Structure Tree Generation Algorithm]

In the case that DT is a set of arcs:

\hangafter=1\hangindent=2em
\quad A-Step1(Identification of the Top Single Edge): Let *e$_t$ be the top single edge top\_edge(DT).

\hangafter=1\hangindent=2em
\quad A-Step2 (Identification of a Path): Identify a path from *E$_r$ to *e$_t$. Let PATH be a set of single edges in the path except for *e$_t$.

\hangafter=1\hangindent=2em
\quad A-Step3 (Division of DT): Divide DT by removing edges in ds(top\_edge(DT)) to get a set of partial dependency trees DT$_i$($1{\leq}i{\leq}m$) and root packed edges *E$_i$=root\_Edge(DT$_i$).

\hangafter=1\hangindent=2em
\quad A-Step4 (Computation of Partial Phrase Structure Trees): Apply the phrase structure tree generation algorithm to each DT$_i$ and *E$_i$($1{\leq}i{\leq}m$) and compute each PT$_i$.

\hangafter=1\hangindent=2em
\quad A-Step5 (Construction of Phrase Structure Tree): Returns PT=PATH ${\cup}$ \{*e$_t$\} ${\cup}$ PT$_1$ ${\cup}$ ${\cdots}$ ${\cup}$PT$_m$ as a phrase structure tree for DT, *E$_r$.

In the case that DT is a set of a node (DT=\{n\}):

\hangafter=1\hangindent=2em
\quad N-Step1 (Identification of Lexical Edge): Identify the lexical edge @e$_{lex}$ that generated node n.

\hangafter=1\hangindent=2em
\quad N-Step2 (Identification of a Path): Identify a path from *E$_r$ to e$_{lex}$ and return a set of single edges in the path as a phrase structure tree for DT, *E$_r$.

\end{minipage}
}
\vspace{0.5\Cvs}

\noindent
When DT is s a set of arcs, the phrase structure tree PT is
constructed through A-Step1 to
A-Step5. Figure \ref{fig:DttoParseTreeExplain} shows the behavior of the
algorithm diagrammatically. A-Step1 computes the top single edge
*e$_t$($=$top\_edge(DT)) (Figure \ref{fig:DttoParseTreeExplain}
(S1)). A-Step2 identifies a path from *E$_r$ to *e$_t$ and computes
PATH, a set of single edges. The existence of this path
(*E$_r \xrightarrow[{\rm PF}]{+}$*e$_t$) is assured as follows:

\begin{figure}[b]
 \begin{center}
\includegraphics{21-4ia6f33.eps}
 \end{center}
\caption{Generation of PT from DT and *E$_r$}
\label{fig:DttoParseTreeExplain}
\end{figure}

In the case that *E$_r$ is *E$_{root}$, it is obvious. In the case
that *E$_r$ is obtained by the division of a dependency graph (*E$_i$
in A-Step4), it is assured by Lemma
\ref{lem:app2RootEdgeAndTopEdge}. The dependency structure parts of
the single edges in PATH are \{\}. This is obvious because *E$_r$ and
*e$_t$ have the same coverage since all nodes in DT are in the
coverage of *e$_t$.

\[
\text{dependency\_tree(PATH)} = \{\} \eqno{(A)}
\]
A-Step3 performs the division of DT and generates DT$_i$,
*E$_i$($1{\leq}i{\leq}m$), as shown in
Figure \ref{fig:DttoParseTreeExplain} (S3). According to Lemma
\ref{lem:app2-6}, DT$_i$ is a well-formed dependency tree covering the
coverage of *E$_i$ and the phrase structure tree generation algorithm
is applicable recursively at A-Step4. A-Step5 computes the phrase
structure tree PT (Figure \ref{fig:DttoParseTreeExplain} (S5)). From the
definition of the phrase structure tree, it is obvious that PT is a
phrase structure tree if each PT$_i$ is a phrase structure tree.

When DT is a set of a node (DT=\{n\}), N-Step1 and N-Step2 compute a
phrase structure tree PT. The existence of a path from *E$_r$ to
e$_{lex}$ is assured for the same reason described in the explanation
of A-Step1.

The phrase structure tree PT generated by the phrase structure tree
generation algorithm satisfying DT=dependency\_tree(PT) is shown as
follows:

In the case that DT=\{n$_r$\}, the algorithm generates PT at
N-Step2. PT is a phrase structure tree containing one node n$_r$. From
the definition of dependency\_tree, dependency\_tree(PT)=\{n$_r$\} is
true. In the case that DT is a dependency tree that consists of a set
of arcs, the phrase structure tree generation algorithm, the
definition of dependency\_tree, and (A) give the following equation.
\begin{align*}
 & \text{dependency\_tree(PT)}\\
 & \quad = \text{dependency\_tree}(\mathrm{PATH} \cup \{\mathrm{*e}_t\} \cup \mathrm{PT}_1 \cup \cdots \cup \mathrm{PT}_m) \\
 & \quad = \text{dependency\_tree}(\mathrm{PATH}) \uplus \text{dependency\_tree}(\{\mathrm{*e}_t\}) {\uplus}\text{dependency\_tree}(\mathrm{PT}_1)\\
 & \qquad \uplus \cdots \uplus \text{dependency\_tree}(\mathrm{PT}_m)\\
 & \quad = \text{dependency\_tree}(\{\mathrm{*e}_t\}) \uplus \text{dependency\_tree}(\mathrm{PT}_1) \uplus \cdots \uplus \text{dependency\_tree}(\mathrm{PT}_m)
\end{align*}
Assume that PT$_i$ corresponding to DT$_i$, *E$_i$ in A-Step4 satisfies the following.

dependency\_tree(PT$_i$)=DT$_i$  (1${\leq}i{\leq}$m)

\noindent
Now, PT generated at A-Step5 generates DT as shown below.
\begin{align*}
 & \text{dependency\_tree(PT)} \\
 & \quad = \mathrm{ds}_t \uplus \mathrm{DT}_1 \uplus \cdots \uplus \mathrm{DT}_m \\
 & \quad = \mathrm{DT}
\end{align*}
}
\end{theorem}

\begin{biography}

\bioauthor[:]{Hideki Hirakawa}{Hideki Hirakawa
  received the B.E. and M.E. degrees in electrical engineering from
  Kyoto University, Kyoto, Japan, in 1978 and 1980, respectively. He
  received the PhD degree of Information and Science Technology from
  Tokyo University, Tokyo, Japan, in 2007. In April 1980, he joined
  the Toshiba Corp, Kawasaki, Japan. From 1982 to 1984, he was a
  researcher at Institute for New Generation Computer Technology
  (ICOT), Tokyo, Japan. From 1994 to 1995, he worked as a research
  affiliate of the MIT Media Laboratory, Boston, U.S.A. He is
  currently a Chief Specialist of the Government \& External Relations
  Office, Toshiba Corp., Tokyo, Japan. He is a member of the
  Association for Natural Language Processing of Japan, the
  Information Processing Society of Japan, and the Japanese Society for
  Artificial Intelligence.}

\end{biography}

\biodate





\clearpage





















































\clearpage














































\end{document}
