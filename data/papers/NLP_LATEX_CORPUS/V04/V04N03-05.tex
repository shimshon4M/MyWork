



\documentstyle[epsf,jnlpbbl]{jnlp_j_b5}

\setcounter{page}{83}
\setcounter{巻数}{4}
\setcounter{号数}{3}
\setcounter{年}{1997}
\setcounter{月}{7}
\受付{1996}{12}{27}
\採録{1997}{3}{27}

\setcounter{secnumdepth}{5}

\title{文字連鎖を用いた複合語同音異義語誤りの検出手法とその評価}
\author{奥 雅博\affiref{NTT} \and 松岡 浩司\affiref{NTT}}

\headauthor{奥 雅博・松岡 浩司}
\headtitle{文字連鎖を用いた複合語同音異義語誤りの検出手法とその評価}

\affilabel{NTT}{NTT情報通信研究所}
{NTT Information and Communication Systems Laboratories}

\jabstract{
本論文では,文字連鎖を用いた複合語同音異義語誤りの検出手法とその評価について述べる.ワードプロセッサによって作成された日本語文書には,変換誤りに起因する同音異義語誤りが生じやすい.同音異義語誤りは,同じ読みの単語を誤った単語へと変換してしまう誤りである.このため,推敲支援システムにおいて同音異義語誤りを検出する機能を実現することは重要な課題の1つとなっている.我々は,意味的制約に基づく複合語同音異義語誤りの検出/訂正支援手法を提案した.しかし,この手法においてもいくつかの短所が存在する.本論文では,これらの短所を補うための手法として,文字連鎖を誤り検出知識として用いた複合語同音異義語誤りの検出手法について述べる.文字連鎖は,既存の文書を解析することなしに容易に収集することができる.また,本手法は文字連鎖のみを用いているので,複合語同音異義語誤りに限らず,文字削除誤りなどの別のタイプの誤りに適用することも可能である.
さらに本論文では,本手法の有効性を検証するために行った評価実験の結果についても述べ,意味的制約を用いた複合語同音異義語誤り検出/訂正支援手法との比較についても述べる.}

\jkeywords{日本文推敲支援,同音異義語誤り検出,文字連鎖,テキスト処理,自然言語処理}

\etitle{A Method for Detecting Japanese \\
Homophone Errors in Compound Nouns based on \\
Character Cooccurrence and Its Evaluation}
\eauthor{Masahiro Oku \affiref{NTT} \and Koji Matsuoka\affiref{NTT}} 

\eabstract{
Most Japanese texts are produced with Japanese word processors. As Japanese texts consist of phonograms, KANA, and ideograms, KANJI, Japanese word processors always use KANA-KANJI conversion in which KANA sequences input through the keyboard are converted into KANA-KANJI sequences. Therefore, Japanese texts suffer from homophone errors caused by erroneous KANA-KANJI conversion. A homophone error occurs when a KANA sequence is converted into the wrong word which has the same reading. Detecting homophone errors is an important topic in Japanese text revision support systems. We have already proposed a high performance method for handling Japanese homophone errors in compound nouns used in REVISE. The method, however, has some drawbacks. To compensate for these drawbacks, this paper describes a method for detecting Japanese homophone errors in compound nouns that uses character cooccurrence. Character cooccurrence can be easily collected from existing texts without any analysis. Therefore, this method can be used, in a Japanese revision support system, as a complementary method for handling Japanese homophone errors in compound nouns. Moreover, as this method depends only on character cooccurrence, it can be applied not only to homophone errors but also other types of errors such as character deletion.}

\ekeywords{Japanese Proofreading Support System, Homophone Error Detection, Character Cooccurrence, Text Processing, Natural Language Processing}

\begin{document}
\thispagestyle{myheadings}
\maketitle


\section{まえがき}
最近の文書作成はほとんどの場合,日本語ワードプロセッサ(ワープロ)を用いて行われている.これに伴い,ワープロ文書中に含まれる誤りを自動的に検出するシステムの研究が行われている~\cite{FukushimaAndOtakeAndOyamaAndShuto1986,Kuga1986,IkeharaAndYasudaAndShimazakiAndTakagi1987,SuzukiAndTakeda1989,OharaAndTakagiAndHayashiAndTakeishi1991,IkeharaAndOharaAndTakagi1993}.
ワープロの入力方法としては一般にかな漢字変換が用いられている.このため,ワープロによって作成された文書中には変換ミスに起因する同音異義語誤りが生じやすい.同音異義語誤りは,所望の単語と同じ読みを持つ別表記の単語へと誤って変換してしまう誤りである.従って,同音異義語誤りを自動的に検出する手法を確立することは,文書の誤り検出/訂正作業を支援するシステムにおいて重要な課題の1つとなっている.

同音異義語誤りを避けたり,同音異義語誤りを検出するために種々の方法が提案されている~\cite{FukushimaAndOtakeAndOyamaAndShuto1986,MakinoAndKizawa1981,Nakano1982,OshimaAndAbeAndYuuraAndTakeichi1986,SuzukiAndTakeda1989,TanakaAndMizutaniAndYoshida1984a,TanakaAndYoshida1987}.
われわれは,日本文推敲支援システムREVISE~\cite{OharaAndTakagiAndHayashiAndTakeishi1991}において,意味的制約に基づく複合語同音異義語誤りの検出/訂正支援手法を採用している~\cite{Oku1994,Oku1996}.この手法の基本的な考え方は,「複合語を構成する単語はその隣に来うる単語(隣接単語)を意味的に制約する」というものである(3章参照).しかしながら,この手法においても以下のような問題点があった;

\begin{description}
\item{(1)}
同音異義語ごとに前方/後方隣接単語に対する意味的制約を,誤り検出知識及び訂正支援のための知識として収集しなければならない.しかし,このような意味的制約を人手を介さずに自動的に収集することは困難である.
\item{(2)}
検出すべき同音異義語誤りを変更すると,意味的制約を記述した辞書を新たに構築する必要が生じる.
\end{description}

これらの問題点を解決するためには,誤り検出知識として収集が容易な情報を使用する必要がある.この条件に合致する情報の1つとして文書中の文字連鎖がある.文字連鎖の情報は既存の文書から容易に収集することができる.

3文字連鎖を用いてかな漢字変換の誤りを減らす手法については~\cite{TochinaiAndItoAndSuzuki1986}が報告されているが,この手法は漢字をすべて1つのキャラクタとして扱っているため,複合語に含まれる同音異義語誤りを検出することができない.
また,文字の2重マルコフ連鎖確率を用いて日本文の誤りを検出し,その訂正を支援する手法が提案されている~\cite{ArakiAndIkeharaAndTsukahara1993}.この手法は,「漢字仮名混じり文節中に誤字または誤挿入の文字列が存在するときは,m重マルコフ連鎖確率が一定区間だけ連続してあるしきい値以下の値を取る」という仮説に基づいて誤字,脱字及び誤挿入文字列の誤り種別及び位置を検出するものである.同音異義語誤りは単語単位の誤字と捉えることができるが,この手法が同音異義語誤りに対して有効であるか否かについては報告されていない.

一方,日本文推敲支援システムREVISEは,ルールに基づく形態素解析を基本にしたシステムであり,その中に誤り検出知識として収集が容易な統計的な情報を導入した誤り検出手法を確立することも重要な課題である.
そこで,本論文では,収集が容易な統計的な誤り検出知識として文字連鎖に焦点をあて,文字連鎖を用いた複合語同音異義語誤りの検出手法について述べる.さらに,その有効性を検証するために行った評価実験の結果についても述べる.

以下,2章において本論文で用いる用語の定義を行い,3章において日本文推敲支援システムREVISEにおける誤り検出の流れと,意味的制約に基づく複合語同音異義語誤りの検出手法の概要及びその問題点について述べる.4章では,3章で述べる問題点を解決するために,文字連鎖を用いて複合語に含まれる同音異義語誤りを検出する手法を提案する.5章では,本手法の有効性を評価するために行った同音異義語誤り検定の評価実験について述べ,意味的制約を用いた同音異義語誤り検出/訂正支援手法との比較を含めた考察を加える.

\section{用語の定義}
\begin{description}
\item[○複合語:]
2つ以上の語が助詞を介さずに複合して成立した語.
\item[○同音語:]
読みが同じで意味の異なる2つ以上の語の集合と定義する.表1に同音語の種類を示す~\cite{OshimaAndAbeAndYuuraAndTakeichi1986}.
\item[○同音異義語集合:]
同じ読みを持つ単語を要素とする集合であって,各要素が同品詞であり,かつ表記ゆらぎでないもの(表1参照)を同音異義語集合と定義する.例えば“科学”と“化学”は同じ同音異義語集合に属する.
\item[○同音異義語:]
一般には同音語と同じ意味で用いられるが,本論文では,同音異義語集合の各要素のことを同音異義語と定義する.単語Aが単語Bと同じ読みを持つとき,「単語Aは同音異義語Bを持つ」,「単語Aの同音異義語はBである」などと表現する.
\item[○同音異義語誤り:]
ある単語から,それと同じ同音異義語集合に属する単語への置換誤りを同音異義語誤りと定義する.表2に同音異義語誤りの例を示す.

\begin{figure}[htbp]
\begin{center}
\epsfile{file=table1-2.eps,scale=0.7}
\end{center}
\end{figure}

\item[○前方隣接単語,後方隣接単語,隣接単語:]
複合語において,ある単語の直前に位置する単語をその単語の前方隣接単語,直後に位置する単語をその単語の後方隣接単語とよぶ.また,隣接単語とは,前方隣接単語,後方隣接単語のいずれかを指す.
\item[○確定単語:]
ある単語が属する同音異義語集合の要素数が1であるとき,すなわち,ある単語に同音異義語が存在しないとき,この単語を確定単語とよぶ.
\item[○不確定単語:]
ある単語が属する同音異義語集合の要素数が2以上であるとき,すなわち,ある単語に同音異義語が存在するとき,この単語を不確定単語とよぶ.
\item[○意味属性:]
単語をその意味により有限個の概念に写像したもの.例えば,単語“自然”,“天然”はともに意味属性[自然]に属する.以降,意味属性を表すのに[ ]を用いる.
\item[○n文字連鎖:]
n文字連鎖とは,実文書に現れるn文字の並びである.誤りのない大量の文書からn文字の並び(n文字連鎖)を集めることによって,正しいn文字連鎖,すなわち,誤りを検出するための知識を収集することができる.
\end{description}

\section{意味的制約に基づく複合語同音異義語誤り検出の概要}
\subsection{日本文推敲支援システムREVISEにおける位置づけ}
日本文推敲支援システムREVISEの概略フロー図を図1に示す.REVISEは,入力された日本語文書に対して形態素解析処理を行う.次に綴り誤りや助詞抜けなどの誤りを誤り検出知識を参照して検定し,発見した誤りに対して訂正候補を提示する.
REVISEにおける誤り検出の基本的な考え方は,「正しい文に対して高い解析精度を持つ形態素解析を適用したとき,解析に失敗した箇所には誤りが含まれている可能性が高い」というものである.しかし,誤りの種類によっては形態素解析に成功するものが存在する.同音異義語誤り,特に複合語に含まれる同音異義語誤りはこの種の誤りの1つである.REVISEでは,あらかじめ用意しておいた特定の同音異義語が複合語に含まれるときに3.2節で述べる同音異義語誤り検定処理が起動される.


\begin{figure}[ht]
\begin{center}
\epsfile{file=fig1.eps,scale=0.7}
\end{center}
\end{figure}


\subsection{処理の概要と問題点}
従来より,ある単語に関係する語は限られており,特に複合語において隣接する単語の組合せは限られていることが指摘されている~\cite{TanakaAndMizutaniAndYoshida1984b,TanakaAndYoshida1987}.また,人間は前後一語の環境があれば,ある単語を認定することができ,特に複合語においてその傾向が顕著であると言われている~\cite{Nakano1982}.
これらのことは複合語において隣接する単語間には意味的制約が存在することを示唆している.そこで我々は,複合語において隣接する単語間に成立する意味的制約に着目して複合語に含まれる同音異義語誤りを検出する手法を提案した~\cite{Oku1994,Oku1996}.この手法は,同音異義語とこれに隣接する単語との間に成立する意味的制約のみを利用して複合語に含まれる同音異義語誤りを検出する.以下に例を用いてこの手法の概要を述べる.

図2に不確定単語“化学”を含む複合語“自然化学”における同音異義語誤りの検出/訂正候補推定の例を示す.ここで,“化学”は同音異義語“科学”を持つ不確定単語,“自然”は確定単語とする.なお,“化学”は“科学”のほかに同音異義語を持たないものとする.

不確定単語“化学”の前方に対する意味的制約を表す意味属性集合を$PS_1$,不確定単語“科学”の前方に対する意味的制約を表す意味属性集合を$PS_2$とする.複合語“自然化学”において不確定単語“化学”の前方隣接単語“自然”の持つ意味属性は[自然]であるが,図2に示すように不確定単語“化学”の前方に対する意味的制約を表す意味属性集合$PS_1$は,
\[[自然] \notin  PS_1 \]
を満足する.すなわち,不確定単語“化学”の前方に位置する“自然”は,不確定単語“化学”の持つ前方に対する意味的制約を満足しない.従って,不確定単語“化学”を複合語“自然化学”において同音異義語誤りとして検出することができる.

 次に同音異義語誤り“化学”に対する訂正候補を推定する過程に入る.同音異義語誤り“化学”に対する訂正候補は,その前方に対する意味的制約を表す意味属性集合に意味属性[自然]を含んでいなければならない.図2より,読み“かがく”を持つ“化学”と同音異義の関係にある“科学”の意味的制約を表す意味属性集合$PS_2$は,
\[[自然] \in  PS_2 \]
を満足する.すなわち,“自然”は不確定単語“化学”の同音異義語である“科学”の前方に対する意味的制約を満足する.従って,“化学”の同音異義語“科学”を訂正候補として推定し,ユーザに提示することができる.


\begin{figure}[ht]
\begin{center}
\epsfile{file=fig2.eps,scale=0.7}
\end{center}
\end{figure}


しかしながら,この手法には以下のような問題点が存在する;

\begin{description}
\item{(1)}
同音異義語ごとに前方/後方隣接単語に対する意味的制約を,誤り検出/訂正支援知識としてあらかじめ収集しておかなければならない.しかし,このような意味的制約を人手を介さずに自動的に収集することは困難である.すなわち,誤り検出/訂正支援知識の収集に大きな工数を要する.
\item{(2)}
検出すべき同音異義語誤りを変更すると,意味的制約を記述した辞書を新たに構築する必要が生じる.
\end{description}

\section{文字連鎖を用いた複合語同音異義語誤り検出手法の提案}
3.2節で述べた問題点を解決するために,本論文では誤り検出知識として収集が容易な情報である文字連鎖を用いた手法について提案する.
\subsection{基本的な考え方}
日本文推敲支援システムREVISEは,ルールベースの形態素解析によって複合語を単語単位に分割するので,その単語間の結びつきを見るには文字や単語の連鎖確率ではなく,連鎖そのものを調べればよい\footnote{連鎖確率を用いる場合には,形態素解析とともに誤り検出も行うという戦略がとられる.REVISEのようにルールベースの形態素解析が完了している状態で誤り検出を行うには,文字連鎖確率ではなく文字連鎖そのものを誤り検出知識として利用すれば十分であると考えられる.また,確率値は誤り検出の確からしさを計算するのには有効であろう.}.
また,単語連鎖を収集するには,誤り検出知識の収集の段階で,形態素解析を大量の文書に対して正確に行わなければならず,誤り検出知識を容易に収集するという目的に反する.一方,文字連鎖は,大量の文書から機械的にかつ容易に収集することができる.
以上のことから,本論文で述べる文字連鎖を用いた複合語同音異義語誤りの検出手法の基本的な考え方は,「既存の文書に現れているn文字連鎖をあらかじめ大量に集めておき,検定対象の不確定単語を含むn文字連鎖がその中に含まれているかを検証することにより,検定対象の不確定単語が誤りか否かを判定する」というものである.

図3に不確定単語を含む複合語に対して,どの部分のn文字連鎖を誤り検出に用いるかを模式的に示す.複合語中で不確定単語に前方隣接単語が存在する場合には,不確定単語の先頭から(n-i)文字と前方隣接単語の末尾i文字とから構成されるn文字連鎖を用いて検定を行う.複合語中で不確定単語に後方隣接単語が存在する場合には,不確定単語の末尾から(n-i)文字と後方隣接単語の先頭i文字とから構成されるn文字連鎖を用いて検定を行う.


\begin{figure}[ht]
\begin{center}
\epsfile{file=fig3.eps,scale=0.7}
\end{center}
\end{figure}


\subsection{処理の流れ}
図4にn文字連鎖を用いた同音異義語誤り検出の概略フローを示す.ここで,複合語の範囲および単語への分割は,図1に示した形態素解析処理の段階で既に行われているものとする.

最初に,入力された複合語が検定対象の不確定単語を含むか否かを調べる.
次に,検定対象の不確定単語を含む場合には,不確定単語の先頭(n-i)文字とその前方隣接単語の末尾i文字あるいは不確定単語の末尾(n-i)文字とその後方隣接単語の先頭i文字とからn文字連鎖を作成し,
このn文字連鎖があらかじめ収集してあるn文字連鎖辞書に存在するか否かを調べる.存在すれば着目している不確定単語は正しく使用されていると判定し,存在しない場合にはその不確定単語を同音異義語誤りであると判定する.


\begin{figure}[ht]
\begin{center}
\epsfile{file=fig4.eps,scale=0.7}
\end{center}
\end{figure}


\section{評価実験}
本手法の有効性を確認するために同音異義語誤り検出の評価実験を行った.なお,入力となる複合語はすでに形態素解析処理により構成単語は既知となっているものとする.
\subsection{文字連鎖辞書の作成}
n文字連鎖辞書としては,2文字連鎖辞書(n=2, i=1)と3文字連鎖辞書(n=3, i=1)を用意して,誤り検出の精度という観点から両者の比較を行う.さらに,辞書の大きさ,すなわち文字連鎖の収集度合いによる誤り検出の変化を調べるために,それぞれ3種類の2文字連鎖辞書,3文字連鎖辞書を用意した.表3に評価実験に用いた6種類の文字連鎖辞書の概要を示す.なお,これらの辞書はすべて新聞記事から作成した.

誤り検出の知識として長い文字連鎖を用いる方が誤りの検出精度が高くなることが予想されるが,反面,正しいものまで誤りとして検出してしまう可能性も同時に高くなると考えられる.
2文字連鎖辞書と3文字連鎖辞書とを用いた実験を行うことにより,このような傾向についても考察することができる.

\begin{figure}[h]
\begin{center}
\epsfile{file=table3.eps,scale=0.7}
\end{center}
\end{figure}

\vspace*{-5mm}
\subsection{評価実験1}
評価実験1では正しく使用されている不確定単語をn文字連鎖によってどの程度正しいと判定できるかを調べた.
\vspace{-0.2mm}
\subsubsection{評価実験1に用いた評価用データ}
\begin{description}
\item{○評価実験に用いた不確定単語}

表4に評価実験に用いた23個の同音異義語集合とこれらに含まれる同音異義語77語を示す.
\begin{figure}[htbp]
\begin{center}
\epsfile{file=table4.eps,scale=0.7}
\end{center}
\end{figure}

\item{○評価実験1に用いた複合語正解データ}

表5に評価実験に用いた複合語データの概要を示す.高校の教科書および新聞記事において,表4に示した同音異義語のいずれかを含む複合語を抽出し,それぞれデータ1,データ2とした.なお,データ2を抽出した新聞記事は5.1節で述べた文字連鎖辞書を作成した新聞記事とは異なる\footnote{データ1を用いた評価実験は,文字連鎖辞書を作成した文書(新聞記事)と異なる分野の文書(教科書)に対して行うものであり,データも分野もオープンという性格を持つ.また,データ2を用いた評価実験は,文字連鎖辞書と同じ分\\野(新聞記事)であるが,異なる文書に対して行うものであり,分野はクローズであるがデータはオープンという性格を持つ.}.

\begin{figure}[htbp]
\begin{center}
\epsfile{file=table5.eps,scale=0.7}
\end{center}
\end{figure}
\end{description}

\subsection{評価実験2}
評価実験2では,誤った不確定単語を含む複合語を用意し,n文字連鎖によってどの程度誤りを検出できるのかを調べた.
本論文では,同音異義語集合を前記の表4に示すものに限定し,評価用の複合語誤りデータを作成した\footnote{本来,表4に示す23個の同音異義語集合に属する同音異義語77語だけでなく,同じ読みを持つ同音異義語すべてを対象に評価実験を行うべきであろう.しかし,表4に示すもの以外の表記を持つ同音異義語の出現頻度は低いため,今回の評価実験ではこのような限定を設けた.}.
\subsubsection{評価実験2に用いた評価用データ}
\begin{description}
\item{○評価実験2に用いた複合語誤りデータ}

表5に示すように,複合語正解データに含まれる不確定単語をその同音異義語で置き換えることによって複合語誤りデータ(データ3)を作成した.元とした複合語正解データは評価実験1で用いた教科書から抽出した複合語正解データ(データ1)である.
\end{description}
\subsection{実験方法}
\subsubsection{正解/誤りの判定}
2文字連鎖辞書を用いる場合,不確定単語が複合語の先頭にあるときには,その不確定単語の末尾1文字とその後方隣接単語の先頭1文字とから2文字連鎖を作成する.不確定単語が複合語の末尾にあるときにはその不確定単語の先頭1文字とその前方隣接単語の末尾1文字とから2文字連鎖を作成する.そして,これらの2文字連鎖をキーとして2文字連鎖辞書を検索する.
また,3文字連鎖辞書を用いる場合,不確定単語が複合語の先頭にあるときには,その不確定単語の末尾2文字とその後方隣接単語の先頭1文字とから3文字連鎖を作成する.不確定単語が複合語の末尾にあるときにはその不確定単語の先頭2文字とその前方隣接単語の末尾1文字とから3文字連鎖を作成する.そして,これらの3文字連鎖をキーとして3文字連鎖辞書を検索する.検索の結果,前記の2(または3)文字連鎖が存在すれば不確定単語は正しく使用されている(正解)と判定し,存在しない場合には同音異義語誤りであると判定する.なお,不確定単語が複合語の中間に位置するときには,両方の2(または3)文字連鎖を調べ,どちらか一方でも2(または3)文字連鎖辞書に存在すれば正解と判定する.
\subsubsection{実験結果を評価するための指標}
本論文では,実験結果を以下の2つの指標を用いて表現する.
\begin{description}
\item{(1)正解指摘率:}
正解語を正しいと指摘できる能力を示す指標;評価実験1の結果を表すのに用いる.
\[ ○正解指摘率 = \frac{正解と判定できた件数}{正解データの全件数} \]
\item{(2)誤り検出率:}
誤り語を誤りとして検出できる能力を示す指標;評価実験2の結果を表すのに用いる.
\[ ○誤り検出率 = \frac{誤りとして検出した件数}{誤りデータの全件数} \]
\end{description}

\subsubsection{実験結果}

評価実験1,評価実験2の結果を表6および図5,図6に示す.

\noindent{\bf ○評価実験1の結果}

\indent
表6および図5,図6より以下のことが分かる;

\begin{description}
\item{(1)}
2文字連鎖辞書を用いた場合には,3文字連鎖辞書を用いた場合に比較して正解指摘率は6〜15\%程度高い.また,この差は辞書サイズが大きくなるにつれて小さくなる.
\item{(2)}
正解指摘率は辞書の大きさに依存して,2文字連鎖辞書を用いた場合には40〜83\%の間で,3文字連鎖辞書を用いた場合には27〜77\%の間で大幅に変動しているが,辞書サイズを大きくするにつれて正解指摘率は向上する.
\item{(3)}
文字連鎖辞書と同一の分野である新聞記事を対象とした場合(データ2)の方が,教科書を対象とした場合(データ1)に比較して15〜30\%程度高い値を示しており,誤り検出における分野依存性が見られる.
\item{(4)}
誤り検出知識である文字連鎖を収集する分野と同一分野の文書を対象とすると,3文字連鎖辞書3の場合で正解指摘率=77\%となり,意味的制約を用いた手法~\cite{Oku1996}と同等の能力を有している.
\item{(5)}
文字連鎖辞書1を用いた場合と文字連鎖辞書2を用いた場合とを比較すると,2文字連鎖,3文字連鎖ともに文字連鎖辞書2を用いた場合の方が高い正解指摘率を示している.このことは,既存の文書に1回でもその文字連鎖が現れれば正解とすべきであることを示している\footnote{文字連鎖辞書1と文字連鎖辞書2とは同じ量の新聞記事から作成した(表3参照).異なるのは文字連鎖辞書1が2回以上現れた文字連鎖のみをカウントしているのに対して,文字連鎖辞書2は1度でも現れた文字連鎖をカウントしている点のみである.}.
\end{description}


\noindent{\bf ○評価実験2の結果}

\smallskip
\indent
評価実験2の結果を示した表6および図5,図6より以下のことが分かる;
\begin{description}
\item{(1)}
3文字連鎖辞書を用いた場合には,2文字連鎖辞書を用いた場合に比較して誤り検出率は21〜34\%程度高い.また,この差は辞書サイズが大きくなるほど,大きくなっている.すなわち,2文字連鎖辞書を用いた場合には,辞書サイズが大きくなるほど誤り検出率が急激に悪くなることを示している.
\item{(2)}
誤り検出率は辞書の大きさに依存して,2文字連鎖辞書を用いた場合には60〜78\%の間で,3文字連鎖辞書を用いた場合には94〜99\%の間で変動しているが,辞書サイズを大きくするにつれて誤り検出率は徐々に減少する.また,この減少割合は正解指摘率の増加割合に比較して緩やかである.
\item{(3)}
データ3は教科書から作成したものであり,文字連鎖辞書を作成した分野(新聞記事)とは異なる.しかし,分野が異なるにも関わらず,どの3文字連鎖辞書でも90\%以上の誤り検出率が得られている.
\end{description}

\begin{figure}[htbp]
\begin{center}
\epsfile{file=table6.eps,scale=0.7}
\end{center}
\vspace{-4mm}
\end{figure}



\subsubsection{考察}
\noindent{\bf ○文字連鎖を用いた複合語同音異義語誤り検出手法のポテンシャル}

\smallskip
\indent
上記の結果から,2文字連鎖辞書を用いた場合には,3文字連鎖辞書を用いた場合に比較して,正解指摘率が高く,誤り検出率が低いことが分かる.推敲支援という目的を考えれば,2文字連鎖辞書を用いた場合の誤り検出率が60〜78\%というのは低すぎると考えられる.これに対して3文字連鎖辞書を用いた場合には辞書サイズにかかわらず,90\%以上の誤り検出率が得られている.
また,正解指摘率で見た場合,確かに2文字連鎖辞書を用いた方が3文字連鎖辞書を用いた場合に比べて高い値を示しているが,辞書サイズを大きくするとその差は6\%程度と非常に小さくなる.すなわち,推敲支援という目的に使用するには,2文字連鎖よりも3文字連鎖を誤り検出知識として用いる方がよいと言える.

\begin{figure}
\begin{center}
\epsfile{file=fig5.eps,scale=0.7}
\end{center}
\vspace*{-2mm}
\end{figure}

\begin{figure}
\begin{center}
\epsfile{file=fig6.eps,scale=0.7}
\end{center}
\end{figure}

3文字連鎖辞書による検定の方が2文字連鎖辞書による検定よりも優れている理由の1つとして次のことが考えられる.

多くの単語は2文字(特に2漢字)から構成されているため,3文字連鎖はこれらの単語とその前後1文字との連鎖を表すことになる.すなわち,単語そのものとその前後の文字との連鎖として3文字連鎖を近似的にとらえることができるためである.従って,さらに文字連鎖の長さを長くしても,必要な連鎖を集める文書量が大幅に増えるだけで,誤り検出率や正解指摘率の向上はあまり望めないと考えられる.

次に3文字連鎖辞書を用いた場合,意味的制約を用いた複合語同音異義語誤りの検出手法と同程度の精度となる,誤り検出率90\%,正解指摘率70\%のときの辞書サイズの概算を行う.図6の誤り検出率と正解指摘率を簡単に直線近似すると,求める辞書サイズは約250万件となる.概算すると1年分の新聞記事から3文字連鎖辞書を作成すればこの辞書サイズを得ることができる.3文字連鎖の収集の容易さから考えて,1年分の新聞記事から辞書を作成することは,実現性の上からも問題はない.すなわち,3文字連鎖辞書を用いた複合語同音異義語誤り検出手法によって,意味的制約を用いたそれと同程度の精度を実現することが可能であろうと推定できる.

以上のことから,3文字連鎖辞書を誤り検出の知識として用いれば,誤り検出率90\%以上が得られ,しかも,3文字連鎖辞書と検定対象データとが同一の分野であれば,正解指摘率も70\%以上と高い値を示す.また,意味的制約を用いた複合語同音異義語誤りの検出手法と同程度の精度を得ることも可能であると予測される.すなわち,本論文で提案した文字連鎖を用いた複合語同音異義語誤りの検出手法は,検定対象の文書の分野を限定し,その分野において3文字連鎖辞書を収集することによって,日本文推敲支援システムにおける複合語同音異義語誤りの検出手法として十分に利用することが可能であると言える.

\clearpage
\noindent{\bf ○日本文推敲支援システムへの適用形態}

\smallskip
\indent
表7に本手法と意味的制約を用いた手法との長所と短所を示す.意味的制約を用いた手法は高精度の同音異義語誤りの検出と訂正候補の提示が可能であり,分野依存性が小さい反面,意味的制約辞書すなわち誤りを検出するための知識の収集が容易ではない.これに対して,本論文で提案した手法において誤り検出知識として利用する3文字連鎖は,既存の文書から容易に収集することができる.

従って,日本文推敲支援システムにおいては,高頻度の同音異義語誤りに対しては意味的制約を用いた手法により検出/訂正支援を行い,それ以外の幅広い範囲の同音異義語誤りに対しては本論文で述べた3文字連鎖を用いた手法によって誤りを検出するといった,両手法を統合した利用形態が望ましいと考えられる.

\begin{figure}
\begin{center}
\epsfile{file=table7.eps,scale=0.7}
\end{center}
\end{figure}



\section{むすび}
本論文では誤り検出知識として文字連鎖を用いた複合語同音異義語誤りの検出手法について述べた.評価実験の結果,3文字連鎖辞書を誤り検出知識として用いれば,分野依存性はあるものの,3文字連鎖を収集した分野と同一の分野に属する文書に対しては,誤り検出率90\%以上,正解指摘率70\%以上の値が得られた.
このことは,本論文で提案した文字連鎖を用いた複合語同音異義語誤りの検出手法は,検定対象の文書の分野を限定し,その分野において3文字連鎖辞書を収集することによって,日本文推敲支援システムにおける複合語同音異義語誤りの検出手法として十分に利用することが可能であることを示している.

本論文では3文字連鎖によって複合語同音異義語誤りの検出が高精度で可能であることを述べたが,3文字連鎖を用いれば,他の種類の誤り,例えば文字脱落を検出することも可能である~\cite{MatsuokaAndTakagi1989}.




\acknowledgment

本研究の遂行にあたり,有益な討論,助言を頂いた高木伸一郎氏,島崎勝美氏の両氏に深謝致します.


\bibliographystyle{jnlpbbl}
\bibliography{jpaper}      

\begin{biography}
\biotitle{略歴}
\bioauthor{奥 雅博}{
1982年大阪府立大学工学部電子工学科卒業.
1984年同大学院博士前期課程修了.
同年,日本電信電話公社(現NTT)入社.現在,NTT情報通信研究所主任研究員.
自然言語処理の研究実用化に従事.
慣用表現,比喩などの非標準的な言語現象に興味を持つ.
情報処理学会,電子情報通信学会,人工知能学会各会員.}
\bioauthor{松岡 浩司}{
1979年九州大学電子工学科卒業.
同年,日本電信電話公社(現NTT)入社.現在,NTT情報通信研究所主任研究員.
自然言語処理の研究実用化に従事.
推敲支援、音声合成などの形態素解析を応用したシステムに興味を持つ.
情報処理学会,電子情報通信学会各会員.}

\bioreceived{受付}
\bioaccepted{採録}

\end{biography}

\end{document}

