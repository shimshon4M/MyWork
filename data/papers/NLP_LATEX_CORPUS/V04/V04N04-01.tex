



\newcommand{\susumu}{}
\newcommand{\nokou}{}
\newcommand{\scbfkou}{}
\newcommand{\smbfkou}{}

\newenvironment{ftr}{}{}

\documentstyle[epsf,jnlpbbl]{jnlp_j_b5}

\setcounter{page}{3}
\setcounter{巻数}{4}
\setcounter{号数}{4}
\setcounter{年}{1997}
\setcounter{月}{10}
\受付{1996}{4}{24}
\再受付{1997}{5}{20}
\採録{1997}{7}{18}

\setcounter{secnumdepth}{2}

\title{動詞と主体の属性を用いた複文の連接関係の解析}
\author{向仲 \nokou\affiref{EDOGAWA}}

\headauthor{向仲 \scbfkou}
\headtitle{動詞と主体の属性を用いた複文の連接関係の解析}

\affilabel{EDOGAWA}{江戸川大学社会学部 環境情報学科}
{Department of Environmental Information, Edogawa University}

\jabstract{
本論文は,動詞と主体の属性を用いて,複文中の連接関係を解析
するモデルを作成し,評価した結果を述べる.複文中の連接関係の
関係的意味は,接続詞,助詞等の接続の表現だけでは決まらず,曖
昧性がある.例えば,助詞「て」による連接関係には,「時間的継
起」のほかに「方法」,「付帯状態」,「理由」,「目的」,
「並列」などがある.これらの関係的意味は,従属節や主節の述語
の表している事象の意味タイプ,およびその組み合わせによって決
まってくる.従って,動詞と名詞の意味的関係を表すために,動詞
と名詞の意味分類を用いた格パターンがあると同様に,従属節と主
節の連接関係にも,各々の節を構成する動詞と主体の属性を用いた
連接関係パターンが存在すると考えることができる.本論文のモデ
ルでは,従属節と主節の,動詞と主体の属性を用いて,連接関係の
関係的意味を推定する.動詞の属性として,意志性,意味分類,慣
用的表現,ムード・アスペクト・ヴォイス,主体の属性として,主
節と従属節の主体が同一かどうか,無生物主体かどうかを用いた.
このモデルを,技術文書に適用した結果,95\%の文が正しく解析で
きた.}

\jkeywords{連接関係,多義性解消,動詞の属性,主体の属性,連接関係パ
ターン}

\etitle{Analysis of Coherence Relation for Complex Sentences \\
Using Attributes of Verbs and Subjects}
\eauthor{Kou Mukainaka \affiref{EDOGAWA}} 

\eabstract{
This paper presents a model to analyze the coherence 
relation within complex sentences by using the attributes 
of verbs and subjects. The relations between subordinate 
and main clauses can not be understood only by the 
connectives that link them. The coherence relations by 
connectives are often ambiguous.  For example, a Japanese 
conjunctive particle ``te" expresses coherence relations, 
such as sequential, method, manner, reason, purpose and 
parallel. The coherence relations by connectives depend on 
the semantic types of the predicates in subordinate and 
main clauses, and the combination of them. It is presumed 
that there are the patterns of coherence relations in the 
relations between subordinate and main clauses, just like 
there are the patterns of thematic roles to express 
semantic relations between verbs and nouns. The pattern of 
coherence relation uses the attributes of verbs and 
subjects in a subordinate and a main clause, just like the 
pattern of thematic roles uses a verb and the semantic 
types of nouns. The model infers coherence relations by 
using the attributes of verbs and subjects in subordinate 
and main clauses. It uses volition, semantic types, 
idiomatic expressions, mood/aspect/voice, as the attributes 
of verbs, and uses whether the subject in a subordinate 
clause is identical with the one in the main clause or not, 
whether the subject is unanimate or not as the attributes 
of subjects. The model is evaluated and the result shows 
that 95\% of the text taken from science and technical 
documents can be analyzed successfully.\vspace*{5mm}}

\ekeywords{Coherance relation, Disambiguation, Attribute of verb, 
Attribute of subject, Coherence relation pattern}

\begin{document}
\thispagestyle{myheadings}
\maketitle


\section{はじめに}

連接関係の関係的意味は,接続詞,助詞等により一意に決まるも
のもあるが,一般的には曖昧性を含む場合が多い. 一般的には,
複文の連接関係の関係的意味は,従属節や主節の表している事象の
意味,およびそれらの事象の相互関係によって決まってくる.しか
し,各々の単文の意味とそれらの間の関係を理解するためには広範
囲の知識が必要になる.それらの背景知識を記述して,談話理解に
利用する研究\cite[など]{ZadroznyAndJensen1991,Dalgren1988} も行わ
れているが,現状では,非常に範囲を限定したモデルでなければ実
現できない.従って,連接関係を解析するためには,少なくともど
のような知識が必要になり,それを用いてどのように解析するのか
が問題になる.

シテ型接続に関する研究\cite{Jinta1995} では,助詞「て」による
連接関係を解析し,「時間的継起」のほかに「方法」,「付帯状
態」,「理由」,「目的」,「並列」などの意味があることを述べ
ている.これらの関係的意味は,動詞の意志性,意味分類,アスペ
クト,慣用的な表現,同一主体,無生物主体などによって決まるこ
とを解析している.

しかし,動詞の意志性自体が,動詞の語義や文脈によって決まる
場合が多い.また,主体が省略されていることも多い.さらに,
「て」以外の接続の表現に対して,同じ属性で識別できるかどうか
も不明である.

表層表現中の情報に基づいて,文章構造を理解しようとする研究
\cite{KurohasiAndNagao1994} では,種々の手掛かり表現,同一/同種の語
/句の出現,2文間の類似性を利用することによって連接関係を推
定している.しかし,手掛かり表現に多義のある時は,ある程度の
意味情報を用いる必要がある.

日本語マニュアル文においてアスペクトにより省略された主語を
推定する研究\cite{NakagawaAndMori1995} や,知覚思考,心理,言語活動,
感情,動きなど述語の意味分類を用いて,「ので」順接複文におけ
る意味解析を行う研究\cite{KimuraAndNisizawaAndNakagawa1996} などがあり,アス
ペクトや動詞の意味分類が連接関係の意味解析に有効なことが分か
る.しかし,連接関係全般について,動詞と主体のどのような属性
を用いて,どの程度まで解析できるかが分からない.

本論文では,「て」以外の曖昧性の多い接続の表現についても,
その意味を識別するために必要な属性を調べ,曖昧性を解消するモ
デルを作成した.動詞の意志性については,予め単文で動詞の格パ
ターンを適用して解析して,できるだけ曖昧性を無くすようにした.
省略された主体については,技術論文,解説書,マニュアルなどの
技術文書を前提にして,必要な属性を復元するようにした.

\section{連接関係の曖昧性}
接続詞,助詞等の接続の表現には曖昧性がある.特に,「て」,
「ため」,「が」,「と」などで表現される従属節の関係的意味は
様々である.これは,「から」,「ので」などと違って,これらの
接続の表現自体が明確な固有の意味を有していないからである.
従って,これらの接続の表現では,連接関係の関係的意味が,従属
節や主節の表している事象の意味,およびそれらの事象の相互関係
によって決まってくる.

「〜して」形式で表される連接関係の関係的意味には,次の幾つ
かの例で示すように,「時間的継起」,「方法」,「付帯状態」,
「原因」,「目的」,「並列」などがある.
\begin{description}
\item[〔例文1〕] それぞれのセグメントにTCPヘッダを加えて,相手のプ
ロトコルモジュールに送っています.(時間的継起)
\item[〔例文2〕] ネットワークを利用して,常に最新のデータを取り出す
ことができます.(方法)
\item[〔例文3〕] そこで女中が鍵を持って,私を待っていた.(付帯状
態)
\item[〔例文4〕] この部屋は静かで,よく眠れる.(原因)
\item[〔例文5〕] 私がだれかにこの暗号を伝達する前に詳細を知ろうとし
て,私をつけ狙った.(目的)
\item[〔例文6〕]交通は混乱して,人心は険悪である.(並列)
\end{description}

「〜と」形式で表される連接関係の関係的意味には,「時」,
「条件」,「原因」などがある.

\begin{description}
\item[〔例文7〕] 朝起きると,すぐシャワーを浴びる.(時)
\item[〔例文8〕] 同じユーザ名でもユーザIDやグループIDが異なっている
と,うまくログインやファイルのコピーができません.(条
件)
\item[〔例文9〕] 窓を開けると,寒い風が入った.(原因)
\end{description}

「〜ため」形式で表される連接関係の関係的意味には,「原因」
と「目的」がある.

\begin{description}
\item[〔例文10〕] AとCは,異なったネットワークにあるため,データ
リンク層のプロトコルも違います.(原因)
\item[〔例文12〕] 異なる地域,都市,国の間を結ぶため,自分達で勝手
に通信回線の敷設はできません.(目的)
\end{description}

「〜が」形式で表される連接関係の関係的意味には,「逆接」,
「対比」,「前置き」などがある.

\begin{description}
\item[〔例文13〕] 何度も説明しましたが,あの人は分からなかった.
(逆接)
\item[〔例文14〕] 兄は勤勉だが,弟はずぼらだ.(対比)
\item[〔例文15〕] ちょっと伺いますが,駅はどこですか.(前置き)
\end{description}

\section{動詞と主体の属性と連接関係の関係的意味}
複文中の連接関係は,従属節や主節の述語の表している事象の
意味タイプ,およびその組み合わせによって決まってくる.ここで,
事象の意味タイプは,ある主体が行う動作または状態の分類,意志
性などを表す~\cite{Jinta1995}.例えば,生物主体の姿勢変化,携帯,
心的状態,使用,作成,助言などである.従って,事象の意味タイ
プは,動詞と名詞の属性を用いて表すことができると考えることが
できる.従って,動詞と名詞の意味的関係を表すために,動詞と名
詞の意味分類を用いた格パターンがあると同様に,従属節と主節の
連接関係にも,動詞と名詞の属性を用いた連接関係パターンが存在
すると考えることができる.

本論文では,従属節と主節の,動詞と主体の属性を用いて,連接
関係の関係的意味を推定する方法をとった.動詞の属性として,意
志性,意味分類,慣用的表現,ムード・アスペクト・ヴォイス,主
体の属性として,主節と従属節の主体が同一かどうか,無生物主体
かどうかを採用した.

次に,各々の属性によって,連接関係の関係的意味がどのように
決まるかを,いくつかの例で示す.

\subsection{動詞の意志性と主体の同一性,無生物性}
従属節と主節の動詞の意志性および主体が同一かどうか,無生物
主体かどうかの組み合わせによって,連接関係の関係的意味が次の
ように影響を受ける.ここで,主体は,動作主,経験者などを含む
概念である\cite{Jinta1995} .動詞の意志性は,人間の意志的な行為
を表し,誘い掛けや命令の意味を表す.生物/無生物は,生命の有
る/無しではなく,情意,特に自由意志による行動可能なものを表
す.

「〜して」形式接続で,従属節と主節が共に意志動詞で形成され,
両者の主体が同一の時は,「時間的継起」を表すことが多い.これ
は,同一主体による制御可能な動きは,通常,継起的に引き起こさ
れるためである.本論文のモデルで用いた用例では,このパターン
に属する連接関係の94\%が「時間的継起」を表した.ただし,後に
述べる従属節の意味分類から「方法」,「付帯状態」と識別される
ものは除いてある.用例の内容については\ref{section:evaluation} 章で述べる.
\begin{description}
\item[〔例文16〕] ユーザは,IDとパスワードを指定して,OKボタンをク
リックします.
\end{description}

「〜して」形式接続で,従属節と主節の主体が無生物で動詞が共
に無意志動詞の時は,「時間的継起」を表すことが多い.自然界に
おける無生物的な2現象が,時間的継起の下に生じることはよくあ
る.この場合がそれに相当する.用例では,従属節の意味分類から
「原因」と識別されるものを除くと,このパターンに属する連接関
係の85\%が「時間的継起」を表した.
\begin{description}
\item[〔例文17〕] ログインが成立して,プロンプトが戻ってきます.
\end{description}

「〜して」形式接続で,従属節と主節が共に無意志動詞で従属節
が無生物主体,主節が生物主体の時は,「原因」を表すことが多い.
これは,主たる事象が人間に関するものでありながら,従属節で人
間の意志で制御できない事象が生じたためである.一般に技術文書
ではこのパターンは少なく,用例でも2例しかなかったが,2例の連
接関係はいずれも「原因」を表した.
\begin{description}
\item[〔例文18〕] 彼は,車が故障して,遅れた.
\end{description}

「〜ため」形式接続で,従属節と主節が共に意志動詞で生物主体
の時は,「目的」を表すことが多い.これは,同一主体が,従属節
の事象を達成する目的で,主たる事象を行うためである.用例では
このパターンに適合する連接関係は,全て「目的」を表した.
\begin{description}
\item[〔例文19〕] ユーザは,FDDIでUNIXワークステーションを接続する
ために,通常FDDIの通信用ボードを購入しなければなりません.
\end{description}

「〜ため」形式接続で,従属節が無意志動詞で,主節も無意志動
詞の時は「原因」を表すことが多い.これは,従属節で人間の意志
で制御できない事象が生じ,それからある事象が起こったときは,
その事象の原因と解釈されるためである.

無意志動詞には,心的作用を表す動詞や,状態を表す動詞のほか,
意志動詞に「される」が後接した受動態や,意志動詞に「ている」,
「てある」が後接した「単純状態」や「変化状態の維持」のアスペ
クトなどを含む.このような場合を含めると,用例では,このパ
ターンに適合する連接関係はすべて「原因」を表した.
\begin{description}
\item[〔例文20〕] FDDIはトークンパッシング方式を採用しているため,
ネットワークトラフィックによる性能の低下がない.
\end{description}

「〜と」形式接続で,従属節と主節が共に意志動詞で同一主体の
時は,「時」を表すことが多い.「〜と」形式接続は一般的には
「条件」を表すことが多いが,同一主体による制御可能な動きの場
合には,「その時」または「〜してすぐ」という意味になりやすい.
主節のヴォイスが「可能」を表す時には「条件」を表すことが多い
ので,この場合を除くと,用例ではこのパターンに適合する連接関
係は全て「時」を表した.
\begin{description}
\item[〔例文21〕]彼女は,部屋に入ると,窓を開けた.
\end{description}

\subsection{動詞の意味分類}
動詞の意味分類によっても,連接関係の関係的意味が影響を受け
ることがある.次に,幾つかの例を挙げる.

「〜して」形式接続で従属節の動詞が,姿勢変化,着脱,携帯,
心的状態などの意味分類であるときは「付帯状態」を表すことが多
い.「付帯状態」とは,従属節と主節の事象が時間的に同存し,同
一主体で,従属節で主節の事象の実現のされ方を表しているもので
ある.
\begin{description}
\item[〔例文22〕] わたしは汗で湿った服をそのまま着て,また寮に出か
けていった.
\end{description}

「〜して」形式接続で,従属節の動詞が,使用,作成,助言など
の意味分類であるときは,「方法」を表すことが多い.つまり,従
属節の事象が,主節の事象を実現するための方法的要因になってい
る場合である.技術文書では,「〜を利用して」,「〜を作成し
て」などの表現が多いので,連接関係の関係的意味にも「方法」が
かなりある.意味分類の定義は,基本的には分類語彙表に因った.
分類語彙表の意味分類に用例を適用して,正の用例のみを含む場合
は分類語彙表の分類で定義したが,負の用例を含む場合は正の用例
の単語をそのまま定義に加えた.
\begin{description}
\item[〔例文23〕] そのケーブルを使って,データを送る必要があります.
\end{description}

「〜が」形式接続で,主節と従属節の動詞又は形容詞の意味分類
が,反意語の時は「対比」を表すことが多い.用例では,2例だけ
であったが,2例とも「対比」を表した.
\begin{description}
\item[〔例文24〕] 昨日まで寒かったが,今日から急に暖かくなった.
\end{description}

\subsection{ムード・アスペクト・ヴォイス}
ムード・アスペクト・ヴォイスによっても連接関係の関係的意味
が変わってくる.前述のように,「される」が後接した受動態や,
「ている」,「てある」が後接した「単純状態」や「変化状態の維
持」のアスペクトなどを含む節は無意志的に解釈されるが,そのほ
かにも次に示すような幾つかの例がある.

「〜して」形式接続で,従属節が「〜(よ)うとして」といった
将然相の形をとる場合は,「目的」を表すことが多い.主節の事象
を引き起こす計画を,従属節で述べているためである.
\begin{description}
\item[〔例文25〕] 体を鍛えようとして,毎日ジョギングをやっている.
\end{description}

「〜が」形式接続で,主節が疑問文の時は,「前置き」を表すこ
とが多い.一般的には,「〜が」は「逆接」または「対比」を表す
ことが多いが,主節が疑問文の時は,質問に対する前提条件を表す
ことが多いためである.用例では,このパターンに属する連接関係
は全て「前置き」を表した.
\begin{description}
\item[〔例文26〕]ここに鍵が置いてありますが,誰のですか.
\end{description}

「〜と」形式接続で,従属節が生物主体で「ている」,「かけ
る」,「はじめる」の場合は,「時」を表すことが多い.「〜と」
は一般的には,「条件」を表すことが多いが,従属節が生物主体で
動作の「進行」または「開始」を表すアスペクトの場合は,「その
時」または「〜してすぐ」の意味になることが多いためである.用
例では,このパターンは少なかったが,適合する連接関係は全て
「時」を表した.
\begin{description}
\item[〔例文27〕]食事をしていると,急にグラッと揺れた.
\end{description}

\subsection{従属節が慣用句的になっているもの}
従属節が慣用句化して副詞的に用いられる場合がある.この場合
は,接続の表現もそれぞれの慣用句に対応した関係的意味を持つ.

「〜して」形式接続で,従属節が「体力をふり絞って」,「先を
争って」,「まとまって」,「だまって」などの表現をとるときは,
「付帯状態」を表す.
\begin{description}
\item[〔例文28〕]体力を振り絞って,走った.
\end{description}

\section{複文の連接関係解析モデル}
前述のように複文の連接関係は,動詞の意志性,意味分類,ムー
ド・アスペクト・ヴォイス,慣用表現,主体の同一性,無生物主体
かどうかなどによって決まる場合が多い.これらの情報により,連
接関係を解析することができる.しかし,動詞の意志性自体に曖昧
性がある.また,主体が省略されていることも多い.従って,連接
関係を解析する前に,これらの情報を解析しておく必要がある.複
文の連接関係の解析モデルは図1に示すように,動詞と主体の属性
を解析し,それらの属性を用いて,連接関係の解析を行う.
\vspace*{2mm}
\begin{figure}[htbp]
  \begin{center}
    
	\epsfile{file=fig1.eps,width=78mm}
    \bigskip

    \caption{連接関係の解析モデル}
    \label{fig:1}
  \end{center}
\end{figure}

本論文のモデルは連用修飾節の解析を対象としたので,「の」,
「こと」などによる連体修飾節は,入力文から省いてある.連接関
係パターン,格パターン,名詞の属性などをHPSG\cite[など]{PollardAndSag1987,PollardAndSag1993} 
の素性構造に似た形式で表し,辞書に登録した.
これらの辞書の情報を用いて,動詞と主体の属性の解析,連接関係
の解析を行い,複文の素性構造に相当する出力を生成する.解析は,
簡単なHPSGパーザをprologで作成して行った.このパーザは,格パ
ターンの解析と連接関係パターンの解析の機能だけを持ったもので,
この目的のために作成した.

辞書は,IPAL辞書\cite[など]{IPA1987,IPA1990} と分類
語彙表\cite{Kokuritukokugokenkyujo1989} に基づき作成した.動詞の格パター
ン,動詞の意志性はIPAL辞書のものを採用した.動詞の意味分類は
分類語彙表の分類を用いた.IPAL辞書の格パターンで用いている名
詞の意味分類は,比較的粗い分類になっている.従って,動詞の意
味分類の方が名詞の意味分類よりも詳しくなっている.

名詞の生物/無生物の区分については,意味分類が「人間」と
「組織」の場合を「生物」とし,それ以外を「無生物」とした.用
例は,主としてネットワークプログラムの解説書からとったため,
プログラムの機能説明が多数含まれていた.その中で,プログラム
が生物と同様の行動をするので,プログラムも生物に分類した.コ
ンピュータ用語でIPAL辞書にも分類語彙表にも無い用語が多数出て
きたが,同じ分類の用語に準じて定義し辞書に追加した.

\subsection{動詞と主体の属性の解析}
複数の語義のある動詞は,語義を確定しないと,意志性を確定で
きない.そのため,動詞の格パターンを適用して,語義を確定する
ようにした.図2に,辞書における動詞の格パターンの記載例を示
す.図2ではHPSGの形式で記載しているが,実際の辞書ではprolog
の形式に変換して格納している.この格パターンを用いて入力文を
解析する過程で,適合したパターンの語義を選択し,意志性を確定
した.

しかし,意志動詞でも,特定の語義で無意志用法のあるものがあ
る.
\begin{description}
\item[〔例文29〕] B29は工場に爆弾を落とした.(意志用法)
\item[〔例文30〕] 彼女はお皿を落として割ってしまった.(無意志用
法)
\end{description}

\begin{figure}[htbp]
  \small
  (a)動詞の記載例\\[1mm]
  \hspace*{10mm}$
    \begin{ftr}
      & PHON & 開ける & \\
      & & & \\[-3mm]
      & SYN|LOC|CAT & \begin{ftr}
        & HEAD & \begin{ftr}
          & MAJ   & ~V & \\
          & VFORM & ~DICTIONARY - FORM & \\
          & \multicolumn{2}{l}{ADJUNCTS\{[LOC|HEAD|MAJ \;\; ADV]\}} & \\
          & LEX   & ~+ & \\
        \end{ftr} & \\
        & & & \\[-3mm]
        & SUBCAT &
          \left\langle
          \begin{array}[c]{l}
            PP[NOM][\,1\,][+ANIMATE,HUMAN], \\
            PP[ACC][\,2\,][-ANIMATE,PRODUCT] \\
          \end{array}
          \right\rangle & \\
      \end{ftr} & \\
      & & & \\[-3mm]
      & SEM|CONT & \begin{ftr}
        & RELN     & ~AKERU   & \\
        & OPENER   & ~[\,1\,] & \\
        & OPEND    & ~[\,2\,] & \\
        & VOLITION & ~+       & \\
        & SEM-TYPE & ~開・閉  & \\
      \end{ftr} & \\
    \end{ftr}
  $\\[1mm]
  \hspace*{15mm}注) $VP[DICTIONARY-FORM][+ VOLITION]$ と略記する

  \bigskip
  (b) 名詞句 $PP[ACC][-ANIMATE,PRODUCT]$ の内容\\[1mm]
  \hspace*{10mm}$
    \begin{ftr}
      & PHON & 箱を & \\
      & & & \\[-3mm]
      & SYN|LOC|CAT & \begin{ftr}
        & HEAD & \begin{ftr}
          & MAJ   & ~P   & \\
          & PFORM & ~WO  & \\
          & CASE  & ~ACC & \\
        \end{ftr} & \\
        & & & \\[-3mm]
        & SUBCAT & \langle NP[- ANIMATE,PRODUCT]\rangle & \\
      \end{ftr} & \\
    \end{ftr}
  $

  \bigskip
  (c) 名詞 $NP[- ANIMATE,PRODUCT]$ の内容\\[1mm]
  \hspace*{10mm}$
    \begin{ftr}
      & PHON & 箱 & \\
      & & & \\[-3mm]
      & SYN|LOC|CAT & \begin{ftr}
        & HEAD|MAJ  & ~N   & \\
        & SUBCAT    & ~\langle \;\; \rangle & \\
      \end{ftr} & \\
      & & & \\[-3mm]
      & SEM|LOC|CONT & \begin{ftr}
        & ANIMATE  & ~-       & \\
        & SEM-TYPE & ~PRODUCT & \\
      \end{ftr} & \\
    \end{ftr}
  $

  \bigskip

  \caption{辞書における動詞の格パターンの記載例}
  \label{fig:2}
\end{figure}

ただし,技術文書を考えた場合,意志動詞の無意志用法は比較的
少ない.実際に用例を調べた結果,意志動詞の無意志用法は,非常
に少く2\%であったので無視することにした.従って,IPAL辞書の
動詞の意志性で無意志用法の有るものは,意志動詞として分類した.

主体が省略されている時には,マニュアル,技術論文などの技術
文書であることを前提にして,比較的単純な方式により,生物主体
か無生物主体か,同一主体か異主体かを推定するようにした.

一般的に,著者や読者が主題になっているときは,先行文脈から
推定可能であり,特に強調する必要があるときなど,特別の場合以
外は,省略されるのが普通でる.例えば,マニュアル類では,装置
の開発者が,利用者に説明することを前提にして書かれているので,
開発者や利用者が省略されることが多い.技術文書でも,開発者や
研究者又は読者が省略される.

文全体の主題と従属節の主題が同じ時は,従属節の主題は省略さ
れる.ただし,従属節に主題でない主体があるにもかかわらず,主
節の主体が省略されることは有り得る.この場合は,省略された主
体は,先行文脈の主題であり,かつ,文全体が主題寄りの視点で記
述されている可能性が高い~\cite{Kuno1978}.技術文書では著者又は
読者寄りの視点から書かれているのが普通であるから,著者又は読
者が省略されている場合が多い.

省略された主体を埋める候補が意味的に矛盾しないかどうかの検
証は,動詞の要求する主体が生物か無生物かによった.

連接関係の解析に必要な情報は,同一主体か,無生物主体かだけ
であり,HPSGパーザは格パターン,連接関係パターンの順序で縦形
探索を用いて解析しているので,次のような簡単な処理方法により
省略された主体を推定した.
\begin{enumerate}
\item 
格パターンを適用するとき,生物主体のパターンを優先す
る.\label{item:animate}
\item 
最終的に動詞の格パターンのSUBCATに残った項を,省略を
復元する候補とする.\label{item:subcat}
\item 
連接関係パターンを適用するとき,同一主体のパターンを
優先する.\label{item:same}
\item 
従属節が無生物主体で,主節の主体が省略されている時だ
け特別扱いし,生物/無生物の異主体のパターンを優先す
る.\label{item:different}
\item 
優先するパターンが無いときは,次に確率の高いパターン
を適用する.
\end{enumerate}

\ref{item:animate} は,著者や読者が省略されている可能
性が高い事を表す規則であり,\ref{item:same} は,次
の\ref{item:different} の場合を除いて,文全体の主題と
従属節の主題が同じ場合,および従属節の主体と主節の主体が同じ
場合が多い事を表す規則である.\ref{item:different} は,従属節が無生物主体で,
主節の主体が著者または読者の場合に相当する.

従属節の動詞の意味分類によって,連接関係の関係的意味が決
まってくることがある.従って,動詞が特定の意味分類に属してい
るかどうかを,解析する必要がある.このため,動詞の意味分類に
よって決まる接続の表現に対しては,辞書の連接関係パターンの記
載に,その関係的意味が要求する動詞の意味分類を記載して,動詞
の意味分類と単一化するようにした.動詞の意味分類は,前述のよ
うに連接関係の関係的意味を識別する必要上,名詞の意味分類より
詳細になっている.

従属節が慣用句化して副詞的に用いられる場合は,慣用句として
辞書に記載しておき,入力文を解析する時に優先的に選択する.

上記のようにHPSGパーザの解析過程で,格パターンを適用するこ
とにより,動詞と主体の属性を求め,引き続き連接関係パターンを
適用することにより,連接関係を解析する.

\subsection{連接関係の解析}
接続の表現は,図3に示すような形式で辞書に記載される.図3
には,「〜して」形式接続で,従属節と主節が共に意志動詞,主体
が同一で,「時間的継起」を表す場合の連接関係パターンを示す.
このような連接関係パターンが,接続の表現別,連接関係の関係的
意味別に存在する.1つの関係的意味を複数のパターンで表すこと
も有り得る.

\begin{figure}[htbp]
  \small
  \bigskip
  \begin{displaymath}
    \begin{ftr}
      & PHON & て & \\
      & & & \\[-3mm]
      & \multicolumn{2}{l}{SYN \; \begin{ftr}
        & \multicolumn{2}{l}{LOC|CAT \; \begin{ftr}
          & HEAD & \begin{ftr}
            & MAJ   & ~ADV & \\
            & JFORM & ~TE & \\
          \end{ftr} & \\
          & & & \\[-3mm]
          & SUBCAT & 
            \left\langle
            \begin{array}[c]{l}
              VP[ADVERVIAL-FORM][\,1\,][+VOLITION], \\
              SUBCAT\langle PP[NOM][\,3\,][+ANIMATE] \rangle \\
            \end{array}
            \right\rangle & \\
        \end{ftr} } & \\
        & & & \\[-3mm]
        & BIND|CAT|MAIN &
          \left\langle
          \begin{array}[c]{l}
            VP[\,2\,][+ VOLITION], \\
            SUBCAT\langle PP[NOM][\,3\,][+ ANIMATE] \rangle \\
          \end{array}
          \right\rangle & \\
      \end{ftr} } & \\
      & & & \\[-3mm]
      & SEM|LOC|CONT & \begin{ftr}
        & CONN  & ~SEQUENTIAL & \\
        & JUNCT & ~\{[\,1\,],[\,2\,]\} & \\ 
      \end{ftr} & \\
    \end{ftr}
  \end{displaymath}
    
  \bigskip

  \caption{辞書における連接関係パターンの記載例}
  \label{fig:3}
\end{figure}

\newpage
連接関係パターンで使用している素性は,表1の通りである.た
だし,一般的なHPSGの素性は省いてある.

\begin{table}[htbp]
  \begin{center}
    \caption{連接関係パターンの素性}
    \label{tab:1}

    \smallskip

    \begin{tabular}{|l|l|l|} \hline
      \multicolumn{1}{|c}{素性} & 
      \multicolumn{1}{|c}{値}   &
      \multicolumn{1}{|c|}{意味} \\ \hline
      VOLITION   & +,$-$ & 意志性 \\
      ANIMATE    & +,$-$ & 生物/無生物 \\
      SEM-TYPE   & \{使用, 製造, 教育\}, $\cdots$ & 意味分類 \\
      VOICE      & PASSIVE, ACTIVE, & ヴォイス(受動/能動/使役/可能) \\
      & CAUSATIVE, POSSIBLE & \\
      ASPECT     & \{TEIRU,TEARU\}, $\cdots$ & アスペクト \\
      MOOD       & QUESTION, VOLITION & ムード(疑問/意志) \\
      IDIOM-TYPE & TE-SEQ, GA-PRE & 慣用句の分類 \\ \hline
    \end{tabular}
  \end{center}
\end{table}

特定の接続の表現の連接関係パターンで,従属節の要求する素性
はSUBCATで表され,主節の要求する素性はMAINで表される.解析の
過程で単一化に成功したパターンが選択され,適合する連接関係の
関係的意味が選択されることになる.

\section{連接関係解析モデルの評価結果}\label{section:evaluation}
本論文の連接関係の解析モデルを,実際の技術文書に適用して評
価した.評価結果は,表2に示す通りである.

\begin{table}[htbp]
  \begin{center}
    \caption{連接関係解析モデルの評価結果}
    \label{tab:2}

    \smallskip

    \begin{tabular}{|l|l|r|r|r|r|} \hline
      \makebox[18mm][c]{接続の表現} & 
      \makebox[18mm][c]{関係的意味} & 
      \makebox[18mm][c]{パターン数} & 
      \makebox[18mm][c]{文数} & 
      \makebox[18mm][l]{正しく解析で} & 
      \makebox[18mm][l]{正しく解析で} \\[-1mm]
      & & & & \multicolumn{1}{|l|}{きた文} &
              \multicolumn{1}{|l|}{きなかった文} \\ \hline
      あいだ   & 時         &   0 &   1 &     &   1 \\
      うえで   & 目的       &   1 &   2 &   2 &     \\
      が       & 逆説       &   1 &  12 &  12 &     \\
               & 前置き     &   2 &   4 &   4 &     \\
               & 対比       &   1 &   3 &   1 &   2 \\
      から     & 原因       &   1 &   3 &   3 &     \\
               & 理由       &   1 &   1 &   1 &     \\
      さいに   & 時         &   1 &   4 &   4 &     \\
      し       & 並列       &   1 &   6 &   6 &     \\
               & 理由       &   1 &   0 &     &     \\
      ため(に) & 目的       &   2 &  13 &  13 &     \\
               & 原因       &   2 &   9 &   8 &   1 \\
      たら(ば) & 条件       &   1 &   1 &   1 &     \\
               & 理由       &   1 &   0 &     &     \\
               & 時         &   1 &   0 &     &     \\
      たり     & 並列       &   1 &   4 &   4 &     \\
      て/で    & 時間的継起 &   3 &  14 &  10 &   4 \\
               & 方法       &   3 &   8 &   8 &     \\
               & 付帯状態   &   1 &   6 &   6 &     \\
               & 原因       &   2 &   4 &   4 &     \\
               & 目的       &   1 &   0 &     &     \\
               & 並列       &   1 &   1 &   1 &     \\
      ても     & 逆接条件   &   1 &  10 &  10 &     \\
      と       & 条件       &   2 &  13 &  13 &     \\
               & 時         &   2 &   2 &   2 &     \\
               & 原因       &   1 &   0 &     &     \\
      といえば & 題材       &   0 &   1 &     &   1 \\
      とき     & 時         &   1 &   9 &   9 &     \\
      とともに & 時         &   0 &   1 &     &   1 \\
      ながら   & 同時動作   &   1 &   1 &   1 &     \\
               & 逆接       &   1 &   0 &     &     \\
      なら(ば) & 条件       &   1 &   0 &     &     \\
      ので     & 原因       &   1 &  12 &  12 &     \\
               & 理由       &   1 &  14 &  14 &     \\
      のに     & 逆接条件   &   1 &   2 &   2 &     \\
      ば       & 条件       &   1 &  22 &  22 &     \\
               & 並列       &   1 &   0 &     &     \\
      ばあい   & 条件       &   1 &  14 &  14 &     \\
      ように   & 対比       &   2 &   3 &   3 &     \\
               & 推量       &   1 &   1 &   1 &     \\
      より     & 対比       &   0 &   1 &     &   1 \\
      連用中止 & 並列       &   2 &  28 &  28 &     \\
               & 時間的継起 &   2 &   8 &   7 &   1 \\ \hline
      \multicolumn{1}{|c|}{合計} &
                            &  52 & 238 & 226 &  12 \\ \hline
      \multicolumn{1}{|c|}{比率(\%)} &
                            &     & 100 &  95 &   5 \\ \hline
    \end{tabular}
  \end{center}
\end{table}

まず,接続の表現として最も多く使われる20の接続の表現を選択
した.表2には,この20の表現に,評価した例文に出てきた4つの
表現を追加して挙げてある.複文中の単文と単文を接続する表現を
中心として選択し,複文と複文を接続する表現は除いた.また,本
論文では,前述のように連用修飾節を解析の対象としたので,「の」,
「こと」などによる連体修飾節に対する接続の表現は省いてある.

上記20の接続の表現に対する連接関係パターンを解析した.解析
のための例文としては,複文に関する論文集\cite{Jinta1995} から145
文,日本語教育の参考書\cite[など]{YokobayasiAndSimomura1988,Houjou1992} から323
文,ネットワークの解説書\cite{KaneutiAndImayasu1993} の前半から312文を
選択した.合計780文の連接関係に適合する52の連接関係パターン
を抽出した.連接関係パターンは図3に示すような構成をしており,
表1に示す素性の有/無がパターン毎に異なる.各々の接続の表現
毎の関係的意味に対するパターン数は表2に示す通りである.

この連接関係パターンを,ネットワークの解説書の後半からとっ
た別の238文に適用した結果が,表2の連接関係モデルの評価結果
である.用意した20の接続の表現に含まれないものが4例,正しく
解釈できなかったものが8例あった.合計して5\%の文は正しく解釈
できなかったが,残りの95\%は正しく解析することができた.正し
く解析できたかどうかの判断の基準は,「〜して」形式接続につい
ては複文に関する論文集\cite{Jinta1995} ,その他については日本語
教育の参考書\cite[など]{YokobayasiAndSimomura1988,Houjou1992} の例文に因った.似
た例文を探して,その例文の連接関係の意味を採用した.連接関係
の意味が正しいかどうかの判断の結果は,3名の語学の研究者に見
てもらい,異議の出たものは修正した.

\section{むすび}
本論文では,従属節と主節の,動詞と主体の属性を用いて,連接
関係の関係的意味を解析するシステムを作成した.動詞と名詞の意
味的関係を表すために,動詞と名詞の意味分類を用いた格パターン
があると同様に,従属節と主節の連接関係にも,各々の節を構成す
る動詞と主体の属性を用いた連接関係パターンが存在すると考える
ことができる.動詞の属性として,動詞の意志性,意味分類,慣用
的表現,ムード・アスペクト・ヴォイス,主体の属性として,主節
と従属節の主体が同一かどうか,無生物主体かどうかを採用した.

このシステムを,実際の技術文書に適用して評価した結果,95\%
の正しい解析結果を得ることができた.

今回は,連接関係パターンを手作業で抽出したが,確率モデルを
採用し,パターンの属性の組み合わせを学習するシステムに拡張す
る事が今後の研究課題である.



\bibliographystyle{jnlpbbl}
\bibliography{jpaper}

\begin{biography}
\biotitle{略歴}
\bioauthor{向仲 \smbfkou}{
1953九州大学工学部電気工学科卒業.
同年,日本電気(株)入社.
基本ソフトウェア開発に従事.
平成5年より金沢経済大学教授.
平成9年より江戸川大学教授,現在に至る.
自然言語理解,エキスパートシステムの研究に従事.
情報処理学会,言語処理学会,人工知能学会,ACL,ACM,IEEE各会員.}

\bioreceived{受付}
\biorevised{再受付}
\bioaccepted{採録}

\end{biography}

\end{document}
