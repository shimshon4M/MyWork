




\documentstyle[jnlpbbl]{jnlp_j_b5}

\setcounter{page}{41}
\setcounter{巻数}{4}
\setcounter{号数}{2}
\setcounter{年}{1997}
\setcounter{月}{4}
\受付{1996}{4}{25}
\再受付{1996}{6}{14}
\再々受付{1996}{8}{2}
\採録{1996}{9}{19}

\setcounter{secnumdepth}{2}

\title{意味的制約を用いた日本語名詞における間接照応解析}
\author{村田 真樹\affiref{KUEE} \and 長尾 真\affiref{KUEE}}

\headauthor{村田 真樹・長尾 真}
\headtitle{意味的制約を用いた日本語名詞における間接照応解析}

\affilabel{KUEE}{京都大学工学部 電子通信工学教室}
{Department of Electronics and Communication, Kyoto University}

\jabstract{
照応現象の一つに,文章中に現れていないがすでに言及されたこと
に関係する事物を間接的に指示する間接照応という用法がある.
間接照応の研究はこれまで自然言語処理においてあまり行な
われていなかったが,文章の結束性の把握や意味理解において重要
な問題である.間接照応の解析を行なうには,
二つの名詞間の関係に関する知識として
名詞格フレーム辞書が必要となるが,
名詞格フレーム辞書はまだ存在していないので,
「名詞Aの名詞B」の用例と用言格フレーム辞書を代わりに利用することにした.
この方法で,テストサンプルにおいて
再現率63\%,適合率68\%の精度で解析できた.
このことは,名詞格フレーム辞書が存在しない現在においても
ある程度の精度で間接照応の解析ができることを意味している.
また,完全な名詞格フレーム辞書が利用できることを仮定した実験も
行なったが,この精度はテストサンプルにおいて
再現率71\%,適合率82\%であった.
また,名詞格フレーム辞書の作成に
「名詞Aの名詞B」を利用する方法を示した.
}

\jkeywords{機械翻訳,間接照応,AのB,用例,格フレーム}

\etitle{
Indirect Anaphora Resolution in Japanese Nouns\\
using Semantic Constraint}
\eauthor{Masaki Murata \affiref{KUEE} \and Makoto Nagao\affiref{KUEE}} 

\eabstract{
A definite noun phrase can indirectly refer to an entity that has 
already been mentioned before. For example, ``There is a house. The roof is
white.'' indicates that ``the roof'' is associated with ``a house'',
which was mentioned in a previous sentense. This kind of references 
(indirect anaphora) has not been 
studied well in natural language processing, but is important for 
coherence resolution, language understanding, and machine translation. 
When we analyze indirect anaphora, 
we need a case frame dictionary for nouns 
containing a knowledge about relations between two nouns. 
But no noun case frame dictionary exists at present. 
Therefore, we are forced to use examples of ``A of B'' 
and a verb case frame dictionary, instead. 
We experimented the estimation of indirect anaphoras 
by using this information, 
and obtained a recall rate of 63\% and 
a precision rate of 68\% on held-out test sentences. 
This indicates that 
the information of `A of B` 
is useful to a certain extent 
when we can not make use of a noun case frame dictionary. 
We made an estimation in the case 
that we can use a good noun case frame dictionary, 
and obtained the result with the recall and the precision rates of 
71\% and 82\%, respectively. 
Finally we proposed how to construct a noun case frame dictionary 
from examples of ``A of B''.
}

\ekeywords{Machine translation, Indirect anaphora, A of B, Example, Case frame}

\begin{document}
\maketitle


\section{はじめに}

\newenvironment{indention}[1]{}{}

照応現象の一つに,文章中に現れていないがすでに言及されたこと
に関係する事物を間接的に指示する間接照応という用法がある
\cite{yamanashi92}
.
たとえば,「家がある.屋根は白い.」の場合,
「屋根」は前文の「家」の屋根である.
間接照応の研究はこれまで自然言語処理においてあまり行な
われていなかったが
\footnote{文献\cite{Tanaka1}では
化学の世界に限定して名詞「体積」の間接照応の解析をしているが,
一般の名詞すべてに対して間接照応の解析を行なっている研究はない.}
,文章の結束性の把握や意味理解において重要
な問題である.そこで,我々は二つの名詞間の関係に関する知識を用いて
日本語文章上でこの問題を解決することを試みた.
間接照応の照応詞としては
名詞句,指示詞,ゼロ代名詞が考えられるが,
本論文では,名詞句が照応詞である場合の間接照応だけを対象とする.

\section{間接照応の解析方法}
\label{sec:how_to}

間接照応の照応先
になりえる事物は,
間接照応の照応詞によってある程度限定される.
例えば,
以下の例文のように「屋根」が照応詞である場合は,
照応先は「家」などの建物に限定される.
\begin{equation}
  \begin{minipage}[h]{10cm}
家がある.\underline{屋根}は白い.
  \end{minipage}
\label{eqn:mouhitori_ojiisan_hoho_kobu}
\end{equation}
そこで,間接照応の解析を行なうには,
間接照応の照応先と照応詞の間の条件を
記載した辞書が必要となる.

照応先と照応詞の間の条件を記載するとき,
照応先でまとめて記載するか,
照応詞でまとめて記載するかの問題がある.
照応先でまとめて記載すると関係の種類が爆発的に増加することになるが,
照応詞でまとめて記載すると関係の種類の数をある程度の数に抑えることができる.
例えば,「家族」が間接照応の照応詞である場合,
照応先としては「人」ぐらいであるが,
「家族」が間接照応の照応先である場合,
照応詞としては「人数」「成員」「文化的水準」などと
多様なものが想定される.
よって,照応詞でまとめて記載する方が効率的であることがわかる.

\begin{table}[t]
  \begin{center}
    \caption{名詞格フレーム辞書の例}
    \label{tab:noun_case_frame}
\begin{tabular}{|l|l|l|}\hline
照応詞       & 照応先となりえるもの    & 照応詞と照応先の関係\\\hline
家族         & 人                      & 所属\\\hline
国民         & 国                      & 所属\\\hline
元首         & 国                      & 所属\\\hline
屋根         & 建物                    & 全体--部分\\\hline
模型         & 生産物(飛行機,船)      & 対象\\\hline
行事         & 組織                    & 関与\\\hline
人格         & 人                      & 所有\\\hline
教育         & 人                      & 行為者\\
             & 人                      & 受益者\\
             & 能力(数学,技術)        & 対象\\\hline
研究         & 人,組織                & 行為者\\
             & 学問(数学,技術)        & 対象\\\hline
\end{tabular}
\end{center}
\end{table}

照応詞でまとめると二つの名詞の間の関係に関する辞書の表現は
動詞の格フレームに似たものになる.
この名詞に対する辞書を{\bf 名詞格フレーム辞書}と呼ぶことにする.
名詞格フレームの例は表\ref{tab:noun_case_frame} のようなものである.

ところが,このような辞書は今のところ存在しない.
そこで,名詞格フレーム辞書の代わりに,
照応詞が用言の派生語である場合は
用言の格フレーム辞書を用い,
照応詞が用言の派生語でない場合は,
「名詞Aの名詞B」の用例を用いる.

間接照応の解析は以下の手順で行なう.
\begin{enumerate}
\item 
\label{enum:youso_kenshutu}
解析する名詞に対して,
用言格フレーム辞書と「AのB」の用例を用いて,
間接照応先を求める必要のある空の要素を検出する.
解析する名詞が用言からの派生語である場合は
用言格フレーム辞書を用い,
用言からの派生語でない場合は「AのB」の用例を用いる.

\begin{table}[t]
  \caption{動詞「解析する」の格フレーム}
  \label{tab:kuitigau_frame}
  \begin{center}
\begin{tabular}[h]{|l|l|l|}\hline
表層格   & 意味素性 &  用例\\\hline
ガ       & HUM(人間)       & 生徒,彼\\
ヲ       & ABS(抽象名詞)/PRO(具体物)     & 値/資料\\\hline
\end{tabular}\\
  \end{center}
\end{table}

\begin{table}[t]
  \caption{主題の重み}
  \label{fig:shudai_omomi}
\begin{center}
    \newcommand{\mn}[1]{}
\begin{tabular}[c]{|l|l|r|}\hline
  \multicolumn{1}{|c|}{表層表現} & \multicolumn{1}{|c|}{例} & 重み
  \\\hline
  ガ格の指示詞・代名詞・ゼロ代名詞 &
  (\underline{太郎}が)した.&21 \\\hline
名詞 は/には        &  \underline{太郎}はした.  &20 \\\hline
\end{tabular}
\end{center}
\end{table}

\begin{table}[t]
  \caption{焦点の重み}
  \label{fig:shouten_omomi}
\begin{center}
    \newcommand{\mn}[1]{}
\begin{tabular}[c]{|l|l|r|}\hline
\multicolumn{1}{|l|}{
{表層表現(「は」がつかないもので)}}  & \multicolumn{1}{|c|}{例}   & 重み \\\hline
{ガ格以外の指示詞・代名詞・ゼロ代名詞} & (\underline{太郎}に)した.& 16 \\\hline
{名詞 が/も/だ/なら} & \underline{太郎}がした.  & 15 \\\hline
名詞 を/に/,/.        & \underline{太郎}にした.  & 14 \\\hline
名詞 へ/で/から/より    & \underline{学校}へ行く.  & 13 \\\hline
\end{tabular}
\end{center}
\end{table}

例えば,以下の例文の名詞「解析」の解析を行なう場合は,
名詞「解析」は用言からの派生語なので,
動詞「解析する」の格フレームを取り出す(表\ref{tab:kuitigau_frame}).
表\ref{tab:kuitigau_frame} の動詞「解析する」の格フレームには,
ガ格とヲ格の二つの格要素があるので,
ガ格とヲ格の二つのものが間接照応先を求めるべき要素となる.
\begin{equation}
  \begin{minipage}[h]{9cm}
電気信号を利用したおかげで
物理学者たちは大量のデータを収集できるようになった.\\
そこで,素早い\underline{解析}のための方法が必要となった.
  \end{minipage}
\label{eqn:data_kuitigai}
\end{equation}

\item 
\label{enum:kouho_age}
($\,$\ref{enum:youso_kenshutu}$\,$)で検出した
間接照応先を求める必要のある空の要素に対して,
主語や主題や焦点から照応先の候補をあげる.
主語,主題,焦点の順に照応先のなりやすさがあるので,
推定にはそれに応じた重みを与える.
本論文で想定している主題や焦点とその重みを
表\ref{fig:shudai_omomi},表\ref{fig:shouten_omomi} にあげる.

例えば,
「家がある.屋根は白い.」の「屋根」が
照応詞である場合は,
前方の焦点の「家」が照応先の候補となる.
また,例文(\ref{eqn:data_kuitigai})の「解析」のガ格の空の要素
の解析をする場合だと,
主題・焦点などから
「電気信号」「物理学者たち」「大量のデータ」が照応先の候補となる.
このとき,これらの候補には表\ref{fig:shudai_omomi},
表\ref{fig:shouten_omomi} から重みを与え,
ある種の優先性を与える.

\item 
「AのB」の用例と用言格フレーム辞書による
意味的制約と,解析している名詞と候補の距離から照応先を判定する.
意味的制約としては
解析する名詞が用言からの派生語である場合は用言格フレーム辞書を用い,
その格フレームの格要素に記載されている用例との
類似度が大きいほど間接照応先になりやすいとする.
用言からの派生語でない場合は「AのB」の用例を用い,
「名詞Aの<解析する名詞>」の用例を集め,
名詞A との類似度が大きいほど間接照応先になりやすいとする.
このときの類似度は分類語彙表における類似レベルを利用する.

例えば,
「家がある.屋根は白い.」の「屋根」の間接照応先を求める場合では,
「\verb+<+名詞A\verb+>+の屋根」の用例を集め,
\verb+<+名詞A\verb+>+と意味的に近い名詞を間接照応先とする.
また,例文(\ref{eqn:data_kuitigai})の「解析」の
ガ格の空の要素の解析の場合では,
($\,$\ref{enum:kouho_age}$\,$)であげた候補
「電気信号」「物理学者たち」「大量のデータ」のうち,
動詞「解析する」の格フレームのガ格の意味素性HUM(人間)を満足し,
ガ格の用例「生徒」「彼」と意味的に近く,
照応詞「解析」と比較的近いところにある
「物理学者たち」がガ格の照応先と判定される.
同様にヲ格も解析され,「大量のデータ」が照応先と判定され,
「解析」が「物理学者たち」が「大量のデータ」に対して行なう
解析であると解析できる.

\end{enumerate}

用言の格フレーム辞書を代用する場合は
代用による誤りはあまり生じないと考えられる.
「名詞Aの名詞B」を代用する場合は,
「名詞Aの名詞B」が多様な意味関係を持つので
間接照応していない名詞対に対しても
間接照応すると判定する誤りがかなり生じると考えられる.
そこで,以下の処理を施すことによりこの誤りを減らすことにした.
\begin{enumerate}
\item 
名詞Aが「本当」などの形容詞的な名詞,
数量表現,時間を示す表現である場合,その用例は用いない.
例えば,「本当の壁」などの用例は用いない.
これは「本当」と「壁」が間接照応の関係になるとは
考えられないからである.
\item 
名詞Bが「鶴」「人間」などのような間接照応の
照応詞になりにくいものである場合,その用例は用いない.
つまり,
「鶴」「人間」などのような間接照応の照応詞になりにくいものである場合,
間接照応の解析を行なわない.
本研究で照応詞になりにくいものとみなした語の例を
表\ref{fig:hi_shouousi} にあげる.
これらの語は飽和名詞\cite{houwameishi}
とよばれるものや
関係名詞以外の名詞群と似ているが,
飽和名詞などは間接照応をする場合があるので,
これらの名詞よりもより限定されたものであると考える.
\end{enumerate}
この二つの処理だけでは代用による誤りを完全に消すことはできないが,
少しは軽減できると期待できる.

\begin{table}[t]
  \caption{間接照応の照応詞になりにくいものとみなした語の例}
  \label{fig:hi_shouousi}

\begin{center}
\begin{tabular}[c]{|p{11.5cm}|}\hline
LEP実験装置 データ解析プログラム ドイツ語 ドル相場 バス パターン認識プログラム 宇宙 烏 映画 英語 火 機械装置 共産 空き地 軍人 顕微鏡 湖 国語 昆虫 坂道 山 山道 事故 自転車 自動車 実業学校 社会 車 酒樽 植物 深夜バス 人間 杉 雪山 先進国 先進主要国 川 太鼓 団扇 男 茶 中学校 鳥 鶴 笛 天狗 動物 日本語 日本人 物干しざお 物置 物理学 物理学者 物理法則 泡箱 
\\\hline
\end{tabular}
\end{center}
\end{table}

照応詞が用言の派生語でない場合は,
基本的には上で述べたように
「名詞Aの名詞B」の用例を用いるが,
体(からだ)の一部を表わす名詞と親族呼称の場合は
間接照応先は人間と動物に絞られるのが明らかなので,
「AのB」の用例を用いず
``照応詞「体の一部を表わす名詞」-- 照応先「人間と動物を表わす名詞」'',
``照応詞「親族呼称」-- 照応先「人間と動物を表わす名詞」''
という知識を用いて解析する.
「体の一部を表わす名詞」「人間と動物を表わす名詞」の検出には
名詞意味素性辞書\cite{imiso-in-BGH}を利用した.
また,「親族呼称」の検出は
分類語彙表において分類番号が121ではじまる名詞を
「親族呼称」とすることで行なった.
この種の知識を用いて解析することは
名詞格フレーム辞書において簡単に作れる部分は作っておき,
作るのが難しい部分については「AのB」で
代用するという考え方に基づいている.
本研究では上の二つの知識の他は
すべて「AのB」の用例で対処したが,
この種の知識としては他に
``照応詞「病気・感情を表わす名詞」-- 照応先「人間と動物」'',
``照応詞「物体に対する属性名詞(色,大きさなど)」-- 照応先「物体」''
などが考えられる.
これらの知識も規則化して用いた方がよいと思われるが,
きりがないので本研究では先に述べた二つの知識だけを利用した.

以上の方法で一般の名詞における間接照応は解析できるが,
「一部」などの部分を表わす名詞や
「隣」などの空間語については
特有の処理が必要となる.
以下の例文の「一部」のように
用言の格要素である場合は,
その用言との意味的整合性の情報を利用する.
意味的整合性の情報は,その用言の格フレームのものを利用する.
\begin{equation}
\smallskip
  \begin{minipage}[h]{10cm}
物資は水、戦車、弾薬が目につくが、
おおいをかけられ積み荷が分からない車も多数ある.

\underline{一部}はさらに北西のラフハに向かい、
積み荷を降ろしてダンマンに戻るトラックと行き交うたびに砂ぼこりをあげていた.
  \end{minipage}
\label{eqn:kuruma_itibu}
\smallskip
\end{equation}
例えば,この例文では「一部」は「向かう」のガ格であるので,
「向かう」の格フレームのガ格を参照する.
ガ格にはガ格に入ることができる名詞の用例が記載されていて,
この場合は「彼」や「船」などの移動できるものが
入ることができると記載されている.
このため,間接照応先は
「彼」や「船」と意味的に近いものであることがわかる.
例文中の前文の「車」は移動できるという意味で
「彼」や「船」と意味的に近いので,
間接照応先として妥当であると判定される.

また,以下の例文の「隣」のように
体言に係る場合は,
その体言と意味的に近いものだけを
照応先とすることによって解析する.
\begin{equation}
  \begin{minipage}[h]{10cm}
お爺さんは大喜びをして家に帰りました。

そして、その夜起こったことを人々に話して聞かせるのでした。

さて、\underline{隣}の家に瘤のあるお爺さんがもう一人住んでおりました。
  \end{minipage}
\label{eqn:tonari_ie}
\end{equation}
例えば,この例文では
「隣」は係り先の体言が「家」であるので,
一文目の「家」と間接照応すると解析できる.

\section{照応処理システム}

\begin{figure}[t]
  \leavevmode
  \begin{center}
\fbox{
    \begin{minipage}[c]{6cm}
      \hspace*{0.7cm}条件部 $\Rightarrow$ \{提案 提案 ..$\;$\}\\[-0.1cm]
      \hspace*{0.7cm}提案 := ( 解の候補 \, 得点 )
    \end{minipage}
}
    \smallskip
    \caption{規則の表現}
    \label{fig:kouho_rekkyo}
  \end{center}
\end{figure}

\subsection{システムの枠組}
\label{wakugumi}

本研究では,
名詞における間接照応の解析を行なう際,
名詞,指示詞,代名詞,ゼロ代名詞などによる
直接照応の解析も同時に行なう.
まず,解析する文章を構文解析・格解析する\cite{csan2_ieice}.
その結果に対して文頭から順に文節ごとに照応解析を行なう.
照応解析は,
照応解析の手がかりとなる複数の情報をそれぞれ規則にし,
これらの規則を用いて解の候補に得点を与えて,
合計点が最も高い解の候補を
システムの解とすることによって実現する.
これは,照応解析のように複雑な問題では複数の情報が絡み合っており,
複数の情報を総合的に判断することにより解析を行なうためである.
規則に応じて候補に得点を足していく操作は,
その候補が指示対象であるという確信度が高まっていくことに対応している.

規則は,図\ref{fig:kouho_rekkyo} の構造をしている.
図中の「条件部」には文章中のあらゆる語や
その分類語彙表\cite{bgh}の分類番号や
IPALの格フレーム\cite{ipal}の情報や
名詞の指示性の情報や
構文解析・格解析の結果の情報などを条件として書くことができる.
「解の候補」には
照応先となる名詞の位置を書くことができる.
「得点」は解としての適切さの度合を表している.


\subsection{照応解析に用いる規則}

名詞の解析のために規則を13個作成したが,
これらすべてを適用順序に従って以下に示す.
以下の規則のうち,間接照応の解析のための規則は
規則\ref{enum:間接照応_非サ変名詞} 〜\ 
規則\ref{enum:itibu_case} の四つである.
規則\ref{enum:ika_kisoku} 〜\ 
規則\ref{enum:定名詞以外探索} の9個の規則は
直接照応の解析のためのものであり,
それぞれの規則で用いている専門用語および
規則の詳細については文献\cite{murata_noun_nlp}を参照せよ.
また,代名詞などの指示先も同時に解析するが
これのための規則は文献\cite{murata_deno_nl95}を参照せよ.


{
\begin{enumerate}
\item 
  \label{enum:ika_kisoku}
  「以下」「後述」の名詞や
  「次のような/次のように/次の〜点」における「次」の場合\\
  \{(次の文 \,$50$)\}
  \footnote{列挙判定規則の提案のリストを表わす.図\ref{fig:kouho_rekkyo} 参照.}

\item 
  「それぞれの」「各々の」「各」などに修飾された名詞の場合\\
  \{(特定指示として個体導入 \,$50$)\}

\item 
  「自分」の場合\\
  \{(「自分」が存在する文の主格,「自分」が主格の場合は
  「自分」を含む文の主節の主格 \,$50$)\}

\item 
  \label{enum:定名詞探索}
  推定した名詞の指示性が定名詞の場合で,
  その名詞を末尾に含み修飾語や所有者が同じ
  名詞Aが前方にある場合
  \\ \{(名詞A 
  \,$30$)\}

\item 
  \label{enum:総称名詞導入}
  名詞の指示性が総称名詞の場合\\ 
  \{(総称指示として個体導入 \,$10$)\} 


\item 名詞の指示性が不特定性の不定名詞の場合\\ 
  \{(不特定指示として個体導入 \,$10$)\}

\item
  名詞の指示性が総称名詞でも不特定性の不定名詞でもない場合\\ 
  \{(特定指示として個体導入 \,$10$)\}

\item 
  「普通」「様」「大分」「一緒」「本当」「何」などの指示対象を持たない名詞の場合\\
  \{(指示対象なし \,$50$)\}

  \begin{table}[t]
    \begin{center}
      \caption{直接照応の解析の際に名詞の指示性の情報から与える得点}
    \label{tab:teimeishidenai_doai1}
  \begin{tabular}[h]{|l|r|}\hline
  指示性の推定における得点の状況                           & 定名詞でない度合$d$\\\hline
  定名詞の得点を越える得点を総称名詞と不定名詞が持たない時 & 0\\
  定名詞の得点より1点高い得点を総称名詞か不定名詞が持つ時  & $3$\\
  定名詞の得点より2点高い得点を総称名詞か不定名詞が持つ時  & $6$\\
  定名詞の得点より3点以上高い得点を総称名詞か不定名詞が持つ時 &  規則は適用されない \\\hline
  \end{tabular}
    \end{center}
  \end{table}

\item 
  \label{enum:定名詞以外探索}
  この規則は名詞の指示性が定名詞以外の場合に適用される.
  以下の得点で用いるdとwとnの説明をする.
  dは,文献\cite{match}によって推定した指示性に基づいて
  表\ref{tab:teimeishidenai_doai1} から定まる定名詞でない度合である.
  wは,表\ref{fig:shudai_omomi},表\ref{fig:shouten_omomi}
  から定まる主題と焦点の重みである.
  nは,今解析している名詞と指示対象の候補とする名詞との間の距離を
  反映した数字である.\\
  \{
  (修飾語や所有者が同じで
  重みが$w$で$n$個前
  \footnote{
    主題が何個前かを調べる方法は,
    主題だけを数えることによって行なう.
    主題がかかる用言の位置が
    今解析している文節よりも前にある場合は,
    その用言の位置にその主題があるとして数える.
    そうでない場合はそのままの位置で数える.} 
  の同一名詞の主題 \,$w-n-d+4$)\\ 
  (修飾語や所有者が同じで,
  今解析している名詞を末尾に含む
  重みが$w$で$n$個前の主題 \,$w-n-d+4-5$)\\ 
  (修飾語や所有者が同じで重みが$w$で$n$個前の同一名詞の焦点 \,$w-n-d+4$)\\ 
  (修飾語や所有者が同じで,
  今解析している名詞を末尾に含む
  重みが$w$で$n$個前の焦点 \,$w-n-d+4-5$)\}

\item 
  \label{enum:間接照応_非サ変名詞}
  修飾節を持たず
  \footnote{
    修飾節を持っている名詞は,
    修飾節を持っている分だけ限定されていると考えられ,
    間接照応を行ないにくいと考えるため.
    },
  定名詞である度合が$d$で,用言からの派生語ではないが,
  間接照応の照応詞となる名詞Bの場合
  (定名詞である度合$d$は
  文献\cite{match}での指示性の推定における得点の状況から
  表\ref{tab:teimeishidenai_doai2} によって与えられる.
  これは,定名詞の方が不定名詞よりも間接照応しやすいと
  考えたためである.  )\\
  \{
  (「名詞Aの名詞B」の用例の名詞Aとの類似度により与えられる得点が$s$で,
  重みが$w$ で$n$個前
  にある主題を間接照応の照応先とする \,$w-n+d+s$)\\ 
  (用例との類似度により与えられる得点が$s$で,
  重みが$w$ で$n$個前の\.焦\.点を間接照応の照応先とする \,$w-n+d+s$)\\ 
  (用例との類似度により与えられる得点が$s$で,
  解析している名詞が係る動詞の主格 \,$23+d+s$)\\ 
  (用例との類似度により与えられる得点が$s$で,
  解析している名詞が係る動詞が係る名詞を間接照応の照応先とする \,$23+d+s$)\}\\ 
  主題や焦点の定義と重み$w$ は表\ref{fig:shudai_omomi},
  表\ref{fig:shouten_omomi} のとおりである.
  「名詞Aの名詞B」の用例の名詞Aとの類似度により与えられる$s$は,
  分類語彙表における名詞Aと照応先の類似レベルに応じて
  表\ref{tab:ruijido_hisahen} により与えられる.
  このとき,名詞Aが形容詞的な名詞である用例は利用しない.

  \begin{table}[t]
    \begin{center}
      \caption{間接照応の解析の際に名詞の指示性の情報から与える得点}
    \label{tab:teimeishidenai_doai2}
  \begin{tabular}[h]{|l|r|}\hline
  指示性の推定における得点の状況                           & 定名詞である度合$d$\\\hline
  定名詞の得点が最も高い時 & 5\\
  定名詞の得点が総称名詞か不定名詞の得点と同点の時 & 0\\
  定名詞の得点より1点高い得点を総称名詞か不定名詞が持つ時  & $-5$\\
  定名詞の得点より2点高い得点を総称名詞か不定名詞が持つ時  & $-10$\\
  定名詞の得点より3点以上高い得点を総称名詞か不定名詞が持つ時 &  規則は適用されない \\\hline
  \end{tabular}
    \end{center}
\bigskip
  \end{table}

\begin{table}[t]
  \leavevmode
    \caption{用言からの派生語でない場合に与える得点}
    \label{tab:ruijido_hisahen}

  \begin{center}
\begin{tabular}[c]{|l|r|r|r|r|r|r|r|r|}\hline
類似レベル & 0   & 1  & 2  & 3 & 4 & 5 & 6 & 一致\\\hline
得点   & $-$30 & $-$20 & $-$10 & $-5$ & 0 & 5 & 7 & 10\\\hline
\end{tabular}
\end{center}
\end{table}


\begin{table}[t]
 \vspace*{-1.4mm}
  \leavevmode
    \caption{用言からの派生語の場合に与える得点}
    \label{tab:ruijido_sahen}

  \begin{center}
\begin{tabular}[c]{|l|r|r|r|r|r|r|r|r|}\hline
類似レベル & 0 & 1 & 2 & 3 & 4 & 5 & 6 & 一致\\\hline
得点   & $-$10 & $-$2 & 1 & 2 & 2.5& 3 & 3.5 & 4\\\hline
\end{tabular}
\end{center}
  \vspace*{-1.3mm}
\end{table}

\item 
  \label{enum:サ変名詞}
  修飾節を持たず,
  用言からの派生語の場合,\\
  \{(ゼロ代名詞解析モジュール\cite{murata_deno_nl95}で解析する \,20)\}\\
  ゼロ代名詞解析モジュールでは,
  解析する用言からの派生語の空の格要素すべてに対して
  以下のような規則により候補をあげ,
  最も得点の大きい候補を照応先とする.
  ただし,最も得点の大きい候補が
  閾値の得点よりも小さい場合は,
  間接照応先の解析は行なわない.
  この閾値は,
  ガ格,ヲ格,ニ格,デ格の場合,それぞれ,
  15点,14点,15点,16点とし,その他の表層格の場合17点とした.
  また,格フレームに任意格の指定がある格の場合はさらに閾値に3点を加算した.

\begin{indention}{0.8cm}\noindent
  ガ格の省略の場合の規則\\
  \{(用言の格フレームのガ格の用例との類似度がsで重みが$w$で$n$個前の主題 \,$w-n*2+1+s$)\\
  (重みが$w$で$n$個前の焦点 \,$w-n+1+s$)\\
  (用例との類似度がsで今解析している節と並列の節の主格 \,$25+s$)\\
  (用例との類似度がsで今解析している節の従属節か主節の主格 \,$23+s$)\\
  (用例との類似度がsで今解析している節が埋め込み文の場合で主節の主格 \,$22+s$)\}
\end{indention}
\begin{indention}{0.8cm}\noindent
  ガ格以外の省略の場合の規則\\
  \{(用言の格フレームの格要素の用例との類似度がsで重みがwでn個前の主題 \,$w-n*2-3+s$)\\
  (用例との類似度がsで重みがwでn個前の焦点 \,$w-n*2+1+s$)\}
\end{indention}

この規則中の$s$は,用言の格フレームの格要素の用例と照応先の候補の
分類語彙表における類似レベルに応じて
表\ref{tab:ruijido_sahen} により与えられる.

\item 
  「一部」「隣」などの特殊な名詞で,助詞「の」がつく場合\\
  \{(今解析している名詞が係る名詞と同一の前方にある名詞を間接照応の照応先とする
  \, $30$)\}

\item 
  \label{enum:itibu_case}
  「一部」「隣」などの特殊な名詞で,それが用言の格要素になっている場合\\
  \{(\ref{enum:サ変名詞}と同様な解析モジュールで解析する \,30)\}

\end{enumerate}
}

上記の規則で与える 50, 30, 20, 10点などの値は,
特殊な名詞のための規則,直接照応のための規則,
用言からの派生語に対する間接照応のための規則,
照応せず新しく個体として導入されるもの
のための規則の優先順序を指定するためのものである.
また,規則\ref{enum:間接照応_非サ変名詞} で
与える 23点は主題や焦点の重みとの関係で実験的に定めた.
また,分類語彙表での類似レベルや名詞の指示性の情報から与える得点なども
実験的に定めた.

また,\ref{sec:how_to} 節で述べた
``照応詞「体の一部を表わす名詞」-- 照応先「人間と動物を表わす名詞」'',
``照応詞「親族呼称」-- 照応先「人間と動物を表わす名詞」''
の知識を用いた解析は以上の規則による解析とは別の解析によって行なう.
この種の確信度の高い間接照応の解析は
直接照応の解析を行なうよりも前に行なった方が
直接照応の解析精度があがるため,
規則の得点による解析を行なう前に行なう.
具体的にはこれらの解析は
規則\ref{enum:定名詞探索}, \ref{enum:定名詞以外探索} において
用いる名詞の所有者を推定することによって行なわれ,
この所有者が間接照応先に相当する.
所有者の推定は
意味素性が体の一部を意味するPARである名詞か,
分類語彙表の分類番号の最初の3桁が``121''である名詞
(親族呼称)に対してのみ行なう.
その名詞が存在する文の主語かそれまでの主題の中から
意味素性がHUM(人間)かANI(動物)のものを
探し出して,それを所有者とする.


\subsection{解析例}

間接照応の解析例を図\ref{tab:dousarei} に示す.
図\ref{tab:dousarei} は名詞「公定歩合」の解析を
正しく行なったことを示している.これを以下で説明する.

文献\cite{match}の方法で「公定歩合」の指示性の解析を行なうと,
不定名詞と推定されたので
\footnote{
  文献\cite{match}での推定では不定名詞となったが,
  「公定歩合」の正しい指示性は,
  「公定歩合」が「西独」の「公定歩合」であるので,
  定名詞である.
  本研究の間接照応の解析を行ない,
  間接照応となった場合それを定名詞とすることで,
  文献\cite{match}での指示性の推定精度が上がると思われる.
  しかし,間接照応する名詞がすべて定名詞となるわけではないので,
  問題はそう簡単ではない.
},前節の3番目の規則により
「不定名詞」という候補があげられそれに10点を与える.
この候補の得点が最も高い場合は
間接照応先を求めないことになる.
また,「公定歩合」は用言からの派生語でないので,
前節の4番目の規則が用いられる.
この規則により主語,主題,焦点から
「西独」「自国通貨安」「政策協調」「このドル高」といった候補があげられ,
それぞれに得点が与えられる.
さらに「公定歩合」との間の距離に応じて得点が与えられ,
また,推定した「公定歩合」の指示性が定名詞でなかったので,
間接照応しにくくなるという意味で$-5$点を各候補に与える.
さらに,「名詞Aの公定歩合」の用例の「名詞A」になっている名詞に
「日本」「米国」があり,これらの名詞との類似度に応じて
得点を与える.
「日本」「米国」と類似度の高い「西独」が
最も高い合計点をとり,
間接照応先として正しく解析された.

\begin{figure}[t]
\fbox{
\begin{minipage}[h]{13.5cm}


このドル高は、政策協調をぎくしゃくさせている.

自国通貨安を防ごうと、西独が\underline{公定歩合}を引き上げた.

\vspace{0.2cm}

\begin{center}
\begin{tabular}[h]{|l|l@{ }|r@{ }|r@{ }|r@{ }|r@{ }|r@{ }|}\hline
\multicolumn{2}{|l|}{}                 & 不定名詞    &  西独       & 自国通貨安& 政策協調   &このドル高 \\\hline
\multicolumn{2}{|l|}{3番目の規則}      &   10        &             &           &            &           \\\hline
\multicolumn{2}{|l|}{4番目の規則}      &             &   25        &  $-23$    &  $-24$     &  $-17$    \\\hline
                    &主語             &             &   23        &           &            &           \\
                     &主題 焦点 w      &             &             &    14     &    14      &    20     \\
                     &距離 n           &             &             &   $-2$    &   $-3$     &   $-2$    \\
                   &定名詞である度合$d$&             &  $-5$       &   $-5$    &   $-5$     &   $-5$    \\
                    &用例との類似度$s$ &             &    7        &  $-30$    &  $-30$     &   $-30$   \\\hline
\multicolumn{2}{|l|}{合計}             &   10        &   25        &  $-23$    &  $-24$     &   $-17$    \\\hline
\end{tabular}
\end{center}

\vspace{0.2cm}

「名詞Aの公定歩合」の用例

\hspace{0.5cm}
日本の公定歩合,米国の公定歩合

\smallskip
\end{minipage}
}
\smallskip\smallskip
\caption{間接照応の解析例}
\label{tab:dousarei}
\end{figure}

\section{実験}
\subsection{実験}

間接照応の解析を行なう前に構文解析・格解析を行なうが,
そこでの誤りは人手で修正した.
格フレームはIPALの辞書のものを用いたが,
IPALの辞書にない用言に対しては人手で格フレームを作成した.
「名詞Aの名詞B」の用例は
EDRの共起辞書\cite{edr_kyouki_1.0}のものを利用した.
格解析の修正では実験テキスト中の「名詞Aの名詞B」の格解析も
正しく行なえることを仮定して修正した.
たとえば,「主治医のすすめ」という句が実験テキスト中にある場合,
「主治医」は「すすめ」のガ格に入るということはわかっているとする.

本研究で提案した
「名詞Aの名詞B」の用例と用言格フレーム辞書を用いる方法で
間接照応の解析を行なった結果を
表\ref{tab:sougoukekka} に示す.
テストサンプルにおいても再現率63\%,適合率68\%の精度を得ているので,
名詞格フレーム辞書が存在しない現在においても
6割以上の精度で間接照応の解析ができることがわかる.

\begin{table*}[t]
\begin{minipage}[h]{14cm}
    \caption{本研究の実験結果}
    \label{tab:sougoukekka}
  \begin{center}
\begin{tabular}[c]{|@{\,}l@{\,}|@{\,}r@{}c@{\,}|@{\,}r@{}c@{\,}|@{\,}r@{}c@{\,}|@{\,}r@{}c@{\,}|@{\,}r@{}c@{\,}|@{\,}r@{}c@{\,}|}\hline
        &\multicolumn{4}{c|@{\,}}{\small 用言からの派生名詞以外} 
        &\multicolumn{4}{c|@{\,}}{\small 用言からの派生名詞}  
        &\multicolumn{4}{c|}{\small 合計}\\\cline{2-13}
        &\multicolumn{2}{c|@{\,}}{\small 再現率}
        &\multicolumn{2}{c|@{\,}}{\small 適合率}
        &\multicolumn{2}{c|@{\,}}{\small 再現率}
        &\multicolumn{2}{c|@{\,}}{\small 適合率}
        &\multicolumn{2}{c|@{\,}}{\small 再現率}
        &\multicolumn{2}{c|}{\small 適合率}\\\hline
\multicolumn{13}{|c|}{「名詞Aの名詞B」と用言の格フレームを用いた実験}\\\hline
{\small 学習サンプル}
  & 91\% & (60/66) & 86\% & (60/70) & 66\% & (23/35) & 79\% & (23/29) & 82\% & (83/101)& 84\% & (83/99)\\\hline
{\small テストサンプル}
  & 63\% & (24/38) & 83\% & (24/29) & 63\% & (20/32) & 56\% & (20/36) & 63\% & (44/70) & 68\% & (44/65)\\\hline
\multicolumn{13}{|c|}{完全な名詞格フレーム辞書を用いることができる場合の評価}\\\hline
{\small 学習サンプル}
  & 91\% & (60/66) & 88\% & (60/68) & 69\% & (24/35) & 89\% & (24/27) & 83\% & (84/101)& 88\% & (84/95)\\\hline
{\small テストサンプル}
  & 79\% & (30/38) & 86\% & (30/35) & 63\% & (20/32) & 77\% & (20/26) & 71\% & (50/70) & 82\% & (50/61)\\\hline
\end{tabular}
\end{center}

\begin{center}\begin{minipage}{0.9\textwidth}\baselineskip=14.5pt
各規則で与える得点は学習サンプルにおいて人手で調節した.\\
{
学習サンプル\{例文(43文)\cite{walker2},童話「こぶとりじいさん」全文(93文)\cite{kobu},天声人語一日分(26文),社説1テーマ(26文)\}

テストサンプル\{童話「つるのおんがえし」前から91文抜粋\cite{kobu},天声人語二日分(50文),社説半日分(30文)\}

評価に適合率と再現率を用いたのは,
間接照応を行なわない名詞を
システムが誤って間接照応を行なうと解析することがあり,
この誤りを適切に調べるためである.
適合率は間接照応の照応先を持つ名詞の要素のうち
正解した要素の個数を,システムが間接照応の照応先を持つと
解析した要素の個数で割ったもので,
再現率は間接照応の照応先を持つ要素のうち
正解した要素の個数を,
間接照応の照応先を持つ要素の個数で割ったものである.}
\end{minipage}\end{center}
\end{minipage}
\bigskip
\vspace{-1.5mm}
\end{table*}

また,
「名詞Aの名詞B」と用言の格フレームを用いた近似的な方法による実験の他に,
完全な名詞格フレーム辞書を用いることができることを仮定した評価も行なった
(表\ref{tab:sougoukekka} の二段目).
この評価は,
「名詞Aの名詞B」と用言の格フレームを用いる近似的な方法で解析した結果において,
以下の三つの理由で誤ったものを正解として数えることによって行なった.
\begin{enumerate}
\item 
適切な用例が不足している.
\item 
副作用を示す用例が存在している.
\item 
名詞と動詞の場合で格フレームが異なる.
\end{enumerate}
実際に名詞格フレーム辞書を作って解析する場合は,
辞書に誤りが含まれることが予想され,
精度はここで示したものよりも若干低くなると思われる.


\subsection{誤りの考察}

\vspace{-0.5mm}
名詞格フレーム辞書を用いることができたとして
本手法では誤りとなるものとしては次のようなものがあった.
\begin{equation}
 \vspace{-0.5mm}
  
  \begin{minipage}[h]{10cm}
こんなひどいふぶきの中をいったいだれがきたのかと
いぶかりながら、お婆さんは言いました。\\
「どなたじゃな」\\
戸を開けてみると、そこには
全身雪でまっ\hspace{-0.2mm}しろになった\hspace{-0.1mm}\underline{娘}\hspace{-0.1mm}が立っておりました。
\end{minipage}
\end{equation}
この例文の下線部の「娘」は若い女の人という意味で用いられていて
間接照応しないが,
「お婆さん」の「娘」であると解析してしまった.
これは,名詞の役割における多義性の問題で
あり,非常に難しい問題である.

また,次のような誤りもあった.
\begin{equation}
  \begin{minipage}[h]{10cm}
各国が国内経済政策優先に走るのは、...\\
(中略)\\
日本の政策当局には、
労働需給のひっ迫や消費税導入によって便乗値上げなどから
インフレ懸念が強まるとの見方が出始めている。\\
この見方は、G7で日本への過度の\underline{期待}をけん制し
金融政策のフリーハンドを確保しておくべきだとの意見につながる。
\end{minipage}
\end{equation}
この例文の下線部の「期待」の間接照応先は,
「期待」のガ格の間接照応先を「政策当局」と解析したが,
「期待」のガ格の正しい間接照応先は「各国」である.
この解析ができるようにするには,
「期待」がかかる動詞の意味を利用する必要がある.
この例の「けん制する」の場合,
「けん制する」のガ格と「けん制する」のヲ格にくる動詞のガ格とは
ほとんど一致しないと考えられる.
この知識を用いると,
ゼロ代名詞の解析において「けん制する」のガ格には「政策当局」がすでに
入っているので,
「期待」のガ格に「政策当局」が入るという誤りはなくなると考えられる.
このような知識も蓄えて解析する必要がある.

また,本研究の方法では,
知識の利用は一段階しか行なわないが,
二段階の知識の利用が必要な例があった.
\begin{equation}
  \begin{minipage}[h]{10cm}
30日に竹下首相をまじえて、自民党の関係者が話し合った内容によると、1票の格差は3倍以内を目標に是正する、総定数を1減らして511に戻す、\underline{抵抗}の強い\underline{2人区}の解消は見送り、定数6の北海道1区だけ分区する方針で合意した、という。
  \end{minipage}
\label{eqn:nininku_ayamari}
\end{equation}
この例文の下線部の「抵抗」の間接照応先は一見すれば「2人区」と
思ってしまうが,深く考えれば「2人区の立候補者」であることがわかる.
このような場合は,「2人区」から「抵抗」まで間接照応するとき,
「2人区---立候補者」,「立候補者が抵抗する」
の二段階の知識の利用が必要となる.
このような場合も処理できるような枠組にしていく必要がある.
しかし,知識の二段階の利用をあらゆるものに適用すると,
間接照応でないのに誤って間接照応と解析する場合が増加し,
適合率を低下させることになる.
二段階の知識を利用する場合は,
また新たな考え方が必要である.

\vspace*{-1mm}
\begin{table}[t]
    \caption{「名詞Aの名詞B」を分類語彙表に基づいて並べかえたもの}
    \label{tab:noun_bgh}
    \vspace{-1mm}
  \begin{center}
\begin{tabular}{|l|l|}\hline
名詞B   & 名詞Aを分類語彙表に基づいて並べかえたもの\\\hline
国民    & (人間) 相手 \, (組織) 国 先進国 両国 内地 全国 日本 ソ連\\
        & 英国 アメリカ スイス デンマーク 世界 \\\hline
元首    & (人間) 来賓 \, (組織) 外国 各国 ポーランド \\\hline
屋根    & (組織) 北海道 世界 学校 工場 ガソリンスタンド スーパー\\
        & 自宅 本部 \, (生産物) 車 住宅 家 邸宅 民家 神殿 玄関 車\\
        & 車体 新車 (現象) 緑 オレンジ色 (動作) かわらぶき\\
        & (精神) 方式 (特徴) 形式 車体\\\hline
模型    & (動物) 象 \,(自然) 富士山 \,(生産物) 鋳物 マンション\\
        &  カプセル 電車 船 軍艦 飛行機 ジェット機 \,(動作) 造船\\
        & (精神) プラン \, (性質) 運行\\\hline
行事    & (人間) 皇室 王室 官民 家元 \, (組織) 全国 農村 県 日本\\
        & ソ連 寺 学校 学園 母校 \, (動作) 就任 まつり 祭り\\
        & 祝い 巡礼 \, (精神) 祝い 恒例 公式 \\\hline
人格    & (人間) わたし 私 人間 青少年 政治家 \\\hline
\end{tabular}
\vspace*{-3mm}
\end{center}
\end{table}

\begin{table}[t]
    \caption{分類語彙表の分類番号の変更}
    \vspace{-1mm}
    \label{tab:bunrui_code_change}
  \begin{center}
\begin{tabular}[c]{|l|l|l|}\hline
意味素性          & 分類語彙表の        & 変更後の\\
                  &   分類番号            &        分類番号\\\hline
ANI(動物)         &  156                & 511\\[0cm]\hline
HUM(人間)         &  12[0-4]            & 52[0-4]\\[0cm]\hline
ORG(組織・機関)   &  125,126,127,128    & 535,536,537,538\\[0cm]\hline
PLA(植物)         &  155                & 611\\[0cm]\hline
PAR(生物の部分)   &  157                & 621\\[0cm]\hline
NAT(自然物)       &  152                & 631\\[0cm]\hline
PRO(生産物・道具) &  14[0-9]            & 64[0-9]\\[0cm]\hline
LOC(空間・方角)   &  117                & 651\\[0cm]\hline
PHE(現象名詞)     &  150,151            & 711,712\\[0cm]\hline
ACT(動作・作用)   &  13[3-8]            & 81[3-8]\\[0cm]\hline
MEN(精神)         &  130                & 821\\[0cm]\hline
CHA(性質)         &  11[2-58],158       & 83[2-58],839\\[0cm]\hline
REL(関係)         &  111                & 841\\[0cm]\hline
LIN(言語作品)     &  131,132            & 851,852\\[0cm]\hline
その他            &  110                & 861\\[0cm]\hline
TIM(時間)         &  116                & a11\\[0cm]\hline
QUA(数量)         &  119                & b11\\[0cm]\hline
\end{tabular}
\vspace*{-3mm}
\end{center}
\end{table}

\section{名詞格フレーム辞書の作成に関する考察}

本論文では,完全な形の名詞格フレーム辞書を用いる代わりに,
「名詞Aの名詞B」を利用して間接照応の解析を行なった.
しかし,完全な形の名詞関係辞書をなんらかの方法で作成し,
これを用いて解析することの方が精度も高くなると期待できるし,
自然言語処理に対する正しいスタンスであると考えられる.
そこで,名詞格フレーム辞書の作成に関する考察を行なった.

「名詞Aの名詞B」の意味解析の研究が進めば,
名詞格フレーム辞書は自動的に作成することができると思われる.
この場合は非常に小さいコストで
名詞格フレーム辞書を作成できることになる.
「名詞Aの名詞B」の意味解析の研究が今のような状態でなかなか困難で,
人手で作成せざるをえない場合は,どうだろうか.
このときも,「名詞Aの名詞B」の用例に注目して作成するのがよいと考えられる.
例えば,EDRの「名詞Aの名詞B」の用例を
名詞Bの分類語彙表\cite{bgh}の分類番号に応じて並べかえ,
また,名詞Aの分類語彙表の分類番号に応じて並べかえ,
さらに,名詞Aが形容詞的な名詞である用例を省くと
表\ref{tab:noun_bgh} のようになる.
このとき並べかえに用いる分類語彙表の分類番号は,
表\ref{tab:bunrui_code_change} の変更を行なったものを用いた.
表~\ref{tab:bunrui_code_change} は,
IPAL動詞辞書\cite{ipal}の意味素性を参考にして作成したものである.
「AのB」の用例が表\ref{tab:noun_bgh} の形になれば,
そこから名詞格フレーム辞書を人手で作成するのは
そんなに大変なことではないと考えられる.
「国民」の欄の「相手」,「元首」の欄の「来賓」を取り除いたり,
特徴,性質を意味する名詞を取り除くことによって作成することになる.
また,「名詞A」にある名詞を大雑把に眺めて国を意味する名詞が多いというときは
名詞Aには国を意味するものが入りやすいという
意味素性による指定も行なうことにもなる.
ただし,用例が不足していることが考えられるので,
用例が不足していることを念頭において
人手で用例を足しながら作成する必要がある.
しかし,意味的な順番に並べかえていて
意味的に近いものが近くにあるので,
人手で用例を足すときもそれほど困難ではないと考えられる.


\section{おわりに}

本研究では,名詞における間接照応の解析方法の提案を行なった.
このとき,名詞格フレーム辞書を作成することが望ましいが,
名詞格フレーム辞書はまだ存在しないので,
「名詞Aの名詞B」の用例と用言格フレーム辞書を代わりに利用することにした.
この方法で,テストサンプルにおいて
再現率63\%,適合率68\%の精度で解析できた.
このことは,名詞格フレーム辞書が存在しない現在においても
ある程度の精度で間接照応の解析ができることを意味している.
また,完全な名詞格フレーム辞書が利用できることを仮定した実験も
行なったが,この精度はテストサンプルにおいて
再現率71\%,適合率82\%であった.
また,名詞格フレーム辞書の作成に
「名詞Aの名詞B」を利用する方法を示した.



\bibliographystyle{jnlpbbl}
\bibliography{jpaper}

\begin{biography}
\biotitle{略歴}
\bioauthor{村田 真樹}{
1993年京都大学工学部電気工学第二学科卒業.
1995年同大学院修士課程修了.
同年,同大学院博士課程進学,現在に至る.
自然言語処理,機械翻訳の研究に従事.}
\bioauthor{長尾 真}{
1959年京都大学工学部電子工学科卒業.工学博士.京都大学工学部助手,
助教授を経て,1973年より京都大学工学部教授.国立
民族学博物館教授を兼任(1976.2 -- 1994.3).
京都大学大型計算機センター長(1986.4 -- 1990.3),
日本認知科学会会長(1989.1 -- 1990.12),パターン認識
国際学会副会長(1982 -- 1984),日本機械翻訳協会初代会長(1991.3 -- 1996.6),
機械翻訳国際連盟初代会長(1991.7 -- 1993.7).
電子情報通信学会副会長(1993.5 -- 1995.4).
情報処理学会副会長(1994.5 -- 1996.4).
京都大学附属図書館長(1995 -- ).
パターン認識,画像処理,機械翻訳,自然言語処理等の分野を並行して研究.}

\bioreceived{受付}
\biorevised{再受付}
\biorerevised{再々受付}
\bioaccepted{採録}

\end{biography}

\end{document}
