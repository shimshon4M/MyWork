    \documentclass[japanese]{jnlp_1.4}
\usepackage{jnlpbbl_1.1}
\usepackage[dvips]{graphicx}
\usepackage{amsmath}
\usepackage{amssymb}
\usepackage{hangcaption_jnlp}
\usepackage{ascmac}
\usepackage{udline}
\setulminsep{1.2ex}{0.2ex}
\setlength{\tabcolsep}{1zw}

\newcommand{\argmax}{}
\newcounter{exp}
\setcounter{exp}{0}
\def\exp#1{}
\def\toolref#1{}

\Volume{15}
\Number{3}
\Month{July}
\Year{2008}

\received{2008}{2}{7}
\revised{2008}{3}{27}
\accepted{2008}{3}{31}

\setcounter{page}{115}


\jtitle{要約事例を用例として模倣利用したニュース記事要約}
\jauthor{山本 和英\affiref{Author_1} \and 牧野  恵\affiref{Author_1}}
\jabstract{
現在,文書の要約をユーザへ提示することで支援を行う自動要約の研究が盛ん
に行われている.既存研究の多くは語や文に対して重要度を計算し,その重要
度に基づいて要約を行うものである.しかし我々人間が要約を行うときには文
法などの知識やどのように要約を行ったら良いのかという様々な経験を用いて
いるため,我々は人間が要約に必要だと考える語や文と相関のあるような重要
度の設定は難しいと考える.さらに人間が要約を行う際は様々な文の語や文節
など織り交ぜて要約を作成するため,文圧縮や文抽出の既存研究ではこのよう
な人間が作成する要約文は作ることができない.そこで本論文ではこれらの問
題点を解決し,人間が作成するような要約を得るため用例利用型の要約手法を
提案した.この要約手法の基本的なアイデアは人間が作成した要約文(用例)
を模倣して文書を要約することである.提案手法は類似用例文の獲得,
文節の対応付け,そして文節の組合せの3つの過程から構成される.評価実験
では従来法の一つを比較手法として挙げ,自動評価と人手による評価を行った.
人手の評価では要約文が読みやすいかという可読性の評価と要約の内容として
適切であるかという内容適切性の評価を行った.実験結果では自動評価及び人
手による評価共に従来法に比べ,本手法の方が有効であることが確認できた.
また本研究で目的としていた複数文の情報を含んだ要約文が作成されたことも
確認できた.
}
\jkeywords{複数文要約,要約事例,類似用例文,用例利用型}

\etitle{Example-based News Article Summarization \\
	by Imitating Summary Instances}
\eauthor{Kazuhide Yamamoto\affiref{Author_1} \and Megumi Makino\affiref{Author_1}} 
\eabstract{
Recently, there are a lot of automatic summarization systems.  Almost
all previous works figure an importance for each word or each
sentence, and compress or extract a sentence by using the importance
of each word or each sentence.  However, when we generate a summary,
we use much knowledge and experience in our mind.  Therefore, it is
difficult to compute the importance which correlates with human sense.
This paper proposes a new summarization method which is based on
example-based approach.  The method has three steps.  First, system
retrieves a similar instance in a instance collection to an input. The
instance collection indicates summaries which are generated by human.
In the second step, the system links the similar phrases in the input
to a phrase in the similar instance.  As third step, the system
combines the corresponding phrases, and outputs summary candidates.
Experimental results have proven that the summarization system attains
approximately 1.81 accuracy on a scale 1 to 4 by human judgments. And
the system has obtained better accuracy than previous work.  From the
examinations, the system has confirmed that the summaries were
generated by combining the phrases in many position of the input,
while those summaries are not given just by common methods such as
sentence extraction methods and sentence compression methods.
}
\ekeywords{multi-sentence summarization, summary instances, similar instance, example-besed}

\headauthor{山本,牧野}
\headtitle{要約事例を用例として模倣利用したニュース記事要約}

\affilabel{Author_1}{長岡技術科学大学電気系}{Department of Electrical Engineering, Nagaoka University of Technology}



\begin{document}
\maketitle


\section{はじめに}\label{節:背景}

近年,インターネットの普及や企業に対するe-文書法等の施行に伴い,我々の周
りには膨大な電子化文書が存在するようになってきた.そこで,ユーザが必要な
情報へ効率よくアクセスするための支援技術の研究として自動要約の研究が盛ん
に行われている.

自動要約の既存研究としては,要約する前の文章(原文)とそれを要約したもの(要
約文)のパラレルコーパスを使用し,どのような語が要約文へ採用されているのか
確率を用いることによってモデル化する手法 
\cite{Jing:2000,Daume:2002,Vandeghinste:2004}や,大量のコーパスから単語や
文に対して重要度を計算し,重要であると判断された語や文を要約文に採用する
方法\shortcite{oguro:1991論文,Hori:2003}がある.これらは計算機のスペックや大量
の言語資源を手に入れることが出来るようになったことにより近年多く研究され
ている.


しかし言語を全て統計的に処理してしまうことはあまりにも大局的過ぎ,個々
の入力に合った出力が難しくなってしまう.また我々人間が要約を行うときに
は文法などの知識やどのように要約を行ったら良いのか等,様々な経験を用い
ている.そのため人間が要約に必要だと考える語や文と相関のあるような重要
度の設定は難しい.さらに人間が要約を行う際は様々な文の語や文節など織り
交ぜて要約を作成するが,既存手法である文圧縮や文抽出ではこのような人間
が作成する要約文は作ることができない.

そこで本論文では人間が作成するような要約文,つまり複数の文の情報を織り
交ぜて作成する要約文の作成を目指す.また上述のように語や文などに人間と
同じように重要度を設定することは困難であるため,本論文ではこれらに対し
て重要度の設定を行わずに用例を模倣利用することによって要約文を獲得する
方法を提案する.


以下,\ref{章:用例利用型のアプローチ}章にて用例利用型の考え方と既存研
究,また用例利用型を要約にどう適用するのか述べる.続いて\ref{章:提案法
のシステム概要}章にて提案法のシステム概要を述べ,\ref{章:類似用例文の
選択}章から\ref{章:文節の組合せ}章にて提案法の詳細を述べる.そして
\ref{章:評価実験及び考察}章にて実験,\ref{章:結果及び考察}章にて結果及
び考察を行う.




\section{用例利用型のアプローチ}\label{章:用例利用型のアプローチ}

\subsection{用例利用型の既存研究}

用例利用型のアプローチ(example-based approach)とは事例を模倣して要約や翻
訳など言語を作成することであり,アナロジーに基づく翻訳手法として
Nagao~\cite{Nagao:1984}によって提唱された.Nagaoは人間が第二言語を学ぶ際の
学習過程に注目し,機械もその人間の学習過程を真似れば翻訳ができるのではな
いかと提案した.人間が言語を習得する際にはまずは基本的で単純な文や文節を
学習し,さらに学習を進める際には今までに学習した事例の中から類似した文や
文節を模倣利用して組合せたり語句を置き換えたりすることによって新しく文を
作成している.機械も同様に今までに収集した事例を真似ることで文が作れるの
ではないかという考えである.


機械翻訳の分野では用例利用型の翻訳が実装され,これまでに良好な結果を得て
いる\shortcite{佐藤理史:1989NL,imamura:2004,Kurohashi:2005,Sato:1995}.これら
の既存研究では原言語(日本語)と目的言語(英語)のパラレルコーパスを用例とし
て使用している.そして用例がどのように翻訳が行われているか参照し,その対
を利用することで翻訳を行っている.

さらに換言処理の分野でも用例利用の考え方が用いられている.大竹\cite{大竹清敬:
2003NLP}は同一言語内における翻訳と捉え被換言表現と換言表現の対を用例とし
て利用している.そして入力となる被換言表現と類似した用例の被換言表現側を
検索しそれと対となっている換言表現を模倣して換言を行っており,良好な結果
を得ている.

このように用例利用型のアプローチは翻訳や換言で用いられており,良好な結果
が得られている.

\subsection{自動要約へ適応}\label{節:自動要約へ適応}

人間が要約を行う際にはコツがあるという.それは対象物を読んでどのような文
や文節が要旨として必要なのか判断し,より短い表現に置き換えることである.
その要旨判断の元となっているのは要約とはどういうものなのか,どういう傾向
で作成すれば良いのか等の今までの要約事例を元にした知識や経験である.図
\ref{人間が考える要約}に人間が行う要約方法の例を示す.
図\ref{人間が考える要約}では人間が要約の対象となる文章を読んで,この記
事は「監査に乗り出す」という内容の記事だと認識する.また「監査」と「捜
査」が似ている単語であるという知識も持っている.そして人間は監査や捜査
という内容の記事がどのような傾向で要約されるのかという経験も用いて,要
約文を作成する.

つまり人間が行う要約過程も前節で述べた人間が第二言語を学習する際の翻訳す
る過程と似ており,経験,事例を利用していることが分かる.そこで本論文では
機械も人間の要約する過程を真似れば自動要約ができるのではないかと考え,用
例利用型の要約を行う.

\begin{figure}[t]
\begin{center}
\includegraphics{15-3ia6f1.eps}
\caption{人間が行う事例をもとにした要約} \label{人間が考える要約}
\end{center}
\end{figure}


\subsection{用例利用型要約の利点}\label{節:用例利用型要約の利点}

用例利用型のアプローチを自動要約に用いることの利点は以下である.
\begin{enumerate}
 \item 保守が容易である\\ 
システムを構築する際には一般的に管理,保守の容易性が求められる.用例利
用型のアプローチでは使用する状況に応じて用例を変更,または加えることだ
けで容易に改良ができるため,管理や保守が容易に行うことができる.なお追
加した用例が他の用例に副作用を及ぼすことがない.これに対して,人手で要
約規則を作成する規則利用の要約は修正や保守に高いコストがかかる.
また,修正するための規則が他の規則と競合してしまうことや人手による
作成がゆえ,要約する際に必要な規則が欠ける等の問題も起こってしまう.

\item 重要度の設定が不要\\ 
これまでに行われてきた重要文抽出や文圧縮の研究の多くは頻度情報や位置情
報,タイトル情報などを用いて語や文節に対して重要度を計算していた.しか
しながら人間はさまざまな情報を考慮して要点を判断しているため,その要点
と相関のあるような重要度を設定することは難しい.これに対して用例利用型
要約では重要度の計算を必要としない.その代わりに2つの表現間の類似度を
用いるのである.本論文ではある表現に対して計算する重要度よりも2 つの表
現間対して計算する類似度の方が容易であると考える.

\item 入力の内容により適した出力が得られる\\
統計的な要約手法は一般的に要約文のコーパスに含まれる単語に対して確率を
計算する\cite{Witbrock:1999,Knight:2002}.統計的なアプローチではより大
局的な確率に注目するため入力の内容により適した出力を得ることが難しい.
しかしながら用例利用型要約では用例の中から類似した用例を見つけ,その用
例に従って要約を行うため入力によりふさわしい出力が得られる.
\end{enumerate}



\section{提案法のシステム概要}\label{章:提案法のシステム概要}

本章では用例利用型の要約である提案法の概要を示す.まず本論文で使用する
用例を紹介し,その後提案システムの全体の流れを説明する.最後に,用例を
用いた要約の既存手法を取り上げ,本手法との違いを述べる.

\subsection{用例として用いる言語資源}\label{節:用例として用いる言語資源}

本論文の用例利用型要約では用例として,日経ニュースメール
Nikkei-goo\toolref{言語資源:nikkei-goo}から配信されている速報ニュース
を使用した.この速報ニュースは月曜日から金曜日までの週に5日,1日3回の
メールによって配信されており,以下のようなものである.

\begin{screen}
\exp{例:日経メール} 日経ニュースメールNikkei-gooで配信されているメールの例


\ul{記事のタイトル}

{\setlength{\leftskip}{1zw}
九州新幹線長崎ルート、JR九州が並行在来線の運行継続
\par}

\ul{本文}

{\setlength{\leftskip}{1zw}
九州新幹線長崎ルート問題で、並行する在来線の運行をJR九州が続けることに。
沿線市町村の反対根拠が消え着工へ前進。
\par}

…………………………………………………………………………………………………………

\ul{記事のタイトル}

{\setlength{\leftskip}{1zw}
素材各社、高機能品を中国で生産
\par}

\ul{本文}

{\setlength{\leftskip}{1zw}
素材各社が中国で高機能素材を生産へ。三井化学など車などに使う高機能樹脂工
場を建設。中国素材市場の需要急伸に対応。
\par}
\end{screen}

例\ref{例:日経メール}のように配信されているニュースは記事のタイトル及
び本文(1文から3文)から構成されておりニュース記事の要点を短くまとめたも
のである.しかし2文目以降の文は1文目の付加情報であることが多いため,本
論文では用例文として利用する対象を本文の1文目に限定した.またシステム
で使用する構文解析器での誤りや解析の揺れを最小限に抑えるため,この
速報ニュースには前処理を施してから用例データベースに格納している.この
前処理は \ref {章:類似用例文の選択}章で述べる.




\subsection{システムの流れ}\label{節:システムの流れ}

提案法の要約システムは大きく分けて3つのステップから構成される.入力は
複数の文を持つ文書または1文のみの文書であり,これを用例に従い1文に要約
する.用例には前節で述べた速報ニュースを用いている.図\ref{提案システ
ムの流れ図}を用いてシステムの流れを説明する.

\begin{figure}[b]
\begin{center}
\includegraphics{15-3ia6f2.eps}
 \caption{提案システムの流れ図} \label{提案システムの流れ図}
\end{center}
\end{figure}

まずStep1では入力となるニュース記事(複数文)を受け取り,用例の集合の中
から内容の類似した類似用例文を検索する.続いてStep2で先程選ばれた類似
用例文と入力ニュース記事の文節を比較する.類似用例文の文節1つに対して
類似していると判断された入力の文節が全て対応候補として選択される.また
類似用例文の文節に対して,対応する入力の文節が1つも無い場合はその類似
用例文を使用せず,次に検索された類似用例文を使って対応付けを行う.そし
てStep3では選択された対応候補を使用し,用例により類似した文節で日本語
として繋がりの良い組合せを探索する.このステップによって出力の要約文が
得られる.

\subsection{用例に基づく要約の関連研究}

用例に基づいた言語生成の研究は主に機械翻訳または換言で実
装されているが,要約の分野ではNguyenら\cite{Le:2004}以外に前例はない.

Nguyenら\cite{Le:2004}は用例利用型の要約として,1文を入力としてそれを圧縮
する手法を提案している.この要約手法は用例として原文と要約文の対から作成
したテンプレート規則を使用している.またこのテンプレートの他にも換言
規則(lexical rule)も作成している.これは例えば,The documentをDocumentに,Two
companiesをCompaniesに換言するものである.そして入力として``The document
is very good and includes a tutorial to get you starte.''という文が入って
きたときに,``Document is very good.''  と文圧縮されるシステムである.

これに対し,提案法のシステムでは人間が作成するような要約文により近づける
ため入力の単位を複数文(ニュース1記事)として文節を組合せることにより1文に
要約する.つまり文圧縮とは異なる.そのため文字の削除率は必然的に高くなり,
Nguyenらよりも本論文で取り組む問題はさらに困難である.また本論文では人手
で作成された要約文のみを用例として用いているため原文との対応を取る必要が
ない.そのため容易に用例を収集することができ,原文と要約
文の対応コーパスが少ない特許や医学文書の分野でも効果的に要約が作成できる.


\section{類似用例文の選択}\label{章:類似用例文の選択}

本章ではニュース1記事を入力として受け取り,その記事の内容に類似した用例
「類似用例文」の選択方法を述べる.この類似用例文は入力ニュース記事が要約
文を作成する際に模倣利用するものである.

類似用例文を選ぶ際には解析誤り等を防ぐためまず初めに入力に対して文の整形
を施す.これは用例として使用している速報ニュースに対して施した文の整形と
同様の方法であり,本章で説明する.続いて類似用例文の選択や\ref{章:文節の
対応付け}章で用いる類似単語データベースについて説明し,最後に類似用例文の
選択での中心部分を述べる.

\subsection{前処理}

類似用例文の選択を行う前に,前処理として文の整形を行う.これは構文解析を
行う際に解析ミスや解析の揺れが出てしまうのを未然に防ぐために行うものであ
る.解析ミスや解析の揺れは\ref{節:類似用例文の選択}節で行う類似用例文の
選択で不当な類似スコアを与えてしまうことに繋がってしまうためできるだけ抑
えなくてはいけない.

整形を施す部分は括弧の部分であり,これらについては括弧及びその中の情報を
削除するまたは括弧のみを削除する操作を行っている.この整形は入力される
ニュース記事に対して,また用例データベースを作成する際の用例にあらかじめ
施すものである.用例文や入力のニュース記事に含まれる括弧には丸括弧( )やカ
ギ括弧「 」がある.丸括弧では直前にくる語に関する年齢,地名,代表者名,補
足情報,換言であり,このような役割である丸括弧の情報は削除してもニュース
の内容は変わらない.そのため丸括弧については括弧及びその括弧内に含まれて
いる語を削除する操作を行う.

続いて,カギ括弧の整形について述べる.カギ括弧の役割としては引用や強調が
あるが,入力記事や用例文には同じような使われ方であるにも関わらず,カギ括
弧が付いている場合とそうでない場合がある.またカギ括弧の有無だけで構文解
析の結果が異なってしまうことがあるため,これを防ぐには引用部分がある場合
には全てカギ括弧を付ける,もしくは全てカギ括弧を削除するどちらかをしなく
てはならない.しかしカギ括弧の付いていないものに対して範囲を指定し括弧を
付加することは意味解析など様々な技術を用いても確実にできるものは
少ないため後者のカギ括弧を削除する操作を行った.

\subsection{類似単語データベース}\label{節:類似単語データベース}


類似単語データベースは使われ方の似た単語を類似したものとして格納したもの
である.この場合の使われ方が似ているとは同じ語と係り受け関係を持ちやすい
ことを意味する.例えば例\ref{類似単語DB}に示すように「参入」と「進出」は
同じ係り受け関係を持ちやすい.このような語を似ていると判断し,データベー
スに格納する.

\begin{screen}
\exp{類似単語DB} 使われ方の似た単語


 参入を予定—進出を予定

 参入を発表—進出を発表
\end{screen}

この類似度を測る際にはLin~\cite{Lin:1998}の相互情報量を用いた類似度を日本
語に対応できるように改良して単語間の類似度計算を行った.Linはテキストコー
パスから係り受け関係にある2文節とその文法関係に対して``(have, SUBJ, I)''
のように3つ組($w, r, w'$)を作成している.3つ組には以下の式で与えられる相
互情報量も付加している.この相互情報量は係り受け関係の繋がりの強さを意味
する.
\begin{equation}
\begin{aligned}[b]
 I(w,r,w') & = \log\frac{P(w,r,w')}{P(r)\times P(w|r)\times P(w'|r)} \\
	   & = \log\frac{||w,r,w'||\times||*,r,*||}{||w,r,*||\times||*,r,w'||}
\end{aligned}\label{IM}
\end{equation}
上式の$*$は任意の単語であり,$||\cdot||$は出現頻度を表す.
例えば$||*,r,*||$ならば文法関係が$r$である3 つ組の出現頻度を表す.さらに
単語$w_1$と単語$w_2$の類似度を算出するために以下の式を用いて計算を行って
いる.
\begin{equation}
{Sim_w(w_1,w_2)}=\frac{\sum_{(r,w)\in T(w_1)\cap
  T(w_2)}(I(w_1,r,w)+I(w_2,r,w))}{\sum_{(r,w)\in
  T(w_1)}I(w_1,r,w)+\sum_{(r,w)\in T(w_2)}I(w_2,r,w)}
\label{simMI}
\end{equation}
式(\ref{simMI})における$T(w_i)$は式(\ref{IM})の$I(w_i,r,w')$が正となるよ
うな$(r,w')$の集合を表す.この類似度によって同じ係り受け関係を持ちやすい
ような単語が類似していると判断できる.

本論文ではあらかじめ日本経済新聞コーパス1990--2004 年\toolref{言語資源:日経}を
使用し,類似単語データベースを作成した.類似用例データベースには単語
$w_1$ と類似度が高かった上位10\%の類似した単語$w_2$を格納している.また相
互情報量を計算する3つ組($w, r, w'$)を日本語に対応できるよう,係り受け関係
にある2文節の各々の主辞($w, w'$)と係り元文節に含まれる機能語$r$へと改良し
た.この例を図\ref{図:3つ組}に示す.

続いて,類似単語データベースの一例を以下に示す.例の括弧内の数値の1つ目
は類似度を示す.2つ目はその類似度をある単語(例中の「事務所」)と類似して
いる単語(例中の「支店,室,センター」など)の中で最も高かった類似度で正規
化したものである.
\begin{screen}
\exp{nolabel01} 類似単語データベースの一例

\ul{事務所}

{\setlength{\leftskip}{1zw}
支店(0.22606~:~1.00000)、室 (0.22601~:~0.99977)、
センター(0.21635~:~0.95705)、所 (0.21327~:~0.94342)、局(0.20754~:~0.91807)
\par}

\ul{値上げ}

{\setlength{\leftskip}{1zw}
引き上げ(0.24755~:~1.00000)、値下げ(0.22883~:~0.92438)、
引き下げ(0.22575~:~0.91194)、削減(0.21192~:~0.85606)、
増産(0.20626~:~0.83321)
\par}

\ul{再建}

{\setlength{\leftskip}{1zw}
解決(0.20423~:~1.00000)、復興(0.19219~:~0.94104)、構築(0.19162~:~0.93826)、
実現(0.18707~:~0.91598)、開拓(0.17891~:~0.87602)
\par}

\ul{スポーツ}

{\setlength{\leftskip}{1zw}
サッカー(0.16291~:~1.00000)、ゴルフ(0.15383~:~0.94426)、
ビジネス(0.14445~:~0.88668)、野球(0.14405~:~0.88423)、
競技(0.14355~:~0.88116)
\par}

\ul{描く}

{\setlength{\leftskip}{1zw}
作る(0.13313~:~1.00000)、書く(0.13185~:~0.99039)、埋める(0.13058~:~0.98085)、
つくる(0.12920~:~0.97048)、生かす(0.12774~:~0.95951)
\par}
\end{screen}

\begin{figure}[t]
\begin{center}
\includegraphics{15-3ia6f3.eps}
 \caption{相互情報量を計算する3つ組の例} \label{図:3つ組}
 \end{center}
\vspace{-1\baselineskip}
\end{figure}


\subsection{類似用例文の選択}\label{節:類似用例文の選択}

類似用例文の選択では入力ニュース記事と用例データベースに格納されている用
例とを比較し,入力の内容や話題の類似した用例「類似用例文」を選択する.提
案法のシステムではこの選択された類似用例文を模倣利用して入力ニュース記事
を要約する.

内容や話題の類似した文章には共通して出現する語が多い.そ
のため提案法では入力ニュース記事に対して同じ語を多く含むような用例文を獲
得する.しかしどんな語でも共通して出現していればよいというわけではない.
文章の内容や話題を表すのに貢献する語が両者に共通して出現したときにはそれ
らの文章は類似していると言える.よって類似用例文の選択では内容をより表し
ている語に注目して比較を行わなくてはならない.本論文では
注目する語によって類似用例文の選択部分を2つの段階に分ける.


\subsubsection{述語の一致による用例文の限定}

まずは述語に注目する.述語は記事の骨子であるため内容を表す最も重要な語で
ある.そのため類似用例文として獲得する用例文を入力の述語を1つ以上含むも
のに限定する.この段階で行う操作の流れを図\ref{図:類似用例1}に示す.

図\ref{図:類似用例1}の入力及び用例文の述語群の獲得をするための規則を説明
する.規則を適用する際は文節情報,品詞情報を使用するため入力ニュース記事
が複数文であるときは1文単位に分割し各々の文に対して構文解析を行う.用例文
は全て1文であるのでそのまま構文解析を施す.本論文では構文解析器
CaboCha\toolref{ツール:cabocha}を用いた.このツールには文節内の最も重要と
なる語である主辞や助詞などの機能語を判定する機能も含まれている.続いて構
文解析した結果を用いて作成した規則を適用する.なお述語には活用を考慮せず,
全て基本形を用いている.述語を取り出す規則を以下に示す.

\begin{figure}[t]
\begin{center}
\includegraphics{15-3ia6f4.eps}
 \caption{述語に注目して選択する用例文を限定} \label{図:類似用例1}
 \end{center}
\end{figure}

\begin{enumerate}
\item 文末の1文節に含まれている主辞(例外処理(4)(5)も行う)\\
	例)\\
	 決めた。   →  決める\\
	 発表した。 →  発表
\item 動詞--自立\footnote{この品詞は構文解析器Cabochaが使用している形
       態素解析器ChaSen\toolref{ツール:chasen}の品詞体系に基づいてい
       る.}+読点が含まれている文節の主辞(例外処理(4)も行う)\\
	例)\\
	 決め、→ 決める\\
	 発表し、→ 発表
\item 動詞--自立+助詞--引用が含まれている文節の主辞(例外処理(4)(5)も行う)\\
	例)\\
	 決めたと→ 決める\\
	 発表したと→ 発表
\item \uc{\mbox{例外処理1}}\\ 
	動詞「する,ある,なる」が主辞と判定された場合,そ
       の直前の形態素がサ変名詞であればそのサ変名詞を述語とする.直前の形
       態素がサ変名詞でなければ1つ前の文節を参照しその文節の主辞を述語と
       する.\\
	(「する」は機能語的に使用されていることが多いため.また
       「ある,なる」は文の内容を表すものとして情報は少ない.そのためこれ
       らの語は述語として用いず,他の語を述語とした.)\\
	例)\\
	 発表を|\kern-0.25zw\footnote{`|'は文節区切りを表す.}\ する。→ 発表\\
	 発表する。→ 発表\\
	 狙いが|ある。→ 狙い\\
	 先送りに|なる。→ 先送り
\item \uc{\mbox{例外処理2}} \\
	「予定,計画,方針,方向,見込み,見通し」のいず
       れかが主辞と判定された場合,その直前の文節の主辞を述語とする.\\
	(これらの表現は本来の意味よりも未来,推量などを
       表すムード\cite{寺村}であることが多く,「だろう」などの助動詞とし
       て扱うこともできる.そのためその語が持つ記事の内容を表す貢献度は低
       く,直前の形態素がサ変名詞または動詞であればその語を述語とした.)\\
	例)発表する|予定。→ 発表
\end{enumerate}
本論文では以上の規則を用いて述語を取り出した.続いて用例文の述語に対して
使われ方の類似した単語を付加し,用例の述語群を拡張する.これは述語群の一
致をみるときに,完全に一致した述語でなくても使われ方が似ていれば,次節で
述べる対応付けも可能であり,類似用例文として有用であるた
めである. 類似した単語は,あらかじめ作成しておいた類似単語データベース
のその述語と類似した上位3単語とした.

この処理では上述した述語群の一致を使用して類似用例文となりうる用例文を限
定した.続いて,述語が一致した用例文からさらに内容語に注目して類似用例文
を獲得する.


\subsubsection{内容語の一致による用例文の選択}\label{節:内容語の一致}

内容語の一致では先ほど述語に注目し類似用例文となりうる用例文の中から内容
のより類似したものを獲得する.この段階で行う操作を図\ref{図:類似用例2} 
に示す.

本論文で注目している内容語とは助詞,助動詞,記号以外の形態素であり,数字
は$\#$として汎化してある.ここで機能語は用いず内容語のみに限定した理由は
機能語の役割が文法的な意味を付け加えるだけで内容自体を含むものではないた
めである.また内容語の中でもより注目すべき語とそうではな
い語がある.まずは連体修飾部に含まれる内容語は注目すべきではない語である
.以下に図\ref{図:連体修飾}を示して連体修飾部について考えを示す.

図\ref{図:連体修飾}では「家電メーカの」という連体修飾部が「松
下電器産業は、」という文節に係っている.しかしこの連体修飾部は記事の内容
を直接表しているのではなく,被修飾部である松下電器産業のみと関係を持って
いる.そのため入力ニュース記事の連体修飾部の内容語は類似用例文の検索時に
は使用しない.用例文の連体修飾部も同様に使用するべきではないが,文がもとも
と短いため削除すると検索される情報が減りすぎてしまう.そのため連体修飾部
の単語を使用しない制約は入力ニュース記事のみに限る.図\ref{図:連体修飾2}
に連体修飾部の判断方法を示す.
連体修飾部の判断ではまず構文解析を行った結果を文末から参照していく.そし
て以下の条件に\ul{当てはまらない文節}(体言)を被連体修飾部として特定し,
その被連体修飾部の文節に係る部分を再帰的に参照し,連体修飾部とする.この
連体修飾部に含まれる内容語は類似用例文の選択では用いない.

\begin{figure}[t]
\begin{center}
\includegraphics{15-3ia6f5.eps}
 \caption{内容語に注目して類似用例文を選択} \label{図:類似用例2}
 \end{center}
\end{figure}
\begin{figure}[t]
\begin{center}
\includegraphics{15-3ia6f6.eps}
 \caption{連体修飾部を含む文の例} \label{図:連体修飾}
 \end{center}
\end{figure}
\begin{figure}[t]
\begin{center}
\includegraphics{15-3ia6f7.eps}
 \caption{連体修飾部の判定} \label{図:連体修飾2}
 \end{center}
\end{figure}

\begin{enumerate}
 \item   動詞--自立を含む文節
 \item   形容詞--自立を含む文節
 \item   名詞--サ変接続+読点,または名詞--サ変接続+句点を含む文節
\end{enumerate}

また複数の形態素から構成される複合名詞を形態素に分割して扱うと意味内容の
異なる用例文が出力される恐れがある.この例を例\ref{複合語}に示す.

\begin{screen}
\exp{複合語} 複合語であることを考慮しない場合の出力例

〈入力ニュース記事1〉

{\setlength{\leftskip}{2zw}
大手私鉄や\underline{\underline{地下鉄}}でもJRと同様に安全上、コインロッ
カーやごみ箱の使用をやめる動きが広がっている。JR東日本では主要駅などで
警備員を増員。(以下省略)
\par}


〈入力ニュース記事2〉

{\setlength{\leftskip}{2zw}
\ul{地下鉄サリン事件}から三年。被害者と遺族ら計四十二人がオウム真理教の
 破産管財人に損害賠償の上積みを求めていた訴訟は二十五日、東京地裁で開か
 れた口頭弁論で和解が成立し、賠償額を約十一億千九百万円とすることで合意
 した。(以下省略)
\par}

〈用例文〉

{\setlength{\leftskip}{2zw}
\underline{\underline{地下鉄}サリン事件}の実行犯のオウム真理教元幹部、北
 村被告に対する控訴審で東京高裁は無期懲役の一審判決支持し控訴棄却。
\par}
\end{screen}

\noindent
例\ref{複合語}の入力ニュース記事1では入力ニュース記事の内容語「地下鉄」が
用例文と一致している.しかし用例文で使用されている地下鉄は事件名の一部で
あり,入力で用いられている用法とは異なる.このように複合語を形態素に分割
すると本来注目すべきではない部分とも一致してしまう.また例\ref{複合語} の
入力ニュース記事2では用例文と「地下鉄」「サリン」「事件」という3形態素の
一致がある.しかしこの記事の内容は類似していない.このように長い複合語が
あると記事の内容が似ていないのも関わらず局所的に一致してしまい類似してい
ると判断してしまう可能性がある.そのため名詞が連続して出現する場合は複合
名詞として判断し,形態素で分割するのではなく1語として扱う,もしくは複合名
詞の中心的意味を表す主辞のみを扱う必要がある.提案法では汎用性を高めるた
め後者の主辞のみを扱うことにした.また主辞が接尾辞である場合には直前の形
態素も結合させている.またその直前の形態素も接尾辞である場合(例
\ref{syuzi}の2つ目の例),再帰的に接尾辞でない形態素までを結合した.

\begin{screen}
\exp{syuzi} 複合名詞の主辞が接尾辞と判定された場合 \\
 製鉄\ul{所}  →  製鉄所\\
 五人\ul{目}    →  人目  →  \#人目
\end{screen}

\noindent
これにより以下の例に示すような入力ニュース記事にも例\ref{複合語}の用例文
が類似していると判断できる.この例は複合名詞全体が一致していないが複合名
詞の主辞「事件」が一致している.

\begin{screen}
\exp{nolabel02} 例\ref{複合語}の用例と似ている入力ニュース記事の例

〈入力ニュース記事〉

{\setlength{\leftskip}{2zw}
衆院選で比例代表近畿ブロックで当選した自民党の野田実議員派の選挙違反
\ul{事件}で、公選法違反の罪に問われた同議員の地元事務所職員、仲修
平被告の上告審で、最高裁第一小法廷は七日までに、被告側の上告を棄却。(以
下省略)
\par}
\end{screen}

上記の方法により,得た内容語群を使用して入力ニュース記事と述語の一致して
いる用例文とを比較する.本論文では述語一致数により優先的に出力する類似用
例文を選び,同数の述語一致数の中でも用例文の内容語数に対する入力の内容語
一致率が高いものをより優先的に類似用例文として出力する.述語では一致数,
内容語では一致率に注目している.この理由は述語を多く含む用例文を選ぶこと
により多くの情報を圧縮でき,さらに内容語一致率の高い用例
文を選ぶことで後で行う対応付けがしやすい類似用例文を得ることが可能とな
る. 以下,例\ref{例:類似用例出力}に得られた類似用例文の例を示す.

\begin{screen}
\exp{例:類似用例出力} 類似用例文の出力例

〈入力ニュース記事〉

{\setlength{\leftskip}{2zw}
山梨県がまとめた九七年十二月の甲府市消費者物価指数は一〇一・五で前月に比
 べ〇・三%下落した。冬物衣料やゴルフプレー料が値下がりした。前年同月比で
 は一・二%の上昇。気候要因による値動きが激しい生鮮食品を除いた総合指数は
 一〇二・三で、前月比では〇・五%下落、前年同月比は一・九%上昇した。前年
 同月比では、医療保険制度の改革に伴う医療費本人負担の増加で保健医療が一二・
 四%の上昇。被服・履物、光熱・水道なども上昇した。住居家具・家事用品は家
 賃の値下げなどで低下した。
\par}

〈得られた類似用例文(文頭の数字は順位)〉
\begin{itemize}
\item[(1)]
6月の国内企業物価指数は100.5と前月比では0.1%下落、前年同月
 比3.3%上昇した。

\item[(2)]
9月の企業向けサービス価格は、前月比0.1%上昇、前年同月比では0.5%下落。

\item[(3)]
7月の国内企業物価指数は、前月比で0.3%上昇、前年同月比0.7%下落した。

\item[(4)]
日銀が26日発表した10月の企業向けサービス価格指数は、前月比0.2%上昇、前年同月比では1.2%下落した。

\item[(5)]
米国のガソリン小売価格が高騰、2月中旬に比べて24%上昇。
\end{itemize}
\end{screen}




\section{文節の対応付け}\label{章:文節の対応付け}

本章では\ref{章:類似用例文の選択}章で得られた類似用例文と入力ニュース記事
とを比較し,両者の文節で類似したものを対応付ける方法を述べる.まず対応付
けを行う単位である文節を説明する.続いて対応付けに用いた
3つの尺度である助詞の一致,固有表現タグの一致,そして単語間類似度
を述べる.

\subsection{対応付けの単位}\label{節:対応付けの単位}

類似用例文と入力ニュース記事の対応付けで用いることができ
る単位としては,形態素や文節がある.しかし形態素では単位が小さく対応付け
られない語が出現しやすい.以下の例を用いて説明する.

\begin{screen}
\exp{例:形態素対応付け} 形態素単位では対応付けられない文の例

〈入力ニュース記事1〉

{\setlength{\leftskip}{2zw}
愛知県半田市などで二十二日から二十七日にかけて、自動販売機から、変造した
韓国の\ul{五百ウォン硬貨}が見つかった。(以下省略)
\par}

〈類似用例文1〉

{\setlength{\leftskip}{2zw}
東京都千代田区で偽造された\ul{千円札}が見つかる。
\par}

…………………………………………………………………………………………………………

〈入力ニュース記事2〉

{\setlength{\leftskip}{2zw}
ウィンが十一月十日、\ul{大阪証券取引所第二部}に株式を上場する。(以下省略)
\par}

〈類似用例文2〉

{\setlength{\leftskip}{2zw}
幻冬舎は来年1月にも、\ul{ジャスダック市場}に株式を上場する。
\par}
\end{screen}

\noindent
例\ref{例:形態素対応付け}における類似用例文の「五/\footnote{`/'は形態素
の区切りを表す.}百/ウォン/硬貨/」と入力の「千/円/札/」は類似しており,
対応付けの候補となる.しかし「五/百/」という2形態素に対して「千/」は1形
態素であるため対応が取れない.また2つ目の例では類似用例文の「ジャスダッ
ク/市場/」と「大阪/証券/取引/所/第/二/部/」という部分が類似しているが,2 
形態素に対して7形態素を対応付けなくてはならず,形態素の単位では対応がと
れない.これに対して文節を用いれば「五百ウォン硬貨が」と「千円札が」が,
「ジャスダック市場に」と「大阪証券取引所第二部に」が対応付けすることがで
き,対応付けの難しさは軽減される.そのため提案法では対応付けを行
う単位として文節を考慮する.

しかし対応付けを行う単位が文節であっても明らかに対応付けが不可能な例が存
在する.以下に例を示して説明を行う.
\begin{screen}
\exp{例:文節対応付けA} 文節単位でも対応付けられない文の例

〈入力ニュース記事1〉

{\setlength{\leftskip}{2zw}
\ul{ソニーは}二十五日、有機ELディスプレー技術を開発したと発表。(以下省略)
\par}

〈類似用例文1〉

{\setlength{\leftskip}{2zw}
\setnami\uc{米有力総合病院の}\ul{メイヨー・クリニックは}炭疽菌の高性能検出技術を開発し、
 今秋にも発売を開始する。
\par}

…………………………………………………………………………………………………………

〈入力ニュース記事2〉

{\setlength{\leftskip}{2zw}
富士商は二十日、平生町に\setnami\uc{\mbox{県内最大規模の書籍販売とAVレンタルの}}
\ul{複合店アップルクラブ平生店を}オープンした。(以下省略)
\par}

〈類似用例文2〉

{\setlength{\leftskip}{2zw}
JR秋葉原駅前で9日、\setnami\uc{\mbox{地上22階建ての}}\ul{\mbox{複合ビル秋葉原UDXが}} オープン。
\par}
\end{screen}
例\ref{例:文節対応付けA}における用例文の「メイヨー・クリニックは」と入力の
「ソニーは」という類似しており,対応付けの候補となる.しかし類似用例文の
「米有力病院の」という連体修飾部の文節に対応した文節は入力には存在せず対
応が取れない.また2つ目の例における類似用例文「複合ビル秋葉原UDXが」と入
力の「複合店アップルクラブ平生店を」という文節は類似している.しかし連体
    修飾部である「地上|\kern-0.25zw$^{2}$ 22階建ての
|」に対して「県内最大規模の|書籍販売と|AVレンタルの|」という文節
は文節数が異なるため対応付けすることができない.このように連体修飾部は被
修飾語によって内容や長さが異なる.よって被修飾語が異なる場合はそれに係る
連体修飾部を対応付けることが困難である.そのため提案法では連体修飾部が存
在する文節に対しては,対応付けの単位としてその連体修飾部と被修飾部を結合
した文節を用いた.また連体修飾部は\ref{節:内容語の一致}節と同様の方法で判
断している.なお連体修飾部が文の内容の本筋ではないため,文
節が対応付けられるか比較する際は被修飾部同士による比較を行う.

以下に例\ref{例:文節対応付けB}の2つ目の例を用いて対応付けを行う際に使用す
る文節を具体的に示す.
\begin{screen}
\exp{例:文節対応付けB} 対応付けに用いる文節の例

〈入力ニュース記事〉

{\setlength{\leftskip}{2zw}
富士商は二十日、平生町に県内最大規模の書籍販売とAVレンタルの複合店アッ
プルクラブ平生店をオープンした。(以下省略)
\par}

〈入力ニュース記事から作成した文節〉

{\setlength{\leftskip}{2zw}
富士商は\\
二十日、\\
平生町に\\
県内最大規模の$|$書籍販売と$|$AVレンタルの$|$複合店アップルクラブ平生店を\\
オープンした。
\par}

〈類似用例文〉

{\setlength{\leftskip}{2zw}
JR秋葉原駅前で9日、地上22階建ての複合ビル秋葉原UDXがオープン。
\par}

〈類似用例文から作成した文節〉

{\setlength{\leftskip}{2zw}
JR秋葉原駅前で\\
9日、\\
地上22階建ての$|$複合ビル秋葉原UDXが\\
オープン。
\par}
\end{screen}


\subsection{3つの尺度を用いた文節の対応付け}\label{節:3つの尺度を用いた文節の対応付け}

本論文では\ref{節:対応付けの単位}節で説明した文節を対応付けの単位とし,類
似用例文の文節に対して類似している入力ニュース記事の文節
複数個を対応付ける.なお,対応付けを行う際は文節内に含まれる語をすべて基
本形に変更した形で行う.

対応付けは助詞の一致,固有表現タグの一致,そして単語間類似度の3つの尺度を
用いて行う.提案法では類似用例文の文節1つに対して,類似し
ている入力の文節を全て対応付ける.つまり1文節に対して複数個の対応文節が
得られる.この時,3つの尺度を用いても対応文節が見つからない場合は,
\ref{章:類似用例文の選択}章の類似用例文の選択で次に内容の似ていると判断さ
れた類似用例文を使用して再度対応付けを行う.

図\ref{図:対応付け例}を用いて以下にそれぞれの尺度について述べる.


\subsubsection{助詞の一致}

本論文では類似用例文と入力の文節末尾にある助詞が一致した場合,それらの文
節を対応付ける.日本語では助詞をみることで概ね主語や目的語などを判断する
ことができる.そのため助詞が一致した文節は同じような使われ方をしている
ため,文節の対応付ける尺度の1つとして用いた.

\begin{figure}[t]
\begin{center}
\includegraphics{15-3ia6f8.eps}
 \caption{入力ニュース記事と類似用例文での文節対応例} \label{図:対応付け例}
 \end{center}
\vspace{-1\baselineskip}
\end{figure}

助詞の一致を測る際には文節が複数の文節で構成されている場合(連体修飾部と被
修飾語の文節)は1番末尾の文節,1文節で構成されている場合はその文節から助詞
を取り出す.そして入力と類似用例文のその助詞を比較し,一致した場合文節を
対応付ける.但し例\ref{例:格助詞相当句}のように複数の助詞が連接して出てく
る場合,その助詞は格助詞相当句として連結した形で用いる.
\begin{screen}
\exp{例:格助詞相当句} 格助詞相当句となる助詞の一例

〈では〉

{\setlength{\leftskip}{2zw}
 で(助詞--格助詞--一般)、 は(助詞--係助詞)として出現  → 格助詞相当句「で
 は」として連結して用いる
\par}

〈には〉

{\setlength{\leftskip}{2zw}
 に(助詞--格助詞--一般)、 は(助詞--係助詞)として出現  → 格助詞相当句「に
 は」として連結して用いる
\par}
\end{screen}
係助詞「は」や「も」,格助詞「が」は主語となる文節に対して付与されている
助詞である.そのため,この3つの助詞に対しては同一視する.図\ref{図:対応付
け例}の類似用例文の文節「NTTが」に対して,入力の文節「東芝は」「同社は」
が対応付けられているのはこのためである.

また助詞の一致で対応付ける文節は入力と類似用例文の文節で主辞が同じ品詞の
グループに属するもののみとした.これは助詞が一致したとしても文節内容が明
らかに類似していない場合に対応付けを行わないようにするためである.品詞の
グループは各文節末尾の文節内に含まれている主辞の品詞\footnote{品詞の判断
にはChaSen\toolref{ツール:chasen}の品詞体系に従っている.}を参照し表\ref{表:品詞グループ}に従う.
よって助詞の一致で対応付けられる文節は助詞が一致しており,さらに品詞のグ
ループが同じものである.

\begin{table}[b]
\caption{品詞のグループ} \label{表:品詞グループ}
\input{06table01.txt}
\end{table}




\subsubsection{固有表現タグの一致}\label{節:固有表現タグの一致}

固有表現とは人名,地名,組織名などの固有名詞の他,日付や金額などの数値表
現などのまとまりを示す.提案法では構文解析器CaboChaの
出力\footnote {構文解析器CaboChaには固有表現タグの付与機
能も備わっている.}に従って固有表現タグが一致した文節を対応付ける.この固
有表現タグが一致する文節は固有表現のまとまりが同じであることを意味し,類
似した文節である.

対応付けの際に参照する部分は助詞の一致の場合と同様,複数の文節から構成さ
れる文節は末尾の文節,1文節で構成されている文節ならばその文節である.その
文節の主辞の固有表現タグが類似用例文と入力で一致した場合に対応付けを行う.
CaboChaが持つ固有表現タグは9つである.なお本論文では`DATE'タグと`TIME'タ
グは同じ時間表現を表しているタグであると判断し,同一視する.また類似用例
文の文節に固有表現タグ`DATE' もしくは`TIME'があるにも関わらず,入力の文節
にそのようなタグが無かった場合は空情報``$\epsilon$''を入れる.これは
`DATE'及び`TIME'タグの表現に限り,空であっても文の内容には大きく関係せず,
次のステップである文節の組合せで日本語やその内容としても正しい要約文が作
成できると考えたためである.よって空情報``$\epsilon$''があるだけでは
\ref{章:類似用例文の選択}章に戻って次に内容の似た類似用例文を選ぶことはし
ない.


図\ref{図:対応付け例}では類似用例文の文節「NTTが」に対して入力の文節
「東芝は」が組織名を表す`ORGANIZATION'タグが一致したため対応付けられてい
る.また類似用例文の「今秋にも」と入力の「来年六月に」は日付表現`DATE'タ
グが一致したため対応付けられたものである.

\subsubsection{単語間の類似度}\label{節:単語間の類似度}

単語間の類似度では上述の2つの尺度と同様,文節末尾の文節を比較する.提案法
では特に文節内の主辞同士の類似に注目して対応付けを行った.対応付けには
\ref{節:類似単語データベース}節で説明した類似単語データベースを用いた.類
似用例文と入力における文節末尾の文節内の主辞同士を比較し,その2語が類似単
語データベースに含まれている場合はその文節を対応付けている.またこの対応
付けでも助詞の一致と同様に品詞のグループによる制限を設けた.つまり単語間
の類似度では文節内の主辞を比較して類似単語データベースに含まれ,さらに主
辞の品詞が表\ref{表:品詞グループ}で同じグループに属するときにそれらの文節
を対応付ける.

\section{文節の組合せ}\label{章:文節の組合せ}

\ref{章:文節の対応付け}章の文節の対応付けでは,類似用例文の文節1つに
対して入力の複数個の文節が対応付けられた.本章ではその得られた対応文節を
組み合せることによって要約文を作成する方法を述べる.用例利用型の要約では
出力される要約文が類似用例文の文節により似たもので構成され,かつ日本語と
して連接のよいものが理想である.したがって提案法では要約文を得る操作を類
似用例文の文節に対して得られる類似度を最大に,かつ日本語としてできるだけ
自然な文節列を取り出す組合せ最適化問題として定式化し,これを動的計画法に
よって解く.

また\ref{章:文節の対応付け} 章の対応付けや本章の組合せは全ての語を基本形
にして扱っている.そのため提案法では組合せを行うことによって出力された文
に対し,規則を用いて類似用例文の形と同じ形になるよう変更した.


\subsection{組合せ最適化問題}\label{節:組合せ最適化問題}

前章で得られた類似用例文に対する対応文節を組合せて要約文を作成する.組合
わせの方法を図\ref{図:組合せ}を用いて説明する.

図\ref{図:組合せ}のノード$a_i$は類似用例文の文節$A$に対して\ref{章:文節の
対応付け}章で得られた入力の対応文節を指す.また同様にノード$bi$は類似用例
文の文節$B$に対して得られた入力の文節である.\ref {章:文節の対応付け} 章
の対応付けでは類似用例文の文節1つに対して類似していると判断した入力の文節
複数個を対応付けているため図\ref{図:組合せ}では$a_i\ (i \geqq 1)$となる.なお
組合せの際には初期状態と最終状態を明確にするため文頭記号$\langle
\rm{s}\rangle$と文末$\langle/\rm{s}\rangle$を挿入する.

\begin{figure}[t]
\begin{center}
\includegraphics{15-3ia6f9.eps}
\caption{対応文節の組合せ図の例} \label{図:組合せ}
\end{center}
\vspace{-1\baselineskip}
\end{figure}


ここでノード$n_i$に対してのスコア$N(n_i)$として類似用例文の文節にどれだけ
類似しているかを与え,エッジのスコア$E(n_{i-1},n_i)$としてフレー
ズ間の繋がりの良さを与える.これにより本章の目的である,類似用例文の文節
により似たもので構成され,かつ日本語として連接の良い部分文節列を得るため
にはこのノードとエッジのスコアの総和を最大にするような経路を求める問題に帰
着できる.さらに図\ref{図:組合せ}では文頭から文末に向かう全ての組合せを2
次元空間に示したものであり,探索領域は限られている.そのためこの問題は動
的計画法で解くことができる.続いて式を用いて具体的にどのような問題を解く
のかを考える.

経路列$W_p=\{n_0,n_1,n_2,\cdots ,n_m\}$\footnote{図\ref{図:組合せ}におけ
る太線ならば$W_p=\{\langle{\rm s}\rangle,a_3,b_2,c_1,d_2,\langle/{\rm
s}\rangle\}$を通る経路.} に対し,以下のスコアを最大にするような経路を求め
る問題を考える.このとき最適経路列$\hat W_p$ は以下で与えられる.
\begin{equation}
 \hat{W_p}=W_p\hspace{5mm} {\rm s.t.}\hspace{3mm}\argmax_p \mathit{Path}(W_p)
\end{equation}
またスコア$Path(W_p)$を次式で表す.{\small
\begin{equation}
\label{scoredp}
 \mathit{Path}(W_p)=\sum_{i=0}^{m}N(n_i)+\sum_{i=1}^{m}E(n_{i-1},n_i)
\end{equation}
}
$m$は類似用例文の文節の最終番号を表す\footnote{図\ref{図:組合せ}ならば
$m=5$である.}.以下にノード重みを定義する.
\begin{equation}
 N(n_i) = \alpha\cdot \mathit{particle}(n_i)+\beta\cdot \mathit{NEtag}(n_i)
	+\gamma\cdot \mathit{MI}(n_i)
\label{nodescore}
\end{equation}
式\ref{nodescore}の$particle(n_i), NEtag(n_i), MI(n_i)$は以下の式で表さ
れる.また$\alpha,\beta,\gamma$は各スコアに対するバランスパラメータであ
る.
{\allowdisplaybreaks
\begin{align}
 \mathit{particle}(n_i) & =
  \begin{cases}
	1 & \text{ノード $n_i$ が助詞の一致で対応付けされた場合}\\
	0 & \text{それ以外}
  \end{cases}
  \label{particle} \\
 \mathit{NEtag}(n_i) & =
   \begin{cases}
	1 & \text{ノード $n_i$ が固有表現タグの一致で対応付けされた場合}\\
	0 & \text{それ以外}
   \end{cases}
   \label{NEtag}\\
 \mathit{MI}(n_i) & =
   \begin{cases}
	\mathit{sim}(n_i, ph) & \text{ノード $n_i$ が単語間の類似度で対応付けされた場合}\\
	0 & \text{それ以外}
   \end{cases}
   \label{sim}
\end{align}
}
式(\ref{sim})内の$ph$は類似用例文のある文節を示し,$sim(n_i,ph)$は類似用
例文の文節$ph$と\ref{章:文節の対応付け}章で対応付けられた入力の文節との類
似度を正規化した数値である.この数値は\ref{節:類似単語データベース}節での
類似単語データベースの作成時であらかじめ計算してあるものを使用した.

次にエッジ重みを以下に定義する.
\begin{equation}
 E(n_{i-1},n_i)=
  \begin{cases}
	\sigma\cdot\frac{1}{loc(n_{i})-loc(n_{i-1})+1} 
		& \text{$\mathit{loc}(n_{i}) \geqq \mathit{loc}(n_{i-1})$ の場合}\\
	0 & \text{それ以外}
  \end{cases}
  \label{edge}
\end{equation}
エッジスコアは文節間の繋がりの良さを示す.本論文では様々な文の文節を組み
合せることにより文を作成するが,1文目 → 5文目 → 2文目 → 10 文目の様
に1つ1つの文節があまりにも様々な文を跨いで組み合わさるようなものは多くの
話題が混在することとなり,連接も悪くな
る.そのため式(\ref{edge})では対応付けられた文節が存在す
る入力の文の位置$\mathit{loc}(n_i)$を考慮した.$\mathit{loc}(n_i)$ はノードつまり
対応文節$n_i$ が入力したニュース記事の何文目に存在しているかという情報で
ある.連接する文節$(n_{i-1},n_i)$がどれだけ離れているかを
$\mathit{loc}(\cdot)$の差の絶対値を取ることで測っている.このとき文節が文頭に向かっ
て(4文目 → 2文目のように)戻る場合は話題が戻ることとなり,連接をより悪く
してしまう可能性があるため,このような場合にはスコア0を
与えている.

以上の方法により,類似用例文の文節に似ており,さらに日本語の連接としてよ
り正しい文が作成される.しかし提案法の文節の対応付けでは図\ref{図:対応付
け例}のように入力中の同じ文節「…ヘリカルCTを」が類似用例文の文節複数「…
ネット技術を」,「サービスを」に似ていると判断される場合が存在する.この
時文節の組合せを行うと1文中に同じ文節が複数個出現してしまう恐れがある.要
約文1文中に全く同じ文節が2回以上出現する文は不自然で冗長である.そのため
提案法では,組合せによって得られた文に同じ文節が2つ以上存在した場合,組合
せによって得られた全体のスコアが次に高かった組合せ結果を採用する.本論文
では複数の組合せ解を効率的に得るために永田の後向き$A^*$アルゴリズム
\cite{永田:1999論文}を用いた.





\makeatletter
\def\footnotemark{}
\def\footnotetext{}
\makeatother

\subsection{用例の形へと合わせる}\label{節:用例の形へ合わせる}

対応文節の組合せを最適化問題として解いて得られた文は以下の例に示すように
基本形の文節で,さらに句点など入力ニュースで用いられていたそのままの形で
表されている.
\begin{screen}
\exp{例:組合せ結果} 対応文節の組合せで得られた組合せの例

〈入力ニュース記事〉

{\setlength{\leftskip}{2zw}
    中国地域ニュービジネス協議会は、|\kern-0.25zw$^{2}$ 財務や営業、技術、マーケ
ティングの分野で助言するシニアアドバイザーを|募集した。|ベンチャー企
業から|財務や営業面で弱点補強の要請が|あれば、|アドバイザーとして|派遣する。(以下省略)
\par}

〈類似用例文〉

{\setlength{\leftskip}{2zw}
産業再生機構は|リストラ特命隊を|募集し、|カネボウ化粧品へ|派遣。|
\par}

〈対応文節の組合せ結果〉

{\setlength{\leftskip}{2zw}
中国地域ニュービジネス協議会は、|財務や営業、技術、マーケティングの分野
で助言するシニアアドバイザーを|募集する\footnotemark た。|ベンチャー
企業から|派遣する。|
\par}
\end{screen}

\footnotetext{文節の対応付け,組合せは全て基本形で行ったため,組合せ
結果も基本形となっている.}

そのため本節では得られた組合せ結果を類似用例文の形に倣い変更する.例
\ref{例:組合せ結果}を類似用例文の形へと合わせると以下のような文が得られる.


\begin{screen}
\exp{nolabel03} 例\ref{例:組合せ結果}) で得られた組合せ結果を類似用例文の形へと変更

中国地域ニュービジネス協議会は、財務や営業、技術、マーケティングの分野
で助言するシニアアドバイザーを募集し、ベンチャー企業へ派遣。

\end{screen}

以下に類似用例文の形へと合わせるための規則を示す.この規則は対応文節末尾
の文節に対して行うものである.

\begin{enumerate}
\item 類似用例文の文節が文末の場合\\
\ul{対応文節も文末である場合}\\
対応する入力の文節も文末で活用語がある場合,それらの語を入力
ニュース記事の元の活用へ変更する.\\
 述べるた。→ 述べた。\\
但し対応文節が「サ変名詞+する」である場合,冗長な表現であるため文末
整形も兼ね「する」を削除する.\\
 派遣する。→ 派遣。\\
\ul{対応文節が文末以外の場合}\\
対応する入力の文節が文末以外で活用語がある場合,それらの語は基
本形のままにする.この時対応文節の助詞は削除する.\\
 述べるて、(述べて、)→ 述べる。\\
但し対応する入力の文節が過去形を表す助動詞「た」を含む場合,時
制を変えないようにするため,入力ニュース記事の元の活用へ変更する.\\
 述べるたと、(元の形は「述べたと、」)→ 述べた。

\item 類似用例文の文節が文末以外の場合\\
\ul{対応文節が文末である場合}\\
対応する入力の文節が文末で活用語がある場合,それらの語を類似用
例文の文節の活用に合わせる.この時対応文節に類似用例文のフ
レーズに含まれる助詞を付加する.\\
 類似用例文の文節:発表して\\
 述べる。→ 述べて\\
但し対応する入力の文節が過去形を表す助動詞「た」を含み,さらに類似
用例文の文節にも「た」を含む場合,時制を変えないようにするため,入
力ニュース記事の元の活用へ変更する.この時対応文節に類似用例文の文
節に含まれる助詞を付加する.\\
 類似用例文の文節:発表したと\\
 述べるた。(元の形は「述べた。」)→ 述べたと\\
\ul{対応文節が文末以外である場合}\\
対応する入力の文節も文末以外で活用語がある場合,それらの語を類似用
例文の文節の活用に合わせる.この時対応文節に類似用例文の文節に含ま
れる助詞を付加する.\\
 類似用例文の文節:発表して\\
 述べる、(元の形は「述べ、」)→ 述べて\\
但し対応する入力の文節が過去形を表す助動詞「た」を含み,さらに
類似用例文の文節にも「た」を含む場合,時制を変えないようにする
ため,入力ニュース記事の元の活用へ変更する.この時対応文節に類
似用例文の文節に含まれる助詞を付加する.\\
 類似用例文の文節:発表したと\\
 述べるたと(元の形は「述べたと」)→ 述べたと

\end{enumerate}


なお文節内の連体修飾部の文節に対しては例\ref{例:連体修飾部の活用}に示
すように全て入力ニュース記事の元の活用へと変更した.
\begin{screen}
\exp{例:連体修飾部の活用} 連体修飾部に対する変更の処理

〈入力ニュース記事〉

  小泉首相は、|\ul{○○を事前に公表した}\footnotemark ことを|受け、|…。|(以下省略)

〈類似用例文〉

  政府は、|××したものを|…。|

〈対応文節の組合せ結果〉

  小泉首相は、|\ul{○○を事前に公表}\setnami\uc{する}\ul{た}ことを
|…。|

〈連体修飾部は活用を元の記事に合わせる〉

小泉首相は、|\ul{○○を事前に公表}\setnami\uc{し}\ul{た}ことを|…。|
\end{screen}
\footnotetext{下線部は文節内の連体修飾部を示す.}


これらの規則により組合せで得られた文を類似用例文の形へと合わせた.




\section{実験}\label{章:評価実験及び考察}

\subsection{実験条件}

評価実験の際に用いるデータや設定するパラメータについて述べる.また提案法
と同じく複数文を要約するシステムである従来研究を比較対象として説明する.


\subsubsection{使用するデータ}\label{節:使用するデータ}

この節では実験で使用する用例データベース,パラメータの調整,テストに用い
たデータについて述べる.

\noindent \ul{用例データベース}

用例データベース内の用例には日経ニュースメールNikkei-goo\toolref{言語資
源:nikkei-goo}から配信されているニュースの要約文を用いた.このニュース要
約文は人手で作成されているものであり\footnote{用例データベースについては
\ref{節:用例として用いる言語資源}節の用例として用いる言語資源でも述べてい
る.},1999年12月から2007年12月までに収集した27036件を用いた.

1文あたりの平均形態素数は23.1形態素,平均文節数は6.6文節である.


\noindent \ul{パラメータ調整用のデータ}

従来手法,提案法のシステムにおけるパラメータを調整するためのデータとして,
日本経済新聞1999年のデータ121件を用いた.またこの新聞データの日付やタイト
ル情報を利用し,用例データベースと同じ形式である日経ニュースメール
Nikkei-goo\toolref{言語資源:nikkei-goo}の中から記事タイトルと日付情報が一
致したものを正解データとして利用した.このパラメータ調整用のデータは入力
ニュース記事1件に対して,正解要約文1件が1対1で存在する.

1記事当たりの平均文数は10.6文であり,1文当たりの平均形態素数は30.6形態
素,平均文節数は9.6文節である.

\noindent \ul{テストデータ}

日本経済新聞1998年のデータ200件を用いた.このデータはパラメータ調整用のデー
タとは異なり,オープンテストである.この200件のうち100件は1記事あたりの文
数が3文以下で比較的短いものである.また他の100件は1記事あたりの文数が4文
以上10文以下の長めの記事である.テストデータの詳細は表\ref{表:テストデー
タ詳細}に示す.

\begin{table}[b]
\caption{テストデータの詳細} \label{表:テストデータ詳細}
\input{06table02.txt}
\end{table}

なおテストデータ200件には3人が独立で作成した正解データ(人手で作成した要約
文)が存在する.つまりこのデータは入力ニュース記事1件に対して,正解要約文
3件が1対3で存在する.要約文の自動評価ではこれら複数の正解データを使用して
\ref{節:自動評価}節に示すBLEUとROUGEにて評価を行う.

\subsubsection{パラメータの調整}\label{節:パラメータの調整}

提案法で使用するパラメータの調整方法について述べる.調整するパラメータは
文節の組合せ時のノードとエッジのバランスを取るためのものであり,式
(\ref{nodescore})と式(\ref{edge})の$\alpha, \beta, \gamma, \sigma$である.
このパラメータの調整には\ref{節:使用するデータ}節のパラメータ調整用のデー
タを用いた.ここで対応文節の組合せを行うには,まず類似用例文を獲得し,文
節の対応付けをしなくてはならない.本論文では類似用例文の検索精度によらず
パラメータを調整するため,正解データを類似用例文として使用した.そして提
案法により文節の対応付けを行い,パラメータを適宜変動させることで組合せを
行った.

パラメータは式(\ref{nodescore})と式(\ref{edge})の$\alpha, \beta, \gamma,
\sigma$をそれぞれ0.1刻みで変動を行った.調整用のデータ121件で出力された要
約文と正解データでのBLEU値の合計が最も高かったパラメータをテストで用いる.
このBLEU値は要約や翻訳で用いられている自動評価手法であり,\ref{節:自動評
価}節で詳細を述べる.調整したパラメータを表\ref{表:提案法パラメータ}に示
す.

\begin{table}[t]
\caption{提案法のパラメータ最適値} \label{表:提案法パラメータ}
\input{06table03.txt}
\end{table}


\subsubsection{従来手法}\label{節:比較手法}

提案法と同じく複数文の語を使用して1文に要約するHori~\cite{hori:2002th}の手
法を従来手法として挙げる.この手法は複数文要約を複数の入力文中に含まれる
単語列から部分単語列を抽出する問題として定式化することによって1文の要約文
を作成している.またこの要約手法は要約率を自由に設定することができる.そ
こで本論文では従来手法が出力する要約文の形態素数と提案法が出力する要約文
と形態素数を同じに設定して比較実験を行う.


\subsection{評価方法}\label{節:評価方法}

本論文ではシステムが出力した要約文を評価するために2つの評価方法を用いた.
まず自動要約の分野でよく用いられている自動評価方法,もう1つは出力した要
約文を人手で評価する方法である.


\subsubsection{自動評価}\label{節:自動評価}

本論文ではBLEUスコア\shortcite{BLEU}とROUGEスコア\cite{ROUGE}を用いて自動評価
を行う.これらは入力に対してあらかじめ作成した正解要約文とシステムが出力
した要約文を比較し,正解にどれだけ近い文が得られたか評価することによって
システムの優劣を測るものである.

\noindent \ul{\mbox{BLEUスコア}}

BLEUスコアは1~gramから4~gramまでの適合率の重み付き和で以下の式で定義され,
複数の正解文にも対応した評価尺度である.
\begin{align}
 {\rm BLEU}({\rm sys},{\bf ref}) & = \mathit{BP}\cdot{\rm exp}\left(\sum_{n=1}^{4}\frac{1}{n}\log p_n\right)
	\label{bleu1}\\
 p_n & = \frac{\sum^{M_n}_{j=1}\min(s^n_j,\max_{k=1,\dots,L}r_{jk}^n)}{\sum^{M_n}_{j=1}s^n_j}
	\label{bleu2}
\end{align}
式(\ref{bleu2})の$s_j^n$はシステムが出力した要約文に含まれる$j$番目の
$n$gramの出現数を表す($j=1,...,M_n;\ n=1,...,4$).また$r_{jk}^n$はその
$j$番目の$n$gramが$k$番目の正解要約文に出現する数である($k=1,...,L$).こ
のとき$L$は正解要約文の数である.よって\ref{節:パラメータの調整}節のパラ
メータの調整時は正解要約文が1つであるため$L=1$,テスト時には正解要約文が
3つ存在するため$L=3$となる.

またこの評価式は適合率であるため,システムが出力した文が正解文に対してあ
まりにも短いと評価を不当に上げてしまう恐れがある.そのためペナルティ
($BP$)が導入されている.しかし,要約文の評価では短い出力文である方が高圧
縮であり一般的に良いとされ,この$BP$は課さない場合が多い.そのため本論文
でも同様に$BP=1$として評価を行う.

\noindent \ul{\mbox{ROUGEスコア}}

ROUGEスコアは正解要約文とシステムが出力した要約文を比較して$N$gram再現率
を算出することにより正解文にどの程度近いかを評価することができる.以下に
ROUGEスコアの式を示す.
\begin{equation}
 {\rm ROUGE-}N({\rm sys,ref_\mathit{k}})
	=\frac{\sum^{M_N}_{j=1}\min(s^N_j,r_{jk}^N)}{\sum^{M_N}_{j=1}r^N_{jk}}
	\label{rouge}
\end{equation}
式(\ref{rouge})の分母は正解要約文に含まれる$N$gram総数であり,分子はシス
テムと正解要約文で一致した$N$gramの総数である.Lin~\cite{ROUGE}によると
1~gram再現率または2~gram再現率を測ったときに人手の評価と相関の高い結果が得
られたとしている.つまりROUGE-1またはROUGE-2の場合である.そのため本論文
でもこのROUGEスコア(ROUGE-1及びROUGE-2)を用いて自動評価を行う.なお
\ref{節:使用するデータ}節で示したようにテストデータには1件の入力ニュース
記事に対して3つの正解要約文が存在する.そのためROUGE尺度による評価実験で
は入力ニュース記事1件に対して,システムが出力した要約文と正解要約文それぞれ
とでROUGE-1及びROUGE-2を算出し,最も高い値が得られたものを評価値として採
用する.

\subsubsection{人手による評価}\label{節:人手による評価}

人手による評価では,評価者3人が提案法と従来手法それぞれが出力した要約文を
可読性と内容の適切性の2点について評価を行った.

\noindent \ul{可読性の評価}

可読性の評価では評価者3人が独立にシステムが出力した要約文のみを読み,表
\ref{表:可読性の評価}の指標に基づいて4段階評価を行った.この評価ではシス
テムが作成した要約文が日本語として読み易いかを評価するものであり,値が小
さいほど可読性は良い.

\noindent
また評価者に与えた教示は以下の通りである.
\begin{screen}
 システムが出力した要約文を読み,表\ref{表:可読性の評価}に基づいて,評価
 値を付与しなさい.

 また,以下に示す例のように述語(「逮捕」及び「述べた」)はどちらも「ガ,デ,
 ヲ格」を取るが,例2の文は明らかに日本語として不適切な表現(述べた)がある
 ため,4文節中1文節を変更しなくてはならない.そのため例2の場合は評価2とな
 る.

例1)

 新潟県警が/○○疑惑で/××容疑者を/\ul{逮捕。}/

例2)

 新潟県警が/○○疑惑で/××容疑者を/\ul{述べた。}/

\end{screen}
\noindent \ul{内容の適切性評価}

内容の適切性評価では可読性の評価を行った同じ評価者3人が入力のニュース記事
とシステムが出力した要約文を読んで,表\ref{表:内容の適切性評価}の指標に基
づいて内容の適切性評価を行った.この評価値は小さい方が内容の適切性は良い.

\begin{table}[b]
\caption{要約文の可読性評価の指標} \label{表:可読性の評価}
\input{06table04.txt}
\end{table}
\begin{table}[b]
\caption{要約文の内容の適切性評価の指標} \label{表:内容の適切性評価}
\input{06table05.txt}
\end{table}


また評価者に与えた教示は以下の通りである.
\begin{screen}
 システムが出力した要約文を読む前に,システムに入力した記事のみを読んで,
 自分が要約文に必要だと考える内容を考えなさい.その後,自分が考えた内容と
 システムが出力した要約文を比較して,表\ref{表:内容の適切性評価}に基づい
 て,評価値を付与しなさい.
\end{screen}



\section{結果及び考察}\label{章:結果及び考察}

\subsection{実験結果}\label{節:実験結果}

\subsubsection{自動評価による結果}\label{節:自動評価による結果}

\noindent \ul{\mbox{BLEUスコアによる結果}}

表\ref{表:BLEU結果}に自動評価尺度BLEUにて評価を行った結果を示す.

\begin{table}[b]
\caption{自動評価尺度BLEUによる評価結果} \label{表:BLEU結果}
\input{06table06.txt}
\end{table}

表\ref{表:BLEU結果}より本手法は1--3文の入力データの場合も,4--10文の入力デー
タの場合も従来手法よりも良好な結果が得られていることが分かる.次に4--10文
のデータを入力した場合と1--3文のデータを入力した場合を比較してBLEU 値の低
下率を以下の式により算出した.
\begin{equation}
\rm BLEU値低下率=1-\frac{4文から10文の入力データで得られたBLEU値}{1文か
 ら3文の入力データで得られたBLEU値}
\end{equation}


上式によりBLEU値の低下率を算出したところ,従来手法の低下率は56\%であり,
一方本手法は29\%であった.長い記事を入力として1文の要約を
作成する場合,必然的に削除率が高くなりタスクとしては困難である.しかし上
述の精度低下率をみると,本手法は長い記事を入力した場合でも短い記事を入力
したときの精度とそれ程変わらずに1文の要約文を作成できることが分かる.


\noindent \ul{\mbox{ROUGEスコアによる結果}}

続いて表\ref{表:ROUGE-1結果}と表\ref{表:ROUGE-2結果}にて自動評価尺度
ROUGEにて評価を行った結果を示す.本論文ではROUGE-1及びROUGE-2を使用して評
価を行った.まず表\ref{表:ROUGE-1結果}にROUGE-1における評価結果を示す.

表\ref{表:ROUGE-1結果}より,1文から3文のデータを入力した場合,本手法が出
力した要約文は正解文と比較して1~gram再現率が0.631で従来手法の0.462を大きく
上回っていることが分かる.また4文から10文のデータを入力した場合でも本手法
の方が優位である結果が得られた.


続いて表\ref{表:ROUGE-2結果}にROUGE-2における評価結果を示す.

\begin{table}[t]
\caption{自動評価尺度ROUGE-1による評価結果} \label{表:ROUGE-1結果}
\input{06table07.txt}
\end{table}
\begin{table}[t]
\caption{自動評価尺度ROUGE-2による評価結果} \label{表:ROUGE-2結果}
\input{06table08.txt}
\end{table}

表\ref{表:ROUGE-2結果}でもROUGE-1の結果同様に,従来手法の結果に比べて本手
法の方が良好である結果が得られた.また長めの記事データを入力した場合,従
来手法における100件中の最大ROUGE-2値は0.667に留まっているが,本手法の方は
ROUGE-2の最大値1を獲得していることも表から分かる.

\subsubsection{人手による評価の結果}\label{節:人手による評価の結果}

人手による評価では被験者3人がシステムが出力した要約文と従来手法である
Hori~\cite{hori:2002th}手法が出力した要約文の評価を行った.評価尺度等は
\ref{節:人手による評価}節に示した通り,可読性の評価(4段階評価)と内容適切
性の評価(4段階評価)の2点である.


\vspace{1\baselineskip}
\noindent \ul{可読性の評価結果}

表\ref{表:人手可読性}と表\ref{表:人手可読性2}に1文から3文から構成されるデー
タを入力した場合の要約文を評価者が可読性評価した結果を示す.可読性の評価
は1--4の4段階で1が良好であり,4が不良である.表\ref{表:人手可読性}は従来手
法が出力した要約文を評価した結果であり,表\ref{表:人手可読性2}は本手法の
結果を表している.

\setlength{\tabcolsep}{0.5zw}
\setlength{\captionwidth}{200pt}
\begin{table}[b]
\begin{minipage}{200pt}
\hangcaption{従来手法の要約文の可読性評価(1文から3文の入力データ時)}
\label{表:人手可読性}
\input{06table09.txt}
\end{minipage}
\hfill
\begin{minipage}{200pt}
\hangcaption{本手法の要約文の可読性評価(1文から3文の入力時)}
\label{表:人手可読性2}
\input{06table10.txt}
\end{minipage}
\end{table}

続いて表\ref{表:人手可読性3}と表\ref{表:人手可読性4}に4文から10文から構成さ
れるデータを入力した場合の要約文を評価者が可読性評価した結果を示す.
表\ref{表:人手可読性3}は従来手法の結果であり,表\ref{表:人手可読性4}は本
手法の結果である.
表\ref{表:人手可読性}から表\ref{表:人手可読性4}の結果より,本手法の可読性
評価の平均値はどの評価者においても従来手法に比べて優位で
ある結果が得られたことが分かる.また1--3文から構成されているデータを入力と
した場合と4--10文のデータを入力した場合で結果を比較すると,本手法は従来手
法に比べ評価者全員を通して評価値の低下があまり見られなかった.そのためデー
タを入力しても本手法は可読性が保たれると考えることができる.


\begin{table}[b]
\begin{minipage}{200pt}
\hangcaption{従来手法の要約文の可読性評価(4文から10文の入力時)}
\label{表:人手可読性3}
\input{06table11.txt}
\end{minipage}
\hfill
\begin{minipage}{200pt}
\caption{本手法の要約文の可読性評価(4文から10文の入力時)} \label{表:人手可読性4}
\input{06table12.txt}
\end{minipage}
\end{table}
\setlength{\tabcolsep}{1zw}
\begin{table}[b]
\caption{評価者の人数と入力データに含まれる評価値1(良好)の件数} \label{表:人手可読性5}
\input{06table13.txt}
\end{table}


続いて表\ref{表:人手可読性5}において,評価者3人全員が良好である評価値1を
付与したのが入力データ中にどの程度存在するか,また評価者2人以上が評価値1
を付与したのがどの程度存在するか調査した結果を示す.

表\ref{表:人手可読性5}より本手法において,評価者3人が共に評価値1を付与し
た入力データの件数は1--3文の入力データだと52件,4--10文の入力データだと53件
という結果が得られた.またいずれの場合も本手法は従来手法の結果よりも優位
な結果が得られたことが分かる.評価者の過半数が最も良好である評価値1を付与
したものを正解とすると本手法の正解率は1--3文の入力データだと84\%,4--10文の
入力データだと79\%である.

\vspace{1\baselineskip}
\noindent\ul{内容適切性の評価結果}

表\ref{表:人手内容適切性}と表\ref{表:人手内容適切性2}に1文から3文から構成
されるデータを入力した場合の要約文を評価者が内容適切性を評価した結果を示
す.内容適切性の評価は1--4の4段階で1が良好であり,4が不良である.表
\ref{表:人手内容適切性}は従来手法が出力した要約文を評価した結果であり,表
\ref{表:人手内容適切性2}は本手法の結果を表している.

\setlength{\tabcolsep}{0.5zw}
\begin{table}[b]
\begin{minipage}{200pt}
\hangcaption{従来手法の要約文の内容適切性の評価(1文から3文の入力時)}
\label{表:人手内容適切性}
\input{06table14.txt}
\end{minipage}
\hfill
\begin{minipage}{200pt}
\hangcaption{本手法の要約文の内容適切性の評価(1文から3文の入力時)} \label{表:人手内容適切性2}
\input{06table15.txt}
\end{minipage}
\end{table}

\noindent
ここで評価値1は評価者が考えた要約文の内容とシステムが出力
した要約文を比較して内容がほとんど一致するという評価であり,評価値2は評価
者が考える要約文と50\%以上75\%未満で内容が一致する評価である.表\ref{表:
人手内容適切性2}をみると,1文から3文の短いデータを入力した場合,本手法の
評価の平均値はいずれの評価者においても評価値2前後であり,人間が作成する要
約文の内容に半分以上は一致することが分かる.

続いて表\ref{表:人手内容適切性3}と表\ref{表:人手内容適切性4}に4文から10文
から構成されるデータを入力した場合の要約文を評価者が内容適切性を評価した
結果を示す.表\ref{表:人手内容適切性3}は従来手法の結果であり,表\ref{表:
人手内容適切性4}は本手法の結果である.

\begin{table}[b]
\begin{minipage}{200pt}
\hangcaption{従来手法の要約文の内容適切性の評価(4文から10文の入力時)}
\label{表:人手内容適切性3}
\input{06table16.txt}
\end{minipage}
\hfill
\begin{minipage}{200pt}
\hangcaption{本手法の要約文の内容適切性の評価(4文から10文の入力時)} \label{表:人手内容適切性4}
\input{06table17.txt}
\end{minipage}
\end{table}

表\ref{表:人手内容適切性}から表\ref{表:人手内容適切性4}の結果よりどの評価
者を見ても従来手法より本手法の方が平均値が低い,つまり評価が高かったこと
が分かる.

続いて表\ref{表:人手内容適切性5}において,評価者3人全員が良好である評価値1を
付与したのが入力データ中にどの程度存在するか,また評価者2人以上が評価値1を
付与したのがどの程度存在するか調査した結果を示す.

\setlength{\tabcolsep}{1zw}
\begin{table}[t]
\caption{評価者の人数と入力データに含まれる評価値1(良好)の件数}
\label{表:人手内容適切性5}
\input{06table18.txt}
\end{table}

表\ref{表:人手内容適切性5}より本手法において,評価者3人が共に評価値1を付
与した入力データの件数は1文から3文の入力データだと26件,4文から10文の入力
データだと21件という結果が得られた.またいずれの場合も本手法は従来手法の
結果よりも優位な結果が得られたことが分かる.評価者の過半数が最も良好であ
る評価値1を付与したものを正解とすると本手法の正解率は1文から3文で構成され
る短めの入力データ,4文から10文で構成される長めのデータ共に43\%を獲得でき
たこととなる.また内容の適切性を十分に満足するような要約文を得るタスクは
従来手法の結果である表\ref{表:人手内容適切性} や表\ref{表:人手内容適切性
3}を見ても分かるように難しいタスクであることが分かる.しかし表\ref{表:人
手内容適切性2}や表\ref{表:人手内容適切性4} をみると,評価者の多くは評価値
1と評価値2の合計が50\%を超えている.そのため,正解率が
43\%であると言っても不正解である57\%の多くは評価値1に近いことが言える.


\subsubsection{実験結果のまとめ}\label{節:実験結果のまとめ}

自動評価尺度であるBLEUやROUGE-$N$,また人手による可読性評価,内容適切性評
価のいずれにおいても従来手法より優位な結果が得られた.また結果より,短め
の記事を入力した場合と長めの記事を入力した場合で,比較対象と精度を比べる
と本手法の精度はほとんど変化しなかったことが分かった.そのため,長い記事
を入力とした場合でも良好な結果が得られる.

\subsection{要約文が含んでいる情報の量}\label{章:要約文が含んでいる情報の量}

本論文では入力を1記事として,それを1文に要約する手法を述べた.また目的の
1つとして複数の文を1文に圧縮することを挙げた.そのため本手法で出力した
要約文が何文の情報を含んでいるのか調査した.これには入力した記事と本手法
が出力した要約文の形態素で比較することにより調査を行った.表\ref{表:情報
量1}に1文から3文で構成されているデータを入力したときの結果を示す.

表\ref{表:情報量1}より,出力した要約文の60\%は2文以上の文を圧縮したもの
であるという結果が得られた.続いて,4文から10文で構成される長い記事を
入力したときに得られた要約文が何文を圧縮したものなのかを調査した.この結
果を表\ref{表:情報量2}に示す.

\begin{table}[t]
\begin{minipage}[t]{200pt}
\hangcaption{1文から3文のデータを入力したときに得られた要約文の圧縮文数}
\label{表:情報量1}
\input{06table19.txt}
\end{minipage}
\hfill
\begin{minipage}[t]{200pt}
\hangcaption{4文から10文のデータを入力したときに得られた要約文の圧縮文数}
\label{表:情報量2}
\input{06table20.txt}
\end{minipage}
\end{table}

表\ref{表:情報量2}より2文以上を圧縮した要約文は100件中80件であることが分
かる.これらの結果より,本論文で目的としていた複数の文を1文に圧縮した要
約文が作成できたことが分かる.

\subsection{用例の収集時期の影響}\label{章:}

本実験では1999年から2007年までに収集した用例を最大限使用
し,さらにオープンテストを行うため,入力するテストデータとしては1998年の
記事を使用した.本節では 用例と入力の時間的関係が精度にどの程度の影響を
与えるかについて,以下のような調査を行った.


収集した用例データベースを1999年から2007年で時系列に沿って10分割し,1998 
年のテストデータを用いてそれぞれのデータに対して類似用例文の獲得,文節の
対応付け,対応文節の組合せを行った.そして対応文節の組合せ時における動的
計画法のスコアを比較する.この動的計画法のスコアは入力した記事と獲得され
た類似用例文がどれ程似ているのかを表している.この結果を図\ref{図:用例デー
タベースと動的計画法のスコアの関係} に示す.

図\ref{図:用例データベースと動的計画法のスコアの関係}の結
果を見ると用例データベースの収集時期によらず結果がほぼ同一であることが分
かる.つまりテストデータとして用いた1998年の記事と近い時期の用例データベー
スを用いた場合でも,また離れた時期である2007年の用例データベースを用いて
も結果に変化はない.よって用例データベースの収集時期に関わらず,要約文の
作成が可能である.


\subsection{用例数と精度との関係}\label{章:用例数を変更}

類似用例文を検索する先の用例データベースに含まれる用例文数を変更したとき
の要約文の精度を調査した.用例文数を変更するときは無作為で用例文を設定
した件数になるまで獲得している.また要約文の評価は自動評価尺度である
ROUGE-1及びROUGE-2を用いている.この調査結果を図\ref{図:DB大きさと精度}に
示す.

\begin{figure}[t]
\begin{center}
\includegraphics{15-3ia6f10.eps}
\caption{用例データベースの収集時期と動的計画法のスコアの関係}
 \label{図:用例データベースと動的計画法のスコアの関係}
 \end{center}
\end{figure}

図\ref{図:DB大きさと精度}の横軸は対数表示になっている.また図中の直線は対
数近似を行った結果である.用例文の数が1000件に満たない場合は,ランダムで
選んできた用例文の種類によって大きく精度が変わってしまうため,これによっ
てROUGE-$N$スコアが大きく変化している.近似した回帰グラフ
を見ると右上がりのグラフが作成されている.また用例数を増加させることによっ
ていずれ精度は飽和することが考えられるが,現在約27000件の用例文を用いてい
る現段階では精度が飽和していない.よって今後,用例文を増加させることによっ
て更なる精度向上が期待できる.



\subsection{誤った要約文に対する考察}\label{節:誤った文に対する考察}

誤った要約文について例を挙げて示す.

\begin{figure}[t]
\begin{center}
\includegraphics{15-3ia6f11.eps}
\caption{用例データベースの規模と要約精度}
\label{図:DB大きさと精度}
\end{center}
\end{figure}

\begin{screen}
\exp{例:例1} 誤って出力した要約文の例
 
〈入力記事〉

{\setlength{\leftskip}{2zw}
米石油大手のコノコは二十九日、従業員九百七
十五人を\setnami\uc{削減し、}十--十二月期に五千万ドルの特別損失を\ul{計上
すると}発表した。九九年の投資額も石油探査・開発関連を中心に約五億ドル減ら
し、九八年比で二一%少ない十八億ドル規模にする。
\par}

〈類似用例文〉

  米AOLは9日、従業員の7%に相当する1300人を削減すると発表

〈出力結果〉

  米石油大手のコノコは二十九日、従業員九百七十五人を\ul{計上すると}発表
\end{screen}

例\ref{例:例1}における出力結果では下線の部分が不自然であ
り,可読性が良くない.文節の対応付けの際には類似用例文の文節「削減すると」
に対して,助詞の一致で「計上すると」が対応付けられ,さらに単語間類似度に
よる対応付けで「削除し、」と「計上すると」が対応付けられた.本手法の対応
文節の組合せでは助詞の一致と固有表現タグの一致,単語間類似度の信頼度を重
み付き和で表している.このため助詞の一致,単語間類似度で共に対応付けが行
われた「計上すると」という文節の方が優位であると判断された.このように組
合せを行う際に求めた唯一の解ではこのような例が存在するため,N-best解を出
力することや複数の類似用例文を使用して複数個の要約文を作成,そして最終
的に可読性の良さを連接確率などで測って要約文を選定することによってこの問
題は解決できるのではないかと考える.



\begin{screen}
\exp{例:例3} 誤って出力した要約文の例

〈入力記事〉

{\setlength{\leftskip}{2zw}
ハンバーガー大手のロッテリアは一月四日から十四日まで、\setnami\uc{全店で
骨なしフライドチキンチキンテンダーを}\setniju\uc{半額で}販売する。通常二
百四十円の五個入りを百二十円で、十個入りを二百四十円で販売する。サイドメ
ニューの人気商品の半額キャンペーンで、ハンバーガー類とのセット購入を促し
て客単価のアップを狙う。
\par}

〈類似用例文〉

  イトーヨーカ堂は24日から5日間限定で8200円の紳士・婦人スーツを販売。

〈出力結果〉

  ハンバーガー大手のロッテリアは一月四日から全店で\ul{十個入りを}販売。
\end{screen}

例\ref{例:例3}では類似用例文の「8200円の紳士・婦人スー
ツを」に対応する文節として「十個入りを」という文節を対応付けている.本手
法では連体修飾部を連結させて,対応付けを行う時には被修飾部のみを参照する
ことで類似した文節の対応付けを行っている.この例では連体修飾部の「8200円の」
という部分がその文の最も言いたいことであるが,出力結果にはそのよ
うな表現は一切現れていない.本手法では文節単位で対応付けを行い,連体修飾
部の内容は比較していないためこのような要約文を作ってしまった.文節を用い
たのは形態素単位での対応付けでは対応が取れない例が多く見られたためである
が,逆に文節単位ではこのような連体修飾部の対応付けまでができない.よって
今後の展望として形態素で対応付けする場合と,そうではなく文節で対応付けを
行う場合どちらも使用して要約文を作成することが挙げられる.

\begin{screen}
\exp{例:例4} 誤って出力した要約文の例

〈入力記事〉\par
  太平工業第一回無担保債一億二百万円。二百万円の利益。

〈類似用例文〉\par
  米穀物大手ADMの10--12月は純利益が20%増の4億4100万ドル。

〈出力結果〉\par
  二百万円の利益が太平工業第一回無担保債一億二百万円。
\end{screen}
この例では入力記事にもともと助詞があまり存在しない.その
ため助詞に一致による文節の対応付けは行われなかった.また入力記事の1文目は
連体修飾部を連結した場合,1つのまとまりであると判断された.そのため類似用
例文の「純利益が」という文節に対して入力記事の2文目「利益」が対応付けられ,
さらに類似用例文の文節「$\cdots$ドル」に対して1文目の「$\cdots$円」が対応
付けられた.これらを理由に対応付けが行われたため,出力結果をみると可読性
も内容適切性も悪い結果となったと考えられる.





\section{結論}

重要度の設定を必要とせず,さらに複数文の情報を圧縮した要約文を作成するこ
とを目標とした要約手法を提案した.この要約手法は文書を入力として受け取り,
その文書内の複数の文から文節を抽出,また組み合わせることで複数文の情報を
含む要約文を作成する.ここで,単に文節を抽出して組み合わせるだけでは,日本語とし
て正しくかつ適切な内容を含む要約文を作ることはできない.これに対して,
本論文では過去に人間が作成した要約文(要約事例)を用例文として用い,その用
例文を模倣して文節を抽出,組み合わせることで,日本語として連接が良く,さ
らに適切な内容を含む要約文の作成を可能にした.

評価実験ではBLEU,ROUGE-Nによる自動評価と人手による評価の2つから評価を行
い,また従来法の1つを比較手法として取り上げた.評価結果では自動評価,人手
による評価結果ともに,従来手法に比べ本手法の方が良好な結果が得られたこと
が分かった.さらに2つの評価方法各々の結果からも本手法の有効性が確認できた.

\def\labelenumi{}
\section*{使用したツール及び言語資源}
\begin{enumerate}
\def\newblock{}
\item\label{ツール:cabocha} 構文解析器 {CaboCha},  {Ver}.0.53,
     \newblock 奈良先端科学技術大学院大学 松本研究室,\\
     \newblock http://chasen.org/\~{}taku/software/cabocha/
\item\label{ツール:chasen} 形態素解析器 {ChaSen},  {Ver}.2.3.3,
     \newblock 奈良先端科学技術大学院大学 松本研究室,\\
     \newblock http://chasen.naist.jp/hiki/ChaSen
\item\label{言語資源:nikkei-goo} 日経ニュースメール,NIKKEI-goo, \\
     \newblock http://nikkeimail.goo.ne.jp/
\item\label{言語資源:日経}日本経済新聞全記事データベース
     1990--2004年度版,
     \newblock 日本経済新聞社
\end{enumerate}


\bibliographystyle{jnlpbbl_1.3}
\begin{thebibliography}{}

\bibitem[\protect\BCAY{{Daume III} \BBA\ Marcu}{{Daume III} \BBA\
  Marcu}{2002}]{Daume:2002}
{Daume III}, H.\BBACOMMA\ \BBA\ Marcu, D. \BBOP 2002\BBCP.
\newblock \BBOQ A Noisy-Channel Model for Document Compression\BBCQ\
\newblock In {\Bem Proceedings of the 40th Annual Meeting of the Association
  for Computational Linguistics}, \mbox{\BPGS\ 449--456}.

\bibitem[\protect\BCAY{Hori}{Hori}{2002}]{hori:2002th}
Hori, C. \BBOP 2002\BBCP.
\newblock {\Bem A Study on Statistical Methods for Automatic Speech
  Summarization}.
\newblock Ph.D.\ thesis, Tokyo Institute of Technology.

\bibitem[\protect\BCAY{Hori, Furui, Malkin, Yu, \BBA\ Waibel}{Hori
  et~al.}{2003}]{Hori:2003}
Hori, C., Furui, S., Malkin, R., Yu, H., \BBA\ Waibel, A. \BBOP 2003\BBCP.
\newblock \BBOQ A Statistical Approach to Automatic Speech Summarization\BBCQ\
\newblock {\Bem Artificial Intelligence}, {\Bbf 2}, \mbox{\BPGS\ 128--139}.

\bibitem[\protect\BCAY{Imamura}{Imamura}{2004}]{imamura:2004}
Imamura, K. \BBOP 2004\BBCP.
\newblock {\Bem Automatic Construction of Translation Knowledge for
  Corpus-based Machine Translation}.
\newblock Ph.D.\ thesis, Nara Institute of Science and Technology.

\bibitem[\protect\BCAY{Jing}{Jing}{2000}]{Jing:2000}
Jing, H. \BBOP 2000\BBCP.
\newblock \BBOQ Sentence Reduction for Automatic Text Summarization\BBCQ\
\newblock In {\Bem Proceedings of The 6th Conference on Applied Natural
  Language Processing}, \mbox{\BPGS\ 310--315}.

\bibitem[\protect\BCAY{Knight \BBA\ Marcu}{Knight \BBA\
  Marcu}{2002}]{Knight:2002}
Knight, K.\BBACOMMA\ \BBA\ Marcu, D. \BBOP 2002\BBCP.
\newblock \BBOQ Summarization Beyond Sentence Extraction: A Probabilistic
  Approach to Sentence Compression\BBCQ\
\newblock {\Bem Artificial Intelligence}, {\Bbf 139}  (1), \mbox{\BPGS\
  91--107}.

\bibitem[\protect\BCAY{Kurohashi, Nakazawa, Alexis, \BBA\ Kawahara}{Kurohashi
  et~al.}{2005}]{Kurohashi:2005}
Kurohashi, S., Nakazawa, T., Alexis, K., \BBA\ Kawahara, D. \BBOP 2005\BBCP.
\newblock \BBOQ Example-based Machine Translation Pursuing Fully Structural
  NLP\BBCQ\
\newblock In {\Bem Proceedings of International Workshop on Spoken Language
  Translation 2005 (IWSLT2005)}, \mbox{\BPGS\ 207--212}.

\bibitem[\protect\BCAY{Lin}{Lin}{2004}]{ROUGE}
Lin, C.-Y. \BBOP 2004\BBCP.
\newblock \BBOQ Looking for a Good Metrics: ROUGE and its Evaluation\BBCQ\
\newblock In {\Bem Proceedings of the 4th NTCIR Workshops}, \mbox{\BPGS\ 1--8}.

\bibitem[\protect\BCAY{Lin}{Lin}{1998}]{Lin:1998}
Lin, D. \BBOP 1998\BBCP.
\newblock \BBOQ Automatic Retrieval and Clustering of Similar Words\BBCQ\
\newblock In {\Bem The 36th Annual Meeting of the Association for Computational
  Linguistics and the 17th International Conference on Computational
  Linguistics (COLING-ACL)}, \mbox{\BPGS\ 768--774}.

\bibitem[\protect\BCAY{Nagao}{Nagao}{1984}]{Nagao:1984}
Nagao, M. \BBOP 1984\BBCP.
\newblock \BBOQ A Framework for a Mechanical Translation between Japanese and
  English by Analogy Principle\BBCQ\
\newblock In {\Bem Artificial and Human Intelligence}, \mbox{\BPGS\ 173--180}.

\bibitem[\protect\BCAY{永田}{永田}{1999}]{永田:1999論文}
永田昌明 \BBOP 1999\BBCP.
\newblock \JBOQ 統計的言語モデルとN-best探索を用いた日本語形態素解析法\JBCQ\
\newblock \Jem{情報処理学会論文誌}, {\Bbf 40}  (9), \mbox{\BPGS\ 3420--3431}.

\bibitem[\protect\BCAY{Nguyen, Horiguchi, Shimazu, \BBA\ Bao}{Nguyen
  et~al.}{2004}]{Le:2004}
Nguyen, M.~L., Horiguchi, S., Shimazu, A., \BBA\ Bao, H. \BBOP 2004\BBCP.
\newblock \BBOQ Example-based Sentence Reduction Using Hidden Markov
  Model\BBCQ\
\newblock {\Bem ACM Transactions on Asian Language Information Processing},
  {\Bbf 3}  (2), \mbox{\BPGS\ 146--158}.

\bibitem[\protect\BCAY{小黒\JBA 尾関\JBA 張\JBA 高木}{小黒\Jetal
  }{2001}]{oguro:1991論文}
小黒玲\JBA 尾関和彦\JBA 張玉潔\JBA 高木一幸 \BBOP 2001\BBCP.
\newblock \JBOQ
  文節重要度と係り受け整合度に基づく日本語文簡約アルゴリズム\JBCQ\
\newblock \Jem{自然言語処理}, {\Bbf 8}  (3), \mbox{\BPGS\ 3--18}.

\bibitem[\protect\BCAY{大竹}{大竹}{2003}]{大竹清敬:2003NLP}
大竹清敬 \BBOP 2003\BBCP.
\newblock \JBOQ 用例に基づく換言:中日旅行会話翻訳への適用\JBCQ\
\newblock \Jem{言語処理学会年次大会発表論文集}, \mbox{\BPGS\ 345--348}.

\bibitem[\protect\BCAY{Papineni, Roukos, Ward, \BBA\ Zhu}{Papineni
  et~al.}{2002}]{BLEU}
Papineni, K., Roukos, S., Ward, T., \BBA\ Zhu, W.-J. \BBOP 2002\BBCP.
\newblock \BBOQ BLEU : a Method for Automatic Evaluation of Machine
  Translation\BBCQ\
\newblock In {\Bem Proceedings of the Annual Meeting of the Association for
  Computational Linguistics (ACL'02)}, \mbox{\BPGS\ 311--318}.

\bibitem[\protect\BCAY{Sato}{Sato}{1995}]{Sato:1995}
Sato, S. \BBOP 1995\BBCP.
\newblock \BBOQ MBT2: A Method for Combining Fragments of Examples in
  Example-based Translation\BBCQ\
\newblock {\Bem Artificial Intelligence}, {\Bbf 75}  (1), \mbox{\BPGS\ 31--49}.

\bibitem[\protect\BCAY{佐藤\JBA 長尾}{佐藤\JBA 長尾}{1989}]{佐藤理史:1989NL}
佐藤理史\JBA 長尾真 \BBOP 1989\BBCP.
\newblock \JBOQ 実例に基づいた翻訳\JBCQ\
\newblock \Jem{情報処理学会研究報告}, {\Bbf NL70-9}, \mbox{\BPGS\ 1--8}.

\bibitem[\protect\BCAY{寺村}{寺村}{1993}]{寺村}
寺村秀夫 \BBOP 1993\BBCP.
\newblock \Jem{日本語のシンタクスと意味 I, 第1章}.
\newblock くろしお出版.

\bibitem[\protect\BCAY{Vandeghinste \BBA\ {Kim Sang}}{Vandeghinste \BBA\ {Kim
  Sang}}{2004}]{Vandeghinste:2004}
Vandeghinste, V.\BBACOMMA\ \BBA\ {Kim Sang}, E.~T. \BBOP 2004\BBCP.
\newblock \BBOQ Using a Parallel Transcript/Subtitle Corpus for Sentence
  Compression\BBCQ\
\newblock In {\Bem Proceedings of the 4th International Conference on Language
  Resources and Evaluation}, \mbox{\BPGS\ 231--234}.

\bibitem[\protect\BCAY{Witbrock \BBA\ Mittal}{Witbrock \BBA\
  Mittal}{1999}]{Witbrock:1999}
Witbrock, M.\BBACOMMA\ \BBA\ Mittal, V. \BBOP 1999\BBCP.
\newblock \BBOQ Ultra-Summarization: A Statistical Approach to Generating
  Highly Condensed Non-Extractive Summaries\BBCQ\
\newblock In {\Bem Proceedings of Research and Development in Information
  Retrieval}, \mbox{\BPGS\ 315--316}.

\end{thebibliography}



\section*{付録 出力された要約文の例}

\vspace{-0.5\baselineskip}
以下に本手法で得られた要約文の出力例を示す.

\newenvironment{InL}{}{}
\small
\vspace{0.5\baselineskip}
\hrule
\vspace{0.5\baselineskip}

\noindent {\bf 入力記事}

\begin{InL}
三十日午後二時十分ごろ、剣淵町の国道40号で、旭川市東旭川町下兵村二二八、農業南部正さんの乗用車と、旭川市流通団地二条二ノ四三、運転手原政運さんのトラックが正面衝突した。乗用車の四人のうち、南部さんと妻の喜美子さん、士別市東山町三〇二、無職池沢一郎さんの三人が頭を打つなどして死亡、旭川市東旭川北一条四ノ一ノ二八、無職真岩高子さんも左足の骨を折る重傷を負った。原さんにけがはなかった。
\end{InL}

\noindent {\bf 類似用例文}

\begin{InL}
イラク中部で28日深夜、油送管が爆発し74人が死亡。
\end{InL}

\noindent {\bf 出力した要約文}

\begin{InL}
剣淵町の国道40号で三十日午後二時十分ごろ、旭川市流通団地二条二ノ四三、運転手原政運さんのトラックが正面衝突し南部さんと妻の喜美子さん、士別市東山町三〇二、無職池沢一郎さんの三人が死亡
\end{InL}

\vspace{0.5\baselineskip}
\hrule
\vspace{0.5\baselineskip}

\noindent {\bf 入力記事}

\begin{InL}
衛星携帯電話イリジウムの日本法人、日本イリジウムは一月一日から衛星携帯電話の本サービスを開始する。これまでは各国での通話品質の向上を待つため、試験サービスとして実施していた。一月以降は携帯電話利用で月額五十ドル、ポケットベルの併用で月額八十ドルの基本料を徴収する。
\end{InL}

\noindent {\bf 類似用例文}

\begin{InL}
イランは、次期国会議員選挙を08年3月14日に実施。
\end{InL}

\noindent {\bf 出力した要約文}

\begin{InL}
衛星携帯電話イリジウムの日本法人、日本イリジウムは、衛星携帯電話の本サービスを一月一日に開始
\end{InL}

\vspace{0.5\baselineskip}
\hrule
\vspace{0.5\baselineskip}

\noindent {\bf 入力記事}

\begin{InL}
大阪府警柏原署は二十九日、親子げんかの末に父親を殴って死なせたとして、柏原市内の中学三年生の少年を傷害致死の疑いで逮捕した。調べによると、少年は二十八日午後十時五十分ごろ、自宅で会社員の父親に日ごろの生活態度を注意されたことから口論となり、取っ組み合いの末に父親の胸を強く殴り死亡させた疑い。父親は殴られた直後にめまいがすると言って両手をけいれんさせたため、家族が一一九番通報、病院に運ばれたが、二十九日朝、胸部打撲による大動脈の裂傷が原因で死亡した。
\end{InL}

\noindent {\bf 類似用例文}

\begin{InL}
警視庁は14日、元藍沢証券社員を業務上横領の疑いで逮捕した。
\end{InL}

\noindent {\bf 出力した要約文}

\begin{InL}
大阪府警柏原署は二十九日、柏原市内の中学三年生の少年を傷害致死の疑いで逮捕
\end{InL}

\vspace{0.5\baselineskip}
\hrule
\vspace{0.5\baselineskip}

\noindent {\bf 入力記事}

\begin{InL}
九九年三月に大阪商工会議所の次期会頭に就任する田代和副会頭は二十八日、鳥井信一郎・サントリー社長に副会頭への就任を要請した。田代氏の昇格で空席となる分をまず決める狙い。残りの六副会頭のうち半数程度が十一月の改選期に入れ替わると見られており、鳥井氏も同時に就任する可能性が高い。
\end{InL}

\noindent {\bf 類似用例文}

\begin{InL}
米GMは著名投資家カーコリアン氏側近のジェリー・ヨーク氏に社外取締役就任を要請へ。
\end{InL}

\noindent {\bf 出力した要約文}

\begin{InL}
九九年三月に大阪商工会議所の次期会頭に就任する田代和副会頭は鳥井信一郎・サントリー社長に副会頭への就任を要請
\end{InL}

\vspace{0.5\baselineskip}
\hrule
\vspace{0.5\baselineskip}

\noindent {\bf 入力記事}

\begin{InL}
栃木県がまとめた十月の県鉱工業生産指数は九三・三で前月比〇・九%、前年同月比八・四%それぞれ低下した。業種別では金属製品、鉄鋼、精密機械などが低下し、木材、化学工業などは上昇した。
\end{InL}

\noindent {\bf 類似用例文}

\begin{InL}
ゴールデンウイーク期間中の格安国際航空券の価格が上昇、前年比1--7割高に。
\end{InL}

\noindent {\bf 出力した要約文}

\begin{InL}
栃木県がまとめた十月の県鉱工業生産指数が低下、前月比〇・九%に
\end{InL}

\vspace{0.5\baselineskip}
\hrule
\vspace{0.5\baselineskip}

\noindent {\bf 入力記事}

\begin{InL}
北海道中小企業家同友会旭川支部は一月五日、旭川市内のホテルで合同就職説明会を開催する。同友会加盟企業十五社が参加、Uターン就職希望者や大学、短大などの卒業予定者を対象に直接面談する。入退場自由で無料。
\end{InL}

\noindent {\bf 類似用例文}

\begin{InL}
自民党は16日、都内のホテルで党大会を開催。
\end{InL}

\noindent {\bf 出力した要約文}

\begin{InL}
北海道中小企業家同友会旭川支部は一月五日、旭川市内のホテルで合同就職説明会を開催
\end{InL}

\vspace{0.5\baselineskip}
\hrule
\vspace{0.5\baselineskip}

\noindent {\bf 入力記事}

\begin{InL}
日本トランスオーシャン航空は来年四月、客室乗務員の制服を十一年ぶりに一新する。新しい制服は濃紺のジャケットに赤、青、グリーンの三種類のブラウスなど、オーソドックスな色合いとデザインで落ち着いた雰囲気を演出。帽子や胸章などを廃止して、従来の制服より約三〇%コストを削減した。
\end{InL}

\noindent {\bf 類似用例文}

\begin{InL}
仏エールフランス航空は5日から、客室乗務員などの制服を18年ぶりに一新する。
\end{InL}

\noindent {\bf 出力した要約文}

\begin{InL}
日本トランスオーシャン航空は来年四月から、客室乗務員の制服を十一年ぶりに一新
\end{InL}

\vspace{0.5\baselineskip}
\hrule
\vspace{0.5\baselineskip}

\noindent {\bf 入力記事}

\begin{InL}
宮崎銀行は二十八日、住宅金融公庫の住宅ローン返済問題の相談コーナーを本店営業部内に設置した。政府が住宅公庫ローン返済者に対する救済策を打ち出したため、専門的にアドバイスする統括相談員を配置した。設置は来年三月末まで。
\end{InL}

\noindent {\bf 類似用例文}

\begin{InL}
政府は7月にもバイオテクノロジー戦略会議を設置する。
\end{InL}

\noindent {\bf 出力した要約文}

\begin{InL}
政府は二十八日にも住宅金融公庫の住宅ローン返済問題の相談コーナーを設置
\end{InL}

\vspace{0.5\baselineskip}
\hrule
\vspace{0.5\baselineskip}

なお,最後の例では要約結果が「宮崎銀行は...」とはならず,「政府は...」
と出力された.これは,類似用例文の「政府は」に対して,入力記事1文目の
「宮崎銀行は」よりも入力記事2文目の「政府が」のほうが,内容語と助詞の
類似性の点からより類似していると判断されたためである.

\normalsize


\begin{biography}
\bioauthor{山本 和英}{
1996年3月豊橋技術科学大学大学院工学研究科博士後期課程システム情報工学専攻修了.
博士(工学).
1996年〜2005年(株)国際電気通信基礎技術研究所(ATR)研究員(2002年〜2005年客員研究員).
1998年中国科学院自動化研究所国外訪問学者.
2002年より長岡技術科学大学電気系,現在准教授.
言語表現加工技術(要約,換言,翻訳),主観表現処理(評判,意見,感情)などに興味がある.
言語処理学会,人工知能学会,情報処理学会,各会員.
e-mail: yamamoto@fw.ipsj.or.jp
}
\bioauthor{牧野  恵}{
2006年3月長岡技術科学大学電気電子情報工学課程卒業.
2008年3月同大学大学院工学研究科修士課程電気電子情報工学専攻修了.
修士(工学).
在学中は自動要約の研究に従事.
言語処理学会学生会員. 
e-mail: makino@nlp.nagaokaut.ac.jp
}

\end{biography}


\biodate


\end{document}
