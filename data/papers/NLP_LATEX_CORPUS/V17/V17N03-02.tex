    \documentclass[english]{jnlp_1.4}

\usepackage{jnlpbbl_1.2}
\usepackage[dvips]{graphicx}
\usepackage{hangcaption_jnlp}


\Volume{17}
\Number{3}
\Month{April}
\Year{2010}

\received{2009}{5}{15}
\revised{2009}{12}{11}
\accepted{2009}{12}{16}

\setcounter{page}{25}

\etitle{Collecting Object-attribute Noun Pairs and Constructing 
	Concept Graphs for the Argument of Adjectives from Japanese N1-Adj-N2 
	Constructions}
\eauthor{Takeshi Abekawa\affiref{NII} \and Manabu Okumura\affiref{Precision}} 
\eabstract{
In this paper, we propose a method for exploring the Japanese
construction N1-Adj-N2, which often establishes a relationship between
an object (N2), an attribute (N1), and an evaluation of that attribute
(Adj). As this construction connects two nouns, our method involves
constructing a graph of the noun relations, which can be considered 
as representing selectional restrictions for the argument of a target 
adjective. The exploration of N1-Adj-N2 constructions is useful for
opinion mining, lexicographical analysis of adjectives, and writing
aid, among others.
}
\ekeywords{Relative Clause, Adjective, Concept graph, Object attribute pair}

\headauthor{Abekawa and Okumura}
\headtitle{Collecting Object-attribute Noun Pairs and Constructing Concept Graphs}

\affilabel{NII}{}{National Institute of Informatics}
\affilabel{Precision}{}{Precision and Intelligence Laboratory, Tokyo Institute of Technology}


\begin{document}

\maketitle

\newcommand{\argmax}{}

\section{Introduction}

``おいしい イタリア'' (``Delicious Italy''). This phrase was used in a
promotional brochure put out by a travel agency in Japan. Though the
adjective ``おいしい'' (``delicious'') modifies the noun ``イタリア''
(``Italy'') here, what is delicious is not Italy itself but something associated
with Italy. The brochure is designed to make an impression by prompting
readers to imagine what this ``something'' might be. In Japanese there
are many phrases that indicate a relationship between ``something'' and
a target noun, such as ``A no B (B of A)''. The phrase ``something {\it ga}
adjective noun'' is also widely used.
In this phrase, ``something {\it ga} adjective'' means ``something is
adjective'' in English and ``{\it ga}'' is a case marker which
indicates a subject in Japanese.  In the example above, if we search
for the phrase ``* が おいしい イタリア'' (``* {\it ga oishii Itaria}'';
``Italy, whose * is tasty'') on the Internet, we can identify what the
``something'' might be, for example nouns like ``wine'', ``pasta'' and
``pizza''.

This paper proposes a method for exploring the Japanese construction
N1-{\it ga}-Adj-N2, which often establishes a relationship between an object
(N2), an attribute (N1), and an evaluation of that attribute (Adj: adjective). For
instance, in ``鼻 が 短い 象'' (``nose {\it ga}''-``short''-``elephant'': ``an
elephant with a short nose'') and ``クレマカタラナ が おいしい レストラン''
(``crema catalana {\it ga}''-``delicious''-``restaurant''; ``a restaurant that
serves delicious crema catalana''), the nouns in N1 position (``nose''
and ``crema catalana'') serve as the attributes of the nouns in N2
position (``elephant'' and ``restaurant''), while the adjectives
(``short'' and ``tasty'') are evaluations of the attributes. 

When noun A in N2 position is acquired from noun B in N1 position by
using the phrase ``N1 {\it ga} Adj N2'', we can acquire noun C by using
another phrase, ``N1 {\it ga} Adj B'', in which C takes N1 position.
From this, we can create the chain of nouns
A${\rightarrow}$B${\rightarrow}$C.  For instance, from ``スープ(soup) が
({\it ga}) おいしい(delicious) ラーメン(noodles)'' and ``塩味(salty taste)
 が おいしい(delicious) スープ(soup)'', we can obtain the chain ``ラーメン''
(``noodle'') ${\rightarrow}$ スープ (``soup'') ${\rightarrow}$ ``塩味''
(``salty taste'').  By connecting various chains targeting one
adjective, a graph structure can be constructed. This graph structure
can represent a selectional restriction for the argument position of a
target adjective. Thus it will be useful for the lexicographical
analysis of adjectives.

The analysis proposed in this paper is also useful for opinion mining.
There is currently a great deal of interest in finding ways to extract
individuals' opinions on products, services and other objects from the
wealth of consumer-generated media on the Internet, for use in a variety
of activities such as marketing and risk management, sociological
analysis, and the post-processing of IR results, among others
\cite{suzuki:2006,kobayashi:2007}. The main
components of this information can be defined as an object, an
attribute, and an evaluation of that attribute (in other words, the
individual's opinion). In order to mine the Internet for information on
people's opinions, it is first necessary to define objects and
attributes. As manual construction of this information is costly, a
variety of automatic methods have been proposed by researchers. Some use
fixed patterns \cite{matthew:1999,abdulrahman:2004}, some delve into the
structure of tables \cite{yoshida:2004}.

Our main contributions in this paper are following three points. First
one is automatically collecting attribute nouns for specific
object. Our method helps the producer of products/services to
comprehend what they are their main points as viewed by the
consumers. The second contribution is an accuracy improvement of
dependency parsing. Though an applicable structure is limited to the
decision of N1 dependency in N1-Adj-N2 form, our method is more
accurate than existing parsers which can parse all structures.  The
third contribution is for lexicographical analysis of adjectives.  The
concept graph for an adjective represents selectional restriction for
adjectives which have two arguments and facilitate the semantic
analysis as well as sentence generation.


In section 2, we discuss the nature of the N1-Adj-N2
construction. Section 3 explains the way we collected object-attribute
noun pairs from Web texts, and introduces a method for automatically
collecting object-attribute-evaluation triplets using machine
learning. We then describe the construction of an adjective case frame
from object-attribute noun pairs in section 4. Section 5 concludes the
paper.


\section{The nature of the N1-Adj-N2 construction}

As briefly explained in the introduction, the N1-Adj-N2 construction in
Japanese frequently takes N1 as the attribute of the object N2, with Adj
expressing an evaluation of the attribute specified by N1
\cite{mikami:60}. This construction has the following characteristics:
\begin{itemize}
\item Because most adjectives take only one obligatory argument or case
slot, the adjective whose case slot is filled by N1 (which is frequently
the case in the N1-Adj-N2 construction) cannot establish a case relation
with N2. N2 tends to establish an object-attribute relation with N1 as a
consequence.
\item The construction is relatively stable and simple. Though it may
undergo grammatical transformations, it is widely held that the degree
of transformation this construction undergoes is relatively less than
that undergone by a complete clausal construction. This characteristic
results in part from the fact that this construction essentially occurs
as an embedded part of a sentence.
\end{itemize}

From these characteristics, we can reasonably expect that:
\begin{enumerate}
\item the accuracy of the object-attribute-evaluation triplet obtained
by exploring this construction will be high;
\item in applying machine learning methods, the construction of training 
data will be less cumbersome than exploring most other constructions, and 
automatic processing will be rather straightforward;
\item the range of object-attribute-evaluation relations which can potentially
be collected will be wider than those constructed manually,
or by focusing on tables, or another grammatical construction.
\end{enumerate}
The third point needs further explanation. A typical occurrence 
of the N1-Adj-N2 construction is:
\begin{enumerate}
\item[(1)] 夜景が綺麗なレストランで彼女と食事をした。
(I dined with my partner in a restaurant with a beautiful night view.)
\end{enumerate}
where the N1-Adj-N2 construction ``夜景 が 綺麗な レストラン'' (a restaurant
with a beautiful night view) constitutes a noun phrase in the subject 
position, and is not the focal topic of the sentence (the focal topic
is ``I dined with my partner''). Compare this with 
\begin{enumerate}
\item[(2)] そのレストランは夜景が綺麗だった。
(The restaurant has a beautiful night view.)
\end{enumerate}
in which the focal topic is the ``beautiful night view''. Although we do
not have rigid statistical information, it can be expected that exploiting
the N1-Adj-N2 construction will capture object-attribute-evaluation relations
which are mentioned in passing and without being the focal topic of 
Web texts. The potential range should therefore be wider in the sense
that we shall be able to collect the type of relations that are not normally 
represented in table format or manifested as an explicit proposition in
the form of a main sentence, in other words, more than just the relations 
that consciously focused in the human mind.


\section{Collecting object-attribute pairs}

\begin{figure}[b]
\begin{center}
\includegraphics{17-3ia2f1.eps}
\end{center}
  \caption{Framework of our proposed method.}
  \label{fig:framework}
\end{figure}

We collected attribute nouns from a Web corpus to construct
the training data. Using the training data, a classifier was made by
using the SVM that detects object-attribute relations from N1-Adj-N2
constructions. The framework of our proposed method is shown in Figure \ref{fig:framework}.


\subsection{Phrase retrieval}

We used a search engine to collect N1-Adj-N2 constructions from the
Web. In order to do so, we first collected the top 200 ``{\it
i}''-inflectional adjectives and top 200 ``{\it na}''-inflectional
adjectives from a newspaper corpus\footnote{We counted frequency of each
adjective from 10 years' worth of newspaper articles (10M
sentences) and used top 400 adjectives because of restriction on
an API of search engine.}. For the object nouns, we collected 633 nouns
from a Web site that lists the prices of commercial
products\footnote{http://kakaku.com/}.  For each of these 400
adjectives, we retrieved Web snippets by using ``が({\it ga}) Adj
Object-Noun'' where Adj stands for the adjectives\footnote{We can
expect the existence of nouns in front of this search query because ``が
({\it ga})'' is a case maker.}. The Yahoo!  Japan search engine was
used for the retrieval of Web snippets \cite{yahoo:2008}.  The total
number of snippets obtained at this stage was about 1.3 million (the
maximum number of hits allowed by the search engine was 1,000 for each
search).


\vspace{0.3\baselineskip}

\subsection{Filtering}

The data obtained by phrase retrieval contained some irrelevant cases.
We removed the following:
\begin{itemize}
\item As the Yahoo! Japan phrasal search ignores punctuation, 
cases such as ``...がおいしい。レストランでは...'' (... is good. The
restaurant ...) were retrieved for the search phrase 
\mbox{``が} おいしい レストラン''. We removed these cases systematically using
a simple filtering program.
\item The patterns ``N1 {\it ga} Adj N2 {\it no} N3'' (N3 of N2 whose N1 is Adj) are
ambiguous in that the object of the attribute N1 can be N2 or N3. We
removed these cases in order to reduce the burden of manual checking when 
constructing the training corpus and to achieve higher precision.
\end{itemize}
In addition, we removed identical sentences. As a result, we obtained in total
about 220,000 sentences from the 1.3 million snippets.

\vspace{0.3\baselineskip}

\subsection{Manual classification of the data}

After filtering the data, we manually checked some instances whether
items taking N1 position constitute attribute nouns or not. In the
process, we paid attention to noun phrases in the position N1. For
instance, in the following examples that take the form `N1$_{a}$ {\it
no} N1$_{b}$'-Adj-N2,
\begin{enumerate}
\item[1)] 大豆の甘味 が    おいしい  ドリンク (``a drink whose sweet flavour of beans is tasty'')
\item[2)] 振り子の動き が  かわいい  時計 (``a clock whose pendulum movement is cute'')
\end{enumerate}
it is enough to see the nouns immediately preceding the adjective, i.e.
``甘味'' (``sweetness'') and ``動き'' (``movement''), to understand the meaning. But 
``大豆'' (``beans'') and ``振り子'' (``pendulum'') is more concrete and, in a 
sense, more relevant as an attribute of the object nouns. On 
the other hand, in the following examples that also take the form 
`N1a の N1b'-Adj-N2,
\begin{enumerate}
\item[3)] 青磁の色 が 魅力の コーヒーカップ (``a coffee cup whose blue color is attractive'')
\item[4)] 北アルプスの景観 が 自慢の キャンプ場 (``a camping ground that boasts views of the Northern Alps'')
\item[5)] 円筒形のボディ が おもしろい パソコン (``a personal computer
whose cylindrical shape of the body is interesting'')
\end{enumerate}
N1$_{b}$, not N1$_{a}$, of the attribute noun phrase `N1$_{a}$ no
N1$_{b}$' bears the actual content of the attribute: ``色'' (``color'')
rather than ``青磁'' (``blue glaze'') in 3), ``景観'' (``landscape'')
rather than ``北アルプス'' (``Northern Alps'') in 4), and ``ボディ''
(``body'') rather than ``円筒形'' (``cylindrical shape''). Therefore, we
regard all the nouns in the noun phrases in N1 position as attribute
noun candidates. When compound nouns occur in N1 position, we apply the
same treatment, because any of the constituents in compound nouns can be
an attribute. For instance, in
\begin{enumerate}
\item[6)] 利用 頻度 が    高い    財布 (``a wallet with a high frequency of use'')
\item[7)] 資産 価値 が    高い    自動車 (``a car with a high property value'')
\end{enumerate}
``利用'' (``use'') and ``価値'' (``value'') were taken as the attribute nouns 
in 6) and 7), respectively.

We randomly selected 2,668 sentences and tagged them as training data.
When the noun phrase in N1 position does not contain an attribute noun, 0 
was assigned as a tag. When the noun phrase in N1 position contains an 
attribute noun, the number that shows the distance from the adjective (Adj) 
was assigned. Table \ref{tab:tag} shows examples of tags. Table
\ref{tab:number_of_tag}
 shows the quantitative
distribution of tags.

\begin{table}[b]
  \caption{Examples of tagged data.}
  \label{tab:tag}
\input{06table01.txt}
\end{table}

\begin{table}[t]
  \caption{Distribution of tags.}
  \label{tab:number_of_tag}
\input{06table02.txt}
\end{table}


\subsection{Classification method}

\subsubsection{Training and application}

Based on training data constructed in the manner described above, we 
constructed the classifier by using SVM. This classifier identifies 
whether the noun 
contained in position N1 constitutes the attribute of noun N2 or not,
for the construction N1-Adj-N2. The training and application were
carried out as follows.

In the training stage, when there was only one candidate noun in position
N1, it is treated as positive when it was the attribute noun, and negative
when it was not. When there were multiple candidates, we treated each candidate
separately, and regarded those that were attribute nouns as positive and others
as negative. If there were no attribute nouns among multiple candidates,
we used them as negative data.

In the application or classification stage, when there were multiple candidates, 
each candidate was judged separately by the classifier and the distance from 
the hyperplane was calculated. If there were more than one candidate with 
positive values, the one that was situated furthest from the hyperplane 
was chosen. When all the candidates had negative values, they were all 
treated as non-attribute nouns.
We used TinySVM\footnote{http://chasen.org/\~{}taku/software/TinySVM/} with 
linear kernel. 


\subsubsection{Features}

We extracted features for the machine learning method.  After applying
the morphological analysis, the chunk (``{\it bunsetsu}''\footnote{ {\it
Bunsetsu} consists of one or more content words that may be followed by
any number of function words.})  that include ``N1 {\it ga} Adj N2'' and
the part preceding it were extracted.  We used
JUMAN\cite{kurohasi:98:ae} and KNP\cite{kurohasi:94:b} for morphological
analysis and part extraction, respectively. Features used in the
classifier are listed in Table \ref{tab:features}. The definition of
heads and word forms in Table \ref{tab:features} follows
\cite{utimoto:2000e}. As for the concept represented by nouns, we used
the concept number assigned by Japanese thesaurus
\cite{ikehara:1997e}\footnote{For proper nouns, we used the concept
number at depth 2 from the root, and for general nouns, we used the
concept number at depth 3 from the root.}.

\begin{table}[b]
  \caption{Features.}
  \label{tab:features}
\input{06table03.txt}
\end{table}



\subsection{Preliminary experiment}

In order to observe the performance of the classifier thus constructed, 
we carried out a preliminary experiment using the internal data, applying
10-fold cross validation to the tagged data. For evaluation, we calculated 
the accuracy of binary classification of each candidate noun and the
accuracy of classification of N1, i.e. whether N1 contains an attribute
noun or not. Table \ref{tab:svm_result} shows the result. We also
conducted the experiment with existing Japanese dependency parser KNP
and Cabocha\cite{kudo:2002e}. In this experiment, we measured whether
noun phrase N1 has a dependency relation with the target adjective or
not. The result shows the accuracies of existing parsers are much lower
than that of our method.


\subsection{Collecting object-attribute pairs}

\begin{table}[b]
\caption{Experimental results.}
\label{tab:svm_result}
\input{06table04.txt}
\end{table}
\begin{table}[b]
\caption{Examples of attribute nouns.}
\label{tab:example_of_nouns}
\input{06table05.txt}
\end{table}


We constructed the classifier by using all the tagged data as training
data, and collected attribute nouns for the objective nouns ``hotel'',
``company'' and ``digital camera''. Table \ref{tab:example_of_nouns}
shows the list of obtained nouns, in the descending order of their
frequencies.

From manual observation of the results, we have found that (i) the
method extracts general nouns which represent classes as attributes
successfully, but is not very good at extracting specific names that
represent product names and other proper names, etc.; (ii) attribute
nouns that were not extracted by existing methods have been extracted,
and (iii) nouns that are ranked high are mostly attribute nouns,
i.e. high precision was obtained for higher ranks, while there are many
interesting attribute nouns among nouns ranked low, which suggests that
manual examination of those ranked low will be useful. As we have the
attribute and its values for the given object nouns, some of which have
not been successfully collected using existing methods, the data we were
able to obtain will be useful for opinion mining.


\section{Construction of concept graphs for the argument of adjectives}

In the previous section, we introduced a method of extracting
attribute nouns of an object noun with the evaluation of the attributes,
using the Japanese construction N1-Adj-N2. The results are promising. 
If we represent the relation between object nouns and attribute nouns 
for each adjective with the symbol ``\verb|->|'', we obtain the pattern 
``B \verb+->+A'', where A is an attribute noun and B is an object noun. 
For instance, from ``ラーメン が おいしい お店'' (``a restaurant 
that serves tasty noodles''), we obtain ``お店\verb|->|ラーメン'' 
(``restaurant\verb|->|noodle''), and from \mbox{``スー}プ が おいしい ラーメン''
(``noodles, whose soup is tasty'') we 
obtain ``ラーメン\verb|->|スープ'' (``noodle\verb|->|soup'').
This means that many of the object nouns can also be attribute nouns
in different contexts.

From the set of these relations for each adjective, we can construct
a chain of relations such as ``お店\verb+->+ラーメン\verb+->+スープ\verb+->+塩味''
(``restaurant\verb|->|noodle\verb|->|soup\verb|->|salty taste''). As
there are many attribute-object relations in which either an object
or an attribute is shared, we can expect that this will produce
a graph structure of object-attribute relations for each adjective,
in which many of the object nouns also work as attributes.
Instead of viewing the object-attribute patterns from the point of view of
objects, if we see them from the point of view of individual adjectives,
we can observe through this graph the structure of concepts that fit for 
the argument of the adjectives.

\subsection{Obtaining object-attribute structure}

Here, we observe the attribute-object structure constructed around the adjective
``おいしい'' (``delicious''). We applied phrase retrieval to the Web using 
``* が おいしいN'' (``* ga oishii N'') and obtained the data. For N, we selected 
3,586 nouns that belong to the category of ``food'' in a thesaurus \cite{ikehara:1997e}, 
which 
are expected to be related to the adjective ``おいしい''. We obtained a total of
248,627 snippets from the retrieval. By applying the method described above, 
we obtained 1,217 attribute nouns, 420 of which were not in the initial list 
of 3,586 nouns.  Using the 420 nouns as N, we repeated the same process, 
and obtained 1,810 attribute nouns, among which 586 are new. This process can 
be repeated further, but it may not converge due to erroneous nouns. 
We therefore stopped the repetition at this point. 

\subsection{Constructing the attribute-object graph}

From the set of attribute-object noun pairs for each adjective, we can
define a directed graph by treating each noun as a vertex and the
relation from an object noun to an attribute noun as a directed edge. As
mentioned above, this graph can be regarded as the conceptual graph for
the argument of the adjective.  Figure \ref{fig:oishii_graph} shows the
resultant graph obtained for the adjective ``おいしい'' (``delicious''). In the graph, we
generalised vertices by using the thesaurus for simplicity and removed
low-frequency edges in order to remove errors and unimportant relations.

\begin{figure}[t]
  \begin{center}
\includegraphics{17-3ia2f2.eps}
 \end{center}
  \caption{Graph of the adjective ``おいしい'' (``delicious'').}
  \label{fig:oishii_graph}
\end{figure}



\subsection{Observations}

A few interesting points can be observed in these graphs:
\begin{itemize}
\item There are vertices with no incoming edge or with no outgoing edge.
For the construction N1-Adj-N2, the nodes with no incoming edge
only appear in the position N2 and the nodes with no outgoing edge
only appear in the position N1, i.e. the selectional restriction for the
adjective is provided. For instance, as ``レストラン'' (``restaurants''), 
``日本'' (``Japan''), and ``季節'' (``season'') have no outgoing edges, we cannot 
say ``\{レストラン$|$日本$|$季節\} が おいしい'' (``\{restaurant$|$Japan$|$season\}
is delicious''). Among the set of concepts relevant to the adjective,
a few nouns cannot become arguments of the adjective.
\item The terminating vertices tend to represent abstract concepts. As can be 
seen in Figure \ref{fig:oishii_graph}, many paths terminate at the vertex
``刺激'' (``stimulus''). The actual nouns belonging to this node
are rather abstract, such as ``甘味'' (``sweetness''), ``喉ごし'' 
(``smoothness in the process of swallowing''), representing attributes
of which we can identify no further attributes. Note that this is a
generalisable characteristic. As long as the nouns represents
concrete objects or abstract objects, attributes can be extracted from them.
On the other hand, from the nouns representing general or abstract attributes,
no further attributes can be extracted\footnote{In ``色 が きれいな 野菜'' (``a 
vegetable with a beautiful color''), ``color'' is the terminating vertex, 
but from a different point of view, ``color'' can be an object noun, as 
in ``周波数 が 短い 色'' (``a color with short light waves'').}.
\item Between nouns that have ancestral relations (with one or more 
intervening vertices), only loose object-attribute relations can be
identified. In many cases, nouns that are not directly linked do not
have object-attribute relations. However, in some situations this can
be overridden, as in the example of ``おいしい イタリア'' (``Delicious Italy'')
mentioned in the introduction. The object-attribute relation obtained
in this work is related to the association made by human beings. In
ordinary situations, the object-attribute relation is limited to the
neighboring vertices, but there are situations where this can be extended.
We could perfectly say ``塩 が おいしい イタリア'' (``Italy whose salt is tasty'')
or ``オイル が おいしい イタリア'' (``Italy whose oil is tasty'').
\end{itemize}
This concept graph for each adjective will be useful for the 
lexicographical analysis of adjectives.


\section{Related work}

Related work is divided into three: (i) collecting attribute nouns; (ii)
mining opinion information; and (iii) disambiguating modifications.  For
work in collecting attribute nouns, there are studies that resort to the
HTML table structure \cite{yoshida:2004} and/or document layouts
\cite{tokunaga:2005,yoshinaga:2007}, those that use some patterns such
as ``A of B'' \cite{matthew:1999,abdulrahman:2004} and ``A no B''
\cite{tokunaga:2005,yoshinaga:2007}, and those that use search queries
by users. These methods proposed in this paper are characterised by the
fact that it can also collect relevant adjectives.

Collecting object-attribute-value was studied in opinion
mining. \cite{suzuki:2006} proposed a semi-supervised learning method
using a small number of triplets. 
They use
dependency information, and the performance depends on the result of
dependency analysis. Compared to these, our method depends on shallower
information and thus robust.

\cite{hashimoto:2001} tried to disambiguate N1-{\it no}-Adj-N2 construction,
i.e. whether N1 modifies Adj or N2, using decision list. It only uses
semantic information, while our method make use of wider range of
features like syntactic information.

\section{Conclusions}


In this paper, we extracted object-attribute noun relations, using the
Japanese construction ``N1 {\it ga} Adj N2'' and taking advantage of the
fact that N2 and N1 tend to take object-attribute relations. From the
object-attribute noun pairs obtained for each adjective Adj, we
constructed a structured relational frame for the adjective.

From the point of view of applications, we are currently exploring 
the extension of the method and the data in the following three directions:
\begin{enumerate}
\item The extraction of evaluations and opinions about the object,
in relation to the attributes. 
\item The description of overall ``relational frame'' patterns and 
``selectional restriction'' adjectives.
\item Facilitating the writing of catchy phrases for advertisements
and related types of writing assistance.
\end{enumerate}
For these (especially for the second direction), it is necessary to 
construct structured relational frames for a substantial number of adjectives.

Technically, we are now trying to improve the performance of the method
by incorporating the ``tournament model'' \cite{iida:2003} to obtain the 
most appropriate attribute noun from among multiple candidates. 


\bibliographystyle{jnlpbbl_1.4}
\begin{thebibliography}{}

\bibitem[\protect\BCAY{Almuhareb \BBA\ Poesio}{Almuhareb \BBA\
  Poesio}{2004}]{abdulrahman:2004}
Almuhareb, A.\BBACOMMA\ \BBA\ Poesio, M. \BBOP 2004\BBCP.
\newblock \BBOQ Attribute-based and value-based clustering: an
  evaluation.\BBCQ\
\newblock In {\Bem Proceedings of Empirical Methods in Natural Language
  Processing (EMNLP 2004)}, \mbox{\BPGS\ 158--165}.

\bibitem[\protect\BCAY{Hashimoto, Nishidate, Shirai, Tokunaga, \BBA\
  Tanaka}{Hashimoto et~al.}{2001}]{hashimoto:2001}
Hashimoto, T., Nishidate, K., Shirai, K., Tokunaga, T., \BBA\ Tanaka, H. \BBOP
  2001\BBCP.
\newblock \BBOQ Decision Lists for Determining Adjective Dependency in
  Japanese.\BBCQ\
\newblock In {\Bem Proceedings of Machine Translation Summit VIII},
  \mbox{\BPGS\ 151--156}.

\bibitem[\protect\BCAY{Iida, Inui, Takamura, \BBA\ Matsumoto}{Iida
  et~al.}{2003}]{iida:2003}
Iida, R., Inui, K., Takamura, H., \BBA\ Matsumoto, Y. \BBOP 2003\BBCP.
\newblock \BBOQ Incorporating Contextual Cues in Trainable Models for
  Coreference Resolution.\BBCQ\
\newblock In {\Bem Proceedings of EACL Workshop `The Computational Treatment of
  Anaphora'}, \mbox{\BPGS\ 23--30}.

\bibitem[\protect\BCAY{Ikehara, Miyazaki, Shirai, Yokoo, Nakaiwa, Ogura,
  Ooyama, \BBA\ Hayashi}{Ikehara et~al.}{1997}]{ikehara:1997e}
Ikehara, S., Miyazaki, M., Shirai, S., Yokoo, A., Nakaiwa, H., Ogura, K.,
  Ooyama, Y., \BBA\ Hayashi, Y. \BBOP 1997\BBCP.
\newblock {\Bem Goi-Taikei---A Japanese Lexicon}.
\newblock Iwanami Shoten, Tokyo.

\bibitem[\protect\BCAY{Kobayashi, Inui, \BBA\ Matsumoto}{Kobayashi
  et~al.}{2007}]{kobayashi:2007}
Kobayashi, N., Inui, K., \BBA\ Matsumoto, Y. \BBOP 2007\BBCP.
\newblock \BBOQ Extracting Aspect-Evaluation and Aspect-Of Relations in Opinion
  Mining.\BBCQ\
\newblock In {\Bem Proceedings of EMNLP-CoNLL 2007}, \mbox{\BPGS\ 1065--1074}.

\bibitem[\protect\BCAY{Kudo \BBA\ Matsumoto}{Kudo \BBA\
  Matsumoto}{2002}]{kudo:2002e}
Kudo, T.\BBACOMMA\ \BBA\ Matsumoto, Y. \BBOP 2002\BBCP.
\newblock \BBOQ Japanese Dependency Analysis using Cascaded Chunking.\BBCQ\
\newblock In {\Bem CoNLL 2002: Proceedings of the 6th Conference on Natural
  Language Learning 2002 (COLING 2002 Post-Conference Workshops)}, \mbox{\BPGS\
  63--69}.

\bibitem[\protect\BCAY{Kurohashi \BBA\ Nagao}{Kurohashi \BBA\
  Nagao}{1994}]{kurohasi:94:b}
Kurohashi, S.\BBACOMMA\ \BBA\ Nagao, M. \BBOP 1994\BBCP.
\newblock \BBOQ KN Parser: Japanese Dependency/Case Structure Analyzer.\BBCQ\
\newblock In {\Bem Proceedings of the Workshop on Sharable Natural Language
  Resources}, \mbox{\BPGS\ 48--55}.

\bibitem[\protect\BCAY{Kurohashi \BBA\ Nagao}{Kurohashi \BBA\
  Nagao}{1998}]{kurohasi:98:ae}
Kurohashi, S.\BBACOMMA\ \BBA\ Nagao, M. \BBOP 1998\BBCP.
\newblock {\Bem Japanese Morphological Analysis System JUMAN version 3.5}.
\newblock Department of Informatics, Kyoto University.
\newblock (in Japanese).

\bibitem[\protect\BCAY{Berland \BBA\ Charniak}{Berland \BBA\
  Charniak}{1999}]{matthew:1999}
Berland, M.\BBACOMMA\ \BBA\ Charniak, E. \BBOP 1999\BBCP.
\newblock \BBOQ Finding Parts in Very Large Corpora.\BBCQ\
\newblock In {\Bem Proceedings of the 37th Annual Meeting of the ACL},
  \mbox{\BPGS\ 57--64}.

\bibitem[\protect\BCAY{Mikami}{Mikami}{1960}]{mikami:60}
Mikami, A. \BBOP 1960\BBCP.
\newblock {\Bem Elephants have long trunks (zou wa hana ga nagai)}.
\newblock Kuroshio-shuppan.
\newblock (in Japanese).

\bibitem[\protect\BCAY{Suzuki, Takamura, \BBA\ Okumura}{Suzuki
  et~al.}{2006}]{suzuki:2006}
Suzuki, Y., Takamura, H., \BBA\ Okumura, M. \BBOP 2006\BBCP.
\newblock \BBOQ Application of Semi-supervised Learning to Evaluative
  Expression Classification.\BBCQ\
\newblock In {\Bem Proceedings of 7th International Conference on Intelligent
  Text Processing and Computational Linguistics (CICLing'06)}, \mbox{\BPGS\
  502--513}.

\bibitem[\protect\BCAY{Tokunaga, Kazama, \BBA\ Torisawa}{Tokunaga
  et~al.}{2005}]{tokunaga:2005}
Tokunaga, K., Kazama, J., \BBA\ Torisawa, K. \BBOP 2005\BBCP.
\newblock \BBOQ Automatic Discovery of Attribute Words from Web
  Documents.\BBCQ\
\newblock In {\Bem Proceedings of 2nd International Joint Conference on Natural
  Language Processing (IJCNLP2005)}, \mbox{\BPGS\ 106--118}.

\bibitem[\protect\BCAY{Uchimoto, Murata, Sekine, \BBA\ Isahara}{Uchimoto
  et~al.}{2000}]{utimoto:2000e}
Uchimoto, K., Murata, M., Sekine, S., \BBA\ Isahara, H. \BBOP 2000\BBCP.
\newblock \BBOQ Dependency Model Using Posterior Context.\BBCQ\
\newblock In {\Bem Proceedings of the Sixth International Workshop on Parsing
  Technology (IWPT2000)}, \mbox{\BPGS\ 321--322}.

\bibitem[\protect\BCAY{{Yahoo! JAPAN}}{{Yahoo! JAPAN}}{2008}]{yahoo:2008}
{Yahoo! JAPAN} \BBOP 2008\BBCP.
\newblock \BBOQ Developer Network {Web} Search {API}.\BBCQ\
  http://{\linebreak[2]}developer.{\linebreak[2]}yahoo.co.jp/{\linebreak[2]}se
arch/.

\bibitem[\protect\BCAY{Yoshida \BBA\ Nakagawa}{Yoshida \BBA\
  Nakagawa}{2004}]{yoshida:2004}
Yoshida, M.\BBACOMMA\ \BBA\ Nakagawa, H. \BBOP 2004\BBCP.
\newblock \BBOQ Specification Retrieval---How to Find Attribute-Value
  Information on the Web.\BBCQ\
\newblock In {\Bem Proceedings of 1st International Joint Conference on Natural
  Language Processing (IJCNLP2004)}, \mbox{\BPGS\ 338--347}.

\bibitem[\protect\BCAY{Yoshinaga \BBA\ Torisawa}{Yoshinaga \BBA\
  Torisawa}{2007}]{yoshinaga:2007}
Yoshinaga, N.\BBACOMMA\ \BBA\ Torisawa, K. \BBOP 2007\BBCP.
\newblock \BBOQ Open-Domain Attribute-Value Acquisition from Semi-Structured
  Texts.\BBCQ\
\newblock In {\Bem Proceedings of Workshop of OntoLex07---From Text to
  Knowledge: The Lexicon/Ontology Interface held at the sixth International
  Semantic Web Conference}, \mbox{\BPGS\ 55--66}.

\end{thebibliography}


\begin{biography}
\bioauthor[:]{Takeshi Abekawa}{
Takeshi Abekawa is a research assistant professor at National Institute
of Informatics. His research interests include natural language
processing and computational linguistics, especially information
extraction from large corpus. He received Doctor of Engineering from
Tokyo Institute of Technology in 2006.
}
\bioauthor[:]{Manabu Okumura}{
Manabu Okumura is a professor at Precision and Intelligence Laboratory in
Tokyo Institute of Technology. His recent work in natural language
processing focuses on sentiment analysis in text, automated text
summarization and text data mining. He received Doctor of Engineering
from Tokyo Institute of Technology in 1989.
}
\end{biography}

\biodate

\end{document}
