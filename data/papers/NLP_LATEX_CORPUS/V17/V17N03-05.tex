    \documentclass[english]{jnlp_1.4}

\usepackage{jnlpbbl_1.2}
\usepackage[dvips]{graphicx}
\usepackage{hangcaption_jnlp}




\Volume{17}
\Number{3}
\Month{April}
\Year{2010}


\received{2009}{5}{15}
\revised{2009}{12}{9}
\accepted{2009}{12}{16}

\setcounter{page}{81}


\etitle{Treatment of Legal Sentences Including Itemization Written in Japanese, English and Vietnamese\\
---Towards Translation into Logical Forms---}
\eauthor{Makoto Nakamura\affiref{Author_1} 
	\and Yusuke Kimura\affiref{Author_1}
	\and Minh Quang Nhat Pham\affiref{Author_1}
	\and \\
	Le Minh Nguyen\affiref{Author_1}
	\and Akira Shimazu\affiref{Author_1}} 
\eabstract{
 This paper reports how to treat legal sentences including itemized
 expressions in three languages.
 Thus far, we have developed a system for
 translating legal sentences into logical formulae. Although our system
 basically converts words and phrases in a target sentence into
 predicates in a logical formula, it generates some useless predicates
 for itemized and referential expressions. 
 In the previous study, focusing on Japanese Law, we
 have made a front end system which substitutes corresponding
 referent phrases for these expressions.
 In this paper, we examine our approach to the Vietnamese Law and the
 United States Code.
 Our linguistic analysis shows the difference in notation among
 languages or nations, and we extracted conventional expressions
 denoting itemization for each language.
 The experimental result shows high accuracy in terms of generating
 independent, plain sentences from the law articles including itemization. 
 The proposed system generates a meaningful
 text with high readability, which can be input into our translation
 system.
}
\ekeywords{Legal Engineering, Itemization, Vietnamese Law}

\headauthor{Nakamura et al.}
\headtitle{Treatment of Legal Sentences Including Itemization}

\affilabel{Author_1}{}{School of Information Science, Japan Advanced Institute of Science and Technology}


\begin{document}

\maketitle

\section{Introduction}

A new research field called \textit{Legal Engineering} was proposed in
the 21st Century COE Program, Verifiable and Evolvable
e-Society~\cite{katayama05,katayama07,katayama08}.
Legal Engineering serves for computer-aided examination and verification
of whether a law has been established appropriately according to its purpose,
whether there are logical contradictions or problems in the document per se,
whether the law is consistent with related laws, and whether its
revisions have been modified, added, and deleted consistently.
One approach to verifying law sentences is to convert law sentences into
logical or formal expressions~\cite{nakamura07jurisin} and to verify them based on
inference~\cite{hagi06jurix,hagi06}.

Thus far, in order to take charge of text processing,
we have developed a system for automatically converting
Japanese legal documents into logical forms~\cite{nakamura07jurisin}.  
The system analyzes law
sentences, determines logical structures, and then generates logical
expressions. We have shown our system provides high accuracy in terms of 
generating logical predicates corresponding to words and their semantic
relations.
However, some predicates generated concerned with itemization and reference
were meaningless, because predicates converted from words and phrases,
such as ``the items below,'' ``Article 5,'' and so on are not intrinsic
to a logical representation of the sentence.
These words should be replaced with appropriate phrases
before the process of translation.
Since Japanese legal documents have strict rules concerning its description
and modification, we succeeded to extract conventional expressions in
the documents by some regular expressions~\cite{nakamura08jurisin_lnai}.
In order to investigate whether the proper method depends on the
language or the nation establishing laws, we try to apply our
approach to the English and Vietnamese versions of Vietnamese Law and
the United States Code.
Therefore, our purpose in this paper is to show a difference
of the method to generate independent, plain sentences from legal texts
including itemization. 
This study is regarded as a derivation of the series of our main study
to translate legal documents into logical forms~\cite{nakamura07jurisin}.
We expect that these fruitful results are able to be applied not only to
the English and Vietnamese versions of the translation system into logical forms, but 
also to a support system for reading legal documents and a text-to-speech
system.





In this paper, we introduce our current system and its problems in
Section~2. In Section~3 we show analysis of
law sentences including itemization or reference, and we propose
a method to rewrite the law sentences into plain sentences in
Section~4. 
We also examine our new method and report its results in Section~5. 
Finally, we conclude and describe our future work in Section~6.


\section{The current system and problems}

In this section, we describe our current system for translating legal
documents into logical forms, and its problems. We call our
system WILDCATS\footnote{WILDCATS is a recursive acronym of `` `Wildcats' Is a Legal
Domain Controller As a Translation System.''}.

\subsection{Wildcats}

Acquisition of knowledge bases by automatically reading natural language
texts has widely been studied.
While the definition of semantic representation differs depending on
what the language processing systems deal with, some systems try to
generate logical formulae based on first order predicate
logic~\cite{hobbs88,mulkar07learning,mulkar07ijcai}.
Legal documents are suited for translating into logical representation, since 
they are different from daily-use texts in that they are
described with characteristic expressions in order to avoid ambiguous
description. 
Taking into account the linguistic analysis of the expressions, we can
extract the logical structure of legal documents.

Our system, Wildcats, derives logical forms forms from law sentences~\cite{nakamura07jurisin}.
We explain an outline of our current system, showing an example of input and output in Fig.~\ref{fig:prev_problem}.

In most cases, a law sentence in Japanese Law consists of a law
requisite part 
and a law effectuation part, which designate its legal logical
structure~\cite{tanaka93,nagai95}.
The structure of a sentence in terms of these parts is shown in
Fig.~\ref{fig:youken}.
The law requisite part is further divided into a subject part and a
condition part, and the law effectuation part is divided into an object,
content, and provision part.

\begin{figure}[b]
\begin{center}
\includegraphics{17-3ia5f1.eps}
\end{center}
\caption{Converting a law sentence including a reference phrase.}
\label{fig:prev_problem}
\end{figure}
\begin{figure}[b]
\begin{center}
\includegraphics{17-3ia5f2.eps}
\end{center}
\caption{Structure of requisition and effectuation.}
\label{fig:youken}
\end{figure}

Dividing a sentence into these two parts in the pre-processing stage makes
the main procedure more efficient and accurate.
Nagai~et~al.~\cite{nagai95} proposed an acquisition model for this
structure from Japanese law sentences.
Dealing with strict linguistic constraints of law sentences, their model
succeeded in acquiring the structures at fairly high accuracy using a simple
method, which specifies the surface forms of law sentences.
Our approach is different from theirs in that we consider some semantic
analyses in order to represent logical formulae.


The following list is the procedure for one sentence (See Fig.~\ref{fig:flow}).
We repeat it when we process a set of sentences.
\begin{enumerate}
 \item Analyzing morphology by JUMAN\footnote{JUMAN is a morphological
       analyzer of Japanese developed by Kyoto
       University~\cite{kurohashi94juman}. It segments sentences into
       morpheme 
       sequences with many additional pieces of information such as
       a semantic category of each word, a part of speech, and so on, 
       based on a hidden
       Markov model.} and parsing
       a target sentence by KNP\footnote{KNP is a rule-based Japanese
       dependency analyzer developed by Kyoto University~\cite{kurohashi94knp}.}.
       
 \item Splitting the sentence based on the characteristic structure of
       a law sentence.
 \item Assignment of modal operators with the cue of auxiliary verbs. 
 \item Making one paraphrase of multiple similar expressions for 
       unified expression.
 \item Analyzing clauses and noun phrases using a case frame dictionary.
 \item Assigning variables and logical predicates.  We generally assign verb phrases and
       {\it sahen}-nouns\footnote{A {\it sahen}-noun is a noun which can
       become a verb with the suffix {\it -suru}.} to a logical
       predicate and an event variable, $e_i$,
       and other content words to a case role predicate and $x_j$, which
       represents an argument of a logical predicate.
 \item Building a logical formula based on fragments of logical
       connectives, modal operators, and predicates.
\end{enumerate}

\begin{figure}[t]
\begin{center}
\includegraphics{17-3ia5f3.eps}
\end{center}
\caption{Flow chart of Wildcats.}
\label{fig:flow}
\end{figure}


The procedure is roughly divided into two parts.
One is to make the outside frame of the logical form (Step 1 to 3 and 7),
which corresponds to the legal logical structure shown in Fig.~\ref{fig:youken}.
The other (Step 4 to 6) is for the inside frame. We assign noun phrases to bound
variables and predicates using a case frame dictionary.


\subsection{Problems of wildcats}

When our system converts a law sentence including referential phrases,
it is not interpreted correctly.
For example, in Fig.~\ref{fig:prev_problem}, the enclosed predicate
``items\_below(x6)'' is useless.
Even if the post-processing system performs logical operations with the
generated logical expressions, it does not result in line with our expectations.
This is because the generated predicates lack information which must
be referred.
In case of Fig.~\ref{fig:prev_problem}, there is no connection between
the predicate ``items\_below(x6)'' and the ones in the following items.

One solution is to replace these phrases with appropriate ones in the
items before the process of translation into logical forms.
In other words, the front end system of Wildcats rewrites a
sentence including itemization into a plain sentence.
Therefore, substituting corresponding referent phrases for these
expressions appropriately, our proposed system in this paper generates a
meaningful text with high readability, and then the generated text can
be input to the translation system.
For example, the system should process the following instead of the
input sentences in Fig.~\ref{fig:prev_problem}; ``The right to receive a
survivor's basic pension lapses when the recipient dies.''
As long as treating meaningful sentences like this, the system does not generate
any more redundant predicate such as ``items\_below(x6).''
In this paper, we propose a pre-processing system which modifies input
sentences including itemization.


\subsection{Scope of our study}

The scope of the study in this paper is restricted to 
sentences including itemized expressions written in Japanese,
Vietnamese, and English.
In the preceding study~\cite{nakamura08jurisin_lnai}, focusing on Japanese
legal sentences, we showed that the system worked well
using some simple regular expressions.
In this paper, we apply it to the Vietnamese Law on Enterprises and the
United States Code: Title 39-Postal Service.



\section{Analysis of law sentences with itemization}

In general, a notation of law sentences is strictly affected not only by the written 
language, but also by the nation establishing the law.
In other words, it depends on the legislative process whether or not our simple 
approach is useful for other countries' law.
In this section, we analyze law sentences including itemization from the
following two aspects; One is linguistic characteristics,
and the other is comparison of legislative proceedings among nations.


\subsection{Characteristics of law documents among nations}

The legislation system of Japanese Law is rational to keep the notation
of expressions of law\footnote{
This section is written based on the discussion with Prof. Matsuura in
  Graduate School of Law, Nagoya University.
For more detail about the administrative structure of legislation of
Japanese Law, see Nagano~\cite{nagano05}.
}.
A bill is basically proposed by the proper authority 
of the law.  Once the authority has made a draft of the bill, it negotiates 
with other authorities.  After that, the cabinet strictly examines the 
draft in terms of inconsistency with other laws, expressions, formats and 
so on, using the database of legislation.  
As a result, this system keeps even the usage of comma and period.

Not all other countries have the system similar to Japan.  
In the United Kingdom, the description check by the legislature is
not as strict as Japan, since in most cases the draft of a bill is
prepared by an outsider of the ministry.  In the United States of America, 
there is no organization or system for consolidating expressions of laws.
In Asian countries except Japan and Korea, each ministry independently
makes out a draft of a bill without coordinating various opinions from other
ministries. As a result, the notation of bills becomes different among
ministries.  Moreover, in some countries bills are often modified
during deliberation in the national assembly, while bills mostly pass the
National Diet in Japan as drafted.  This political process causes 
inconsistencies in notation.

Vietnamese laws are also strictly examined by a number of organizations
concerning the laws before passing the National Diet~\cite{endo07}.
Particularly, there is a rigorous inspection about interpretation of
laws by the standing committee in Parliament, though it is unknown
whether the notation of expressions is surveyed by some authorities as
strict as Japan.
In fact, in order to reform the legislative structure,
the Vietnamese government enacted the Law on Promulgation of the Legal
Documents\footnote{http://vietlaw.gov.vn/LAWNET/docView.do?docid=22443\&type=html
(in Vietnamese)}
in 2008~\cite{endo08}.
This law is considered as an evidence that laws have carelessly been
proposed in a number of independent organizations so far.
Thus, it seems difficult to keep a writing notation in administrative
documents in Vietnam.
This is a difference between Japan and Vietnam from a point of view of
keeping the notation.


\subsection{Definition of itemization}

In general, a law consists of a number of articles, each of which is
further subdivided into a number of paragraphs or items.
Both articles, paragraphs, and items have sequential numbers with
a typeface different from each other.  For example, in the English
version of Vietnamese Law, articles start with ``Article 1,'' ``Article
2'' and so on, paragraphs with ``$1., 2., \ldots$,'' and items with
``$(a), (b), \ldots$.''
Although there are a few differences in notation between English and
Vietnamese, itemization structure can be dealt with easily by pre-processing.


In the case of Japanese Law, we basically recognize an itemized
expression as a noun phrase or a subordinate clause following the upper
paragraph or article, which consists of sentences.
An example of itemized expression is shown in Fig.~\ref{fig:prev_problem}.
On the contrary, in the case of Vietnamese Law, even some articles
are expressed as a phrase which lacks the subject or the main verb.
We show an example in Fig.~\ref{fig:art21}.
Taking it into consideration, we define itemization, with which we deal in this
study, as a phrase or a sentence following a sentence in an article or a
paragraph.
The article shown in Fig.~\ref{fig:art21} is not
recognized as itemization, because it does not have a complete sentence.



\subsection{Analysis of itemization in Japanese law sentences}
\label{sec:analysis_japanese}


\begin{figure}[b]
\begin{center}
\includegraphics{17-3ia5f4.eps}
\end{center}
\caption{Article 21 in the Vietnamese Law on Enterprises.}
\label{fig:art21}
\end{figure}
\begin{figure}[b]
\begin{center}
\includegraphics{17-3ia5f5.eps}
\end{center}
\caption{Itemization of conditions in the law requisite part.}
\label{fig:itemization}
\end{figure}

Some law sentences include itemization of conditions in the law
requisite part, an example of which is shown in Fig.~\ref{fig:itemization}. 
The enclosed phrase should be replaced with one of the items
denoting actual conditions.
When one or more conditions are satisfied, the description in the law
effectuation part becomes effective.
We found 34 sentences of such a style in National Pension Law.
Therefore, we considered a method to embed itemized conditions
instead of cue phrases of itemization.

We defined \textit{Key Phrases}, which always appear in sentences before
an itemization\footnote{There may be a proviso between the sentence and
itemization.}.
As we analyzed sentences from all 215 articles of the National Pension
Law, the set of Key Phrases can be expressed as a regular expression,
the diagram of which is shown in Fig.~\ref{fig:keyphrase}.
For example, the phrase \textit{``Tsugi no kaku gou ni gaitou suru ni
itatta,''} meaning ``to result in coming under either of the items
below\footnote{If we do not care about word-to-word translation for the
Japanese law sentence, the following phrase is more appropriate; ``to be
included in one of the following cases.''},'' which is derived from the
generative rule in Fig.~\ref{fig:keyphrase}, is regarded as a Key Phrase.

Itemized condition sentences appear next to sentences which contain Key Phrases.
The last words of these sentences are ``{\it Toki} (time),'' ``{\it
Mono} (person),'' and so on.
In this paper, we call these sentences excluding the last words \textit{Condition Items}.
Key Phrases and Condition Items appearing in National Pension Law are
shown in Table~\ref{tab:appearancekey} and
Table~\ref{tab:appearancecon}, respectively.


\begin{figure}[b]
\begin{center}
\includegraphics{17-3ia5f6.eps}
\end{center}
\caption{Key phrases for itemization.}
\label{fig:keyphrase}
\end{figure}
\begin{table}[b]
\begin{minipage}[t]{0.45\textwidth}
\caption{Frequency of Key Phrases.}
\label{tab:appearancekey}
\input{05table01.txt}
   \end{minipage}
\hfill
   \begin{minipage}[t]{0.45\textwidth}
\caption{Frequency of Condition Items.}
\label{tab:appearancecon}
\input{05table02.txt}
\end{minipage}
\end{table}


We will describe a method to remove itemization using Key Phrases and
Condition Items in Section 4.



\subsection{Analyses of itemization in Vietnamese Law sentences written
  in Vietnamese and English}

In order to find {\it Key Phrases} and {\it Condition Items},
we analyzed 100 out of 172 articles in the Vietnamese Law on Enterprises.
Although all Key Phrases identify one regular expression in Japanese 
Law, 
we defined 15 and 14 rules of regular expression for Vietnamese and
English, respectively.
This means there are a variety of expressions denoting itemization
in Vietnamese Law.
We show the set of rules for the English and Vietnamese versions
of Vietnamese Law in Fig.~\ref{fig:kp_english} and
Fig.~\ref{fig:kp_vietnamese}, respectively.
Since we manually made the sets of rules in English and Vietnamese separately, 
each rule in the English version does not correspond to that of
Vietnamese with the same label, and vice versa.

\begin{figure}[b]
\begin{center}
\includegraphics{17-3ia5f7.eps}
\end{center}
\caption{Key Phrases for the English version of Vietnamese Law.}
\label{fig:kp_english}
\end{figure}
\begin{figure}[b]
\begin{center}
\includegraphics{17-3ia5f8.eps}
\end{center}
\caption{Key Phrases for Vietnamese Law.}
\label{fig:kp_vietnamese}
\end{figure}


In the English version of the law, we need to deal with inflection of
words.
The number of rules would be reduced,
if we did not consider an irregular conjugation for some particular nouns.
For example, Rule 9 accepts the phrase ``following rights,'' ``following 
obligations,'' or ``following undertakings'' and generates an
appropriate phrase with the condition item, omitting the word
``following'' and the suffix `-s.' 
Because some irregular conjugations such as `duties' are not accepted
by Rule 10, an additional rule is added to the rule-set.
Rule 10 works the same as Rule 9 except replacing the suffix `-ies' to 
`-y.'
Moreover, each regular expression accepts a number of Key Phrases 
corresponding to a Condition Item.
In other words, some rules which could be merged with other rules are
separated due to the different Condition Items.

Similar to Japanese, Vietnamese does not distinguish singular or plural
nouns by inflection.  
Vietnamese distinguish singular and plural nouns by quantifiers which
precede corresponding nouns, such as `c\'ac (all),' 
`nh\~u\hspace{-0.6mm}'ng (some),'
`t$\acute{\hat{\mbox{a}}}$t c$\acute{\mbox{a}}$ (every),'
`m\d{\^o}t (one),' 
`hai (two),' 
and so on.
The rules of Vietnamese are simpler than that of English, being not
necessary for the process of inflection.
Some rules similar to each other are distinguished depending on the
Condition Items corresponding to the rule.





\subsection{Analysis of itemization in the United States Code}

We analyzed Postal Service of the United States Code.
As a result, we defined 4 rules of regular expression, shown in
Fig.~\ref{fig:kp_us}.

\begin{figure}[b]
\begin{center}
\includegraphics{17-3ia5f9.eps}
\end{center}
\caption{Key Phrases for the United States Code.}
\label{fig:kp_us}
\end{figure}
\begin{figure}[b]
\begin{center}
\includegraphics{17-3ia5f10.eps}
\end{center}
\caption{Example of itemization in the United States Code.}
\label{fig:us_sample}
\end{figure}


Rule 1 in Fig.~\ref{fig:kp_us} covers most of the items, since the main
clause becomes a complete sentence, replacing a hyphen (`-') at the
last of the clause with each item.
Figure~\ref{fig:us_sample} shows an example of itemization.
Therefore, we rarely took care of inflection for the rule extraction.
This simple expression seems to be a result of study for high
readability, although it differs from the English version of the Vietnamese Law.


\section{Method for removing itemization}

In Section \ref{sec:analysis_japanese}, we defined Key Phrases as cue phrases that always appear
with itemization, like ``{\it tsugi-no kaku gou no izureka ni
gaitou-suru} ((something) to which either of the following items is
applicable),'' and we search for itemization with it.
If a Key Phrase is found, we regard the following items as Condition
Items, and replace the Key Phrase with one of the Condition Items for
each. Then we have sentences which are understandable
separately, as shown in Fig.~\ref{fig:itemize1}.
We show an example of the pair of input and output in
Fig.~\ref{fig:itemize2}.


\begin{figure}[b]
\begin{center}
\includegraphics{17-3ia5f11.eps}
\end{center}
 \caption{Removing itemization.}
 \label{fig:itemize1}
\end{figure}
\begin{figure}[b]
\begin{center}
\includegraphics{17-3ia5f12.eps}
\end{center}
 \caption{An example of removing itemization.}
 \label{fig:itemize2}
\end{figure}


The process of Vietnamese Law is different from that of Japanese in that 
there are a number of rules of regular expression.
Since some rules conflict with other rules, priority is established
in the order of the rule number.
Each rule has a corresponding Condition Item, which is defined as
regular expression.
We show an example of rewriting itemized expressions in
Fig.~\ref{fig:itemize_eng}, in which the dashed box labeled `Key
Phrases' denotes a part of rewriting rules.
Rule 9 rewrites a phrase matching the regular expression in the left
hand side to one of the words in the bracket.
In this case, Rule 10 matches the phrase `following duties' in the
article, which is replaced to the word `duty' described in the right
hand side in Rule 10.



\section{Experiments and results}\label{sec:exp}

We tested our system on itemization.  The test set for each language is
shown in Table~\ref{tab:lawlist}.  
Since we extract Key Phrases of Japanese from National Pension Law, we
used it for a closed test.  
For an open test we used Income Tax Law as the test set.

\begin{figure}[b]
\begin{center}
\includegraphics{17-3ia5f13.eps}
\end{center}
 \caption{An example of removing itemization 
(Article 71-2, the Vietnamese Law on Enterprises).}
 \label{fig:itemize_eng}
\end{figure}
\begin{table}[b]
\caption{Input texts for open and closed tests.}
\label{tab:lawlist}
\input{05table03.txt}
\end{table}



In these experiments, it is difficult to establish a baseline due to
the distinctiveness of our model and its target.
Some studies which extract web contents from HTML or XML
documents~\cite{bing} may be able to deal with itemization in HTML
or XML documents.
However, our method is different from them in that it includes 
process to find itemized phrases without a tag, and to make plain
sentences.
We examine whether or not our model works well to the law documents
in some languages, regardless of the linguistic characteristics, or of
its nation which established the law.



We extract Key Phrases of both Vietnamese and English from 100 out of
172 articles in the Vietnamese Law on Enterprises.
Therefore, we use sentences from Article 1 to 100 for a closed test and
from Article 101 to 172 for an open test.
Hereafter, we call the open test {\it vopen1}.
In addition, we examine the Law on Bankruptcy for another open test, called
{\it vopen2}.
We have two kinds of experiments; one is to examine whether the system
successfully identify articles including itemization, and the other is
to measure the accuracy of removing itemized expressions that the system
successfully found.

For the United States Code, Key Phrases are extracted from Postal
Service.
For an open test we used the US Code-Title41:Public Contracts as the test set.

\begin{table}[b]
\caption{Experimental results for identifying itemization.}
\label{tab:result_id}
\input{05table04.txt}
\end{table}


Firstly, we show the experimental result for identifying itemization by key
phrases in Table~\ref{tab:result_id}.
The labels `\#Art,' `Find,' `Over,' `Err,' `P,' and `R' denote `the
number of articles including itemization,' `the number of articles that
the system identified as itemized expressions,' `the number of
oversights,' `the number of errors,' `precision' and `recall,' respectively.
The result shows that the system sufficiently found articles including
itemization except {\it vopen2} in the English test.
In the Law on Bankruptcy, expressions are quite different from those in the
Law on Enterprises.
We show an example of the Law on Bankruptcy in Fig.~\ref{fig:err_open2_english}.
In this case, the complete sentences can be generated by adding each
item to the last of the main clause with some minor modifications to remove
a colon (:) and replace a semi-colon (;) with a period (.).
Although we did not extract any rule processing this expression from the
Law on Enterprises, it is often used in the United States Code.
The difference between the Law on Enterprises and the Law on Bankruptcy in
notation may be caused by different interpreters.

\begin{table}[t]
\caption{Experimental results for removing itemization.}
\label{tab:result_rm}
\input{05table05.txt}
\end{table}
\begin{figure}[t]
\begin{center}
\includegraphics{17-3ia5f14.eps}
\end{center}
\caption{Example of failure in the English version of the Law on Bankruptcy.}
\label{fig:err_open2_english}
\end{figure}


Secondly, the experimental results for removing itemization are shown in Table~\ref{tab:result_rm}.
The labels `\#item,' `Succ,' `Err' and `Acc' denote `the number of items
to be processed,' `the number of items that the system successfully
processed,' `the number of errors,' and `accuracy,' respectively.
In the closed test of Japanese, 
we found that 11 of the whole errors were items which denote a combination
of a Condition Item and an object part in the law effectuation part.
In other words, the objects of these sentences change depending on the
Condition Items.
An example is shown in Fig.~\ref{fig:err_japanese}.
This article determines the revision of the rate after the base year about
the national pension.  An important thing here is that each item
consists of a condition part and its result, separated with a
space\footnote{In Japanese writing, no spaces are left between words.
Since there is a space only between the condition part and the result in the item,
they are absolutely identified.}.
That is, the first Key Phrase denoting ``In the case of the following items,''
which is emphasized corresponds to the first phrases of each item, while
the second Key Phrase denoting ``on the basis of the rate on the item''
which is underlined corresponds to the second phrases of each item.
Our system did not deal with this type of itemization.


For the result of open test with Income Tax Law, a little more than half
of the sentences were processed well. There seems to be some 
difference in notation between National Pension Law and Income Tax
Law. 
Particularly, we found the increase of itemization consisting of a
combination of a Condition Item and an object part to 84.
Results will be improved after an analysis of the mistakes.



The results of both English and Vietnamese show higher accuracy than
that of Japanese in terms of removing itemized expressions.
This is because the number of rules of regular expression is increased
to 14 and 15, while there is only one rule for Japanese.
In the English test, 
we found that some Key Phrases were followed by a number of types of
Condition Items different from each other,
so that the set of rules did not cover all the Key Phrases even in the closed test.
Because this decision becomes much more difficult in the open tests, the
accuracies in the open tests come down to 73.2\% for {\it vopen1} and 68.2\%
for {\it vopen2}.
We show an example of failure which occurred in {\it vopen1} in Fig.~\ref{fig:faileng}.
There are two rules which deal with the phrase ``in the following
cases'' in Rule 4 and 5 in Fig.~\ref{fig:kp_english}, which rewrite it to the corresponding
phrases ``in the case of'' and ``in the case that,'' respectively.
Figure~\ref{fig:faileng} shows that the key phrase was replaced to
the phrase ``in the case that'' although the following item is a noun phrase.

\begin{figure}[b]
\begin{center}
\includegraphics{17-3ia5f15.eps}
\end{center}
\caption{Example of failure in Japanese National Pension Law.}
\label{fig:err_japanese}
\end{figure}
\begin{figure}[b]
\begin{center}
\includegraphics{17-3ia5f16.eps}
\end{center}
 \caption{An example of failure
(Article 127-1, the Vietnamese Law on Enterprises, English).}
 \label{fig:faileng}
\end{figure}

In the case of Vietnamese law text, we also found some errors that the meaning
of sentences generated are different from original meaning, and
ungrammatical sentences may be generated.
However, the accuracy keeps high even in the open test.
This is because the Vietnamese grammar is not as strict as English in terms
of distinction between a phrase and a clause.
Figure~\ref{fig:succviet} shows that the key phrase is successfully
replaced in the itemized expression where it failed in the English version.

\begin{figure}[t]
\begin{center}
\includegraphics{17-3ia5f17.eps}
\end{center}
 \caption{An example of success
(Article 127-1, the Vietnamese Law on Enterprises, Vietnamese).}
 \label{fig:succviet}
\end{figure}

In the case of the US Code, the accuracy decreased in the open test.
The difference in notation of itemization between the titles affects not
only the result on identifying itemization, but also the one on removing itemization.
This problem can be overcome by adding extra rules to the Key Phrase.

Overall accuracy would be improved depending on the rule set of regular expression.
Therefore, we can conclude that our method is quite suitable not only for Japanese
legal texts but also for other languages with some modification.









\section{Conclusion}

In this paper we proposed a method to rewrite legal sentences including
itemization into independent, plain sentences, focusing on laws written
in three languages.
From the linguistic analyses, we showed the difference of the number of
regular expressions for extracting Key Phrases between Japanese,
Vietnamese and English. 
It implies that fixed expressions are often used in Japanese Law.
In Vietnamese law documents, there are some common words and phrases
which appear with high frequency at the Key Phrases.
In the United States Code, most of itemized forms are identified with a
hyphen at the last of the main clause.
Further investigation of Vietnamese words and phrases is required for
making accurate regular expression rules.

In the experiments, 
we showed that the system successfully extracted
itemized expressions with some exceptions.
We consider that the system is useful not only for the front end of our
main system, Wildcats, but also for assistance in reading 
legal documents.
We can improve this system by introducing a method for
enhancing readability of the output sentences.
Concerning the Vietnamese version of translation system (Wildcats),
we need to wait for the development of a dependency parser for Vietnamese.




\acknowledgment

We would like to give special thanks to Prof. Yoshiharu Matsuura in Nagoya
University for discussion about the differences among nations in notation 
of law documents.
This research was partly supported by the 21st Century COE Program
`Verifiable and Evolvable e-Society' and Grant-in-Aid for Scientific
Research (19650028 and 20300057).  




\bibliographystyle{jnlpbbl_1.4}
\begin{thebibliography}{}

\bibitem[\protect\BCAY{Endo}{Endo}{2007}]{endo07}
Endo, S. \BBOP 2007\BBCP.
\newblock \BBOQ The national assembly and legislative process in {V}ietnam (in
  {J}apanese).\BBCQ\
\newblock {\Bem Gaikoku no Rippou ({F}oreign Legislation)}, {\Bbf 231},
  \mbox{\BPGS\ 110--151}.

\bibitem[\protect\BCAY{Endo}{Endo}{2008}]{endo08}
Endo, S. \BBOP 2008\BBCP.
\newblock \BBOQ Legal structure reform in {V}ietnam: {L}aw on {P}romulgation of
  the {L}egal {D}ocuments 2008 (in {J}apanese).\BBCQ\
\newblock {\Bem Gaikoku no Rippou ({F}oreign Legislation)}, {\Bbf 238},
  \mbox{\BPGS\ 177--190}.

\bibitem[\protect\BCAY{Hagiwara \BBA\ Tojo}{Hagiwara \BBA\
  Tojo}{2006a}]{hagi06jurix}
Hagiwara, S.\BBACOMMA\ \BBA\ Tojo, S. \BBOP 2006a\BBCP.
\newblock \BBOQ Discordance Detection in Regional Ordinance: Ontology-based
  Validation.\BBCQ\
\newblock In {\Bem Legal Knowledge and Information Systems: JURIX 2006: The
  Nineteenth Annual Conference (Frontiers in Artificial Intelligence and
  Applications)}, \mbox{\BPGS\ 111--120}. IOS Press.

\bibitem[\protect\BCAY{Hagiwara \BBA\ Tojo}{Hagiwara \BBA\
  Tojo}{2006b}]{hagi06}
Hagiwara, S.\BBACOMMA\ \BBA\ Tojo, S. \BBOP 2006b\BBCP.
\newblock \BBOQ Stable Legal Knowledge with Regard to Contradictory
  Arguments.\BBCQ\
\newblock In {\Bem AIA'06: Proceedings of the 24th IASTED international
  conference on Artificial intelligence and applications}, \mbox{\BPGS\
  323--328}\ Anaheim, CA, USA. ACTA Press.

\bibitem[\protect\BCAY{Hobbs, Stickel, Martin, \BBA\ Edwards}{Hobbs
  et~al.}{1988}]{hobbs88}
Hobbs, J.~R., Stickel, M., Martin, P., \BBA\ Edwards, D. \BBOP 1988\BBCP.
\newblock \BBOQ Interpretation as abduction.\BBCQ\
\newblock In {\Bem Proceedings of the 26th annual meeting on Association for
  Computational Linguistics}, \mbox{\BPGS\ 95--103}\ Morristown, NJ, USA.
  Association for Computational Linguistics.

\bibitem[\protect\BCAY{Katayama}{Katayama}{2005}]{katayama05}
Katayama, T. \BBOP 2005\BBCP.
\newblock \BBOQ The Current Status of the Art of the 21st {COE} Programs in the
  Information Sciences Field (2): Verifiable and Evolvable
  e-Society---Realization of Trustworthy e-Society by Computer Science---(in
  {J}apanese).\BBCQ\
\newblock {\Bem Information Processing Society of Japan Journal}, {\Bbf 46}
  (5), \mbox{\BPGS\ 515--521}.

\bibitem[\protect\BCAY{Katayama}{Katayama}{2007}]{katayama07}
Katayama, T. \BBOP 2007\BBCP.
\newblock \BBOQ Legal Engineering---An Engineering Approach to Laws in
  e-Society {\linebreak}Age---.\BBCQ\
\newblock In {\Bem Proc. of the 1st Intl. Workshop on JURISIN}.

\bibitem[\protect\BCAY{Katayama, Shimazu, Tojo, Futatsugi, \BBA\
  Ochimizu}{Katayama et~al.}{2008}]{katayama08}
Katayama, T., Shimazu, A., Tojo, S., Futatsugi, K., \BBA\ Ochimizu, K. \BBOP
  2008\BBCP.
\newblock \BBOQ e-{S}ociety and {L}egal {E}ngineering (in {J}apanese).\BBCQ\
\newblock {\Bem Journal of the Japanese Society for Artificial Intelligence},
  {\Bbf 23}  (4), \mbox{\BPGS\ 529--536}.

\bibitem[\protect\BCAY{Kimura, Nakamura, \BBA\ Shimazu}{Kimura
  et~al.}{2009}]{nakamura08jurisin_lnai}
Kimura, Y., Nakamura, M., \BBA\ Shimazu, A. \BBOP 2009\BBCP.
\newblock \BBOQ Treatment of Legal Sentences Including Itemized and Referential
  Expressions---Towards Translation into Logical Forms---.\BBCQ\
\newblock In Hattori, H., Kawamura, T., Ide, T., Yokoo, M., \BBA\ Murakami,
  Y.\BEDS, {\Bem New Frontiers in Artificial Intelligence: JSAI2008 Conference
  and Workshops, Asahikawa, Japan, June 11--13, 2008, Revised Selected Papers},
  \lowercase{\BVOL}\ 5447 of {\Bem Lecture Notes in Artificial Intelligence},
  \mbox{\BPGS\ 242--253}. Springer.

\bibitem[\protect\BCAY{Kurohashi \BBA\ Nagao}{Kurohashi \BBA\
  Nagao}{1994}]{kurohashi94knp}
Kurohashi, S.\BBACOMMA\ \BBA\ Nagao, M. \BBOP 1994\BBCP.
\newblock \BBOQ {KN} Parser: {J}apanese Dependency/Case Structure
  Analyzer.\BBCQ\
\newblock In {\Bem Proceedings of the Workshop on Sharable Natural Language
  Resources}, \mbox{\BPGS\ 48--55}.

\bibitem[\protect\BCAY{Kurohashi, Nakamura, Matsumoto, \BBA\ Nagao}{Kurohashi
  et~al.}{1994}]{kurohashi94juman}
Kurohashi, S., Nakamura, T., Matsumoto, Y., \BBA\ Nagao, M. \BBOP 1994\BBCP.
\newblock \BBOQ Improvements of {J}apanese Morphological Analyzer
  {JUMAN}.\BBCQ\
\newblock In {\Bem Proceedings of the Workshop on Sharable Natural Language
  Resources}, \mbox{\BPGS\ 22--28}.

\bibitem[\protect\BCAY{Liu, Grossman, \BBA\ Zhai}{Liu et~al.}{2003}]{bing}
Liu, B., Grossman, R., \BBA\ Zhai, Y. \BBOP 2003\BBCP.
\newblock \BBOQ Mining Data Records in Web Pages.\BBCQ\
\newblock In {\Bem Proceedings of the ninth ACM SIGKDD}, \mbox{\BPGS\
  601--606}.

\bibitem[\protect\BCAY{Mulkar, Hobbs, \BBA\ Hovy}{Mulkar
  et~al.}{2007a}]{mulkar07learning}
Mulkar, R., Hobbs, J.~R., \BBA\ Hovy, E. \BBOP 2007a\BBCP.
\newblock \BBOQ Learning from Reading Syntactically Complex Biology
  Texts.\BBCQ\
\newblock In {\Bem Proceedings of the 8th International Symposium on Logical
  Formalizations of Commonsense Reasoning, part of the AAAI Spring Symposium
  Series}.

\bibitem[\protect\BCAY{Mulkar, Hobbs, Hovy, Chalupsky, \BBA\ Lin}{Mulkar
  et~al.}{2007b}]{mulkar07ijcai}
Mulkar, R., Hobbs, J.~R., Hovy, E., Chalupsky, H., \BBA\ Lin, C.-Y. \BBOP
  2007b\BBCP.
\newblock \BBOQ Learning by Reading: Two Experiments.\BBCQ\
\newblock In {\Bem Proceedings of IJCAI 2007 workshop on Knowledge and
  Reasoning for Answering Questions}.

\bibitem[\protect\BCAY{Nagai, Nakamura, \BBA\ Nomura}{Nagai
  et~al.}{1995}]{nagai95}
Nagai, H., Nakamura, T., \BBA\ Nomura, H. \BBOP 1995\BBCP.
\newblock \BBOQ Skeleton Structure Acquisition of {J}apanese Law Sentences
  based on Linguistic Characteristics.\BBCQ\
\newblock In {\Bem Proceedings of NLPRS'95, Vol. 1}, \mbox{\BPGS\ 143--148}.

\bibitem[\protect\BCAY{Nagano}{Nagano}{2005}]{nagano05}
Nagano, H. \BBOP 2005\BBCP.
\newblock \BBOQ {F}oundation and {C}ommon sence for legislation (in
  {J}apanese).\BBCQ\
\newblock {\Bem Jichitai Houmu Kenkyuu}, {\Bbf \protect{\mdseries winter}}.

\bibitem[\protect\BCAY{Nakamura, Nobuoka, \BBA\ Shimazu}{Nakamura
  et~al.}{2008}]{nakamura07jurisin}
Nakamura, M., Nobuoka, S., \BBA\ Shimazu, A. \BBOP 2008\BBCP.
\newblock \BBOQ Towards Translation of Legal Sentences into Logical
  Forms.\BBCQ\
\newblock In Satoh, K., Inokuchi, A., Nagao, K., \BBA\ Kawamura, T.\BEDS, {\Bem
  New Frontiers in Artificial Intelligence: JSAI 2007 Conference and Workshops,
  Miyazaki, Japan, June 18--22, 2007, Revised Selected Papers},
  \lowercase{\BVOL}\ 4914 of {\Bem Lecture Notes in Artificial Intelligence},
  \mbox{\BPGS\ 349--362}. Springer.

\bibitem[\protect\BCAY{Tanaka, Kawazoe, \BBA\ Narita}{Tanaka
  et~al.}{1993}]{tanaka93}
Tanaka, K., Kawazoe, I., \BBA\ Narita, H. \BBOP 1993\BBCP.
\newblock \BBOQ Standard Structure of Legal Provisions---For The Legal
  Knowledge Processing by Natural Language--- (In {J}apanese).\BBCQ\
\newblock In {\Bem IPSJ Research Report on Natural Language Processing},
  \mbox{\BPGS\ 79--86}.

\end{thebibliography}



\begin{biography}

\bioauthor[:]{Makoto Nakamura}{
 received the Bachelor degree in Information Engineering from Kyushu
 Institute of Technology in 1995.  He received the Master and Doctoral
 degrees in Information Science from JAIST in 1997 and 2004,
 respectively.  He is now an assistant professor at School of
 Information Science, JAIST.  His research interests include
 Legal Text Processing, and Simulation of Language Evolution.
}

\bioauthor[:]{Yusuke Kimura}{
received the Bachelor degree in Engineering from Osaka
University in 2005. He received the Master degree in Information
Science from JAIST in 2008. He now works at Pixela Corporation as a
software engineer.
}

\bioauthor[:]{Minh Quang Nhat Pham}{
 is now a Ph.D student at Natural Language
 Processing Laboratory, School of Information Science, Japan Advanced
 Institute of Science and Technology (JAIST). He  received the  B.S
 degree of  Information Technology from Vietnam National University of
 Hanoi (VNUH) in 2006, and received M.S degree of Information Science
 from JAIST in 2010. His research interests include Machine Learning,
 Text Generation, and Legal Text Processing.
}

\bioauthor[:]{Le Minh Nguyen}{
received the BS degree in information technology from Hanoi University
 of Science, and the MS degree in information technology from Vietnam
 National University, Hanoi in 1998 and 2001, respectively. He received
 the Ph.D in information science from Graduate School of Information
 Science, Japan Advanced Institute of Science and Technology (JAIST) in
 2004. He is now an assistant professor at Graduate School of
 Information Science, JAIST. His research interests include text
 summarization, machine translation, natural language processing,
 machine learning, and information retrieval.
}

\bioauthor[:]{Akira Shimazu}{
 received the Bachelor and Master degrees in mathematics from Kyushu
 University in 1971 and 1973, respectively, and the Doctoral degree in
 Natural Language Processing from Kyushu University in 1991.  From 1973
 to 1997, he worked at Musashino Electrical Communication Laboratories
 of Nippon Telegram and Telephone Public Corporation, and at Basic
 Research Laboratories of Nippon Telegraph and Telephone Corporation.
 From 2002 to 2004, he was the president of the Association for Natural
 Language Processing.  He has been a professor in the Graduate school of
 Information Science, JAIST since 1997.
}

\end{biography}

\biodate



\end{document}
