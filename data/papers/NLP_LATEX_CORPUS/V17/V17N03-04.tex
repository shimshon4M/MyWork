    \documentclass[english]{jnlp_1.4}

\usepackage{jnlpbbl_1.2}
\usepackage[dvips]{graphicx}
\usepackage[noreplace,multi]{otf}
\makeatletter
\renewcommand{\thesubsubsection}{}
\renewcommand{\subsubsection}{}
\makeatother


\Volume{17}
\Number{3}
\Month{April}
\Year{2010}


\received{2009}{4}{30}
\accepted{2009}{9}{2}

\setcounter{page}{61}

\etitle{Comparison of Chinese Treebanks for Corpus-oriented HPSG Grammar Development}
\eauthor{Kun Yu\affiref{univTokyo} \and Yusuke Miyao\affiref{univTokyo} 
	\and Takuya Matsuzaki\affiref{univTokyo} \and Xiangli Wang\affiref{univTokyo} 
	\and Yaozhong Zhang\affiref{univTokyo} \and \\
	Kiyotaka Uchimoto\affiref{InfoComm} 
	\and Junichi Tsujii\affiref{univTokyo}\affiref{univManch}} 
\eabstract{
Comparing with the traditional way of manually developing grammar based on 
linguistic theory, corpus-oriented grammar development is more promising. To 
develop HPSG grammar through the corpus-oriented way, a treebank is an 
indispensable part. This paper first compares existing Chinese treebanks and 
chooses one of them as the basic resource for HPSG grammar development. Then 
it proposes a new design of part-of-speech tags based on the assumption that 
it is not only simple enough to reduce ambiguity of morphological analysis 
as much as possible, but also rich enough for HPSG grammar development. 
Finally, it introduces some on-going work about utilizing a Chinese 
scientific paper treebank in HPSG grammar development.
}
\ekeywords{Chinese treebank, HPSG, part-of-speech tag}

\headauthor{Yu et al.}
\headtitle{Comparison of Chinese Treebanks for Corpus-oriented HPSG Grammar Development}

\affilabel{univTokyo}{}{Graduate School of Information Science and Technology, The University of Tokyo}
\affilabel{InfoComm}{}{National Institute of Information and Communications Technology}
\affilabel{univManch}{}{School of Computer Science, University of Manchester}


\begin{document}

\maketitle

\section{Introduction}

As the most popular way for grammar development, manually writing grammar 
relies on the linguistic intuition of grammar writers; however, this 
technique is hard because the developer need to resort to all the possible 
resources to support the grammar development. Recently, a new strategy, 
which is called as `corpus-oriented grammar development', has been proposed 
to automatically acquire grammar from an annotated corpus (Miyao 2006; Guo 
2009; Xia 1999; Chen and Shanker 2000; Hockenmaier and Steedman 2002; Chiang 
2000; Cramer and Zhang 2009). This strategy first externalizes the 
linguistic intuition as annotations to a corpus. Then a system of linguistic 
entities (i.e. a grammar) is automatically induced which obeys a linguistic 
theory that could explain the given annotations (Miyao et al. 2005). 

The corpus-oriented grammar development strategy is adequate for lexicalized 
grammar acquisition, such as Head-driven Phrase Structure Grammar (HPSG) 
(Pollard and Sag 1994), Lexical Functional Grammar (LFG) (Dalrymple et al. 
1995), Combinatory Categorial Grammar (CCG) (Steedman 2000), and Lexicalized 
Tree Adjoining Grammar (LTAG) (Donovan et al. 2005). Because lexicalized 
grammars are formulated with a small number of grammar rules and a large 
lexicon. With manually written grammar rules, the lexicon can be 
automatically obtained (Miyao et al. 2005). For example, Miyao (2006) 
automatically acquired an English HPSG grammar from the Penn Treebank; Guo 
(2009) obtained wide-coverage LFG resources for Chinese from the Penn 
Chinese Treebank; and Cramer and Zhang (2009) constructed a German HPSG 
grammar from the Tiger Treebank. 

To our current knowledge, previous works about developing Chinese HPSG 
grammar were fulfilled by hand (Zhang 2004; Li 1997; Wang et al. 2009). To 
develop HPSG grammar through corpus-oriented strategy, an indispensable part 
is the annotated corpus (i.e. a treebank). Thus, in this paper, we first 
compare the word segmentation specification, part-of-speech (POS) tags, and 
syntactic bracketing criteria of the existing Chinese treebanks. After that, 
we select one of them as the basic annotated corpus and propose a new POS 
tag design for corpus-oriented HPSG grammar development. The new POS tag 
design is based on the assumption that it is not only simple enough to 
reduce ambiguity of morphological analysis as much as possible, but also 
rich enough for HPSG grammar development. At last, we introduce some 
on-going work about utilizing a Chinese scientific paper treebank in HPSG 
grammar development.



\section{Comparison of Chinese Treebanks}

\subsection{Statistics}

There exist several treebanks for simplified Chinese, in which the Peking 
University Treebank (PKU), the Tsinghua University Treebank (TSU), and the 
Penn Chinese Treebank (CTB) are the most popular ones. 

Table 1 lists the statistical data of these treebanks (Computational 
Linguistics Lab 2001; Jin et al. 2003; Yu et al. 2002; Zhou 2004; PKU 
Treebank). There are several differences among them:


\subsubsection{The resource of the treebanks are different} 

CTB contains sentences from newspapers. But PKU and TSU are balanced 
corpora, including news, governmental documentation, academic documentation 
etc.

\subsubsection{The size of the treebanks differs a lot}

TSU contains the largest number of sentences and hanzi. PKU has the smallest 
number of sentences and words. 

\subsubsection{The POS tag definition of the treebanks are different}

Among these treebanks, CTB has the simplest POS tag design, including 33 POS 
tags. While, both PKU and TSU introduce detailed definition of POS tags 
besides of the basic POS tags. 

\begin{table}[t]
\caption{Statistics of Chinese Treebanks.}
\input{04table01.txt}
\end{table}

\subsubsection{The phrase tag definition of the treebanks are also different}

The numbers of basic phrases of the three treebanks are similar. But PKU 
contains 7 detailed phrase tags. Besides, both CTB and TSU annotate the 
shallow semantic information by using functional tag and relational tag, 
respectively. Furthermore, CTB tags 6 types of verb compounds and 7 types of 
empty categories, including trace and dropped element, and so on. 



\subsection{Comparison of Word Segmentation Criteria}

\noindent
\textbullet\ \textbf{Comparison of word definition}

A \textit{word} in CTB is a syntactic atom, which is anything that can be inserted into 
an $X^{0}$ position in syntax (Xia 2000(a)). It includes compounds and simple 
words (Xia 2000(a)). In PKU, a \textit{word} means the basic unit that is used in 
information processing and has specified semantic and syntactic function (Yu 
et al. 2002). It also includes simple words and compounds (Yu et al. 2002). 
Besides, in both CTB and PKU, it is possible that some isolated morphemes 
and non-morphemes are treated as words (Xia 2000(a); Yu et al. 2002). 

\noindent
\textbullet\ \textbf{Differences of word segmentation standards}

There are several differences between the word segmentation standard of CTB 
and PKU.


\subsubsection{Verb compound}

One difference is for verb compound (Xia 2000(a)). For example, in CTB, the 
string `\UTFC{8D70}\UTFC{51FA}\UTFC{6765} (walk out)' is treated as two 
segments `\UTFC{8D70} (walk)' and `\UTFC{51FA}\UTFC{6765} (out)' in the word 
segmentation phase and marked as a verb resultative compound in the 
syntactic bracketing phase. But in PKU, the string is segmented into two 
parts and not grouped together as a compound (Institute of Computational 
Linguistics of Peking University1999; Yu et al. 2002).

\subsubsection{Number}

In CTB, when a cardinal number is before a morpheme such as 
`\UTFC{4F59} (odd)', they are segmented as one word (e.g. 
`\UTFC{56DB}\UTFC{767E}\UTFC{4F59} (around 400)') (Xia 2000(a)). But in PKU, the 
cardinal number and following morpheme are segmented as two words, such as 
`\UTFC{4E09}\UTFC{5341} (thirty) \UTFC{591A} (about)' (Yu et al. 2002). 
For ordinal number, which is normally expressed by adding a prefix 
`\UTFC{7B2C}' before a cardinal number like `\UTFC{7B2C}\UTFC{5341}\UTFC{4E09} (the 13th)', it is treated as one word in both CTB and PKU (Xia 
2000(a); Yu et al. 2002).

\subsubsection{Abbreviation}

In CTB, abbreviation is usually treated as one word. But when it is longer 
than 3 syllables, it will be segmented according to phonologic evidence (Xia 
2000(a)). While, in PKU abbreviation is always treated as one word if only 
it represents a whole concept (Yu et al. 2002).

\subsubsection{Prefix and suffix}

Both CTB and PKU do not segment a prefix as a word (Xia 2000(a); Yu et al., 
2002). But for suffix representing the plural form of noun (i.e. `\CIDC{2764}'), 
the standard of CTB and that of PKU treat it differently. CTB does not 
segment the suffix `\CIDC{2764}' from its previous string like 
`\UTFC{5B66}\UTFC{751F}\CIDC{2764} (students)' (Xia 2000(a)). While PKU segments 
`\UTFC{5B66}\UTFC{751F} (student)' and `\CIDC{2764}' into two words (Yu et al. 
2002).

\subsubsection{Reduplication of AA, ABAB, AABB, AAB, AB, and ABAC}

All these types of reduplications are treated as one word in CTB (Xia 
2000(a)). But in PKU (Yu et al. 2002), the reduplication of ABAB are 
segmented into two words, such as `\UTFC{7814}\UTFC{7A76} (research) 
\UTFC{7814}\UTFC{7A76} (research)'.

\subsubsection{Reduplication of AA-kan, A-one-A, A-le-one-A, A-le-A, A-not-A, and reduced A-not-A}

Except for reduced A-not-A (e.g. \UTFC{559C}\UTFC{4E0D}\UTFC{559C}\CIDC{1995} (like or not)), other types of 
reduplications are segmented into words in both CTB and PKU (Xia 2000(a); Yu 
et al. 2002). For the type of reduced A-not-A, CTB segments it into several 
words, such as `\UTFC{76F8} (believe) \UTFC{4E0D} (not) 
\UTFC{76F8}\UTFC{4FE1}(believe)'. But PKU treats it as one word like 
`\UTFC{76F8}\UTFC{4E0D}\UTFC{76F8}\UTFC{4FE1}(believe or not)'.


\subsection{Comparison of POS Tags}


\noindent
\textbullet\ \textbf{Comparison of POS tagging criteria}

There exist two different viewpoints about POS tagging criteria: tagging 
based on meaning and tagging based on syntactic distribution (Xia 2000(b); 
Gong 1997). For example, for a word `\UTFC{6E38}\UTFC{6CF3} (swim)', it 
can be used as both noun and verb. From the first point-of-view, the word 
`\UTFC{6E38}\UTFC{6CF3} (swim)' should be tagged as verb in both the two 
sentences `\UTFC{6211}\UTFC{5728}\UTFC{6E56}\UTFC{91CC}\UTFC{6E38}\UTFC{6CF3} (I swim in the lake)' and 
`\UTFC{6211}\UTFC{559C}\CIDC{1995}\UTFC{6E38}\UTFC{6CF3} (I like 
swimming)'. The POS tag design of PKU and TSU is based on this (i.e. 
meaning). But in CTB, POS tagging is based on the second point-of-view (i.e. 
syntactic distribution). Therefore, the word 
`\UTFC{6E38}\UTFC{6CF3} (swim)' is tagged as verb in the first sentence 
but tagged as noun in the second sentence. 


\noindent
\textbullet\ \textbf{Comparison of POS tag definition}

The POS tag definitions of the three treebanks are different from each 
other. In the following, we first briefly introduce the POS tag definition 
of CTB. Then we compare it with the POS tag definition of PKU. Because the 
POS tag definition of TSU is more similar to that of PKU, we make a 
comparison between the POS tag design of PKU and TSU finally.


\noindent
\textendash\ \textit{POS tag definition of CTB}

33 POS tags are defined in CTB (see Table 2) (Xia 2000(b)). There are some 
special treatments in this definition, which includes:


\begin{table}[b]
\caption{POS Tags of Penn Chinese Treebank.}
\input{04table02.txt}
\end{table}


\subsubsection{Ba-construction}

Ba-construction moves the object of a verb to the pre-verbal position. For 
example, in the sentence `\UTFC{6211}/PN \UTFC{628A}/BA \CIDC{2762}/NN 
\UTFC{6253}\CIDC{2359}/VV (I open the door)'. The word `\CIDC{2762} (door)' is the 
object of verb `\UTFC{6253}\CIDC{2359} (open)'. But it is moved before 
`\UTFC{6253}\CIDC{2359} (open)' by using `\UTFC{628A} (ba)'. There are three ways to 
tag this word `\UTFC{628A} (ba)': as a verb (Huang 1991; Bender 2000), as a 
preposition (Gao 1992), and as a case-marker (Gao 2000). The tagging 
criterion used in CTB is similar to the first one: as a verb that forms verb 
phrase and takes subject and object. But in order to differ them from other 
verbs, a tag BA is defined for `\UTFC{628A} (ba)' and `\UTFC{5C06} (jiang)' when 
they occur in the ba-construction.

\subsubsection{Bei-construction}

Bei-construction introduces the passive voice of a verb. In CTB, there exist 
two types of bei-constructions that take different complements. The first 
one is long bei-construction, for which the syntactic pattern is `NP0 + bei 
+ NP1 + VP' (Xia 2000(b)). In long bei-construction, the word 
`\UTFC{88AB} (bei)' (LB) is treated as a verb, which takes `NP0' as a 
subject and `NP1 + VP' as a sentential complement (Xue and Xia 2000). For 
example, in the sentence `\CIDC{2762}/NN \UTFC{88AB}/LB 
\UTFC{4ED6}/PN \CIDC{1841}\UTFC{4E0A}/VV 
(The door is closed by him)', `\CIDC{2762} (door)' is the subject of 
`\UTFC{88AB} (bei)', and `\UTFC{4ED6} (him) 
\CIDC{1841}\UTFC{4E0A} (close)' is a sentential complement of 
`\UTFC{88AB} (bei)'. The second one is short bei-construction (Xia 
2000(b)). Its syntactic pattern is `NP0 + bei + VP' (Xia 2000(b)). In short 
bei-construction, `\UTFC{88AB} (bei)' (SB) is treated on a par with modal 
verbs in which it takes a VP complement. An example sentence is 
`\UTFC{5DEE}\UTFC{8DDD}/NN \UTFC{88AB}/SB \CIDC{4409}\UTFC{5927}/VV 
(The difference is enlarged)'. In this sentence, the verb phrase 
`\CIDC{4409}\UTFC{5927} (enlarge)' is the complement of 
`\UTFC{88AB} (bei)'.


\subsubsection{Particle \UTFC{7684} (de)}

In CTB, the particle `\UTFC{7684} (de)' is assigned with different tags 
according to different functions. When it functions as a complementizer or a 
nominalizer, it is tagged as DEC, for example `\CIDC{4483}/DT \UTFC{662F}/VC \UTFC{6211}/PN \CIDC{2713}/VV \UTFC{7684}/DEC \CIDC{3456}/NN (This is the book 
that I buy)'. But when it acts as a genitive marker or an associative 
marker, it is tagged as DEG, such as `\UTFC{7F8E}\CIDC{2529}/JJ \UTFC{7684}/DEG 
\UTFC{6751}\UTFC{5E84}/NN (beautiful village)'. Such division introduces 
syntactic information into POS tag definition, but increase the ambiguity of 
morphological analysis.


\subsubsection{Predictive adjective}

There are some adjectives that can act as predicate. For example, in the 
sentence `\CIDC{4483}/DT \CIDC{1571}/M \UTFC{82B1}/NN \UTFC{5F88}/AD 
\UTFC{6F02}\UTFC{4EAE}/VA (This flower is very beautiful)', the adjective 
`\UTFC{6F02}\UTFC{4EAE}(beautiful)' is the predicate of this sentence. CTB 
tags these words as predicative adjective (VA). It includes two types (Xia 
2000(b)): $(a)$ predicates that have no object and can be modified by 
`\UTFC{5F88}(very)'; $(b)$ predicates derived from the first type through 
reduplication or the pattern `N + A' meaning `as A as an N'. 


\noindent
\textendash\ \textit{Comparison of POS tag definition of CTB and PKU}

The POS tag definition of PKU is based on meaning. Thus besides of 26 basic 
POS tags, there are 72 detailed POS tags defined in PKU (see Table 3) (PKU 
Treebank). Following are the differences between the POS tag definition of 
CTB and that of PKU.

\begin{table}[t]
\caption{POS Tags of Peking University Treebank.} 
\input{04table03.txt}
\end{table}


\setcounter{subsubsection}{0}
\subsubsection{Division of proper noun}

In CTB, there is only one tag (NR) for proper noun. But in PKU, proper noun 
is divided into four types according to the different named entities it 
belongs to: personal proper noun (nr), location proper noun (ns), 
organization proper noun (nt), and other proper noun (nz). 

\subsubsection{Division of pronoun}

Likewise, only one tag (PN) is defined for pronoun in CTB. While, there 
exist one basic tag (r) and seven detailed tags for pronoun in PKU. This 
division is based on the function of a pronoun, such as demonstrative 
pronoun (rb) and temporal pronoun (rt).

\subsubsection{Classification of verb}

Verbs in CTB are classified into four types based on their syntactic 
distribution. But in PKU, besides of one basic tag (v), verbs are assigned 
with two detailed tags according to their meanings, which are verb as 
adverbial (vd) and verb as noun (vn).

\subsubsection{Classification of conjunction}

In CTB, conjunction is divided into two types based on its syntactic 
function: coordination conjunction (CC) and subordinating conjunction (CS). 
But in PKU, there are two detailed types for conjunction defined by 
position, which are pre-conjunction (ch) and post-conjunction (ck).

\subsubsection{Classification of adjective}

Adjectives in CTB are classified into adjective as predicate (VA) and 
adjective as noun modifier (JJ) according to their functions. But in PKU, 
adjectives are assigned with two detailed tags according to the meaning of 
words, which are adjective as adverbial (ad) and adjective as noun (an).

\subsubsection{Ba-construction and Bei-construction}

Another obvious distinction between the POS tag definition of CTB and that 
of PKU is the tagging for ba-construction and bei-construction. In CTB, both 
the two constructions are treated as verbs and have their own tags, 
including BA, LB, and SB. But in PKU, both `\UTFC{628A} (ba)' and 
`\UTFC{88AB} (bei)' are treated as preposition.

\subsubsection{Particle \UTFC{7684} (de)}

The treatment of particle `\UTFC{7684} (de)' is also a distinct difference 
between the two treebanks. In CTB, the tag of this particle is divided into 
two types (DEC and DEG) according to its function. In PKU, only one tag 
(ude1) is assigned to this particle.

\subsubsection{Division of punctuation}

PKU divides punctuations into 22 detailed types, which includes almost all 
the punctuations in common usage. In CTB, all the punctuations are assigned 
with one tag PU.

\subsubsection{Annotation of additive word}

PKU also adds several new POS tags. Parts of them are tags for additive 
words, such as morpheme (g), non-morpheme (x), prefix (h), suffix (k) and 
their detailed types.

\subsubsection{Other new POS tags }

Idiom (i), locution (l), and abbreviation (j) and their detailed types are 
the other parts of new POS tags in PKU. In CTB, idioms and abbreviations 
that are no longer than 3 syllables are also segmented as words, but they 
are tagged as verbs or nouns according to their functions. Besides of that, 
PKU annotates place (s), mood (y) and mode (z), which are not defined in the 
POS tag set of CTB.


\noindent
\textendash\ \textit{Comparison of POS tag definition of PKU and TSU}

The POS tag definition of TSU is also based on meaning, including 24 basic 
tags, 27 detailed tags, and tags for punctuations (see Table 4) (Tsinghua 
National Lab for Information Science and Technology). This design is very 
similar to that of PKU, except for the following differences:

\begin{table}[t]
\caption{POS Tags of Tsinghua University Treebank.}
\input{04table04.txt}
\end{table}



\setcounter{subsubsection}{0}
\subsubsection{Division of adverb}

In PKU, there is only one tag for adverb. But in TSU, adverbs are divided 
into three types by their functions, including negative pre-adverb (dB), 
negative adverb (dN), and degree adverb (dD).

\subsubsection{Division of classifier}

Not like PKU, TSU divides classifiers (q) into four detailed types according 
to the type of words that they can be connected to. These detailed types 
include noun classifier (qN), verb classifier (qV), temporal classifier 
(qT), and compound classifier (qC). 

\subsubsection{Division of verb}

The division of verb in TSU is similar to that of PKU. Besides of the 
detailed tags designed for verb in PKU, TSU introduces copula (vC) (like VC 
in CTB) and verb as complement (vB). In addition, TSU defines two detailed 
tags for verb based on valence, which are divalent verb (vSB) and pivotal 
verb (vJY).

\subsubsection{Deletion of tags}

Some tags that exist in the definition of PKU are deleted from that of TSU. 
These tags include adjective as adverbial (ad), morpheme (g) and its 
detailed tags, non-morpheme (x), locution (l) and its detailed tags, and 
detailed tags for particles.

\subsubsection{Some new tags}

In TSU, some new tags are added, including connection (l), string (x), and 
hanzi (g). 

\subsection{Comparison of Syntactic Bracketing Criteria}

The syntactic bracketing criteria of the three treebanks are also different 
from each other. 

\noindent
\textbullet\ \textbf{Syntactic Bracketing Criterion of CTB}

The syntactic bracketing criterion of CTB makes compromise between 
linguistic correctness and engineering convenience (Xue and Xia 2000) and 
has good consistency. The phrase structure of CTB is quite flat. For 
example, in the sentence in Figure 1, the nouns 
`\UTFC{5916}\UTFC{5546} (foreign businessman)', `\UTFC{6295}\CIDC{4647} (investment)', 
`\UTFC{4F01}\CIDC{4156} (enterprise)' are parallel. 

\begin{figure}[b]
\begin{center}
\includegraphics{17-3ia4f1.eps}
\end{center}
\caption{An example sentence from CTB.} 
\begin{center}
(\textit{The foreign investment enterprises become an important growth pole of Chinese foreign trade}).
\end{center}
\end{figure}

Besides of phrases, CTB annotates 26 functional tags to express shallow 
semantic information. For an instance, the tag OBJ in Figure 1 means the NP 
with this tag is the object of the predicate of this sentence. In addition, 
empty categories are annotated. These empty categories include the traces in 
topic construction, relative clause, ba-construction, and bei-construction; 
the null elements in control constructions; the pro-dropped subjects and 
objects; and the null elements occurring when two coordinated verbs share a 
complement (Xue and Xia 2000).


\noindent
\textbullet\ \textbf{Syntactic bracketing Criterion of PKU}

In PKU, the phrase structure is still flat. But some constructions such as 
coordination are annotated hierarchically. Figure 2 (a) shows an example of 
a coordination in PKU and Figure 2 (b) shows the corresponding annotation in 
CTB. Compared with the annotation in CTB, the coordination annotated by PKU 
specification is more hierarchical. Besides, unlike CTB, PKU annotates head 
word and head phrase using the `!' mark (PKU Treebank).


\begin{figure}[t]
\begin{center}
\includegraphics{17-3ia4f2.eps}
\end{center}
\caption{An example sentence with coordination (\textit{constantly, rapidly, healthily develop}).} 
\end{figure}


\noindent
\textbullet\ \textbf{Syntactic bracketing Criterion of TSU}

The phrase structure annotation of TSU is the most hierarchical one among 
the three treebanks. Simple sentences and complex sentences are also 
distinguished in TSU (Zhou 2004). In addition, similar to CTB, TSU uses 27 
relational tags to annotate the syntactic structure of phrases (Zhou 2004), 
which could be looked as shallow semantic information of sentences. For 
example, ZW means the phrase has subject-predicate structure, and PO means 
the phrase has predicate-complement structure.


\section{A Chinese Part-of-speech Tag Design for HPSG Grammar Development}

\subsection{Corpus-oriented HPSG Grammar Development}

HPSG (Pollard and Sag 1994) is a linguistic theory based on lexicalized 
grammar formalism. It consists of a small number of schemata that explain 
general grammatical constraints and a large number of lexical entries that 
express word-specific characteristics (Miyao et al. 2005). Typed feature 
structure is used to represent both the two elements. The constraints are 
checked by unification (Pollard and Sag 1994). 

As introduced in Section 1, we would like to develop Chinese HPSG grammar 
through corpus-oriented way. Figure 3 shows the framework of our 
corpus-oriented HPSG grammar development. With manually defined linguistic 
principles and POS tag, a treebank is annotated to be derivation trees (i.e. 
an HPSG treebank) first. Then, both a large lexicon and some super tags, 
which are some lexical templates shared among lexical entries for different 
words (Matsuzaki 2006), are extracted automatically. The extracted lexicon 
and the designed linguistic principles formulate an HPSG grammar. The super 
tags, such as the lexical template of transitive verb, introduce rich 
syntactic information that cannot be represented by POS tags. 

\begin{figure}[b]
\begin{center}
\includegraphics{17-3ia4f3.eps}
\end{center}
\caption{Framework of Chinese corpus-oriented HPSG grammar development.} 
\end{figure}


The first thing that we need to do in this framework is defining linguistic 
principles and POS tags for HPSG treebank acquisition. The linguistic 
principle means a type hierarchy of HPSG signs, ID schemas and principles 
that are regulated by the theory of HPSG. The design of ID schemas and 
principles basically follows the definition by Pollard and Sag (1994). The 
HPSG sign is defined in Figure 4. It shows the syntactic and semantic 
information of words and phrases. \textit{HEAD} feature represents the syntactic category 
of head word. \textit{MOD}, \textit{SUBJ}, \textit{SPR}, \textit{SPEC}, \textit{COMPS}, and \textit{CONJ} features show the constraints of modifiee, 
subject, specifier, specifiee, complement, and conjunction. \textit{MARKING} feature 
represents if the word or phrase is marked by a marker. \textit{CONT} is used to 
represent the semantic information of a word or phrase. \textit{GAP} and \textit{STOPGAP} features 
explain the trace of movements. 


\begin{figure}[t]
\begin{center}
\includegraphics{17-3ia4f4.eps}
\end{center}
\caption{Definition of a Chinese HPSG sign.} 
\end{figure}


\subsection{A Chinese POS Tag Design for HPSG Grammar Development}

Compared with the other two treebanks, CTB has good consistency. 
Furthermore, it annotates functional tags to express shallow semantic 
information and empty categories to show traces in specific constructions. 
Therefore, in our Chinese HPSG grammar development, we choose CTB as the 
basic treebank. 

The linguistic principles are already defined in Section 3.1. Therefore, we 
need to design a proper set of POS tag to develop HPSG grammar from 
treebank. Previous comparison about POS tag definition of the three 
treebanks shows that PKU and TSU encode more detailed information than CTB. 
But this detailed information can be easily induced both from an HPSG 
treebank as super tags and from other tools such as recognizing organization 
proper noun by a named entity recognizer. Therefore, we do not need to 
encode a lot of information in our POS tag definition in order to reduce the 
ambiguity of morphological analysis as much as possible. Based on this 
precondition, we make a new design of Chinese POS tag based on the POS tag 
definition of CTB for our HPSG grammar development. In this new design, some 
POS tags are merged together because the information they introduce can be 
easily acquired from the constructed HPSG treebank. Besides, some new POS 
tags are added because they provide essential information for HPSG treebank 
acquisition.

Table 5 lists the POS tags defined in this new design. There are some 
differences between the new design and the POS tags in CTB. We will explain 
each difference with the corresponding reason in detail.

\begin{table}[t]
\caption{Design of Chinese POS tags for HPSG grammar development.}
\input{04table05.txt}
\end{table}



\subsubsection{Classification of verb}

In the new design, we do not distinguish the verb `\UTFC{6709} (have)' as what 
is done in CTB because it has similar function as other transitive verbs. 
But we define two tags for auxiliary verb (VX) and tendency verb (VT). The 
reason is that auxiliary verbs always take a verbal complement, and tendency 
verbs could be treated as complement of verbal predicate.

\subsubsection{Tags for marker}

Similar to CTB, we define special POS tags for `\UTFC{628A} (ba)' in 
ba-construction and `\UTFC{88AB} (bei)' in bei-construction. But we treat 
them as marker. The design of this category corresponds to the design of 
MARKING feature in the HPSG sign (see Figure 4). This feature is defined 
based on the assumption that all the nouns (except for subject and object) 
must be marked by prepositions or markers (Li, 1985; Travis, 1984). In 
addition, we do not distinguish short-bei construction and long-bei 
construction as what is done in CTB. But we keep the trace in 
bei-construction annotated in CTB.


\subsubsection{Combination of particle}

In CTB, the particle \UTFC{7684} (de) is assigned with two tags (DEC and DEG) 
when it has different functions. Such division is useful for obtaining 
correct syntactic structure but introduces ambiguity to morphological 
analysis. Because the syntactic information brought by this division could 
be acquired from an HPSG treebank, we combine the two tags (DEC and DEG) in 
CTB as one tag (NPA) in our design.

Besides, one difference between the two particles ETC and MSP in CTB is 
whether it appears before a VP. It could also be induced from the 
constructed HPSG treebank. Therefore, we merge the two tags as one tag OPA 
in the new design.

\subsubsection{Division of punctuation }

Because semi-comma functions as coordinating conjunction in Chinese, in our 
new design we divide punctuations into two types: semi-comma (SPU) and other 
punctuations (OPU). With this division, we could easily find coordination 
structures connected by semi-commas.



\section{Applying A Chinese Scientific Paper Treebank in HPSG Grammar Development}

To provide training data for a Chinese syntactic analyzer developed for a 
Japanese-Chinese scientific paper machine translation system, National 
Institute of Information and Communications Technology developed a Chinese 
scientific paper treebank. This treebank includes 8,000 Chinese sentences 
that are randomly selected from the manual translation of Japanese journal 
papers in the filed of information science. The annotation criteria are 
basically the same as that of Penn Chinese Treebank (Xia 2000(a); Xia 
2000(b); Xue and Xia 2000), except that this treebank removes the annotation 
of functional tags and empty categories for simplification. 

Considering that this treebank follows the annotation guideline of CTB and 
is composed of sentences from a specific domain (i.e. scientific paper), we 
would like to utilize this treebank in our Chinese HPSG grammar development, 
While, there are several works that need to be done beforehand.

\noindent
\textbullet\ \textbf{POS tag transformation}

In order to utilize this scientific paper treebank to acquire HPSG grammar, 
we first need to transform its POS tag definition to our new POS tag design. 
There are 31 POS tags in our new design, in which 23 POS tags have 
one-to-one correspondence with the POS tags in CTB and 8 POS tags (italic 
tags in Table 5) has no precise correspondence. Therefore, we use following 
rules to semi-automatically obtain the 8 POS tags from the POS tags of CTB.

\textit{Rule1: Transformation of verb }

Several steps are applied to determine the POS tag of a verb in the new 
design. (1) auxiliary verbs (VX) are listed by hand, including 
\UTFC{80FD} (can), \UTFC{80FD}\CIDC{1812} (can), 
\UTFC{53EF}\UTFC{4EE5}(can) and so on. (2) copula (VC in CTB) keeps the 
same tag. (3) if a verb is the second verb of a verb resultative compound 
(VRD) in CTB, it is tagged as tendency verb (VT). (4) the remaining verbs 
tagged as VV or VE in CTB are marked as VV.

\textit{Rule2: Classification of punctuation}

Semi-comma is tagged as SPU in the new design. Other punctuations are tagged 
as OPU. 

\textit{Rule3: Bei-construction }

If a word is tagged as LB or SB in CTB, it is tagged as BEI in the new 
design.

\textit{Rule4: Combination of particle}

DEC and DEG in CTB are combined as one POS tag NPA. MSP and ETC in CTB are 
also combined as one tag OPA.



\noindent
\textbullet\ \textbf{Functional tag annotation}

The scientific paper treebank does not annotate functional tags. There are 
26 functional tags in CTB, in which 7 tags mark grammatical roles (Xue and 
Xia 2000). Among these tags, SBJ (subject), OBJ (object), and IO (indirect 
object) show the arguments of a predicate. EXT (extent) marks post-verbal 
complements that describe the extent, frequency, or quantity of an activity. 
FOC (focus) marks an object fronted to a preverbal but post-subject 
position. PRD (predicate) represents non-verbal predicate. These tags are 
essential for HPSG grammar development to obtain shallow semantic 
information. Besides, TPC (topic) indicates the topic construction, which is 
a special construction in Chinese. Therefore, we manually annotate these 
functional tags before using the scientific paper treebank for HPSG grammar 
development. 


\noindent
\textbullet\ \textbf{Trace annotation}

Traces that are represented by empty categories in CTB are also important to 
get long distance dependency in HPSG grammar development. But the scientific 
paper treebank does not annotate this information. In CTB, traces exist in 
four constructions: relative clause, ba-construction, bei-construction, and 
topic construction. So we first extract the sentences with words tagged as 
DEC (relative clause), words tagged as LB or SB (bei-construction), and 
phrases tagged with TPC functional tag (topic construction). Then, we 
annotate these extracted sentences with the empty categories according to 
the guideline of CTB (Xue and Xia 2000) manually.



\section{Conclusion and Future Work}

Corpus-oriented grammar development works well for lexicalized grammar, such 
as HPSG, LFG, CCG, and LTAG. In this strategy, a suitable treebank is an 
indispensible part. This paper first compares the word segmentation 
criteria, the POS tag definition, and the syntactic bracketing criteria of 
three most popular Chinese treebanks: Penn Chinese Treebank, Peking 
University Treebank, and Tsinghua Univesity Treebank. Then, it chooses the 
Penn Chinese Treebank as the resource of HPSG grammar development because of 
its good consistency and rich information and then designs a new set of POS 
tags for HPSG grammar development. Finally, with this new definition of 
Chinese POS tag and the designed Chinese HPSG linguistic principles, this 
paper utilizes a Chinese scientific paper treebank to acquire Chinese HPSG 
grammar and introduces the on-going related work.

In the future, the linguistic principles for our corpus-oriented Chinese 
HPSG grammar development will be enriched. Subsequently, two HPSG treebanks 
that are based on CTB and the Chinese scientific paper treebank will be 
constructed. Through these HPSG treebanks, we could extract a large lexicon 
and a large set of super tags of words from both the news domain and science 
domain.

\bibliographystyle{jnlpbbl_1.4}
\begin{thebibliography}{}

\item
Bender, E. (2000). ``The Syntax of Madarin Ba: Reconsidering the Verbal 
Analysis.'' \textit{Journal of East Asian Linguistics}, \textbf{9} (2), pp. 105--145.

\item
Chen, J. and Shanker, V. (2004). ``Automated Extraction of TAGs from the 
Penn Treebank.'' In \textit{Proceedings of the 6}$^{th}$\textit{ IWPT}.

\item
Chiang, D. (2000). ``Statistical Parsing with an Automatically-extracted 
Tree Adjoining Grammar.'' In \textit{Proceedings of the 38}$^{th}$\textit{ ACL}. pp. 456--463.

\item
Cramer, B. and Zhang, Y. (2009). ``Construction of a German HPSG Grammar 
from a Detailed Treebank.'' In \textit{Proceedings of the 2009 Workshop on Grammar Engineering Across Frameworks, ACL-IJCNLP 2009}. pp. 37--45. 

\item
Chinese Knowledge Information Processing Group (1996). ``Shouwen Jiezi---a 
Study of Chinese Word Boundaries and Segmentation Standard for Information 
Processing.'' Technical Report. Taipei: Academia Sinica.

\item
Computational Linguistics Lab. (2001). ``Standardized Set of Chinese POS 
Markers for Computational Uses.'' \textit{Applied Linguistics}, 2001 (3).

\item
Dalrymple, M. et al. (1995). \textit{Formal Issues in Lexical-Functional Grammar}. 
Cambridge University Press, Stanford, CA.

\item
Donovan, R. et al. (2005). ``Large-Scale Induction and Evaluation of Lexical 
Resources from the Penn-II and Penn-III Treebanks.'' \textit{Computational Linguistics}, pp. 329--366.

\item
Gao, Q. (1992). ``Chinese Ba Construction: its Syntax and Semantics.'' 
Unpublished. The Ohio State University.

\item
Gao, Q. (2000). ``Argument Structure, HPSG and Chinese Grammar.'' PhD 
Thesis. The Ohio State University.

\item
Gong, Q. (1997). ``Zhongguo Yufaxue Shi (The History of Chinese Syntax).'' 
Yuwen Press.

\item
Guo. Y. (2009). ``Treebank-based acquisition of Chinese LFG Resources for 
Parsing and Generation.'' PhD Thesis. School of Computing, Dublin City 
University.

\item
Hockenmaier, J. and Steedman, M. (2002). ``Acquiring Compact Lexicalized 
Grammars from a Cleaner Treebank.'' In \textit{Proceedings of the 3}$^{rd}$\textit{ LREC}.

\item
Huang, C. (1991). ``Mandarin Chinese and the Lexical Mapping Theory: A Study 
of the Interaction of Morphology and Argument Changing.'' \textit{Bulletin of the Institute of History and Philosophy 62}.

\item
Institute of Computational Linguistics of Peking University (1999). ``The 
Processing of Contemporary Chinese Corpus Specification---Word Segmentation 
and POS Tagging.'' Technical Report.

\item
Jin, G. et al. (2003). ``Standardization for Corpus Processing.'' \textit{Applied Linguistics}, 2003 (4).

\item
Li, A. (1985). ``Abstract Case in Chinese.'' PhD Thesis. University of 
Southern California. 

\item
Li, W. (1997). ``Outline of an HPSG-style Chinese Reversible Grammar.'' In 
\textit{Proceedings of the 13}$^{th}$\textit{ North West Linguistics Conference}.

\item
Matsuzaki, T. (2006). ``Efficient HPSG Parsing with Supertagging and 
CFG-filtering.'' PhD Thesis. The University of Tokyo.

\item
Miyao, Y. (2006). ``From Linguistic Theory to Syntactic Analysis: 
Corpus-oriented Grammar Development and Feature Forest Model.'' PhD Thesis. 
The University of Tokyo.

\item
Miyao, Y. et al. (2005). ``Corpus-oriented Grammar Development for Acquiring 
a Head-driven Phrase Structure Grammar from the Penn Treebank.'' \textit{Natural Language Processing---\mbox{IJCNLP} 2004}. pp. 684--693.

\item
PKU Treebank. ``Web Page.'', 
\texttt{http://ccl.pku.edu.cn:8080/WebTreebank/WebTreebank{\_}Readme.\linebreak[2]html}.

\item
Pollard, C. and Sag, I.A. (1994). \textit{Head-driven Phrase Structure Grammar}. 
University of Chicago Press.

\item
Steedman, M. (2000). \textit{The Syntactic Process}. The MIT Press.

\item
Travis, L. (1984). ``A Topic-comment Approach to the Ba Construction.'' 
\textit{Journal of Chinese Linguistics}, pp. 1--55.

\item
Tsinghua National Lab for Information Science and Technology ``Hanyu Jiben 
Duanyu Biaozhu Guifan (The Specification of Chinese Basic Phrase 
Annotation).'' Unpublished.

\item
Wang, X. et al. (2009). ``Design of Chinese HPSG Framework for Data-driven 
Parsing.'' In \textit{Proceedings of the 23}$^{rd}$\textit{ Pacific Asia Conference on Language, Information and Computation}.

\item
Xia, F. (1999). ``Extracting Tree Adjoining Grammars from Bracketed 
Corpora.'' In \textit{Proceedings of the 5}$^{th}$\textit{ NLPRS}.

\item
Xia, F. (2000a). ``The Segmentation Guidelines for the Penn Chinese 
Treebank.'' Technical Report. University of Pennsylvania.

\item
Xia, F. (2000b). ``The Part-Of-Speech Tagging Guidelines for the Penn 
Chinese Treebank.'' Technical Report. University of Pennsylvania.

\item
Xue, N. and Xia, F. (2000). ``The Bracketing Guidelines for the Penn Chinese 
Treebank.'' Technical Report. University of Pennsylvania.

\item
Yu, S. et al. (2002). ``The Basic Processing of Contemporary Chinese Corpus 
at Peking University Specification.'' \textit{Journal of Chinese Information Processing}, \textbf{16} (5).

\item
Zhou, Q. (2004). ``Annotation Scheme for Chinese Treebank.'' \textit{Journal of Chinese Information Processing}, \textbf{18} (4).

\item
Zhang, Y. (2004). ``Starting to Implement Chinese Resource Grammar Using LKB 
and LinGO Grammar Matrix.'' Technical Report.

\end{thebibliography}



\begin{biography}

\bioauthor[:]{Kun Yu}{
born in Xi'an, China, 1978, received her Ph.D. degree in
Computer Science and Technology in 2005, from University of Science and
Technology of China (USTC). She is currently a researcher of Tsujii Lab.,
Graduate School of Information Science and Technology, The University of
Tokyo. Her research interests include Chinese HPSG grammar development,
Chinese dependency parsing, Chinese morphological analysis, machine
translation, and information extraction.
}
\end{biography}

\biodate





\end{document}

