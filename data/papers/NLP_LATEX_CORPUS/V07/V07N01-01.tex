\documentstyle[epsf,jnlpbbl]{jnlp_j}

\setcounter{page}{3}
\setcounter{巻数}{7}
\setcounter{号数}{1}
\setcounter{年}{2000}
\setcounter{月}{1}
\受付{1999}{1}{6}
\採録{1999}{10}{18}

\setcounter{secnumdepth}{2}

\title{ソフトウエアの要求獲得会議での\\言い直しに注目した要求獲得方法論}
\author{土井 晃一\affiref{Fujitsu}}
\headauthor{土井 晃一}
\headtitle{ソフトウエアの要求獲得会議での言い直しに注目した要求獲得方
法論}

\affilabel{Fujitsu}{株式会社 富士通研究所 コンピュータシステム研究所}
{Computer System Laboratories, Fujitsu Laboratories Ltd.}

\jabstract{
ソフトウエアの要求獲得会議では会議参加者の関心のあることをきちんと堀起
こすことが重要である.関心のあることを堀起こすためには会議参加者の無意
識の部分を知る方法が考えられる.無意識の兆候としては古来から言い直しが
挙げられている.言い直しを利用するとしても,言い直しを解釈するやり方は
高度の技術を要する.そこで本研究では,言い直しを解釈しないで利用する方
法を考えることにする.そこでまず,言い直した語と言い直された語との間で
どちらに関心が高いかを調べた.その結果,言い直された語の中にも話し手の
関心が高いものが見受けられた.そこで,言い直された語を抽出して,次の会
議の話題展開に用いる方法論を考案した.}

\jkeywords{言い直し、要求獲得、会議}

\etitle{A Requirements Capturing Method Focusing\\ in the Corrected
Words\\ in the Software Requirements Capturing Meeting.}

\eauthor{Kouichi Doi\affiref{Fujitsu}}

\eabstract{
In software requirements capturing meeting, it is important to find
out interesting items of clients. There is a method of finding out
interesting items by knowing their structure of unconsciousness. There
are some indications of unconsciousness. For example, there are some
mistakes in speaking. To interpret mistakes in speaking seems to need
a high technic. We aim at the method of not interpreting the mistakes
or corrects in speaking. We experimented the relationship between
interests and corrects in speaking. We found the case that the speaker
was interested in the corrected word. We made a requirements capturing
method to find the topics in the next meeting extracting the corrected
words.}

\ekeywords{mistake, requirements capturing, meeting}

\begin{document}
\maketitle

\section{はじめに}

まず,言い間違いの原因について考察してみる.フロイト\cite{freud1917a}
は言い
間違いの原因として身体的理由と精神的理由を挙げている.フロイトは身体的
理由として,

\begin{enumerate}
\item 気分が悪い・疲れ気味である
\item あがっている
\item 注意が他にそれている
\end{enumerate}

\noindent を挙げている.1は確かに身体的理由であるが,2と3はむしろその場
の精神的理由である.フロイトが言いたいことは,確かに上記のような身体的
理由があるにしろ,言い間違いが生じている時は必ず何らかの深層心理的・無
意識的理由があるということである.フロイトは深層心理的・無意識的理由の
ない言い間違いはありえない,つまり偶然生じる言い間違いはあり得ないと断
言している.さらにフロイトは言い間違いで探索すべき概念の範囲として,似
た言葉(発音・言語類似・言語連想)と反対の意味の言葉を挙げている.

しかしながら,あまりよく知らない単語であったり,関心が薄い単語であれば
言い間違えることが考えられる.

また,ラカンの流れを汲むNasioは,無意識は相互作用であり,コミュニケー
ションあるいは精神分析の中でしか無意識は存在しないと言っている
\cite{nasio1995a}.これは精神分析者が被精神分析者の無意識を被精神分析
者に示し,理解させ,相互に了解しながら精神分析が進んでいくということを
意味しているものと思われる.その無意識の兆候の一つとして挙げられるのが
言い間違いである.つまり言い間違いのすべてが無意識を顕現化しているもの
ではない.

このような無意識を第三者が観察することで見い出すことは可能であろうか? 
もし可能であれば,会議支援につながる.会議参加者が意識的には気づいてい
ないが無意識的に重要だと思っていることを会議へフィードバックすることが
できるからである.しかるに言い間違いは無意識の兆候を示しているのである
から,言い間違いを調べることによって会議支援ができることが期待できる.

しかし,前述のような精神分析的方法は分析者の解釈がどうしても必要であり,
かなりの能力が必要となり,誰にでもできるというわけにはいかない.しかも,
その解釈にはかなり主観的要素がつきまとう.実際の言い間違いの利用方法に
は,

\begin{enumerate}
\item 解釈しない(客観的)
\item 解釈する(主観的)
\end{enumerate}

\noindent の二種類が考えられる.前者は言い間違えた事実だけを客観的に使
う方法であり,後者は言い間違いを解釈して使う方法である.我々は,解釈に
は分析者にかなりの能力が必要であり,利用の条件が厳しくなり,また,分析
者の主観性が強く現れすぎて,結果が恣意的になると考え,前者の方法を採用
する.

言い間違いに関する用語を定義しておく.言い間違いにはいわゆる言い間違い,
言い淀み,言い直しなどが含まれる.本論文では,言い淀みとは不要な語句
(感動詞を含む)が挿入された発話を指すことにし,言い直しとは途中で発言が
中断され別の語句に発話し直したことを指すことにし,言い間違いとは言い淀
み・言い直し以外の言い間違いのことを指すことにする.ソフトウエアの要求
獲得会議のコーパスから言い間違いの例を挙げると,

\begin{verbatim}
言い淀み: 「電話で何だけ,留守番電話みたいに」
言い直し: 「たとえば何らかのシステムが出て,出たとしても」
言い間違い: 「自分の手帳でやってや書くでしょう」
\end{verbatim}

\noindent のようになる.なお,言い直しの例で,「出て」を言い直す前の単
語,「出た」を言い直した後の単語と呼ぶことにする.

言い直し以外の言い間違いを利用するためにはどう
しても解釈する必要が出てくる.我々は客観的に分析するという観点から,主
として,言い直しに限って分析を進める.

さらに言い直しは,客観的に判断できる,形態論的な観点から,

\begin{enumerate}
\item\label{どの文法単位の言い直しか?} どの文法単位の言い直しか?
\begin{enumerate}
\item\label{単語レベルの言い直し} 単語レベルの言い直し
\item\label{文節レベル以上の言い直し} 文節レベル以上の言い直し
\end{enumerate}
\item\label{言い直しの間に他の発話が入っているか?} 言い直しの間に他の発話が入っているか?
\begin{enumerate}
\item\label{直後の言い直し} 直後の言い直し
\item\label{他の発話が入った言い直し} 他の発話が入った言い直し
\end{enumerate}
\end{enumerate}

\noindent に分類される.もちろん,\ref{どの文法単位の言い直しか?}と
\ref{言い直しの間に他の発話が入っているか?}の間には重複があり得るので,
全体では四通りに分類できる.

それぞれ,単独の場合の例を,実際の発話から挙げておく.

まず,\ref{単語レベルの言い直し}の例としては,

\begin{quote}
あ,メ,電話の取り次ぎってことね.
\end{quote}

\noindent が挙げられる.この例は,文脈から「メモ」を「電話の取り次ぎ」
に言い直したことがわかる.

次に,\ref{文節レベル以上の言い直し}の例としては,

\begin{quote}
離席の,リフレッシュルームに電話番号はないわけだから,
\end{quote}

\noindent が挙げられる.この例は,「離席の」という名詞と格助詞からなる
文節を「リフレッシュルームに」に言い直している.このように,\ref{単語
レベルの言い直し}と\ref{文節レベル以上の言い直し}との違いは,言い直す
前の語句が単語か文節かの違いである.

次に,\ref{直後の言い直し}の例としては,先ほどの,

\begin{quote}
あ,メ,電話の取り次ぎってことね.
\end{quote}

\noindent が挙げられる.また,\ref{他の発話が入った言い直し}の例として
は,

\begin{quote}
ファッ……だからE−mailからFAXは簡単だよね.
\end{quote}

\noindent が挙げられる.この例では,「ファックス」が「E−mail」に
言い直され,両方の語句の間に「だから」が挿入されている.このように,
\ref{直後の言い直し}と\ref{他の発話が入った言い直し}の違いは,言い直さ
れた語句の間に他の語句が挿入されたかどうかの違いによる.

前述のように,言い直しのすべてが無意識の兆候になっているかどうかは若干
の疑念がある.そこで本論文では,第2節で,言い直す前の単語と言い直した
後の単語のどちらにより関心があるかを調べる.次に,第3節で,言い直しを
ソフトウエアの要求獲得に使う考え方について述べる.次に,第4節で,言い
直しを利用した,要求獲得方法論について述べる.第5節では,本要求獲得方
法論を例題を挙げて説明する.第6節では,全体のまとめと今後の課題につい
て述べる.

\section{言い直しと関心の高さ}

言い直す前の単語と言い直した後の単語のどちらにより関心が高いかを調べる
ために,次のような実験を行なった.

まず,実験に先だって,会議を行なった.会議の議題は本研究所の在席管理・
会議室予約システムを取り上げた.会議は,コーディネータ1名(実験を行な
うグループから選出した),マネージャー1名(実際の研究所の部長であり,予
算の権限を持っている),開発者役の研究員1名(いい仕様書ができれば,彼が
ソフトウェア外注を使ってシステムの構築に当たることになる),書記1名(事
務処理を業務とする者),研究員3名(実際に要求を出してもらう者)の計7名
を会議の参加者とした.また,分析を効率良く進めるために,先のコーディネー
タの他に,議事録をとる人,話題の項目を書き出す人,機器の操作を行なうも
のの計4人を置いた.

会議終了後,オーディオ・テープとビデオテープを用いて,発話の書き起こし
を行なった.
この書き起こされたものを以後,コーパスと呼ぶ.

実験は二回目の会議が終了した後に行なった.まず,言い直しの中で,体言
(句)の言い直しに絞った.なぜなら,それ以上(例えば,文の言い直し),それ
以外のもの(例えば,用言(句)の言い直し)は解釈に曖昧性が大きいからである.
さらに,その中でも,同じ単語の繰り返しなどではない,二つの言葉の差異が
適度に大きい言い直しを抽出した.

図\ref{questionnaire3}のようなフォーマットでアンケートを行なった.項目内の順
番は乱数で適度に替えた.例えば,2では,実際のコーパスでの出現順に,言
い直す前の単語が「メモ」,言い直した後の単語が「電話の取り次ぎ」となっ
ている.一方,3では,実際には「在席」を「出社」に言い直したのだが,ア
ンケートでの順番は「出社」,「在席」の順番になっている.この変化はいつ
も同じ側が来ることを被験者に意識させないためである.

\begin{figure*}[t]
\begin{center}
\begin{tabular}{|r|r|r|}
\multicolumn{3}{p{9.5cm}}{今回のシステムのことを念頭においたうえで,次の二つ
の概念のうちでより(関心の深い方/望ましい方)を丸で囲んで下さい.}\\	\hline
1 & 在席 & 出張\\	\hline
2 & メモ & 電話の取り次ぎ\\	\hline
3 & コンピュータ & マシーン\\	\hline
4 & 机 & ドア\\	\hline
5 & 出社 & 在席\\	\hline
6 & ファックス & E-mail\\	\hline
7 & いるかいないか & 外出 \\	\hline
8 & スクリーニング & フィルタリング\\	\hline
9 & 出社 & 在席 \\	\hline
10 & 離席 & リフレッシュルーム\\	\hline
11 & 翌日 & 朝\\	\hline
12 & サービス出社 & サービス残業\\	\hline
13 & 在席 & 離席\\	\hline
14 & 一人 & 依田さん\\	\hline
15 & 退社 & 出張\\	\hline
16 & リセット & リフレッシュ\\	\hline
17 & パージング & 認識 \\	\hline
18 & キーボード & キャラクタ\\	\hline
19 & 故意に & 間違って\\	\hline
20 & 退社 & 在席\\	\hline
\end{tabular}
\caption{アンケートのフォーマット}
\ecaption{The format of the questionnaire}
\label{questionnaire3}
\end{center}
\end{figure*}

1と20は,実際のコーパスの中には現れなかったものである.これは,初頭効
果と最終効果をなくすためである.

また,4も実際のコーパスの中には現れなかったものである.1と20を入れた
理由にもなるが,言い直し以外のものを混入させ,被験者に何を測定している
かをわかりにくくするためである.

さらに,5と9は同じ項目を聞いているが,これも,被験者に何を測定している
かをわかりにくくするためである.

\begin{table*}
\begin{scriptsize}
\begin{tabular}{|r|r||r|r|r||r|r|r|r|r|r|}	\hline
項番 & A & B & (a) & (b) & V & Y & X & W & Z & T\\	\hline
1 & メモ & 電話の取り次ぎ & ○ & ○ & B & B & B\hspace{-3mm}$\bigcirc$ & B & B & A\\	\hline
2 & コンピュータ & マシーン & ○ & ○ & A & A & B\hspace{-3mm}$\bigcirc$ & A & A & A\\	\hline
*3 & 出社 & 在席 & ○ & ○ & A & B\hspace{-3mm}$\bigcirc$ & A & A & A & A\\	\hline
4 & ファックス & E-mail & ○ & & B\hspace{-3mm}$\bigcirc$ & B & B & B & B & B\\	\hline
*5 & いるかいないか & 外出 & ○ & & A & B\hspace{-3mm}$\bigcirc$ & A & B & B & B\\	\hline
6 & スクリーニング & フィルタリング & ○ & ○ & A & B\hspace{-3mm}$\bigcirc$ & B && B & A\\	\hline
*7 & 出社 & 在席 & ○ & ○ & B & A & B & A & A\hspace{-3mm}$\bigcirc$ & A\\	\hline
8 & 離席 & リフレッシュルーム & & & A & A\hspace{-3mm}$\bigcirc$ & A & A & A & A\\	\hline
9 & 翌日 & 朝 & ○ & & B & A\hspace{-3mm}$\bigcirc$ & A & B & B & B\\	\hline
*10 & サービス出社 & サービス残業 & & & A & B & B & A & A\hspace{-3mm}$\bigcirc$ & A\\	\hline
*11 & 在席 & 離席 & ○ & ○ & B\hspace{-3mm}$\bigcirc$ & A & B & B & B & B\\	\hline
12 & 一人 & 依田さん & & ○ & B\hspace{-3mm}$\bigcirc$ & B & B & B & A & A\\	\hline
13 & 退社 & 出張 & & ○ & B\hspace{-3mm}$\bigcirc$ & A & B & B & A & B\\	\hline
*14 & リセット & リフレッシュ & ○ & & B\hspace{-3mm}$\bigcirc$ & B & A & A & B & A\\	\hline
*15 & パージング & 認識 & ○ & & A & A & A & A & A\hspace{-3mm}$\bigcirc$ & A\\	\hline
16 & キーボード & キャラクタ & ○ & ○ & B\hspace{-3mm}$\bigcirc$ & A & A & A & A & A\\	\hline
17 & 故意に & 間違って & ○ & ○ & A\hspace{-3mm}$\bigcirc$ & B & B & B & B & B\\	\hline
計 & & & 13(4:9)α & 6(1:5)β & 7(1:6) & 5(2:3) & 2(0:2) & 0 & 3(3:0) &\\	\hline
\end{tabular}
\end{scriptsize}
\caption{アンケートの結果}
\ecaption{The result of the questionnaire}
\label{questionnaire4}
\end{table*}

表\ref{questionnaire4}に結果の生データを示す.表中,Aの列は言い直す前の単語,
Bの列は言い直した後の単語である.また,(a)の列は単語レベルの言い直し,
(b)の列は直後の言い直しにそれぞれ該当するものに○印を付けてある.また,
項番中*印は,被験者に意図を悟られないように,乱数で選んでAとBの順番を
反転させたものを示す.T, V, W, X, Y, Zは個人名を示す.Tは司会者である
ので,ここでは測定の対象としなかった.VからZの列で,○で囲んだAとBはそ
の人に言い直しがあったことを示す.一番下の合計欄は,(a)と(b)は該当数,
括弧の中は該当するうちで,AとBのそれぞれの合計を示す.個人の合計は言い
直しがあったもので,括弧の中はAとBのそれぞれの合計を示す.

この結果から言えることは,言い直したのであるから,当然のことながらB,
すなわち,言い直した後の単語の方が関心が高い.しかし,A,すなわち,言
い直す前の単語でも,関心の高いものがある.さらに,Zのように言い直す
前の単語にだけ関心が高いものもいる.

ここで,特に,(b)は,$\chi^2 > \chi_{0.5}^2$となり,単語レベルの直後の
言い直しは、後に発話した単語に関心があることが分かる.

つまり,言い直しは総じて言い直した後の単語の方に関心が高いが,言い直す
前の単語にも,関心の高いものがあり,人によっては,言い直した後の単語に
だけ関心を持つものもいることが言える.

\section{要求獲得オフライン法での利用}

本節では,ソフトウエアの要求獲得で関心事項を抽出するための方法論の一つ
として言い直しをいかに利用するかを述べる.

まず,要求獲得オフライン法\cite{doi94a}について概説する.ソフトウエア
要求獲得プロセスでよく使われている方法として会議が挙げられる.会議では
「要求獲得者の容認と理解」というフィルター,時間的な制約などにより,要
求獲得が的確に行われないことがある.つまり要求を網羅的にキャッチアップ
するのが困難であり,話題の展開が不十分なまま終ってしまうこともあり得る.
これらの問題点を解決するために,会議を録音・ビデオ撮影して,オフライン
で観察・分析する方法(オフライン法)がある.

また,ソフトウエア開発に当たっては,下流ではコストをかけるが上流ではあ
まりコストをかけない.そのため,顧客の本当の要求が正確に獲得できないで,
システム構築に入ってしまうことがよくある.その結果,使いにくい,あるい
は,使われないシステムが往々にして出来上がる.使いにくい,あるいは,使
われないシステムを作らないために,上流,特に要求獲得のフェーズではかな
りのコストをかけてもよい,と考えられる\cite{doi93b}.

以上のような問題意識を背景にして,我々はオフライン法の一実現方法として
USP-Offline法を構築している\cite{kata96a}.USP-Offline法では,ビデオテー
プから発話を書き起こし,コーパスを作ることで解析を始める.USP-Offline
の主な出力としては,構造化された精密な議事録と関心事項があるが,本稿で
はその関心事項獲得方法論について述べる.なお,本方法論は顧客の関心事項
を抽出することを目的とするため,ここで扱う会議は,少なくとも顧客が参加
し,コーディネータがいる要求獲得会議を対象にし,開発者側が参加するかど
うかは関知しない.

関心事項の抽出の仕方の一例として,ここでは言い直しを取り上げる.前節の
実験結果が示す通り,言い直す前の単語に対しても発話者の関心が見られるも
のがある.つまり,言い直す前の単語には,関心があるかもしれないものがあ
る.この関心があるかもしれないものは,前述のフィルターの問題,時間的な
制約により,会議で充分に話されていない可能性がある.この関心があるかも
しれないものの中に,顧客の本当の要求があると,要求が正確に獲得できない
ことになる.そこで,この関心があるかもしれないものを会議にフィードバッ
クすることを考える.関心があるかもしれないものを会議に戻すのであるから,
ここで対象とする会議は,顧客・開発者・コーディネータ・分析者のいずれか
二者以上がよく知らないもの同士であるときに効果的である.

具体的な手順としては,まず,分析者は,この関心があるかもしれないものが
会議中に話されているかどうかをコーパスの中で調べる.会議中に充分話され
ていればよいが,会議中に充分話されていない時は次回の会議で,その関心が
あるかもしれないものを優先的に話させることにする.

\section{言い直しを利用した関心事項の抽出法}

本節では,言い直しを利用した関心事項の抽出法について述べる.本方法論は,
言い直す前の単語に焦点を当て,関心事項を推論する方法論である.本方法論
は,言い直す前の単語を関心があるかもしれないものとして抽出し,それを関
心事項として次の会議にフィードバックすることによって要求獲得を行ない,
システム設計に役立てる方法論である.

二つの方法が考えられる.一つの方法は会議中に言い直しを調べる専門のスタッ
フがいて,言い直しの起きたところを忠実に記録する方法である.もう一つの
方法はコーパスを起こし,言い直しの起きたところを正確に把握する方法であ
る.この二つの方法は正確さと工数がトレードオフの関係になっている.

いずれの方法をとるにせよ,会議中に言い直しの起きた場所が記述できる.言
い直しは,体言句の言い直しで,かつ,同一語句の言い直しではないものを対
象とする.

次に,分析者はその言い直しの起きた場所を見ながら,

\begin{enumerate}
\item 誤発言
\item 誤表現
\item 正発言
\item 正表現
\end{enumerate}

\noindent を埋める.ここで,誤発言とはコーパスの中で言い直す前の語句,
誤表現とは誤表現が完結した語句になっていない時,分析者の推論で埋めた語句,
正発言とはコーパスの中で言い直した後の語句,正表現とは正発言が完結した
語句になっていない時,分析者の推論で埋めた語句をそれぞれ指す.

まず,分析者は,コーパスから誤発言と正発言を書き出し,次に,誤表現と正
表現を推論によって埋める.推論不明の場合は「不明」と書く.

次に,分析者は,誤表現の内容が当該会議の中で充分議論されているかどうか
を調べる.充分議論されていない項目を関心があるかもしれないもののリスト
とする.

分析者は関心があるかもしれないもののリストを元にコーディネータとともに
次の会議の前に事前会議をする.事前会議では関心があるかもしれないものの
リストについてお互いに話し合い,次の本会議で話し合う事項を決める.この
事項は関心があるかもしれないものがどれぐらい会議で話されていないかが選
出の基準になる.話されていない関心があるかもしれないものほど優先順位が
高い.ただ,これらの事項は,フロイトらの分析にもあるように,システムの
構築に携わる開発側にも知らせられない性格のものである可能性があるので,
扱いには注意しなければならない.顧客側に何らかの感情的なわだかまりがあっ
て,本会議でどうも話題展開がすっきりいかないときは,開発者・コーディネー
タが席を外して,顧客・ユーザだけの会議で解決をはかってもらうことも必要
となろう.いずれにしてもコーディネータの裁量が要求される.

\section{例題}

本節では,例を挙げて,本方法論の運用方法を説明する.

前述の,

\begin{quote}
あ,メ,電話の取り次ぎってことね.
\end{quote}

の例を使う.分析者が,コーパス中,あるいは,会議中にこの発話を発見した
時に,まず,誤発言に「メ」を正発言に「電話の取り次ぎ」を記入する.次に,
誤表現を文脈などの推論から「メモ」と記入し,正表現を「電話の取り次ぎ」
と記入する.出来上がりは以下のようになる.

\begin{enumerate}
\item 誤発言 メ
\item 誤表現 メモ
\item 正発言 電話の取り次ぎ
\item 正表現 電話の取り次ぎ
\end{enumerate}

会議終了後,分析者は,会議の内容と照らし合わせて,「電話の取り次ぎメモ」
に関する議論が充分であったかどうかを調べる.もし充分でなければ,これを
関心があるかもしれないもののリストに加える.

次に,前述のように,分析者は,コーディネータと共に事前会議を行なう.こ
こで,分析者とコーディネータの二人の判断で,「電話の取り次ぎメモ」を次
の会議の議題の一つにするかどうかを決定する.

実際,この会議では,「電話の取り次ぎメモ」に関する議論はこの後発展しな
かった.本当に話さなくて良いのかどうか,会議参加者たちに確認を求めるべ
きであろう.

\section{おわりに}

本論文では,まず,精神分析の立場から,フロイトとラカンの説を検討し,言
い間違いには,なんらかの精神的理由が存在することがあることを検討した.

次に,言い間違いを言い直しに絞り,言い直した語と言い直された語との間で
どちらに関心が高いかを分析した.分析の結果,言い直しは総じて言い直した
後の単語の方に関心が高いが,言い直す前の単語にも,関心の高いものがあり,
人によっては,言い直した後の単語にだけ関心を持つものもいることが言える
ことがわかった.

これらの知見を利用して,言い直しをソフトウエアの要求獲得の方法論に役立
てる,考え方と,方法論を示し,さらに,適用例を述べた.その結果,顧客の
関心があるかもしれないものを抽出することができた.

今後は,この方法論を現実のソフトウエアの要求獲得の場で用いて,その効果
のほどを調べていきたい.

\bibliographystyle{jnlpbbl}

\bibliography{muisiki,usp}

\begin{biography}
\biotitle{略歴}
\bioauthor{土井 晃一}{
1961年生.
1991年東京大学工学部情報工学専攻博士課程修了.
工学博士.
同年富士通研究所国際情報社会科学研究所入社(現富士通研究所コンピュータシス
テム研究所).
自然言語理解,人工知能,ソフトウエア工学などの研究に従事.
1998年9月より文部省学術情報センター客員助教授併任.
情報処理学会,人工知能学会,認知科学会,ソフトウエア科学会,言語処理学
会各会員.}

\bioreceived{受付}
\biorevised{再受付}
\bioaccepted{採録}

\end{biography}

\end{document}
